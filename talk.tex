\documentclass[t,aspectratio=169]{beamer} %tlmgr install translator
%\includeonlyframes{current}  %%%% USE: \begin{frame}[label=current]
\usepackage{macros}
\usepackage{talk}
\begin{comment}
\begin{frame}
  \frametitle{}
  \begin{itemize}
  \item 
  \end{itemize}
\end{frame}
    \begin{itemize}
    \item 
    \end{itemize}

    Great Beamer guide here:
    https://www.overleaf.com/learn/latex/Beamer_Presentations:_A_Tutorial_for_Beginners_(Part_4)—Overlay_Specifications

To convert from PDF to KeyNOTE: use the automator suggestion here
https://apple.stackexchange.com/questions/95856/import-multiple-pages-from-a-pdf-as-separate-slides-in-keynote
I put the automator app here: https://www.dropbox.com/sh/1o36nes45cydmbx/AADPVWpeEBZj2vemVXkGKQXVa?dl=0
- Copy that to your machine somewhere.
- In the finder, drop talk.pdf onto "PDF Save Pages as Images.app".
- This will create a bunch of png files on the desktop
- Now, open keynote with a blank document.
- In the finder, select the files, drop onto the leftmost "slides" pane of Keynote.

Another display option is to use http://dspdfviewer.danny-edel.de, which requires
  \usepackage{pgfpages}
  \setbeameroption{show notes on second screen}
  You can use the keys "t" and "d" to get different things on the presenter screen

\pause
\onslide<>

\visible<>{during}                 visibleenv
\invisible<>{other}                invisibleenv

\onslide<>{during}
\uncover<>{during}                 uncoverenv

\only<>{during}                    onlyenv
\alt<>{during}{other}              altenv
\temporal<>{before}{during}{after}
\setbeamercovered{transparent=35}


\begin{itemize}[<+-| alert@+>]
\item Apple
\item Peach
\item Plum
\item Orange
\end{itemize}
\end{comment}

\title{Pomsets with Preconditions}
\subtitle{A Simple Model of Relaxed Memory}
\author{Radha Jagadeesan$^\dagger$ \and Alan Jeffrey$^*$ \and James Riely$^\dagger$}
\date{October 2020}
\institute{$^\dagger$DePaul University \and $^*$Mozilla Research and the Servo Project}

%\usepackage{pgfpages}\setbeameroption{show notes on second screen}
\setbeamertemplate{navigation symbols}{\insertframenumber/\inserttotalframenumber}
\begin{document}
\begin{frame}
  \maketitle
\end{frame}
% \begin{frame}
%   \begin{framecenter}
%     \Huge
%     Shared Memory Concurrency \\
%     \vfill
%     \uncover<2->{Transactions} %\\ to augment or replace locks} \\
%     \vfill
%     \uncover<3->{Programmer Model?}
%   \end{framecenter}
% \end{frame}

\begin{frame}
  \frametitle{Can this program write 1 to z?}
  \begin{gather*}
    y\GETS x
    \PAR
    r\GETS y\SEMI
    x\GETS r\SEMI
    z\GETS r
  \end{gather*}
  $r$ is a register\\
  $x$, $y$, $z$ are shared memory locations, initially $0$

 
  \pause
  \vspace{1in}
  what about
  \begin{gather*}
    y\GETS x
    \PAR
    r\GETS y\SEMI
    \alert{\IF{r}\THEN} x\GETS r\SEMI z\GETS r \alert{\ELSE x\GETS1 \FI}
  \end{gather*}
\end{frame}

\begin{frame}
  \frametitle{Can this program write 1 to z?}
  \begin{gather*}
    y\GETS x
    \PAR
    r\GETS y\SEMI
    \IF{r}\THEN x\GETS r\SEMI z\GETS r \ELSE x\GETS1 \FI
  \end{gather*}
  \pause
  $x$ and $y$ can only be $0$ or $1$  
  \begin{gather*}
    y\GETS x
    \PAR
    r\GETS y\SEMI
    \IF{r}\THEN x\GETS r\SEMI z\GETS 1 \ELSE x\GETS1 \FI
  \end{gather*}
  \pause
  lift common code
\begin{gather*}
  %\taglabelpp{OOTA?}
  y\GETS x
  \PAR
  r\GETS y\SEMI
  x\GETS1\SEMI
  \IF{r}\THEN z\GETS 1 \FI
\end{gather*}
  \pause
  commute independent statements
\begin{gather*}
  %\taglabelpp{OOTA?}
  y\GETS x
  \PAR
  x\GETS1\SEMI
  r\GETS y\SEMI
  \IF{r}\THEN z\GETS 1 \FI
\end{gather*}
  \pause
  interleaving
\begin{gather*}
  %\taglabelpp{OOTA?}
  x\GETS1\SEMI
  y\GETS x\SEMI
  r\GETS y\SEMI
  \IF{r}\THEN z\GETS 1 \FI
\end{gather*}
\end{frame}

{
  \setbeamertemplate{navigation symbols}{
    \href{https://en.wikipedia.org/wiki/Nuclear_explosion\#/media/File:Operation_Upshot-Knothole_-_Badger_001.jpg}{Photo in the Public domain}
  }
  \usebackgroundtemplate{\includegraphics[width=\paperwidth]{images/abomb.jpg}}
\begin{frame}
\end{frame}
}

{
  \setbeamertemplate{navigation symbols}{\href{https://commons.wikimedia.org/wiki/File:Tulips,Hitachi_Seaside_Park,Hitachinaka-city,Japan.JPG}{Photo
  by Katorisi, 2010, Creative Commons}}
\usebackgroundtemplate{\includegraphics[height=\paperheight]{images/katorisi.jpeg}}
\begin{frame}
\end{frame}
}


\begin{frame}
  \frametitle{It's going to be okay}
  \begin{itemize}[<+->]
  \item Concurrent observers see more [Plotkin and Pratt , 1990]: \emph{Teams
      Can See Pomsets}]
    \begin{itemize}
    \item compiler optimization only sound for sequential programs      
    \end{itemize}
  \item Use \emph{Sequential consistency} as correctness criterion [Lamport,
    1979]: \emph{How to Make a Multiprocessor Computer That Correctly
      Executes Multiprocess Programs}
    \begin{itemize}
    \item It's too slow
    \end{itemize}
  \item 
  \item Orange
  \end{itemize}
\end{frame}

\begin{frame}
  \fontsize{80}{80}\selectfont
  Hello
\end{frame}
\end{document}

% Local Variables:
% mode: latex
% TeX-master: t
% End:
