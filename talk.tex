\documentclass[t,aspectratio=169]{beamer} %tlmgr install translator
% \includeonlyframes{current}  %%%% USE: \begin{frame}[label=current]
\usepackage{macros}
\usepackage{talk}
\begin{comment}
  \begin{frame}
    \frametitle{}
    \begin{itemize}
    \item 
    \end{itemize}
  \end{frame}
  \begin{itemize}
  \item 
  \end{itemize}

  Great Beamer guide here:
  https://www.overleaf.com/learn/latex/Beamer_Presentations:_A_Tutorial_for_Beginners_(Part_4)—Overlay_Specifications

  To convert from PDF to KeyNOTE: use the automator suggestion here
  https://apple.stackexchange.com/questions/95856/import-multiple-pages-from-a-pdf-as-separate-slides-in-keynote
  I put the automator app here: https://www.dropbox.com/sh/1o36nes45cydmbx/AADPVWpeEBZj2vemVXkGKQXVa?dl=0
  - Copy that to your machine somewhere.
  - In the finder, drop talk.pdf onto "PDF Save Pages as Images.app".
  - This will create a bunch of png files on the desktop
  - Now, open keynote with a blank document.
  - In the finder, select the files, drop onto the leftmost "slides" pane of Keynote.

  Another display option is to use http://dspdfviewer.danny-edel.de, which requires
  \usepackage{pgfpages}
  \setbeameroption{show notes on second screen}
  You can use the keys "t" and "d" to get different things on the presenter screen

  \pause
  \onslide<>

  \visible<>{during}                 visibleenv
  \invisible<>{other}                invisibleenv

  \onslide<>{during}
  \uncover<>{during}                 uncoverenv

  \only<>{during}                    onlyenv
  \alt<>{during}{other}              altenv
  \temporal<>{before}{during}{after}
  \setbeamercovered{transparent=35}


  \begin{itemize}[<+-| alert@+>]
  \item Apple
  \item Peach
  \item Plum
  \item Orange
  \end{itemize}
\end{comment}

\title{Pomsets with Preconditions}
\subtitle{A Simple Model of Relaxed Memory}
\author{Radha Jagadeesan$^\dagger$ \and Alan Jeffrey$^*$ \and James Riely$^\dagger$}
\date{October 2020}
\institute{$^\dagger$DePaul University \and $^*$Mozilla Research and the Servo Project}

% \usepackage{pgfpages}\setbeameroption{show notes on second screen}
\setbeamertemplate{navigation symbols}{\insertframenumber/\inserttotalframenumber}
\begin{document}
\begin{frame}
  \maketitle
\end{frame}
% \begin{frame}
%   \begin{framecenter}
%     \Huge
%     Shared Memory Concurrency \\
%     \vfill
%     \uncover<2->{Transactions} %\\ to augment or replace locks} \\
%     \vfill
%     \uncover<3->{Programmer Model?}
%   \end{framecenter}
% \end{frame}

\begin{frame}
  \frametitle{Can this program write 1 to z?}
  \begin{gather*}
    y\GETS x
    \PAR
    \alt<4>{x\GETS y}{r\GETS y \SEMI x\GETS r}
    \invisible<3-4>{\SEMI z\GETS r}
  \end{gather*}
  \onslide<2->
  $r$ is a register\\
  $x$, $y$, $z$ are shared memory locations, initially $0$

  \bigskip \bigskip
  \onslide<5->
  No!
  
  \onslide<6->
  \bigskip \bigskip
  what about
  \begin{gather*}
    y\GETS x
    \PAR
    r\GETS y\SEMI
    \alert{\IF{r}\THEN} x\GETS r\SEMI z\GETS r \alert{\ELSE x\GETS1 \FI}
  \end{gather*}
\end{frame}

\begin{frame}
  \frametitle{Can this program write 1 to z?}
  \begin{gather*}
    y\GETS x
    \PAR
    r\GETS y\SEMI
    \IF{r}\THEN x\GETS r\SEMI z\GETS r \ELSE x\GETS1 \FI
  \end{gather*}
  \pause
  $x$ and $y$ can only be $0$ or $1$  
  \begin{gather*}
    y\GETS x
    \PAR
    r\GETS y\SEMI
    \IF{r}\THEN x\GETS r\SEMI z\GETS 1 \ELSE x\GETS1 \FI
  \end{gather*}
  \pause
  lift common code
  \begin{gather*}
    % \taglabelpp{OOTA?}
    y\GETS x
    \PAR
    r\GETS y\SEMI
    x\GETS1\SEMI
    \IF{r}\THEN z\GETS 1 \FI
  \end{gather*}
  \pause
  commute independent statements
  \begin{gather*}
    % \taglabelpp{OOTA?}
    \only<5->{\alert}{y\GETS x}
    \PAR
    x\GETS1\SEMI
    r\GETS y\SEMI
    \IF{r}\THEN z\GETS 1 \FI
  \end{gather*}
  \pause
  interleaving
  \begin{gather*}
    % \taglabelpp{OOTA?}
    x\GETS1\SEMI
    \alert{y\GETS x}\SEMI
    r\GETS y\SEMI
    \IF{r}\THEN z\GETS 1 \FI
  \end{gather*}
\end{frame}

\imagepage{images/abomb3.jpg}{\href{https://www.publicdomainpictures.net/en/view-image.php?image=198310}{Image in the Public domain}}
%\imagepage{images/flowers.jpg}{\href{https://www.publicdomainpictures.net/en/view-image.php?image=365530}{Image in the Public domain}}
\imagepage{images/keynote-images/keynote-images.006.jpeg}{\href{https://www.publicdomainpictures.net/en/view-image.php?image=365530}{Image in the Public domain}}
\imagepage{images/keynote-images/keynote-images.007.jpeg}{\href{https://www.publicdomainpictures.net/en/view-image.php?image=365530}{Image in the Public domain}}

% \imagepage{images/katorisi.jpeg}{\href{https://commons.wikimedia.org/wiki/File:Tulips,Hitachi_Seaside_Park,Hitachinaka-city,Japan.JPG}{Image
% by Katorisi, 2010, Creative Commons}}


\begin{frame}
  \frametitle{Definitions}
  A \emph{pomset with preconditions} is a tuple
  $(\Event, {\le}, %{\leloc},
  \labeling)$, such that
  \begin{itemize}
  \item $\Event$ is a set of \emph{events},
  \item ${\le} \subseteq (\Event\times\Event)$ is a partial order, 
  \item $\labeling: \Event \fun (\Formulae\times\Act)$ is a \emph{labeling},
    from which we derive functions
    \begin{itemize}
    \item $\labelingForm:\Event\fun\Formulae$ and
    \item $\labelingAct:\Event\fun\Act$,
    \end{itemize}
  \item<2-> $\bigwedge_{\aEv}\labelingForm(\aEv)$ is satisfiable
    \emph{(consistency)}, and
  \item<3-> if $\bEv\le\aEv$ then $\labelingForm(\aEv)$ implies
    $\labelingForm(\bEv)$ \emph{(causal strengthening)}.
  \end{itemize}
\end{frame}
\begin{frame}
  \frametitle{Definitions}
  % Two actions \emph{conflict} if one writes a location and the other
  % either reads or writes the same location.

  We say $\bEv$ \emph{fulfills $\aEv$} (on $\aLoc$) if 
  \begin{itemize}
  \item[{\labeltextsc[F1]{(F1)}{rf1}}] $\bEv$ writes $\aVal$ to $\aLoc$,
  \item[{\labeltextsc[F2]{(F2)}{rf2}}] $\aEv$ reads $\aVal$ from $\aLoc$,
  \item[{\labeltextsc[F3]{(F3)}{rf3}}] $\bEv \lt \aEv$, and
  \item[{\labeltextsc[F4]{(F4)}{rf4}}] for every conflicting write $\cEv$,
    either $\cEv \gtN \bEv$ or $\aEv \gtN \cEv$.
  \end{itemize}
  \pause
  A pomset is \emph{$\aLoc$-closed} if every read on $\aLoc$ is fulfilled,
  and $\labelingForm(\aEv) \vDash \labelingForm(\aEv)[\aVal/\aLoc] \vDash \labelingForm(\aEv)$. 
\end{frame}

\begin{frame}
  \frametitle{Operators}
  \begin{itemize}[<+->]
  \item 
    Let $(\nu\aLoc\!\DOT\!\aPSS)$ be  $\aPSS'{\subseteq}\aPSS$ such that $\aPS'{\in}\aPSS'$
    when $\aPS'$ is $\aLoc$-closed.
  \item 
    Let $(\aForm \guard \aPSS)$ be the set $\aPSS'\subseteq\aPSS$ such that
    $\aPS'\in\aPSS'$ when $\labelingForm(\aEv')$ implies $\aForm$
    $(\forall\aEv'\in\Event')$. 
  \item 
    Let $(\aPSS\aSub)$ be the set $\aPSS'$ where $\aPS'\in\aPSS'$ when there is
    $\aPS\in\aPSS$ such that: $\Event' = \Event$, ${\le'} = {\le}$,
    $\labelingAct' = \labelingAct$, and
    $\labelingForm'(\aEv) = \labelingForm(\aEv)\aSub$.
    
    
  \item 
    Let $\aPS' \in (\aPSS^1 \parallel \aPSS^2)$ when there are
    $\aPS^1 \in \aPSS^1$ and $\aPS^2 \in \aPSS^2$ such that
    % $\aPS^1$ is completed exactly when $\aPS^2$ is completed, there is at most one termination in $\Event'$,
    $\Event' = \Event^1 \cup \Event^2$,
    ${\le'}\supseteq{\le^1}\cup{\le^2}$, and for all $\aEv\in\Event'$, either:
    \begin{gather*}
      \begin{aligned}
        \;\;&\labelingAct'(\aEv) = \labelingAct^1(\aEv) = \labelingAct^2(\aEv) \textand \labelingForm'(\aEv) \textimplies \labelingForm^1(\aEv) \lor \labelingForm^2(\aEv),\\[-1ex]
        &\;\;\aEv \not\in \Event^2,\; \labelingAct'(\aEv) = \labelingAct^1(\aEv) \textand \labelingForm'(\aEv) \textimplies \labelingForm^1(\aEv),\; \textor\\[-1ex]
        &\;\;\aEv \not\in \Event^1,\; \labelingAct'(\aEv) = \labelingAct^2(\aEv) \textand \labelingForm'(\aEv) \textimplies \labelingForm^2(\aEv).
      \end{aligned}
    \end{gather*}
  \end{itemize}
\end{frame}

\begin{frame}
  \frametitle{Operators}
  \begin{itemize}
  \item 
  Let $(\aForm \mid \aAct) \prefix \aPSS$ be the set
  $\PRE{\aPSS'}$ 
  where
  $\aPS'\in\aPSS'$ when 
  there is $\aPS\in\aPSS$ such that
  \begin{enumerate}
  \item[{\labeltextsc[P1]{(P1)}{1}}] $\Event' = \Event \cup \{\bEv\}$,
  \item[{\labeltextsc[P2]{(P2)}{2}}]  ${\le'}\supseteq{\le}$, 
  \item[{\labeltextsc[P3]{(P3a)}{3a}}]\labeltextsc[P3]{}{3}$\labelingAct'(\bEv) = \aAct$,
  \item[{\labeltextsc[P3b]{(P3b)}{3b}}] $\labelingAct'(\aEv) = \labelingAct(\aEv)$, 
  \item[{\labeltextsc[P4a]{(P4a)}{4a}}]\labeltextsc[P4]{}{4}$\labelingForm'(\bEv)$ implies $\aForm$, 
  \item[{\labeltextsc[P4b]{(P4b)}{4b}}]
    if $\bEv$ \externally reads $\aVal$ from $\aLoc$ then
    $\labelingForm'(\aEv)$ implies $\labelingForm(\aEv)[\aVal/\aLoc]$,
  \item[{\labeltextsc[P4c]{(P4c)}{4c}}]
    if $\bEv$ does not \externally read then
    $\labelingForm'(\aEv)$ implies $\labelingForm(\aEv)$, 
  \item[{\labeltextsc[P5a]{(P5a)}{5a}}]\labeltextsc[P5]{}{5}if $\aEv$ writes
    then either $\bEv\lt'\aEv$ or $\labelingForm'(\aEv)$ implies
    $\labelingForm(\aEv)$,
  \item[{\labeltextsc[P5b]{(P5b)}{5b}}]
    if $\bEv$ and $\aEv$ are \external actions in conflict,
    then $\bEv\lt'\aEv$, 
  \item[{\labeltextsc[P5c]{(P5c)}{5c}}]
    if $\bEv$ is an acquire or $\aEv$ is a release, then $\bEv \lt' \aEv$, and
  \item[{\labeltextsc[P5d]{(P5d)}{5d}}]
    if $\bEv$ is an SC write and $\aEv$ is an SC read, then $\bEv \lt' \aEv$.
  \end{enumerate}
\end{itemize}
\end{frame}

\begin{frame}
  \frametitle{Language {\normalsize ($\amode \BNFDEF \modeRLX \BNFSEP \modeRA \BNFSEP \modeSC$)}}
  \begin{gather*}
  % \begin{aligned}
  %   \aCmd,\,\bCmd
  %   \BNFDEF& \SKIP
  %   \BNFSEP \aReg\GETS\aExp\SEMI \aCmd
  %   \BNFSEP \aReg\GETS\aLoc^{\amode}\SEMI \aCmd 
  %   \BNFSEP \aLoc^{\amode}\GETS\aExp\SEMI \aCmd
  %   \\[-.5ex]
  %   \BNFSEP&\aCmd \PAR[\aThrd][\bThrd] \bCmd
  %   \BNFSEP \VAR\aLoc\SEMI \aCmd
  %   \BNFSEP \IF{\aExp} \THEN \aCmd \ELSE \bCmd \FI
  % \end{aligned}
  % \\
  \begin{aligned}
    \sem[\aThrd]{\SKIP} & \eqdef
    \{ \DSTOP \}
    \\
    \sem[\aThrd]{\aReg\GETS\aExp\SEMI \aCmd} & \eqdef
    \sem[\aThrd]{\aCmd}[\aExp/\aReg] 
    \\ 
    \sem{\aReg\GETS\aLoc^\amode\SEMI \aCmd} & \eqdef \textstyle\bigcup_\aVal\;
    (\DRmode\aLoc\aVal) \prefix \sem{\aCmd} [\aLoc/\aReg]
    \\
    \sem{\aLoc^\amode\GETS\aExp\SEMI \aCmd} & \eqdef
    \textstyle\bigcup_\aVal\; (\aExp=\aVal \mid \DWmode\aLoc\aVal)
    \prefix \sem{\aCmd}[\aExp/\aLoc]
    \\
    \sem[\aThrd]{\IF{\aExp} \THEN \aCmd \ELSE \bCmd \FI} & \eqdef
    \bigl(\aExp \guard \sem[\aThrd]{\aCmd}\bigr) \parallel \bigl(\lnot\aExp \guard \sem[\aThrd]{\bCmd}\bigr) 
    \\
    \sem[\aThrd]{\aCmd \PAR[\bThrd][\bThrd'] \bCmd} & \eqdef
    \sem[\bThrd]{\aCmd} \parallel \sem[\bThrd']{\bCmd} 
    \\
    \sem[\aThrd]{\VAR\aLoc\SEMI \aCmd} & \eqdef
    \nu \aLoc \DOT \sem[\aThrd]{\aCmd}  
  \end{aligned}
\end{gather*}
\end{frame}

\imagepage{images/keynote-images/keynote-images.007.jpeg}{\href{https://www.publicdomainpictures.net/en/view-image.php?image=365530}{Image in the Public domain}}

\begin{frame}
  \frametitle{Sequential Semantics}

  \begin{gather*}
    \begin{gathered}
      \only<9->{\IF{r}\THEN}
      \only<4->{x\GETS r\SEMI}
      z\GETS r
      \only<9->{\FI}
      \\
      \only<1>{\hbox{\begin{tikzinline}[node distance=.5em]
            \event{wz}{r{=}0\mid\DW{z}{0}}{}
          \end{tikzinline}}}
      \only<2>{\hbox{\begin{tikzinline}[node distance=.5em]
            \event{wz}{r{=}1\mid\DW{z}{1}}{}
          \end{tikzinline}}}
      \only<3>{\hbox{\begin{tikzinline}[node distance=.5em]
            \event{wz}{r{=}2\mid\DW{z}{2}}{}
          \end{tikzinline}}}
      \only<4>{\hbox{\begin{tikzinline}[node distance=.5em]
            \event{wz}{r{=}0\mid\DW{z}{0}}{}
            \event{wx}{r{=}0\mid\DW{x}{0}}{left=of wz}
          \end{tikzinline}}}
      \only<5,8>{\hbox{\begin{tikzinline}[node distance=.5em]
            \event{wz}{r{=}1\mid\DW{z}{1}}{}
            \event{wx}{r{=}1\mid\DW{x}{1}}{left=of wz}
          \end{tikzinline}}}
      \only<6-7>{\hbox{\begin{tikzinline}[node distance=.5em]
            \nonevent{wz}{r{=}1\mid\DW{z}{1}}{}
            \event{wx}{r{=}0\mid\DW{x}{0}}{left=of wz}
          \end{tikzinline}}}
      \only<9>{\hbox{\begin{tikzinline}[node distance=.5em]
            \event{wz}{r{\neq}0 \land r{=}1\mid\DW{z}{1}}{}
            \event{wx}{r{\neq}0 \land r{=}1\mid\DW{x}{1}}{left=of wz}
          \end{tikzinline}}}
      \only<10->{\hbox{\begin{tikzinline}[node distance=.5em]
            \event{wz}{r{=}1\mid\DW{z}{1}}{}
            \event{wx}{r{=}1\mid\DW{x}{1}}{left=of wz}
          \end{tikzinline}}}
    \end{gathered}
    \only<8->{\qquad \qquad\qquad\begin{gathered}
        \only<9->{\IF{\lnot r}\THEN}
        x\GETS \alt<12>{2}{1}
        \only<9->{\FI}
        \\
        \only<8>{\hbox{\begin{tikzinline}[node distance=.5em]
              \event{wx}{1{=}1\mid\DW{x}{1}}{}
            \end{tikzinline}}}
        \only<9>{\hbox{\begin{tikzinline}[node distance=.5em]
              \event{wx}{r{=}0 \land 1{=}1\mid\DW{x}{1}}{}
            \end{tikzinline}}}
        \only<10-11,13>{\hbox{\begin{tikzinline}[node distance=.5em]
              \event{wx}{r{=}0\mid\DW{x}{1}}{}
            \end{tikzinline}}}
        \only<12>{\hbox{\begin{tikzinline}[node distance=.5em]
              \event{wx}{r{=}0\mid\DW{x}{2}}{}
            \end{tikzinline}}}
      \end{gathered}}
    \\[3ex]
    \begin{gathered}
      \only<11->{\IF{r}\THEN x\GETS r\SEMI z\GETS r \ELSE x\GETS \alt<12>{2}{1}\FI}
     \\
      \only<11,13>{\hbox{\begin{tikzinline}[node distance=.5em]
            \event{wz}{r{=}1\mid\DW{z}{1}}{}
            \event{wx}{r{=}0\lor r{=}1\mid\DW{x}{1}}{left=of wz}
          \end{tikzinline}}}    
      \only<12>{\hbox{\begin{tikzinline}[node distance=.5em]
            \event{wz}{r{=}1\mid\DW{z}{1}}{}
            \event{wx}{r{=}1\mid\DW{x}{1}}{left=of wz}
            \nonevent{wxx}{r{=}0\mid\DW{x}{2}}{right=of wz}
          \end{tikzinline}}}    
    \end{gathered}
  \end{gather*}
  \begin{center}
    \only<7>{$r{=}0$ and $r{=}1$ are \emph{inconsistent}}
  \end{center}
\end{frame}


\begin{frame}
  \frametitle{Sequential Semantics}

  \begin{gather*}
    \begin{gathered}
      \IF{r}\THEN x\GETS r\SEMI z\GETS r \ELSE x\GETS 1\FI
      \\
      \hbox{\begin{tikzinline}[node distance=.5em]
          \event{wz}{r{=}1\mid\DW{z}{1}}{}
          \event{wx}{r{=}0\lor r{=}1\mid\DW{x}{1}}{left=of wz}
        \end{tikzinline}}
    \end{gathered}
  \end{gather*}
\end{frame}




\imagepage{images/keynote-images/keynote-images.008.jpeg}{\href{https://www.publicdomainpictures.net/en/view-image.php?image=365530}{Image in the Public domain}}
\imagepage{images/keynote-images/keynote-images.009.jpeg}{\href{https://www.publicdomainpictures.net/en/view-image.php?image=365530}{Image in the Public domain}}
\imagepage{images/keynote-images/keynote-images.010.jpeg}{\href{https://www.publicdomainpictures.net/en/view-image.php?image=365530}{Image in the Public domain}}
\imagepage{images/keynote-images/keynote-images.011.jpeg}{\href{https://www.publicdomainpictures.net/en/view-image.php?image=365530}{Image in the Public domain}}
\imagepage{images/keynote-images/keynote-images.012.jpeg}{\href{https://www.publicdomainpictures.net/en/view-image.php?image=365530}{Image in the Public domain}}
\imagepage{images/keynote-images/keynote-images.013.jpeg}{\href{https://www.publicdomainpictures.net/en/view-image.php?image=365530}{Image in the Public domain}}


\begin{frame}
  \frametitle{Examples?}
  lochbihler
oota1
alanAddress
ex-1
SB
ex-2
\end{frame}


% \begin{frame}
%   \frametitle{It's going to be okay}
%   \begin{itemize}[<+->]
%   \item Concurrent observers see more [Plotkin and Pratt , 1990]: \emph{Teams
%       Can See Pomsets}]
%     \begin{itemize}
%     \item compiler optimization only sound for sequential programs      
%     \end{itemize}
%   \item Use \emph{Sequential consistency} as correctness criterion [Lamport,
%     1979]: \emph{How to Make a Multiprocessor Computer That Correctly
%       Executes Multiprocess Programs}
%     \begin{itemize}
%     \item It's too slow
%     \end{itemize}
%   \item 
%   \item Orange
%   \end{itemize}
% \end{frame}

\end{document}

% Local Variables:
% mode: latex
% TeX-master: t
% End:
