\section{Additional RMW Examples}


\begin{comment}
Status of IRIW is unclear in our model, since we allow everything allowed by
power...
\begin{gather*}
  \begin{gathered}
    % x\GETS0\SEMI
    x\GETS 1
    \PAR
    % y\GETS0\SEMI
    y\GETS 1
    \PAR
    r\GETS x^\modeRA\SEMI s\GETS y
    \PAR
    s\GETS y^\modeRA \SEMI r\GETS x
    \\
    \smash[b]{\hbox{\begin{tikzinline}[node distance=1.5em]
          % \event{wx0}{\DW{x}{0}}{}
          % \event{wx1}{\DW{x}{1}}{right=of wx0}
          % \event{wy0}{\DW{y}{0}}{below=4ex of wx0}
          % \event{wy1}{\DW{y}{1}}{right=of wy0}
          \event{wx1}{\DW{x}{1}}{}
          \event{wy1}{\DW{y}{1}}{below=4ex of wx1}
          \event{ry1}{\DR[\modeRA]{y}{1}}{right=2.5em of wy1}
          \event{rx0}{\DR{x}{0}}{right=of ry1}
          \event{rx1}{\DR[\modeRA]{x}{1}}{right=2.5 em of wx1}
          \event{ry0}{\DR{y}{0}}{right=of rx1}
          % \wk{wx0}{wx1}
          % \wk{wy0}{wy1}
          % \rf[bend left]{wy0}{ry0}
          % \rf[bend right]{wx0}{rx0}
          \sync{rx1}{ry0}
          \sync{ry1}{rx0}
          \rf{wx1}{rx1}
          \rf{wy1}{ry1}
          \wk{rx0}{wx1}
          \wk{ry0}{wy1}
        \end{tikzinline}}}
  \end{gathered}
\end{gather*}
\end{comment}

It is not possible for two \RMW{}s to see the same write.
\begin{gather*}
  \begin{gathered}
    x\GETS 0 \SEMI \bigl(\FADD^{\modeRLX,\modeRLX}(x,1) \PAR \FADD^{\modeRLX,\modeRLX}(x,1)\bigr)
    \\
    \hbox{\begin{tikzinline}[node distance=2em]
        \event{a0}{\DW{x}{0}}{}
        \event{a1}{\DR{x}{0}}{right=3em of a0}
        \event{a2}{\DW{x}{1}}{right=of a1}
        \event{b1}{\DR{x}{0}}{right=3em of a2}
        \event{b2}{\DW{x}{1}}{right=of b1}
        \rmw{a1}{a2}
        \rf{a0}{a1}
        \rf[out=-15,in=-165]{a0}{b1}
        \wk[out=-15,in=-165]{a1}{b2}
        \wk{b1}{a2}
        \graywk[bend left]{a2}{b1}
        \rmw{b1}{b2}
      \end{tikzinline}}
  \end{gathered}
  \taglabel{rmw0}
\end{gather*}
The gray arrow is required the \RMW{} atomicity axioms.

\citet{DBLP:conf/pldi/LeeCPCHLV20} introduce \ps2.0 to refine the treatment of
\RMW{}s in the promising semantics (\ps).  Their examples have the expected
results here, with far less work.  First they recall that \ps{} requires
quantification over multiple futures in order to disallow executions such as
\ref{CDRF}:
\begin{gather*}
  \taglabel{CDRF}
    \begin{gathered}
      r\GETS \FADD^{\modeRA,\modeRA}(x,1)\SEMI \IF{r{=}0}\THEN y\GETS 1 \FI
      \PAR
      r\GETS \FADD^{\modeRA,\modeRA}(x,1)\SEMI \IF{r{=}0}\THEN \IF{y}\THEN x\GETS 0 \FI \FI
      \\
      \hbox{\begin{tikzinline}[node distance=2em]
          \event{a1}{\DR[\modeRA]{x}{0}}{}
          \event{a1b}{\DW[\modeRA]{x}{1}}{below=1em of a1}
          \event{a2}{\DW{y}{1}}{right=of a1}
          \event{b0}{\DR[\modeRA]{x}{0}}{right=3em of a2}
          \event{b0b}{\DW[\modeRA]{x}{1}}{below=1em of b0}
          \event{b1}{\DR{y}{1}}{right=of b0}
          \event{b2}{\DW{x}{0}}{right=of b1}
          \rmw{a1}{a1b}
          \rmw{b0}{b0b}
          \rf[out=-13,in=-163]{a2}{b1}
          \po{a1}{a2}
          \sync{b0}{b1}
          \po{b1}{b2}
          \rf[out=-165,in=-12]{b2}{a1}
        \end{tikzinline}}
    \end{gathered}
  \end{gather*}
This execution is clearly impossible, due to the cycle above.  In this
diagram, we have not drawn order adjacent to the writes of the \RMW{}s, since
this is not necessary to produce the cycle.
  
\ps{} does not support global value range analysis, as modeled by \ref{GA+E} below.  Our
semantics permits \ref{GA+E}:
\begin{gather*}
  \taglabel{GA+E}
    \begin{gathered}
      x\GETS0 \SEMI
      \bigl(
        r\GETS \CAS^{\modeRLX,\modeRLX}(x,0,1)\SEMI \IF{r{<}10}\THEN y\GETS 1 \FI
        \PAR
        x\GETS{42}\SEMI x\GETS y
      \bigr)
      \\
      \hbox{\begin{tikzinline}[node distance=2em]
          \event{a0}{\DW{x}{0}}{}
          \event{a1}{\DR{x}{1}}{right=3em of a0}
          \event{a2}{0{<}10\mid\DW{y}{1}}{right=of a1}
          \event{b0}{\DW{x}{42}}{right=3em of a2}
          \event{b1}{\DR{y}{1}}{right=of b0}
          \event{b2}{\DW{x}{1}}{right=of b1}
          %\rmw{a1}{a2}
          \rf[out=-15,in=-160]{a2}{b1}
          \po{b1}{b2}
          \rf[out=-165,in=-15]{b2}{a1}
          \wk[out=10,in=170]{a0}{b0}
          \wk[out=15,in=165]{b0}{b2}
        \end{tikzinline}}
    \end{gathered}
\end{gather*}
\ps{} also does not support register promotion, as modeled by \ref{RP} below.    Our
semantics permits \ref{RP}:
\begin{gather*}
  \taglabel{RP}
    \begin{gathered}
      r\GETS x\SEMI
      s\GETS \FADD^{\modeRLX,\modeRLX}(z,r)\SEMI y\GETS s{+}1
      \PAR
      x\GETS y
      \\
      \hbox{\begin{tikzinline}[node distance=2em]
          \event{a0}{\DR{x}{1}}{}
          \event{a1}{\DR{z}{0}}{right=of a0}
          \event{a1b}{\DW{z}{1}}{right=of a1}
          \event{a2}{\DW{y}{1}}{right=of a1b}
          \event{b0}{\DR{y}{1}}{right=3em of a2}
          \event{b1}{\DW{x}{1}}{right=of b0}
          \rmw{a1}{a1b}
          \po[out=20,in=160]{a0}{a1b}
          \po[out=20,in=160]{a1}{a2}
          \po{b0}{b1}
          \rf{a2}{b0}
          \rf[out=-165,in=-15]{b1}{a0}
        \end{tikzinline}}
    \end{gathered}
\end{gather*}
  
\section{Proof of Efficient Implementation on ARM8}
\label{sec:arm:proof}

In this section, we develop the proof of correctness of compilation to \armeight.  

Our language can be translated to \armeight{} following
\citet{DBLP:journals/pacmpl/PodkopaevLV19}, thus using \texttt{ldr} for
relaxed read, \texttt{ldar} for $\modeRA$/$\modeSC$ reads are acquires,
\texttt{str} for relaxed write, and \texttt{stlr} for $\modeRA$/$\modeSC$
writes.  $\FENCE$ instruction can be translated to \texttt{dmb.sy}, since it
has release-acquire semantics.  Acquire fences map to \texttt{dmb.ld}, and
release fences to \texttt{dmb.sy} --- \texttt{dmb.st} does not provide order
to prior reads.

Relative to the \armeight{} specification, we have removed loops, and
read-modify-write (RMW) operations.  The implementation of RMW operations
follows \citet{DBLP:journals/pacmpl/PodkopaevLV19}.  We only omit it because
a systematic treatment includes a loop to account for the possible failure of
the RMW operation.

Given a relation $R$, $R^?$ denotes reflexive closure, $R^+$ denotes
transitive closure and $R^*$ denotes reflexive and transitive closure.  Given
relations $R$ and $S$, $R;S$ denotes composition.


The \armeight{} model is described using the following relations.
\begin{itemize}
\item $\IDR$, $\IDW$, $\IDAcq$, $\IDRel$: identity on reads, writes, acquires
  and releases.
% \item $\IDR$ identity on reads
% \item $\IDW$: identity on writes
% \item $\IDAcq$: identity on acquires
% \item $\IDRel$: identity on releases
\item $\IDLoc$: relates any two events that touch the same location.
\item $\rpox$: program order.
\item $\rdata$, $\rctrl$, $\raddr$: data, control and address dependencies.
\item $\rrfx$: reads-from. $\rrfx^{-1}$ relates each read to a matching write
  on the same location.
\item $\rco$: coherence, which is a total order on the writes to a single
  location.
\item ${\rfr}\eqdef{\rco};\rrfx^{-1}$: from-read, which relates reads to
  subsequent writes.
\end{itemize}
For any relation, the cross-thread subrelation is denoted by appending $e$;
the intra-thread subrelation is denoted by appending $i$.  For example,
${\rrfe}\eqdef{\rrfx}\setminus{\rpox}$ and ${\rrfi}\eqdef{\rrfx}\cap{\rpox}$.
The subrelation restriction attention to actions on the same location is
given by appending $\mathsf{loc}$.  For example, ${\rpoloc}\eqdef{\rpox}\cap{\IDLoc}$.

The \armeight{} model defines the following relations.
In our presentation, we have elided rules concerning fences and RMW operations.
\begin{align*}
  \tag{Extended coherence}
  {\reco} &\eqdef {\rrf} \cup {\rfr} \cup {\rco}
  \\
  \tag{Observed externally}
  {\robs} &\eqdef \smash{
    {\rrfe} \cup {\rfre} \cup {\rcoe}
  }
  \\
  \tag{Dependency order}
  {\rdob} &\eqdef \smash{({\raddr}\cup{\rdata}); {\rrfi}^?}
  \\[-1ex]
  &\cup \smash{({\rctrl}\cup{\rdata}); {\IDW}; {\rcoi}^?}
  \\[-1ex]
  &\cup \smash{{\raddr}; {\rpox}; {\IDW}}
  \\
  \tag{Barrier order}
  {\rbob} &\eqdef\smash{
    {\IDAcq}; {\rpox}
    \cup {\rpox};{\IDRel}; {\rcoi}^?
  }
  \\
  \tag{Acyclic order}
  {\rob} &\eqdef\smash{
    ({\robs} \cup {\rdob} \cup {\rbob})^+
  }
\end{align*}
\begin{definition}
  An RMW-free and fence-free execution is \emph{\armeight{}-consistent} if
  \begin{align*}&
    \tag{\textsc{$\rrfx$-completeness}}\label{rf-comp}
    \fcodom(\rrfx)=\fdom(\rreads)
    \\[-1ex]&
    \tag{\textsc{$\rco$-totality}}\label{co-tot}
    \text{For every location $\aLoc$, $\rco$ totally orders the writes of $\aLoc$}  
    \\[-1ex]&
    \tag{\textsc{sc-per-loc}}\label{sc-per-loc}
    {\rpoloc} \cup {\rrfx} \cup {\rfr} \cup {\rco}\;\text{is acyclic}
    \\[-1ex]&
    \tag{\textsc{external}}\label{external}
    {\rob}\;\text{is acyclic}
  \end{align*}
\end{definition}

% Use these to refer to the rules in text:
%\ref{rf-comp} 
%\ref{co-tot}
%\ref{sc-per-loc}
%\ref{external}


Given an execution graph $G$, we say that $\aEv$ is an \emph{internal read} if
$\aEv\in\fcodom(\mathsf{po}\cap \mathsf{rf})$.    

% We are going to translate internal reads of execution graphs into 
% internal reads of the semantics.  

From $G$ we construct a candidate pomset $\aPS$ as follows:
\begin{itemize}
\item $\Event= \textsf{E}$,
% \item $\labelingAct(\aEv)=\tau \mathsf{lab}(e)$, if $\aEv$ is a % relaxed internal read, 
\item $\labelingAct(\aEv)=\mathsf{lab}(e)$, if $\aEv$ is not a relaxed
  internal read,
\item $\labelingForm(\aEv)=\TRUE$,
\item ${\gtN}=({\reco}\cup{\robi})^*$.
\end{itemize}
The relation ${\robi}$ is defined from ${\rob}$ by restricting the order into and out of an read that is in
the codomain of the $\rrfi$ relation.  More formally, let $\bEv\xdobi\aEv$ when $\bEv\xdob\aEv$ and
$\bEv\notin\fcodom(\rrfi), \aEv \notin\fcodom(\rrfi)$.  

Let $\robi$ be defined as for $\rob$, simply replacing $\rdob$ with $\rdobi$.

% considering the definition of $\rob$ without $\rrfi$, ie.:
% \begin{align*}
 %  \tag{Dependency order}
%   {\rdob} &\eqdef\smash{
 %    ({\raddr}\cup{\rdata});
 %    \cup ({\rctrl}\cup{\rdata}); {\IDW}; {\rcoi}^?
 %    \cup {\raddr}; {\rpox}; {\IDW}
 %  }
 %  \\
 %  \tag{Barrier order}
 %  {\rbob} &\eqdef\smash{
 %    {\IDAcq}; {\rpox}
 %    \cup {\rpox};{\IDRel}; {\rcoi}^?
 %  }
 %  \\
 %  \tag{Acyclic order}
 % %  {\rob} &\eqdef\smash{
 %   ({\robs} \cup {\rdob} \cup {\rbob})^+
  % }
% \end{align*}


We show that $\aPS$ is a top-level pomset, reasoning as follows.
% We establish the criteria for a top-level memory-model pomset:
\begin{itemize}
%\item ${\le}$ is a partial order.  This holds since $G.{\rar}$ is 
%acyclic.
% \item If $\bEv \le \aEv$ then $\bEv \gtN \aEv$.  By construction.
% \item If $\bEv \le \aEv$ and $\aEv \gtN \bEv$ then $\bEv = \aEv$.  Proved below.
% \item If $\cEv \le \bEv \gtN \aEv$ or $\cEv \gtN \bEv \le \aEv$ then
%   $\cEv \gtN \aEv$. By construction.
% \end{itemize}

% Next, we establish the criteria for a 3-valued pomset with preconditions (Definition~\ref{def:3pre}).
% \begin{itemize}
\item $\labelingForm(\aEv)$ implies $\labelingForm(\bEv)$ whenever
  $\bEv\le\aEv$.   Trivial, since every formula is $\TRUE$.
% \item $\aPS$ is $\aLoc$-coherent; that is, when restricted to events that
%   read or write $\aLoc$, $\gtN$ forms a partial order.
% \end{itemize}

% Finally, we establish the criteria for a top-level pomset
% (Definition~\ref{def:x-closed}).
% \begin{itemize}
%\item $\aEv$ is location independent. Trivial, since every formula 
% is $\TRUE$.
\item If $\aEv$ reads $\aVal$ from $\aLoc$, then there is some $\bEv$ such that
  \begin{itemize}
  \item $\bEv \lt \aEv$,  
  \item $\bEv$ writes $\aVal$ to $\aLoc$, and
  \item if $\cEv$ writes to $\aLoc$
    then either $\cEv \gtN \bEv$ or $\aEv \gtN \cEv$.
  \end{itemize}    
\end{itemize}

\subsection{Proof that  $({\robi} \cup {\reco})^*$ is irreflexive. }


\begin{lemma}\label{extendob}
Let $\aEv, \bEv$ be distinct events and $\bEv'\ ({\xob}\cap {\xpox}) \ \bEv\allowbreak\ (({\xeco}\cap {\xpox}) \setminus {\xrfi}) \  \aEv\ ({\xob} \cap {\xpox})  \ \aEv'$.  Then $\bEv' \xob \aEv'$.

\begin{proof}
If $\bEv'$ is an acquire,  or $\aEv$ is an release, or $\aEv'$ is a release, result is immediate.

We next consider the case where $\aEv$ is a read.  In this case,  $\bEv$ is a write.  Since $\bEv\ ((\xeco\cap \xpox) \setminus \xrfi) \  \aEv$, there is a write $\bEv_1$ such that $ \bEv \xcoe\ \bEv_1 \ \xrfe\ \aEv' $.  So, $\bEv \xob \aEv$ and result follows in this case. 


So, it suffices to prove the following assuming that $\bEv'$ is not an acquire and $\aEv'$ is not a release and $\aEv$ is not a release or a read and $\aEv, \bEv$ are distinct.
\begin{itemize}
\item If $\bEv'\ (\xob\cap \xpox)  \ \bEv(\xeco\cap\xpox)\aEv$ then $\bEv'\xob\aEv$.
\item If $\bEv\ (\xeco\cap\xpox) \ \aEv(\xob\cap\xpox)\aEv'$ then $\bEv\xob\aEv'$.
\end{itemize}


We first prove that if $\bEv'\ (\xob \cap \xpox) \ \bEv\ (\xeco \cap \xpox) \ \aEv$ then $\bEv'\xob\aEv$.   Proof proceeds by cases on the witness for $\bEv'\ (\xob\cap \xpox) \ \bEv$. 
\begin{itemize}
\item  If $\bEv' \xbob  \bEv$, then: 
\[ \bEv'\ (\smash{
    {\IDAcq}; {\rpox}
    \cup {\rpox};{\IDRel}; {\rcoi}^?) \ 
  }
\bEv
\]
Since $\bEv'$ is not an acquire, $\bEv' ({\rpox};{\IDRel}; {\rcoi}^?) \bEv$, so $\bEv$ is a write.  Since $\aEv$ is not a read,  $\bEv \xcoi\ \aEv$. Thus, result follows.

\item If $\bEv' \xdob  \bEv$, then: 
\[ \bEv'\ 
\smash{
    ( ({\rctrl}\cup{\rdata}); {\IDW}; {\rcoi}^?
    \cup {\raddr}; {\rpox}; {\IDW}
  } \
\bEv
\]
So, $\bEv$ is a write.  Since $\aEv$ is also a write, we deduce that 
\[ \bEv'\ 
\smash{
    ( ({\rctrl}\cup{\rdata}); {\IDW}; {\rcoi}^?
    \cup {\raddr}; {\rpox}; {\IDW}
  } \
\aEv
\]
\end{itemize}


We next prove  that if $\bEv\ (\xeco \cap \xpox) \ \aEv\ (\xob\cap \xpox) \ \aEv'$ then $\bEv\xob\aEv'$, under the assumptions that  $\aEv'$ is not a release and $\aEv$ is not a release or a read and $\aEv, \bEv$ are distinct.


 Proof proceeds by cases on the witness for $\aEv (\xob\cap \xpox) \aEv'$.  

\begin{itemize}
\item  If $\aEv \xbob  \aEv'$, then: 
\[ \aEv\ (\smash{
    {\IDAcq}; {\rpox}
    \cup {\rpox};{\IDRel}; {\rcoi}^?) \ 
  }
\aEv'
\]
Since $\aEv$ is not a read, $\aEv ({\rpox};{\IDRel}; {\rcoi}^?) \aEv'$.  Result follows since  $\bEv \xpox\ \aEv$.


\item If $\aEv \xdob  \aEv'$, then $\aEv$ is a read.  \popQED
\end{itemize}
\end{proof}
\end{lemma}


We now turn to proving that $({\robi} \cup {\reco})^*$ is acyclic.  

It suffices to consider possible cycles in $({\robi} \cup {\reco})^*$ that do not involve the read events fulfilled by $\rrfi$.    This is because the read events that are fulfilled by $\rrfi$ have the following properties:
\begin{itemize}
\item Any in-edge into the event factors through an in-edge from an acquire or the fulfilling write to the read
\item Any out-edge from the event factors through an out-edge to a release
\end{itemize}
Thus, if there is a cycle involving a read event fulfilled by $\rrfi$, there is also a cycle without such a read event. 


{\em In the rest of this section, we only consider events that are not read events fulfilled by $\rrfi$.}

\begin{lemma}\label{obeco1}
If $\bEv\xobi\aEv$ then $\lnot(\aEv\xeco\bEv)$.

\begin{proof}

Proof by contradiction.  Let 
\[  \aEv \xeco \bEv \xobi \aEv  \]

At least one of $\aEv,\bEv$ is a write.  So, if $\aEv \not\xpox \bEv$, then $\aEv \xob \bEv$, and we have a cycle in $\xob$.  

So, we conclude that $\aEv \xpox \bEv$

Since $\bEv \xobi \aEv$, there exists $\bEv \not\xpox \cEv$, $\bEv (\xeco \cap \xobi)  \cEv$, $\cEv \xobi \aEv$.

We reason by cases.
\begin{itemize}
\item If $\cEv$ is a write or $\aEv$ is a write,  $\cEv \xeco \aEv$ and we have an $\xeco$ cycle.

\item Otherwise, $\aEv$ is a read, $\bEv$ is a write. 

Let $\bEv \xobi \cEv_0 \xobi \cEv_1 \ldots \cEv_n \xobi \aEv$ be the witness.   If $\cEv_n \xpox \aEv$, then $\cEv_n \xobi \bEv$ and we have an $\xob$ cycle.

So, $\cEv_n \not\xpox \aEv$.  Thus, $\cEv_n$ is the write fulfilling
$\aEv$. So, we deduce: $\cEv_n \xeco \aEv$ and $\bEv \xeco \cEv_n$ yielding an $\xeco$ cycle. \popQED
\end{itemize}
\end{proof}
\end{lemma}

\begin{lemma}\label{obeco2}
$({\robi} \cup {\reco})^*$ is irreflexive.
\end{lemma}
\begin{proof}
The simple case that $\robi; \reco$ is irreflexive is proved above.  The full proof is by contradiction.  

Let $n \geq 1$ be  such that:
\begin{align*} 
&\aEv^0_0 \xobi \aEv^0_1 \xeco \bEv^0_0 \xobi \bEv^0_1  \\
(\xeco \cap \xobi) &  \   \aEv^1_0 \xobi \aEv^1_1 \xeco \bEv^1_0 \xobi \bEv^1_1 \\
(\xeco \cap \xobi) & \ \ldots \\
& \ldots \bEv^n_1 \\
 (\xeco \cap \xobi) & \  \aEv^0_0
\end{align*}
where  for all $i$, we have:
\[ \aEv^i_0 \xpox \aEv^i_1 (\xeco \cap \xpox) \bEv^i_0 \xpox \bEv^i_1\] and 
\[ \neg (\bEv^i_1 \xpox \aEv^{(i+1) \mod n}_0 ) \]
with the proviso that we have chosen the cycle with the minimum number of events.  

For any $i$, if $\aEv^i_0 \not= \aEv^i_1$ or $\bEv^i_0 \xpox \bEv^i_1$, via lemma~\ref{extendob}, we deduce that $\aEv^i_0  \xob \bEv^i_1$, contradicting minimality of number of events in cycle.  

So, we can assume that $n \geq 1$ is such that:
\begin{align*} 
&\ \aEv^0 \xeco  \bEv^0 \\
(\xeco \cap \xob) &  \   \aEv^1  \xeco \bEv^1 \\
(\xeco \cap \xob) & \ \ldots \\
& \ldots \bEv^n \\
 (\xeco \cap \xob) & \ \aEv^0
\end{align*}
which is a contradiction since it is a cycle in $\xeco$. 
% and since at least one of $\aEv^i ,\bEv^i$ is a write for all $i$. 
\end{proof}


