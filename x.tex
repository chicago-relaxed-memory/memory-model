\RequirePackage{amssymb}  %% bizarre error previous def of Bbbk if after documentclass
% at most 12 pages, % excluding references. 
\documentclass[sigplan,10pt,
review,anonymous,
]{acmart}
\makeatletter\if@ACM@anonymous\acmSubmissionID{{}}\fi\makeatother %Including this in anonymous so page breaks are right for version with institutions
\settopmatter{printfolios=true}
\AtEndPreamble{%
  % \theoremstyle{acmdefinition}
  % \theoremstyle{plain}
  % \newtheorem{theorem}{Theorem}[section]
  % \newtheorem{proposition}[theorem]{Proposition}
  % \theoremstyle{acmdefinition}
  % \newtheorem{definition}[theorem]{Definition}
  \theoremstyle{acmdefinition}
  \newtheorem{remark}[theorem]{Remark}
  \newtheorem{candidate}[theorem]{Candidate}
  \renewcommand{\theequation}{\fnsymbol{equation}}
}
%% For single-blind review submission

% \documentclass[acmsmall]{acmart}\settopmatter{printfolios=true}
% \hypersetup{bookmarksnumbered,bookmarksopen=true,bookmarksdepth=2}
%% For final camera-ready submission
%\documentclass[acmsmall,10pt]{acmart}\settopmatter{}

%% Note: Authors migrating a paper from PACMPL format to traditional
%% SIGPLAN proceedings format should change 'acmsmall' to
%% 'sigplan'.

%\documentclass[acmsmall,screen]{acmart}


%% Some recommended packages.
\usepackage{booktabs}   %% For formal tables:
                        %% http://ctan.org/pkg/booktabs
\usepackage{subcaption} %% For complex figures with subfigures/subcaptions
                        %% http://ctan.org/pkg/subcaption



% \makeatletter\if@ACM@journal\makeatother
% %% Journal information (used by PACMPL format)
% %% Supplied to authors by publisher for camera-ready submission
% \acmJournal{PACMPL}
% \acmVolume{1}
% \acmNumber{1}
% \acmArticle{1}
% \acmYear{2017}
% \acmMonth{1}
% \acmDOI{10.1145/nnnnnnn.nnnnnnn}
% \startPage{1}
% \else\makeatother
% %% Conference information (used by SIGPLAN proceedings format)
% %% Supplied to authors by publisher for camera-ready submission
% \acmConference[PL'17]{ACM SIGPLAN Conference on Programming Languages}{January 01--03, 2017}{New York, NY, USA}
% \acmYear{2017}
% \acmISBN{978-x-xxxx-xxxx-x/YY/MM}
% \acmDOI{10.1145/nnnnnnn.nnnnnnn}
% \startPage{1}
% \fi


%% Copyright information
%% Supplied to authors (based on authors' rights management selection;
%% see authors.acm.org) by publisher for camera-ready submission
% \setcopyright{none}             %% For review submission
%\setcopyright{acmcopyright}
%\setcopyright{acmlicensed}
%\setcopyright{rightsretained}
%\copyrightyear{2017}           %% If different from \acmYear


%% Bibliography style
\bibliographystyle{ACM-Reference-Format}
%% Citation style
%% Note: author/year citations are required for papers published as an
%% issue of PACMPL.
\citestyle{acmauthoryear}   %% For author/year citations

% AFAICT this is the only way to get the copyright notice to be CC-BY-4.0
% which is the recommended license for POPL 2020.
\setcopyright{rightsretained}
\begin{makeatletter}
  \gdef\@copyrightpermission{%
    This paper is published under the Creative Commons Attribution~4.0
    International (CC-BY~4.0) license.
  }
\end{makeatletter}

\usepackage{macros}

% % at most 12 pages, excluding references, not anonymous
% % Titles and Short Abstracts Due	4 January 2019
% % Full Papers Due	11 January 2019
% % Author Feedback/Rebuttal Period	4–8 March 2019
% % Author Notification	29 March 2019
% % Conference	24–27 June 2019
% \documentclass[conference]{IEEEtran}
% \IEEEoverridecommandlockouts

\usepackage{macros}

\begin{document}

%% Title information
\title{Weak Ideas}
\author{Radha Jagadeesan}
\affiliation{
  %\department{School of Computing}
  \institution{DePaul University}
  %\country{USA}
}
%\email{rjagadeesan@cs.depaul.edu}

\author{Alan Jeffrey}
\orcid{0000-0001-6342-0318}
\affiliation{
  \institution{Mozilla Research}
  %\country{USA}
}
%\email{ajeffrey@mozilla.com}

\author{James Riely}
\orcid{0000-0002-8731-1463}
\affiliation{
  %\department{School of Computing}
  \institution{DePaul University}
  %\country{USA}
}
%\email{jriely@cs.depaul.edu}


%% Paper note
\thanks{Radha Jagadeesan and James Riely were supported in part by NSF CCR}
%% The \thanks command may be used to createame a "paper note" ---
%% similar to a title note or an author note, but not explicitly
%% associated with a particular element.  It will appear immediately
%% above the permission/copyright statement.
%                                         %% can be repeated if necesary
%                                         %% contents suppressed with 'anonymous'


%% Abstract
%% Note: \begin{abstract}...\end{abstract} environment must come
%% before \maketitle command
\begin{abstract}
Relaxed memory models must simultaneously achieve efficient implementability
and thread-compositional reasoning.  Is that why they have become so
complicated?  We argue that the answer is no: It is possible to achieve these
goals by combining an idea from the 60s (preconditions) with an idea from the
 80s (pomsets), at least for \textsc{x64} and \armeight.  We show that the
resulting model (1) supports compositional reasoning for temporal safety
properties, (2) supports all expected sequential compiler optimizations,
(3) satisfies the \drfsc\ criterion, and (4) compiles to \textsc{x64} and \armeight{}
microprocessors without requiring extra fences on relaxed accesses.

\endinput

Relaxed memory models must simultaneously achieve efficient implementability and thread-compositional reasoning.  Is that why they have become so complicated?  We argue that the answer is no: It is possible to achieve these goals by combining an idea from the 60s (preconditions) with an idea from the 80s (pomsets), at least for X64 and ARMv8.  We show that the resulting model (1) supports compositional reasoning for temporal safety properties, (2) supports all expected sequential compiler optimizations, (3) satisfies the DRF-SC criterion, and (4) compiles to X64 and ARMv8 microprocessors without requiring extra fences on relaxed accesses.



DePaul Abstract:

A memory model is a contract between a programmer and a system implementor
which indicates the allowed outcomes of any given program.  Some of the
things allowed on your computer might surprise you!

In your first systems class you learned a simple model of virtual memory: a
nice flat address space.  You also learned that this is a lie!  Memory
systems are remarkably complicated.  The situation is even more complex when
you consider the aggressive optimization performed by current compilers.
Although system designers are able to hide much of this complexity, they
can't hide it all without killing performance.  For fifteen years,
researchers have been looking for a model that is understandable to
programmers while still allowing efficient implementation.

In this talk, we present a (relatively) simple model that does almost
everything we want.  The model combines an idea from the 60s (preconditions)
with an idea from the 80s (pomsets).  We show that the resulting model (1)
supports compositional reasoning for temporal safety properties, (2) supports
all reasonable sequential compiler optimizations, (3) allows programmers to
use a simplistic model for race-free programs, and (4) compiles to X64 and
ARMv8 microprocessors without requiring extra fences on relaxed accesses.





Video material description:

This is a video presentation of our OOPSLA 2020 paper.  The paper presents a
model of relaxed memory using partial orders and preconditions.

The talk is a meditation on the notion of "Thin Air Reads" in relaxed memory
models.  We argue that the whole concept of "out of thin air" is too nebulous
to be useful.  Instead, we should concern ourselves with finding
compositional proof rules for expressive logics.

If Goldilocks were to visit the house of relaxed memory models, she would
find Promises too hot (allowing thin air reads) and Event structures too cold
(disallowing efficient implementation).  Pomsets with Preconditions are "just
right".





Info for session chairs:

This is our third major paper on relaxed memory.  In the talk I will argue
that: Promises/speculation (2010) are too hot, allowing thin air reads.
Event structures (2016) are too cold, disallowing efficient implementation.
Pomsets with Preconditions (2020) are "just right".




Twitter-length summary:

Goldilocks visits relaxed memory: Promises are too hot (allowing thin air).
Event structures are too cold (disallowing efficient implementation).
Pomsets with Preconditions are "just right".

Goldilocks visits relaxed memory: Promises too hot. Event structures too cold. Pomsets with Preconditions "just right".
https://2020.splashcon.org/details/splash-2020-oopsla/70/Pomsets-with-Preconditions-A-Simple-Model-of-Relaxed-Memory
@jriely @asajeffrey

\end{abstract}


%% 2012 ACM Computing Classification System (CSS) concepts
%% Generate at 'http://dl.acm.org/ccs/ccs.cfm'.
\begin{CCSXML}
<ccs2012>
<concept>
<concept_id>10003752.10003753.10003761.10003762</concept_id>
<concept_desc>Theory of computation~Parallel computing models</concept_desc>
<concept_significance>500</concept_significance>
</concept>
% <concept>
% <concept_id>10003752.10003753.10003761.10003763</concept_id>
% <concept_desc>Theory of computation~Distributed computing models</concept_desc>
% <concept_significance>500</concept_significance>
% </concept>
% <concept>
% <concept_id>10003752.10003753.10003761.10003764</concept_id>
% <concept_desc>Theory of computation~Process calculi</concept_desc>
% <concept_significance>500</concept_significance>
% </concept>
% <concept>
% <concept_id>10003752.10003790.10002990</concept_id>
% <concept_desc>Theory of computation~Logic and verification</concept_desc>
% <concept_significance>500</concept_significance>
% </concept>
% <concept>
% <concept_id>10003752.10010124.10010131.10010133</concept_id>
% <concept_desc>Theory of computation~Denotational semantics</concept_desc>
% <concept_significance>500</concept_significance>
% </concept>TT
% <concept>
% <concept_id>10003752.10010124.10010131.10010134</concept_id>
% <concept_desc>Theory of computation~Operational semantics</concept_desc>
% <concept_significance>500</concept_significance>
% </concept>
% <concept>
% <concept_id>10003752.10010124.10010131.10010135</concept_id>
% <concept_desc>Theory of computation~Axiomatic semantics</concept_desc>
% <concept_significance>500</concept_significance>
% </concept>
<concept>
<concept_id>10003752.10010124.10010138.10011119</concept_id>
<concept_desc>Theory of computation~Abstraction</concept_desc>
<concept_significance>500</concept_significance>
</concept>
</ccs2012>
\end{CCSXML}

\ccsdesc[500]{Theory of computation~Parallel computing models}
% \ccsdesc[500]{Theory of computation~Distributed computing models}
% \ccsdesc[500]{Theory of computation~Process calculi}
% \ccsdesc[500]{Theory of computation~Logic and verification}
% \ccsdesc[500]{Theory of computation~Denotational semantics}
% \ccsdesc[500]{Theory of computation~Operational semantics}
% \ccsdesc[500]{Theory of computation~Axiomatic semantics}
\ccsdesc[500]{Theory of computation~Abstraction}
 

%% End of generated code


%% Keywords
%% comma separated list
\keywords{Relaxed Memory Models, Hardware Transactional Memory}  %% \keywords is optional


%% \maketitle
%% Note: \maketitle command must come after title commands, author
%% commands, abstract environment, Computing Classification System
%% environment and commands, and keywords command.
\maketitle

% \title{What Happened}

% \author{
% \IEEEauthorblockN{Radha Jagadeesan}
% \IEEEauthorblockA{\textit{DePaul University}\\
%   \email{rjagadeesan@cs.depaul.edu}}
% \and
% \IEEEauthorblockN{Alan Jeffrey}
% \IEEEauthorblockA{\textit{Mozilla Research}\\
%   \email{ajeffrey@mozilla.com}}
% \and
% \IEEEauthorblockN{James Riely}
% \IEEEauthorblockA{\textit{DePaul University}\\
%   \email{jriely@cs.depaul.edu}}
% }

\clearpage
\section{Weak Ideas}
Introduce \emph{weak order} $\wle$\footnote{Note we can \emph{not} require
\begin{itemize}
\item if $\bEv\;({\le}\cup{\wle})\;\aEv$ then $\labelingSub(\bEv)$ subsumes
  $\labelingSub(\aEv)$.
\end{itemize}
This does not hold, for example, in $\sem{x\GETS1\SEMI x\GETS2}$.}.

\begin{definition}[2.1]
  A \emph{(memory order) pomset} is a tuple
  $(\Event, {\le}, {\wle}, \labeling)$: 
  \begin{itemize}
  \item $\Event$ is a set of \emph{states},
  \item ${\le}\subseteq (\Event\times\Event)$ and ${\wle}\subseteq (\Event\times\Event)$ are partial orders, 
  \item $\labeling: \Event \fun (\Formulae\times\Act)$ is a \emph{labeling},
    from which we derive functions $\labelingForm:\Event\fun\Formulae$ and $\labelingAct:\Event\fun\Act$,
  % \item $\labeling: \Event \fun (\Formulae\times\Act\times\Sub)$ is a \emph{labeling},
  %   from which we derive functions $\labelingForm:\Event\fun\Formulae$, $\labelingAct:\Event\fun\Act$, and  $\labelingSub:\Event\fun\Sub$,
  \item if $\bEv\;({\le}\cup{\wle})\;\aEv$ then $\labelingForm(\aEv)$ implies $\labelingForm(\bEv)$, and
  % \item $\bigwedge\{\labelingForm(\aEv_i)\mid\forall i,j.\; \aEv_i\xpox\aEv_j \textor \aEv_j\xpox\aEv_i\}$ is satisfiable.
  \item $\bigwedge_{\aEv}\labelingForm(\aEv)$ is satisfiable.
  \end{itemize}
  Additional stuff:
  \begin{itemize}
  \item if $\bEv\;({\le}\sequence{\wle})\;\aEv$ then $\bEv\neq\aEv$, and
  % \item if $\bEv\le\aEv$ and $\bEv$ and $\aEv$ conflict, then $\bEv\wle\aEv$,
  %\item if $\bEv\wle\aEv$ and $\bEv$ and $\aEv$ are SC, then $\bEv\le\aEv$,
  \item if $\bEv\;({\le}\sequence{\wlt}\sequence{\le})\;\aEv$, $\bEv$ is SC,
    and $\aEv$ is SC, then $\bEv\lt\aEv$.
  \end{itemize}
\end{definition}
Update the definitions to use $\wle$ instead of $\le$ in two places:
\begin{itemize}
\item the last item defining fulfillment, and
\item item 5b defining prefixing.
\end{itemize}

\begin{definition}[2.4]
  We say $\bEv$ \emph{fulfills $\aEv$ on $\aLoc$} if 
  \begin{itemize}    
  \item $\bEv$ writes
    $\aVal$ to $\aLoc$, 
  \item $\aEv$ reads $\aVal$ from $\aLoc$,
  \item
    $\bEv \le \aEv$, and
  \item
    if $\cEv$ writes to $\aLoc$ then either $\cEv \wle \bEv$ or $\aEv \wle \cEv$.
  \end{itemize}
\end{definition}

\begin{definition}[2.10]
  \label{def:prefix}
Let $(\aForm \mid \aAct) \prefix \aPSS$ be the set $\PRE(\aPSS')$ where
$\aPS'\in\aPSS'$ when
there is some $\aPS\in\aPSS$ that satisfies items 1-4 of
Definition 2.8 %\ref{def:pre-sc}
such that:
\begin{enumerate}
\item[5a.]  if $\aEv$ writes then either $\bEv\le'\aEv$ or
  $\labelingForm'(\aEv)$ implies $\labelingForm(\aEv)$,
\item[5b.] if $\bEv$ and $\aEv$ are actions in conflict, then $\bEv\wle'\aEv$,
\item[5c.] if $\bEv$ is an acquire or $\aEv$ is a release, then $\bEv \le' \aEv$, and
\item[5d.] if $\bEv$ is an SC write and $\aEv$ is an SC read, then $\bEv \le' \aEv$.
\end{enumerate}
\end{definition}


\begin{gather*}
  x\GETS0\SEMI
  y\GETS0\SEMI
  (
  x\GETS1\SEMI\aReg\GETS y
  \PAR
  y\GETS1\SEMI \aReg\GETS x)
  \\
  \tag{\textsc{sb}}\label{SB}
  \hbox{\begin{tikzinline}[node distance=1em]
      \event{wx0}{\DW{x}{0}}{}
      \event{wy0}{\DW{y}{0}}{below=of wx0}
      \event{wx}{\DW{x}{1}}{right=of wx0}
      \event{wy}{\DW{y}{1}}{right=of wy0}
      \event{ry}{\DR{y}{0}}{right=of wx}
      \event{rx}{\DR{x}{0}}{right=of wy}
      \wk{wx0}{wx}
      \wk{wy0}{wy}
      \po{wy}{rx}
      \po{wx}{ry}
      \rf{wy0}{ry}
      \rf{wx0}{rx}
      \wk{ry}{wy}
      \wk{rx}{wx}
      % \po{rx}{wy}
    \end{tikzinline}}
\end{gather*}

\begin{gather*}
  \\[-2.5ex]
  \renewcommand{\arraycolsep}{1pt}
  \hbox{\small
    $\begin{array}{ccccc}
    &x\GETS0\SEMI x\GETS 1
    &\PAR&
    r\GETS x\ACQ \SEMI s\GETS y
    %\IF{x}\THEN r\GETS y \FI
    \\
    \PAR
    &y\GETS0\SEMI y\GETS 1
    &\PAR&
    r\GETS y\ACQ \SEMI s\GETS x
    %\IF{y}\THEN s\GETS x \FI
  \end{array}$}
  \quad
  \smash{\hbox{\begin{tikzinlinesmall}[baseline=-10pt,node distance=.5em and 1em]
  \event{wx0}{\DW{x}{0}}{}
  \event{wx1}{\DW{x}{1}}{right=of wx0}
  \event{wy0}{\DW{y}{0}}{below=2ex of wx0}
  \event{wy1}{\DW{y}{1}}{right=of wy0}
  \event{ry1}{\DRAcq{y}{1}}{right=2.5em of wy1}
  \event{rx0}{\DR{x}{0}}{right=of ry1}
  \event{rx1}{\DRAcq{x}{1}}{right=2.5 em of wx1}
  \event{ry0}{\DR{y}{0}}{right=of rx1}
  \wk{wx0}{wx1}
  \wk{wy0}{wy1}
  \rf{wx1}{rx1}
  \rf[bend left]{wy0}{ry0}
  \rf{wy1}{ry1}
  \rf[bend right]{wx0}{rx0}
  \wk{rx0}{wx1}
  \wk{ry0}{wy1}
  \po{rx1}{ry0}
  \po{ry1}{rx0}
    \end{tikzinlinesmall}}}
  \\[-1ex]
\end{gather*}

\begin{gather*}
  \hbox{\small$\IF{x}\THEN y\GETS0 \FI \SEMI y\GETS1
  {\PAR}
  \IF{y}\THEN x\GETS0 \FI \SEMI x\GETS1
  $}
  \\[-.5ex]
  \hbox{\begin{tikzinlinesmall}[node distance=1em]
  \event{a1}{\DR{x}{1}}{}
  \event{a2}{\DW{y}{0}}{right=of a1}
  \po{a1}{a2}
  \event{a3}{\DW{y}{1}}{right=of a2}
  \wk{a2}{a3}
  \event{b1}{\DR{y}{1}}{right=of a3}
  \event{b2}{\DW{x}{0}}{right=of b1}
  \po{b1}{b2}
  \event{b3}{\DW{x}{1}}{right=of b2}
  \wk{b2}{b3}
  \rf{a3}{b1}
  \rf[out=173,in=7]{b3}{a1}  
    \end{tikzinlinesmall}}
\end{gather*}
\begin{gather*}
  \hbox{\small$\IF{x}\THEN y\GETS0 \FI \SEMI y\GETS1
  {\PAR}
  \IF{y}\THEN z\GETS0 \FI \SEMI z\GETS1
  {\PAR}
  \IF{z}\THEN x\GETS0 \FI \SEMI x\GETS1
  $}
  \\[-.5ex]
  \hbox{\begin{tikzinlinesmall}[node distance=1em]
  \event{a1}{\DR{x}{1}}{}
  \event{a2}{\DW{y}{0}}{right=of a1}
  \po{a1}{a2}
  \event{a3}{\DW{y}{1}}{right=of a2}
  \wk{a2}{a3}
  \event{b1}{\DR{y}{1}}{right=of a3}
  \event{b2}{\DW{z}{0}}{right=of b1}
  \po{b1}{b2}
  \event{b3}{\DW{z}{1}}{right=of b2}
  \wk{b2}{b3}
  \event{c1}{\DR{z}{1}}{right=of b3}
  \event{c2}{\DW{x}{0}}{right=of c1}
  \po{c1}{c2}
  \event{c3}{\DW{x}{1}}{right=of c2}
  \wk{c2}{c3}
  \rf{a3}{b1}
  \rf{b3}{c1}
  \rf[out=173,in=7]{c3}{a1}  
    \end{tikzinlinesmall}}
\end{gather*}

\begin{gather*}
    x\GETS1
    \PAR
    r\GETS x\SEMI   
    \FENCE^{\modeSC}\SEMI
    r\GETS y  
    \PAR
    y\GETS 1 \SEMI
    \FENCE^{\modeSC}\SEMI
    r\GETS x  
    \\[-.1ex]
  \hbox{\begin{tikzinline}[node distance=1em]
  \event{a1}{\DW{x}{1}}{}
  \event{b1}{\DR{x}{1}}{right=3em of a1}
  \event{b2}{\DFS{\modeSC}}{right=of b1}
  \po{b1}{b2}
  \event{b3}{\DR{y}{0}}{right=of b2}
  \po{b2}{b3}
  \event{c1}{\DW{y}{1}}{right=3em of b3}
  \event{c2}{\DFS{\modeSC}}{right=of c1}
  \po{c1}{c2}
  \event{c3}{\DR{x}{0}}{right=of c2}
  \po{c2}{c3}
  \wk{b3}{c1}
  \rf{a1}{b1}
  \wk[out=-170,in=-10]{c3}{a1}
    \end{tikzinline}}
  \\
    x\GETS1\SEMI   
    z\REL\GETS1\SEMI   
    \PAR
    r\ACQ\GETS z\SEMI   
    \FENCE^{\modeSC}\SEMI
    r\GETS y  
    \PAR
    y\GETS 1 \SEMI
    \FENCE^{\modeSC}\SEMI
    r\GETS x  
    \\[-.1ex]
  \hbox{\begin{tikzinline}[node distance=.7em]
  \event{a1}{\DW{x}{1}}{}
  \event{a2}{\DWRel{z}{1}}{right=of a1}
  \po{a1}{a2}
  \event{b1}{\DRAcq{z}{1}}{right=2em of a2}
  \event{b2}{\DFS{\modeSC}}{right=of b1}
  \po{b1}{b2}
  \event{b3}{\DR{y}{0}}{right=of b2}
  \po{b2}{b3}
  \event{c1}{\DW{y}{1}}{right=2em of b3}
  \event{c2}{\DFS{\modeSC}}{right=of c1}
  \po{c1}{c2}
  \event{c3}{\DR{x}{0}}{right=of c2}
  \po{c2}{c3}
  \wk{b3}{c1}
  \rf{a2}{b1}
  \wk[out=-170,in=-10]{c3}{a1}
    \end{tikzinline}}
\end{gather*}

\clearpage
\begin{small}
\bibliography{bib}
\end{small}
\end{document}

% Local Variables:
% mode: latex
% TeX-master: t
% End:
