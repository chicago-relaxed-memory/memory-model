%  13 pages of text and up to 5 pages for references and appendices, totalling no more than 18 pages

\documentclass[conference]{IEEEtran}

\usepackage[numbers]{natbib}
\usepackage{amsmath}
\usepackage{amssymb}
\usepackage{stmaryrd}
% For a version without hyperlinks, use the [draft] option.
\usepackage{hyperref}
\newcommand{\email}[1]{\href{mailto:#1}{{\UrlFont #1}}}
\usepackage{amsthm}
\theoremstyle{plain}
\newtheorem{theorem}{Theorem}[section]
\newtheorem{proposition}[theorem]{Proposition}
\theoremstyle{definition}
\newtheorem{definition}[theorem]{Definition}
% \usepackage{verbatim}
% \makeatletter
% \def\verbatim@font{\sffamily\upshape}
% \makeatother

\bibliographystyle{plainnat}
\usepackage{macros}
\newcommand{\ignore}[1]{}
\newcommand{\todo}[1]{{\color{red}\textbf{\{#1\}}}}

\pagestyle{plain}

\begin{document}

\title{The Code That Never Ran:\\Modeling Attacks on Speculative Evaluation}

\author{
\IEEEauthorblockN{Craig Disselkoen}
\IEEEauthorblockA{\textit{University of California San Diego}\\
  \textit{Mozilla Research Internship}\\
  \email{cdisselk@cs.ucsd.edu}}
\and
\IEEEauthorblockN{Radha Jagadeesan}
\IEEEauthorblockA{\textit{DePaul University}\\
  \email{rjagadeesan@cs.depaul.edu}}
\and
\IEEEauthorblockN{Alan Jeffrey}
\IEEEauthorblockA{\textit{Mozilla Research}\\
  \email{ajeffrey@mozilla.com}}
\and
\IEEEauthorblockN{James Riely}
\IEEEauthorblockA{\textit{DePaul University}\\
  \email{jriely@cs.depaul.edu}}
}


\maketitle

\begin{abstract}
  This paper studies information flow caused by speculation mechanisms
  in hardware and software.  The Spectre attack shows that there are
  practical information flow attacks which use an interaction of
  dynamic security checks, speculative evaluation and cache timing.
  Previous formal models of program execution are designed to capture
  computer architecture, rather than micro-architecture, 
  % have not been designed to model speculative evaluation,
  and so do not capture attacks such
  as Spectre.  In this paper, we propose a model based on pomsets which
  is designed to model speculative evaluation.  
  % The model provides a
  % compositional semantics for a simple shared-memory concurrent
  % language, which captures features such as data and control
  % dependencies, relaxed memory and transactions.
  The model is abstract with respect to specific micro-architectural
  features, such as caches and pipelines, yet is powerful enough to express
  known attacks such as Spectre and \textsc{Prime+Abort},
  and verify their countermeasures.
  The model also allows for
  the prediction of new information flow attacks.  We derive two such
  attacks, which exploit compiler optimizations, and validate these
  experimentally against gcc and clang.
  % We provide models for  
  % existing information flow attacks based on speculative evaluation
  % and transactions.  We also model new information flow attacks based on compiler
  % optimizations, which are experimentally validated against
  % gcc and clang.
\end{abstract}

\section{Introduction}

This paper is about some of the lies we tell when we talk about programs.

An example lie (or to be more formal, a ``leaky abstraction'') is the order
of reads and writes in a program. We pretend that these happen in the
order specified by the program text, for example
we think of the program
\(( \aLoc\GETS0 \SEMI \aLoc\GETS1 \SEMI \bLoc\GETS2 )\)
as having three sequentially ordered write events:
\begin{tikzdisplay}[node distance=1em]
  \event{wx0}{\DW{\aLoc}{0}}{}
  \event{wx1}{\DW{\aLoc}{1}}{right=of wx0}
  \event{wy2}{\DW{\bLoc}{2}}{right=of wx1}
  \po{wx0}{wx1}
  \po{wx1}{wy2}
\end{tikzdisplay}
However, due to optimizations in hardware or compilers, instructions
may be reordered, resulting in executions where the accesses of
$\aLoc$ and $\bLoc$ are independent, and the hardware
or compiler is free to reorder them:
\begin{tikzdisplay}[node distance=1em]
  \event{wx0}{\DW{\aLoc}{0}}{}
  \event{wx1}{\DW{\aLoc}{1}}{right=of wx0}
  \event{wy2}{\DW{\bLoc}{2}}{right=of wx1}
  \po{wx0}{wx1}
\end{tikzdisplay}
Instruction reordering optimizations are not problematic as long
as they are not visible to user code, that is if programs
are sequentially consistent. Unfortunately, multi-threaded
programs can often observe reorderings. For example
running the above writing thread concurrently with
an observing thread \((\IF{\bLoc}\THEN \cLoc\GETS\aLoc \FI)\)
can result in a sequentially inconsistent execution
(where we highlight the matching write for each read):
\begin{tikzdisplay}[node distance=1em]
  \node(th1){Writing thread:};
  \event{wx0}{\DW{\aLoc}{0}}{right=of th1}
  \event{wx1}{\DW{\aLoc}{1}}{right=of wx0}
  \event{wy2}{\DW{\bLoc}{2}}{right=of wx1}
  \po{wx0}{wx1}
  \node(th2)[below=2.5ex of th1]{Observing thread:};
  \event{ry2}{\DR{\bLoc}{2}}{below=of wx0}
  \event{rx0}{\DR{\aLoc}{0}}{below=of wx1}
  \event{wz0}{\DW{\cLoc}{0}}{below=of wy2}
  \po{ry2}{rx0}
  \po{rx0}{wz0}
  \rf{wx0}{rx0}
  \rf[out=210,in=35]{wy2}{ry2}
\end{tikzdisplay}

This leaky abstraction has resulted in a literature
of \emph{relaxed memory models}~\cite{Manson:2005:JMM:1047659.1040336,SevcikThesis,Jagadeesan:2010:GOS:2175486.2175503,DBLP:journals/toplas/Lochbihler13,DBLP:conf/esop/BattyMNPS15,DBLP:conf/lics/JeffreyR16,Kang-promising-2017},
which try to state precisely the memory guarantees
a compiler is expected to provide, without requiring the use of
expensive memory barriers to ensure sequential consistency.

Relaxed memory is an example of how simple models can become
complex. Instruction reordering was originally intended to be
visible only to the microarchitecture or compiler,
not to the architecture or user code.
Reordering optimizations are so important to the performance
of modern systems that hardware and programming language designers
have now accepted the complexity of relaxed memory models as the price
that has to be paid for acceptable performance.

This paper looks at another leaky abstraction:
\emph{speculative evaluation}. This is similar to reordering,
in that it is an optimization that was intended to be visible only to the microarchitecture,
but the arrival of Spectre~\cite{DBLP:journals/corr/abs-1801-01203}
shows that not only is speculation visible, it has serious
security implications.

The simplest example of speculative evaluation
is branch prediction. The expected observable behavior of
a conditional such as
\(( \IF{\aLoc} \THEN \bLoc\GETS1 \ELSE \cLoc\GETS1 \FI )\)
is that just one branch will execute, for example:
\begin{tikzdisplay}[node distance=1em]
  \event{rx0}{\DR{\aLoc}{0}}{}
  \event{wz1}{\DW{\cLoc}{1}}{right=of rx0}
  \po{rx0}{wz1}
\end{tikzdisplay}
To improve instruction throughput, hardware will often evaluate
branches speculatively, and roll back any failed speculations. For example,
hardware might incorrectly speculate that $\aLoc$ is
nonzero, speculatively execute a write to $\bLoc$,
but then roll it back and execute a write to $\cLoc$:
\begin{tikzdisplay}[node distance=1em]
  \event{rx0}{\DR{\aLoc}{0}}{}
  \nonevent{nwy1}{\DW{\bLoc}{1}}{right=of rx0}
  \event{wz1}{\DW{\cLoc}{1}}{right=of nwy1}
  \po{rx0}{nwy1}
  \po[out=30,in=150]{rx0}{wz1}
\end{tikzdisplay}
Speculation is intended only to be visible at the microarchitectural
level, but as Spectre shows, this abstraction is leaky, and in a way
that allows side-channel attacks to be mounted.

Instruction reordering and speculative evaluation are similar leaky abstractions.
Both were intended originally not to
be visible to user code, but both abstractions have leaked.
This opens some possible areas of investigation:
\begin{itemize}

\item \emph{Using ideas from relaxed memory for speculation}.  There is a
  significant literature showing how to build models of relaxed memory, for
  use in validating compilers, or proving correctness of programs.  Less
  formally, they provide programmers with a way to visualize and communicate
  the behavior of their systems.  Inspired by these models, we give a
  compositional model of program execution that includes speculation
  (\S\ref{sec:model} and \S\ref{sec:semantics}) and show how it can be used
  to model known attacks (\S\ref{sec:examples}) on branch prediction
  \cite{DBLP:journals/corr/abs-1801-01203} and transactional memory
  \cite{ChongSW18,DBLP:conf/uss/DisselkoenKPT17}.

\item \emph{Mounting attacks against speculation on relaxed memory}.
  Spectre shows how a leaky abstraction allows for the construction of
  side-channels which bypass dynamic security checks.
  %% Since defenses
  %% against buffer overflows are often dynamic checks, these attacks
  %% allow all memory in a process to be read.
  Inspired by these
  attacks, we show how to mount information flow attacks against
  compiler optimizations, both against the model (\S\ref{sec:compiler})
  and against existing compilers (\S\ref{sec:experiments}).
  Fortunately, we were only able to mount the attacks against
  ahead-of-time compilers (where optimizations require secrets
  to be known at compile-time) and not just-in-time compilers
  (which can optimize based on run-time secrets).
  With the addition of shared-memory concurrency
  to JavaScript~\cite[\S24.2]{ecma-262}, the attacks described in this paper might
  become feasible.
  %% We hope that compiler designers become as
  %% aware of information flow attacks against optimizations as their
  %% hardware designing colleagues.

\end{itemize}
Readers who wish to focus on the impact of the model can skip to \S\ref{sec:examples}
on first reading, referring to prior sections as needed.

\paragraph*{Acknowledgments}

We would like to thank the anonymous referees,
and the paper shepherd Frank Piessens, whose
comments helped greatly improve this paper.
Jagadeesan and Riely are supported by National Science Foundation CCR-1617175.

\section{Related work}

Information flow provides a formal
foundation for end-to-end security.  Informally, a program is secure
if there is no observable dependency of low-security outputs on high-security inputs.
The precise formalization of this intuitive idea has been the topic of
extensive research \cite{Sabelfeld:2006:LIS:2312191.2314769}, encompassing a variety of language
features such as non-determinism~\cite{Wittbold1990InformationFI},
concurrency~\cite{Smith:1998:SIF:268946.268975}, reactivity~\cite{O'Neill:2006:ISI:1155442.1155677}, and
probability~\cite{Gray:1992:TMF:2699806.2699811}. The static and dynamic enforcement
of these definitions in general purpose languages~\cite{myers-popl99} has % also
% been studied extensively and has
influenced language design and implementation.

A key parameter in defining information flow is the \emph{observational power} of the attacker model. Whereas the classical
input-output behavior is often an adequate foundation,
it has long been known~\cite{Lampson:1973:NCP:362375.362389,Biswas:2017:STC:3058791.3023872} that side-channels that leak
information arise from other observables such as execution time and
power consumption.
Recently, the Spectre family of attacks~\cite{DBLP:journals/corr/abs-1801-01203} has
shown that branch prediction, in conjunction with cache-timing side-channels,
allows adversaries to bypass dynamic security checks.

\citet{Chien:2018} argues that Spectre-like attacks ``extend the functional
specification of the architecture to include its detailed performance'' and
thus ``making strong assurances of application security on a computing system
requires detailed performance information.''
This approach has been pursued in the information flow literature, by
enriching language semantics with observables such as execution time and  power consumption
\cite{Zhang:2012:LCM:2345156.2254078,hyperflow}.   This approach
has also been pursued to develop model-checking techniques for Spectre-like
attacks \cite{DBLP:conf/micro/TrippelLM18}.  Like our work,
\cite{DBLP:conf/micro/TrippelLM18} recognizes the role played by adapting
techniques from relaxed memory.

In this paper, we adopt the opposite approach, attempting to understand
Spectre-like attacks as \emph{abstractly} as possible and thus to reveal the
``essence'' of Spectre.  We develop a novel model of speculative
evaluation and show that it is sufficient to both capture known attacks and
predict new attacks.  Our model is defined at the \emph{language} level,
rather than the hardware level; thus, we do not model micro-architectural
details such as caches or timing.

Relaxed memory models
\cite{SparcV9,Manson:2005:JMM:1047659.1040336,Boehm:2008:FCC:1375581.1375591,DBLP:conf/popl/ZhaoNMZ12,
  Jagadeesan:2010:GOS:2175486.2175503,Kang-promising-2017} allow
speculative execution to varying degrees. Relaxed execution is known to
affect the validity of information flow analyses
\cite{6957104,Vaughan:2012:SIF}.
In these models, a valid execution is defined
with reference to other possible executions of the program. These
models are not, however, designed for modeling Spectre-style attacks
on speculation. For example all of these models will consider the
straight-line code:
\[
  r\GETS x\SEMI s\GETS\SEC \SEMI
  a[r]\GETS 1
\]
to be the same as the conditional code:
\[\begin{array}{ll}
  r\GETS x\SEMI s\GETS\SEC \SEMI \\[\jot]
  \IF{r\EQ s} \THEN a[s]\GETS 1 \ELSE a[r]\GETS 1 \FI
\end{array}\]
and indeed an optimizing compiler might choose to rewrite
either of these programs to be the other.

An attacker can mount a Spectre-style attack on the
conditional code, for example by setting $x$ to be~$0$,
flushing the cache,
executing the program, then using timing effects to
determine if $a[1]$ is in the cache. If it is, then $\SEC$
must have been~$1$. This attack is not possible against
the straight-line code, and so any model trying to
capture Spectre must distinguish them.

Most definitions of non-interference will say that in both
programs, there is no observable dependency of the low-security
outputs ($a$) on the high-security inputs ($\SEC$) and so both programs
are safe.
  The only existing models of
non-interference which capture this information flow are ones such
as~\cite{Zhang:2012:LCM:2345156.2254078} which model
micro-architectural features such as caching and timing.

In our model, the straight-line and conditional programs are not equated, since the conditional code has the execution:
\begin{tikzdisplay}[node distance=1em]
  \event{rx0}{\DR{\aLoc}{0}}{}
  \event{rs1}{\DR{\SEC}{1}}{below=of rx0}
  \event{wa01}{\DW{a[0]}{1}}{right=3em of rx0}
  \nonevent{wa11}{\DW{a[1]}{1}}{below=of wa01}
  \po{rx0}{wa01}
  \po{rs1}{wa01}
  \po{rx0}{wa11}
  \po{rs1}{wa11}
\end{tikzdisplay}
which is not matched in the straight-line code.
Indeed, from an information-flow perspective,
this refined treatment of dependencies in conditionals identifies a novel
distinguishing feature of our model, namely that the traditional conditional
is a self-composition operator in the sense
of~\cite{Barthe:2004:SIF:1009380.1009669}.

Static analyses such as the Smith-Volpano type
system~\cite{Smith:1998:SIF:268946.268975} will reject the conditional
program, due to $a[s]\GETS 1$, in which a low-security assignment depends on
a high-security variable.  We show how to circumvent such analyses in
\S\ref{sec:spectre}.

\section{UNUSED: Old intro}
Information flow provides a formal
foundation for end-to-end security.  Informally, a program is secure
if there is no observable dependency of low-security outputs on high-security inputs.
The precise formalization of this intuitive idea has been the topic of
extensive research \cite{Sabelfeld:2006:LIS:2312191.2314769}, encompassing a variety of language
features such as non-determinism~\cite{Wittbold1990InformationFI},
concurrency~\cite{Smith:1998:SIF:268946.268975}, reactivity~\cite{O'Neill:2006:ISI:1155442.1155677}, and
probability~\cite{Gray:1992:TMF:2699806.2699811}. The static and dynamic enforcement
of these definitions in general purpose languages~\cite{myers-popl99} has % also
% been studied extensively and has
influenced language design and implementation.

A key parameter in defining information flow is the \emph{observational power} of the attacker model. Whereas the classical
input-output behavior is often an adequate foundation,
it has long been known~\cite{Lampson:1973:NCP:362375.362389,Biswas:2017:STC:3058791.3023872} that side-channels that leak
information arise from other observables such as execution time and
power consumption.
Recently, the Spectre family of attacks~\cite{DBLP:journals/corr/abs-1801-01203} has
shown that branch prediction, in conjunction with cache-timing side-channels,
allows adversaries to bypass dynamic security checks.

\citet{Chien:2018} argues that Spectre-like attacks ``extend the functional
specification of the architecture to include its detailed performance'' and
thus ``making strong assurances of application security on a computing system
requires detailed performance information.''
This approach has been pursued in the information flow literature, by
enriching language semantics with observables such as execution time and  power consumption
\cite{Zhang:2012:LCM:2345156.2254078,hyperflow}.

In this paper, we adopt the opposite approach, attempting to understand
Spectre-like attacks as \emph{abstractly} as possible and thus to reveal the
``essence'' of Spectre.  We develop a novel model of \emph{speculative
  evaluation} and show that it is sufficient to both capture known attacks and
predict new attacks.  Our model is defined at the \emph{language} level,
rather than the hardware level; thus, we do not model micro-architectural
details such as caches or timing, as in
\cite{Zhang:2012:LCM:2345156.2254078,hyperflow}.
% We try to give as
% simple a model as possible, while still capturing shared-memory concurrency
% and speculation.


There are several sources of speculative evaluation in modern computer
systems, intended to improve performance without affecting the observable
behavior of the program: Failed speculations are meant to be
undetectable. Yet, Spectre-like attacks show that failed speculations are not
always undetectable.  Our model provides a unifying mechanism to understand
these sources of speculation.  Because failed speculations are part of the
model, it is easily enriched with operators to detect operations that occur
in failed speculations.
% In the spirit of \citet{Chien:2018}, our model ``extends the functional
% specification of the programming language to include its detailed performance''.
% so that failed speculation does not affect the input-output behavior
% of the program, but may affect other observable behavior, opening an opportunity
% for side-channels:
\begin{itemize}
\item Relaxed memory models
  \cite{SparcV9,Manson:2005:JMM:1047659.1040336,Boehm:2008:FCC:1375581.1375591,DBLP:conf/popl/ZhaoNMZ12}
  allow speculative execution to varying degrees. Relaxed execution
  is known to affect the validity of information flow analyses
  \cite{6957104,Vaughan:2012:SIF}.  More troubling, relaxed memory models
  allow for the observation of control and data dependencies. This creates an
  opportunity for information flows caused by optimizing compilers, whose
  behavior is driven by dependency analysis.  Our basic model captures this
  dependency analysis.
  %% For example,
  %% $(\IF{\aReg}\THEN \aLoc\GETS1 \ELSE \aLoc\GETS1 \FI)$ can be optimized to
  %% $(\aLoc\GETS1)$, whereas  %% $(\IF{\aReg}\THEN \aLoc\GETS1 \ELSE \aLoc\GETS2 \FI)$ cannot be so
  %% optimized.
\item Pipelined micro-architectures use \emph{branch prediction}
  to speculatively execute the result of
  a conditional jump or indirect jump instruction.
  Spectre \cite{DBLP:journals/corr/abs-1801-01203} exploits
  cache timing to detect the operations performed in a mispredicted branch
  before being flushed from the pipeline.  We capture Spectre by enriching
  our language with a single operation that allows one to test whether a
  location has been touched. %, even in a failed speculation.
  %% This means,
  %% for example, that a single execution of
  %% $(\IF{\aExp}\THEN \aCmd \ELSE \bCmd \FI)$ may depend on both $\aCmd$ and
  %% $\bCmd$.  This differs from the standard semantics of the conditional, in
  %% which executions of $\aCmd$ and $\bCmd$ are disjoint.
\item Some microprocessors support transactional
  memory~\cite{ChongSW18}, where aborted transactions are meant to be
  unobservable.  \textsc{Prime+Abort}
  \cite{DBLP:conf/uss/DisselkoenKPT17} uses cache timing to detect the
  operations performed in an aborted transaction.  We capture \textsc{Prime+Abort} by enriching
  our language with transactions that abort when a location used by the
  transaction is touched outside the transaction.
\end{itemize}

%% This line of research was initiated by~\citet{Zhang:2012:LCM:2345156.2254078}.
%% Whereas they explore static annotations to address side channels in the context of hardware description languages, we explore a model of programs
%% that captures enough detail to reveal and analyze the presence of side
%% channels revealed by speculative execution.
%

Our model is based on \emph{partially ordered multisets}~\cite{GISCHER1988199,Plotkin:1997:TSP:266557.266600}
(``pomsets''), whose labels are given by read and write actions. These can be
visualized as a graph where the edges indicate dependencies, for example
$(\aReg\GETS\aLoc\SEMI \bLoc\GETS1\SEMI \cLoc\GETS\aReg+1)$
has an execution modeled by the pomset:
\begin{tikzdisplay}[node distance=1em]
  \event{rx1}{\DR{\aLoc}{1}}{}
  \event{wy1}{\DW{\bLoc}{1}}{right=of rx1}
  \event{wz2}{\DW{\cLoc}{2}}{right=of wy1}
  \po[out=25,in=155]{rx1}{wz2}
\end{tikzdisplay}
The edge from $(\DR{\aLoc}{1})$ to $(\DW{\cLoc}{2})$ indicates a
data dependency. Since there is no dependency between
$(\DW{\bLoc}{1})$ and $(\DW{\cLoc}{2})$, the write actions may
take place in either order.  Such reorderings may arise in
hardware (for example, caching) or in the compiler (for example,
instruction reordering).

The novel aspect of the model is that events have
\emph{preconditions} which may be false. These are used in giving the
semantics of conditionals and transactions, modeling failed branch
prediction and aborted transactions. For example the program
$(\IF{\aLoc}\THEN \bLoc\GETS1\SEMI\cLoc\GETS1 \ELSE \bLoc\GETS2\SEMI\cLoc\GETS1\FI)$
has an execution:
\begin{tikzdisplay}[node distance=1em]
  \event{rx1}{\DR{\aLoc}{1}}{}
  \event{wy1}{\DW{\bLoc}{1}}{right=of rx1}
  \nonevent{wy2}{\DW{\bLoc}{2}}{below=of wy1}
  \event{wz1}{\DW{\cLoc}{1}}{right=of wy1}
  \po{rx1}{wy1}
  \po{rx1}{wy2}
\end{tikzdisplay}
The edges from $(\DR{\aLoc}{1})$ to $(\DW{\bLoc}{1})$ and
$(\DW{\bLoc}{2})$ indicate control dependencies. The presence of
a crossed out $(\DW{\bLoc}{2})$ indicates an event with an unsatisfiable precondition,
modeling an unsuccessful speculation.
Since the $(\DW{\cLoc}{1})$ action is performed on both branches of the conditional,
there is no control dependency from $(\DR{\aLoc}{1})$.  Indeed, from an information-flow perspective,
this refined treatment of dependencies in conditionals identifies a novel distinguishing feature of our model, namely that the traditional conditional is a self-composition operator in the sense of~\cite{Barthe:2004:SIF:1009380.1009669}.

There do exist models of programs which include speculation, notably
the Java Memory Model~\cite{Manson:2005:JMM:1047659.1040336}, and the
generative~\cite{Jagadeesan:2010:GOS:2175486.2175503} and
promising~\cite{Kang-promising-2017} operational semantics for
relaxed memory.  In all of these models a valid execution is defined
with reference to other possible executions of the program. These
models are not, however, designed for modeling Spectre-style attacks
on speculation. For example all of these models will consider the
straight-line code:
\[
  r\GETS x\SEMI s\GETS\SEC \SEMI
  a[r]\GETS 1
\]
to be the same as the conditional code:
\[\begin{array}{ll}
  r\GETS x\SEMI s\GETS\SEC \SEMI \\[\jot]
  \IF{r\EQ s} \THEN a[s]\GETS 1 \ELSE a[r]\GETS 1 \FI
\end{array}\]
and indeed an optimizing compiler might choose to rewrite
either of these programs to be the other.

An attacker can mount a Spectre-style attack on the
conditional code, for example by setting $x$ to be~$0$,
flushing the cache,
executing the program, then using timing effects to
determine if $a[1]$ is in the cache. If it is, then $\SEC$
must have been~$1$. This attack is not possible against
the straight-line code, and so any model trying to
capture Spectre must distinguish them.

Most definitions of non-interference will say that in both
programs, there is no observable dependency of the low-security
outputs ($a$) on the high-security inputs ($\SEC$) and so both programs
are safe.
  The only existing models of
non-interference which capture this information flow are ones such
as~\cite{Zhang:2012:LCM:2345156.2254078} which model
micro-architectural features such as caching and timing.

In our model, the straight-line and conditional programs are not equated, since the conditional code has the execution:
\begin{tikzdisplay}[node distance=1em]
  \event{rx0}{\DR{\aLoc}{0}}{}
  \event{rs1}{\DR{\SEC}{1}}{below=of rx0}
  \event{wa01}{\DW{a[0]}{1}}{right=3em of rx0}
  \nonevent{wa11}{\DW{a[1]}{1}}{below=of wa01}
  \po{rx0}{wa01}
  \po{rs1}{wa01}
  \po{rx0}{wa11}
  \po{rs1}{wa11}
\end{tikzdisplay}
which is not matched in the straight-line code.

Static analyses such as the Smith-Volpano type
system~\cite{Smith:1998:SIF:268946.268975} will reject the conditional
program, due to $a[s]\GETS 1$, in which a low-security assignment depends on
a high-security variable.  We show how to circumvent such analyses in
\S\ref{sec:spectre}.


The model in this paper leads to new attacks on optimizing
compilers~(\S\ref{sec:info-flow-attack} and~\S\ref{sec:dse}) which
were discovered as a consequence of building the model. A natural
question is whether these attacks are an artifact of the model, or if
they can be mounted in practice? We mounted the attacks on gcc and
clang, where they succeeded in leaking a $\SEC$ as long as the secret
was a constant known at compile time. By itself this is not too
worrying, since secrets are not normally static constants. If the same
attacks could be mounted against Just-In-Time~(JIT) compilers, this
is potentially significant, as secrets are often known at JIT-compile
time.

Fortunately, our attempts to mount the attacks against SpiderMonkey,
V8 and HotSpot did not succeed. We speculate that this is because
JIT compilers do not perform as aggressive optimizations as
ahead-of-time compilers, and not because of mitigations against
information flows. With the addition of shared-memory concurrency
to JavaScript~\cite[\S24.2]{ecma-262}, the attacks described in this paper might
become feasible. We hope that compiler designers become as
aware of information flow attacks against optimizations as their
hardware designing colleagues.

The novel contributions of this paper are:
\begin{itemize}

\item a compositional model of program execution that includes speculation
  (\S\ref{sec:model} and \S\ref{sec:semantics}),

\item examples showing how the model can be applied,
  including existing information flow attacks on
  hardware and transactional memory, and new attacks on optimizing compilers (\S\ref{sec:examples}), and

\item experimental evidence about how practical it is to mount
  the new class of attacks (\S\ref{sec:experiments}).

\end{itemize}
Readers who wish to focus on the impact of the model can skip to \S\ref{sec:examples}
on first reading, referring to prior sections as needed.


\section{Model}
\label{sec:model}

Our model is based on \emph{partially ordered multisets}~\cite{GISCHER1988199,Plotkin:1997:TSP:266557.266600}
(``pomsets''), whose labels are given by read and write actions. These can be
visualized as a graph where the edges indicate dependencies, for example
$(\aReg\GETS\aLoc\SEMI \bLoc\GETS1\SEMI \cLoc\GETS\aReg+1)$
has an execution modeled by the pomset:
\begin{tikzdisplay}[node distance=1em]
  \event{rx1}{\DR{\aLoc}{1}}{}
  \event{wy1}{\DW{\bLoc}{1}}{right=of rx1}
  \event{wz2}{\DW{\cLoc}{2}}{right=of wy1}
  \po[out=25,in=155]{rx1}{wz2}
\end{tikzdisplay}
The edge from $(\DR{\aLoc}{1})$ to $(\DW{\cLoc}{2})$ indicates a
data dependency. Since there is no dependency between
$(\DW{\bLoc}{1})$ and $(\DW{\cLoc}{2})$, the write actions may
take place in either order.  Such reorderings may arise in
hardware (for example, caching) or in the compiler (for example,
instruction reordering).

The novel aspect of the model is that events have
\emph{preconditions}, which give the thread-local view of memory. These are used in giving the
semantics of conditionals and transactions, modeling failed branch
prediction and aborted transactions. For example the program
$(\IF{\aLoc}\THEN \bLoc\GETS1\SEMI\cLoc\GETS1 \ELSE \bLoc\GETS2\SEMI\cLoc\GETS1\FI)$
has an execution:
\begin{tikzdisplay}[node distance=1em]
  \event{rx1}{\DR{\aLoc}{1}}{}
  \event{wy1}{\DW{\bLoc}{1}}{right=of rx1}
  \nonevent{wy2}{\DW{\bLoc}{2}}{below=of wy1}
  \event{wz1}{\DW{\cLoc}{1}}{right=of wy1}
  \po{rx1}{wy1}
  \po{rx1}{wy2}
\end{tikzdisplay}
The edges from $(\DR{\aLoc}{1})$ to $(\DW{\bLoc}{1})$ and
$(\DW{\bLoc}{2})$ indicate control dependencies. The presence of
a crossed out $(\DW{\bLoc}{2})$ indicates an event with an unsatisfiable precondition,
modeling an unsuccessful speculation.
Since the $(\DW{\cLoc}{1})$ action is performed on both branches of the conditional,
there is no control dependency from $(\DR{\aLoc}{1})$.  

We give the semantics of a program as a set of pomsets with event labels of the form
$(\aForm \mid \aAct)$, where $\aForm$ is the event's precondition
(such as $\aExp=\aVal$) and $\aAct$ is the event's action (such as $\DW\aLoc\aVal$).
For example the semantics of the program $(\aLoc\GETS\aExp)$ includes the case
where $\aExp$ is $\aVal$, which is written to $\aLoc$, and is captured
by the one-event pomset:
\begin{tikzdisplay}[node distance=1em]
  \event{wxv}{\aExp=\aVal \mid \DW{\aLoc}{\aVal}}{}
\end{tikzdisplay}
We make few requirements of the logic of preconditions, save that it
includes equalities between expressions, is closed under substitution,
and supports a notion of implication.

% For example, the set of pomsets $\sem{\aReg\GETS\bLoc\SEMI \aLoc\GETS\aReg+1}$ contains:
% \begin{tikzdisplay}[node distance=1em]
%   \event{ry1}{\DR{\bLoc}{1}}{}
%   \event{wx2}{\DW{\aLoc}{2}}{right=of ry1}
%   \po{ry1}{wx2}
% \end{tikzdisplay}
The semantics is defined compositionally. As an example, we show how to 
construct one of the pomsets in
$\sem{\aReg\GETS\bLoc\SEMI \aLoc\GETS\aReg+1}$.
First, $\sem{\aLoc\GETS\aReg+1}$
contains the pomset:
\begin{tikzdisplay}[node distance=1em]
  \event{wx2}{\aReg=1 \mid \DW{\aLoc}{2}}{}
\end{tikzdisplay}
Next, we perform the substitution of $\aReg$ with $1$ in every precondition,
to get that $\sem{\aLoc\GETS\aReg+1}[1/\aReg]$
contains the pomset:
\begin{tikzdisplay}[node distance=1em]
  \event{wx2}{1=1 \mid \DW{\aLoc}{2}}{}
\end{tikzdisplay}
and since $(1=1)$ is a tautology, we elide it:
\begin{tikzdisplay}[node distance=1em]
  \event{wx2}{\DW{\aLoc}{2}}{}
\end{tikzdisplay}
This substitution is performed in defining
$\sem{\aReg\GETS\bLoc\SEMI \aLoc\GETS\aReg+1}$, which contains
the pomset:
\begin{tikzdisplay}[node distance=1em]
  \event{ry1}{\DR{\bLoc}{1}}{}
  \event{wx2}{\DW{\aLoc}{2}}{right=of ry1}
  \po{ry1}{wx2}
\end{tikzdisplay}
There is an ordering $(\DR{\bLoc}{1}) \lt (\DW{\aLoc}{2})$ (represented pictorially as an arrow)
because the precondition $(\aReg=1)$ depends on $\aReg$. If the precondition
was independent of $\aReg$ then there would be no ordering, for example
$\sem{\aReg\GETS\bLoc\SEMI \aLoc\GETS\aReg+1-\aReg}$ contains
the pomset:
\begin{tikzdisplay}[node distance=1em]
  \event{ry1}{\DR{\bLoc}{1}}{}
  \event{wx1}{\DW{\aLoc}{1}}{right=of ry1}
\end{tikzdisplay}
since the precondition $(\aReg+1-\aReg=1)$ is independent of $\aReg$.

The main novelty of our semantics is the use of preconditions, which allow us
to provide an unusual model of conditionals. In most
models, an execution of
$\sem{\IF{\aExp}\THEN \aCmd \ELSE \bCmd \FI}$ would either be
given by an execution from $\sem{\aCmd}$ or from $\sem{\bCmd}$, but not both.
In our semantics, a pomset
in $\sem{\IF{\aExp}\THEN \aCmd \ELSE \bCmd \FI}$ may include
both a pomset from $\sem{\aCmd}$ \emph{and} a pomset from $\sem{\bCmd}$.
For example, $\sem{\IF{\aExp}\THEN \aLoc\GETS1 \ELSE \aLoc\GETS2 \FI}$
contains:
\begin{tikzdisplay}[node distance=1em]
  \event{wx1}{\aExp\neq0 \mid \DW{\aLoc}{1}}{}
  \event{wx2}{\aExp=0    \mid \DW{\aLoc}{2}}{right=of wx1}
\end{tikzdisplay}
that is we have behavior from both branches of execution.

Moreover, two events representing the same action on both sides of a
conditional can be merged, producing a single event.
The precondition of the merged event is the disjunction of the preconditions
of the original events.
For example
$\sem{\IF{\aExp}\THEN \aLoc\GETS1\SEMI \bLoc\GETS3 \ELSE \aLoc\GETS2\SEMI \bLoc\GETS3 \FI}$
contains:
\begin{tikzdisplay}[node distance=1em]
  \event{wx1}{\aExp\neq0 \mid \DW{\aLoc}{1}}{}
  \event{wx2}{\aExp=0    \mid \DW{\aLoc}{2}}{right=of wx1}
  \event{wy3}{(\aExp\neq0) \lor (\aExp=0) \mid \DW{\bLoc}{3}}{below=3ex of $(wx1)!0.5!(wx2)$}
\end{tikzdisplay}
and since $(\aExp\neq0) \lor (\aExp=0)$ is a tautology, this is:
\begin{tikzdisplay}[node distance=1em]1
  \event{wx1}{\aExp\neq0 \mid \DW{\aLoc}{1}}{}
  \event{wx2}{\aExp=0    \mid \DW{\aLoc}{2}}{right=of wx1}
  \event{wy3}{\DW{\bLoc}{3}}{right=of wx2}
\end{tikzdisplay}

Combining this model of conditionals with the previously discussed model of memory using substitutions
gives that
$\sem{\IF{\cLoc}\THEN \aLoc\GETS1\SEMI \bLoc\GETS3 \ELSE \aLoc\GETS2\SEMI \bLoc\GETS3 \FI}$
contains:
\begin{tikzdisplay}[node distance=1em]
  \event{rz1}{\DR{\cLoc}{1}}{}
  \event{wx1}{1\neq0 \mid \DW{\aLoc}{1}}{right=of rz1}
  \event{wx2}{1=0    \mid \DW{\aLoc}{2}}{right=of wx1}
  \event{wy3}{\DW{\bLoc}{3}}{right=of wx2}
  \po{rz1}{wx1}
  \po[out=25,in=155]{rz1}{wx2}
\end{tikzdisplay}
and we visualize unsatisfiable preconditions as crossed out:
\begin{tikzdisplay}[node distance=1em]
  \event{rz1}{\DR{\cLoc}{1}}{}
  \event{wx1}{\DW{\aLoc}{1}}{right=of rz1}
  \nonevent{wx2}{\DW{\aLoc}{2}}{right=of wx1}
  \event{wy3}{\DW{\bLoc}{3}}{right=of wx2}
  \po{rz1}{wx1}
  \po[out=25,in=155]{rz1}{wx2}
\end{tikzdisplay}
Note that this semantics captures control dependencies
such as $(\DR\cLoc1)\lt(\DW\aLoc1)$, independencies
such as $(\DR\cLoc1)\not\lt(\DW\bLoc3)$, and failed
speculations such as the crossed out $(\DW\aLoc2)$.

In summary, the features we need of the underlying data model are:
\begin{itemize}
\item \emph{actions}, which may read or write memory locations, and
\item \emph{preconditions}, which are closed under substitution.
\end{itemize}
In rest of this section we make data models precise %in~\S\ref{sec:preliminaries},
and define pomsets.  In the next section we give the semantics of a simple
imperative language as sets of pomsets.
% in \S\ref{sec:pomsets}.
% In \refapp{sets-of-pomsets}, we define
% operations on sets of pomsets, which are used in~\S\ref{sec:semantics}
% to give a compositional semantics for
% a simple imperative language.

\subsection{Data models}
\label{sec:preliminaries}

A \emph{data model} consists of:
\begin{itemize}
\item a set of \emph{memory locations} $\Loc$, ranged over by
  $\aLoc$ and $\bLoc$,
\item a set of \emph{registers} $\Reg$, ranged over by
  $\aReg$ and $\bReg$,
\item a set of \emph{values} $\Val$, ranged over by
  $\aVal$ and $\bVal$,
\item a set of \emph{expressions} $\Exp$, ranged over by
  $\aExp$ and $\bExp$,
\item a set of \emph{logical formulae} $\Formulae$, ranged over by
  $\aForm$ and $\bForm$, and
\item a set of \emph{actions} $\Act$, ranged over by $\aAct$ and $\bAct$,
\end{itemize}
such that:
\begin{itemize}
\item values include at least the constants $0$ and $1$,
\item expressions include at least registers and values,
\item expressions are closed under substitutions of the form $\aExp[\bExp/\aReg]$,
\item formulae include at least $\TRUE$, $\FALSE$, and equalities of the form $(\aExp=\bExp)$ and $(\aLoc=\bExp)$,
\item formulae are closed under negation, conjunction, disjunction,
\item formulae are closed under substitutions of the form $\aForm[\aLoc/\aReg]$ or $\aForm[\bExp/\aLoc]$,
\item there is a relation $\vDash$ between formulae, and
\item there are partial functions $\rreads$ and $\rwrites: \Act \fun (\Loc \times \Val)$.
\end{itemize}
We shall say $\aAct$ \emph{reads} $\aVal$ \emph{from} $\aLoc$ whenever
$\rreads(\aAct) = (\aLoc,\aVal)$, and
$\aAct$ \emph{writes} $\aVal$ \emph{to} $\aLoc$ whenever
$\rwrites(\aAct) = (\aLoc,\aVal)$.
We shall say $\aForm$ \emph{implies} $\bForm$ whenever $\aForm\vDash\bForm$,
$\aForm$ is a \emph{tautology} whenever $\TRUE\vDash\aForm$,
$\aForm$ is \emph{unsatisfiable} whenever $\aForm\vDash\FALSE$, and
$\aForm$ is \emph{independent of $\aLoc$} whenever $\aForm \vDash \aForm[\aVal/\aLoc] \vDash \aForm$ for every $\aVal$.
In examples, the actions are of the form $(\DR{\aLoc}{\aVal})$, which reads $\aVal$ from $\aLoc$,
and $(\DW{\aLoc}{\aVal})$, which writes $\aVal$ to $\aLoc$.

\subsection{3-valued pomsets}
\label{sec:pomsets}

Recall the definition of a pomset from~\cite{GISCHER1988199}:
\begin{definition}
  A \emph{pomset} $(\Event, {\le}, \labeling)$ with alphabet $\Alphabet$
  is a partial order $(\Event, {\le})$ together with
  $\labeling: \Event \fun \Alphabet$.
\end{definition}
Going forward, we fix the alphabet $\Alphabet=(\Formulae\times\Act)$.
We will write $(\aForm \mid \aAct)$ for the pair $(\aForm,\aAct)$,
elide $\aForm$ when $\aForm$ is a tautology, and write $\aAct$ crossed-out ($\NEVER\aAct$)
when $\aForm$ is unsatisfiable.
We lift terminology from logical formulae and actions to events,
for example if $\labeling(\aEv)=(\aForm\mid\aAct)$
then we say
$\aEv$ is unsatisfiable whenever $\aForm$ is unsatisfiable,
$\aEv$ writes $\aVal$ to $\aLoc$ whenever $\aAct$ writes $\aVal$ to $\aLoc$, and
so forth.
We visualize a pomset as a graph where the nodes are drawn from
$\Event$, each node $\aEv$ is labeled with $\labeling(\aEv)$,
and an edge $\bEv \rightarrow \aEv$ corresponds to an ordering
$\bEv\le\aEv$. For example:
\begin{tikzdisplay}[node distance=1em]
  \event{rx1}{\DR{\aLoc}{1}}{}
  \nonevent{wy0}{\DW{\bLoc}{0}}{right=of rx1}
  \event{wy1}{\DW{\bLoc}{1}}{right=of wy0}
  \po{rx1}{wy0}
  \po[out=30,in=150]{rx1}{wy1}
\end{tikzdisplay}
is a visualization of the pomset where:
\[\begin{array}{c}
  E = \{ 0,1,2 \} \quad
  0 \le 1 \quad
  0 \le 2 \quad
  \labeling(0) = (\TRUE, \DR{\aLoc}{1}) \\
  \labeling(1) = (\FALSE, \DW{\bLoc}{0}) \quad
  \labeling(2) = (\TRUE, \DW{\bLoc}{1}) \quad
\end{array}\]

We are building a compositional semantics of shared memory
concurrency, which means we require a notion of when
a read has a matching write. This is a property we require
of closed programs, but \emph{not} of open programs.
For example a program whose semantics includes:
\begin{tikzdisplay}[node distance=1em]
  \event{wx1}{\DW{\aLoc}{1}}{}
  \event{rx0}{\DR{\aLoc}{0}}{right=2.5em of wx1}
  \event{wy0}{\DW{\bLoc}{0}}{right=of rx0}
  \nonevent{wy1}{\DW{\bLoc}{1}}{right=of wy0}
  \po{rx0}{wy0}
  \po[out=30,in=150]{rx0}{wy1}
\end{tikzdisplay}
may be put in parallel
with another program which writes $0$ to $\aLoc$.
If the program is closed with respect to $\aLoc$ though, such an execution cannot exist,
so we need each read of $\aLoc$ to have a matching write.
This is captured by defining when $\aEv$ \emph{reads $\aLoc$ from} $\bEv$~\cite{alglave}.
A preliminary definition (which, as we shall see, needs to be strengthened) is:
\begin{itemize}
\item $\bEv \lt \aEv$,
\item $\aEv$ implies $\bEv$,
\item $\bEv$ writes $\aVal$ to $\aLoc$,
  and $\aEv$ reads $\aVal$ from $\aLoc$, and
\item there is no $\bEv \lt \cEv \lt \aEv$ such that
  $\cEv$ writes to $\aLoc$.
\end{itemize}
% In diagrams, for readability we often highlight the reads-from edges,
% for example:
%% We visualize rf-pomsets by drawing a dashed edge between nodes in $\RF$,
%% labeled with the memory location,
%% for example:
%% \begin{tikzdisplay}[node distance=1em]
%%   \event{wx1}{\DW{\aLoc}{1}}{}
%%   \event{x1}{\DR{\aLoc}{1}}{right=5em of wx1}
%%   \nonevent{y0}{\DW{\bLoc}{0}}{below left=of x1}
%%   \event{y1}{\DW{\bLoc}{1}}{below right=of x1}
%%   \rfx{wx1}{x}{x1}
%%   \po{x1}{y0}
%%   \po{x1}{y1}
%% \end{tikzdisplay}
%% In most cases, the memory location is obvious from context,
%% so we elide it:
% \begin{tikzdisplay}[node distance=1em]
%   \event{wx1}{\DW{\aLoc}{1}}{}
%   \event{x1}{\DR{\aLoc}{1}}{right=2.5em of wx1}
%   \nonevent{y0}{\DW{\bLoc}{0}}{right=of x1}
%   \event{y1}{\DW{\bLoc}{1}}{right=of y0}
%   \rf{wx1}{x1}
%   \po{x1}{y0}
%   \po[out=30,in=150]{x1}{y1}
% \end{tikzdisplay}
Unfortunately by itself, this is not enough. The problem is the final
clause saying that there does not exist an $\aLoc$-\emph{blocking}
event $\cEv$ between $\bEv$ and $\aEv$. Unfortunately, concurrency can
turn events that were not $\aLoc$-blockers into an $\aLoc$-blocker,
\emph{even if the new thread does not mention $\aLoc$.}
We give an example to show this in \refapp{blockers}.
This is a problem in that it means the preliminary model violates
\emph{scope extrusion}~\cite{Milner:1999:CMS:329902},
in that we can find programs $\aCmd$ and $\bCmd$ such that
$\sem{\VAR\aLoc\SEMI(\aCmd\PAR\bCmd)}$ is not the same as
$\sem{(\VAR\aLoc\SEMI\aCmd)\PAR\bCmd}$, even if $\bCmd$ does not mention~$\aLoc$.

There are a number of ways this can be addressed; for example,
in models such as~\cite{Batty:2011:MCC:1926385.1926394} the reads-from relation is taken
as a primitive. In this paper, we propose \emph{3-valued pomsets}
as a solution. These are pomsets in which, in addition to positive statements
$(\bEv \lt \aEv)$ (interpreted as $\aEv$ depends on $\bEv$),
we also have negative statements $(\aEv \gtN \bEv)$
(interpreted as $\aEv$ cannot depend on $\bEv$).

\begin{definition}
  A \emph{3-valued pomset} $(\Event, {\le}, {\gtN}, \labeling)$ 
  is a \emph{pomset} $(\Event, {\le}, \labeling)$
  together with ${\gtN} \subseteq (\Event\times\Event)$ such that:
  \begin{itemize}
  \item if $\bEv \le \aEv$ then $\bEv \gtN \aEv$,
  \item if $\bEv \le \aEv$ and $\aEv \gtN \bEv$ then $\bEv = \aEv$,
  \item if $\cEv \ge \aEv \gtN \bEv$ or $\bEv \gtN \cEv \ge \aEv$ then $\aEv \gtN \cEv$.
  \end{itemize}
\end{definition}

% \begin{definition}
%   A \emph{3-valued poset} $(\Event,{\le},{\gtN})$ is a poset $(\Event,{\le})$
%   together with ${\gtN} \subseteq (\Event\times\Event)$ such that:
%   \begin{itemize}
%   \item if $\bEv \le \aEv$ then $\bEv \gtN \aEv$,
%   \item if $\bEv \le \aEv$ and $\aEv \gtN \bEv$ then $\bEv = \aEv$,
%   \item if $\cEv \ge \aEv \gtN \bEv$ or $\bEv \gtN \cEv \ge \aEv$ then $\aEv \gtN \cEv$.
%   \end{itemize}
% \end{definition}

% \begin{definition}
%   A \emph{3-valued pomset} $(\Event, {\le}, {\gtN}, \labeling)$
%   is a 3-valued poset $(\Event, {\le}, {\gtN})$ and
%   a pomset $(\Event, {\le}, \labeling)$.
% \end{definition}

Structures similar to 3-valued pomsets have come up in many guises, for example
rough sets~\cite{Pawlak1982} or ultrametrics over
$\{0,{}^1\!/_2,1\}$. They correspond to axioms A1--A3 of Lamport's
\emph{system executions}~\cite{DBLP:journals/dc/Lamport86}.
They are the notion of pomset given by interpreting
$\bEv\le\aEv$ in a 3-valued logic~\cite{Urquhart1986}. 

In diagrams, we visualize $(\bEv \gtN \aEv)$ as a dashed
arrow from $\bEv$ to $\aEv$ (note the change of direction).
We refer to edges introduced by $(\bEv \lt \aEv)$ as
\emph{strong edges} and by $(\bEv \gtN \aEv)$
as \emph{weak edges}.
For readability, we often highlight the reads-from edges as well.
% for example:
For example one execution of
\(
  (\aLoc\GETS0\SEMI\aLoc\GETS1) \PAR (\aLoc\GETS\aLoc+1)
\)
is:
\begin{tikzdisplay}[node distance=1em]
  \event{wx0}{\DW{\aLoc}{0}}{}
  \event{wx1}{\DW{\aLoc}{1}}{right=of wx0}
  \event{rx1}{\DR{\aLoc}{1}}{right=2.5 em of wx1}
  \event{wx2}{\DW{\aLoc}{2}}{right=of rx1}
  \rf{wx1}{rx1}
  \wk{wx0}{wx1}
  \wk{rx1}{wx2}
\end{tikzdisplay}
We strengthen the definition of reads-from to require not just that
no blocker exists, but that any candidate blocker must either
have $\cEv \gtN \bEv$ or $\aEv \gtN \cEv$. This ensures that any
further concurrency cannot turn a non-blocker into a blocker.
\begin{definition}\label{def:rf}
  In a 3-valued pomset, $\aEv$ \emph{can read $\aLoc$ from} $\bEv$ whenever: 
  \begin{itemize}
  \item $\bEv \lt \aEv$,
  \item $\aEv$ implies $\bEv$,
  \item $\bEv$ writes $\aVal$ to $\aLoc$,
    and $\aEv$ reads $\aVal$ from $\aLoc$, and
  \item if $\cEv$ writes to $\aLoc$
    then either $\cEv \gtN \bEv$ or $\aEv \gtN \cEv$.
  \end{itemize}
\end{definition}
One of the requirements of closed programs is that
every read event reads from a write event.

In the remainder of the paper, we drop the prefix ``3-valued'', referring to
3-valued pomsets simply as \emph{pomsets}.

\section{Semantics of programs}
\label{sec:semantics}

\begin{figure*}
\begin{eqnarray*}
  \sem{\SKIP} & = & \{ \emptyset \} \\
  \sem{\aLoc\GETS\aExp\SEMI \aCmd} & = & \textstyle\bigcup_\aVal\; \bigl((\aExp=\aVal) \guard (\DW\aLoc\aVal) \prefix \sem{\aCmd}[\aExp/\aLoc]\bigr) \\
  \sem{\aReg\GETS\aLoc\SEMI \aCmd} & = & \sem{\aCmd}[\aLoc/\aReg] \cup \textstyle\bigcup_\aVal\; (\DR\aLoc\aVal) \prefix \sem{\aCmd}[\aLoc/\aReg] \\
  \sem{\IF{\aExp} \THEN \aCmd \ELSE \bCmd \FI} & = & \bigl((\aExp \neq 0) \guard \sem{\aCmd}\bigr) \parallel \bigl((\aExp=0) \guard \sem{\bCmd}\bigr) \\
  \sem{\aCmd \PAR \bCmd} & = & \sem{\aCmd} \parallel \sem{\bCmd} \\
  \sem{\VAR\aLoc\SEMI \aCmd} & = & \nu \aLoc \DOT \sem{\aCmd}
\end{eqnarray*}
\caption{Semantics of a concurrent shared-memory language}
\label{fig:programs}
\end{figure*}

In Figure~\ref{fig:programs}, we give the semantics of a simple shared-memory
concurrent language as sets of pomsets.  
Each pomset
$\aPS\in\sem{\aCmd}$ represents a single execution of $\aCmd$.  We do not
expect $\sem{\aCmd}$ to be prefixed closed; thus, one may view each
$\aPS\in\sem{\aCmd}$ as a \emph{completed} execution.  However, the sets of
pomsets given by our semantics \emph{are} closed with respect to
augmentation, which may create additional order and strengthening
preconditions:
\begin{definition}
  $\aPS'$ is an augmentation of $\aPS$ if $\Event'=\Event$, $\aEv\le\bEv$
  implies $\aEv\le'\bEv$, $\aEv\gtN\bEv$ implies $\aEv\gtN'\bEv$, and
  % $\labeling'(\aEv)=\labeling(\aEv)$
  if $\labeling(\aEv) = (\bForm \mid \bAct)$ then
  $\labeling'(\aEv) = (\bForm' \mid \bAct)$ where $\bForm'$ implies
  $\bForm$.
\end{definition}

We give the semantics using combinators over sets of pomsets, defined in
\refapp{sets-of-pomsets}.  Using $\aPSS$ to range over sets of pomsets, these
are:
\begin{itemize}
\item \emph{restriction} $\nu\aLoc\DOT\aPSS$, which filters $\aPSS$ to include
  only pomsets where every event $\aEv$ that reads from $\aLoc$ \emph{can read} from some
  $\bEv$, following Definition~\ref{def:rf},
  and where no precondition can depend on $\aLoc$,

\item \emph{guarding} $\aForm\guard\aPSS$, which filters $\aPSS$,
  keeping pomsets whose events have preconditions that imply $\aForm$,
\item \emph{substitution} $\aPSS[\aExp/\aLoc]$, which replaces $\aLoc$ with
  $\aExp$ in every precondition of $\aPSS$,
\item \emph{composition} $\aPSS_1\parallel\aPSS_2$, which unions pomsets from
  $\aPSS_1$ and $\aPSS_2$, allowing events to be merged, and
\item \emph{prefixing} $\aAct\prefix\aPSS$, which adds an event with action
  $\aAct$ to pomsets in $\aPSS$, ordering $\aAct$ before any $\aEv$ whose predicate
  depends on the value read by $\aAct$.
\end{itemize}
These operations are similar to those from models of concurrency such
as~\cite{Brookes:1984:TCS:828.833}, but adapted here to the setting of
speculative evaluation.

Restriction and guarding filter the set of pomsets; we have
$(\nu\aLoc\DOT\aPSS)\subseteq\aPSS$ and $(\aForm\guard\aPSS)\subseteq\aPSS$.
Substitution updates the preconditions in a pomset, thus we expect the number
of pomsets to be unchanged; in addition, the number of events in each of the
pomsets is unchanged.
% : $|\aPSS[\aExp/\aLoc]|=|\aPSS|$
The most interesting operators are composition and prefixing, which create
larger pomsets from smaller ones.


Composition is used in giving the semantics for conditionals and concurrency.
$\aPSS_1 \parallel \aPSS_2$ contains the union of pomsets from $\aPSS_1$ and
$\aPSS_2$, allowing overlap as long as they agree on actions. For example, if
$\aPSS_1$ and $\aPSS_2$ contain:
\begin{tikzdisplay}[node distance=1em]
  \event{a}{\aForm \mid \aAct}{}
  \event{b}{\bForm_1 \mid \bAct}{right=of a}
  \po{a}{b}
\end{tikzpicture}\qquad\qquad\begin{tikzpicture}[node distance=1em]
  \event{b}{\bForm_2 \mid \bAct}{}
  \event{c}{\cForm \mid \cAct}{right=of b}
  \wk{b}{c}
\end{tikzdisplay}
then $\aPSS_1 \parallel \aPSS_2$ contains:
\begin{tikzdisplay}[node distance=1em]
  \event{a}{\aForm \mid \aAct}{}
  \event{b}{\bForm_1 \lor \bForm_2 \mid \bAct}{right=of a}
  \event{c}{\cForm \mid \cAct}{right=of b}
  \po{a}{b}
  \wk{b}{c}
\end{tikzdisplay}

Prefixing is used in giving the semantics of reads and writes.
$\aAct\prefix\aPSS$ adds a new event $\cEv$ with action $\aAct$ to each
pomset in $\aPSS$.  As in the definition of parallel composition, the
definition allows the new event to overlap with events in $\aPSS$ as long as
they agree on the action.

If $\cEv$ writes to a location that is also written by $\aEv$ in $\aPSS$,
then prefixing introduces weak order between them: $\aEv \gtN \cEv$.  This
ensures that these writes cannot be given the reverse order in an augmentation.

If $\cEv$ reads from a location that occurs in the predicate of $\aEv$, then
prefixing introduces order from $\cEv$ to any $\aEv$
whose predicate depends on $\aLoc$. 
For example, if $\aAct$ and $\bAct$ write to the same location, $\aAct$ reads
$\aVal$ from $\aLoc$, $\bForm$ is independent of $\aLoc$, and $\aPSS$
contains:
\begin{tikzdisplay}[node distance=1em]
  \event{b}{\bForm \mid \bAct}{}
  \event{c}{\cForm \mid \cAct}{right=of b}
  \po{b}{c}
\end{tikzdisplay}
then $\aAct\prefix\aPSS$ contains:
\begin{tikzdisplay}[node distance=1em]
  \event{a}{\aForm \mid \aAct}{}
  \event{b}{\bForm \mid \bAct}{right=of a}
  \event{c}{\cForm[\vec\aVal/\vec\aLoc] \mid \cAct}{right=of b}
  \po[out=25,in=155]{a}{c}
  \wk{a}{b}
  \po{b}{c}
\end{tikzdisplay}

%% A write generates a write event that may be visible
%% to other threads.  A read may see a
%% thread-local value, or it may generate a read event that must be justified by
%% another thread.  In the latter case, occurrences of $\aReg$ are replaced with
%% $\aLoc$ (rather than $\aVal$) to ensure that dependencies are tracked
%% properly.  The subsequent substitution of $\aVal$ for $\aLoc$ occurs in
%% Definition~\ref{def:prefix} of prefixing.



% We have completed the formal definition of our model of speculative
% evaluation, and now turn to examples.

In the remainder of this section, we give examples to explain the semantics,
concentrating on reads and conditionals.  Security-relevant examples begin in
\S\ref{sec:examples}.

\subsection{Sequential memory accesses}
\label{sec:sequential-memory}

In the semantics of memory, there are two very different ways memory can be
accessed: sequentially or concurrently.  These are modeled differently, since
hardware and compilers give very different guarantees about their behavior.
In the semantics of $\sem{\aReg\GETS\aLoc\SEMI \aCmd}$, given in
Figure~\ref{fig:programs}, these are found on left and right sides of the
union operation.  In this section, we discuss the sequential semantics,
$\sem{\aCmd}[\aLoc/\aReg]$, leaving the concurrent semantics to
\S\ref{sec:concurrent-memory}.

Consider the program $(\aLoc\GETS0\SEMI \bLoc\GETS\aLoc+1)$.  One execution of
this program is where the write to $y$ uses the sequential value of
$x$, which is $0$:
\begin{tikzdisplay}[node distance=1em]
  \event{wx0}{\DW{x}{0}}{}
  \event{wy1}{\DW{y}{1}}{right=of wx0}
\end{tikzdisplay}
To see how this execution is modeled, we first
expand out the syntax sugar to get the program
$(\aLoc\GETS0\SEMI \aReg\GETS\aLoc\SEMI \bLoc\GETS\aReg+1\SEMI\SKIP).$
Now $\sem{\SKIP}$ is just $\{\emptyset\}$, and
$\sem{y \GETS r+1\SEMI \SKIP}$ includes:
\[
   (r+1=1) \guard (\DW y1) \prefix \sem{\SKIP}[1/y]
\]
which contains the pomset:
\begin{tikzdisplay}[node distance=1em]
  \event{wy1}{r+1=1 \mid \DW{y}{1}}{}
\end{tikzdisplay}
expressing that this program can write $1$ to $y$,
as long as the precondition $(r+1=1)$ is satisfied.
Now $\sem{r \GETS x\SEMI y \GETS r+1\SEMI \SKIP}$
has two cases, the sequential case
(which does not introduce a read action)
and the concurrent case (which does).
For the moment, we are interested in the sequential case:
\[
   \sem{y \GETS r+1\SEMI \SKIP}[x/r]
\]
which contains the pomset:
\begin{tikzdisplay}[node distance=1em]
  \event{wy1}{x+1=1 \mid \DW{y}{1}}{}
\end{tikzdisplay}
In this pomset, the precondition is $(x+1=1)$, which specifies a property
of the thread-local value of $x$.
Finally $\sem{x \GETS 0\SEMI r \GETS x\SEMI y \GETS r+1\SEMI \SKIP}$ includes:
\[
   (0=0) \guard (\DW x0) \prefix \sem{r \GETS x\SEMI y \GETS r+1\SEMI \SKIP}[0/x]
\]
which contains the pomset:
\begin{tikzdisplay}[node distance=1em]
  \event{wx0}{0=0 \mid \DW{x}{0}}{}
  \event{wy1}{0=0\land0+1=1 \mid \DW{y}{1}}{right=of wx0}
\end{tikzdisplay}
all of whose preconditions are tautologies, so this has the expected behavior:
\begin{tikzdisplay}[node distance=1em]
  \event{wx0}{\DW{x}{0}}{}
  \event{wy1}{\DW{y}{1}}{right=of wx0}
\end{tikzdisplay}
There is no dependency between $(\DW x0)$ and $(\DW y1)$,
since $(0=0\land0+1=1)$ is independent of $\aLoc$.

This example demonstrates how preconditions
capture the sequential semantics of memory.
In an execution containing an event with label
$(\aForm \mid \aAct)$, one way the precondition $\aForm$
can be discharged is by an assignment $\aLoc\GETS\aExp$,
which performs a substitution $[\aExp/\aLoc]$.
This is a variant of the Hoare semantics of
assignment \cite{Hoare:1969:ABC:363235.363259}, where if $\aCmd$ has precondition $\aForm$
then $\aLoc\GETS\aExp\SEMI\aCmd$ has precondition
$\aForm[\aExp/\aLoc]$.

\subsection{Concurrent memory accesses}
\label{sec:concurrent-memory}

We now turn to the case of concurrent accesses to memory.
Consider the program %a concurrent version of the program from \S\ref{sec:sequential-memory}:
$(\aLoc\GETS1 \PAR \bLoc\GETS\aLoc+1)$.
In executions of this program, it is possible for the second thread to 
perform a concurrent read of $x$:
\begin{tikzdisplay}[node distance=1em]
  \event{wx1}{\DW{x}{1}}{}
  \event{rx1}{\DR{x}{1}}{right=2.5em of wx1}
  \event{wy2}{\DW{y}{2}}{right=of rx1}
  \rf{wx1}{rx1}
  \po{rx1}{wy2}
\end{tikzdisplay}
To see how this execution is modeled, we first
expand out the syntax sugar to get the program
$(\aLoc\GETS1\SEMI\SKIP \PAR \aReg\GETS\aLoc\SEMI \bLoc\GETS\aReg+1\SEMI\SKIP).$
As before, $\sem{y \GETS r+1\SEMI \SKIP}$ includes:
\[
   (r+1=2) \guard (\DW y2) \prefix \sem{\SKIP}[2/y]
\]
which contains the pomset:
\begin{tikzdisplay}[node distance=1em]
  \event{wy2}{r+1=2 \mid \DW{y}{2}}{}
\end{tikzdisplay}
As before, $\sem{r \GETS x\SEMI y \GETS r+1\SEMI \SKIP}$ has two cases.
We are now interested in the concurrent case, which includes:
\[
   (\DR x1) \prefix \sem{y \GETS r+1\SEMI \SKIP}[x/r]
\]
which contains the pomset:
\begin{tikzdisplay}[node distance=1em]
  \event{rx1}{\DR{x}{1}}{}
  \event{wy2}{\DW{y}{2}}{right=of rx1}
  \po{rx1}{wy2}
\end{tikzdisplay}
Note that $(\DR x1)$ reads $1$ from $x$, and while
$(x+1=2)[1/x]$ is a tautology,
$(x+1=2)$ is not,
and so there is a dependency
$(\DR x1) \lt (\DW y2)$.

Now, $\sem{x \GETS 1\SEMI \SKIP}$ includes the pomset:
\begin{tikzdisplay}[node distance=1em]
  \event{wx1}{\DW{x}{1}}{}
\end{tikzdisplay}
and so $\sem{x \GETS 1\SEMI \SKIP \PAR r \GETS x\SEMI y \GETS r+1\SEMI \SKIP}$ includes:
\begin{tikzdisplay}[node distance=1em]
  \event{wx1}{\DW{x}{1}}{}
  \event{rx1}{\DR{x}{1}}{right=2.5em of wx1}
  \event{wy2}{\DW{y}{2}}{right=of rx1}
  \rf{wx1}{rx1}
  \po{rx1}{wy2}
\end{tikzdisplay}
as expected, including a reads-from dependency
$(\DW x1) \lt (\DR x1)$.

This example demonstrates how read and write events
capture the concurrent semantics of memory.
In an execution containing an event with label
$(\DR \aLoc\aVal)$, if the execution is
$\aLoc$-closed, then there must be an event
it reads from, for example one labeled
$(\DW \aLoc\aVal)$.

\subsection{Control dependencies}
\label{sec:control-dep}

Conditionals introduce control dependencies, for example consider the program:
\[
  \aReg\GETS\cLoc\SEMI
  \IF{\aReg}\THEN \aLoc\GETS1 \ELSE \bLoc\GETS2 \FI
\]
This includes executions in which the false branch is taken:
\begin{tikzdisplay}[node distance=1em]
  \event{rz0}{\DR{z}{0}}{}
  \nonevent{wx1}{\DW{x}{1}}{right=of rz0}
  \event{wy2}{\DW{y}{2}}{right=of wx1}
  \po{rz0}{wx1}
  \po[out=30,in=150]{rz0}{wy2}
\end{tikzdisplay}
and ones where the true branch is taken:
\begin{tikzdisplay}[node distance=1em]
  \event{rz1}{\DR{z}{1}}{}
  \event{wx1}{\DW{x}{1}}{right=of rz1}
  \nonevent{wy2}{\DW{y}{2}}{right=of wx1}
  \po{rz1}{wx1}
  \po[out=30,in=150]{rz1}{wy2}
\end{tikzdisplay}
In both cases, we record the actions in the branch that was
not taken. This is a novel feature of this model, and is
intended to capture speculative evaluation. In \S\ref{sec:spectre}
we will show how this model captures Spectre-like information
flow attacks, once the attacker is provided with the ability to
observe such speculations.

To see how these executions are modeled, consider the semantics of
$\sem{x\GETS 1\SEMI\SKIP}$, which contains any pomset of the form:
\begin{tikzdisplay}[node distance=1em]
  \event{wx1}{\aForm \mid \DW{x}{1}}{}
\end{tikzdisplay}
in particular it contains:
\begin{tikzdisplay}[node distance=1em]
  \event{wx1}{r\neq0 \mid \DW{x}{1}}{}
\end{tikzdisplay}
Similarly $\sem{y\GETS 2\SEMI\SKIP}$ contains:
\begin{tikzdisplay}[node distance=1em]
  \event{wy2}{r=0 \mid \DW{y}{2}}{}
\end{tikzdisplay}
and so $\sem{\IF{r}\THEN x\GETS 1\SEMI\SKIP \ELSE y\GETS 2\SEMI\SKIP \FI}$
contains:
\begin{tikzdisplay}[node distance=1em]
  \event{wx1}{r\neq0 \mid \DW{x}{1}}{}
  \event{wy2}{r=0 \mid \DW{y}{2}}{right=of wx1}
\end{tikzdisplay}
Now, the semantics of concurrent read performs substitutions, for example:
\begin{tikzdisplay}[node distance=1em]
  \event{rz0}{\DR{z}{0}}{}
  \event{wx1}{0\neq0 \mid \DW{x}{1}}{right=of rz0}
  \event{wy2}{0=0 \mid \DW{y}{2}}{right=of wx1}
  \po{rz0}{wx1}
  \po[out=25,in=155]{rz0}{wy2}
\end{tikzdisplay}
which gives the required pomset:
\begin{tikzdisplay}[node distance=1em]
  \event{rz0}{\DR{z}{0}}{}
  \nonevent{wx1}{\DW{x}{1}}{right=of rz0}
  \event{wy2}{\DW{y}{2}}{right=of wx1}
  \po{rz0}{wx1}
  \po[out=30,in=150]{rz0}{wy2}
\end{tikzdisplay}
Note that the precondition $r=0$ is dependent on $r$,
and so there is a dependency $(\DR z0) \lt (\DW y2)$,
modeling the control dependency introduced by the conditional.

\subsection{Control independencies}

In most models of control dependencies, the dependency relation
is syntactic, based on whether the action occurs syntactically
inside a conditional. In contrast, the notion in this model is
semantic: if an action can occur on both sides of a conditional,
there is no control dependency. Consider a variant of the example
from \S\ref{sec:control-dep}:
\[
  \aReg\GETS\cLoc\SEMI
  \IF{\aReg}\THEN \aLoc\GETS1 \ELSE \aLoc\GETS1 \FI
\]
This has the expected execution in which the control
dependencies exist:
\begin{tikzdisplay}[node distance=1em]
  \event{rz0}{\DR{z}{0}}{}
  \nonevent{nwx1}{\DW{x}{1}}{right=of rz0}
  \event{wx1}{\DW{x}{1}}{right=of nwx1}
  \po{rz0}{nwx1}
  \po[out=30,in=150]{rz0}{wx1}
\end{tikzdisplay}
but it also has an execution in which the two writes
of $1$ to $x$ are merged, resulting in no dependency:
\begin{tikzdisplay}[node distance=1em]
  \event{rz0}{\DR{z}{0}}{}
  \event{wx1}{\DW{x}{1}}{right=of rz0}
\end{tikzdisplay}
To see how this arises,
consider the definition of $\sem{\IF{r}\THEN x\GETS1\SEMI\SKIP \ELSE x\GETS1\SEMI\SKIP \FI}$:
\[\begin{array}{rl}
   \aPSS_1 \parallel \aPSS_2 \quad\mbox{where}\quad&
   \aPSS_1 = (r\neq 0) \guard \sem{x\GETS1\SEMI\SKIP} \\&
   \aPSS_2 = (r=0) \guard \sem{x\GETS1\SEMI\SKIP}
\end{array}\]
Now, one pomset in $\aPSS_1$ is:
\begin{tikzdisplay}[node distance=1em]
  \event{wx1}{r\neq0 \mid \DW{x}{1}}{}
\end{tikzdisplay}
that is $\aPS_1$ where:
\[
  \Event_1 = \{\aEv\} \quad
  \labeling_1(\aEv) = (r\neq 0, \DW x1)
\]
and similarly, one pomset in $\aPSS_2$ is:
\begin{tikzdisplay}[node distance=1em]
  \event{wx1}{r=0 \mid \DW{x}{1}}{}
\end{tikzdisplay}
that is $\aPS_2$ where:
\[
  \Event_2 = \{\aEv\} \quad
  \labeling_2(\aEv) = (r= 0, \DW x1)
\]
Crucially, in the definition of $\aPSS_1 \parallel \aPSS_2$
there is \emph{no} requirement that $\Event_1$ and $\Event_2$ are disjoint,
and in this case they overlap at $\aEv$. As a result, one pomset in
$\aPSS_1\parallel\aPSS_2$ is $\aPS_0$ where:
\[
  \Event_0 = \{\aEv\} \quad
  \labeling_0(\aEv) = (r\neq0 \lor r=0, \DW x1)
\]
that is:
\begin{tikzdisplay}[node distance=1em]
  \event{wx1}{\DW{x}{1}}{}
\end{tikzdisplay}
Note that this pomset has no precondition dependent on $r$,
since $(r\neq0 \lor r=0)$ does not depend on $r$, which is why
we end up with an execution without a control dependency:
\begin{tikzdisplay}[node distance=1em]
  \event{rz0}{\DR{z}{0}}{}
  \event{wx1}{\DW{x}{1}}{right=of rz0}
\end{tikzdisplay}
This semantics captures compiler optimizations which may, for example,
merge code executed on both branches of a conditional, or hoist
constant assignments out of loops.

We can now see the counterintuitive behavior of conditionals
in the presence of control dependencies.
There are programs such as
\(
  (\IF{\cLoc}\THEN \aLoc\GETS1 \ELSE \aLoc\GETS1 \FI)
\)
with executions in which  $(\DW x1)$ is independent of $(\DR z1)$:
\begin{tikzdisplay}[node distance=1em]
  \event{rz1}{\DR{z}{1}}{}
  \event{wx1}{\DW{x}{1}}{right=of rz1}
\end{tikzdisplay}
while programs such as
\(
  (\IF{\cLoc}\THEN \aLoc\GETS1 \ELSE \bLoc\GETS2 \FI)
\)
only have executions in which $(\DW x1)$ is dependent on $(\DR z1)$:
\begin{tikzdisplay}[node distance=1em]
  \event{rz1}{\DR{z}{1}}{}
  \event{wx1}{\DW{x}{1}}{right=of rz1}
  \nonevent{wy2}{\DW{y}{2}}{right=of wx1}
  \po{rz1}{wx1}
  \po[out=30,in=150]{rz1}{wy2}
\end{tikzdisplay}
These programs have executions with different dependency relations, depending only
on conditional branches that were \emph{not} taken. In \S\ref{sec:info-flow-attack}
we shall see that this has security implications, since relaxed
memory can observe dependency.
% The attack is similar to Spectre, so
% we shall take a detour to see how Spectre can be modeled in this
% setting.

\subsection{UNUSED: Relaxed memory}
\label{sec:relaxed-memory}

In \S\ref{sec:info-flow-attack} we present an information flow attack
on relaxed memory, similar to Spectre in that it relies on speculative
evaluation. Unlike Spectre it does not depend on timing attacks,
but instead is based on the sensitivity of relaxed memory to data
dependencies. % For this reason, we present a simple model of relaxed
% memory, which is strong enough to capture this attack.

Our model includes concurrent memory accesses, which can introduce concurrent
reads-from. 
Since we are allowing events to be partially ordered, this gives a simple
model of relaxed memory.  For example an independent read independent write
(IRIW) example is:
\[\begin{array}{l}
  x\GETS0\SEMI x\GETS x+1
  \PAR
  y\GETS0\SEMI y\GETS y+1
\\{}
  \PAR
  r_1\GETS x\SEMI r_2\GETS y
  \PAR
  s_1\GETS y\SEMI s_2\GETS x
\end{array}\]
which includes the execution:
\begin{tikzdisplay}[node distance=1em]
  \event{wx0}{\DW{x}{0}}{}
  \event{wx1}{\DW{x}{1}}{right=of wx0}
  \event{wy0}{\DW{y}{0}}{right=2.5em of wx1}
  \event{wy1}{\DW{y}{1}}{right=of wy0}
  \event{ry1}{\DR{y}{1}}{below=4ex of wx0}
  \event{rx0}{\DR{x}{0}}{right=of ry1}
  \event{rx1}{\DR{x}{1}}{right=2.5 em of rx0}
  \event{ry0}{\DR{y}{0}}{right=of rx1}
  \rf{wx1}{rx1}
  \rf{wy0}{ry0}
  \rf[out=210,in=30]{wy1}{ry1}
  \rf{wx0}{rx0}
  \wk{rx0}{wx1}
  \wk{ry0}{wy1}
\end{tikzdisplay}
This model does not introduce thin-air reads (TAR).
For example the TAR pit
\((
  x\GETS y \PAR y \GETS x
)\)
fails to produce a value for $x$ from thin air
since this produces a cycle in $\le$, as shown on the left below:
\begin{align*}
\begin{tikzpicture}[node distance=1em]
  \event{ry42}{\DR{y}{42}}{}
  \event{wx42}{\DW{x}{42}}{below=of ry42}
  \event{rx42}{\DR{x}{42}}{right=2.5em of ry42}
  \event{wy42}{\DW{y}{42}}{below=of rx42}
  \po{ry42}{wx42}
  \po{rx42}{wy42}
  \rf{wx42}{rx42}
  \rf{wy42}{ry42}
\end{tikzpicture}
&&
\begin{tikzpicture}[node distance=1em]
  \event{ry1}{\DR{y}{1}}{}
  \event{wx1}{\DW{x}{1}}{below=of ry1}
  \event{rx1}{\DR{x}{1}}{right=2.5em of ry1}
  \event{wy1}{\DW{y}{1}}{below=of rx1}
  \po{ry1}{wx1}
  \rf{wx1}{rx1}
  \rf{wy1}{ry1}
\end{tikzpicture}
\end{align*}
This cycle can be broken by removing a dependency. For example
\((
  x\GETS y \PAR r\GETS x\SEMI y \GETS r+1-r
)\)
has the execution on the right above.
% \begin{tikzdisplay}[node distance=1em]
%   \event{ry1}{\DR{y}{1}}{}
%   \event{wx1}{\DW{x}{1}}{below=of ry1}
%   \event{rx1}{\DR{x}{1}}{right=2.5em of ry1}
%   \event{wy1}{\DW{y}{1}}{below=of rx1}
%   \po{ry1}{wx1}
%   \rf{wx1}{rx1}
%   \rf{wy1}{ry1}
% \end{tikzdisplay}
Note that $(\DR x1) \not\le (\DW y1)$, so this does not introduce a cycle.

Although it is not the primary focus of this paper, our model may be an
attractive model of relaxed memory.  Many prior models either permit
thin-air executions that our model forbids or forbid desirable executions
that our model permits.
%% In \S\ref{sec:logic}, we develop a logic which allows us to prove that our
%% semantics forbids thin air examples that are permitted by prior speculative
%% models
%% \cite{Manson:2005:JMM:1047659.1040336,Jagadeesan:2010:GOS:2175486.2175503,DBLP:conf/popl/KangHLVD17}.
% Our model passes all of the causality test cases
% \cite{PughWebsite}.
%% Significantly, this
%% includes test case 9, which is forbidden by \cite{DBLP:conf/lics/JeffreyR16},
%% one of the few models that disallows the thin air example from
%% \S\ref{sec:logic}.  We present this test case in the appendix, where we also
%% discuss the thread inlining examples from
%% \cite{Manson:2005:JMM:1047659.1040336}.

% In \refapp{logic}, we present a variant of the TAR-pit
% example %from \S\ref{sec:relaxed-memory}
% that is allowed under prior speculative semantics
% \cite{Manson:2005:JMM:1047659.1040336,Jagadeesan:2010:GOS:2175486.2175503,DBLP:conf/popl/KangHLVD17}.
% We develop a logic that allows us to prove that the problematic execution is
% forbidden in our model.  \citet{DBLP:conf/esop/BattyMNPS15} showed that the
% thin-air problem has no per-candidate-execution solution for C++.  This
% result does not apply to our model, which has a different notion of
% dependency.

% as the semantics of a conditional can depend on the semantics
% of both branches.

\citet{PughWebsite} developed a set of twenty {causality test cases} in the
process of revising the Java Memory Model (JMM)
\cite{Manson:2005:JMM:1047659.1040336}.  Using hand calculation, we have
confirmed that our model gives the desired result for all twenty cases,
unrolling loops as necessary.  Our model also gives the desired results for
all of the examples in \citet[\textsection 4]{DBLP:conf/esop/BattyMNPS15} and
all but one in \citet[\textsection 5.3]{SevcikThesis}:
redundant-write-after-read-elimination fails for any
sensible non-coherent semantics.  Our model agrees with the JMM on the
``surprising and controversial behaviors'' of \citet[\textsection
8]{Manson:2005:JMM:1047659.1040336}, and thus fails to validate thread
inlining.
In \refapp{tc}, we discuss three of the causality test cases and the thread
inlining example from \cite{Manson:2005:JMM:1047659.1040336}.%  In presenting the
% examples, we unroll loops, correct typos and simplify the code.  


\section{Attacks on speculative execution}
\label{sec:examples}

In this section, we show how known attacks on speculative execution can be
modeled.  In \S\ref{sec:spectre}, we discuss Spectre. In
\S\ref{sec:spec-barriers}, we describe \emph{speculation barriers} for
defense against Spectre.  In \S\ref{sec:transactions}, we discuss attacks on
transactions.

In each attack, there is a high-security variable $\SEC$,
and the goal of the attacker is to learn one bit of information
from $\SEC$. The Spectre and \textsc{Prime+Abort}
attacks exploit optimizations in hardware, and so can be mounted
against a dynamic $\SEC$.

\subsection{Spectre}
\label{sec:spectre}

We give a simplified model of Spectre attacks, ignoring the details of
cache timing.  In this model, we extend programs with the ability to tell
whether a memory location has been touched (in practice this is
implemented using timing attacks on the cache). For example,
we can model Spectre by:
\[\begin{array}{l}
  \VAR a\SEMI \IF{\CANREAD(\SEC})\THEN a[\SEC]\GETS1
  \brELIF{\TOUCHED a[0]}\THEN x\GETS0
  \brELIF{\TOUCHED a[1]}\THEN x\GETS1 \FI
\end{array}\]
This is a low-security program, which is attempting to discover the
value of a high-security variable $\SEC$. The low-security program
is allowed to attempt to escalate its privileges by checking that it is
allowed to read a high-security variable:
\[\begin{array}{l}
  \IF{\CANREAD(\SEC})\THEN \mbox{code allowed to read $\SEC$}
  \brELSE \mbox{code not allowed to read $\SEC$} \FI
\end{array}\]
In this case, $\CANREAD(\SEC)$ is false, so the fallback code
is executed. Unfortunately, the escalated code is speculatively
evaluated, which allows information to leak by testing for which
memory locations have been touched.

Attacks may realize the abstract notions in various ways.  For example, in
variant 1 of Spectre, the dynamic security check is implemented as an array
bounds check.

We model the $\TOUCHED$ test by introducing a new action
$(\DT{\aLoc})$, and defining:
\[\begin{array}{l}
  \sem{\IF{\TOUCHED\aLoc} \THEN \aCmd \ELSE \bCmd \FI} \\[\jot]\quad =  ((\DT\aLoc) \prefix \sem{\aCmd}) \cup \sem{\bCmd}
\end{array}\]
Implementations of $\TOUCHED$ use cache timing, but their success can be modeled
without needing to be precise about such microarchitectural details:
\begin{itemize}
\item if $\labeling(\aEv)=(\aForm \mid \DT{\aLoc})$
  then there is $\bEv\gtN\aEv$
  where $\bEv$ reads or writes $\aLoc$.
\end{itemize}
Note that there is no requirement that $\bEv$ be satisfiable,
and indeed Spectre has the execution:
\begin{tikzdisplay}[node distance=1em]
  \nonevent{rs}{\DR{\SEC}{1}}{}
  \nonevent{wa}{\DW{a[1]}{1}}{right=of rs}
  \event{ta}{\DT{a[1]}}{right=of wa}
  \event{wx}{\DW{x}{1}}{right=of ta}
  \po{rs}{wa}
  \wk{wa}{ta}
  \po{ta}{wx}
\end{tikzdisplay}
but (assuming a successful implementation of $\TOUCHED$) \emph{not}:
\begin{tikzdisplay}[node distance=1em]
  \nonevent{rs}{\DR{\SEC}{0}}{}
  \nonevent{wa}{\DW{a[0]}{1}}{right=of rs}
  \event{ta}{\DT{a[1]}}{right=of wa}
  \event{wx}{\DW{x}{1}}{right=of ta}
  \po{rs}{wa}
  \wk{wa}{ta}
  \po{ta}{wx}
\end{tikzdisplay}
Thus, the attacker has managed to leak the value of a high-security
location to a low-security one: if $(\DW x1)$ is observed, the \verb|SECRET|
must have been 1.

This shows how our model of speculation can express
the way in which Spectre-like attacks bypass dynamic security checks,
without giving a treatment of microarchitecture.

\subsection{Speculation barriers}
\label{sec:spec-barriers}

The ability to model Spectre is useful, but really we would
like to model defenses against such attacks, and provide some
confidence in the correctness of the defense. One such defense
which fits naturally in our model is \emph{speculation barriers},
which prevent code from being speculatively executed. For example,
we could introduce such a $\BARRIER$, and require that
a barrier is introduced on each security check:
\[\begin{array}{l}
\IF{\CANREAD(\SEC})\THEN \BARRIER\SEMI\cdots
  \ELSE \cdots \FI
\end{array}\]
To model barriers, we introduce a new action $\DSB$
and define:
\[\begin{array}{l}
  \sem{\BARRIER\SEMI\aCmd} =  \{\emptyset\} \cup ((\DSB) \prefix \sem{\aCmd})
\end{array}\]
Implementations of $\BARRIER$ make use of hardware barriers which
halt speculative execution until all instructions up to the barrier
have been retired. Such barriers are successful when:
\begin{itemize}
\item if $\labeling(\aEv)=(\aForm \mid \DSB)$
  then $\aForm$ is satisfiable.
\end{itemize}
For example, a successful implementation of barriers disallows
the execution of Spectre:
\begin{tikzdisplay}[node distance=1em]
  \nonevent{sb}{\DSB}{}
  \nonevent{rs}{\DR{\SEC}{1}}{right=of sb}
  \nonevent{wa}{\DW{a[1]}{1}}{right=of rs}
  \event{ta}{\DT{a[1]}}{right=of wa}
  \event{wx}{\DW{x}{1}}{right=of ta}
  \po{rs}{wa}
  \wk{wa}{ta}
  \po{ta}{wx}
\end{tikzdisplay}
One might expect that this is a successful (albeit expensive) defense
against Spectre, but it is not, unless the compiler is aware that
$\BARRIER$ cannot be lifted out of a conditional. An unaware compiler
might perform common subexpression elimination on barriers, allowing
the attacker to introduce a barrier to fool a compiler into optimizing
the safe:
\[\begin{array}{l}
  \IF{\CANREAD(\SEC})\THEN \BARRIER\SEMI \cdots
  \ELSE \BARRIER\SEMI \cdots \FI
\end{array}\]
into the unsafe:
\[\begin{array}{l}
  \BARRIER\SEMI \IF{\CANREAD(\SEC})\THEN \cdots
  \ELSE \cdots \FI
\end{array}\]
To model the requirement that barriers are not moved past
conditionals, we make them \emph{unmergeable}: $\DSB$ events
on different arms of a conditional cannot be merged.
With this requirement, we can show that barriers act as a
defense against Spectre by first showing that $\sem\bCmd$ has
the same successful executions as:
\[
  \sem{\IF{\CANREAD(\SEC}) \THEN \BARRIER\SEMI\aCmd \ELSE \bCmd \FI}
\]
and then showing that the semantics which only looks at successful
executions is \emph{compositional}: if $\sem\bCmd$ has the same
successful executions as $\sem{\bCmd'}$ then $\sem{\aCmd[\bCmd]}$ has
the same successful executions as $\sem{\aCmd[\bCmd']}$ for any
``program with a hole'' $\aCmd[{\bullet}]$. Compositional reasoning
is what fails when $\DSB$ is mergeable, as shown by the
attack against a compiler which blindly performs common subexpression
elimination.

To realize a speculation barrier in microarchitecture, it is likely
sufficient for the barrier to stop any further speculation until the barrier
is known to succeed.  There is experimental evidence that Intel's
\texttt{mfence} instruction has the effect of a speculation barrier in some contexts
\cite[\S{VII-\textit{D}}]{DBLP:conf/micro/TrippelLM18}.

\subsection{Transactions}
\label{sec:transactions}

We present a model of transactional memory~\cite{Larus:2007:TM:1207012} that is sufficient to capture
\textsc{Prime+Abort} attacks~\cite{DBLP:conf/uss/DisselkoenKPT17}.  We make
several simplifying assumptions: transactions are serializable, strongly
isolated, and only abort due to cache conflicts.

The action $(\DB{\aVal})$ %\in\Acq$
represents the begin of a transaction with
id $\aVal$, and $(\DC{\aVal})$ %\in \Rel$
represents the corresponding commit.
We model a language in which transactions have explicit identifiers (which we
elide in examples) and abort handlers (which we elide when they are empty):
\[\begin{array}{l}
    \sem{\BEGINVAL\SEMI \eCmd\SEMI \RECOVERYVAL \bCmd \ENDREC} \\\quad
    = (\DB{\aVal}) \prefix \bigl(\sem{\eCmd} \cup \bigl((\FALSE \guard \sem{\eCmd}) \parallel \sem{\bCmd}\bigr)\bigr)
    \\[\jot]
    \sem{\COMMITVAL\SEMI \bCmd} \\\quad
    = (\DC{\aVal}) \prefix \sem{\bCmd}
  \end{array}\]
The semantics of a transaction has two cases: a committed case
(executing only the transaction body) and an aborted case (executing both the body and the
recovery code, where the body is marked unsatisfiable). For example, two executions of
\begin{math}
  (\BEGIN\SEMI \aLoc\GETS1\SEMI \aLoc\GETS2\SEMI \COMMIT\SEMI \RECOVERY \bLoc\GETS1\ENDREC)
\end{math}
are:
\[\begin{array}{c}\begin{tikzpicture}[node distance=1em]
  \event{b0}{\DB{}}{}
  \event{wx0}{\DW{x}{1}}{right=of b0}
  \event{wx1}{\DW{x}{2}}{right=of wx0}
  \event{c0}{\DC{}}{right=of wx1}
  \po{b0}{wx0}
  \po[out=30,in=150]{b0}{wx1}
  \po[out=30,in=150]{wx0}{c0}
  \po{wx1}{c0}
  \wk{wx0}{wx1}
\end{tikzpicture}
\\
\begin{tikzpicture}[node distance=1em]
  \event{b0}{\DB{}}{}
  \nonevent{wx0}{\DW{x}{1}}{right=of b0}
  \nonevent{wx1}{\DW{x}{2}}{right=of wx0}
  \nonevent{c0}{\DC{}}{right=of wx1}
  \event{wy1}{\DW{y}{1}}{right=of c0}
  \po{b0}{wx0}
  \po[out=30,in=150]{b0}{wx1}
  \po[out=30,in=160]{b0}{wy1}
  \po[out=30,in=150]{wx0}{c0}
  \po{wx1}{c0}
  \wk{wx0}{wx1}
\end{tikzpicture}\end{array}\]
At top level, we require that pomsets be \emph{serializable}, as defined below.
\begin{definition}
  We say that event $\comEv$ \emph{matches} $\begEv$ if
  $\labeling(\comEv)=(\DC{\aVal})$ and
  $\labeling(\begEv)=(\DB{\aVal})$. %, for some $\aVal$.
  % We say that a begin event \emph{aborts} if every matching commit is
  % unsatisfiable.
  We say that begin event $\begEv$ \emph{begins} $\aEv$ if
  $\begEv\le\aEv$ and there is no intervening matching commit; in this case
  $\aEv$ \emph{belongs to} $\begEv$.
  % event $\comEv$ such that $\begEv\le\comEv\le\aEv$
  We say that commit event $\comEv$ \emph{commits} $\aEv$ if $\aEv\le\comEv$
  and there is no intervening matching begin.
  % event $\begEv$ such that $\aEv\le\begEv\le\comEv$.
\end{definition}
\begin{definition}
  A pomset is \emph{serializable} if:
  \begin{enumerate}
  \item\label{tx:1} no two begins have the same id,
  \item\label{tx:2} every commit follows the matching begin,
  \item\label{tx:3} $\le$ totally orders tautological begins and commits,
  \item\label{tx:4} if $\begEv$ begins $\aEv$, but not $\bEv$, and $\bEv\le\aEv$ then $\bEv\le\begEv$,
  \item\label{tx:5} if $\comEv$ ends $\aEv$, but not $\bEv$, and $\aEv\le\bEv$ then $\comEv\le\bEv$,
  % \item\label{tx:4} if $\begEv$ begins $\aEv$, but not $\bEv$, then
  %   $\bEv\le\aEv$ implies $\bEv\le\begEv$ and $\bEv\gtN\aEv$ implies $\bEv\gtN\begEv$
  % \item\label{tx:5} if $\comEv$ ends $\aEv$, but not $\bEv$, then
  %   $\aEv\le\bEv$ implies $\comEv\le\bEv$ and $\aEv\gtN\bEv$ implies $\comEv\gtN\bEv$,
  \item\label{tx:6} if $\aEv$ and $\bEv$ belong to $\begEv$ and read the same
    location, then both read the same value, and
    % note that read events are optional, so we can assume they come from
    % outside the transaction.
  %
  % \item\label{tx:6} if $\begEv$ begins $\aEv$ then some matching $\comEv$ both implies and ends $\aEv$,
  % \item\label{tx:6} if $\begEv$ begins $\aEv$ then some matching $\comEv$
  %   ends $\aEv$ such that both $\aEv$ implies $\comEv$ and $\comEv$ implies $\aEv$,
  \item\label{tx:7} if $\aEv$ belongs to $\begEv$, then $\aEv$ implies some
    matching $\comEv$ that ends $\aEv$.
  \end{enumerate}
\end{definition}
%In discussion, we identify transactions by their unique begin event.
%A transaction that does not abort is \emph{successful}.
%
Conditions \ref{tx:1}-\ref{tx:5} ensure serializability of committed
transactions.  Conditions \ref{tx:4}-\ref{tx:6} also ensure strong isolation
for non-transactional events
\cite{DBLP:journals/pacmpl/DongolJR18}. Condition \ref{tx:7} ensures that all
events in aborted transactions are unsatisfiable.
%
For example Conditions \ref{tx:5} and \ref{tx:7} rule out
executions (which violate strong isolation and atomicity):
\[\begin{array}{c}\begin{tikzpicture}[node distance=1em]
  \event{b0}{\DB{}}{}
  \event{wx0}{\DW{x}{0}}{right=of b0}
  \event{wx1}{\DW{x}{1}}{right=of wx0}
  \event{c0}{\DC{}}{right=of wx1}
  \event{rx0}{\DR{x}{0}}{right=of c0}
  \po{b0}{wx0}
  \po[out=30,in=150]{b0}{wx1}
  \po[out=30,in=150]{wx0}{c0}
  \po{wx1}{c0}
  \wk{wx0}{wx1}
  \rf[out=30,in=150]{wx0}{rx0}
\end{tikzpicture}
\\
\begin{tikzpicture}[node distance=1em]
  \event{b0}{\DB{}}{}
  \event{wx0}{\DW{x}{0}}{right=of b0}
  \nonevent{wx1}{\DW{x}{1}}{right=of wx0}
  \nonevent{c0}{\DC{}}{right=of wx1}
  \event{wy1}{\DW{y}{1}}{right=of c0}
  \po{b0}{wx0}
  \po[out=30,in=150]{b0}{wx1}
  \po[out=30,in=160]{b0}{wy1}
  \po[out=30,in=150]{wx0}{c0}
  \po{wx1}{c0}
  \wk{wx0}{wx1}
\end{tikzpicture}\end{array}\]

In order to model \textsc{Prime+Abort}, we need a mechanism for modeling
\emph{why} a transaction aborts, as this can be used as a back channel.
We model a simple form of concurrent transaction, which aborts when it
encounters a memory conflict---this is similar to
the treatment of $\TOUCHED$ in \S\ref{sec:spectre}.

\begin{definition}
  \label{def:abort}
  A commit event $\comEv$ matching $\begEv$ \emph{aborts due to memory conflict}
  if there is some $\aEv$ ended by $\comEv$, and some tautologous $\begEv\gtN\bEv\gtN\comEv$ that does not
  belong to $\begEv$ such that $\aEv$ and $\bEv$ touch the same location.
\end{definition}

The attack requires an honest agent whose %cache-set
access pattern depends upon a secret.
% If $a[0]$ and $a[1]$ belong to separate cache sets, then
Such an honest agent is:
\[
  a[\SEC]\,\GETS\,1
\]
% \ignore{
% \begin{verbatim}
%   a[SECRET] := 1
% \end{verbatim}
% }
%The attack relies on discovery of some $y$ which belongs to the cache-set of $a[1]$.
Then the attacker program
\[
\BEGIN\SEMI a[1]\GETS0\SEMI r\GETS\COMMIT\SEMI \RECOVERY x\GETS1\ENDREC
\]
% \ignore{
% \begin{verbatim}
%   begin; y:=0; commit; onabort; x:=1;
% \end{verbatim}
% }
can write $1$ to $x$ if the \texttt{SECRET} is $1$, in which case the
following execution is possible.
\begin{tikzdisplay}[node distance=1em,baselinecenter]
  \event{wa1}{\DW{a[1]}{1}}{}
  \event{b}{\DB{}}{right=2.5em of wa1}
  \nonevent{e}{\DW{a[1]}{0}}{right=of b}
  \nonevent{c}{\DC{}}{right=of e}
  \event{wx1}{\DW{x}{1}}{right=of c}
  \po{b}{e}
  \po{e}{c}
  \po[out=30,in=155]{b}{wx1}
  \wk{b}{wa1}
  \wk[out=25,in=155]{wa1}{c}
\end{tikzdisplay}
If the attacker knows that commits only abort due to memory conflicts,
then this attack is an information flow, since the memory conflict only happens
when the \texttt{SECRET} is $1$.

The attacker code here must have write access to the high security variable
$a$.  Such a ``write up'' is allowed by secrecy analyses such as the
Smith-Volpano type system \cite{Smith:1998:SIF:268946.268975}, which
is meant to guarantee noninterference.

If we require that the attacker and honest agent access disjoint locations in
memory, then we must include a bit of microarchitecture to model the attack.
Suppose that the set of locations $\Loc$ is partitioned into \emph{cache
  sets} and update Definition~\ref{def:abort} so that the commit event
{aborts due to memory conflict} if $\aEv$ and $\bEv$ touch locations \emph{in
  the same cache set}.

\textsc{Prime+Abort} exploits an honest agent whose cache-set
access pattern depends upon a secret.
If $a[0]$ and $a[1]$ belong to separate cache sets, then
such an honest agent is, as before:
\[
  a[\SEC]\,\GETS\,1
\]
The attack relies on discovery of some $y$ which belongs to the cache-set of
$a[1]$.
Then the attack can be written as:
\[
  \BEGIN\SEMI y\GETS0\SEMI r\GETS\COMMIT\SEMI \RECOVERY x\GETS1\ENDREC
\]
As before, if the attacker knows that commits only abort due to memory
conflicts, then there is an information flow, since the memory conflict
only happens when the \texttt{SECRET} is $1$.

This style of attack can be thwarted by requiring that the honest agent and
attack code access disjoint cache sets.  This approach is pursued in \cite{dawg}.

Another defense is to require a speculation barrier at the beginning of each
transaction.  This would have the effect, however, of undermining any
optimistic execution strategy for transactions: the transaction would only be
able to begin when it is known that its commit will succeed.

% \ignore{
% \begin{verbatim}
%   begin; y:=0; commit; onabort; x:=1;
% \end{verbatim}
% }

%   The definition handles simple examples:
% \begin{itemize}
% \item Single threaded example: $\DB_1 \DC_1 \DB_2 \DC_2$.  Because
%   $\DC_2$ is a release, we know that $\DC_1\lt\DC_2$.  Because
%   $\DB_1$ is an acquire, we know that $\DB_1\lt\DB_2$ By lifting, either of
%   these is sufficient to require that $\DC_1\lt\DB_2$.
% \begin{tikzdisplay}[node distance=1em,baselinecenter]
%   \event{b1}{\DB_1}{}
%   \event{c1}{\DC_1}{right=of b1}
%   \event{b2}{\DB_2}{below right=of b1}
%   \event{c2}{\DC_2}{right=of b2}
%   \po{b1}{c1}
%   \po{b1}{b2}
%   \po{b2}{c2}
%   \po{c1}{c2}
% \end{tikzpicture}
% \;\text{implies}\;
% \begin{tikzpicture}[node distance=1em,baselinecenter]
%   \event{b1}{\DB_1}{}
%   \event{c1}{\DC_1}{right=of b1}
%   \event{b2}{\DB_2}{below right=of b1}
%   \event{c2}{\DC_2}{right=of b2}
%   \po{b1}{c1}
%   \po{c1}{b2}
%   \po{b2}{c2}
% \end{tikzdisplay}
% \item Abort example:
% \begin{tikzdisplay}[node distance=1em,baselinecenter]
%   \event{b1}{\DB}{}
%   \event{wx1}{\DW{x}{1}}{right=of b1}
%   \event{c1}{\DC_1}{above right=of wx1}
%   \nonevent{c2}{\DC_2}{below right=of wx1}
%   \event{rx1}{\DR{x}{1}}{right=2.5 em of wx1}
%   \po{b1}{wx1}
%   \po{wx1}{c1}
%   \po{wx1}{c2}
%   \rf{wx1}{rx1}
% \end{tikzpicture}
% \;\text{implies}\;
% \begin{tikzpicture}[node distance=1em,baselinecenter]
%   \event{b1}{\DB}{}
%   \event{wx1}{\DW{x}{1}}{right=of b1}
%   \event{c1}{\DC_1}{above right=of wx1}
%   \nonevent{c2}{\DC_2}{below right=of wx1}
%   \event{rx1}{\DR{x}{1}}{right=2.5 em of wx1}
%   \po{b1}{wx1}
%   \po{wx1}{c1}
%   \po{wx1}{c2}
%   \rf{wx1}{rx1}
%   \po{c1}{rx1}
% \end{tikzdisplay}

% % \item Clause \eqref{xrf} stops transaction from reading two different values
% %   for the same variable from transactions (it is possible with no
% %   transactional writes).
% % \item Clause \eqref{xrf} also stops transactional IRIW.
% \end{itemize}
% % \begin{definition}
% %   An rf-pomset is transaction-closed if the $\DB$ and $\DC$ actions with
% %   satisfiable preconditions are totally ordered by $\lt$.
% % \end{definition}

% Let ``$\END\SEMI \bCmd$'' be syntax sugar for
% ``$\IF{\COMMIT\vec\aLoc}\THEN\bCmd \ELSE \bCmd$'', where $\vec\aLoc$ are the
% free variables of $\bCmd$.

% The semantics of
% \begin{alltt}
%   x:=1; begin; x:=2; end; y:=x;
% \end{alltt}
% includes
% \begin{tikzdisplay}[node distance=1em]
%   \event{wx1}{\DW{x}{1}}{}
%   \event{b}{\DB}{right=of wx1}
%   \event{wx2}{\DW{x}{2}}{right=of b}
%   \event{c}{\DC}{right=of wx2}
%   \event{wy2}{\DW{y}{2}}{right=of c}
%   \nonevent{wy1}{\DW{y}{1}}{below=of wy2}
%   \po{b}{wx2}
%   \po[bend right]{b}{wy1}
%   \po[bend left]{b}{wy2}
%   \po{wx2}{c}
%   \po[bend left]{wx1}{c}
%   %\po{rz0}{wy2}
% \end{tikzdisplay}
% and
% \begin{tikzdisplay}[node distance=1em]
%   \event{wx1}{\DW{x}{1}}{}
%   \event{b}{\DB}{right=of wx1}
%   \nonevent{wx2}{\DW{x}{2}}{right=of b}
%   \nonevent{c}{\DC}{right=of wx2}
%   \nonevent{wy2}{\DW{y}{2}}{right=of c}
%   \event{wy1}{\DW{y}{1}}{below=of wy2}
%   \po{b}{wx2}
%   \po[bend right]{b}{wy1}
%   \po[bend left]{b}{wy2}
%   \po{wx2}{c}
%   \po[bend left]{wx1}{c}
%   %\po{rz0}{wy2}
% \end{tikzdisplay}

% Publication example:
% \begin{alltt}
%   var x; var f; x:=0; f:=0;
%      x:=1; (begin; f:=1; end;) || (begin; r:=f; end; s:=x;)
% \end{alltt}

% Note: we could also include a transaction factory, and close the factory.
% \begin{alltt}
%   TransactionFactory T; var x; var f; x:=0; f:=0; fence;
%      x:=1; (begin T; f:=1; f:=2; end T;) || (begin T; r:=f; end T; s:=x;)
% \end{alltt}

% Before defining atomicity, we provide some auxiliary notation.
%
% We say that $\aEv$ is a \emph{begin event} if
% $\labeling(\aEv)=(\aForm\mid\DB)$ and a \emph{commit event} if
% $\labeling(\aEv)=(\aForm\mid\DC)$.
%
% We write $\aForm_\aEv$ for the formula and $\aAct_\aEv$ for the
% action of $\aEv$; that is, when $\labeling(\aEv)=(\aForm_\aEv\mid\aAct_\aEv)$.
%
% We say that $\aEv$ is \emph{compatible with} $\bEv$ when
% $\aForm_\aEv\land\aForm_\bEv$ is satisfiable.
%
% \begin{definition}
%   A pomset is \emph{atomic} when for any $\aEv$ that belongs to $(\begEv,\vec\comEv)$:
%   \begin{enumerate}
%   \item\label{xcommitform} $\aForm_{\aEv}$ implies $\textstyle\bigvee_i\aForm_{\comEv_i}$,
%   \item\label{xliftb} if $\bEv\lt\aEv$ then $\bEv\lt\begEv$,
%   \item\label{xliftc} if $\aEv\lt\bEv$ and $\bEv$ is compatible with
%     $\comEv_i$ then $\comEv_i\lt\bEv$,
%   \item\label{xtotal} if $\aEv'\neq\aEv$ belongs to $(\begEv',\vec\comEv')$ but not
%     $(\begEv,\dontcare)$ and
%     \begin{itemize}
%     \item $\comEv'_j$ is compatible with $\aEv$, and
%     \item $\comEv_i$ is compatible with $\aEv'$
%     \end{itemize}
%     then either
%     $\comEv'_j\lt\begEv$ or
%     $\comEv_i\lt\begEv'$,
%   % \item\label{xrf} if $\aEv'\neq\aEv$ belongs to $(\begEv,\dontcare)$ but not
%   %   $(\begEv,\dontcare)$ and
%   %   \begin{itemize}
%   %   \item $\aEv$ reads from $\bEv'$ that belongs to $(\begEv',\vec\comEv')$,
%   %   \item $\aEv'$ reads from $\bEv''\neq\beV'$ that belongs to $(\begEv'',\vec\comEv'')$,
%   %   \item $\comEv''_j$ is compatible with $\bEv'$, and
%   %   \item $\comEv'_i$ is compatible with $\bEv''$
%   %   \end{itemize}
%   %   then either
%   %   $\comEv''_j\lt\begEv'$ or
%   %   $\comEv'_i\lt\begEv''$ .
%   \item\label{xcommitvars} if $\aEv$ writes $\aLoc$ and $\comEv_i$ writes
%     $\vec\aLoc$ then $\aLoc=\aLoc_i$, for some $i$, and
%   \item\label{xreadunique} if $\aEv$ reads $\aLoc$ and $\aEv'\neq\aEv$ reads
%     $\aLoc$, belongs to $(\begEv,\dontcare)$ and is compatible with $\aEv$
%     then $\aAct_{\aEv}=\aAct_{\aEv'}$.
%   \end{enumerate}
% \end{definition}
% Clause \eqref{xcommitform} requires that the precondition on $\aEv$ is false on an
% aborted transaction.
% The \emph{lifting clauses}, \eqref{xliftb} and \eqref{xliftc}, require order
% come in or out of $\aEv$ is lifted to the corresponding begin or commit event.
% % Clause \eqref{xrf} requires that whenever a transaction reads from two other
% % transactions, the other transactions must be ordered.
% Clause \eqref{xtotal} requires that transactions be totally ordered.
% Clause \eqref{xcommitvars} requires that all writes be committed.
% Clause \eqref{xreadunique} requires that multiple reads of a location in a
% single transaction must see the same value.
%
% The definition of atomicity guarantees strong isolation.  For weak isolation,
% clauses \eqref{xcommitvars} and \eqref{xreadunique} are unnecessary,
% \eqref{xliftb} only applies when $\bEv$ is a commit, and \eqref{xliftc} only
% applies when $\bEv$ is a begin.


\section{Attacks on compiler optimizations}
\label{sec:compiler}

In this section, we model two attacks on compiler optimizations.  The first
attack exploits reordering allowed by relaxed memory models
(\S\ref{sec:info-flow-attack}).  The second exploits dead store elimination
(\S\ref{sec:dse}).

As in the previous section, the goal of the attacker is to learn one bit of
information from the high-security $\SEC$.   The attacks on compiler
optimizations require the $\SEC$ to be known to the compiler, for example a
static $\SEC$ or a JIT compiler.

To defend against these attacks it is sufficient to require a traditional
memory fence after each security check: compilers do not reorder instructions
over fences.

\subsection{Relaxed memory orders}
\label{sec:info-flow-attack}

Consider an attacker program, again using dynamic security checks to
try to learn a \verb|SECRET|. Whereas \verb|Spectre| uses
hardware capabilities, which have to be modeled by adding
extra capabilities to the language, this new attacker works
by exploiting relaxed memory which can result in
unexpected information flows. The attacker program is:
\[\begin{array}[t]{@{}l}
  \VAR x\GETS0\SEMI \VAR y\GETS0\SEMI\\\quad
    y\GETS x
  \PAR\begin{array}[t]{@{}l}
    \IF{y\EQ0}\THEN x\GETS1
    \brELIF{\CANREAD(\SEC)}\THEN x\GETS\SEC
    \brELSE x\GETS1\SEMI z\GETS1 \FI
\end{array}\end{array}\]
In the case where $\SEC$ is $2$, this has many executions,
one of which is:
\begin{tikzdisplay}[node distance=1em]
  \event{ix}{\DW{x}{0}}{}
  \event{iy}{\DW{y}{0}}{right=of ix}
  \event{rx0}{\DR{x}{0}}{below=of wx0}
  \event{wy0}{\DW{y}{0}}{right=of rx0}
  \event{ry0}{\DR{y}{0}}{below=of wy0}
  \event{wx1}{\DW{x}{1}}{right=of ry0}
  \nonevent{wx2}{\DW{x}{2}}{right=of wx1}
  \nonevent{wz1}{\DW{z}{1}}{right=of wx2}
  \po{rx0}{wy0}
  \po{ry0}{wx1}
  \po[out=30,in=150]{ry0}{wz1}
  \po[out=25,in=155]{ry0}{wx2}
  \rf{ix}{rx0}
  \rf{wy0}{ry0}
  \wk{iy}{wy0}
\end{tikzdisplay}
but there are no executions which exhibit
$(\DW{z}{1})$, since any attempt to do so
produces a cycle, since the value written
to $x$ has a control dependency on the value
read from $y$:
\begin{tikzdisplay}[node distance=1em]
  \event{ix}{\DW{x}{0}}{}
  \event{iy}{\DW{y}{0}}{right=of ix}
  \event{rx1}{\DR{x}{1}}{below=of ix}
  \event{wy1}{\DW{y}{1}}{right=of rx1}
  \event{ry1}{\DR{y}{1}}{below=of wy1}
  \event{wx1}{\DW{x}{1}}{right=of ry0}
  \nonevent{wx2}{\DW{x}{2}}{right=of wx1}
  \event{wz1}{\DW{z}{1}}{right=of wx2}
  \po{rx1}{wy1}
  \po{ry1}{wx1}
  \po[out=30,in=150]{ry1}{wz1}
  \po[out=25,in=155]{ry1}{wx2}
  \rf[in=-90,out=-150]{wx1}{rx1}
  \rf{wy1}{ry1}
  \wk[out=-20,in=90]{ix}{wx1}
  \wk[out=-20,in=120]{ix}{wx2}
  \wk{iy}{wy1}
\end{tikzdisplay}\vskip-\bigskipamount\noindent
In the case where \verb|SECRET| is $1$, there is an execution:
\begin{tikzdisplay}[node distance=1em]
  \event{ix}{\DW{x}{0}}{}
  \event{iy}{\DW{y}{0}}{right=of ix}
  \event{rx1}{\DR{x}{1}}{below=of ix}
  \event{wy1}{\DW{y}{1}}{right=of rx1}
  \event{ry1}{\DR{y}{1}}{below=of wy1}
  \event{wx1}{\DW{x}{1}}{right=of ry0}
  \event{wz1}{\DW{z}{1}}{right=of wx1}
  \po{rx1}{wy1}
  \po[out=30,in=150]{ry1}{wz1}
  \rf[in=-90,out=-150]{wx1}{rx1}
  \rf{wy1}{ry1}
  \wk[out=-20,in=90]{ix}{wx1}
  \wk[out=-20,in=120]{ix}{wx2}
  \wk{iy}{wy1}
\end{tikzdisplay}\vskip-\bigskipamount\noindent
Note that in this case, there is no dependency from
$(\DR{y}{1})$ to $(\DW{x}{1})$.  This lack of dependency makes the
execution possible. Thus, if the attacker sees
an execution with $(\DW{z}{1})$, they can conclude
that \verb|SECRET| is $1$, which is an information flow
attack.

This attack is not just an artifact of the model,
since the same behavior can be exhibited by
compiler optimizations. Consider the program fragment:
\[\begin{array}{l}
    \IF{y = 0}\THEN x\GETS1
    \brELIF{\CANREAD(\SEC)}\THEN x\GETS\SEC
    \brELSE x\GETS1\SEMI z\GETS1 \FI
\end{array}\]
In the case where \verb|SECRET| is a constant \verb|1|,
the compiler can inline it
and lift the assignment to $x$ out of the $\IF{}$ statement:
\[\begin{array}{l}
    x\GETS1\SEMI
    \IF{y = 0}\THEN
    \brELIF{\CANREAD(\SEC)}\THEN
    \brELSE z\GETS1 \FI
\end{array}\]
After this optimization, a sequentially consistent execution
exhibits $(\DW{z}{1})$. We discuss the practicality of this attack
further in \S\ref{sec:experiments}.

\subsection{Dead store elimination}
\label{sec:dse}

A common compiler optimization is \emph{dead store elimination},
in which writes are omitted if they will be overwritten by a subsequent
write later in the same thread. We can model eliminated writes
by ones with an unsatisfiable precondition. For example,
one execution of $(x \GETS 1\SEMI x \GETS 2) \PAR (r \GETS x)$ is:
\begin{tikzdisplay}[node distance=1em]
  \nonevent{wx1}{\DW{x}{1}}{}
  \event{wx2}{\DW{x}{2}}{right=of wx1}
  \event{rx2}{\DR{x}{2}}{right=2.5em of wx2}
  \wk{wx1}{wx2}
  \rf{wx2}{rx2}
\end{tikzdisplay}
Recall that for any satisfiable $\aEv$, if $\aEv$ reads $\aLoc$ from $\bEv$
then $\bEv$ is satisfiable. This means that, although we can eliminate
$(\DW{x}{1})$ we cannot eliminate $(\DW{x}{2})$.

One heuristic that a compiler might adopt is to only eliminate
writes that are guaranteed to be followed by another write
to the same variable. This can be formalized by saying that
a write event $\bEv$ is eliminable if
there is a tautology $\bEv \gtN \aEv$
which writes to the same location.
A model of dead store elimination is one where,
in every pomset, every eliminable event is unsatisfiable.
This model includes the example above.

Note that if dead store
elimination is \emph{always} performed, then there is an information
flow attack similar to the one in \S\ref{sec:info-flow-attack}. Consider
the program:
\[\begin{array}[t]{@{}l}
    y\GETS x
  \PAR\begin{array}[t]{@{}l}
    x\GETS 1\SEMI\\
    \IF{\CANREAD(\SEC})\THEN \IF{\SEC}\THEN x\GETS 2\FI
    \brELSE x\GETS 2\FI
\end{array}\end{array}\]
In the case that \verb|SECRET| is $0$, there is an execution:
\begin{tikzdisplay}[node distance=1em]
  \event{rx1}{\DR{x}{1}}{}
  \event{wy1}{\DW{y}{1}}{right=of rx1}
  \event{wx1}{\DW{x}{1}}{right=2.5em of wy1}
  \event{wx2}{\aForm \mid \DW{x}{2}}{right=of wx1}
  \rf[out=160,in=20]{wx1}{rx1}
  \po{rx1}{wy1}
  \wk{wx1}{wx2}
\end{tikzdisplay}
where $\aForm$ is ($\lnot$\verb|canRead(SECRET)|),
which is not a tautology, and so the $(\DW{x}{1})$ event is not eliminated.
In the case that \verb|SECRET| is not $0$, the matching execution
is:
\begin{tikzdisplay}[node distance=1em]
  \event{rx2}{\DR{x}{2}}{}
  \event{wy2}{\DW{y}{2}}{right=of rx2}
  \nonevent{wx1}{\DW{x}{1}}{right=2.5em of wy2}
  \event{wx2}{\DW{x}{2}}{right=of wx1}
  \rf[out=160,in=20]{wx2}{rx2}
  \po{rx2}{wy2}
  \wk{wx1}{wx2}
\end{tikzdisplay}
Now the $(\DW{x}{2})$ event is a guaranteed write, so the $(\DW{x}{1})$
is eliminated, and so cannot be read.
In the case that the attacker can rely on dead store
elimination taking place, this is an information flow: if the attacker observes
$x$ to be $1$, then they know \verb|SECRET| is $0$. We return to this attack
in \S\ref{sec:experiments}.

\section{Experiments}
\label{sec:experiments}

One theme of this paper is that optimizations not typically part of formal
abstractions can result in information flow leaks.
This is typified by the Spectre attack, which leverages speculative execution,
a hardware optimization.
\S\ref{sec:info-flow-attack} and~\S\ref{sec:dse} presented other attacks
along the same line, which leverage compiler optimizations.
These attacks also, unlike Spectre, do not rely on timing side channels, or
indeed timers of any kind, bypassing many common Spectre
mitigations~\cite{KohlbrennerShacham2016, FirefoxPerformanceNow}.

In this section we present implementations of the attacks described
in~\S\ref{sec:info-flow-attack} and~\S\ref{sec:dse}, in both cases
exploiting compiler optimizations to construct an information flow attack.
\ignore{
The attacker model (detailed in~\S\ref{subsec:attacker-model})
is currently unrealistic, as we attack C compilers rather than scripting
languages, and we require the secret to be a compile-time constant which
the compiler can optimize on.
This renders these attacks proof-of-concepts rather than
immediately exploitable vulnerabilities.
However, we believe their novelty may lead to
interesting discussion, and with much more development, these attacks may
evolve into genuine threats against targets such as JIT compilers.
}
We demonstrate the efficacy of our proof-of-concept attacks against
the {\CLANG} and {\GCC} C compilers.
All of our experiments are performed on a Debian 9 machine with an Intel
i7-6500U processor and 8 GB RAM;
we test against {\GCC}~6.3.0 and {\CLANG}~3.8.

\subsection{Attacker model}
\label{subsec:attacker-model}

As explained in Section~\ref{sec:examples}, our model expresses a variety
of attacks with differing attacker models.
The Spectre~(\S\ref{sec:spectre}) and
\textsc{Prime+Abort}~(\S\ref{sec:transactions}) attacks exploit optimizations
in hardware, and so can be mounted against a dynamic \SEC.
Our model captures this appropriately.
In contrast, the attacks from~\S\ref{sec:info-flow-attack}
and~\S\ref{sec:dse} leverage compiler optimizations and require the \SEC to
be known to the compiler, for example a static \SEC or a JIT compiler.
As our experimental section is devoted to these latter (novel) attacks, we
discuss the attacker model for these attacks in more detail.

In the attacker model for the compiler-optimization attacks, we assume that
there is a {\SEC} hardcoded into an application; for instance, {\SEC} may be
an API key.
This {\SEC} is known at compile time, but may not be
accessed except behind a security check.
Since the attacker is running with low security privileges,
the security check always fails,
so the attacker can only access the {\SEC} in dead code.
\ignore{
The attacker has no capabilities other than writing and executing code --- in
particular the attacker may not disassemble the compiler or libraries to learn
the {\SEC} directly; may not examine the internal state of the compiler;
may not access timers of any kind; and may not leverage hardware side channels.
}
The attacker's goal is to learn the value of the {\SEC}.

As a running hypothetical example, suppose there is a library that contains
a hardcoded {\SEC}: % such as an API key: --- API key mentioned just above
\begin{verbatim}
  static const uint SECRET = 0x1234;
  static volatile bool canReadSecret = false;
\end{verbatim}
The attacker is not allowed to write to \verb|canReadSecret| or read from {\SEC}
except after performing an \verb|if(canReadSecret)| check.

This is not necessarily a realistic attacker model,
since in most cases secrets are only known at run time rather than compile time,
which means that the attacks presented in this section
are more proof-of-concepts rather than immediately exploitable vulnerabilities.
However, the mechanisms we use are novel and could potentially be applied
in other contexts.
For instance, many real-world contexts allow untrusted or
third-party entities to write code in a scripting language which is then
compiled alongside and integrated into a larger application, often
using a just-in-time (JIT) compiler.
JavaScript code from third-party websites running in a browser is a common
example of this.
% We give an attacker similar capabilities against a
% compiler, except that we consider the simpler setting of using C code against a C
% compiler.
Although we consider only attacks using C code against a C compiler,
one could imagine a similar attack using JavaScript against browser JIT
compilers, where the compiler may have access to interesting secrets such as the
browser's cookie store, and may be able to optimize based on those secrets.
We plan to explore JavaScript attacks of this type as future work.

\subsection{Load-store reordering attack}
\label{subsec:exp-rel-mem}

We begin by examining the attack in~\S\ref{sec:info-flow-attack} in
more detail.
We show that by exploiting compiler optimizations which perform
load-store reordering, an attacker can learn the value of a compile-time
{\SEC} despite only being allowed to use it inside dead code.
%This attack was tested and works against {\GCC} version 6.3.0.
We verified that this attack succeeds against {\GCC} version 6.3.0.

The form of the attack presented in~\S\ref{sec:info-flow-attack} works in
theory, but in practice, just because a compiler is \emph{allowed} to perform a
load-store reordering doesn't mean that it \emph{will}.
We found that {\GCC} and {\CLANG} chose to read $y$ into a
register first (before writing to $x$), regardless of the value of
{\SEC}.
% However, we did find a related pattern in which {\GCC} will emit a
However, using a similar program we were able to coax {\GCC} to emit a
different ordering of the read of $y$ and the write of $x$ depending
on the value of a {\SEC}:
\[\begin{array}[t]{@{}l}
  \VAR x\GETS0\SEMI \VAR y\GETS0\SEMI\\\quad
    y\GETS x
  \PAR\begin{array}[t]{@{}l}
    x\GETS 1\SEMI\\
    \IF{\texttt{canReadSecret}}\THEN x\GETS\SEC\FI\\
    \IF{y > 0}\THEN z\GETS 0 \ELSE z\GETS 1\FI
\end{array}\end{array}\]
\ignore{
\begin{verbatim}
    x := 0; y := 0;
    (
      y := x;
    ) || (
      x := 1;
      if(canReadSecret) { x = SECRET; }
      if (y) { return 0; }
      else { return 1; }
    )
\end{verbatim}
}

Figure~\ref{fig:lsr-asm} shows the assembly output of {\GCC} on this program in the cases
where {\SEC} is 0 and 1 respectively.
In the case that {\SEC} is $1$, {\GCC} removes the $\IF{}$
statement entirely, and moves the read of $y$ above the write of $x$.
However, when {\SEC} is $0$, the $\IF{}$ statement must remain
intact, and {\GCC} does not move the read of $y$.
This means that if {\SEC} is $1$, the second thread will always
read $y\EQ0$ and always assign $z\GETS1$.
However, if {\SEC} is $0$, it is possible that the first thread
may observe $x\EQ1$ and write $y\GETS1$ in time for the second thread
to observe $y\EQ1$ and thus assign $z\GETS0$.
In this way, we leverage compiler load-store reordering to learn the value of
a compile-time {\SEC}.

\begin{figure}
  \begin{tabular}{p{3.4cm} @{\quad} | p{3.4cm}}
    \texttt{SECRET == 0} & \texttt{SECRET == 1} \\ \hline
\begin{verbatim}
  mov s(%rip), %eax
  mov $1, x(%rip)
  test %eax, %eax
  je label1
  mov $0, x(%rip)
label1:
  mov y(%rip), %eax
  test %eax, %eax
  sete %eax
\end{verbatim}
  &
\begin{verbatim}
  mov s(%rip), %eax
  mov y(%rip), %eax
  mov $1, x(%rip)
  test %eax, %eax
  sete %eax
\end{verbatim}
  \end{tabular}
  \caption{
    Simplified x86 assembly output from \texttt{gcc} for the main thread of
    the load-store reordering attack.
    In particular, note that the order between (\texttt{mov \$1, x(\%rip)})
    and (\texttt{mov y(\%rip), \%eax}) is different in the two cases.
    References to the \texttt{canReadSecret} variable have been shortened to
    \texttt{s} for the figure.
  }
  \label{fig:lsr-asm}
\end{figure}

We extend this attack to leak a secret consisting of an arbitrary number
\verb|N| of bits.
To do this, we compile \verb|N| copies of the test function, each
performing a boolean test on a single bit of {\SEC}.
So that the bit value is
constant at compile time, we must compile a separate function for each bit, rather than execute the same
code repeatedly in a loop.
\ignore{
The function used for reading the \verb|k|th bit is as follows (for
\verb|N <= 64|):
\[\begin{array}[t]{@{}l}
  \VAR x\GETS0\SEMI \VAR y\GETS0\SEMI\\\quad
    y\GETS x
  \PAR\begin{array}[t]{@{}l}
    x\GETS 1\SEMI\\
    \IF{\texttt{canReadSecret}}\THEN x\GETS\texttt{(\SEC\, \& (1 << k)) ? 1 : 0}\SEMI\FI\\
    \IF{y > 0}\THEN \RETURN 0
    \brELSE \RETURN 1\FI
\end{array}\end{array}\]
Following the same analysis as above, this function will always return $1$
if {\SEC} is $1$, but may return $0$ if {\SEC} is $0$.
The extension of the attack to the general case with truly arbitrary \verb|N|
is straightforward; {\SEC} becomes an array of 64-bit values, and we use
\verb|k / 64| and \verb|1 << (k & 63)| as the array index and bitmask
respectively.
}

We make three additional tweaks to improve the reliability, so that the attacker
can confidently infer the value of {\SEC} based on the observed value of $z$.
  First, rather than performing $y\GETS x$ only once in the forwarding thread,
we perform $y\GETS x$ continuously in a loop.
This maximizes the probability that, once $x\GETS 1$ occurs in the main
thread, $y$ will be immediately assigned $1$ by the forwarding thread
and the main thread will be able to read $y\EQ 1$.

Second, we wish to lengthen the timing window between $x\GETS 1$ and the
read of $y$ in the main thread (in the case where
{\SEC} is $0$ and the read of $y$ remains below $x\GETS 1$).
However, we wish to do this in a way that does not block the reordering of the
read of $y$ upwards in the case where {\SEC} is $1$.
We do this by inserting many copies of the line
\[
  \IF{\texttt{canReadSecret}} \THEN x\GETS\SEC\FI
\]
instead of just one.
In the case where {\SEC} is $0$, this
results in many reads of \verb|canReadSecret| and many conditional jumps,
which in practice creates a timing window for the forwarding thread to perform
$y\GETS x$.
However, in the case where {\SEC} is $1$,
all of these inserted lines can be removed just as a single copy could be.
In practice, we found that inserting too many copies of the line prevents
{\GCC} from reordering the read of $y$ above the write to $x$ as
desired; inserting $30$ copies was sufficient to create a timing window
while still allowing the desired reordering.

Finally, we redundantly execute the entire attack several times, noting the
value of $z$ in each case.
We note that if \emph{any} of the redundant runs produces a value of
$z\EQ0$ for a particular bit position, then we can be certain that the
corresponding bit of {\SEC} \emph{must} be $0$, as it implies the
read of $y$ was not reordered upwards in that particular function.
On the other hand, the more runs that produce a value of $z\EQ1$ for a
particular bit position, the more certain we can be that the read of $y$
was reordered above the $x\GETS 1$ assignment, and {\SEC} is $1$.

\begin{figure}
  \small
  \begin{tabular}{ r | l | l | l }
    Redundancy & Bandwidth (bits/s) & Bitwise Acc & Per-run Acc \\ \hline
    1          & 3.14 million       & 90.89\%     & 1.9\%       \\
    2          & 1.56 million       & 96.04\%     & 8.1\%       \\
    3          & 1.04 million       & 98.09\%     & 10.0\%      \\
    4          & 783 thousand       & 98.98\%     & 24.3\%      \\
    5          & 626 thousand       & 99.71\%     & 50.2\%      \\
    7          & 447 thousand       & 99.91\%     & 70.6\%      \\
    10         & 314 thousand       & 99.991\%    & 93.8\%      \\
    15         & 208 thousand       & 99.994\%    & 95.5\%      \\
    20         & 157 thousand       & 99.9995\%   & 99.2\%      \\
    30         & 105 thousand       & 99.99995\%  & 99.9\%      \\
  \end{tabular}
  \caption{
    Performance results for the load-store reordering attack when leaking a
    2048-bit secret.
    `Redundancy' is the number of redundant runs performed for error
    correction; more redundant runs improves accuracy but reduces bandwidth.
    `Bandwidth' is the number of bits leaked per second after accounting for
    any error correction.
    `Bitwise Accuracy' is the percentage of bits that were correct, while
    `Per-run Accuracy' is the percentage of full 2048-bit secrets that were
    correct in all bit positions.
  }
  \label{fig:load-store-perf}
\end{figure}

Figure~\ref{fig:load-store-perf} gives the performance results for this attack
against {\GCC} version 6.3.0.
The attack can sustain hundreds of thousands of bits per second leaked with
near-perfect accuracy, or millions of bits per second with error rates of a
few percent.
This means that an attacker can leak a 2048-bit secret with near-perfect
accuracy in under $10$ ms.
Note that this bandwidth assumes that all copies of the attack function are
already compiled; the cost of compilation is not included here.

\subsection{Dead store elimination attack}
\label{subsec:exp-dse}

In this section we return to the attack in~\S\ref{sec:dse} based on
dead store elimination.
We show that in our attacker model (given in~\S\ref{subsec:attacker-model}),
the attacker is able to exploit dead
store elimination to again learn the value of a compile-time {\SEC}
despite only being allowed to use it inside dead code.
This attack is even more efficient than the attack on load-store reordering,
and further, we were able to demonstrate its effectiveness against both
{\GCC} and {\CLANG}.

We start from the simple form of the attack presented in~\S\ref{sec:dse},
and extend it to leak a secret consisting of an
arbitrary number of bits, in the same way that we extended the load-store
reordering attack.
\ignore{
As we did in the load-store reordering attack, we again compile \verb|N| copies
of the test function, each performing a boolean test on a single bit of the
secret.
The function used for reading the \verb|k|th bit is as follows (for
\verb|N <= 64|):
\[\begin{array}[t]{@{}l}
  \VAR x\GETS0\SEMI\\\quad
    r\GETS x
  \PAR\begin{array}[t]{@{}l}
    x\GETS 1\SEMI\\
    \IF{\CANREAD(\SEC})\THEN \IF{\SEC\,\texttt{ \& (1 << k}}\NOTEQ0)\THEN x\GETS 2\FI
    \brELSE x\GETS 2\FI
\end{array}\end{array}\]
\ignore{
\begin{verbatim}
    (
      r := x;
    ) || (
      x := 1;
      if (canRead(SECRET)) {
        if (SECRET & (1 << k)) { x := 2; }
      } else {
        x := 2;
      }
    )
\end{verbatim}
}
Then, we test each function in turn, each time noting the value of $y$
observed by the `listening' thread.
If {\SEC} is 1, the $x\GETS 2$ assignment is
guaranteed to happen, so the compiler can eliminate the $x \GETS 1$
assignment as a dead store and we will observe $y\EQ 2$; however, if
{\SEC} is 0, the $x\GETS 1$ assignment cannot be
eliminated, and we will observe $y\EQ 1$ with some probability.
The extension of the attack to the general case with truly arbitrary \verb|N|
is straightforward and proceeds exactly as it did for the attack on
load-store reordering.
}
We make three additional tweaks to improve the reliability so that the attacker
can confidently infer the value of {\SEC}.
Two of them follow exactly the same pattern as the reliability tweaks
for the load-store reordering attack in~\S\ref{subsec:exp-rel-mem} ---
continuously forwarding $x$ to $y$ in the forwarding thread, and running the
entire attack multiple times.
The remaining tweak is again motivated by increasing the timing window in
which the forwarding can happen, but differs in some details from the
implementation in~\S\ref{subsec:exp-rel-mem}.

\ignore{
First, rather than simply observing $x$ with $y\GETS x$ in the
`listening' thread, we continuously load $x$ in a loop until a
nonzero value is observed --- i.e., we perform
$\DO{r\GETS x} \WHILE(r\EQ0)$.
\ignore{
\begin{verbatim}
    do {
      r := x;
    } while(r == 0);
\end{verbatim}
}
This remedies the case where $y\GETS x$ could observe a value of $x$
from `before' either of the two possible writes performed by the other thread.
}

To increase the timing window, we insert additional time-consuming
computation immediately following
the $x\GETS 1$ operation in the main thread.
This increases the likelihood that the listening thread will be able to observe
$x\EQ 1$ (unless the $x\GETS 1$ write was eliminated).
Inserting this computation should be done without interfering with the dead store
elimination process itself, so that the compiler will continue to eliminate
the $x\GETS 1$ write if and only if {\SEC} was 1.
For {\GCC}, we have a fair amount of freedom with the time-consuming
computation --- for instance, we can use an arbitrarily long loop.
In fact, we can perform a further optimization by monitoring the value of the
variable $y$ (written to by the listening thread) and breaking out of the
loop early if we see that the listening thread has already observed $x\EQ 1$.
However, with {\CLANG}, we cannot use a loop at all --- the time-consuming
computation must be branch-free and, furthermore, must not consist of too many
instructions.
\ignore{
This is because {\CLANG}'s dead store elimination pass operates only
within basic blocks, and uses a heuristic to stop scanning the basic block
early if it is too large.
}
Nonetheless, we find that even with these restrictions, we are able to
construct a reliable and fast attack against both {\CLANG} and {\GCC}.

\ignore{
Finally, we redundantly execute the entire attack several times, noting the
final value of $y$ (the first observed nonzero value of $x$) in each
case.
We note that if \emph{any} of the redundant runs produces $y\EQ 1$ for a
particular bit position, we can be certain that the corresponding bit of
{\SEC} \emph{must} be $0$, as it implies that the $x\GETS 1$ write
was not eliminated in that particular function.
On the other hand, the more runs that observe $y\EQ 2$ in a particular bit
position despite our other reliability-increasing measures taken above, the
more certain we can be that the $x\GETS 1$ write was eliminated in that
function, and {\SEC} is $1$.
}

\begin{figure}
  \small
  \begin{tabular}{ r | l | l | l }
    Redundancy & Bandwidth (bits/s) & Bitwise Acc & Per-run Acc \\ \hline
    1          & 1.19 million       & 99.991\%    & 95.6\%      \\
    2          & 597 thousand       & 99.99986\%  & 99.7\%      \\
    3          & 397 thousand       & 100.0\%     & 100.0\%     \\
  \end{tabular}
  \caption{
    Performance results for the dead store elimination attack on {\CLANG} when
    leaking a 2048-bit secret.
    Terms are the same as defined in the caption for Figure~\ref{fig:load-store-perf}.
  }
  \label{fig:clang-dse-perf}
\end{figure}

\begin{figure}
  \small
  \begin{tabular}{ r | c | c | c }%| c | c | c }
    Stall amount & 10 &
%                   20 &
%                   50 &
                   100 &
%                   200 &
                   500 \\ \hline
    Redundancy 1 & \makecell{2.54 million\\98.15\%} &
%                   \makecell{2.36 million\\99.80\%} &
%                   \makecell{1.95 million\\99.987\%} &
                   \makecell{1.54 million\\99.996\%} &
%                   \makecell{1.12 million\\99.993\%} &
                   \makecell{584 thousand\\99.998\%} \\ \hline
    Redundancy 2 & \makecell{1.24 million\\99.73\%} &
%                   \makecell{1.17 million\\99.993\%} &
%                   \makecell{989 thousand\\100.0\%} &
                   \makecell{774 thousand\\100.0\%} &
%                   \makecell{553 thousand\\100.0\%} &
                   \makecell{295 thousand\\100.0\%} \\ \hline
    Redundancy 3 & \makecell{841 thousand\\99.94\%} &
%                   \makecell{784 thousand\\100.0\%} &
%                   \makecell{666 thousand\\100.0\%} &
                   \makecell{521 thousand\\100.0\%} &
%                   \makecell{370 thousand\\100.0\%} &
                   \makecell{201 thousand\\100.0\%} \\ \hline
    Redundancy 4 & \makecell{620 thousand\\99.992\%} &
%                   \makecell{585 thousand\\100.0\%} &
%                   \makecell{499 thousand\\100.0\%} &
                   \makecell{387 thousand\\100.0\%} &
%                   \makecell{285 thousand\\100.0\%} &
                   \makecell{145 thousand\\100.0\%} \\
  \end{tabular}
  \caption{
    Performance results for the dead store elimination attack on {\GCC} when
    leaking a 2048-bit secret.
    Rows give different values of `redundancy' (as defined in previous figures),
    while columns give amounts of stall time immediately following the
    $x\GETS 1$ write (as measured in loop iterations).
    Each table cell gives the leak bandwidth in bits/sec, followed by the
    bitwise accuracy.
  }
  \label{fig:gcc-dse-perf}
\end{figure}
% \begin{figure*}
%   \small
%   \begin{tabular}{ r | c | c | c | c | c | c }
%     Stall amount & 10 & 20 & 50 & 100 & 200 & 500 \\ \hline
%     Redundancy 1 & \makecell{2.54 million\\98.15\%} &
%                    \makecell{2.36 million\\99.80\%} &
%                    \makecell{1.95 million\\99.987\%} &
%                    \makecell{1.54 million\\99.996\%} &
%                    \makecell{1.12 million\\99.993\%} &
%                    \makecell{584 thousand\\99.998\%} \\ \hline
%     Redundancy 2 & \makecell{1.24 million\\99.73\%} &
%                    \makecell{1.17 million\\99.993\%} &
%                    \makecell{989 thousand\\100.0\%} &
%                    \makecell{774 thousand\\100.0\%} &
%                    \makecell{553 thousand\\100.0\%} &
%                    \makecell{295 thousand\\100.0\%} \\ \hline
%     Redundancy 3 & \makecell{841 thousand\\99.94\%} &
%                    \makecell{784 thousand\\100.0\%} &
%                    \makecell{666 thousand\\100.0\%} &
%                    \makecell{521 thousand\\100.0\%} &
%                    \makecell{370 thousand\\100.0\%} &
%                    \makecell{201 thousand\\100.0\%} \\ \hline
%     Redundancy 4 & \makecell{620 thousand\\99.992\%} &
%                    \makecell{585 thousand\\100.0\%} &
%                    \makecell{499 thousand\\100.0\%} &
%                    \makecell{387 thousand\\100.0\%} &
%                    \makecell{285 thousand\\100.0\%} &
%                    \makecell{145 thousand\\100.0\%} \\
%   \end{tabular}
%   \caption{
%     Performance results for the dead store elimination attack on {\GCC} when
%     leaking a 2048-bit secret.
%     Rows give different values of `redundancy' (as defined in previous figures),
%     while columns give amounts of stall time immediately following the
%     $x\GETS 1$ write (as measured in loop iterations).
%     Each table cell gives the leak bandwidth in bits/sec, followed by the
%     bitwise accuracy.
%   }
%   \label{fig:gcc-dse-perf}
% \end{figure*}

Performance results for the dead store elimination attack against {\CLANG}
are given in Figure~\ref{fig:clang-dse-perf}, and against {\GCC} are given in
Figure~\ref{fig:gcc-dse-perf}.
Both attacks are faster than the load-store-reordering attack from
\S\ref{subsec:exp-rel-mem} when comparing settings which give the same
accuracy.
In particular, the attack on {\GCC} can leak a 2048-bit cryptographic
key with perfect accuracy (in our tests) in about $2$ ms.

\section{Conclusions and future work}

In this paper, we have presented a model of speculative evaluation and
shown that it captures non-trivial properties of speculations produced
by hardware, compiler optimizations, and transactions. These properties
include information flow attacks: in the case of hardware and transactions
this is modeling known attacks~\cite{DBLP:journals/corr/abs-1801-01203,DBLP:conf/uss/DisselkoenKPT17},
but in the case of compiler optimizations the attacks are new, and were
discovered as a direct result of developing the model. We have experimentally
validated that the attacks can be carried out against \verb|gcc| and \verb|clang|,
though only against secrets known at compile time.

We have tried where possible to abstract away from the
micro-architectural details that enable attackers to exploit
speculation, while still trying to capture the ``essence'' of
Spectre. There are trade-offs with any such abstraction, as
higher-level abstractions make program behavior easier to understand
and reason about, but at the cost of ignoring potential attacks. One
software developer's useful abstraction is another's ignoring the
difficult issues.

As a concrete instance, one feature of Spectre we have glossed over is
the ability of the attacker to influence speculation, for example by
training the branch predictor or influencing the contents of caches.
We expect that such attacker influence could be modeled using a
mechanism similar to the speculation barriers of \S\ref{sec:spec-barriers},
but under the control of the attacker rather than the honest agents.

The paper's primary focus is not weak memory, and the model of relaxed
memory used in this paper is deliberately simplified, compared for
example to
C11~\cite{Boehm:2008:FCC:1375581.1375591,Batty:2011:MCC:1926385.1926394}. Nonetheless,
we believe that the model developed in the paper has promise as a
semantics for relaxed memory. Our model appears to be the first in the
literature that both validates all of the JMM causality test cases and
also forbids thin air behavior; the most prominent existing models are
either too permissive~\cite{Manson:2005:JMM:1047659.1040336,
  Jagadeesan:2010:GOS:2175486.2175503,Kang-promising-2017} or
too conservative~\cite{DBLP:conf/lics/JeffreyR16}.  In separate work,
we are exploring the usual properties of weak memory, such as
comparisons with sequentially consistent models,
optimization soundness, or compilation soundness.  While our model of
transactions shows the flexibility of our model, in this future work,
we will include known features of hardware, including locks, fences,
and read-modify-write instructions.  This development is not core to
the basic findings of this paper.



UNUSED: In particular our model of reads-from is strong, and could be weakened
by replacing the requirement $\bEv\lt\aEv$ in Definition~\ref{def:rf} by $\bEv\gtN\aEv$. It remains to be seen how this impacts the model.
%% in particular the logical formulation of $\aLoc$-closure in
%% \S\ref{sec:logic} as
%% $((\DR{\aLoc}{\aVal}) \Rightarrow \once(\DW{\aLoc}{\aVal}))$
%% would no longer be sound.
% The model is also not considering coherence, though we speculate it
% can be added by requiring that for each $\aLoc$, $\gtN$ form a total
% order when restricted to events that write to $\aLoc$.

The design space for transactions is very rich~\cite{DBLP:journals/pacmpl/DongolJR18}.
We have only presented one design choice, and it remains to be seen how other
design choices could be adopted. For example, we have chosen not to distinguish
commits that are aborted due to transaction failure from commits which are aborted
for other reasons, such as failed speculation.

In future work, it would be interesting to see if full-abstraction
results for pomsets \cite{Plotkin:1997:TSP:266557.266600} can be extended to
3-valued pomsets.

One interesting feature of this model is that (in the language
of~\cite{Pichon-Pharabod:2016:CSR:2837614.2837616}) it is a
\emph{per-candidate execution model}, in that the correctness of an
execution only requires looking at that one execution, not at
others. This is explicit in memory models such
as~\cite{Jagadeesan:2010:GOS:2175486.2175503,Kang-promising-2017} in which
``alternative futures'' are explored, in a style reminiscent of
Abramsky's bisimulation as a testing equivalence~\cite{ABRAMSKY1987225}. Models of
information flow are similar, in that they require comparing different
runs to test for the presence of dependencies~\cite{Clarkson:2010:HYP:1891823.1891830}. In contrast, the model
presented here explicitly captures dependency in the pomset order, and
models multiple runs by giving the semantics of $\IF{}$ in terms of a
concurrent semantics of both branches.
In the parlance of information flow~\cite{Barthe:2004:SIF:1009380.1009669},
the humble conditional suffices to construct a composition operator to detect information flow  in the presence of speculation.

\begin{small}
\bibliography{bib}
\end{small}

\appendix
\subsection{Operations on sets of pomsets}
\label{app:sets-of-pomsets}

Here we give the formal definitions for the operations described at the
beginning of \S\ref{sec:semantics}.

In order to model speculation barriers in \S\ref{sec:spec-barriers},
we partition the actions into \emph{mergeable} and \emph{unmergeable}.

In transactional memory, begin and commit actions are memory fences: that is,
they are a barrier to reordering memory accesses.  To capture this (and other
memory barriers), we identify sets $\Rel$ and $\Acq \subseteq\Act$.  For
transactions, we have $(\DB{\aVal})\in\Acq$ for begins,
$(\DC{\aVal})\in \Rel$ for commits.  We say that $\aAct$ is a \emph{release}
if $\aAct\in\Rel$ and $\aAct$ is an \emph{acquire} if $\aAct\in\Acq$.
% In a pomset, a release event is one
% labeled with a release action, and an acquire event is one labeled by an
% acquire action.  

\begin{definition}
  \label{def:prefix}
Let $\aAct \prefix \aPSS$ be the set $\aPSS'$ where $\aPS'\in\aPSS'$ whenever
there is $\aPS\in\aPSS$ such that:
\begin{itemize}
\item $\Event' = \Event \cup \{\cEv\}$,
\item if $\bEv \le \aEv$ then $\bEv \le' \aEv$,
\item if $\aEv \gtN \bEv$ then $\aEv \gtN' \bEv$,
\item if $\cEv\in\Event$ then $\cEv$ is mergeable,
\item $\labeling'(\cEv) = (\aForm, \aAct)$, and
\item if $\labeling(\aEv) = (\bForm \mid \bAct)$ then $\labeling'(\aEv) =
  (\bForm' \mid \bAct)$, where:
  \begin{itemize}
  \item $\cEv \lt' \aEv$ whenever $\aAct$ is an acquire or $\bAct$ is a release, 
  \item if $\aAct$ is an acquire then $\bForm$ is independent of every $\bLoc$,
  \item if $\aAct$ and $\bAct$ both touch the same location and one is a write,
    then $\cEv \gtN' \aEv$, and
  \item $\bForm'$ implies \(\left\{\begin{array}{l@{~}ll}
    % \bForm[\aVal/\aLoc]                     & \mbox{if $\aAct$ reads $\aVal$ from $\aLoc$ and $\cEv\lt'\aEv$} & \textsc{[dependent read]} \\
    % \bForm[\aVal/\aLoc] \text{ and } \bForm & \mbox{if $\aAct$ reads $\aVal$ from $\aLoc$}                  & \textsc{[independent read]} \\
    % \bForm                                  & \mbox{otherwise}                                              & \textsc{[non-read]} \\        
    \bForm[\aVal/\aLoc]                     \\\quad \mbox{if $\aAct$ reads $\aVal$ from $\aLoc$ and $\cEv\lt'\aEv$} \\\qquad \textsc{[dependent read]} \\[\jot]
    \bForm[\aVal/\aLoc] \text{ and } \bForm \\\quad \mbox{if $\aAct$ reads $\aVal$ from $\aLoc$}                  \\\qquad \textsc{[independent read]} \\[\jot]
    \bForm                                  \\\quad \mbox{otherwise}                                              \\\qquad \textsc{[non-read]} \\
  \end{array}\right.\)
  \end{itemize}
\end{itemize}
\end{definition}
The first constraint ensures that events are ordered before a release and
after an acquire.  The second constraint ensures that thread-local reads do
not cross acquire fences.

% Prefixing is used to define the semantics of reads and writes, and
% adds a new event $\cEv$ with action $\aAct$.  As in the definition
% of parallel composition, the definition allows the new event to overlap with
% events in $\aPSS$ as long as they agree on the action.

The tricky parts of the
definition are the named cases, which place requirements on read
dependencies.  If $\aAct$ reads $\aVal$ from $\aLoc$, we have to
decide whether $\aEv$ depends on $\cEv$ for some $\aEv$ with old
precondition $\bForm$ and new precondition $\bForm'$. The first case
\textsc{[dependent read]} is that the dependency exists, in which case
$\bForm'$ just has to imply $\bForm[\aVal/\aLoc]$. The more interesting 
case is \textsc{[independent read]}, in which case $\bForm'$ has to imply
$\bForm[\aVal/\aLoc]$ and $\bForm$. This corresponds to a case where
$\aEv$ can be performed with or without $\cEv$.
In particular, if $\bForm$ is independent of $\aLoc$ then we can pick
$\bForm'$ to be $\bForm$, and the independent read case will apply.
% For example,
% if $\aAct$ and $\bAct$ write to the same location, $\aAct$ reads $\aVal$ from $\aLoc$, $\bForm$ is independent of $\aLoc$,
% and
% $\aPSS$ contains:
% \begin{tikzdisplay}[node distance=1em]
%   \event{b}{\bForm \mid \bAct}{}
%   \event{c}{\cForm \mid \cAct}{right=of b}
%   \po{b}{c}
% \end{tikzdisplay}
% then $\aAct\prefix\aPSS$ contains:
% \begin{tikzdisplay}[node distance=1em]
%   \event{a}{\aForm \mid \aAct}{}
%   \event{b}{\bForm \mid \bAct}{right=of a}
%   \event{c}{\cForm[\vec\aVal/\vec\aLoc] \mid \cAct}{right=of b}
%   \po[out=25,in=155]{a}{c}
%   \wk{a}{b}
%   \po{b}{c}
% \end{tikzdisplay}


\begin{definition}
Let $\aPS_0 \in (\aPSS_1 \parallel \aPSS_2)$
whenever there are $\aPS_1 \in \aPSS_1$ and $\aPS_2 \in \aPSS_2$ such that:
\begin{itemize}
\item $\Event_0 = \Event_1 \cup \Event_2$,
\item if $\aEv \le_1 \bEv$ or $\aEv \le_2 \bEv$ then $\aEv \le_0 \bEv$,
\item if $\aEv \gtN_1 \bEv$ or $\aEv \gtN_2 \bEv$ then $\aEv \gtN_0 \bEv$,
\item if $\aEv\in\Event_1\cap\Event_2$ then $\aEv$ is mergeable,
\item if $\labeling_0(\aEv) = (\aForm_0 \mid \aAct)$ then either:
  \begin{itemize}
  \item $\labeling_1(\aEv) = (\aForm_1 \mid \aAct)$ and $\labeling_2(\aEv) = (\aForm_2 \mid \aAct)$
    and $\aForm_0$ implies $\aForm_1 \lor \aForm_2$,
  \item $\labeling_1(\aEv) = (\aForm_1 \mid \aAct)$ and $\aEv \not\in \Event_2$
    and $\aForm_0$ implies $\aForm_1$, or
  \item $\labeling_2(\aEv) = (\aForm_2 \mid \aAct)$ and $\aEv \not\in \Event_1$
    and $\aForm_0$ implies $\aForm_2$.
  \end{itemize}
\end{itemize}
\end{definition}
% We use $\aPSS_1 \parallel \aPSS_2$ in defining the semantics of conditionals
% and concurrency.
% It contains the union of pomsets from $\aPSS_1$ and $\aPSS_2$,
% allowing overlap as long as they agree on actions. For example, if
% $\aPSS_1$ and $\aPSS_2$ contain:
% \begin{tikzdisplay}[node distance=1em]
%   \event{a}{\aForm \mid \aAct}{}
%   \event{b}{\bForm_1 \mid \bAct}{right=of a}
%   \po{a}{b}
% \end{tikzpicture}\qquad\qquad\begin{tikzpicture}[node distance=1em]
%   \event{b}{\bForm_2 \mid \bAct}{}
%   \event{c}{\cForm \mid \cAct}{right=of b}
%   \wk{b}{c}
% \end{tikzdisplay}
% then $\aPSS_1 \parallel \aPSS_2$ contains:
% \begin{tikzdisplay}[node distance=1em]
%   \event{a}{\aForm \mid \aAct}{}
%   \event{b}{\bForm_1 \lor \bForm_2 \mid \bAct}{right=of a}
%   \event{c}{\cForm \mid \cAct}{right=of b}
%   \po{a}{b}
%   \wk{b}{c}
% \end{tikzdisplay}

\begin{definition}
Let $\aPSS[\aExp/\aLoc]$ be the set $\aPSS'$ where $\aPS'\in\aPSS'$ whenever
there is $\aPS\in\aPSS$ such that:
\begin{itemize}
\item $\Event' = \Event$,
\item if $\bEv \le \aEv$ then $\bEv \le' \aEv$, and
\item if $\aEv \gtN \bEv$ then $\aEv \gtN' \bEv$, and
\item if $\labeling(\aEv) = (\bForm \mid \aAct)$ then $\labeling'(\aEv) = (\bForm[\aExp/\aLoc] \mid \aAct)$.
\end{itemize}
and similarly for $\aPSS[\aLoc/\aReg]$.
\end{definition}

\begin{definition}
Let $(\aForm \guard \aPSS)$ be the subset of $\aPSS$ such that $\aPS\in\aPSS$ whenever:
\begin{itemize}
\item if $\labeling(\aEv) = (\bForm \mid \aAct)$ then $\aForm$ implies $\bForm$.
\end{itemize}
\end{definition}



\begin{definition}
\label{def:x-closed}
  A 3-valued pomset is $\aLoc$-closed if,
  for every $\aEv\in\Event$:
  \begin{itemize}
  \item $\aEv$ is independent of $\aLoc$, and
  \item if $\aEv$ reads from $\aLoc$, then there is a $\bEv$ such that $\aEv$ can read $\aLoc$ from $\bEv$.
  \end{itemize}
\end{definition}

The definitions as they stand allow cycles in weak edges. This is necessary for examples such
as $(\aLoc\GETS\bLoc-1\SEMI \aLoc\GETS1 \PAR \bLoc\GETS\aLoc-1\SEMI\bLoc\GETS1)$
which has execution:
\begin{tikzdisplay}[node distance=1em]
  \event{ry1}{\DR{\bLoc}{1}}{}
  \event{wx0}{\DW{\aLoc}{0}}{right=of ry1}
  \event{wx1}{\DW{\aLoc}{1}}{right=of wx0}
  \event{rx1}{\DR{\aLoc}{1}}{right=2.5em of wx1}
  \event{wy0}{\DW{\bLoc}{0}}{right=of rx1}
  \event{wy1}{\DW{\bLoc}{1}}{right=of wy0}
  \wk{wx0}{wx1}
  \wk{wy0}{wy1}
  \po{ry1}{wx0}
  \po{rx1}{wy0}
  \rf{wx1}{rx1}
  \rf[in=20,out=160]{wy1}{ry1}
\end{tikzdisplay}
However, in order to model release/acquire fencing in transactions, we need to ban
executions such as:
\begin{tikzdisplay}[node distance=1em]
  \event{wx0}{\DW{\aLoc}{0}}{}
  \event{wx1}{\DW{\aLoc}{1}}{below=7ex of wx0}
  \event{wy1}{\DC{}}{below right=of wx0}
  \event{ry1}{\DB{}}{right=2.5em of wy1}
  % \event{wy1}{\DWRel{\bLoc}{1}}{below right=of wx0}
  % \event{ry1}{\DRAcq{\bLoc}{1}}{right=2.5em of wy1}
  \event{rx0}{\DR{\aLoc}{0}}{above right=of ry1}
  \event{rx1}{\DR{\aLoc}{1}}{below=7ex of rx0}
  \wk[out=-75,in=75]{wx0}{wx1}
  \wk[out=105,in=-105]{wx1}{wx0}
  \po{wx0}{wy1}
  \po{wx1}{wy1}
  \po{ry1}{rx0}
  \po{ry1}{rx1}
  \rf{wy1}{ry1}
  \rf{wx0}{rx0}
  \rf{wx1}{rx1}
\end{tikzdisplay}
The problem here is the weak cycle between $(\DW\aLoc0)$ and $(\DW\aLoc1)$,
which according to Definition~\ref{def:rf}, allows both $(\DR\aLoc0)$ and
$(\DR\aLoc1)$, even though one of them must be a stale value. This can be addressed by
requiring $\gtN$ to form a \emph{per-location} partial order. This is a form
of partial coherence, and can be strengthened to total coherence by requiring
$\gtN$ to be a per-location total order.

\begin{definition}
  A 3-valued pomset is \emph{partially} (resp.~\emph{totally}) $\aLoc$-\emph{coherent}
  if, when restricted to events which touch $\aLoc$,
  $\gtN$ forms a partial (resp.~total) order.
\end{definition}

\begin{definition}
Let $(\nu\aLoc\DOT\aPSS)$ be the subset of $\aPSS$ such that $\aPS\in\aPSS$ whenever
$\aPS$ is $\aLoc$-closed and partially $\aLoc$-coherent.
\end{definition}

\subsection{Blockers}
\label{app:blockers}

Recall the preliminary definition of reads-from in \S\ref{sec:pomsets}, which
defined an $\aLoc$-blocker to be and event $\cEv$ that writes to $\aLoc$ such that
$\bEv \lt \cEv \lt \aEv$.  Were we to adopt this definition, then concurrent
threads could turn events that were not $\aLoc$-blockers into an
$\aLoc$-blocker, even if the new thread does not mention $\aLoc$.

To see this, consider the program
\begin{math}
  (
  \aLoc\GETS1\SEMI
  \bLoc\GETS\aLoc
  \PAR
  \aLoc\GETS\cLoc+1\SEMI
  \bLoc\GETS\aLoc
  \PAR
  \IF{z=2}\THEN\aReg\GETS\aLoc\FI
  )
\end{math}
with execution:
\begin{tikzdisplay}[node distance=1em]
  \event{wx1}{\DW{\aLoc}{1}}{}
  \event{rz1}{\DR{\cLoc}{1}}{right=of wx1}
  \event{wx2}{\DW{\aLoc}{2}}{right=of rz1}
  \event{rz2}{\DR{\cLoc}{2}}{right=of wx2}
  \event{rx1}{\DR{\aLoc}{1}}{right=of rz2}
  \event{rx1a}{\DR{\aLoc}{1}}{below=of wx1}
  \event{wy1}{\DW{\bLoc}{1}}{below=of rx1a}
  \event{rx2a}{\DR{\aLoc}{2}}{below=of wx2}
  \event{wy2}{\DW{\bLoc}{2}}{below=of rx2a}
  \rf{wx1}{rx1a}
  \po{rx1a}{wy1}
  \rf{wx2}{rx2a}
  \po{rx2a}{wy2}
  \po{rz1}{wx2}
  \po{rz2}{rx1}
  \rf[out=20,in=160]{wx1}{rx1}
\end{tikzdisplay}
and the program
\begin{math}
  (
  \cLoc\GETS\bLoc\SEMI
  \cLoc\GETS\bLoc
  )
\end{math}
with execution:
\begin{tikzdisplay}[node distance=1em]
  \event{ry1}{\DR{\bLoc}{1}}{}
  \event{wz1}{\DW{\cLoc}{1}}{right=of ry1}
  \event{ry2}{\DR{\bLoc}{2}}{right=of wz1}
  \event{wz2}{\DW{\cLoc}{2}}{right=of ry2}
  \po{ry1}{wz1}
  \po{ry2}{wz2}
\end{tikzdisplay}
If these are placed in parallel, then a possible execution is:
\begin{tikzdisplay}[node distance=1em]
  \event{wx1}{\DW{\aLoc}{1}}{}
  \event{rz1}{\DR{\cLoc}{1}}{right=of wx1}
  \event{wx2}{\DW{\aLoc}{2}}{right=of rz1}
  \event{rz2}{\DR{\cLoc}{2}}{right=of wx2}
  \event{rx1}{\DR{\aLoc}{1}}{right=of rz2}
  \event{rx1a}{\DR{\aLoc}{1}}{below=of wx1}
  \event{wy1}{\DW{\bLoc}{1}}{below=of rx1a}
  \event{rx2a}{\DR{\aLoc}{2}}{below=of wx2}
  \event{wy2}{\DW{\bLoc}{2}}{below=of rx2a}
  \rf{wx1}{rx1a}
  \po{rx1a}{wy1}
  \rf{wx2}{rx2a}
  \po{rx2a}{wy2}
  \po{rz1}{wx2}
  \po{rz2}{rx1}
  \event{ry1}{\DR{\bLoc}{1}}{below=of wy1}
  \event{wz1}{\DW{\cLoc}{1}}{right=of ry1}
  \event{ry2}{\DR{\bLoc}{2}}{below=of wy2}
  \event{wz2}{\DW{\cLoc}{2}}{right=of ry2}
  \po{ry1}{wz1}
  \po{ry2}{wz2}
  \rf{wy1}{ry1}
  \rf{wz1}{rz1}
  \rf{wy2}{ry2}
  \rf{wz2}{rz2}
\end{tikzdisplay}
and now the $(\DW{\aLoc}{2})$ event is an $\aLoc$-blocker,
so $(\DR{\aLoc}{1})$ cannot
read from $(\DW{\aLoc}{1})$.

In the final definition of reads-from in \S\ref{sec:pomsets} we
ruled out $\aLoc$-blockers by requiring that any
event $\cEv$ that writes to $\aLoc$ has
either $\cEv \gtN \bEv$ or $\aEv \gtN \cEv$.
With this definition, in order for $(\DR{\aLoc}{1})$ to read from
$(\DW{\aLoc}{1})$, we either need $(\DW{\aLoc}{2}) \gtN (\DW{\aLoc}{1})$
or $(\DR{\aLoc}{1}) \gtN (\DW{\aLoc}{2})$, for example:
\begin{tikzdisplay}[node distance=1em]
  \event{wx1}{\DW{\aLoc}{1}}{}
  \event{rz1}{\DR{\cLoc}{1}}{right=of wx1}
  \event{wx2}{\DW{\aLoc}{2}}{right=of rz1}
  \event{rz2}{\DR{\cLoc}{2}}{right=of wx2}
  \event{rx1}{\DR{\aLoc}{1}}{right=of rz2}
  \event{rx1a}{\DR{\aLoc}{1}}{below=of wx1}
  \event{wy1}{\DW{\bLoc}{1}}{below=of rx1a}
  \event{rx2a}{\DR{\aLoc}{2}}{below=of wx2}
  \event{wy2}{\DW{\bLoc}{2}}{below=of rx2a}
  \rf{wx1}{rx1a}
  \po{rx1a}{wy1}
  \rf{wx2}{rx2a}
  \po{rx2a}{wy2}
  \po{rz1}{wx2}
  \po{rz2}{rx1}
  \rf[out=20,in=160]{wx1}{rx1}
  \wk[out=-150,in=-30]{rx1}{wx2}
  \wk{wy1}{wy2}
\end{tikzdisplay}
then putting this in parallel as before results in:
\begin{tikzdisplay}[node distance=1em]
  \event{wx1}{\DW{\aLoc}{1}}{}
  \event{rz1}{\DR{\cLoc}{1}}{right=of wx1}
  \event{wx2}{\DW{\aLoc}{2}}{right=of rz1}
  \event{rz2}{\DR{\cLoc}{2}}{right=of wx2}
  \event{rx1}{\DR{\aLoc}{1}}{right=of rz2}
  \event{rx1a}{\DR{\aLoc}{1}}{below=of wx1}
  \event{wy1}{\DW{\bLoc}{1}}{below=of rx1a}
  \event{rx2a}{\DR{\aLoc}{2}}{below=of wx2}
  \event{wy2}{\DW{\bLoc}{2}}{below=of rx2a}
  \rf{wx1}{rx1a}
  \po{rx1a}{wy1}
  \rf{wx2}{rx2a}
  \po{rx2a}{wy2}
  \po{rz1}{wx2}
  \po{rz2}{rx1}
  \rf[out=20,in=160]{wx1}{rx1}
  \wk[out=-150,in=-30]{rx1}{wx2}
  \wk{wy1}{wy2}
  \event{ry1}{\DR{\bLoc}{1}}{below=of wy1}
  \event{wz1}{\DW{\cLoc}{1}}{right=of ry1}
  \event{ry2}{\DR{\bLoc}{2}}{below=of wy2}
  \event{wz2}{\DW{\cLoc}{2}}{right=of ry2}
  \po{ry1}{wz1}
  \po{ry2}{wz2}
  \rf{wy1}{ry1}
  \rf{wz1}{rz1}
  \rf{wy2}{ry2}
  \rf{wz2}{rz2}
  \wk[out=30,in=150]{wz1}{wz2}
\end{tikzdisplay}
but this is \emph{not} a valid 3-valued pomset,
since $(\DW{\aLoc}{2}) \lt (\DR{\aLoc}{1})$ but also $(\DR{\aLoc}{1}) \gtN (\DW{\aLoc}{2})$,
which is a contradiction.


%\section{Understanding ``Out Of Thin Air'' using Temporal Logic}
\label{sec:logic}

A significant challenge for a software memory model is to relax order enough
to allow efficient implementation without admitting anomalous
behaviors---called \emph{out of thin air} (\oota) in the literature
\cite{vacuous,DBLP:conf/esop/BattyMNPS15,BoehmOOTA}.  The most famous example
is \ref{OOTA3} from \textsection\ref{sec:intro}.  Here we inline
initialization in order to fit the format of our proof rules:
\begin{align*}
  \label{OOTA3}\tag{\textsc{oota2}}
  y\GETS0\SEMI 
  y\GETS x
  \!\PAR\!
  x\GETS0\SEMI
  r\GETS y\SEMI x\GETS r  
  &&
  %\nonumber
  \smash{\hbox{\begin{tikzinline}[node distance=1.2em]
  \event{rx}{\DR{x}{1}}{}
  \event{wy}{\DW{y}{1}}{right=of rx}
  \po{rx}{wy}
  \event{y0}{\DW{y}{0}}{left=of rx}
  \event{x0}{\DW{x}{0}}{right=2em of wy}
  \event{ry}{\DR{y}{1}}{right=2em of x0}
  \event{wx}{\DW{x}{1}}{right=of ry}
  \po{ry}{wx}
  \rf[out=10,in=170]{wy}{ry}
  \rf[out=170,in=10]{wx}{rx}
  \wk[out=-15,in=-165]{y0}{wy}
  \wk[out=-15,in=-165]{x0}{wx}
    \end{tikzinline}}}
\end{align*}
Although Java does not allow \oota{} behaviors of \ref{OOTA3},
\citet{DBLP:journals/toplas/Lochbihler13} showed that it does allow \oota\
behaviors of \ref{OOTA1}, from \textsection\ref{sec:intro}.  In
\cite{DBLP:conf/lics/JeffreyR16}, we described a logic that rules out
\ref{OOTA3} but not \ref{OOTA1}.  In this section, we provide a more accurate
test of \oota{} behaviors by enhancing our previous logic with temporal features.

On first read, we suggest that readers skip to the examples and the
discussion that follows, coming back to the details of the logic as
necessary.  Example~\ref{ex:thin} discusses the canonical \oota{} example
\ref{OOTA3}; the analysis is trivial and well-known
\cite{DBLP:conf/lics/JeffreyR16, DBLP:conf/popl/KangHLVD17}.
Example~\ref{ex:lochb} is more interesting.  It discusses a variant of
\citeauthor{DBLP:journals/toplas/Lochbihler13}'s example \ref{OOTA1},
from the introduction.
% In this case, the violation is a subtler
% temporal property.  We develop a logic sufficient to prove that our semantics
% disallows \oota\ on \eqref{lochbihler}.  

The logic given here is not meant to be definitive; on page
\pageref{page:logic:limits}, we discuss \oota{} examples that require
non-trivial extensions
\cite{DBLP:conf/esop/SvendsenPDLV18,DBLP:journals/pacmpl/ChakrabortyV19}.

\noparagraph{Definitions}
We adapt past linear temporal logic (\pLTL)
\cite{Lichtenstein:1985:GP:648065.747612} to pomsets by dropping the previous
instant operator and adopting strict versions of the temporal operators.
The atoms of our logic are write and read events.
% \begin{displaymath}
%   \afo \QUAD::=\QUAD
%   \DR{\aLoc}{\aVal}
%   \mid
%   \DW\aLoc\aVal
%   \afo \wedge\bfo
%   \mid \lnot \afo
%   \once\afo
%   \mid \always\afo
% \end{displaymath}
%\begin{definition} %[Satisfaction]
Given a pomset $\aPS$ and event $\aEv$, define:\nofootnote{Let $\FALSE$, $\lor$,
  $\Rightarrow$ and $\once$ as usual;
  for example,
  $\once\afo = \lnot(\always\lnot\afo)$.}
\begin{displaymath}
  \renewcommand{\arraycolsep}{.2ex}
    \begin{array}{lrll}
      \aPS,\aEv &\models& \DW{\aLoc}{\aVal} &\text{ if } \labelingAct(\aEv) = \DW{\aLoc}{\aVal} \text{ and } \TRUE \text{ implies } \labelingForm(\aEv) \\
      \aPS,\aEv &\models& \DR{\aLoc}{\aVal} &\text{ if } \labelingAct(\aEv) = \DR{\aLoc}{\aVal} \text{ and } \TRUE \text{ implies } \labelingForm(\aEv) \\
      \aPS,\aEv &\models& \afo\land\bfo &\text{ if } \aPS,\aEv \models  \afo \text{ and } \aPS,\aEv \models  \bfo \\
      \aPS,\aEv &\models& \TRUE\\
      \aPS,\aEv &\models& \lnot\afo &\text{ if } \aPS,\aEv \not\models \afo \\
      \aPS,\aEv &\models& \always\afo &\text{ if } \forall \bEv \lt \aEv.\; \aPS,\bEv \models \afo\\
      \aPS,\aEv &\models& \once\afo &\text{ if } \exists \bEv \lt \aEv.\;  \aPS,\bEv \models \afo 
    \end{array} 
  \end{displaymath}

  Define $\FALSE$, $\lor$, and $\Rightarrow$ as usual.

  % \begin{definition}
  Let $\aPS \models \afo$ if
  $\aPS,\aEv \models\afo$, for all $\aEv \in \Event$.

  Let $\aPSS\models \afo$
  if $\aPS \models\afo$, for all $\aPS \in \aPSS$.
  
Let
  \begin{math}
    \afo, \aPSS \models \bfo  \text{ if } \{ \aPS \mid \aPS \models \afo \} \parallel \aPSS \models \bfo.
  \end{math}
%\end{definition}

  Let $\afo$ be \emph{downclosed} when
  $\{ \aPS \mid \aPS \models \afo \}$ is.

% Thus, $\aPS\models \afo \land \always\afo$ whenever $\aPS \models
  % \afo$. This fact relies on the use of universal quantification in the definition.

% We define other connectives as standard:
% $\once\afo = \lnot(\always\lnot\afo)$,
% %$\FALSE = \lnot(\TRUE)$
% $\afo\lor\bfo = \lnot(\lnot(\afo)\land\lnot(\bfo))$, and
% $\afo\Rightarrow\bfo = \lnot(\afo) \lor\ \bfo$.
% \begin{displaymath}
% \begin{array}{lrl}
% \once\afo &=& \lnot(\always\lnot\afo) \\
% \FALSE &=& \lnot(\TRUE) \\
% \afo\lor\bfo &=& \lnot(\lnot(\afo)\land\lnot(\bfo)) \\
% \afo\Rightarrow\bfo &=& \lnot(\afo) \lor\ \bfo
% \end{array}
% \end{displaymath}
%Let $$ be defined as $$. 
%In addition, let $\FALSE$, $\lor$ and $$ be defined in the
%standard way.
% $\afo\lor\bfo$ for $\lnot(\lnot \afo \land \lnot \bfo)$,
% and $\afo \Rightarrow \bfo$ for $\lnot \afo \lor \bfo$.
  The past operators do not include the current instant, and so
  do \emph{not} satisfy
  $(\always\afo\Rightarrow\once\afo)$. The order-minimal elements always validate
    $\always\afo$ and invalidate
    $\once\afo$.
  However, we can prove the following:
% \begin{align*}  
%   \frac{\aPS \models \afo \Rightarrow\once{\afo}}{\aPS \models \lnot \afo}\text{(Coinduction)}
%   &&
%   \frac{\aPS \models \always\afo \Rightarrow\afo}{\aPS \models \afo}\text{(Induction)}
% \end{align*}
% \begin{lemma}
% Given an pomset $\aPS$.  
\begin{align*}
  \tag{Induction}
  \aPS \models& (\always\afo \Rightarrow\afo) \Rightarrow\afo
  \\[-1ex]
  \tag{Coinduction}
  \aPS \models& (\afo \Rightarrow\once{\afo}) \Rightarrow\lnot \afo
  \\[-1ex]
  \tag{Weakening}
  \aPS \models& (\afo \Rightarrow\once{\bfo}) \Rightarrow (\once\afo \Rightarrow\once{\bfo})
\end{align*}
% \end{lemma}
% \begin{proof}
% We prove that any node in a pomset satisfies these formulas.  
%The proof for both rules proceeds by induction on the length of the maximal path from a root to a node. 
%\end{proof}

% \begin{description}
% \item[Coinduction.]
%   \begin{math}
%     (\afo \Rightarrow\once{\afo}) \Rightarrow\lnot \afo
%   \end{math}
% \item[Induction.] 
%   \begin{math}
%     (\always\afo \Rightarrow\afo) \Rightarrow\afo
%   \end{math}
% \end{description}


%We now present two proof rules for programs. 

%\paragraph*{Proof rules for programs}
We present two additional proof rules. 
The first provides a logical view of \emph{$\aLoc$-closure} (Def.~\ref{def:rf}):
%The soundness proof is straightforward.
% \begin{math}
%   \closed(\aLoc) = (\DR{\aLoc}{\aVal} \Rightarrow \once \DW{\aLoc}{\aVal}).
% \end{math}
% Although this definition does not mention intervening writes, it is
% sufficient for our example.  
\begin{displaymath}
  %\tag{Closing $\aLoc$}
  \frac{
    \afo \text{ is independent of } \aLoc
    \qquad
    %\aPS \models \closed(\aLoc) \Rightarrow \afo
    \aPS \models (\DR{\aLoc}{\aVal} \Rightarrow \once \DW{\aLoc}{\aVal}) \Rightarrow \afo
  }{
    \nu \aLoc \DOT \aPS \models \afo
  }
\end{displaymath}
%It is straightforward to establish that this rule is sound.
% Although it does
% not mention intervening writes, the rule is sufficient for our examples.

The second rule describes concurrent composition, in the style of~\citet{Abadi:1993:CS:151646.151649}.  To simplify the presentation, we
consider the special case with a single invariant.
% We view the
% composition result as capturing key aspects of no-ThinAirRead, as will become
% clearer in the examples below.
% In order to state the theorem, we generalize the satisfaction relation to
% include environment assumptions.

\begin{proposition}%[Composition]
  Let $\afo$ be downclosed.  Let $\aPSS_1, \aPSS_2$ be
  augmentation\hyp{}closed. %\footnote{$\aPS'$ is an augmentation of $\aPS$ if
 %   $\Event'=\Event$, $\aEv\le\bEv$ implies $\aEv\le'\bEv$, $\aEv\gtN\bEv$
 %   implies $\aEv\gtN'\bEv$, and
 %   % $\labeling'(\aEv)=\labeling(\aEv)$
 %   if $\labeling(\aEv) = (\bForm \mid \bAct)$ then
 %   $\labeling'(\aEv) = (\bForm' \mid \bAct)$ where $\bForm'$ implies
 %   $\bForm$.}
  Then:
  \begin{displaymath}
    %\tag{Composition}
    \frac{
      \afo, \aPSS_1 \models\afo
      \qquad
      \afo, \aPSS_2 \models\afo
    }{\aPSS_1 \parallel \aPSS_2 \models \afo}
  \end{displaymath}
\end{proposition}
\begin{proof}[Proof sketch]
  We will show that all downsets in the downset closures of
  $\aPSS_1 \parallel \aPSS_2$ satisfy the required property.  Proof proceeds
  by induction on downsets of $\aPS \in \aPSS_1 \parallel \aPSS_2$.
  %
  The case for empty downset  follows from assumption that  $\afo$ is downset closed.  
  %
  For the inductive case, consider %$\aPS$ in the downset closure of $\aPSS_1 \parallel \aPSS_2$, i.e.
  $\aPS \in \aPS_1 \parallel \aPS_2$ where
  $\aPS_i \in \aPSS_i$.  Since $\aPSS_1$ and $\aPSS_2$ are augmentation
  closed, we can assume that the restriction of $\aPS$ to the events of
  $\aPS_i$ coincides with $\aPS_i$, for $i=1,2$.
  %
  Consider a downset $\aPS'$ derived by removing a maximal element $\aEv$ from
  $\aPS$.  Suppose $\aEv$ comes from $\aPS_1$ (the other case is
  symmetric). Since $\aPS_2$ is a downset of $\aPS'$ and $\aPS' \models \afo$
  by induction hypothesis, we deduce that $\aPS_2 \models \afo$.
  % Thus, $\aPS_2 \in \mods{(\afo)}$.
  Since $\aPS_1 \in \aPSS_1$, by assumption $\afo, \aPSS_1 \models\afo$ we
  deduce that $\aPS \models \afo$.
\end{proof}

% The logic is defined with respect to downclosed sets, but $\sem{\aCmd}$
% includes only completed pomsets.  For reasoning in the logic, we downclose
% the semantics, considering pomsets that may not be completed.  Let
% $\semdown{\aCmd}=\{\aPS'\mid\aPS'$ is a downset of some
% $\aPS \in \sem{\aCmd}\}$.

\begin{example}
\label{ex:thin}
\noparagraph{Basic Examples}
With all variables initialized to $0$, we show that \ref{OOTA3}
satisfies
\begin{math}
  \lnot\DW{x}{1}.
\end{math}

We start with the invariant:
\begin{displaymath}
  [\DW{x}{1}\Rightarrow\once\DR{y}{1}]
  \land
  [\DW{y}{1}\Rightarrow\once\DR{x}{1}]
\end{displaymath}
This invariant holds for each thread; thus, it holds for the
aggregate program by composition.  Closing $y$ yields
\begin{math}
  \DR{y}{1} \Rightarrow \once\DW{y}{1}.
\end{math}
Weakening the right conjunct: % yields
\begin{math}
  \once\DW{y}{1}\Rightarrow\once\DR{x}{1}.
\end{math}
Chaining these together: %yields
\begin{math}
  \DR{y}{1} \Rightarrow \once\DR{x}{1}.
\end{math}
Weakening:  %yields
\begin{math}
  \once\DR{y}{1} \allowbreak\Rightarrow \once\DR{x}{1}. 
\end{math}
Chaining into the left conjunct:  %yields
\begin{math}
  \DW{x}{1} \Rightarrow \once\DR{x}{1}. 
\end{math}
Closing $x$, 
% \begin{math}
%   \DR{x}{1} \Rightarrow \once\DW{x}{1}.
% \end{math}
weakening, 
% \begin{math}
%   \once\DR{x}{1} \Rightarrow \once\DW{x}{1}.
% \end{math}
then chaining: %, yields
\begin{math}
  \DW{x}{1} \Rightarrow \once\DW{x}{1}. 
\end{math}
By coinduction, 
\begin{math}
  \lnot\DW{x}{1}.
\end{math}
%as required.
\end{example}

The same reasoning can be applied to the control flow variant of \ref{OOTA3}
\cite[CYC]{DBLP:conf/popl/VafeiadisBCMN15}:
\begin{math}
  %\tag{\textsc{cyc}}\label{CYC}
  \IF{x}\THEN y\GETS 1 \FI \!\PAR\! \IF{y}\THEN x\GETS 1 \FI.
  % &&
  % %\nonumber
  % \hbox{\begin{tikzinline}[node distance=1.5em]
  % \event{rx}{\DR{x}{1}}{}
  % \event{wy}{\DW{y}{1}}{right=of rx}
  % \po{rx}{wy}
  % \event{ry}{\DR{y}{1}}{right=2em of wy}
  % \event{wx}{\DW{x}{1}}{right=of ry}
  % \po{ry}{wx}
  % \rf{wy}{ry}
  % \rf[out=170,in=10]{wx}{rx}
  %   \end{tikzinline}}
\end{math}
The program is data-race-free. Thus, allowing an execution that writes $1$
would violate \drfsc{}.

\begin{example}
  \label{ex:lochb}
  \noparagraph{Lochbihler's Example} The essential temporal property of
  \ref{OOTA1} is ``allocation at type \texttt{C} is preceded by a read of $1$
  from $z$.''  Because our language lacks object creation, we cannot consider
  \ref{OOTA1} directly.  Instead we study \ref{OOTA4}, which has the same
  temporal structure: ``a write of $1$ to location $a$ is preceded by a read
  of $1$ from $z$.''
% A more general principle, in the spirit of~\citet{Abadi:1993:CS:151646.151649} can be proved.  We chose the simple case of temporal invariants to illustrate the idea in a simple form.  Even this simple version has interesting consequences. 
\begin{gather*}
  \tag{\textsc{oota4}}\label{OOTA4}
  %   Z=1;
  % ||
  %   a=X; // 1
  %   Y=a;
  % ||
  %   b=Z; // 0
  %   if(b){
  %     X=1
  %   } else {
  %     c=Y; // 1
  %     X=c;
  %     W=c;
  %   }
  %     \VAR  x\GETS0\SEMI \VAR  y\GETS0\SEMI \VAR  z\GETS0\SEMI
  \begin{gathered}
  z\GETS0\SEMI
    z\GETS1
  \PAR
  y\GETS0\SEMI
    y\GETS x
  \PAR
  x\GETS0\SEMI
  a\GETS0\SEMI
  \IF{z}\THEN x\GETS1 \ELSE r\GETS y \SEMI x\GETS r \SEMI a\GETS r \FI
  \\
\hbox{\begin{tikzinline}[node distance=1.2em]
  \event{wz0}{\DW{z}{0}}{}
  \event{wz1}{\DW{z}{1}}{right=of wz0}
  \wk{wz0}{wz1}
  \event{wy0}{\DW{y}{0}}{right=2em of wz1}
  \event{rx}{\DR{x}{1}}{right=of wy0}
  \event{wy}{\DW{y}{1}}{right=of rx}
  \po{rx}{wy}
  \event{wx0}{\DW{x}{0}}{right=2em of wy}
  \event{wa0}{\DW{a}{0}}{right=of wx0}
  \event{rz}{\DR{z}{0}}{right=of wa0}
  \event{ry}{\DR{y}{1}}{right=of rz}
  \event{wx}{\DW{x}{1}}{right=of ry}
  \event{wa}{\DW{a}{1}}{right=of wx}
  \po{ry}{wx}
  \rf[out=15,in=165]{wy}{ry}
  \rf[out=-170,in=-10]{wx}{rx}
  \po[out=-18,in=-162]{rz}{wa}
  \po[out=25,in=155]{ry}{wa}
  \wk[out=25,in=155]{wy0}{wy}
  \wk[out=15,in=165]{wx0}{wx}
  \wk[out=15,in=165]{wa0}{wa}
  \rf[out=-10,in=-170]{wz0}{rz}
\end{tikzinline}}
% \hbox{\begin{tikzinline}[node distance=1.5em]
%   \event{rx}{\DR{x}{1}}{}
%   \event{wz1}{\DW{z}{1}}{left=3em of rx}
%   \event{wy}{\DW{y}{1}}{right=of rx}
%   \po{rx}{wy}
%   \event{rz}{\DR{z}{0}}{right=3em of wy}
%   \event{ry}{\DR{y}{1}}{right=of rz}
%   \event{wx}{\DW{x}{1}}{right=of ry}
%   \event{wa}{\DW{a}{1}}{right=of wx}
%   \po{ry}{wx}
%   \rf[out=15,in=165]{wy}{ry}
%   \rf[out=-170,in=-10]{wx}{rx}
%   \po[out=-10,in=-170]{rz}{wa}
%   \po[out=15,in=165]{ry}{wa}
% \end{tikzinline}}
  \end{gathered}  
  \end{gather*}
  Attempting to write $1$ to $a$ results in the cycle shown above.  Thus, the
  outcome is disallowed by our semantics.  It was also disallowed by our
  event structures model, although the logic given there is insufficient to
  establish this fact \citep[\textsection9]{DBLP:journals/lmcs/JeffreyR19}.
  The outcome is \emph{allowed} by
  \citep{DBLP:conf/popl/KangHLVD17,DBLP:conf/esop/JagadeesanPR10,DBLP:journals/pacmpl/ChakrabortyV19}.
% \begin{tikzdisplay}[node distance=1.5em]
%   \event{wy0}{\DW{y}{0}}{}
%   \event{rx}{\DR{x}{1}}{right=4.5em of wy0}
%   \event{wy}{\DW{y}{1}}{right=of rx}
%   \po{rx}{wy}
%   \wk[bend left]{wy0}{wy}
%   \event{wx0}{\DW{x}{0}}{below=of wy0}
%   \event{rz}{\DR{z}{0}}{right=of wx0}
%   \event{ry}{\DR{y}{1}}{right=of rz}
%   \event{wx}{\DW{x}{1}}{right=of ry}
%   \event{ry1}{\DR{y}{1}}{right=of wx}
%   \event{wa}{\DW{a}{1}}{right=of ry1}
%   \rf{wy}{ry1}
%   \po{ry}{wx}
%   \wk[bend right]{wx0}{wx}
%   \rf{wy}{ry}
%   \rf{wx}{rx}
%   \event{wz0}{\DW{z}{0}}{below=of wx0}
%   \event{wz1}{\DW{z}{1}}{right=of wz0}
%   \rf{wz0}{rz}
%   \wk{wz0}{wz1}
%   \po{ry1}{wa}
%   \po[bend right]{rz}{wa}
% \end{tikzdisplay}

To establish that this outcome is disallowed here, we prove 
\begin{math}
  \lnot\DW{a}{1},
\end{math}
starting with invariant:
% which holds for each of the three threads, and thus, by composition, for the
% aggregate program:
\begin{scope}
\small
\begin{align*}
  [\once\DW{y}{1} \Rightarrow \once\DR{x}{1}]
  \land
  [\notonce\DW{a}{1} \Rightarrow (\once\DR{y}{1} \land \always(\DW{x}{1} \Rightarrow \once\DR{y}{1}))]
\end{align*}
\end{scope}
Closing $y$ and chaining into the left conjunct:
% \begin{math}
%   \once\DR{y}{1} \Rightarrow \once\DW{y}{1}. % \Rightarrow \once\DR{x}{1}
% \end{math}
% Chaining this implication on the left:
\begin{math}
  \once\DR{y}{1} \Rightarrow \once\DR{x}{1}.
\end{math}
% We can weaken this to:
% \begin{math}
%   \once\DR{y}{1} \Rightarrow \once\DR{x}{1}. % \Rightarrow \once\DR{x}{1}
% \end{math}
Chaining into the right conjunct:
\begin{displaymath}
  \notonce\DW{a}{1} \Rightarrow (\once\DR{x}{1} \land \always(\DW{x}{1} \Rightarrow \once\DR{x}{1}))
\end{displaymath}
Closing $x$:
% \begin{math}
%   \once\DR{x}{1} \Rightarrow \once\DW{x}{1}.
% \end{math}
%  Weakening and chaining again:
%we can replace $\once\DR{x}{1}$ with $\once\DW{x}{1}$:
\begin{math}
  \notonce\DW{a}{1} \Rightarrow (\once\DW{x}{1} \land \always(\DW{x}{1} \Rightarrow \once\DW{x}{1}).
\end{math}
Applying coinduction to the right conjunct:
\begin{displaymath}
  \notonce\DW{a}{1} \Rightarrow (\once\DW{x}{1} \land \always(\lnot \DW{x}{1}))
\end{displaymath}
Simplifying:
\begin{math}
  \notonce\DW{a}{1} \Rightarrow \FALSE,
\end{math}
as required.
\end{example}


\begin{comment}
  \color{red} Need to sort this out.
  Alan proposes:
\begin{verbatim}
     (W y 2) => <>(R x 1)
     (W y 1) => <>(R x 0)
     (W x 1) => <>(R y 1)
   <>(W x 1) => not(<>(W x 2))  --- which should be???  <>(W x 0) => not(<>(W x 1))
\end{verbatim}

2020/09/30: This seems to go bad because of initialization...
The formula
\begin{verbatim}
<>Wx0 => not(<>Wx1)
\end{verbatim}
does not hold for
\begin{verbatim}
x=0; x=y
\end{verbatim}

2020/09/10:  I am worried about the compositionality of this predicate:
\begin{verbatim}
I think
   <>(W x 0 => not(<>(W x 1)))
holds for 
   x=0; r=y 
and
   x=1
but not
   x=0; r=y || x=1
as shown by the execution
   Wx1 < Wx0 < Ry0
\end{verbatim}
  
It is impossible to fulfill $(\DR{y}{1})$ in the following
\cite[RNG]{DBLP:conf/esop/SvendsenPDLV18}:
\begin{align*}
  \taglabel{OOTA5}
    ( y\GETS x+1
    \PAR
    x\GETS y ) && \hbox{\begin{tikzinline}[node distance=1.5em]
        \event{rx}{\DR{x}{1}}{}
        \event{wy}{\DW{y}{2}}{right=of rx}
        \po{rx}{wy}
        \event{ry}{\DR{y}{1}}{right=3em of wy}
        \event{wx}{\DW{x}{1}}{right=of ry}
        \po{ry}{wx}
        \rf[out=170,in=10]{wx}{rx}
      \end{tikzinline}}
\end{align*}
The proof proceeds as before, starting with the following invariant:
\begin{gather*}
  [\DW{y}{2} \Rightarrow \once\DR{x}{1}] \land
  [\once\DW{x}{1} \Rightarrow \once\DR{y}{1}] \land
  [\once\DW{y}{1} \Rightarrow \once\DR{x}{0}] \land
  [\once\DW{x}{0} \Rightarrow \lnot(\once\DW{x}{1})]
\end{gather*}
\begin{verbatim}
  Wy2 => <>Rx1  /\  <>Wx1 => <>Ry1  /\  <>Wy1 => <>Rx0  
close x and y                                          
  Wy2 => <>Wx1  /\  <>Wx1 => <>Wy1  /\  <>Wy1 => <>Wx0  
chain
  Wy1 => <>Wx0  
chain with <>Wx0 => not(<>Wx1)
\end{verbatim}
\end{comment}

% Many examples are superficially similar, but in fact have fewer dependencies.
% A referee for a previous version of this paper expected that the following example is
% ``the same'':
% \begin{gather*}
%   \tag{OOTA?}\label{OOTA?}
%     y\GETS x
%   \PAR
%     \IF{y}\THEN r\GETS y\SEMI x\GETS r\SEMI a\GETS r \ELSE x\GETS1 \FI
%   \\
%   \hbox{\begin{tikzinline}[node distance=1.5em]
%   \event{rx}{\DR{x}{1}}{}
%   \event{wy}{\DW{y}{1}}{right=of rx}
%   \po{rx}{wy}
%   \event{ry}{\DR{y}{1}}{right=2em of wy}
%   \event{wx}{\DW{x}{1}}{right=of ry}
%   \event{wa}{\DW{a}{1}}{right=of wx}
%   \rf[out=-15,in=-165]{wy}{ry}
%   \rf[out=170,in=10]{wx}{rx}
%   \po[out=-15,in=-165]{ry}{wa}
%     \end{tikzinline}}
% \end{gather*}
% In this execution, $\DW{x}{1}$ is independent of $\DR{y}{1}$, thus there is no
% \oota{} behavior.

Many examples are superficially similar, but in fact have fewer dependencies,
such as \eqref{OOTA?} from \textsection\ref{sec:intro}.

\noparagraph{RFUB: Register assignment From an Unexecuted Branch}
\citeauthor{BoehmOOTA}'s [\citeyear{BoehmOOTA}] \ref{RFUB} example presents
another potential form of \oota{} behavior.
% , in the context of compiler
% optimization.
Our analysis shows that there is no \oota{} behavior in
\ref{RFUB}, only a false dependency:
%\citet{BoehmOOTA} \labeltext{considers}{page:rfub} the following programs:
\begin{gather*}
  \tag{\textsc{rfub}}\label{RFUB}
  \sem{r\GETS y\SEMI x\GETS r}
  \not\supseteq
  \sem{r\GETS y\SEMI \IF{r \NOTEQ 1} \THEN z\GETS 1\SEMI r\GETS 1\FI \SEMI x\GETS r}
\end{gather*}
The left command is half of \ref{OOTA3}. %, from \textsection\ref{sec:logic}.
The right command is dubbed \rfub{}, for \emph{Register assignment From an
  Unexecuted Branch}.  \citeauthor{BoehmOOTA} observes that in the context
$x\GETS y \PAR \hole{}$, these programs have different behaviors.  Yet the
\oota{} example on the left never writes $1$.  Why should the unexecuted
branch change that?  Because of the conditional, the write to $x$ in
\ref{RFUB} is independent of the read from $y$.  It useful to considering the
Hoare logic formulas satisfied by the two threads above: we have
$\hoare{\TRUE}{\text{\rfub}}{x=1}$, but not
$\hoare{\TRUE}{\text{\oota}}{x=1}$.  The change in the thread from
\ref{OOTA3} to \ref{RFUB} is not a valid refinement under Hoare logic; as a
result, it is expected that \ref{RFUB} may have additional behaviors.

Understanding \oota{} behavior is notoriously difficult, even for the
greatest minds in the field!  % We believe that \emph{logic} is the only tool
% that can cut the horrible knot that semanticists have tied themselves in.
% Preconditions provide a \emph{natural} solution to working out these
% dependencies.
This example shows the wisdom of using existing tools, such as preconditions
and Hoare logic, to model new problems, such as relaxed memory.
% We don't
% need to abandon established ideas; we only need to adapt them!
% On page \pageref{page:rfub}, we discuss \citeauthor{BoehmOOTA}'s
% [\citeyear{BoehmOOTA}] \ref{RFUB} example, which presents another potential
% form of \oota{} behavior, in the context of compiler optimization.  Our
% analysis shows that there is no \oota{} behavior in \ref{RFUB}, instead
% \citeauthor{BoehmOOTA}'s analysis has a false dependency.

% Understanding \oota{} behavior is notoriously difficult, even for the
% greatest minds in the field!  We believe that \emph{logic} is the only tool
% that can cut the horrible knot that semanticists have tied themselves in.
% Preconditions provide a \emph{natural} solution to working out these
% dependencies.

% \endinput

% \paragraph{Load buffering and thin air.}
% The program
% \begin{math}
%   %x\GETS0\SEMI y\GETS0\SEMI
%   (y\GETS x \PAR \bReg\GETS y\SEMI x\GETS1)
% \end{math}
% has top level executions that result in the final outcome $x = y = 1$, such as:
% \begin{tikzdisplay}[node distance=1.5em]
%   % \event{wx0}{\DW{x}{0}}{}
%   % \event{wy0}{\DW{y}{0}}{below=wx0}
%   \event{rx}{\DR{x}{1}}{}
%   \event{wy}{\DW{y}{1}}{right=of rx}
%   \po{rx}{wy}
%   \event{ry}{\DR{y}{1}}{right=3em of wy}
%   \event{wx}{\DW{x}{1}}{right=of ry}
%   \rf{wy}{ry}
%   \rf[out=170,in=10]{wx}{rx}
%   %\po{rx}{wy}
% \end{tikzdisplay}
% In \textsection\ref{sec:logic} we provide machinery to prove that this
% outcome is impossible if there is order from read to write in both
% threads.  This order can be achieved by replacing the second thread
% \begin{math}
%   (\bReg\GETS y\SEMI x\GETS1)
% \end{math}
% with 
% \begin{math}
%   (\bReg\GETS y\ACQ\SEMI x\GETS1)
% \end{math}
% or
% \begin{math}
%   (\IF{y}\THEN x\GETS 1\FI)
% \end{math}
% or
% \begin{math}
%   (x\GETS y).
% \end{math}

% A more interesting example is the following variant of \eqref{types}:
% \begin{displaymath}
%   %\label{OOTA4}
%   % x\GETS0\SEMI
%   %y\GETS0\SEMI   
%   (
%     y\GETS x
%   \PAR
%     \IF{z}\THEN x\GETS1 \ELSE x\GETS y\SEMI a\GETS y \FI
%   \PAR
%     z\GETS0\SEMI z\GETS1
%   )
% \end{displaymath}
% This program is allowed to write $1$ to $a$ under many speculative
% memory models
% \cite{Manson:2005:JMM:1047659.1040336,DBLP:conf/esop/JagadeesanPR10,DBLP:conf/popl/KangHLVD17},
% even though the read of $1$ from $y$ in the else branch of the second
% thread arises out of thin air.   \citet{DBLP:journals/toplas/Lochbihler13}
% argues that such executions compromise type safety unless object allocation
% partitions memory by type.
% In our model, the attempted execution is:
% \begin{tikzdisplay}[node distance=1.5em]
%   \event{rx}{\DR{x}{1}}{}
%   \event{wy}{\DW{y}{1}}{below=of rx}
%   \po{rx}{wy}
%   \event{ry}{\DR{y}{1}}{right=of rx}
%   \event{wx}{\DW{x}{1}}{below=of ry}
%   \po{ry}{wx}
%   \rf{wy}{ry}
%   \rf{wx}{rx}
%   \event{rz}{\DR{z}{0}}{right=of ry}
%   \event{wz0}{\DW{z}{0}}{right=of rz}
%   \rf{wz0}{rz}
%   \event{wz1}{\DW{z}{1}}{right=of wz0}
%   \wk{wz0}{wz1}
%   \event{ry1}{\DR{y}{1}}{below=of rz}
%   \rf[bend right]{wy}{ry1}
%   \event{wa}{\DW{a}{1}}{right=of ry1}
%   \po{ry1}{wa}
%   \po{rz}{wa}
% \end{tikzdisplay}
% This is forbidden by the evident cycle.


% \begin{verbatim}



% y=x+1; a=y || x=y
% prove a!=2

% Wyv_1 /\ Wyv_2 => v_1 == v_2 (and maybe v_1==0 \/ v_2==0)
% Wx1 => <>-1 Ry1
% Wy1 => <>-1 Rx1
% \end{verbatim}

% Local Variables:
% mode: latex
% TeX-master: "paper"
% End:



\subsection{Release/acquire synchronization}
\label{app:ra}

% In relaxed memory models, synchronization actions act as memory fences: that
% is, they are a barrier to reordering memory accesses.  In this section, we
% present a simple model of release/acquire fencing. In
% \S\ref{sec:transactions}, we show that this can be scaled up to a model of
% transactional memory.

% We assume there are sets $\Rel$ and $\Acq \subseteq\Act$.  We say that
% $\aAct$ is a \emph{release action} if $\aAct\in\Rel$ and $\aAct$ is an
% \emph{acquire action} if $\aAct\in\Acq$.
% In a pomset, a release event is one labeled with a release action,
% and an acquire event is one labeled by an acquire action.
% To give the semantics of fences, we add extra constraints
% to Definition~\ref{def:prefix} of prefixing %$\aAct\prefix\aPSS$
% (recalling that $\cEv$ is the %$\aAct$-labeled
% event being introduced):
% \begin{itemize}
% \item $\cEv \le \aEv$ whenever $\cEv$ is an acquire event or $\aEv$ is a release event, and
% \item if $\cEv$ is an acquire event then $\aEv$ is independent of $\aLoc$,
%   for every $\aLoc$.
% \end{itemize}
% The first constraint ensures that events are ordered before a release and
% after an acquire.  The second constraint ensures that thread-local reads do
% not cross acquire fences.

We can develop a simple model of release/acquire synchronization using the
following actions: % we will use
% releasing writes and acquiring reads:
\begin{itemize}
\item $(\DWRel{\aLoc}{\aVal})$, a release action that writes $\aVal$ to $\aLoc$, and
\item $(\DRAcq{\aLoc}{\aVal})$, an acquire action that reads $\aVal$ from $\aLoc$.
\end{itemize}
The semantics of programs with releasing write and acquiring read are similar
to regular write and read, with $\DWRel\aLoc\aVal$ replacing
$\DW\aLoc\aVal$ and $\DRAcq\aLoc\aVal$ replacing $\DR\aLoc\aVal$:
\begin{eqnarray*}
  \sem{\aLoc\REL\GETS\aExp\SEMI \aCmd} & = & \textstyle\bigcup_\aVal\; \bigl((\aExp=\aVal) \guard (\DWRel\aLoc\aVal) \prefix \sem{\aCmd}[\aExp/\aLoc]\bigr) \\
  \sem{\aReg\GETS\aLoc\ACQ\SEMI \aCmd} & = & \textstyle\bigcup_\aVal\; (\DRAcq\aLoc\aVal) \prefix \sem{\aCmd}[\aLoc/\aReg]
\end{eqnarray*}

To see the need for the first constraint on prefixing, consider the program:
\[
  \VAR x\GETS0\SEMI \VAR f\GETS0\SEMI
  (x\GETS 1\SEMI f\REL\GETS1 \PAR r\GETS f\ACQ \SEMI s\GETS x)
\]
This has an execution:
\begin{tikzdisplay}[node distance=1em]
  \event{wx0}{\DW{x}{0}}{}
  \event{wf0}{\DW{f}{0}}{right=of wx0}
  \event{wx1}{\DW{x}{1}}{below=of wx0}
  \event{wf1}{\DWRel{f}{1}}{right=of wx1}
  \event{rf1}{\DRAcq{f}{1}}{right=2.5em of wf1}
  \event{rx1}{\DR{x}{1}}{right=of rf1}
  \po{wx0}{wf1}
  \po{wf0}{wf1}
  \po{wx1}{wf1}
  \po{rf1}{rx1}
  \rf{wf1}{rf1}
  \rf[out=20,in=160]{wx1}{rx1}
  \wk{wx0}{wx1}
\end{tikzdisplay}
but \emph{not}:
\begin{tikzdisplay}[node distance=1em]
  \event{wx0}{\DW{x}{0}}{}
  \event{wf0}{\DW{f}{0}}{right=of wx0}
  \event{wx1}{\DW{x}{1}}{below=of wx0}
  \event{wf1}{\DWRel{f}{1}}{right=of wx1}
  \event{rf1}{\DRAcq{f}{1}}{right=2.5em of wf1}
  \event{rx0}{\DR{x}{0}}{right=of rf1}
  \po{wx0}{wf1}
  \po{wf0}{wf1}
  \po{wx1}{wf1}
  \po{rf1}{rx1}
  \rf{wf1}{rf1}
  \rf[out=-20,in=160]{wx0}{rx0}
  \wk{wx0}{wx1}
\end{tikzdisplay}
since $(\DW x0) \gtN (\DW x1) \lt (\DR x0)$, so this pomset does not satisfy the
requirements to be $x$-closed.
If we replace the release
with a plain write, then the outcome $(\DRAcq f1)$ and $(\DR x0)$ is possible:
\begin{tikzdisplay}[node distance=1em]
  \event{wx0}{\DW{x}{0}}{}
  \event{wf0}{\DW{f}{0}}{right=of wx0}
  \event{wx1}{\DW{x}{1}}{below=of wx0}
  \event{wf1}{\DW{f}{1}}{right=of wx1}
  \event{rf1}{\DRAcq{f}{1}}{right=2.5em of wf1}
  \event{rx0}{\DR{x}{0}}{right=of rf1}
  \wk{wf0}{wf1}
  \po{rf1}{rx0}
  \rf{wf1}{rf1}
  \rf[out=-20,in=160]{wx0}{rx0}
  \wk{wx0}{wx1}
\end{tikzdisplay}
since no order is required between $(\DW x1)$ and $(\DW f1)$.  
Symmetrically, if we replace the acquire of the original program
with a plain read, then the outcome $(\DR f1)$ and $(\DR x0)$ is possible.
% \begin{verbatim}
%   x := 0; rel f := 0; ||
%   acq r := f; if (r == 0) { x := x+1; rel f := 1; } ||
%   acq s := f; if (r == 1) { x := x+1; rel f := 2; }
% \end{verbatim}
% This has an execution:
% \begin{tikzdisplay}[node distance=1em]
%   \event{wx0}{\DW{x}{0}}{}
%   \event{wf0}{\DWRel{f}{0}}{below=of wx0}
%   \event{rf0}{\DRAcq{f}{0}}{right=2.5 em of wx0}
%   \event{rx0}{\DR{x}{0}}{below=of rf0}
%   \event{wx1}{\DW{x}{1}}{below=of rx0}
%   \event{wf1}{\DWRel{f}{1}}{below=of wx1}
%   \event{rf1}{\DRAcq{f}{1}}{right=2.5 em of rf0}
%   \event{rx1}{\DR{x}{1}}{below=of rf1}
%   \event{wx2}{\DW{x}{2}}{below=of rx1}
%   \event{wf2}{\DWRel{f}{2}}{below=of wx2}
%   \po{wx0}{wf0}
%   \po{rf0}{rx0}
%   \po{rx0}{wx1}
%   \po{wx1}{wf1}
%   \po{rf1}{rx1}
%   \po{rx1}{wx2}
%   \po{wx2}{wf2}
%   \rf{wf0}{rf0}
%   \rf{wx0}{rx0}
%   \rf{wf1}{rf1}
%   \rf{wx1}{rx1}
% \end{tikzdisplay}
% but \emph{not}:
% \begin{tikzdisplay}[node distance=1em]
%   \event{wx0}{\DW{x}{0}}{}
%   \event{wf0}{\DWRel{f}{0}}{below=of wx0}
%   \event{rf0}{\DRAcq{f}{0}}{right=2.5 em of wx0}
%   \event{rx0}{\DR{x}{0}}{below=of rf0}
%   \event{wx1}{\DW{x}{1}}{below=of rx0}
%   \event{wf1}{\DWRel{f}{1}}{below=of wx1}
%   \event{rf1}{\DRAcq{f}{1}}{right=2.5 em of rf0}
%   \event{rx0b}{\DR{x}{0}}{below=of rf1}
%   \event{wx1b}{\DW{x}{1}}{below=of rx0b}
%   \event{wf2}{\DWRel{f}{2}}{below=of wx1b}
%   \po{wx0}{wf0}
%   \po{rf0}{rx0}
%   \po{rx0}{wx1}
%   \po{wx1}{wf1}
%   \po{rf1}{rx0b}
%   \po{rx0b}{wx1b}
%   \po{wx1b}{wf2}
%   \rf{wf0}{rf0}
%   \rf{wx0}{rx0}
%   \rf{wf1}{rf1}
%   \rf{wx0}{rx0b}
% \end{tikzdisplay}
% since $(\DW x0) \lt (\DW x1) \lt (\DR x0)$, so this pomset does not satisfy the
% requirements to be an rf-pomset.

% The notion rf-pomset is sufficient to capture hardware models and
% release/acquire access in C++, where reads-from implies happens-before
% \cite{alglave}.  To model C++ relaxed access, it
% would be necessary to use a more general notion of rf-pomset, where
% $(\bEv,\aLoc,\aEv) \in \RF$ does not necessarily imply $\bEv \lt \aEv$, instead
% requiring that $(\mathord\lt \cup \mathord\RF)$ be acyclic.

%% To see the need for the second constraint on prefixing, consider the program:
%% \begin{displaymath}
%%   (
%%   x\GETS1\SEMI
%%   \REL f\GETS 1\SEMI
%%   \ACQ r\GETS f\SEMI
%%   y\GETS x
%%   )
%%   \PAR
%%   (
%%   \ACQ s\GETS f\SEMI
%%   x\GETS2\SEMI
%%   \REL f\GETS 2\SEMI
%%   )
%% \end{displaymath}
%% whose semantics includes execution:
%% \begin{displaymath}
%% \begin{tikzpicture}[node distance=1em]
%%   \event{wx1}{\DW{x}{1}}{}
%%   \event{wf1}{\DWRel{f}{1}}{right=of wx1}
%%   \event{rf1}{\DRAcq{f}{2}}{below=of wf1}
%%   \event{wx2}{\DW{x}{2}}{right=of rf1}
%%   \event{wf2}{\DWRel{f}{1}}{right=of wx2}
%%   \event{rf2}{\DRAcq{f}{2}}{above=of wf2}
%%   \event{wy1}{\DW{y}{1}}{right=of rf2}
%%   \po{wx1}{wf1}
%%   \rf{wf1}{rf1}
%%   \po{rf1}{wx2}
%%   \po{wx2}{wf2}
%%   \rf{wf2}{rf2}
%%   \po{rf2}{wy1}
%% \end{tikzpicture}
%% \end{displaymath}
%% This execution exists because
%% \begin{math}
%%   \sem{y\GETS x}
%% \end{math}
%% includes
%% \begin{math}
%%   (x=1\mid \DW{y}{1})
%% \end{math}
%% and the precondition $x=1$ is fulfilled by the preceding write $x\GETS1$.  In
%% implementation term, this execution is reading $1$ from $x$ in a ``stale
%% cache.''  The alternative execution that attempts to read $1$ from the $x$ in
%% ``main memory,'' has an explicit $(\DR{x}{1})$ between $(\DRAcq{f}{2})$ and
%% $(\DW{y}{1})$, and thus will fail to be $x$-closed.

%% To prevent thread-local writes from crossing release/acquire pairs, we
%% require that pomsets in the semantics of acquire have no free locations.
%% This corresponds to the idea that acquires flush the read cache, and
%% therefore reads must reload values from main memory after an acquire.

% In addition, we must change the semantics of write from
% \S\ref{sec:sets-of-pomsets} to ensure that an action is generated for every
% write that might be published by a subsequent release action.
% Formally, $\sem{\aLoc\GETS\aExp\SEMI \aCmd}$ only includes pomsets
% from $\sem{\aCmd}[\aExp/\aLoc]$ that contain a write to
% $\aLoc$ that is not preceded by a release.


\subsection{Relaxed memory}
\label{sec:relaxed-memory}

In \S\ref{sec:info-flow-attack} we presented an information flow attack
on relaxed memory, similar to Spectre in that it relies on speculative
evaluation. Unlike Spectre it does not depend on timing attacks,
but instead is based on the sensitivity of relaxed memory to data
dependencies. % For this reason, we present a simple model of relaxed
% memory, which is strong enough to capture this attack.

Our model includes concurrent memory accesses, which can introduce concurrent
reads-from. 
Since we are allowing events to be partially ordered, this gives a simple
model of relaxed memory.  For example an independent read independent write
(IRIW) example is:
\[\begin{array}{l}
  x\GETS0\SEMI x\GETS x+1
  \PAR
  y\GETS0\SEMI y\GETS y+1
\\{}
  \PAR
  r_1\GETS x\SEMI r_2\GETS y
  \PAR
  s_1\GETS y\SEMI s_2\GETS x
\end{array}\]
which includes the execution:
\begin{tikzdisplay}[node distance=1em]
  \event{wx0}{\DW{x}{0}}{}
  \event{wx1}{\DW{x}{1}}{right=of wx0}
  \event{wy0}{\DW{y}{0}}{right=2.5em of wx1}
  \event{wy1}{\DW{y}{1}}{right=of wy0}
  \event{ry1}{\DR{y}{1}}{below=4ex of wx0}
  \event{rx0}{\DR{x}{0}}{right=of ry1}
  \event{rx1}{\DR{x}{1}}{right=2.5 em of rx0}
  \event{ry0}{\DR{y}{0}}{right=of rx1}
  \rf{wx1}{rx1}
  \rf{wy0}{ry0}
  \rf[out=210,in=30]{wy1}{ry1}
  \rf{wx0}{rx0}
  \wk{rx0}{wx1}
  \wk{ry0}{wy1}
\end{tikzdisplay}
This model does not introduce thin-air reads (TAR).
For example the TAR pit
\((
  x\GETS y \PAR y \GETS x
)\)
fails to produce a value for $x$ from thin air
since this produces a cycle in $\le$, as shown on the left below:
\begin{align*}
\begin{tikzpicture}[node distance=1em]
  \event{ry42}{\DR{y}{42}}{}
  \event{wx42}{\DW{x}{42}}{below=of ry42}
  \event{rx42}{\DR{x}{42}}{right=2.5em of ry42}
  \event{wy42}{\DW{y}{42}}{below=of rx42}
  \po{ry42}{wx42}
  \po{rx42}{wy42}
  \rf{wx42}{rx42}
  \rf{wy42}{ry42}
\end{tikzpicture}
&&
\begin{tikzpicture}[node distance=1em]
  \event{ry1}{\DR{y}{1}}{}
  \event{wx1}{\DW{x}{1}}{below=of ry1}
  \event{rx1}{\DR{x}{1}}{right=2.5em of ry1}
  \event{wy1}{\DW{y}{1}}{below=of rx1}
  \po{ry1}{wx1}
  \rf{wx1}{rx1}
  \rf{wy1}{ry1}
\end{tikzpicture}
\end{align*}
This cycle can be broken by removing a dependency. For example
\((
  x\GETS y \PAR r\GETS x\SEMI y \GETS r+1-r
)\)
has the execution on the right above.
% \begin{tikzdisplay}[node distance=1em]
%   \event{ry1}{\DR{y}{1}}{}
%   \event{wx1}{\DW{x}{1}}{below=of ry1}
%   \event{rx1}{\DR{x}{1}}{right=2.5em of ry1}
%   \event{wy1}{\DW{y}{1}}{below=of rx1}
%   \po{ry1}{wx1}
%   \rf{wx1}{rx1}
%   \rf{wy1}{ry1}
% \end{tikzdisplay}
Note that $(\DR x1) \not\le (\DW y1)$, so this does not introduce a cycle.

Although it is not the primary focus of this paper, our model may be an
attractive model of relaxed memory.  Many prior models either permit
thin-air executions that our model forbids or forbid desirable executions
that our model permits.
%% In \S\ref{sec:logic}, we develop a logic which allows us to prove that our
%% semantics forbids thin air examples that are permitted by prior speculative
%% models
%% \cite{Manson:2005:JMM:1047659.1040336,Jagadeesan:2010:GOS:2175486.2175503,DBLP:conf/popl/KangHLVD17}.
% Our model passes all of the causality test cases
% \cite{PughWebsite}.
%% Significantly, this
%% includes test case 9, which is forbidden by \cite{DBLP:conf/lics/JeffreyR16},
%% one of the few models that disallows the thin air example from
%% \S\ref{sec:logic}.  We present this test case in the appendix, where we also
%% discuss the thread inlining examples from
%% \cite{Manson:2005:JMM:1047659.1040336}.

% In \refapp{logic}, we present a variant of the TAR-pit
% example %from \S\ref{sec:relaxed-memory}
% that is allowed under prior speculative semantics
% \cite{Manson:2005:JMM:1047659.1040336,Jagadeesan:2010:GOS:2175486.2175503,DBLP:conf/popl/KangHLVD17}.
% We develop a logic that allows us to prove that the problematic execution is
% forbidden in our model.  \citet{DBLP:conf/esop/BattyMNPS15} showed that the
% thin-air problem has no per-candidate-execution solution for C++.  This
% result does not apply to our model, which has a different notion of
% dependency.

% as the semantics of a conditional can depend on the semantics
% of both branches.

\citet{PughWebsite} developed a set of twenty {causality test cases} in the
process of revising the Java Memory Model (JMM)
\cite{Manson:2005:JMM:1047659.1040336}.  Using hand calculation, we have
confirmed that our model gives the desired result for all twenty cases,
unrolling loops as necessary.  Our model also gives the desired results for
all of the examples in \citet[\textsection 4]{DBLP:conf/esop/BattyMNPS15} and
all but one in \citet[\textsection 5.3]{SevcikThesis}:
redundant-write-after-read-elimination fails for any
sensible non-coherent semantics.  Our model agrees with the JMM on the
``surprising and controversial behaviors'' of \citet[\textsection
8]{Manson:2005:JMM:1047659.1040336}, and thus fails to validate thread
inlining.
% In \refapp{tc}, we discuss three of the causality test cases and the thread
% inlining example from \cite{Manson:2005:JMM:1047659.1040336}.%  In presenting the
% examples, we unroll loops, correct typos and simplify the code.  

% \subsection{Causality test cases}
% \label{app:tc}

% \citet{PughWebsite} developed a set of twenty {causality test cases} in the
% process of revising the Java Memory Model (JMM)
% \cite{Manson:2005:JMM:1047659.1040336}.  Using hand calculation, we have
% confirmed that our model gives the desired result for all twenty cases,
% unrolling loops as necessary.  Our model also gives the desired results for
% all of the examples in \citet[\textsection 4]{DBLP:conf/esop/BattyMNPS15} and
% all but one in \citet[\textsection 5.3]{SevcikThesis}:
% redundant-write-after-read-elimination fails for any
% sensible non-coherent semantics.  Our model agrees with the JMM on the
% ``surprising and controversial behaviors'' of \citet[\textsection
% 8]{Manson:2005:JMM:1047659.1040336}, and thus fails to validate thread
% inlining.

\section{UNUSED: appendix stuff}

%In this section,
We now discuss three of the causality test cases, as well as the thread
inlining from \cite{Manson:2005:JMM:1047659.1040336}.  In presenting the
examples, we unroll loops, correct typos and simplify the code.  

\subsubsection{Causality test case 8}

Test case 8 asks whether:
\begin{displaymath}
  \VAR x\GETS 0\SEMI
  \VAR y\GETS 0\SEMI
  (\IF{x<2}\THEN y\GETS 1\FI 
  \PAR
  x\GETS y)
\end{displaymath}
may read $1$ for both $x$ and $y$.  This behavior is allowed, since
``interthread analysis could determine that $x$ and $y$ are always either $0$
or $1$.''  This breaks the dependency between the read of $x$ and the write
to $y$ in the first thread, allowing the write to be moved earlier.

The semantics of TC8 includes
\begin{tikzdisplay}[node distance=1em]
  \event{ix}{\DW{x}{0}}{}
  \event{iy}{\DW{y}{0}}{right=of ix}
  \event{rx1}{\DR{x}{1}}{right=2.1em of iy}
  \event{wy1}{\DW{y}{1}}{right=of rx1}
  \event{ry1}{\DR{y}{1}}{right=2.1em of wy1}
  \event{wx1}{\DW{x}{1}}{right=of ry1}
  \po{ry1}{wx1}
  \po[out=30,in=150]{ix}{rx1}
  \rf[in=-25,out=-160]{wx1}{rx1}
  \rf[out=20,in=160]{wy1}{ry1}
  \wk[out=-25,in=-150]{ix}{wx1}
  \wk[out=25,in=155]{iy}{wy1}
\end{tikzdisplay}
Where we require $(\DW{x}{0})\lt(\DR{x}{1})$ but not $(\DR{x}{1})\lt(\DW{y}{1})$.
To see why this execution exists, consider the left thread with syntax sugar
removed:
\begin{displaymath}
  r\GETS x\SEMI \IF{r<2}\THEN y\GETS 1\FI
\end{displaymath}
\begin{math}
  \sem{\IF{r<2}\THEN y\GETS 1\FI}
\end{math}
includes
\begin{math}
  (r<2\mid\DW{y}{1}).
\end{math}
% \begin{tikzdisplay}[node distance=1em]
%   \event{wy1}{r<2\mid\DW{y}{1}}{}
% \end{tikzdisplay}
Thus, by Figure~\ref{fig:programs}, 
\begin{math}
  \sem{r\GETS x\SEMI \IF{r<2}\THEN y\GETS 1\FI}
\end{math}
includes
\begin{math}
  (\DR{x}{1}) \prefix (r<2\mid\DW{y}{1})[x/r]
\end{math}
which simplifies to
\begin{math}
  (\DR{x}{1}) \prefix (x<2\mid\DW{y}{1}),
\end{math}
which, by Definition~\ref{def:prefix}, includes:
\begin{tikzdisplay}[node distance=1em,baselinecenter]
    \event{rx1}{\DR{x}{1}}{}
    \event{wy1}{x<2\mid\DW{y}{1}}{right=of rx1}
  \end{tikzdisplay}
Here we have used the \textsc{[non-ordering read]} clause of Definition~\ref{def:prefix}:
``$\bForm'$ implies $\bForm[\aVal/\aLoc] \land \bForm$, if $\aAct$ reads $\aVal$ from $\aLoc$,''
where $a=(\DR{x}{1})$,  $\bForm=\bForm'=(x<2)$.  We can use this case since
$x<2$ implies $1<2\land x<2$.

Prefixing with $(\DW{x}{0})$ allows us to discharge the assumption $x<2$,
arriving at:
\begin{tikzdisplay}[node distance=1em,baselinecenter]
    \event{ix}{\DW{x}{0}}{}
    \event{rx1}{\DR{x}{1}}{right=2.5 em of ix}
    \event{wy1}{\DW{y}{1}}{right=of rx1}
    \po{ix}{rx1}
  \end{tikzdisplay}
Here we have used the \textsc{[ordering read]}
clause of \ref{def:prefix}:
``$\bForm'$ implies $\bForm[\aVal/\aLoc]$, if $\aAct$ reads $\aVal$ from $\aLoc$ and $\cEv\lt'\aEv$,''
where $a=(\DW{x}{0})$,  $\bForm=(x<2)$ and $\bForm'=\TRUE$.  As long as
require
\begin{math}
  (\DW{x}{0})\lt
  (\DR{x}{1}),
\end{math}
we can use this case since $\TRUE$ implies $0<2$.

\subsubsection{Causality test case 9}

Test case 9 asks whether:
\begin{displaymath}
  \VAR x\GETS 0\SEMI
  \VAR y\GETS 0\SEMI
  (\IF{x<2}\THEN y\GETS 1\FI 
  \PAR
  x\GETS y
  \PAR
  y\GETS 2\SEMI)
\end{displaymath}
may read $1$ for both $x$ and $y$.  This behavior is also allowed.  This is
``similar to test case $8$, except that $x$ is not always $0$ or
$1$. However, a compiler might determine that the read of $x$ by thread $1$
will never see the write by thread $3$ (perhaps because thread $3$ will be
scheduled after thread $1$)''

Reasoning as for test case 8, the semantics of test case 9 includes:
\begin{tikzdisplay}[node distance=1em]
  \event{ix}{\DW{x}{0}}{}
  \event{iy}{\DW{y}{0}}{right=of ix}
  \event{rx1}{\DR{x}{1}}{right=2.2 em of iy}
  \event{wy1}{\DW{y}{1}}{right=of rx1}
  \event{ry1}{\DR{y}{1}}{right=2.2em of wy1}
  \event{wx1}{\DW{x}{1}}{right=of ry1}
  \event{wx2}{\DW{x}{2}}{below=3ex of $(ix)!0.5!(iy)$}%{right=2.5em of wx1}
  \po{ry1}{wx1}
  \po[out=30,in=150]{ix}{rx1}
  \rf[in=-25,out=-160]{wx1}{rx1}
  \rf[out=20,in=160]{wy1}{ry1}
  \wk[out=-25,in=-150]{ix}{wx1}
  \wk[out=25,in=155]{iy}{wy1}
  \wk{ix}{wx2}
  % \wk[out=-25,in=-150]{ix}{wx2}
\end{tikzdisplay}

Thus, with respect to the introduction of new threads, our model appears to
be more robust than the event structures semantics of
\cite{DBLP:conf/lics/JeffreyR16}, which fails on this test case.

\subsubsection{Causality test case 14}

Test case 14 asks whether:
\begin{multline*}
  \VAR a\GETS 0\SEMI
  \VAR b\GETS 0\SEMI
  \VAR y\GETS 0\SEMI\\[-.5ex]
  (\IF{a}\THEN b\GETS 1\ELSE y\GETS 1\FI 
  \PAR\\[-.5ex]
  \WHILE(y+b==0) \THEN\SKIP\FI\; a\GETS1)
\end{multline*}
may read $1$ for $a$ and $b$, yet $0$ for $y$.  Here $a$ and $b$ are regular
variables and $y$ is volatile, which is equivalent to release/acquire in this
example.  This behavior is also disallowed, since ``in all sequentially
consistent executions, [the read of $a$ gets $0$] and the program is
correctly synchronized. Since the program is correctly synchronized in all SC
executions, no non-SC behaviors are allowed.''

Unrolling the loop once, we have:
\begin{multline*}
  \VAR a\GETS 0\SEMI
  \VAR b\GETS 0\SEMI
  \VAR y\GETS 0\SEMI\\[-.5ex]
  (\IF{a}\THEN b\GETS 1\ELSE y\GETS 1\FI 
  \PAR\\[-.5ex]
  \IF{y\lor b}\THEN a\GETS 1\FI)
\end{multline*}
We argue that any execution with $(\DR{a}{1})$, $(\DR{b}{1})$, and
$(\DR{y}{0})$ must be cyclic.  The closure requirements require that
\begin{math}
  (\DW{a}{1})\lt(\DR{a}{1})
  \;\text{and}\;
  (\DR{b}{1})\lt(\DR{b}{1}).
\end{math}
Ignoring initialization, least ordered execution that includes all of these
actions is:
\begin{tikzdisplay}[node distance=1em]
  \event{ra1}{\DR{a}{1}}{}
  \event{wb1}{\DW{b}{1}}{below=of ra1}
  \nonevent{wy1}{\DW{y}{1}}{left=of wb1}
  \event{rb1}{\DR{b}{1}}{right=4.5em of ra1}
  \event{ry0}{\DR{y}{0}}{right=of rb1}
  \event{wa1}{\DW{a}{1}}{below=of rb1}
  \po{ra1}{wb1}
  \po{rb1}{wa1}
  \rf{wa1}{ra1}
  \rf{wb1}{rb1}
\end{tikzdisplay}
where the read of $a$ is ordering for $(\DW{b}{1})$ but
not $(\DW{y}{1})$, and the read of $b$ is ordering for $(\DW{a}{1})$ but the
read of $y$ is not.  $(\DW{y}{1})$ is crossed out, since its
precondition must imply $(\lnot a)[1/a]$, which is equivalent to $\FALSE$.
To avoid order from $(\DR{y}{0})$ to $(\DW{a}{1})$, we
have strengthened the predicate on $(\DW{a}{1})$ from $(y\lor b)$ to
$(y=0\land b=1)$.  Note that we cannot use this trick symmetrically to remove
the order from $(\DR{b}{1})$ to $(\DW{a}{1})$, since $b=1$ does not follow
from the initialization of $b$.


\subsubsection{Thread inlining}

One property one could ask of a model of shared memory is thread
inlining: any execution of $\sem{P\SEMI Q}$ is an execution of $\sem{P
  \PAR Q}$. This is \emph{not} a goal of our model, and indeed is not
satisfied, due to the different semantics of concurrent and sequential
memory accesses. We demonstrate this by considering an example from
the Java Memory Model~\cite{Manson:2005:JMM:1047659.1040336}, which shows that Java does not
satisfy thread inlining either.

The lack of thread inlining is related to the different dependency
relations introduced by sequential and concurrent access.
Recall from \S\ref{sec:sequential-memory} that the program
\verb`(x := 0; y := x+1;)` has execution:
\begin{tikzdisplay}[node distance=1em]
  \event{wx0}{\DW{x}{0}}{}
  \event{wy1}{\DW{y}{1}}{right=of wx0}
\end{tikzdisplay}
but that \verb`(x := 1; || y := x+1;)` has:
\begin{tikzdisplay}[node distance=1em]
  \event{wx1}{\DW{x}{1}}{}
  \event{rx1}{\DR{x}{1}}{right=2.5em of wx1}
  \event{wy2}{\DW{y}{2}}{right=of rx1}
  \rf{wx1}{rx1}
  \po{rx1}{wy2}
\end{tikzdisplay}
That is, in the sequential case there is no dependency from the
write of $x$ to the write of $y$, but in the concurrent case there
is such a dependency.

This can be used to construct a counter-example to thread inlining, based on~\cite[Ex~11]{Manson:2005:JMM:1047659.1040336}:
\begin{verbatim}
  x := 0; if (x == 1) { z := 1; } else { x := 1; } || y := x; || x := y;
\end{verbatim}
This has no execution containing $(\DW z1)$. Any attempt to build such an execution
results in a cycle:
\begin{tikzdisplay}[node distance=1em]
  \event{rx1a}{\DR{x}{1}}{}
  \event{wz1}{\DW{z}{1}}{right=of rx1a}
  \nonevent{wx1a}{\DW{x}{1}}{right=of wz1}
  \event{rx1b}{\DR{x}{1}}{below=of wx1a}%{right=2.5em of wx1a}
  \event{wy1}{\DW{y}{1}}{right=of rx1b}
  \event{ry1}{\DR{y}{1}}{right=2.5em of wy1}
  \event{wx1b}{\DW{x}{1}}{right=of ry1}
  \po{rx1a}{wz1}
  \po[out=25, in=150]{rx1a}{wx1a}
  \po{rx1b}{wy1}
  \po{ry1}{wx1b}
  \rf{wy1}{ry1}
  \rf[out=160, in=30]{wx1b}{rx1a}
  \rf[out=160, in=30]{wx1b}{rx1b}
\end{tikzdisplay}
Inlining the thread \verb|(y := x)| gives~\cite[Ex~12]{Manson:2005:JMM:1047659.1040336}:
\begin{verbatim}
  x := 0; if (x == 1) { z := 1; } else { x := 1; } y := x; || x := y;
\end{verbatim}
with execution:
\begin{tikzdisplay}[node distance=1em]
  \event{rx1a}{\DR{x}{1}}{}
  \event{wz1}{\DW{z}{1}}{right=of rx1a}
  \nonevent{wx1a}{\DW{x}{1}}{right=of wz1}
  \event{wy1}{\DW{y}{1}}{right=of wx1a}
  \event{ry1}{\DR{y}{1}}{right=2.5em of wy1}
  \event{wx1b}{\DW{x}{1}}{right=of ry1}
  \po{rx1a}{wz1}
  \po[out=25, in=150]{rx1a}{wx1a}
  \po{ry1}{wx1b}
  \rf{wy1}{ry1}
  \rf[out=160, in=30]{wx1b}{rx1a}
\end{tikzdisplay}
To see why this execution exists, consider the program fragment:
\begin{verbatim}
  if (x == 1) { z := 1; } else { x := 1; } y := x;
\end{verbatim}
Removing the syntax sugar, this is:
\begin{verbatim}
  r1 := x; if (r1 == 1) {
    z := 1; r2 := x; y := r2; skip
  } else {
    x := 1; r3 := x; y := r3; skip
  }
\end{verbatim}
Now, $\sem{z := 1\SEMI r_2 := x\SEMI y := r_2\SEMI \SKIP}$
includes pomset:
\begin{tikzdisplay}[node distance=1em]
  \event{wz1}{r_1=1 \mid \DW{z}{1}}{}
  \event{wy1}{r_1=x=1 \mid \DW{y}{1}}{right=of wz1}
\end{tikzdisplay}
and $\sem{x := 1\SEMI r_3 := x\SEMI y := r_3\SEMI \SKIP}$
includes pomset:
\begin{tikzdisplay}[node distance=1em]
  \event{wx1a}{r_1\neq 1 \mid \DW{x}{1}}{}
  \event{wy1}{r_1\neq 1 \mid \DW{y}{1}}{right=of wx1a}
\end{tikzdisplay}
so  $\sem{\IF{r_1 = 1} \THEN z := 1\SEMI r_2 := x\SEMI y := r_2\SEMI \SKIP \ELSE x := 1\SEMI r_3 := x\SEMI y := r_3\SEMI \SKIP \FI}$ includes:
\begin{tikzdisplay}[node distance=1em]
  \event{wz1}{r_1=1 \mid \DW{z}{1}}{}
  \event{wx1a}{r_1\neq1 \mid \DW{x}{1}}{right=of wz1}
  \event{wy1}{(r_1=x=1) \lor (r_1\neq1) \mid \DW{y}{1}}{below=3ex of $(wz1)!0.5!(wx1a)$}
\end{tikzdisplay}
which means $\sem{\IF{r_1 = 1} \THEN z := 1\SEMI r_2 := x\SEMI y := r_2\SEMI \SKIP \ELSE x := 1\SEMI r_3 := x\SEMI y := r_3\SEMI \SKIP \FI}[x/r_1]$ includes:
\begin{tikzdisplay}[node distance=1em]
  \event{wz1}{x=1 \mid \DW{z}{1}}{}
  \event{wx1a}{x\neq1 \mid \DW{x}{1}}{right=of wz1}
  \event{wy1}{(x=x=1) \lor (x\neq1)) \mid \DW{y}{1}}{below=3ex of $(wz1)!0.5!(wx1a)$}%{right=of wx1a}
\end{tikzdisplay}
Now $(x=x=1) \lor (x\neq1)$ is a tautology, so this is just:
\begin{tikzdisplay}[node distance=1em]
  \event{wz1}{x=1 \mid \DW{z}{1}}{}
  \event{wx1a}{x\neq1 \mid \DW{x}{1}}{right=of wz1}
  \event{wy1}{\DW{y}{1}}{right=of wx1a}
\end{tikzdisplay}
and so $\sem{r_1 \GETS x\SEMI \IF{r_1 = 1} \THEN z := 1\SEMI r_2 := x\SEMI y := r_2\SEMI \SKIP \ELSE x := 1\SEMI r_3 := x\SEMI y := r_3\SEMI \SKIP \FI}$ includes:
\begin{tikzdisplay}[node distance=1em]
  \event{rx1a}{\DR{x}{1}}{}
  \event{wz1}{1=1 \mid \DW{z}{1}}{right=of rx1a}
  \event{wx1a}{1\neq1 \mid \DW{x}{1}}{right=of wz1}
  \event{wy1}{\DW{y}{1}}{right=of wx1a}
  \po{rx1a}{wz1}
  \po[out=25, in=150]{rx1a}{wx1a}
\end{tikzdisplay}
which simplifies to:
\begin{tikzdisplay}[node distance=1em]
  \event{rx1a}{\DR{x}{1}}{}
  \event{wz1}{\DW{z}{1}}{right=of rx1a}
  \nonevent{wx1a}{\DW{x}{1}}{right=of wz1}
  \event{wy1}{\DW{y}{1}}{right=of wx1a}
  \po{rx1a}{wz1}
  \po[out=25, in=150]{rx1a}{wx1a}
\end{tikzdisplay}
as required. The rest of the example is straightforward, and shows that our semantics
agrees with the JMM in not supporting thread inlining.



% \subsection{Word tearing}

% \todo{Remove this section, since it's not needed for transactions?}

% In \S\ref{sec:transactions}, we shall be considering transactional memory,
% and in \S\ref{sec:transactions} show that we can model a simplified version
% of an information flow attack on transactions. In order to model transactions,
% we need to consider actions that can write many memory locations at once,
% since this is part of the semantics of commitment. To lead up to this, we first
% consider a simpler scenario of many-location writes and reads, which is word
% tearing.

% In word tearing, a program contains a write instruction with data larger
% than the hardware word size, for example copying a byte array, or assigning
% a 64-bit float on a 32-bit architecture. For example, consider the program:
% \begin{verbatim}
%   (x := [0, 0];) || (x := [1, 1];) || (r := x;)
% \end{verbatim}
% This has executions in which the read of $x$ only reads from one of the writes,
% for example:
% \begin{tikzdisplay}[node distance=1em]
%   \event{wx00}{\DW{x}{[0,0]}}{}
%   \event{wx11}{\DW{x}{[1,1]}}{right=2.5em of wx00}
%   \event{rx00}{\DR{x}{[0,0]}}{right=2.5em of wx11}
%   \rf[out=20, in=160]{wx00}{rx00}
% \end{tikzdisplay}
% but also has executions in which the read of $x$ reads from both writes,
% for example:
% \begin{tikzdisplay}[node distance=1em]
%   \event{wx00}{\DW{x}{[0,0]}}{}
%   \event{wx11}{\DW{x}{[1,1]}}{right=2.5em of wx00}
%   \event{rx01}{\DR{x}{[0,1]}}{right=2.5em of wx11}
%   \rfx[out=20, in=160]{wx00}{x[0]}{rx01}
%   \rfx[out=-20, in=-160]{wx11}{x[1]}{rx01}
% \end{tikzdisplay}
% Word tearing can occur, for example, in Java extended floating point~\cite{jmm},
% LLVM 64-bit instructions on 32-bit hardware~\cite{llvm}, or in
% JavaScript SharedArrayBuffers~\cite{js-sab}.

% \newcommand{\rfControl}[4][]{\draw[rf,#1](#2) .. controls (#3) .. (#4);}
% \begin{tikzdisplay}[node distance=1em]
%   \event{wx0}{\DW{x}{0}}{}
%   \event{wx1}{\DW{x}{1}}{right=of wx0}
%   \event{wy0}{\DW{y}{0}}{right=2.5em of wx1}
%   \event{wy1}{\DW{y}{1}}{right=of wy0}
%   \event{rx1}{\DR{x}{1}}{right=2.5 em of wy1}
%   \event{ry0}{\DR{y}{0}}{right=of rx1}
%   \event{ry1}{\DR{y}{1}}{right=2.5 em of ry0}
%   \event{rx0}{\DR{x}{0}}{right=of ry1}
%   \rf[out=20,in=160]{wx1}{rx1}
%   \rf[out=20,in=160]{wy0}{ry0}
%   \rf[out=340,in=200]{wy1}{ry1}
%   \coordinate (a) [below=of wy1];
%   \rfControl[out=340,in=200]{wx0}{a}{rx0}
%   \wk{wx0}{wx1}
%   \wk{wy0}{wy1}
%   \po{rx1}{ry0}
%   \po{ry1}{rx0}
% \end{tikzdisplay}


% Batty section 4:
% \cite[\S4]{DBLP:conf/esop/BattyMNPS15},
% Example LB+ctrldata+ctrl-double (language must allow)
% r1=loadrlx(x) //reads 42
% if (r1 == 42)
%   storerlx(y,r1)

% r2=loadrlx(y) //reads 42
% if (r2 == 42)
%   storerlx (x,42)
% else
% storerlx (x,42)

% a:RRLX x=42 sb,dd,cd
% c:RRLX y=42 sb,cd
%   This is forbidden on hardware if compiled naively, as the architectures respect read-to-write control dependencies, but in practice compilers will collapse conditionals like that of the second thread, removing the control dependencies from the read of y to the writes of x and making the code identical to the previous example. As that example is allowed and observable on hardware (and we pre- sume that it would be impractical to outlaw such optimization for C or C++), the language must also allow this execution. But this execution has a cycle in the union of reads-from and dependency, so we cannot simply exclude all those.
% Then one might hope for some other adaptation of the C/C++11 model, but the following example shows at least that there is no per-candidate-execution solution.
% Example LB+ctrldata+ctrl-single (language can and should forbid)
% r1=loadrlx(x) //reads 42 if (r1 == 42)
% storerlx (y,r1) r2=loadrlx (y) //reads 42 if (r2 == 42)
% a:RRLX x=42 sb,dd,cd
% rf
% b:WRLX y=42
% c:RRLX y=42 sb,cd
% rf
% d:WRLX x=42
% rf rf
% b:WRLX y=42 d:WRLX x=42
%   storerlx (x,42)

\subsection{UNUSED: Logic}
\label{app:logic}

\newcommand{\closed}{\textsf{closed}}
\newcommand{\pLTL}{\textsf{PLTL}}
\newcommand{\once}{\Diamond^{-1}}
\newcommand{\always}{\Box^{-1}}
\newcommand{\afo}{\phi}
\newcommand{\bfo}{\psi}
\newcommand{\mods}{\textsf{Models}}

In this section, we develop sufficient logical infrastructure to prove that
our semantics disallows thin air executions.  We present a variant of the
TAR-pit example from \S\ref{sec:relaxed-memory} which poses difficulties
under many speculative semantics.

We adapt past linear temporal logic (\pLTL)
\cite{Lichtenstein:1985:GP:648065.747612} to pomsets by dropping the previous
instant operator and adopting strict versions of the temporal operators.
The atoms of our logic are write and read events.
% \begin{displaymath}
%   \afo \QUAD::=\QUAD
%   \DR{\aLoc}{\aVal}
%   \mid
%   \DW\aLoc\aVal
%   \afo \wedge\bfo
%   \mid \lnot \afo
%   \once\afo
%   \mid \always\afo
% \end{displaymath}
%\begin{definition} %[Satisfaction]
  Given an pomset $\aPS$ and event $\aEv$, define:
  \begin{displaymath}
    \begin{array}{lrl}
      \aPS,\aEv &\models& \DW{\aLoc}{\aVal}, \text{ if } \labeling(\aEv) =  (\TRUE, \DW{\aLoc}{\aVal}) \\
      \aPS,\aEv &\models& \DR{\aLoc}{\aVal}, \text{ if } \labeling(\aEv) =  (\TRUE, \DR{\aLoc}{\aVal}) \\
      \aPS,\aEv &\models& \afo\land\bfo, \text{ if } \aPS,\aEv \models  \afo \text{ and } \aPS,\aEv \models  \bfo \\
      \aPS,\aEv &\models& \TRUE\\
      \aPS,\aEv &\models& \lnot\afo, \text{ if } \aPS,\aEv \not\models \afo \\
      %\aPS,\aEv &\models& \once\afo, \text{if } (\exists \bEv \le \aEv, \bEv\not=\aEv)  \aPS,\bEv \models \afo \\
      \aPS,\aEv &\models& \always\afo, \text{ if } (\forall \bEv \le \aEv,\,  \bEv\not=\aEv)\; \aPS,\bEv \models \afo
    \end{array} 
  \end{displaymath}
  Define $\aPS \models \afo$ if
  $(\forall \aEv \in \Event) \;\aPS,\aEv \models\afo$ and $\aPSS\models \afo$
  if $(\forall \aPS \in \aPSS)\; \aPS \models\afo$.
%\end{definition}

Let $\once\afo$ be defined as $\lnot(\always\lnot\afo)$. 
In addition, let $\FALSE$, $\lor$ and $\Rightarrow$ be defined in the
standard way.
% $\afo\lor\bfo$ for $\lnot(\lnot \afo \land \lnot \bfo)$,
% and $\afo \Rightarrow \bfo$ for $\lnot \afo \lor \bfo$.

The past operators do not include the current instant, and thus 
they do \emph{not} satisfy the rule
\begin{math}
  \always\afo\Rightarrow\once\afo.
\end{math}
However, they do satisfy:
% \begin{align*}  
%   \frac{\aPS \models \afo \Rightarrow\once{\afo}}{\aPS \models \lnot \afo}\text{(Coinduction)}
%   &&
%   \frac{\aPS \models \always\afo \Rightarrow\afo}{\aPS \models \afo}\text{(Induction)}
% \end{align*}
\begin{gather*}
  \tag{Coinduction}
  (\afo \Rightarrow\once{\afo}) \Rightarrow\lnot \afo
  \\
  \tag{Induction}
  (\always\afo \Rightarrow\afo) \Rightarrow\afo
\end{gather*}
% \begin{description}
% \item[Coinduction.]
%   \begin{math}
%     (\afo \Rightarrow\once{\afo}) \Rightarrow\lnot \afo
%   \end{math}
% \item[Induction.] 
%   \begin{math}
%     (\always\afo \Rightarrow\afo) \Rightarrow\afo
%   \end{math}
% \end{description}
Note that $\aPS \models \afo \land \always\afo$ whenever $\aPS \models \afo$.

We now present two proof rules.  The first rule captures the semantics of
local variables.  Define
\begin{math}
  \closed(\aLoc) = (\DR{\aLoc}{\aVal} \Rightarrow \once \DW{\aLoc}{\aVal}).
\end{math}
Although this definition does not mention intervening writes, it is
sufficient for our example.  It is straightforward to establish that
following rule is sound:
\begin{displaymath}
  \tag{Closing $\aLoc$}
  \frac{
    \afo \text{ is independent of } \aLoc
    \qquad
    \aPS \models \closed(\aLoc) \Rightarrow \afo
  }{
    \nu \aLoc \DOT \aPS \models \afo
  }
\end{displaymath}

The second rule describes composition, in the style of Abadi and
Lamport~\cite{Abadi:1993:CS:151646.151649}.  To simplify the presentation, we
consider the special case with a single invariant.
% We view the
% composition result as capturing key aspects of no-ThinAirRead, as will become
% clearer in the examples below.
In order to state the theorem, we generalize the satisfaction relation to
include environment assumptions.  Let
\begin{math}
  \mods{(\afo)} = \{ \aPSS \mid \aPSS \models \afo \}
\end{math}
be the set of pomsets that satisfy $\afo$.  We say that $\afo$ is prefix
closed if $\mods{(\afo)}$ is prefix-closed\footnote{$\aPS'$ is a prefix of
  $\aPS$ if $\Event'\subseteq\Event$, $\aEv\in\Event'$ and $ \bEv\le\aEv$
  imply $\bEv\in\Event'$, and $(\labeling',\le',\gtN')$ coincide with
  $(\labeling,\le,\gtN)$ for elements of $\Event'$.}.
  % when $\aEv,\,\bEv\in\Event'$: $\labeling'(\aEv)=\labeling(\aEv)$,
  % $\aEv\le'\bEv$ iff $\aEv\le\bEv$, and $\aEv\gtN'\bEv$ iff
  % $\aEv\gtN\bEv$. 
\begin{noenv}
  Define
  \begin{math}
    \afo, \aPSS \models \bfo  \text{ if } \mods{(\afo)} \parallel \aPSS \models \bfo.
  \end{math}
\end{noenv}
\begin{proposition}%[Composition]
  Let $\afo$ be prefix-closed.  Let $\aPSS_1, \aPSS_2$ be
  augmentation-closed.%\footnote{$\aPS'$ is an augmentation of $\aPS$ if
 %   $\Event'=\Event$, $\aEv\le\bEv$ implies $\aEv\le'\bEv$, $\aEv\gtN\bEv$
 %   implies $\aEv\gtN'\bEv$, and
 %   % $\labeling'(\aEv)=\labeling(\aEv)$
 %   if $\labeling(\aEv) = (\bForm \mid \bAct)$ then
 %   $\labeling'(\aEv) = (\bForm' \mid \bAct)$ where $\bForm'$ implies
 %   $\bForm$.}
  Then:
  \begin{displaymath}
    \tag{Composition}
    \frac{
      \afo, \aPSS_1 \models\afo
      \qquad
      \afo, \aPSS_2 \models\afo
    }{\aPSS_1 \parallel \aPSS_2 \models \afo}
  \end{displaymath}
\end{proposition}
\begin{proof}[Proof sketch]
  We will show that all prefixes in the prefix closures of
  $\aPSS_1 \parallel \aPSS_2$ satisfy the required property.  Proof proceeds
  by induction on prefixes of $\aPS \in \aPSS_1 \parallel \aPSS_2$.

  The case for empty prefix  follows from assumption that  $\afo$ is prefix closed.  

  For the inductive case, consider %$\aPS$ in the prefix closure of $\aPSS_1 \parallel \aPSS_2$, i.e.
  $\aPS \in \aPS_1 \parallel \aPS_2$ where
  $\aPS_i \in \aPSS_i$.  Since $\aPSS_1$ and $\aPSS_2$ are augmentation
  closed, we can assume that the restriction of $\aPS$ to the events of
  $\aPS_i$ coincides with $\aPS_i$, for $i=1,2$.
  %
  Consider a prefix $\aPS'$ derived by removing a maximal element $\aEv$ from
  $\aPS$.  Suppose $\aEv$ comes from $\aPS_1$ (the other case is
  symmetric). Since $\aPS_2$ is a prefix of $\aPS'$ and $\aPS' \models \afo$
  by induction hypothesis, we deduce that $\aPS_2 \models \afo$.
  % Thus, $\aPS_2 \in \mods{(\afo)}$.
  Since $\aPS_1 \in \aPSS_1$, by assumption $\afo, \aPSS_1 \models\afo$ we
  deduce that $\aPS \models \afo$.
\end{proof}

We now turn the conditional TAR-pit program, which is a variant of \cite[Figure 8]{DBLP:journals/toplas/Lochbihler13}:
\begin{multline*}
  \VAR x\GETS0\SEMI \VAR y\GETS0\SEMI \VAR z\GETS0\SEMI  \\[-.5ex] (
    y\GETS x
  \PAR
    \IF{z}\THEN x\GETS1 \ELSE x\GETS y\SEMI a\GETS y \FI
  \PAR
    z\GETS1
  )
\end{multline*}
This program is allowed to write $1$ to $a$ under many speculative
memory models
\cite{Manson:2005:JMM:1047659.1040336,Jagadeesan:2010:GOS:2175486.2175503,DBLP:conf/popl/KangHLVD17},
even though the read of $1$ from $y$ in the else branch of the second
thread arises out of thin air.   In contrast, we prove the formula
\begin{math}
  \lnot\once(\DW{a}{1})
\end{math}
holds for the models of this program in our semantics.  We start with the following invariant,
which holds for each of the three threads, and thus, by composition, for the
aggregate program:
\begin{align*}
  &[\once(\DW{y}{1}) \Rightarrow \once(\DR{x}{1})]
  \land\\[-.5ex]
  &[\once(\DW{a}{1}) \Rightarrow (\once(\DR{y}{1}) \land \always(\DW{x}{1} \Rightarrow \once(\DR{y}{1})))]
\end{align*}
Closing $y$, we have,
\begin{math}
  \once(\DR{y}{1}) \Rightarrow \once(\DW{y}{1}) % \Rightarrow \once(\DR{x}{1})
\end{math}
which we substitute into the left conjunct to get:
\begin{displaymath}
  \once(\DR{y}{1}) \Rightarrow \once(\DR{x}{1})
\end{displaymath}
which in turn we substitute into the right conjunct to get:
\begin{displaymath}
  \once(\DW{a}{1}) \Rightarrow (\once(\DR{x}{1}) \land \always(\DW{x}{1} \Rightarrow \once(\DR{x}{1})))
\end{displaymath}
Closing $x$, we can replace $\once(\DR{x}{1})$ with $\once(\DW{x}{1})$:
\begin{displaymath}
  \once(\DW{a}{1}) \Rightarrow (\once(\DW{x}{1}) \land \always(\DW{x}{1} \Rightarrow \once(\DW{x}{1})))
\end{displaymath}
Applying coinduction to the right conjunct, we have:
\begin{displaymath}
  \once(\DW{a}{1}) \Rightarrow (\once(\DW{x}{1}) \land \always(\lnot \DW{x}{1}))
\end{displaymath}
Simplifying, we have, as required:  
\begin{displaymath}
  \once(\DW{a}{1}) \Rightarrow \FALSE
\end{displaymath}

\end{document}
