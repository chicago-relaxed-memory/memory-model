\section{Data Race Free Behaviors are Sequentially Consistent}
\label{sec:sc}

% For any $\aPSS$, then $\closed(\aPSS)$ is set enriched with useless reads
% (preserving augmentation closure) and where we remove any event whose
% precondition is not a tautology.
% \begin{definition}
%   Let $\addRead(\aPSS)$ be the set $\aPSS'$ where $\aPS'\in\aPSS'$ whenever
%   there is $\aPS\in\aPSS$ such that:
%   $\Event' = \Event\cup{\cEv}$,
%   ${\le'} \supseteq {\le}$, 
%   ${\gtN'} \supseteq {\gtN}$,
%   and
%   $\labelingAct'(\cEv) = (\DR{\aLoc}{\aVal})$ and $\labelingAct'(\aEv) = \labelingAct(\aEv)$,
% \end{definition}
% Then $\fclosed(\aPSS)$ 

Programs that require release sequences for happens-before always include a
data race.  Since we are only defining hb for the purpose of defining data
race for the DRF theorem, we do not include release sequences in our
definition of hb.

In this section, we prove the SC-DRF theorem, which states that any program
that lacks data races under the SC semantics must only have executions that
are compatible with SC executions.

The program $x\GETS1\PAR x\GETS2$ is considered to have an SC data race, but
$x\GETS1\SEMI x\GETS2$ does not.  In our semantics the only difference
between these is that $x\GETS1\SEMI x\GETS2$ enforces weak order between the
writes.  Note also that the
$\sem{x\GETS1\SEMI a\GETS y}=\sem{a\GETS y\SEMI x\GETS1}$, yet these two must
be distinguished in SC, as per the load-buffering litmus test.

In order to define SC executions and SC data races, it is necessary to
augment our semantics to record program order and internal reads.  We extend the definitions in
\textsection\ref{sec:semantics} with
${\rpox}\subseteq{\Event}\times{\Event}$, defined as follows
\begin{itemize}
\item
  ${\rpox'} = {\rpox}$
  when $\aPSS'=\aPSS\aSub$
  or $\aPSS'=\aForm\guard\aPSS$
\item
  ${\rpox'} = {\rpox}\restrict{\Event'}$
  when $\aPSS'=\nu\aLoc\st\aPSS$
\item
  ${\rpox'} = {\rpox}^1\cup{\rpox}^2$
  when $\aPSS'=\aPSS^1\parallel\aPSS^2$
\item
  ${\rpox'} = {\rpox}\cup\{(\cEv,\aEv)\mid\aEv\in\Event\}$
  when $\aPSS'=\aAct\prefix\aPSS$ and $\Event' = \Event \cup \{\cEv\}$
\end{itemize}
In addition, we re-define the rules for relaxed reads and writes as follows:
\begin{align*}
  \sem{\REF{\bExp}\GETS\aExp\SEMI \aCmd} & =
  \textstyle\bigcup_\aLoc\; (\REF{\bExp}=\aLoc) \guard \textstyle\bigcup_\aVal\;  (\aExp=\aVal) \guard\bigl((\DW\aLoc\aVal) \prefix \sem{\aCmd}\bigr)[\aExp/\aLoc]
  \\ & \mkern2mu\cup \textstyle\bigcup_\aLoc\; (\REF{\bExp}=\aLoc) \guard \DW{\aLoc}{} \guard (\iDW{\aLoc}{0}) \prefix \sem{\aCmd}[\aExp/\aLoc]
  \\
  \sem{\aReg\GETS\REF{\bExp}\SEMI \aCmd} & =
  \textstyle\bigcup_\aLoc\; (\REF{\bExp}=\aLoc) \guard \textstyle\bigcup_\aVal\; (\DR\aLoc\aVal) \prefix \sem{\aCmd}[\aLoc/\aReg] 
  \\ & \mkern2mu\cup \textstyle\bigcup_\aLoc\; (\REF{\bExp}=\aLoc) \guard (\iDR{\aLoc}{0}) \prefix \sem{\aCmd}[\aLoc/\aReg]
\end{align*}
The difference is that we record internal action
$(\iDR{\aLoc}{})$ or $(\iDW{\aLoc}{})$.  (The value in this action is ignored.)

Throughout our discussion of data races, we assume that programs do not use
restriction, and therefore that all internal actions arise from this rule.



Let $\IDAcq$ be the identity relation on acquire events, and likewise $\IDRel$ on release events.
We say $\aEv\rrf\bEv$ if $\aEv$ writes $\aLoc$, $\bEv$ reads $\aLoc$, and for any $\cEv$ that writes $\aLoc$ either $\cEv\gtN\aEv$ or $\bEv\gtN\cEv$.
Now define $\rsw$ and $\rhb$.
\begin{align*}
  {\rsw} &= \IDRel; ({\rrf}\setminus{\rpox}); \IDAcq
  \\
  {\rhb} &= ({\rpox} \cup {\rsw})^+
\end{align*}
We say that two actions have a \emph{data-race conflict} if at least one
action is a write or internal write to a location $\aLoc$ and the other is a
read, write, internal read or internal write to the same location.  A pomset
has a \emph{data race} if there are events $\aEv$ and $\bEv$ such that
\begin{itemize}
\item $\aEv$ and $\bEv$ are unordered by $\rhb$,
\item $\labelingForm(\aEv)$ and $\labelingForm(\bEv)$ are tautologies,
\item $\labelingAct(\aEv)$ and $\labelingAct(\bEv)$ have a data-race conflict.
\end{itemize}

The semantics of programs includes SC executions.  Let $\semsc{\aCmd}$ be the
executions where
\begin{itemize}
\item ${\gtN}\cup{\rpox}$ is acyclic (and therefore ${\le}\cup{\rpox}$ is acyclic),
\item prefixing ($\prefix$) and composition ($\parallel$) take disjoint union, and
\item all reads are denoted by explicit actions.
\end{itemize}
Our semantics already ensures ${\le}\subseteq{\rsw}$.

We say that $\aCmd$ has an \emph{SC race} if there is some pomset in $\semsc{\aCmd}$
that contains a data race.

In this section we show that if $\semsc{\aCmd}\subseteq\sem{\aCmd}$ has only
race-free executions, then each pomset $\aPS\in\sem{\aCmd}$ is ``equivalent''
to some $\aPS'\in\semsc{\aCmd}$, where $\aPS'$ may have more events, but
preserves labeling and has less order.

We say that $\aPS\suborder\aPS'$ if there is an injection
$\inj:\Event'\rightarrow\Event$ such that:
\begin{itemize}
\item $\labelingAct'(\aEv) = \labelingAct(\inj(\aEv))$
\item $\labelingForm'(\aEv)$ implies $\labelingForm(\inj(\aEv))$
\item $\labelingForm(\bEv)$ implies $\bigvee_{\aEv\in\inj^{-1}(\bEv)}(\labelingForm'(\aEv))$
\item $\aEv\le'\bEv$ implies $\inj(\aEv)\le\inj(\bEv)$
\item $\aEv\gtN'\bEv$ implies $\inj(\aEv)\gtN\inj(\bEv)$
\end{itemize}

\begin{theorem}
  If $\semsc{\aCmd}$ contains only race-free executions and
  $\aPS\in\sem{\aCmd}$ then there exists some $\aPS'\in\semsc{\aCmd}$ such
  that $\aPS\suborder\aPS'$.
\end{theorem}
% \begin{proof}
%   \begin{itemize}
%   \item
%     \begin{math}
%       \sem{\SKIP}
%       =
%       \{ \emptyset \} 
%     \end{math}
%   \item
%     \begin{math}
%       \sem{\FENCE_{\aF}\SEMI \aCmd}
%       =
%       (\DF{\aF}) \prefix \sem{\aCmd}
%     \end{math}
%   \item
%     \begin{math}
%       \sem{\aLoc\GETS\aExp\SEMI \aCmd}
%       =
%       \textstyle\bigcup_\aVal\; \bigl((\aExp=\aVal) \guard (\DW\aLoc\aVal) \prefix \sem{\aCmd}\bigr)[\aExp/\aLoc]
%     \end{math}
%   % \item
%   %   \begin{math}
%   %     \sem{\REL\aLoc\GETS\aExp\SEMI \aCmd}
%   %     =
%   %     \textstyle\bigcup_\aVal\; \bigl((\aExp=\aVal) \guard (\DWRel\aLoc\aVal) \prefix \sem{\aCmd}\bigr)[\aExp/\aLoc]
%   %   \end{math}
%   \item
%     \begin{math}
%       \sem{\aReg\GETS\aLoc\SEMI \aCmd}
%       =
%       \sem{\aCmd}[\aLoc/\aReg] \cup \textstyle\bigcup_\aVal\; (\DR\aLoc\aVal) \prefix \sem{\aCmd}[\aLoc/\aReg]
%     \end{math}
%   % \item
%   %   \begin{math}
%   %     \sem{\ACQ\aReg\GETS\aLoc\SEMI \aCmd}
%   %     =
%   %     \textstyle\bigcup_\aVal\; (\DRAcq\aLoc\aVal) \prefix \sem{\aCmd}[\aLoc/\aReg]
%   %   \end{math}
%   \item
%     \begin{math}
%       \sem{\IF{\aExp} \THEN \aCmd \ELSE \bCmd \FI}
%       =
%       \bigl((\aExp \neq 0) \guard \sem{\aCmd}\bigr) \parallel \bigl((\aExp=0) \guard \sem{\bCmd}\bigr)
%     \end{math}
%   \item
%     \begin{math}
%       \sem{\aCmd \PAR \bCmd}
%       =
%       \sem{\aCmd} \parallel \sem{\bCmd}
%     \end{math}
%   \item
%     \begin{math}
%       \sem{\VAR\aLoc\SEMI \aCmd}
%       =
%       \nu \aLoc \st \sem{\aCmd}
%     \end{math}
% \end{itemize}
  
% \end{proof}
% \end{theorem}


% To define compatibility, we extend the definitions of
% \textsection\ref{sec:semantics} to include an additional order: $\rird$.
% \begin{itemize}
% \item
%   ${\rird'} = {\rird}$
%   when $\aPSS'=\aPSS\aSub$
%   or $\aPSS'=\aForm\guard\aPSS$
% \item
%   ${\rird'} = {\rird}\restrict{\Event'}$
%   when $\aPSS'=\nu\aLoc\st\aPSS$
% \item
%   ${\rird'} = {\rird}^1\cup{\rird}^2$
%   when $\aPSS'=\aPSS^1\parallel\aPSS^2$
% \item
%   ${\rird'} = {\rird}\cup\{(\cEv,\aEv)\mid\labelingForm(\aEv) \;\text{is dependent on}\; \aLoc\}$
%   when $\aPSS'=\aAct\prefix\aPSS$, $\aAct$ writes $\aLoc$, and $\Event' = \Event \cup \{\cEv\}$
% \end{itemize}

% From $\rird$, we define ${\rrb}={\rird}^{-1};{\gtN}$.

% We want that if there is an execution:
% \begin{tikzdisplay}[node distance=1em]
%   \event{a}{\DW{\aLoc}{1}}{}
%   \event{b}{\DW{\bLoc}{1}}{below right=1em and 6em of a}
%   \event{c}{\DW{\aLoc}{2}}{above right=1em and 1em of b}
%   \wk{a}{c}
%   \ird{a}{b}
%   \rb{b}{c}
% \end{tikzdisplay}
% Then there is also
% \begin{tikzdisplay}[node distance=1em]
%   \event{a}{\DW{\aLoc}{1}}{}
%   \event{b}{\DW{\bLoc}{1}}{below right=1em and 6em of a}
%   \event{c}{\DW{\aLoc}{2}}{above right=1em and 1em of b}
%   \event{r}{\DR{\aLoc}{1}}{below right=.1em and 2em of a} 
%   \wk{a}{c}
%   \rf{a}{r}
%   \po{r}{b}
% \end{tikzdisplay}


% To see that we need $[\aExp/\aLoc]$ in the rule for write, rather than $[\aVal/\aLoc]$
% consider example:
% \begin{verbatim}
% r=y; if (r) {x=r} else {x=r}; s=x; if (r==s) {z=1}
% \end{verbatim}
% or simplified:
% \begin{verbatim}
% r=y;x=r;s=x; if(s==r){z=1}
% \end{verbatim}
% If you read 37 for $y$, then the predicate on \texttt{Wz1} before the
% read is either $r=r$ or $v=r$, where $v=37$, for example.  In one case you
% get a dependency and in the other you do not.

