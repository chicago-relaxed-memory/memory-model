\section{Other Related Work}\label{sec:related}
We survey related work not discussed previously.

A memory consistency model for a shared-memory multiprocessor defines the
values that a read may return.  For a survey of hardware models, see
\citep{AlglaveThesis}. For software models, see
\citep{DBLP:journals/toplas/Lochbihler13,DBLP:phd/ethos/Batty15}.  For an
attempt to bridge the two, see \citep{DBLP:journals/pacmpl/PodkopaevLV19}.
\citet{DBLP:conf/pldi/PultePKLH19} present an operational model of
\armeight{} in the style of \cite{DBLP:conf/popl/KangHLVD17}.

In our previous work~\cite{2019-sp}, we introduced the notion of pomsets with
preconditions.  In \citeyear{2019-sp}, we studied \emph{micro\hyp{}architecture},
where failed speculative execution is visible via cache effects.  Here we
presented an \emph{architectural} model, which allows us to impose
\emph{consistency}, ignoring failed speculative execution, and \emph{causal
  strengthening}.  In \citeyear{2019-sp}, we used 3-valued pomsets, as
opposed to the simple pomsets used here; see \textsection\ref{sec:limits} for
further discussion.  The previous paper was not focused on memory models, and
thus did not prove the soundness of compiler optimizations, the absence of
thin air reads, efficient implementability, or \drfsc{}.

Our model shares important structural elements with that of
\citet{DBLP:conf/esop/PaviottiCPWOB20}, who provide a fix for the \oota{}
problem in C11 relaxed access.  Like us, they use true concurrency semantics
to identify independencies in an execution and thus calculate the preserved
program order explicitly.  Our definition of parallel composition allows
events to coalesce, taking preconditions via disjunction---this is mirrored
by \citeauthor{DBLP:conf/esop/PaviottiCPWOB20}'s definition of coproduct.  We
only compose downsets of completed pomsets---this is mirrored by their
condition on $\leq^{\mathcal{X}}$ during coproduct (discussed in their
\textsection6.3).  Nonetheless, the papers have different goals, leading to
different outcomes.  For example, their model can be implemented efficiently
on non-\mca{} architectures; our model clearly can not!  Conversely, our
model provides an intrinsic characterization of optimizations, such as
redundant read elimination, which only hold in their model up to
observational refinement.

True concurrency techniques have been applied to relaxed memory by
\citet{DBLP:conf/esop/CenciarelliKS07}, \citet{Castellan},
\citet{Pichon-Pharabod:2016:CSR:2837614.2837616}, and
\citet{DBLP:conf/cgo/ChakrabortyV17}.  See
\cite[\textsection8]{DBLP:journals/lmcs/JeffreyR19} for a discussion.
A partial order approach to weak memory was sketched by \citet{brookes} and
fleshed out for TSO by \citet{DBLP:journals/lmcs/KavanaghB19}.  Their action
labels include buffers, encoding the operational behavior of TSO inside the
pomsets themselves.

There is a rich literature on the use of transformations over SC
executions to model relaxed memory:
\citet{Saraswat:2007:TMM:1229428.1229469}
aimed to describe a \jmm-like model this way---\drfsc\ holds, but \citet{SP} 
discovered that this model permits \oota\ behavior.
\citet{DBLP:conf/popl/DemangeLZJPV13} developed \textsc{bmm}, which permits
reordering of a relaxed write with a following relaxed read.
\textsc{bmm} is designed as a restriction of the \jmm\ that compiles efficiently to
\tso.  It requires fencing on other architectures.
\citet{DBLP:conf/fm/LahavV16} characterized \tso\ as being derived by
considering Write-Read (\textsf{WR}) reordering and Read-After-Write
(\textsf{RAW}) elimination.  They also showed that the release-acquire
accesses of C11 are less expressive than considering \textsf{WR} and
\textsf{RAW} together with thread-inlining.  Our paper is inspired by their
implicit challenge: ``Some memory models can be defined via transformations.
But there is more to weak memory than transformations.''

\section{Limitations}
\label{sec:limits}

Our work has several limitations, each of which provides an opportunity for
future research.  % In several cases, these limitations have allowed us to

We have not modeled loops or recursive functions.  These
introduce complexities\allowbreak{}---such as liveness and continuity---that
are orthogonal to the main topic of the paper.

We have not modeled general sequencing of the form $(\aCmd\SEMI\aCmd')$.  The
definition of prefixing (\textsection\ref{sec:pop}) is relatively simple
since we only prepend one action at a time.

We have not modeled mixed-size access
\cite{DBLP:conf/popl/FlurSPNMGSBS17,DBLP:conf/pldi/WattPPBDFPG20}.  
\armeight{} captures multibyte access using multiple events related by
\emph{same instruction} $(\rsi)$ \cite{alglave-git}. To capture the use of
$\rsi$ when defining \emph{locally ordered before} $(\rlob)$, it is likely
sufficient to modify \ref{5b}.  To capture the use of $\rsi$ when defining
\emph{observation} $(\rob)$, it is likely sufficient to modify \ref{rf3} and
\ref{rf4}.

We have only considered peephole optimizations---other optimizations may be
valid.  For example, the model does \emph{not} validate redundant write after
read elimination as a peephole optimization.  Consider this example from
\cite[\textsection 5.3.1]{SevcikThesis}:
\begin{displaymath}
  \sem{r\GETS x\SEMI x\GETS r\SEMI s\GETS x \SEMI \IF{\aReg {\neq} \bReg} \THEN y \GETS 1 \FI}
  \not\supseteq
  \sem{r\GETS x\SEMI s\GETS x  \SEMI \IF{\aReg {\neq} \bReg} \THEN y \GETS 1 \FI}
\end{displaymath}
In the JMM, these are distinguished by the context
\begin{math}
  x\GETS1 \SEMI z_1^\mRA\GETS 1
  \PAR x\GETS2 \SEMI z_2^\mRA\GETS 1
  \PAR \IF{z_1^\mRA \land z_2^\mRA} \THEN \hole{}\FI.
\end{math}
That is not case here, however, due to local \drfsc{}.  Nonetheless,
completed pomsets from the left include a write to $x$, which is not found on
the right.  This write to $x$ cannot be eliminated using $\relfilt[]{\aLoc}$
(Def.~\ref{def:cover}) because there is no following write.  At top-level,
however, the read of $x$ must be fulfilled by \emph{some} write---assuming
that these writes can coalesce, the optimization can be validated.

We have not attempted to validate all program transformations that involve
synchronization, fences or \RMWs{}
\cite{DBLP:conf/popl/VafeiadisBCMN15,DBLP:phd/hal/Morisset17}.  Nonetheless,
our semantics validates roach motel (\ref{AcqW}, \ref{RelW}), redundant load
(\ref{RL}), fence removal (\ref{RF}), and store forwarding (\ref{SF}).  We
expect that dead store elimination (\ref{DS}) generalizes to non-relaxed
access by generalizing $\relfilt[]{\aLoc}$.  Some transformations are not
sound in the model, but we expect them to be provable as metaproperties.  For
example, access-mode strengthening (such as replacing $\mRLX$ by
$\mRA$) is valid up to an equivalence that ignores access modes on
actions.  Other transformations worth studying include commuting adjacent
acquires, commuting adjacent releases, and implementing non-relaxed access
using relaxed access and fences.
Lock elision and access-mode weakening are also interesting, although these
are only sound in certain contexts.
We expect all of these transformations to be valid, given an appropriate
notional of validity.

\begin{comment}
  https://preshing.com/20131125/acquire-and-release-fences-dont-work-the-way-youd-expect/

  Cannot encode R/A actions with actions+fences...

  A release operation prevents preceding memory operations from being delayed
  past it (a;Rel =/=> Rel;a)
  
  A release fence prevents preceding memory operations from being delayed past
  subsequent writes (a;FR;w =/=> w;a;FR)

  An acquire operation prevents subsequent memory operations from being advanced
  before it (Acq;a =/=> a;Acq)

  An acquire fence prevents subsequent memory operations from being advanced
  before prior reads (r;FA;a =/=> FA;a;r)

  https://www.modernescpp.com/index.php/fences-as-memory-barriers

  StoreLoad: Full fence allows a store before to be reordered with respect to a
  load after (wx;F;ry) ===> (ry;F;wx)

  StoreLoad+LoadLoad: Release fence also allows (rx;FR;ry) ===> (ry;FR;rx)

  StoreLoad+StoreStore: Acquire fence also allows (wx;FR;wy) ===> (wy;FR;wx)

  LoadStore: No fence allows a prior load to reorder w.r.t. a subsequent store
  (rx;FR;wy) =/=> (wy;FR;rx)

  https://preshing.com/20120710/memory-barriers-are-like-source-control-operations/
  Good news is that a fullFence does it.

  Bizarrely, it seems this is not supported in C++... You have to go to assembly.
\end{comment}

\begin{comment}
  \color{red} perhaps add example here from referee 1, or discuss \ref{Internal2}.

  Although our model forces some ordering within the thread that includes the
  RA access, threads that lack RA access are free to reorder.  Perhaps this
  gives the desired effect.  For example
  $$
  (y:=1; x^{ra}:=1) || (s:=x; z:=1)
  $$
  allows the execution $(Wz1) \leq (Wy1) \leq (W^{ra}x1) \leq (Rx1)$.  
  (Here, we are using augmentation to add order from $Wz1$ to $Wy1$.)

  We do expect that lock-based implementation of atomics should be sound, if
  locks were introduced as first class entities, but we have not explored this.
  Intuitively, lock elision can be validated by using roach motel to move the
  entire program into the scope of the lock.
\end{comment}

\labeltext{The}{page:logic:limits} logic we presented in \textsection\ref{sec:logic} is only strong enough
to prove a few examples.  \citet{DBLP:conf/esop/SvendsenPDLV18} presented a
different a logic, capable of showing that the following program cannot write
2:
\begin{align*}
  \tag{\textsc{oota6}}
  ( y\GETS x+1
  \PAR
  x\GETS y ) && \hbox{\begin{tikzinline}[node distance=1.5em]
      \event{rx}{\DR{x}{1}}{}
      \event{wy}{\DW{y}{2}}{right=of rx}
      \po{rx}{wy}
      \event{ry}{\DR{y}{1}}{right=3em of wy}
      \event{wx}{\DW{x}{1}}{right=of ry}
      \po{ry}{wx}
      \rf[out=170,in=10]{wx}{rx}
    \end{tikzinline}}
\end{align*}
The attempted execution is \emph{not} allowed by our semantics, since there is no write
to fulfill $(\DR{y}{1})$.  % Proving this requires the ability to reason about
As another example,
consider the following, from \citet[Fig.~3]{DBLP:journals/pacmpl/ChakrabortyV19}:
\begin{gather*}
  \tag{\textsc{oota7}}
  \begin{gathered}
    x\GETS 2\SEMI
    \IF{x\NOTEQ2}\THEN y\GETS 1 \FI
    \PAR
    x\GETS 1\SEMI
    r\GETS x\SEMI
    \IF{y}\THEN x\GETS 3 \FI
    \\\nonumber
    \hbox{\begin{tikzinline}[node distance=1.5em]
        \event{wx2}{\DW{x}{2}}{}
        \event{rx3}{\DR{x}{3}}{right=of wx2}
        \event{wy1}{\DW{y}{1}}{right=of rx3}
        \po{rx3}{wy1}
        \event{wx1}{\DW{x}{1}}{right=2em of wy1}
        \event{rx2}{\DR{x}{2}}{right=of wx1}
        \wk{wx1}{rx2}
        \event{ry1}{\DR{y}{1}}{right=of rx2}
        \event{wx3}{\DW{x}{3}}{right=of ry1}
        \po{ry1}{wx3}
        \wk[in=165,out=15]{rx2}{wx3}
        \rf[in=-170,out=-10]{wy1}{ry1}
        \rf[in=170,out=10]{wx2}{rx2}
        \rf[out=-170,in=-10]{wx3}{rx3}
        \wk[out=-170,in=-10]{wx1}{wx2}
        \wk{wx2}{rx3}
      \end{tikzinline}}
  \end{gathered}
\end{gather*}
The attempted execution is \emph{not} allowed by our semantics, due to the
evident cycle.  Intuitively, it is not possible for the left thread to
read $3$ for $x$ when the right thread reads $2$.  Proving this may require a
logic with modalities to deal with intervening writes and coherence.  Surprisingly, this outcome is allowed by the promising
semantics \cite{DBLP:conf/popl/KangHLVD17}.
\citeauthor{DBLP:journals/pacmpl/ChakrabortyV19} developed \weakestmo{} to
address this example; however, \weakestmo{} does not address \ref{OOTA4}.  

Our model realizes \emph{multi-copy atomicity} (\mca).  Thus it will not
compile efficiently to non-\mca{} architectures, such as \ppc\ and \armseven.
To do so, one cannot include the order required by \ref{rf4} in pomset order.
In \cite{2019-sp}, we modeled programs using \emph{3-valued pomsets}, using
the \emph{weak} relation ($\gtNsp$) for \ref{rf4} and \ref{5b}.  This does
\emph{not} provide a suitable model for non-\mca{} behavior, however, since
it disallows \ref{MCA2}.\footnote{Following \cite{DBLP:journals/dc/Lamport86},
  3-valued pomset require: (1) if $d\leq e$ then $d\gtNsp e$, (2) if
  $d\leq e$ and $e\gtNsp d$ then $d=e$, and (3) if $c\leq d \gtNsp e$ or
  $c\gtNsp d\leq e$ then $c\gtNsp e$.  \ref{MCA2} is disallowed by (2).
  Top-level 3-valued pomsets also require that $\gtNsp$ is a partial order
  per-location; a sufficient condition is (4) if $d\gtNsp e$ and $e\gtNsp d$
  then $d$ and $e$ do not touch the same location.  Although \ref{MCA1} is
  allowed, its two-thread variant is not, due to the combination of
  \emph{semi\hyp{}transitivity (3)} and \emph{partial coherence (4)}.
  It is plausible, perhaps, to require only that $\gtNsp$ is a partial
  order and that $\bEv\gtNsp\aEv$ when $\labelingAct(\bEv)$ and
  $\labelingAct(\aEv)$ conflict and $\bEv\le\aEv$.}

In the discussion of \ref{MCA2}, we noted that our model does not enforce
order between reads dues to address and control dependencies.  This has
implications for Java's final field semantics.  Consider:
\begin{gather*}
  \taglabel{addr2}
  \begin{gathered}
    (r\GETS 1 \SEMI \REF{r}\GETS 0 \SEMI \REF{r}\GETS 1\SEMI  x^\mRA\GETS r)
    \PAR
    (r\GETS x^\mRA \SEMI s \GETS \REF{r})
    \\[-1ex]
    \hbox{\begin{tikzinline}[node distance=1.5em]
        \event{a1}{\DW{\REF{1}}{0}}{}
        \event{a2}{\DW{\REF{1}}{1}}{right=of a1}
        \wk{a1}{a2}
        \event{a4}{\DWRel{x}{1}}{right=of a2}
        \sync{a2}{a4}
        \event{b1}{\DR[\mRA]{x}{1}}{right=3em of a4}
        \event{b2}{\DR{\REF{1}}{0}}{right=of b1}
        \sync{b1}{b2}
        \rf{a4}{b1}
        \wk[out=170,in=10]{b2}{a2}
      \end{tikzinline}}
  \end{gathered}
\end{gather*}
If we changed the read of $x^\mRA$ to $x^\mRLX$, then there would be no order from
$(\DR[\mRLX]{x}{1})$ to $(\DR{\REF{1}}{0})$, and the execution would be allowed.  In
order to allow this relaxation in certain cases without invalidating the
publication of the ``field initializer'' $(\DW{\REF{1}}{1})$, it may be
desirable to distinguish address dependencies from other dependencies---doing
so would likely require two separate preconditions for each event.

\section*{Acknowledgements}
This paper was greatly improved by the comments of the anonymous reviewers.
Jagadeesan and Riely were supported by the National Science Foundation under
grant No.~CCR-1617175.
