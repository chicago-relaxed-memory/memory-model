\section{Understanding ``Out Of Thin Air'' using Temporal Logic}
\label{sec:logic}

A significant challenge for a software memory model is to relax order enough
to allow efficient implementation without admitting anomalous
behaviors---called \emph{out of thin air} (\oota) in the literature
\cite{vacuous,DBLP:conf/esop/BattyMNPS15,BoehmOOTA}.  The most famous example
is \ref{OOTA3} from \textsection\ref{sec:intro}.  Here we inline
initialization in order to fit the format of our proof rules:
\begin{align}
  \label{OOTA3} \tag{\textsc{oota1}}
  \PW{y}{0}\SEMI 
  \PW{y}{x}
  \!\PAR\!
  \PW{x}{0}\SEMI
  \PR{y}{r}\SEMI \PW{x}{r}  
  &&
  \smash{\hbox{\begin{tikzinline}[node distance=1.2em]
        \event{rx}{\DR{x}{1}}{}
        \event{wy}{\DW{y}{1}}{right=of rx}
        \po{rx}{wy}
        \event{y0}{\DW{y}{0}}{left=of rx}
        \event{x0}{\DW{x}{0}}{right=2em of wy}
        \event{ry}{\DR{y}{1}}{right=2em of x0}
        \event{wx}{\DW{x}{1}}{right=of ry}
        \po{ry}{wx}
        \rf[out=10,in=170]{wy}{ry}
        \rf[out=170,in=10]{wx}{rx}
        \wk[out=-15,in=-165]{y0}{wy}
        \wk[out=-15,in=-165]{x0}{wx}
      \end{tikzinline}}}
\end{align}
Although Java does not allow \oota{} behaviors of \ref{OOTA3},
\citet{DBLP:journals/toplas/Lochbihler13} showed that it does allow \oota\
behaviors of \ref{OOTA1}, also from \textsection\ref{sec:intro}.  In
\cite{DBLP:conf/lics/JeffreyR16}, we described a logic that rules out
\ref{OOTA3} but not \ref{OOTA1} or its variant \ref{OOTA4}.  In this section,
we provide a more accurate test of \oota{} behaviors by enhancing our
previous logic with temporal features.

On first read, we suggest that readers skip to the examples and the
discussion that follows, coming back to the definitions as necessary.
Example~\ref{ex:thin} discusses the canonical \oota{} example \ref{OOTA3};
the analysis is trivial and well-known \cite{DBLP:conf/lics/JeffreyR16,
  DBLP:conf/popl/KangHLVD17}.  Example~\ref{ex:lochb} is more interesting.
There, we discuss \ref{OOTA4}, which is a variant of
\citeauthor{DBLP:journals/toplas/Lochbihler13}'s \ref{OOTA1}.

The logic given here is not meant to be definitive; in
\textsection\ref{sec:limits}, we discuss \oota{} examples that appear to
require non-trivial extensions
\cite{DBLP:conf/esop/SvendsenPDLV18,DBLP:journals/pacmpl/ChakrabortyV19}.

\noparagraph{Definitions}
We adapt past linear temporal logic (\pLTL)
\cite{Lichtenstein:1985:GP:648065.747612} to pomsets by dropping the previous
instant operator and adopting strict versions of the temporal operators.
The atoms of our logic are write and read events.
Given a pomset $\aPS$ and event $\aEv$, define:\nofootnote{Let $\FALSE$, $\lor$,
  $\Rightarrow$ and $\once$ as usual;
  for example,
  $\once\afo = \lnot(\always\lnot\afo)$.}
\begin{displaymath}
  \renewcommand{\arraycolsep}{.2ex}
  \begin{array}{lrll}
    \aPS,\aEv &\models& \DW{\aLoc}{\aVal} &\text{ if } \labelingAct(\aEv) = \DW{\aLoc}{\aVal} \text{ and } \TRUE \text{ implies } \labelingForm(\aEv) \\
    \aPS,\aEv &\models& \DR{\aLoc}{\aVal} &\text{ if } \labelingAct(\aEv) = \DR{\aLoc}{\aVal} \text{ and } \TRUE \text{ implies } \labelingForm(\aEv) \\
    \aPS,\aEv &\models& \afo\land\bfo &\text{ if } \aPS,\aEv \models  \afo \text{ and } \aPS,\aEv \models  \bfo \\
    \aPS,\aEv &\models& \TRUE\\
    \aPS,\aEv &\models& \lnot\afo &\text{ if } \aPS,\aEv \not\models \afo \\
    \aPS,\aEv &\models& \always\afo &\text{ if } \forall \bEv \lt \aEv.\; \aPS,\bEv \models \afo\\
    \aPS,\aEv &\models& \once\afo &\text{ if } \exists \bEv \lt \aEv.\;  \aPS,\bEv \models \afo 
  \end{array} 
\end{displaymath}

Define $\FALSE$, $\lor$, and $\Rightarrow$ as usual.

Let $\aPS \models \afo$ if
$\aPS,\aEv \models\afo$, for all $\aEv \in \Event$.

Let $\aPSS\models \afo$
if $\aPS \models\afo$, for all $\aPS \in \aPSS$.

Let
\begin{math}
  \afo, \aPSS \models \bfo  \text{ if } \{ \aPS \mid \aPS \models \afo \} \parallel \aPSS \models \bfo.
\end{math}

Let $\afo$ be \emph{downclosed} when
$\{ \aPS \mid \aPS \models \afo \}$ is.

The past operators do not include the current instant, and so
do \emph{not} satisfy
$(\always\afo\Rightarrow\once\afo)$. The order-minimal elements always validate
$\always\afo$ and invalidate
$\once\afo$.
However, we can prove the following:
\begin{align*}
  \tag{Induction}
  \aPS \models& (\always\afo \Rightarrow\afo) \Rightarrow\afo
  \\[-1ex]
  \tag{Coinduction}
  \aPS \models& (\afo \Rightarrow\once{\afo}) \Rightarrow\lnot \afo
  \\[-1ex]
  \tag{Weakening}
  \aPS \models& (\afo \Rightarrow\once{\bfo}) \Rightarrow (\once\afo \Rightarrow\once{\bfo})
\end{align*}

We present two additional proof rules. 
The first provides a logical view of \emph{$\aLoc$-closure} (Def.~\ref{def:rf}):
\begin{displaymath}
  \frac{
    \afo \text{ is independent of } \aLoc
    \qquad
    \aPS \models (\DR{\aLoc}{\aVal} \Rightarrow \once \DW{\aLoc}{\aVal}) \Rightarrow \afo
  }{
    \nu \aLoc \DOT \aPS \models \afo
  }
\end{displaymath}

The second rule describes concurrent composition, in the style of~\citet{Abadi:1993:CS:151646.151649}.  To simplify the presentation, we
consider the special case with a single invariant.

\begin{proposition}%[Composition]
  Let $\afo$ be downclosed.  Let $\aPSS_1, \aPSS_2$ be
  augmentation\hyp{}closed. %\footnote{$\aPS'$ is an augmentation of $\aPS$ if
  Then:
  \begin{displaymath}
    \frac{
      \afo, \aPSS_1 \models\afo
      \qquad
      \afo, \aPSS_2 \models\afo
    }{\aPSS_1 \parallel \aPSS_2 \models \afo}
  \end{displaymath}
\end{proposition}
\begin{proof}[Proof sketch]
  We will show that all downsets in the downset closures of
  $\aPSS_1 \parallel \aPSS_2$ satisfy the required property.  Proof proceeds
  by induction on downsets of $\aPS \in \aPSS_1 \parallel \aPSS_2$.
  The case for empty downset  follows from assumption that  $\afo$ is downset closed.  
  For the inductive case, consider %$\aPS$ in the downset closure of $\aPSS_1 \parallel \aPSS_2$, i.e.
  $\aPS \in \aPS_1 \parallel \aPS_2$ where
  $\aPS_i \in \aPSS_i$.  Since $\aPSS_1$ and $\aPSS_2$ are augmentation
  closed, we can assume that the restriction of $\aPS$ to the events of
  $\aPS_i$ coincides with $\aPS_i$, for $i=1,2$.
  Consider a downset $\aPS'$ derived by removing a maximal element $\aEv$ from
  $\aPS$.  Suppose $\aEv$ comes from $\aPS_1$ (the other case is
  symmetric). Since $\aPS_2$ is a downset of $\aPS'$ and $\aPS' \models \afo$
  by induction hypothesis, we deduce that $\aPS_2 \models \afo$.
  Since $\aPS_1 \in \aPSS_1$, by assumption $\afo, \aPSS_1 \models\afo$ we
  deduce that $\aPS \models \afo$.
\end{proof}

\begin{example}
  \label{ex:thin}
  \noparagraph{Basic Examples}
  With all variables initialized to $0$, we show that \ref{OOTA3}
  satisfies
  \begin{math}
    \lnot\DW{x}{1}.
  \end{math}

  We start with the invariant:
  \begin{displaymath}
    [\DW{x}{1}\Rightarrow\once\DR{y}{1}]
    \land
    [\DW{y}{1}\Rightarrow\once\DR{x}{1}]
  \end{displaymath}
  This invariant holds for each thread; thus, it holds for the
  aggregate program by composition.  Closing $y$ yields
  \begin{math}
    \DR{y}{1} \Rightarrow \once\DW{y}{1}.
  \end{math}
  Weakening the right conjunct: % yields
  \begin{math}
    \once\DW{y}{1}\Rightarrow\once\DR{x}{1}.
  \end{math}
  Chaining these together: %yields
  \begin{math}
    \DR{y}{1} \Rightarrow \once\DR{x}{1}.
  \end{math}
  Weakening:  %yields
  \begin{math}
    \once\DR{y}{1} \allowbreak\Rightarrow \once\DR{x}{1}. 
  \end{math}
  Chaining into the left conjunct:  %yields
  \begin{math}
    \DW{x}{1} \Rightarrow \once\DR{x}{1}. 
  \end{math}
  Closing $x$, 
  weakening, 
  then chaining: %, yields
  \begin{math}
    \DW{x}{1} \Rightarrow \once\DW{x}{1}. 
  \end{math}
  By coinduction, 
  \begin{math}
    \lnot\DW{x}{1}.
  \end{math}
\end{example}

\begin{example}
  \label{ex:lochb}
  \noparagraph{Lochbihler's Example} %The essential temporal property of
  Because our language lacks object creation, we cannot consider
  \citeauthor{DBLP:journals/toplas/Lochbihler13}'s example (\ref{OOTA1}) directly.  Instead we study \ref{OOTA4}, which has the same
  temporal structure.
  The essential temporal property of
  \ref{OOTA4} is: \emph{A write of $1$ to $y$ must be preceded by a read of
    $1$ from $x$, and if $1$ is written to $z$ then a write of $1$ to $x$
    must be preceded by a read of $1$ from $y$.}
  We show an attempted execution that violates this invariant, eliding
  initialization:
  \begin{gather}
    \label{OOTA4}\tag{\textsc{oota4}}
    \begin{gathered}
      \PW{y}{x}
      \PAR
      \PR{y}{r} \SEMI \IF{b}\THEN  \PW{x}{r} \SEMI \PW{z}{r} \ELSE \PW{x}{1} \FI
      \PAR
      \PW{b}{1}
      \\[-1ex]
      \hbox{\begin{tikzinline}[node distance=1.5em]
          \event{rx}{\DR{x}{1}}{}
          \event{wy}{\DW{y}{1}}{right=of rx}
          \po{rx}{wy}
          \event{ry}{\DR{y}{1}}{right=3em of wy} 
          \event{wx}{\DW{x}{1}}{right=of ry}
          \event{wz}{\DW{z}{1}}{right=of wx}
          \event{rb}{\DR{b}{1}}{right=of wz}
          \event{wb1}{\DW{b}{1}}{right=3em of rb}
          \po{ry}{wx}
          \rf{wb1}{rb}
          \rf{wy}{ry}
          \rf[out=-170,in=-10]{wx}{rx}
          \po{rb}{wz}
          \po[out=15,in=165]{ry}{wz}
        \end{tikzinline}}
    \end{gathered}  
  \end{gather}
  As we discussed in \textsection\ref{sec:pop} there is a dependency from
  $(\DR{y}{1})$ to $(\DW{x}{1})$; thus, the outcome is disallowed.  This
  outcome is also disallowed by our event structures model
  \citep[\textsection9]{DBLP:journals/lmcs/JeffreyR19}, although the logic
  given in that paper is insufficient to establish this fact.  The outcome is
  \emph{allowed} by \citet{Manson:2005:JMM:1047659.1040336}, \citet{DBLP:conf/esop/JagadeesanPR10},
  \citet{DBLP:conf/popl/KangHLVD17}, and
  \citet{DBLP:journals/pacmpl/ChakrabortyV19}.

  To establish that this outcome is disallowed here, we prove 
  \begin{math}
    \lnot\DW{z}{1},
  \end{math}
  starting with invariant:
  \begin{align*}
    [\once\DW{y}{1} \Rightarrow \once\DR{x}{1}]
    \land
    [\notonce\DW{z}{1} \Rightarrow (\once\DR{y}{1} \land \always(\DW{x}{1} \Rightarrow \once\DR{y}{1}))]
  \end{align*}
  Closing $y$ and chaining into the left conjunct:
  \begin{math}
    \once\DR{y}{1} \Rightarrow \once\DR{x}{1}.
  \end{math}
  Chaining into the right conjunct:
  \begin{displaymath}
    \notonce\DW{z}{1} \Rightarrow (\once\DR{x}{1} \land \always(\DW{x}{1} \Rightarrow \once\DR{x}{1}))
  \end{displaymath}
  Closing $x$:
  \begin{math}
    \notonce\DW{z}{1} \Rightarrow (\once\DW{x}{1} \land \always(\DW{x}{1} \Rightarrow \once\DW{x}{1}).
  \end{math}
  Applying coinduction to the right conjunct:
  \begin{displaymath}
    \notonce\DW{z}{1} \Rightarrow (\once\DW{x}{1} \land \always(\lnot \DW{x}{1}))
  \end{displaymath}
  Simplifying:
  \begin{math}
    \notonce\DW{z}{1} \Rightarrow \FALSE,
  \end{math}
  as required.
\end{example}

\noparagraph{RFUB: Register assignment From an Unexecuted Branch}
Many examples are superficially similar, but in fact have fewer dependencies,
such as \eqref{OOTA?} in \textsection\ref{sec:intro}.
\citeauthor{BoehmOOTA}'s [\citeyear{BoehmOOTA}] \ref{RFUB} example presents
another potential form of \oota{} behavior.
Our analysis shows that there is no \oota{} behavior in
\ref{RFUB}, only a false dependency:
\begin{gather*}
  \tag{\textsc{rfub}}\label{RFUB}
  \sem{\PR{y}{r}\SEMI \PW{x}{r}}
  \not\supseteq
  \sem{\PR{y}{r}\SEMI \IF{r \NOTEQ 1} \THEN \PW{z}{1}\SEMI \LET{r}{1}\FI \SEMI \PW{x}{r}}
\end{gather*}
The left command is half of \ref{OOTA3}. %, from \textsection\ref{sec:logic}.
The right command is dubbed \rfub{}, for \emph{Register assignment From an
  Unexecuted Branch}.  \citeauthor{BoehmOOTA} observes that in the context
$\PW{x}{y} \PAR \hole{}$, these programs have different behaviors.  Yet the
\oota{} example on the left never writes $1$.  Why should the unexecuted
branch change that?  Because of the conditional, the write to $x$ in
\ref{RFUB} is independent of the read from $y$.  It useful to considering the
Hoare logic formulas satisfied by the two threads above: we have
$\hoare{\TRUE}{\ref{RFUB}}{x=1}$ for the right thread of \ref{RFUB}, but not
$\hoare{\TRUE}{\ref{OOTA3}}{x=1}$ for the right thread of \ref{OOTA3}.  The
change in the thread from \ref{OOTA3} to \ref{RFUB} is not a valid refinement
under Hoare logic; thus, it is expected that \ref{RFUB} may have additional
behaviors.

Understanding \oota{} behavior is notoriously difficult, even for the
greatest minds in the field!  % We believe that \emph{logic} is the only tool
This example shows the wisdom of using existing tools, such as preconditions
and Hoare logic, to model new problems, such as relaxed memory.

\begin{comment}
  \color{red} Need to sort this out.
  Alan proposes:
\begin{verbatim}
     (W y 2) => <>(R x 1)
     (W y 1) => <>(R x 0)
     (W x 1) => <>(R y 1)
   <>(W x 1) => not(<>(W x 2))  --- which should be???  <>(W x 0) => not(<>(W x 1))
\end{verbatim}

  2020/09/30: This seems to go bad because of initialization...
  The formula
\begin{verbatim}
<>Wx0 => not(<>Wx1)
\end{verbatim}
  does not hold for
\begin{verbatim}
x=0; x=y
\end{verbatim}

  2020/09/10:  I am worried about the compositionality of this predicate:
\begin{verbatim}
I think
   <>(W x 0 => not(<>(W x 1)))
holds for 
   x=0; r=y 
and
   x=1
but not
   x=0; r=y || x=1
as shown by the execution
   Wx1 < Wx0 < Ry0
\end{verbatim}
  
  It is impossible to fulfill $(\DR{y}{1})$ in the following
  \cite[RNG]{DBLP:conf/esop/SvendsenPDLV18}:
  \begin{align*}
    \taglabel{OOTA5}
    ( \PW{y}{x{+}1}
    \PAR
    \PW{x}{y} ) && \hbox{\begin{tikzinline}[node distance=1.5em]
        \event{rx}{\DR{x}{1}}{}
        \event{wy}{\DW{y}{2}}{right=of rx}
        \po{rx}{wy}
        \event{ry}{\DR{y}{1}}{right=3em of wy}
        \event{wx}{\DW{x}{1}}{right=of ry}
        \po{ry}{wx}
        \rf[out=170,in=10]{wx}{rx}
      \end{tikzinline}}
  \end{align*}
  The proof proceeds as before, starting with the following invariant:
  \begin{gather*}
    [\DW{y}{2} \Rightarrow \once\DR{x}{1}] \land
    [\once\DW{x}{1} \Rightarrow \once\DR{y}{1}] \land
    [\once\DW{y}{1} \Rightarrow \once\DR{x}{0}] \land
    [\once\DW{x}{0} \Rightarrow \lnot(\once\DW{x}{1})]
  \end{gather*}
\begin{verbatim}
  Wy2 => <>Rx1  /\  <>Wx1 => <>Ry1  /\  <>Wy1 => <>Rx0  
close x and y                                          
  Wy2 => <>Wx1  /\  <>Wx1 => <>Wy1  /\  <>Wy1 => <>Wx0  
chain
  Wy1 => <>Wx0  
chain with <>Wx0 => not(<>Wx1)
\end{verbatim}
\end{comment}
