\section{Properties of the Basic Model}
\label{sec:props}

It is amazing how much the semantics of \textsection\ref{sec:model} ``gets
right'' out of the box, including value range analysis, internal reads, and
SC access, all of which can be complex in other models.  In this section, we
walk through several litmus tests, valid rewrites and invalid rewrites.  The
examples show that \ref{5a}--\ref{5d} and \ref{rf3}--\ref{rf4} are
understandable as \emph{general principles}.  The interaction of these
principles is limited to a single, global, pomset order.  We discuss tweaks
to the semantics in \textsection\ref{sec:refine}.

\subsection{Litmus Tests}
\label{sec:litmus}

\citet{PughWebsite} developed a set of litmus tests for the java memory
model.  Our model gives the expected result for all but cases 16, 19 and 20
(unrolling loops): we discuss \ref{TC16} below; \textsc{tc19} and
\textsc{tc20} involve a thread join operation, which is not expressible in
our language.  Our model also agrees with the \oota{} examples of
\citet[\textsection 4]{DBLP:conf/esop/BattyMNPS15} and the ``surprising and
controversial behaviors'' of \citet[\textsection
8]{Manson:2005:JMM:1047659.1040336}.  %See \cite{DBLP:conf/esop/PaviottiCPWOB20} for an exhaustive list of litmus tests.

\myparagraph{Buffering}

Consider the \emph{store buffering} and \emph{load buffering} litmus \labeltext[SB]{tests}{SB}:
\begin{align*}
  \taglabel{SB/LB}
  \begin{gathered}
    \PW{x}{0}\SEMI
    \PW{y}{0}\SEMI
    (
    \PW{x}{1}\SEMI\PR{y}{\aReg}
    \PAR
    \PW{y}{1}\SEMI \PR{x}{\aReg})
    \\[-1.5ex]
    \hbox{\begin{tikzinline}[node distance=.9em]
        \event{wx0}{\DW{x}{0}}{}
        \event{wy0}{\DW{y}{0}}{right=of wx0}
        \event{wx}{\DW{x}{1}}{right=3em of wy0}
        \event{ry}{\DR{y}{0}}{right=of wx}
        \event{wy}{\DW{y}{1}}{right=3em of ry}
        \event{rx}{\DR{x}{0}}{right=of wy}
        \rf[out=15,in=165]{wy0}{ry}
        \rf[out=10,in=170]{wx0}{rx}
        \wk{ry}{wy}
        \wk[out=-165,in=-15]{rx}{wx}
      \end{tikzinline}}
  \end{gathered}
  &&
  \begin{gathered}
    \PR{y}{\aReg}\SEMI \PW{x}{1}
    \PAR
    \PR{x}{\aReg}\SEMI \PW{y}{1}
    \\
    \hbox{\begin{tikzinline}[node distance=.9em]
        \event{ry}{\DR{y}{1}}{}
        \event{wx}{\DW{x}{1}}{right=of ry}
        \event{rx}{\DR{x}{1}}{right=3em of wx}
        \event{wy}{\DW{y}{1}}{right=of rx}
        \rf{wx}{rx}
        \rf[out=-165,in=-15]{wy}{ry}
      \end{tikzinline}}
  \end{gathered}
\end{align*}
Because there are no intra-thread dependencies, the desired outcomes are allowed, as shown.

\myparagraph{Publication}
\ref{rf3}--\ref{rf4} and \ref{5b}--\ref{5c} ensure correct publication,
prohibiting stale reads:
\begin{gather}
  \taglabel{Pub1}
  \begin{gathered}
    \PW{x}{0}\SEMI %\PW{y}{0}\SEMI
    \PW{x}{1}\SEMI \PW[\mRA]{y}{1} \PAR \PR[\mRA]{y}{r}\SEMI \PR{x}{s}
    \\[-.4ex]
    \nonumber
    \hbox{\begin{tikzinline}[node distance=1.5em]
        \event{wx0}{\DW{x}{0}}{}
        \event{wx1}{\DW{x}{1}}{right=of wx0}
        \event{wy1}{\DWRel{y}{1}}{right=of wx1}
        \event{ry1}{\DRAcq{y}{1}}{right=2.5em of wy1}
        \event{rx0}{\DR{x}{0}}{right=of ry1}
        \sync{wx1}{wy1}
        \sync{ry1}{rx0}
        \rf{wy1}{ry1}
        \wk{wx0}{wx1}
      \end{tikzinline}}
  \end{gathered}
\end{gather}
This pomset is disallowed, since $(\DR x0)$ fails to satisfy \ref{rf4}:
$(\DW x0) \lt (\DW x1) \lt (\DR x0)$.  Attempting to satisfy this
requirement, one might order $(\DR x0)$ before $(\DW x1)$, but this would
create a cycle.

\myparagraph{Coherence}

Our model of coherence does not correspond to either Java or C11.  We have
chosen the model to validate \ref{CSE} (unlike C11 relaxed atomics) and the
local \drfsc{} theorem (unlike Java).

Since reads are not ordered by \ref{5b},
we {allow} the following unintuitive behavior. C11 includes read-read
coherence between relaxed atomics in order to forbid this:
\begin{gather*}
  \taglabel{Co2}
  \begin{gathered}
    \PW{x}{1}\SEMI \PW{x}{2}
    \PAR
    \PW{y}{x} \SEMI \PW{z}{x}
    \\[-1ex]
    \hbox{\begin{tikzinline}[node distance=1.5em]
        \event{a}{\DW{x}{1}}{}
        \event{b}{\DW{x}{2}}{right=of a}
        \wk{a}{b}
        \event{c}{\DR{x}{2}}{right=3em of b}
        \event{d}{\DW{y}{2}}{right=of c}
        \po{c}{d}
        \event{e}{\DR{x}{1}}{right=of d}
        \event{f}{\DW{z}{1}}{right=of e}
        \po{e}{f}
        \rf{b}{c}
        \rf[out=10,in=170]{a}{e}
      \end{tikzinline}}
  \end{gathered}
\end{gather*}
Here, the reader sees $2$ then $1$, although they are written in the reverse
order.
This behavior is allowed by Java in order to validate \ref{CSE} without requiring
aliasing analysis.

However, our model is more coherent than Java, which permits the following:
\begin{gather*}
  \taglabel{TC16}
  \begin{gathered}
    \PR{x}{r}\SEMI \PW{x}{1}
    \PAR
    \PR{x}{s}\SEMI \PW{x}{2}
    \\[-1ex]
    \hbox{\begin{tikzinline}[node distance=1.5em]
        \event{a1}{\DR{x}{2}}{}
        \event{a2}{\DW{x}{1}}{right=of a1}
        \wk{a1}{a2}
        \event{b1}{\DR{x}{1}}{right=3em of a2}
        \event{b2}{\DW{x}{2}}{right=of b1}
        \wk{b1}{b2}
        \rf{a2}{b1}
        \rf[out=-165,in=-15]{b2}{a1}
      \end{tikzinline}}
  \end{gathered}
\end{gather*}
We also forbid the \labeltext{following}{page:coherence2}, which Java allows:
\begin{gather*}
  \taglabel{Co3}
  \begin{gathered}
    \PW{x}{1}\SEMI \PW[\mRA]{y}{1}
    \PAR
    \PW{x}{2}\SEMI \PW[\mRA]{z}{1}
    \PAR
    \PR[\mRA]{z}{r} \SEMI 
    \PR[\mRA]{y}{r} \SEMI 
    \PR{x}{r} \SEMI 
    \PR{x}{r}
    \\[-1ex]
    \hbox{\begin{tikzinline}[node distance=1.5em]
        \event{a1}{\DW{x}{1}}{}
        \event{a2}{\DW[\mRA]{y}{1}}{right=of a1}
        \sync{a1}{a2}
        \event{b1}{\DW{x}{2}}{right=3em of a2}
        \event{b2}{\DW[\mRA]{\,z}{1}}{right=of b1}
        \sync{b1}{b2}
        \event{c1}{\DR[\mRA]{\,z}{1}}{right=3em of b2}
        \event{c2}{\DR[\mRA]{y}{1}}{right=of c1}
        \event{c3}{\DR{x}{2}}{right=of c2}
        \event{c4}{\DR{x}{1}}{right=of c3}
        \sync{c1}{c2}
        \sync{c2}{c3}
        \sync[out=20,in=160]{c2}{c4}
        \rf[out=8,in=172]{a2}{c2}
        \rf{b2}{c1}
        \wk[out=19,in=161]{a1}{b1}
        \wk[out=-172,in=-8]{c4}{b1}
      \end{tikzinline}}
  \end{gathered}
\end{gather*}
The order from $(\DR{x}{1})$ to $(\DW{x}{2})$ is required to fulfill
$(\DR{x}{1})$. The outcome is disallowed due to the cycle. %by fulfillment.
If this outcome were allowed, then racing writes would be visible, even after
a full synchronization; this would invalidate local reasoning about data
races (\textsection\ref{sec:sc}).

\myparagraph{MCA} We present a few examples that are hallmarks of \mca{}
architectures.  
\begin{scope}
  \allowdisplaybreaks
  \begin{gather*}
    \taglabel{MCA1}
    \begin{gathered}
      \IF{z}\THEN \PW{x}{0} \FI \SEMI \PW{x}{1}
      {\PAR}
      \IF{x}\THEN \PW{y}{0} \FI \SEMI \PW{y}{1}
      {\PAR}
      \IF{y}\THEN \PW{z}{0} \FI \SEMI \PW{z}{1}
      \\[-1ex]
      \hbox{\begin{tikzinline}[node distance=1.5em]
          \event{a1}{\DR{z}{1}}{}
          \event{a2}{\DW{x}{0}}{right=of a1}
          \po{a1}{a2}
          \event{a3}{\DW{x}{1}}{right=of a2}
          \wk{a2}{a3}
          \event{b1}{\DR{x}{1}}{right=3em of a3}
          \event{b2}{\DW{y}{0}}{right=of b1}
          \po{b1}{b2}
          \event{b3}{\DW{y}{1}}{right=of b2}
          \wk{b2}{b3}
          \event{c1}{\DR{y}{1}}{right=3em of b3}
          \event{c2}{\DW{z}{0}}{right=of c1}
          \po{c1}{c2}
          \event{c3}{\DW{z}{1}}{right=of c2}
          \wk{c2}{c3}
          \rf{a3}{b1}
          \rf{b3}{c1}
          \rf[out=173,in=7]{c3}{a1}  
        \end{tikzinline}}
    \end{gathered}
    \\[1ex]
    \taglabel{MCA2}
    \begin{gathered}
      \PW{x}{0}\SEMI \PW{x}{1}
      \PAR
      \PW{y}{x}
      \PAR
      \PR[\mRA]{y}{r} \SEMI \PR{x}{s}
      \\[-1ex]
      \hbox{\begin{tikzinline}[node distance=1.5em]
          \event{wx0}{\DW{x}{0}}{}
          \event{wx1}{\DW{x}{1}}{right=of wx0}
          \wk{wx0}{wx1}
          \event{rx1}{\DR{x}{1}}{right=3em of wx1}
          \event{wy1}{\DW{y}{1}}{right=of rx1}
          \po{rx1}{wy1}
          \event{ry1}{\DRAcq{y}{1}}{right=3em of wy1}
          \event{rx0}{\DR{x}{0}}{right=of ry1}
          \rf{wx1}{rx1}
          \rf{wy1}{ry1}
          \sync{ry1}{rx0}
          \wk[out=170,in=10]{rx0}{wx1}
        \end{tikzinline}}
    \end{gathered}
  \end{gather*}
\end{scope}
These candidate executions are invalid, due to cycles.
\ref{MCA1} is an example of \emph{write subsumption}
\cite[\textsection 3]{DBLP:journals/pacmpl/PulteFDFSS18}.
In \ref{MCA2}, $(\DW{x}{1})$ is delivered to the second thread, but not
the third; this is similar to the well know \iriw{} (Independent
Reads of Independent Writes) litmus test, which is also disallowed by \mca{}
architectures if the reads within each thread are ordered. 

If $y^\mRA$ is changed to $y^\mRLX$ in \ref{MCA2}, then there would be no order
from $(\DR[\mRLX]{y}{1})$ to $(\DR{x}{0})$, and the execution would be
allowed.  Since read-read dependencies do not appear in pomset order, the
execution would still be allowed if a control or address dependency were to
be introduced between the reads. See example \ref{addr2}
(\textsection\ref{sec:limits}) for further discussion.

\myparagraph{Internal Reads and Value Range Analysis}
The JMM causality test cases \citep{PughWebsite} are justified via
compiler analysis, possibly in collusion with the scheduler: If every 
observed value can be shown to satisfy a precondition, then the precondition
can be dropped.  For
example, \ref{TC1} determines that the following execution should be
allowed, as it is in our model:
\begin{gather*}
  \taglabel{TC1}
  \begin{gathered}
    \PW{x}{0} \SEMI
    (\PR{x}{r}\SEMI\IF{r\geq0}\THEN \PW{y}{1} \FI
    \PAR
    \PW{x}{y})
    \\[-1ex]
    \hbox{\begin{tikzinline}[node distance=1.5em]
        \event{wx0}{\DW{x}{0}}{}
        \event{rx1}{\DR{x}{1}}{right=3em of wx0}
        \event{wy1}{0\geq0\mid\DW{y}{1}}{right=of rx1}
        \event{ry1}{\DR{y}{1}}{right=3em of wy1}
        \event{wx1}{\DW{x}{1}}{right=of ry1}
        \po{ry1}{wx1}
        \rf[out=-168,in=-12]{wx1}{rx1}
        \rf{wy1}{ry1}
        \wk[out=10,in=170]{wx0}{wx1}
        \wk{wx0}{rx1}
      \end{tikzinline}}
  \end{gathered}
\end{gather*}
In this example, $(\DW{x}{0})$ ``fulfills'' the read of $x$ that is used in
the guard of the conditional.  This is possible when prefixing $(\DR{x}{1})$
performs the substitution $[x/r]$, but does not weaken the resulting
precondition $(x\geq0\mid\DW{y}{1})$.  Subsequently prefixing $(\DW{x}{0})$
substitutes $[0/x]$, resulting in the tautological precondition
$(0\geq0\mid\DW{y}{1})$.  Note that the execution does not have an action
$(\DR{x}{0})$.

Our semantics is robust with respect to
the introduction of concurrent writes, as in \ref{TC9}:
\begin{gather*}
  \taglabel{TC9}
  \begin{gathered}
    \PW{x}{0} \SEMI
    (\PR{x}{r}\SEMI\IF{r\geq0}\THEN \PW{y}{1} \FI
    \PAR
    \PW{x}{y}
    \PAR
    \PW{x}{-2})
    \\[-1ex]
    \hbox{\begin{tikzinline}[node distance=1.5em]
        \event{wx0}{\DW{x}{0}}{}
        \event{rx1}{\DR{x}{1}}{right=3em of wx0}
        \event{wy1}{0\geq0\mid\DW{y}{1}}{right=of rx1}
        \event{ry1}{\DR{y}{1}}{right=3em of wy1}
        \event{wx1}{\DW{x}{1}}{right=of ry1}
        \event{wx2}{\DW{x}{{-2}}}{right=3em of wx1}
        \po{ry1}{wx1}
        \rf[out=-168,in=-12]{wx1}{rx1}
        \rf{wy1}{ry1}
        \wk[out=10,in=170]{wx0}{wx1}
        \wk{wx0}{rx1}
        \wk{wx1}{wx2}
      \end{tikzinline}}
  \end{gathered}
\end{gather*}
The calculation of this pomset is unchanged from \ref{TC1}.

Examples such as \ref{TC9} present substantial difficulties in other models.
When thought of in terms of compiler optimizations, \ref{TC9} is justified by
global value analysis in collusion with the thread scheduler.  This execution
is disallowed by our event structure model \cite{DBLP:conf/lics/JeffreyR16}.
It is allowed by \citet{Pichon-Pharabod:2016:CSR:2837614.2837616}, at the
cost of introducing \emph{dead reads}.

The reasoning for \ref{TC2} is similar, but in this case no value is necessary to
satisfy the precondition:
\begin{gather*}
  \taglabel{TC2}
  \begin{gathered}
    \PR{x}{r}\SEMI
    \PR{x}{s}\SEMI
    \IF{r{=}s}\THEN \PW{y}{1}\FI
    \PAR
    \PW{x}{y}
    \\[-1ex]
    \nonumber
    \hbox{\begin{tikzinline}[node distance=1.5em]
        \event{a1}{\DR{x}{1}}{}
        \event{a2}{\DR{x}{1}}{right=of a1}
        \event{a3}{(x{=}x)\land(1{=}1)\mid\DW{y}{1}}{right=of a2}
        \event{b1}{\DR{y}{1}}{right=3em of a3}
        \event{b2}{\DW{x}{1}}{right=of b1}
        \rf{a3}{b1}
        \po{b1}{b2}
        \rf[out=169,in=11]{b2}{a2}
        \rf[out=169,in=11]{b2}{a1}
      \end{tikzinline}}
  \end{gathered}
\end{gather*}
Note that in 
\begin{math}
  \sem{\PR{x}{s}\SEMI
    \IF{r{=}s}\THEN \PW{y}{1}\FI},
\end{math}
the precondition on $(\DW{y}{1})$ must imply $(r{=}x \land r{=}1)$.  The
first is imposed by \ref{5a}, the second by \ref{4c}, ensuring that the two
reads see the same value.

Using \armeight{} terminology, these executions involve \emph{internal
  reads}, which are fulfilled by a sequentially preceding write.  Read
actions always generate an event that must be fulfilled, and therefore cannot
be ignored, even if they are unused.  This fact prevents internal reads from
ignoring concurrent blocking writes.
\begin{gather*}
  \taglabel{Internal1}
  \begin{gathered}
    \PW{x}{1} \SEMI
    \PW[\mRA]{a}{1} \SEMI
    \IF{z^\mRA}\THEN  \PW{y}{x} \FI
    \PAR
    \IF{a^\mRA}\THEN  \PW{x}{2}\SEMI \PW[\mRA]{z}{1} \FI
    \\
    \hbox{\begin{tikzinline}[node distance=1.2em]
        \event{a1}{\DW{x}{1}}{}
        \event{a2}{\DWRel{a}{1}}{right=of a1}
        \sync{a1}{a2}
        \event{b3}{\DRAcq{a}{1}}{below right=0em and 3em of a2}
        \rf{a2}{b3}
        \event{b4}{\DW{x}{2}}{right=of b3}
        \sync{b3}{b4}
        \event{b5}{\DWRel{z}{1}}{right=of b4}
        \sync{b4}{b5}
        \event{a6}{\DRAcq{b}{1}}{above right=0em and 3em of b5}
        \rf{b5}{a6}
        \event{a7}{\DR{x}{1}}{right=of a6}
        \sync{a6}{a7}
        \event{a8}{1{=}1\mid\DW{y}{1}}{right=of a7}
        \graypo{a7}{a8}
        \sync[out=-18,in=-162]{a6}{a8}
      \end{tikzinline}}
  \end{gathered}
\end{gather*}
Here $(\DR{x}{1})$ violates \ref{rf4}.  The precondition $(1{=}1)$ is
imposed by \ref{4c}.  The pomset becomes inconsistent if we change
$(\DR{x}{1})$ to $(\DR{x}{2})$, since the precondition would change to $(2{=}1)$.

Internal reads are notoriously difficult to get right.  Consider \cite[Ex 3.6]{DBLP:journals/pacmpl/PodkopaevLV19}:
\begin{gather*}
  \taglabel{Internal2}
  \begin{gathered}
    \PR{x}{\aReg}\SEMI
    \PW[\mRA]{y}{1}\SEMI
    \PR{y}{\bReg}\SEMI
    \PW{z}{\bReg}
    \PAR
    \PW{x}{z}
    \\[-1ex]
    \nonumber
    \hbox{\begin{tikzinline}[node distance=1.5em]
        \event{a1}{\DR{x}{1}}{}
        \event{a2}{\DWRel{y}{1}}{right=of a1}
        \sync{a1}{a2}
        \event{a3}{\DR{y}{1}}{right=of a2}
        \event{a4}{1{=}1\mid\DW{z}{1}}{right=of a3}
        \rf{a2}{a3}
        \event{b1}{\DR{z}{1}}{right=3em of a4}
        \event{b2}{\DW{x}{1}}{right=of b1}
        \po{b1}{b2}
        \rf{a4}{b1}
        \rf[out=170,in=10]{b2}{a1}
      \end{tikzinline}}
  \end{gathered}
\end{gather*}
This behavior is allowed in our model, as it is in \armeight.
Note that $\sem{\PW{z}{\bReg}}$ includes $(\bReg{=}1\mid \DW{z}{1})$.
Prepending a read,
$\sem{\PR{y}{\bReg} \SEMI \PW{z}{\bReg}}$ may update the precondition to
$(y{=}1\mid \DW{z}{1})$ without introducing order.
Further prepending
$(\DWRel{y}{1})$ results in $(1{=}1\mid \DW{z}{1})$.

Our model drops order into actions that depend on a read that can be
fulfilled {internally}, by a prefixed write.  This is natural consequence of
substitution.  The \armeight{} model has to jump through some hoops to ensure
that internal reads are handled correctly.  \armeight{} takes the symmetric
approach: rather than dropping order \emph{out of} an internal read,
\armeight{} drops the order \emph{into} it.  This difference complicates the
proof of correctness for implementing our semantics on \armeight{}
(\textsection\ref{sec:arm}).

\myparagraph{SC access}
\ref{5d} ensures that program order between SC operations is always
preserved.  Combined with \ref{rf3}--\ref{rf4}, this is
sufficient to establish that programs with only SC access have only SC
executions; for example, the executions of \ref{SB/LB} are banned when the
all actions are $\mSC$.  It is also immediate that SC
actions can be totally ordered, using any linearization of pomset order.
Just as SC access in \armeight{} is simplified by \mca, it is simplified here
by the global pomset order.

SC access is not as strict as volatile access in Java.  For example, our
model allows the following, since there is no order from
$(\DW[\mSC]{x}{2})$ to $(\DW{y}{1})$---recall that SC writes are \emph{releases}.
\begin{gather*}
  \taglabel{SC1}
  \begin{gathered}
    \PR{y}{r}\SEMI \PW[\mSC]{x}{1}\SEMI \PR{x}{s}
    \PAR
    \PW[\mSC]{x}{2} \SEMI \PW{y}{1}
    \\[-1ex]
    \hbox{\begin{tikzinline}[node distance=1.5em]
        \event{a}{\DR{y}{1}}{}
        \event{b}{\DW[\mSC]{x}{1}}{right=of a}
        \sync{a}{b}
        \event{bb}{\DR{x}{2}}{right=of b}
        \wk{b}{bb}
        \event{d}{\DW[\mSC]{x}{2}}{right=3em of bb}
        \event{e}{\DW{y}{1}}{right=of d}
        \rf{d}{bb}
        \rf[out=-170,in=-10]{e}{a}
        \wk[in=165,out=15]{b}{d}
      \end{tikzinline}}
  \end{gathered}
\end{gather*}
This execution is disallowed by
\citet[\textsection8.2]{Dolan:2018:BDR:3192366.3192421}, preventing them from
using \texttt{stlr} to implement volatile writes on \armeight{}. Our
implementation strategy does use \texttt{stlr} for SC writes, as is standard.
For further discussion, see examples \ref{past} and \ref{future} in
\textsection\ref{sec:sc}.

\citet[\textsection3.1]{DBLP:conf/pldi/WattPPBDFPG20} noticed a similar
difficulty in Javascript \cite[\textsection27]{ecma2019}:
\begin{gather*}
  \taglabel{SC2}
  \begin{gathered}
    \PW[\mSC]{x}{1} \SEMI \PR[\mSC]{y}{r}
    \PAR
    \PW[\mSC]{y}{1} \SEMI \PW[\mSC]{y}{2} \SEMI \PW{x}{2} \SEMI \PR[\mSC]{x}{s}
    \\[-1ex]
    \hbox{\begin{tikzinline}[node distance=1.5em]
        \event{a}{\DW[\mSC]{x}{1}}{}
        \event{b}{\DR[\mSC]{y}{1}}{right=of a}
        \event{c}{\DW[\mSC]{y}{1}}{right=3em of b}
        \event{d}{\DW[\mSC]{y}{2}}{right=of c}
        \event{e}{\DW{x}{2}}{right=of d}
        \event{f}{\DR[\mSC]{x}{1}}{right=of e}
        \sync{a}{b}
        \sync{c}{d}
        \sync[out=15,in=165]{d}{f}
        \rf{c}{b}
        \rf[out=-8,in=-172]{a}{f}
        \wk[in=10,out=170]{e}{a}
        \wk{e}{f}
      \end{tikzinline}}
  \end{gathered}
\end{gather*}
This execution is allowed both by our semantics and by \armeight{} (using
\texttt{stlr} for SC writes and \texttt{ldar} for SC reads).  However, it is
not allowed by Javascript 2019.  In Javascript, the rules relating SC and
relaxed access are subtle.  As result of these interactions, Javascript 2019
fails to satisfy \drfsc{} \cite{DBLP:journals/pacmpl/WattRP19}.  The rules
are even more complex in C11; see \ref{SC3} and \ref{SC4} in
\textsection\ref{sec:variants} for a discussion of SC fences in C11.
In our model, only \ref{5d} is required to explain SC access.

\subsection{Valid and Invalid Rewrites}
\label{sec:valid}

When $\sem{\aCmd} \supseteq \sem{\aCmd'}$, we say that $\aCmd'$ is a
\emph{valid transformation} of $\aCmd$.  In this subsection, we show the
validity of specific optimizations.  
Let $\free(\aCmd)$ be the set of locations and registers that occur in $\aCmd$.

The semantics validates many peephole optimizations.  Most apply only to
relaxed access.
\begin{align*}
  \taglabel{RR}
  \sem{\PR{\aLoc}{\aReg} \SEMI \PR{\bLoc}{\bReg}\SEMI\aCmd} &=
  \sem{\PR{\bLoc}{\bReg}\SEMI \PR{\aLoc}{\aReg}\SEMI\aCmd} &&\text{if } \aReg\neq\bReg
  \\
  \taglabel{WW}
  \sem{\aLoc \GETS \aExp \SEMI \bLoc  \GETS \bExp\SEMI\aCmd} &=
  \sem{\bLoc  \GETS \bExp\SEMI \aLoc \GETS \aExp\SEMI\aCmd} &&\text{if } \aLoc\neq\bLoc
  \\
  \taglabel{RW}
  \sem{\aLoc \GETS \aExp  \SEMI \PR{\bLoc}{\bReg} \SEMI\aCmd} &=
  \sem{\PR{\bLoc}{\bReg} \SEMI\aLoc \GETS \aExp\SEMI\aCmd} &&\text{if }
  \aLoc\neq\bLoc \textand \bReg\not\in\free(\aExp)%\disjoint{{\free(\aLoc \GETS \aExp)}}{{\free(\PR{\bLoc}{\bReg})}}
\end{align*}
\ref{5} imposes no order between events in \ref{RR}--\ref{RW}.  %Note that \ref{RR} allows aliasing.
Using augmentation closure, \ref{5} also validates roach-motel reorderings \cite{SevcikThesis}.  For
example, on read/write pairs:
\begin{align*}
  \tag{\textsc{roach1}}\label{AcqW}
  \sem{x^\amode \GETS \aExp \SEMI\PR{y}{\bReg} \SEMI\aCmd} &\supseteq
  \sem{\PR{y}{\bReg}  \SEMI x^\amode\GETS \aExp \SEMI \aCmd} 
  &&\text{if }
  \aLoc\neq\bLoc \textand \bReg\not\in\free(\aExp)%\disjoint{{\free(\aLoc \GETS \aExp)}}{{\free(\PR{\bLoc}{\bReg})}}
  \\
  \tag{\textsc{roach2}}\label{RelW}
  \sem{x \GETS \aExp \SEMI\PR[\amode]{y}{\bReg} \SEMI\aCmd} &\supseteq
  \sem{\PR[\amode]{y}{\bReg}  \SEMI x\GETS \aExp \SEMI \aCmd} 
  &&\text{if }
  \aLoc\neq\bLoc \textand \bReg\not\in\free(\aExp)%\disjoint{{\free(\aLoc \GETS \aExp)}}{{\free(\PR{\bLoc}{\bReg})}}
\end{align*}

Redundant load elimination \eqref{RL} follows
from \ref{1}, taking $\bEv\in\Event$, regardless of the access mode:
\begin{align*}
  \taglabel{RL}
  \sem{\PR[\amode]{\aLoc}{\aReg} \SEMI \PR[\amode]{\aLoc}{\bReg}\SEMI\aCmd} &\supseteq 
  \sem{\PR[\amode]{\aLoc}{\aReg} \SEMI \bReg  \GETS \aReg\SEMI\aCmd}
\end{align*}

Since \ref{5b} does not impose order between reads of the same
location, \ref{RR} can allow the possibility that $\aLoc=\bLoc$.  As a
result, read optimizations are not limited by the power of aliasing
analysis.  By composing \ref{RR} and \ref{RL}, we validate \ref{CSE}:
\begin{align*}
  \taglabel{CSE}
  \sem{r_1\GETS \aLoc \SEMI
    s\GETS \bLoc \SEMI  
    r_2\GETS \aLoc\SEMI\aCmd}
  \supseteq
  \sem{r_1\GETS \aLoc \SEMI     
    r_2\GETS r_1\SEMI
    s\GETS \bLoc \SEMI\aCmd}
  &&\textif \aReg_2\neq\bReg&&\hbox{}
\end{align*}

Many laws hold for the conditional, such as dead code elimination \eqref{DC}
and code lifting \eqref{CL}:
\begin{align*}
  \taglabel{DC}
  \sem{\IF{\aExp}\THEN\aCmd\ELSE\bCmd\FI} &=
  \sem{\aCmd}
  &&\textif \aExp \text{ is a tautology}
  \\
  \taglabel{CL}
  \sem{\IF{\aExp}\THEN\aCmd\ELSE\aCmd\FI} &\supseteq
  \sem{\aCmd}
\end{align*}
Code lifting also applies to program fragments inside a conditional.  For example:
\begin{align*}
  \sem{\IF{\aExp}\THEN x\GETS \bExp \SEMI\aCmd\ELSE x\GETS \bExp \SEMI\bCmd\FI} &\supseteq
  \sem{x\GETS \bExp \SEMI \IF{\aExp}\THEN\aCmd\ELSE\bCmd\FI}
\end{align*}
We discuss the inverse of \ref{CL} in \textsection\ref{sec:refine}.

As expected, %sequential and
parallel composition commutes with conditionals and declarations, and
conditionals and declarations commute with each other.  For example,
we have \emph{scope extrusion}~\cite{Milner:1999:CMS:329902}:
\begin{align*}
  \taglabel{SE}
  \sem{\aCmd\PAR \VAR\aLoc\SEMI\bCmd} &=
  \sem{\VAR\aLoc\SEMI(\aCmd\PAR\bCmd)}
  &&\text{if } \aLoc\not\in\free(\aCmd)
\end{align*}

\myparagraph{Invalid Rewrites}

The definition of location binding does not validate renaming of locations:
if $\aLoc\neq\bLoc$ then
$\sem{\VAR\bLoc\SEMI\aCmd}\neq\sem{\VAR\aLoc\SEMI\aCmd[\aLoc/\bLoc]}$, even
if $\aCmd$ does not mention~$\aLoc$.  This is consistent with support for
address calculation, which is required by realistic memory allocators.

\ref{Internal2} shows that---like most relaxed models---our model
fails to validate \emph{thread inlining}.  The given execution is impossible
if the first thread is split, as in
\begin{math}
  \sem{\PR{x}{\aReg}\SEMI
    \PW[\mRA]{y}{1}\PAR
    \PR{y}{\bReg}\SEMI
    \PW{z}{\bReg}
    \PAR
    \PW{x}{z}}.
\end{math}
The write in the first thread cannot discharge the precondition in the
second, now separate.

Some rewrites are invalid in a concurrent setting, such as
relevant read introduction:
\begin{displaymath}
  \sem{\PR{\aLoc}{\aReg} \SEMI \IF{\aReg {\neq} \aReg} \THEN \PW{y}{1} \FI}
  \not\supseteq
  \sem{\PR{\aLoc}{\aReg} \SEMI \PR{\aLoc}{\bReg}  \SEMI \IF{\aReg {\neq}\bReg} \THEN \PW{y}{1} \FI}
\end{displaymath}
Observationally, these are distinguished by the context %
\begin{math}
  \hole{} \PAR \PW{x}{1}\PAR \PW{x}{2}.
\end{math}

Write introduction is also invalid, even when duplicating an existing write:
\begin{displaymath}
  \sem{\PW{\aLoc}{1}} 
  \not\supseteq
  \sem{\PW{\aLoc}{1} \SEMI \PW{\aLoc}{1}}
\end{displaymath}
These are distinguished by the context:
\begin{math}
  \hole{} \PAR
  \PR{x}{r} \SEMI
  \PW{x}{2} \SEMI
  \PR{x}{s}\SEMI
  \IF{\aReg {=} \bReg} \THEN \PW{\cLoc}{1} \FI.
\end{math}

