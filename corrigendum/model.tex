\section{The Basic Model}
\label{sec:model}

The model adapts our previous work on microarchitecture \cite{2019-sp} to the
architectural level.  In this section, we define the model and use it to give
the semantics of a concurrent language.  The semantics given here is
simplified: as discussed in \textsection\ref{sec:refine}, it fails to
validate some important optimizations.  We give the full semantics in
\textsection\ref{sec:refine}.  In \textsection\ref{sec:variants}, we define
extensions that incorporate fences, read-modify-write operations, and address
computation.

The model is based on \emph{partially ordered
  multisets}~\cite{{GISCHER1988199,Plotkin:1997:TSP:266557.266600}}, where
events are labelled with reads and writes, and the partial order tracks
dependencies, which arise within threads due to local dependencies and
across threads due to reads and coherence.

For example the semantics of \eqref{OOTA?} %  (including explicit initialization
contains the
expected pomset (where $\DSTOP$ indicates termination):
\begin{gather}
  \label{OOTA?}
  \begin{gathered}
    x\GETS 0 \SEMI y\GETS 0\SEMI (y\GETS x
    \PAR
    r\GETS y\SEMI \IF{r}\THEN 
    x\GETS r\SEMI z\GETS r \ELSE x\GETS1 \FI)
    \\[-1ex]
    \hbox{\begin{tikzinline}[node distance=1.5em]
        \event{wy0}{\DW{y}{0}}{}
        \event{wx0}{\DW{x}{0}}{left=of wy0}
        \event{rx0}{\DR{x}{0}}{right=3em of wy0}
        \event{wy0a}{\DW{y}{0}}{right=of rx0}
        \event{ry0}{\DR{y}{0}}{right=3em of wy0a}
        \event{wx1}{\DW{x}{1}}{right=of ry0}
        \event{stop}{\DSTOP}{right=of wx1}
        \rf[out=19,in=161]{wx0}{rx0}
        \rf{wy0a}{ry0}
        \wk[out=15,in=165]{rx0}{wx1}
        \wk[out=19,in=161]{wy0}{wy0a}
        \po{rx0}{wy0a}
        \sync{wx1}{stop}
        \sync[out=20,in=160]{ry0}{stop}
      \end{tikzinline}}
  \end{gathered}
  \intertext{but also the unexpected one:}
  \label{unexpected}
  \smash[t]{\hbox{\begin{tikzinline}[node distance=1.5em]
        \event{wy0}{\DW{y}{0}}{}
        \event{wx0}{\DW{x}{0}}{left=of wy0}
        \event{rx1}{\DR{x}{1}}{right=3em of wy0}
        \event{wy1}{\DW{y}{1}}{right=of rx1}
        \event{ry1}{\DR{y}{1}}{right=3em of wy1}
        \event{wx1}{\DW{x}{1}}{right=of ry1}
        \event{wa1}{\DW{z}{1}}{right=of wx1}
        \event{stop}{\DSTOP}{right=of wa1}
        \rf[out=-165,in=-15]{wx1}{rx1}
        \rf{wy1}{ry1}
        \wk[out=12,in=168]{wx0}{wx1}
        \wk[out=19,in=161]{wy0}{wy1}
        \po{rx1}{wy1}
        \po[out=20,in=160]{ry1}{wa1}
        \sync[out=20,in=160]{wx1}{stop}
        \sync{wa1}{stop}
      \end{tikzinline}}}
\end{gather}
The interesting fact about these pomsets is that there is no control dependency between
reading $y$ and writing $x$, since the $(\DW{x}{1})$ event happens on both sides of the conditional.

An attempt to replicate this execution with \ref{OOTA!} fails,
since it introduces a cycle:
\begin{gather}
  \label{OOTA!}\tag{\textsc{oota3}}
  \begin{gathered}
    x\GETS 0 \SEMI y\GETS 0\SEMI (  y\GETS x
    \PAR
    r\GETS y\SEMI \IF{r}\THEN x\GETS r\SEMI z\GETS r \ELSE x\GETS2 \FI)
    \\[-1ex]
    \hbox{\begin{tikzinline}[node distance=1.5em]
        \event{wy0}{\DW{y}{0}}{}
        \event{wx0}{\DW{x}{0}}{left=of wy0}
        \event{rx1}{\DR{x}{1}}{right=3em of wy0}
        \event{wy1}{\DW{y}{1}}{right=of rx1}
        \event{ry1}{\DR{y}{1}}{right=3em of wy1}
        \event{wx1}{\DW{x}{1}}{right=of ry1}
        \event{wa1}{\DW{z}{1}}{right=of wx1}
        \event{stop}{\DSTOP}{right=of wa1}
        \rf[out=-165,in=-15]{wx1}{rx1}
        \rf{wy1}{ry1}
        \wk[out=12,in=168]{wx0}{wx1}
        \wk[out=19,in=161]{wy0}{wy1}
        \po{rx1}{wy1}
        \po[out=20,in=160]{ry1}{wa1}
        \sync[out=20,in=160]{wx1}{stop}
        \sync{wa1}{stop}
        \po{ry1}{wx1}
      \end{tikzinline}}
  \end{gathered}
\end{gather}
In this case, $(\DW{x}{1})$ only happens on one side of the conditional,
causing a control dependency from $(\DR{y}{1})$ to $(\DW{x}{1})$.
Similar cycles arise when attempting to write $z{=}1$ in \ref{OOTA3}--\ref{OOTA4}.

In the diagrams above, there is only one order---color plays no formal role.
We use color only to help the reader see where the order comes from:
\begin{itemize}
\item $(\DR{y}{1})\xpo(\DW{x}{1})$ is a \emph{local} requirement,
  relating reads to writes that depend on them.

\item $(\DW{x}{1})\xrf(\DR{x}{1})$ is a \emph{reads-from} requirement,
  relating writes to reads they fulfill.% : the read of $x$

\item $(\DW{x}{0})\xwk(\DW{x}{1})$ is a \emph{coherence} requirement,
  relating actions that touch the same location. % : the write of $1$

\item $(\DWRel{z}{1})\xsync(\DSTOP)$ is a \emph{fencing} requirement,
  involving fences and synchronization actions.
\end{itemize}

Each pomset event is labeled with a \emph{precondition}, in addition to the
actions shown above.  Whereas read actions represent an obligation that must be
\emph{fulfilled} concurrently by a matching write (Def.~\ref{def:rf}),
preconditions represent an obligation that must be \emph{satisfied}
sequentially via substitution (Def.~\ref{def:prefix}).

To get a sense of how preconditions are satisfied, let us consider the
evolution of the precondition of $(\DW{x}{1})$ during the calculation of
\eqref{unexpected}.
First consider the else-branch of the conditional: the semantics of
``${\IF{\lnot r}\THEN x\GETS1 \FI}$'' contains $(r{=}0 \mid \DW{x}{1})$,
indicating the control dependency.  The then-branch is more complex: the
semantics of ``${\IF{r}\THEN x\GETS r \FI}$'' contains
$(r{\neq}0 \land r{=}1 \mid \DW{x}{1})$ indicating both a control and a data
dependency.  This can be simplified to $(r{=}1 \mid \DW{x}{1})$.  Combining
the two branches of the conditional, we have
$(r{=}0 \lor r{=}1 \mid \DW{x}{1})$.  Prepending $r\GETS y$ substitutes
$[y/r]$, resulting in $(y{=}0 \lor y{=}1 \mid \DW{x}{1})$.  Prepending the
initializer $y\GETS 0$ substitutes $[0/y]$, resulting in
$(0{=}0 \lor 0{=}1 \mid \DW{x}{1})$.  This is a tautology, which we write as
$(\DW{x}{1})$.   We repeat this calculation in \textsection\ref{sec:pop},
after giving the formal definitions.

The same calculation fails for the write to $x$ in \ref{OOTA3}--\ref{OOTA4}.
In \ref{OOTA!}, for example, the writes to $x$ on either side of the
conditional cannot be combined, since one side writes $1$ and the other side
writes $2$.  Thus, the semantics of the conditional contains
$(r{=}1 \mid \DW{x}{1})$, rather than $(r{=}0 \lor r{=}1 \mid \DW{x}{1})$.
As we shall see (Def.~\ref{def:prefix}), in \ref{OOTA!} it is only possible to weaken this
precondition by introducing order from $(\DR{y}{1})$ to $(\DW{x}{1})$.

\subsection{Data Models}
\label{sec:data:models}
A \emph{data model} consists of:
\begin{itemize}
\item a set of \emph{values} $\Val$, ranged over by
  $\aVal$ and $\cVal$,
\item a set of \emph{registers} $\Reg$, ranged over by
  $\aReg$ and $\bReg$,
\item a set of \emph{expressions} $\Exp$, ranged over by
  $\aExp$, $\bExp$, and $\cExp$, % and $\dExp$,
\item a set of \emph{memory locations} $\Loc$, ranged over by $\aLoc$,
  $\bLoc$, $\cLoc$, and $\dLoc$
\item a set of \emph{actions} $\Act$, ranged over by $\aAct$, and %$\bAct$, and
\item a set of \emph{logical formulae} $\Formulae$, ranged over by
  $\aForm$ and $\bForm$.
\end{itemize}

Let $\aSub$ range over substitutions of the form
$[\bExp/\aReg]$ or $[\bExp/\aLoc]$.

We require that data models satisfy the following:
\begin{itemize}
\item values, registers, and memory locations are disjoint,
\item values include at least the constants $0$ and $1$,
\item expressions include at least registers and values,
\item expressions do \emph{not} include memory locations, % or the operator $\REF{\aExp}$,
\item formulae include at least %$\TRUE$, $\FALSE$, and
  equalities %of the form
  $(\aExp=\aVal)$, % and $(\REF{\aExp}=\aLoc)$,
\item formulae are closed under negation, conjunction, disjunction, and
  substitution, and
\item there is a relation $\vDash$ between formulae, with the expected semantics.
\end{itemize}

For the actions of a data model, we require that
there are partial functions $\rreads$ and
$\rwrites: \Act \fun (\Loc \times \Val)$, and there are subsets of $\Act$:
$\Acq$, $\Rel$, $\SC$, and $\Term$,
such that
$\fdom(\rreads)\cap\SC\subseteq\Acq$,
$\fdom(\rwrites)\cap\SC\subseteq\Rel$, and
$\Term\subseteq\Rel$.

\begin{itemize}    
\item We say that action $\aAct$ is a \emph{read} if
  $\aAct\in\fdom(\rreads)$. We say that $\aAct$ is a \emph{write} if
  $\aAct\in\fdom(\rwrites)$.  When $\rreads(\aAct) = (\aLoc,\aVal)$,
  we say that $\aAct$ \emph{reads} $\aVal$ \emph{from} $\aLoc$, and
  similarly for writes.  We say that $\aAct$ \emph{accesses} $\aLoc$ if it
  reads or writes $\aLoc$.

\item Actions in $\Acq$, $\Rel$ and $\SC$, are \emph{synchronization} and
  \emph{fencing} actions.  We say that $\aAct$ is an \emph{acquire} if
  $\aAct\in\Acq$, $\aAct$ is a \emph{release} if $\aAct\in\Rel$, and
  $\aAct$ is \emph{SC} if $\aAct\in\SC$.  We require that every SC read is
  an acquire, and every SC write is a release.

  
\item Actions in $\Term$ are \emph{termination} actions. We
  require %$\Term\subseteq\Rel$.
  that termination events are releasing.
\end{itemize}

Our example language includes actions of the form $(\DSTOP)$, which is a
\emph{termination}, $(\DR[\amode][\aThrd]{\aLoc}{\aVal})$, which
\emph{reads} $\aVal$ from $\aLoc$ and
$(\DW[\amode][\aThrd]{\aLoc}{\aVal})$, which \emph{writes} $\aVal$ to
$\aLoc$.
The \emph{access mode} $(\amode \!\!\BNFDEF\!\! \mRLX \!\BNFSEP\! \mRA \!\BNFSEP\! \mSC)$ is
either \emph{relaxed}, \emph{release-acquire}, or
\emph{sequentially-consistent}.
$\mRA$/$\mSC$ reads are acquires, and $\mRA$/$\mSC$ writes are releases.
We systematically elide the $\mRLX$-mode annotation, writing $(\DR{\aLoc}{\aVal})$
as shorthand for $(\DR[\mRLX]{\aLoc}{\aVal})$.

We do not explicitly include C11-style \emph{plain} access.  If \oota{}
executions are disallowed for C11 relaxed access, then C11 plain access
is the same as relaxed access for data race free programs; data races on
plain access result in undefined behavior \cite{undefined}.

Logical formulae include equations over locations and registers, such
$(\aLoc{=}1)$ and $(\aReg{=}\bReg{+}1)$.
We use expressions as formulae, coercing $\aExp$ to $\aExp\neq 0$.

Formulae are \emph{open}, in that
occurrences of register names and memory locations are subject to
substitutions of the form $\aForm[\bExp/\aReg]$ and $\aForm[\bExp/\aLoc]$.
Actions are not subject to substitution.

\begin{definition}
  \label{def:independent}
  We say $\aForm$ is \emph{independent of $\aLoc$} if, for every
  $\aVal$, $\aForm \vDash \aForm[\aVal/\aLoc] \vDash \aForm$; it is
  \emph{dependent} otherwise.
  We say $\aForm$ is \emph{location independent} if it is independent of
  every location.
  We say
  $\aForm$ \emph{implies} $\bForm$ if $\aForm\vDash\bForm$.
  We say
  $\aForm$ is a \emph{tautology} if $\TRUE\vDash\aForm$.
  We say
  $\aForm$ is \emph{unsatisfiable} if $\aForm\vDash\FALSE$.
\end{definition}

\subsection{Semantic Domain}
\label{sec:domain}
We model single {executions} as \emph{pomsets with
  preconditions}---\emph{pomsets}, for short---ranged over by $\aPS$.  These
extend the well-known model of partially ordered
multisets~\cite{GISCHER1988199} with formulae.

The pomset order relation, $\le$, represents \emph{causality} or
\emph{dependency}.  We visualize pomsets as directed graphs.  For example,
the semantics of
\begin{math}
  \VAR\aLoc\SEMI(
  x\GETS 0
  \SEMI
  x\GETS 1
  \PAR
  y\GETS x
  \SEMI
  z^\mRA\GETS 1
  )
\end{math}
includes:
\begin{tikzdisplay}[node distance=2em]
  \event{wx0}{\DW{x}{0}}{}
  \event{wx1}{\DW{x}{1}}{right=of wx0}
  \event{rx1}{\DR{x}{1}}{right=3em of wx1}
  \event{wy1}{\DW{y}{1}}{right=of rx1}
  \event{wz1}{\DWRel{z}{1}}{right=of wy1}
  \event{stop}{\DSTOP}{right=of wz1}
  \rf{wx1}{rx1}
  \wk{wx0}{wx1}
  \po{rx1}{wy1}
  \sync{wy1}{wz1}
  \sync{wz1}{stop}
\end{tikzdisplay}
We visualize order using arrows that indicate the reason that the order
arises.
$(\DW{x}{0})\xwk(\DW{x}{1})$ is a \emph{coherence} requirement: the write of $1$
must follow the write of $0$, since these are in \emph{conflict} and in program order.
$(\DW{x}{1})\xrf(\DR{x}{1})$ is a \emph{reads-from} requirement: the read of $x$
must be \emph{fulfilled} by a matching write.
$(\DR{x}{1})\xpo(\DW{y}{1})$ is a \emph{local dependency} requirement: the
write to $y$ is 
\emph{data dependent} on the read of $x$; \emph{control} and \emph{address} dependencies are
also local.
$(\DW{y}{1})\xsync(\DWRel{z}{1})$ and $(\DWRel{z}{1})\xsync(\DSTOP)$ are
\emph{fencing/synchronization} requirements: $(\DWRel{z}{1})$ and $(\DSTOP)$
are \emph{release} actions.

\begin{definition}
  \label{def:mmpomset}
  A \emph{pomset with preconditions} is a tuple
  $(\Event, {\le}, %{\leloc},
  \labeling)$, such that
  \begin{itemize}
  \item $\Event$ is a set of \emph{events},
  \item ${\le} \subseteq (\Event\times\Event)$ is a partial order, 
  \item $\labeling: \Event \fun (\Formulae\times\Act)$ is a \emph{labeling},
    from which we derive functions $\labelingForm:\Event\fun\Formulae$ and $\labelingAct:\Event\fun\Act$,
  \item $\bigwedge_{\aEv}\labelingForm(\aEv)$ is satisfiable
    \emph{(consistency)}, and
  \item if $\bEv\le\aEv$ then $\labelingForm(\aEv)$ implies
    $\labelingForm(\bEv)$ \emph{(causal strengthening)}.
  \end{itemize}
\end{definition}

We write pairs in $(\Formulae\times\Act)$ as $(\aForm \mid \aAct)$, eliding
$\aForm$ when it is a tautology.
We write $\bEv\lt\aEv$ when $\bEv\le\aEv$ and $\bEv\neq\aEv$.
We often elide explicit universal quantifiers in phrases such as ``\emph{for
  all $\bEv$ and $\aEv$ in $\Event$}, if $\bEv\le\aEv$ then
$\labelingForm(\aEv)$ implies $\labelingForm(\bEv)$.''  We lift terminology
and notation from actions and formulae to events.  For example, we may say
that $\aEv$ is a read when $\labelingAct(\aEv)$ is a read.

Since each pomset represents a single execution, we require that all
preconditions be \emph{consistent}.  For example, the semantics of
$\IF{r<0}\THEN y\GETS1 \ELSE z\GETS1\FI$ includes pomsets with either
$(r<0\mid\DW{y}{1})$ or $(r\geq0\mid\DW{z}{1})$, but not with both, since
$(r<0\land r\geq0)$ is unsatisfiable.

Preconditions are linked to pomset order via \emph{causal strengthening},
which requires that formulae do not weaken over time, as measured by $\le$.
Example \ref{strengthening} (\textsection\ref{sec:variants}) requires causal
strengthening. % on page \pageref{page:strengthening}.

Let $\aPSS$ range over sets of pomsets.  The semantics of a program is given
as a set of pomsets $\aPSS$ that is closed with respect to \emph{downsets}
(which are similar to prefixes for strings), to \emph{augmentation} (which
may add order), and to \emph{implication} (which may strengthen formulae).
\begin{definition}
  \label{def:downset}
  $\aPS'$ is a \emph{downset} of $\aPS$ if
  $\Event \supseteq \Event' \supseteq \{ \bEv \in \Event \mid
  \exists\aEv\in\Event'.\; \bEv\le\aEv\}$, ${\le'}={\le}\restrict{\Event'}$,
  and ${\labeling'}={\labeling}\restrict{\Event'}$.
  
  
  
  We say that
  $\aPS'$ is an \emph{augment} of $\aPS$ if $\Event'=\Event$,
  ${\labeling'}={\labeling}$, and ${\le'}\supseteq{\le}$. 
  
  We say that
  $\aPS'$ \emph{implies} $\aPS$ if $\Event'=\Event$, ${\le'}={\le}$,
  $\labelingAct'=\labelingAct$, and $\labelingForm'(\aEv)$
  implies $\labelingForm(\aEv)$. 
\end{definition}

In examples, we draw pomsets that are \emph{augmentation\hyp{}minimal} and
\emph{implication\hyp{}minimal}.

A pomset is \emph{completed} if it contains a unique termination action,
ordered after all other events.  Note that, by causal strengthening, the
precondition of the termination event of a completed pomset must imply the
preconditions of all other events.

The semantics of a program includes only completed pomsets and their
downsets.  We systematically elide the termination event in diagrams, unless
it is relevant to the discussion.

\subsection{Example Language}
We define the language by prefixing individual reads and writes.  % In
\begin{align*}
  \aCmd,\,\bCmd
  \BNFDEF& \SKIP
  \BNFSEP \aReg\GETS\aExp\SEMI \aCmd
  \BNFSEP \aReg\GETS\aLoc^{\amode}\SEMI \aCmd 
  \BNFSEP \aLoc^{\amode}\GETS\aExp\SEMI \aCmd
  \\[-.5ex]
  \BNFSEP&\aCmd \PAR[\aThrd][\bThrd] \bCmd
  \BNFSEP \VAR\aLoc\SEMI \aCmd
  \BNFSEP \IF{\aExp} \THEN \aCmd \ELSE \bCmd \FI
\end{align*}

We write
$\IF{\aExp} \THEN \aCmd \FI$ as shorthand for
$\IF{\aExp} \THEN \aCmd\ELSE \SKIP\FI$ and
$\IF{\aExp} \THEN \aCmd^1 \ELSE \aCmd^2 \FI \SEMI \bCmd$ as shorthand for
$\IF{\aExp} \THEN \aCmd^1 \SEMI \bCmd\ELSE \aCmd^2 \SEMI \bCmd\FI$.

The semantic function $\sem[\aThrd]{-}$ takes a command and
yields a set of pomsets.

\subsection{Composition and Concurrency}
\label{sec:par}
Parallel composition is roughly pomset union, allowing that some events may
\emph{coalesce}, with the resulting precondition being the disjunction of the
precondition taken from the two sides.  As in our previous work
\cite{2019-sp}, composition is used to define concurrency and conditionals.
Here, we also use it to define address calculation
(\textsection\ref{sec:variants}).  % In both cases, the use of

\begin{definition}
  \label{def:par}
  Let $\aPS' \in (\aPSS^1 \parallel \aPSS^2)$
  when there are $\aPS^1 \in \aPSS^1$ and $\aPS^2 \in \aPSS^2$
  such that
  $\aPS^1$ is completed exactly when $\aPS^2$ is completed,
  there is at most one termination in $\Event'$,
  $\Event' = \Event^1 \cup \Event^2$,
  ${\le'}\supseteq{\le^1}\cup{\le^2}$,
  and for all $\aEv\in\Event'$,
  either:
  \begin{align*}
    \aEv \not\in \Event^2,\; \labelingAct'(\aEv) &= \labelingAct^1(\aEv) \textand \labelingForm'(\aEv) \textimplies \labelingForm^1(\aEv),
    \\[-1ex]
    \aEv \not\in \Event^1,\; \labelingAct'(\aEv) &= \labelingAct^2(\aEv) \textand \labelingForm'(\aEv) \textimplies \labelingForm^2(\aEv),
    \\[-1ex]
    \DSTOP\neq\labelingAct'(\aEv) = \labelingAct^1(\aEv) &= \labelingAct^2(\aEv) \textand \labelingForm'(\aEv) \textimplies \labelingForm^1(\aEv) \lor \labelingForm^2(\aEv), \textor
    \\[-1ex]
    \DSTOP=\labelingAct'(\aEv) = \labelingAct^1(\aEv) &= \labelingAct^2(\aEv) \textand \labelingForm'(\aEv) \textimplies \labelingForm^1(\aEv) \land \labelingForm^2(\aEv).
  \end{align*}
  We then define:
  \begin{align*}
    \sem[\aThrd]{\aCmd \PAR[\bThrd][\bThrd'] \bCmd} & \eqdef
    \sem[\bThrd]{\aCmd} \parallel \sem[\bThrd']{\bCmd} 
  \end{align*}
\end{definition}
Consider the following pomsets:
\begin{align*}
  \begin{gathered}
    \IF{r<0}\THEN y\GETS1\FI
    \\
    \hbox{\begin{tikzinline}[node distance=1.5em]
        \event{wx1}{r<0\mid\DW{y}{1}}{}
      \end{tikzinline}}
  \end{gathered}
  &&
  \begin{gathered}
    \IF{r\geq0}\THEN y\GETS1\FI
    \\
    \hbox{\begin{tikzinline}[node distance=1.5em]
        \event{wx2}{r\geq0\mid\DW{y}{1}}{right=of wx1}
      \end{tikzinline}}
  \end{gathered}
\end{align*}
The parallel composition of these programs includes pomsets with either one
of the two events, but not both.  Including both would violate consistency
(Def.~\ref{def:mmpomset}).  However, Definition \ref{def:par} allows events
with the same label to \emph{coalesce}, taking the disjunction of their
preconditions.  Thus, the semantics of the combined program also includes
\begin{math}
  (r<0\lor r\geq0\mid\DW{y}{1})
\end{math}.
As discussed in the next subsection, coalesced events inherit order from both sides.

The definition requires that if $\aPS'\in(\aPS^1 \parallel \aPS^2)$ is completed,
then both $\aPS^1$ and $\aPS^2$ are completed, and further, the termination events
\emph{must} coalesce in $\aPS'$.

\subsection{Conditional, Register Assignment, and Skip}
Conditional execution is defined using parallel composition and
\emph{filtering}: $(\aForm \guard \aPSS)$ selects the subset of pomsets in
$\aPSS$ whose preconditions all imply $\aForm$.  Register assignment is
defined using substitution: $(\aPSS\aSub)$ performs the substitution $\aSub$
on every \emph{formula} in $\aPSS$---\emph{actions} are not subject to
substitution.  The semantics of $\SKIP$ is defined using singleton pomsets
with label $\DSTOP$.
\begin{definition}
  Let $\aPS\in(\aForm \guard \aPSS)$ when $\aPS\in\aPSS$ and 
  $\labelingForm(\aEv)$ implies $\aForm$.
  Let $\aPS'\in(\aPSS\aSub)$ when there is
  $\aPS\in\aPSS$ such that $\Event' = \Event$, ${\le'} = {\le}$,
  $\labelingAct' = \labelingAct$, and
  $\labelingForm'(\aEv) = \labelingForm(\aEv)\aSub$.
  \ \ Let $\aPS\in\stopPS$ when $\Event$ has one element labelled with action
  $\DSTOP$\!\!.
  We then define:
  \begin{align*}
    \sem[\aThrd]{\IF{\aExp} \THEN \aCmd \ELSE \bCmd \FI} & \eqdef
    \bigl(\aExp \guard \sem[\aThrd]{\aCmd}\bigr) \parallel \bigl(\lnot\aExp \guard \sem[\aThrd]{\bCmd}\bigr) 
    &
    \sem[\aThrd]{\aReg\GETS\aExp\SEMI \aCmd} & \eqdef
    \sem[\aThrd]{\aCmd}[\aExp/\aReg] 
    &  
    \sem[\aThrd]{\SKIP} & \eqdef
    \stopPS
  \end{align*}
\end{definition}
Substitution applies to formulae, not actions.  For example,
$(x{=}1\mid\DW{x}{2})[0/x]=(0{=}1\mid\DW{x}{2})$. % and $(r{=}1\mid\DW{x}{2})[x/r]=(x{=}1\mid\DW{x}{2})$.

As an example of the conditional, consider the following fragments:
\begin{align}
  \label{ex:cond}
  \begin{gathered}
    \IF{s}\THEN x\GETS 1 \SEMI x\GETS 2\FI
    \\
    \hbox{\begin{tikzinline}[node distance=1.5em]
        \event{a}{s \mathbin{\mid}\DW{x}{1}}{}
        \event{b}{s \mathbin{\mid}\DW{x}{2}}{right=of a}
        \wk{a}{b}
        \event{c}{s \mathbin{\mid}\DSTOP}{right=of b}
        \sync{b}{c}
      \end{tikzinline}}
  \end{gathered}
  &&
  \begin{gathered}
    \IF{\lnot s}\THEN x\GETS 1 \SEMI x\GETS 3\FI
    \\
    \hbox{\begin{tikzinline}[node distance=1.5em]
        \event{a}{\lnot s \mathbin{\mid}\DW{x}{1}}{}
        \event{b}{\lnot s \mathbin{\mid}\DW{x}{3}}{right=of a}
        \wk{a}{b}
        \event{c}{\lnot s \mathbin{\mid}\DSTOP}{right=of b}
        \sync{b}{c}
      \end{tikzinline}}
  \end{gathered}
\end{align}
Putting these together, we can coalesce the $(\DW{x}{1})$ events:
\begin{gather*}
  \aCmd_{\mathsf{cond}} = \IF{s}\THEN x\GETS 1 \SEMI x\GETS 2 \ELSE x\GETS 1 \SEMI x\GETS 3 \FI
  \\
  \hbox{\begin{tikzinline}[node distance=1.5em]
      \event{a}{\DW{x}{1}}{}
      \event{b}{s \mathbin{\mid}\DW{x}{2}}{right=of a}
      \wk{a}{b}
      \event{c}{s \mathbin{\mid}\DSTOP}{right=of b}
      \sync{b}{c}
    \end{tikzinline}}    
  \qquad\qquad\qquad\quad
  \hbox{\begin{tikzinline}[node distance=1.5em]
      \event{a}{\DW{x}{1}}{}
      \event{b}{\lnot s \mathbin{\mid}\DW{x}{3}}{right=of a}
      \wk{a}{b}
      \event{c}{\lnot s \mathbin{\mid}\DSTOP}{right=of b}
      \sync{b}{c}
    \end{tikzinline}}    
\end{gather*}
Let us focus on the left pomset above.  It is derived from the composition:
\begin{gather*}
  \hbox{\begin{tikzinline}[node distance=1.5em]
      \event{a}{s \mathbin{\mid}\DW{x}{1}}{}
      \event{b}{s \mathbin{\mid}\DW{x}{2}}{right=of a}
      \wk{a}{b}
      \event{c}{s \mathbin{\mid}\DSTOP}{right=of b}
      \sync{b}{c}
    \end{tikzinline}}
  \qquad\parallel\qquad
  \hbox{\begin{tikzinline}[node distance=1.5em]
      \event{a}{\lnot s \mathbin{\mid}\DW{x}{1}}{}
    \end{tikzinline}}    
\end{gather*}
The existence of the singleton $(\lnot s \mathbin{\mid}\DW{x}{1})$ is
guaranteed by downset closure on the right pomset in \eqref{ex:cond}.
Consistency prevents any pomset in $\sem{\aCmd_{\mathsf{cond}}}$ from
containing both $(\DW{x}{2})$ and $(\DW{x}{3})$.

Note that the definitions of consistency, downset, and composition prevent
the coalescing of $(\DW{y}{3})$ in
\begin{math}
  \sem{\IF{s}\THEN y\GETS 1 \SEMI y\GETS 3 \ELSE y\GETS 2 \SEMI y\GETS 3 \FI}.
\end{math}
Any pomset that included $(\DW{y}{3})$ would need to contain both $(s
\mathbin{\mid}\DW{y}{1})$ and $(\lnot s \mathbin{\mid}\DW{y}{2})$, which
violates consistency.

\subsection{Prefixing} % for Trivial Expressions.}
\label{sec:pop}

Prefixing adds a new read or write event to the beginning of a pomset.  The
candidate definition given here fails to validate some compiler
optimizations.  We give the final definition in \textsection\ref{sec:refine}.

Maintaining downset closure complicates the definition in uninteresting ways;
therefore, we perform this closure explicitly.  Let
$\PRE{\aPSS}=\{\aPS'\mid\aPS'$ is a downset of some $\aPS \in \aPSS\}$.

\begin{candidate}
  \label{cand:ord}
  \label{def:pre-sc}
  \label{def:prefix}
  Let $\aAct \prefix \aPSS$ be the set
  $\PRE{\aPSS'}$ %$\aPSS'$ 
  where
  $\aPS'\in\aPSS'$ when 
  there is $\aPS\in\aPSS$ such that
  \hbox{{\labeltextsc[P1]{(P1)}{1}} $\Event' = \Event \cup \{\bEv\}$,}
  {\labeltextsc[P2]{(P2)}{2}}  ${\le'}\supseteq{\le}$, % if $\bEv \le \aEv$ then $\bEv \le' \aEv$,
  {\labeltextsc[P3]{(P3a)}{3}}  $\labelingAct'(\aEv) = \labelingAct(\aEv)$, 
  \labeltextsc[P3b]{(P3b)}{3b} $\labelingAct'(\bEv) = \aAct$,
  \begin{enumerate}
  \item[{\labeltextsc[P5b]{(P5b)}{5b}}]\labeltextsc[P5]{}{5}%
    if $\bEv$ and $\aEv$ are \external actions in conflict then $\bEv\le'\aEv$, %$\bEv \gtN' \aEv$,
  \item[{\labeltextsc[P5c]{(P5c)}{5c}}]
    if $\bEv$ is an acquire or $\aEv$ is a release then $\bEv \le' \aEv$, and
  \item[{\labeltextsc[P5d]{(P5d)}{5d}}]
    if $\bEv$ is an SC write and $\aEv$ is an SC read then $\bEv \le' \aEv$.
  \end{enumerate}
  For $(\aReg \mid \DRmode\aLoc\aVal) \prefix \aPSS$ we additionally require
  \begin{enumerate}
  \item[{\labeltextsc[P4c]{(P4c)}{4c}}]{\labeltextsc[P4]{}{4}}%
    $\labelingForm'(\aEv)$ implies $\uReg{\aEv}{=}\aVal\limplies\labelingForm(\aEv)[\uReg{\aEv}/\aReg]$, and
  \item[{\labeltextsc[P5a]{(P5a)}{5a}}]
    if $\aEv$ writes then either $\bEv\le'\aEv$ or $\labelingForm'(\aEv)$
    implies $\uReg{\aEv}{=}\aLoc\limplies\labelingForm(\aEv)[\uReg{\aEv}/\aReg]$.
  \end{enumerate}
  For $(\aForm \mid \DWmode\aLoc\aVal) \prefix \aPSS$ we additionally require
  \begin{enumerate}
  \item[{\labeltextsc[P4a]{(P4a)}{4a}}]{\labeltextsc[P4]{}{4}}%
    if $\bEv\in\Event$ then $\labelingForm'(\bEv)$ implies
    $\aForm\lor\labelingForm(\bEv)$, otherwise $\labelingForm'(\bEv)$ implies $\aForm$,
  \item[{\labeltextsc[P4b]{(P4b)}{4b}}]
    either $\aEv=\bEv$ or
    $\labelingForm'(\aEv)$ implies $\labelingForm(\aEv)$.
  \end{enumerate}
  We then define:
  \begin{gather*}
    \begin{aligned}[t]
      \sem{\aReg\GETS\aLoc^\amode\SEMI \aCmd} & \eqdef \textstyle\bigcup_\aVal\;
      (\aReg \mid \DRmode\aLoc\aVal) \prefix \sem{\aCmd} [\aLoc/\aReg]
      \\
      \sem{\aLoc^\amode\GETS\aExp\SEMI \aCmd} & \eqdef
      \textstyle\bigcup_\aVal\; (\aExp=\aVal \mid \DWmode\aLoc\aVal)
      \prefix \sem{\aCmd}[\aExp/\aLoc]
    \end{aligned}
  \end{gather*}
\end{candidate}

The main work happens in the definition of prefixing ($\prefix$).  \ref{1}
introduces a ``new'' event.  Coalescing is allowed, as it was for composition
(\textsection\ref{sec:par}): the new event $\bEv$ may coalesce with an
``old'' one in $\Event$.  \ref{2} ensures that no order is removed from old
events.  \ref{3} specifies the actions labelling the events.  \ref{4}
specifies the preconditions (along with \ref{5a}).  \ref{5} specifies the
preserved program order.

The semantics of read and write introduce a separate pomset for each possible
value read or written.  The value is fixed in each pomset.  For writes, the
dependence on $\aExp$ appears in the precondition $(\aExp=\aVal)$.  This
precondition must be satisfied using the substitutions in the semantic rules
and \ref{4c}.  Whereas writes introduce a precondition that must be satisfied
sequentially, reads introduce a fulfillment requirement
(\textsection\ref{sec:fulfillment}) that must be satisfied concurrently.

The possibilities for fulfillment are limited by the program
order that is preserved by \ref{5}.  Requirement \ref{5a} connects
the sequential semantics of write to the concurrent semantics of read; it %\ref{5a}
requires order from read to write when the precondition of the write
is \emph{weakened} using \ref{4c}. % (when $\bEv$ is not a read, \ref{4b} trivially implies \ref{5a}).
The other requirements in \ref{5} are standard.  \ref{5b} captures the
extended coherence requirement on sequential actions that touch the same location.
\ref{5c} imposes the order required by acquire and release actions.  \ref{5d}
imposes the additional order required by SC actions.  (Recall that SC reads
are acquires, and that SC writes and termination actions are releases.)

We explain the concurrent semantics in the next section using standard litmus
tests.  In  this subsection, we focus on the sequential
semantics.
Let us revisit %the unexpected execution
\eqref{unexpected}: % of \eqref{OOTA?}:% ,
\begin{gather*}
  \tag{\ref{unexpected}}
  \begin{gathered}
    x\GETS 0 \SEMI y\GETS 0\SEMI (y\GETS x
    \PAR
    r\GETS y\SEMI \IF{r}\THEN 
    x\GETS r\SEMI z\GETS r \ELSE x\GETS1 \FI)
    \\
    \smash[t]{\hbox{\begin{tikzinline}[node distance=1.5em]
          \event{wy0}{\DW{y}{0}}{}
          \event{wx0}{\DW{x}{0}}{left=of wy0}
          \event{rx1}{\DR{x}{1}}{right=3em of wy0}
          \event{wy1}{\DW{y}{1}}{right=of rx1}
          \event{ry1}{\DR{y}{1}}{right=3em of wy1}
          \event{wx1}{\DW{x}{1}}{right=of ry1}
          \event{wa1}{\DW{z}{1}}{right=of wx1}
          \event{stop}{\DSTOP}{right=of wa1}
          \rf[out=-165,in=-15]{wx1}{rx1}
          \rf{wy1}{ry1}
          \wk[out=12,in=168]{wx0}{wx1}
          \wk[out=19,in=161]{wy0}{wy1}
          \po{rx1}{wy1}
          \po[out=20,in=160]{ry1}{wa1}
          \sync[out=20,in=160]{wx1}{stop}
          \sync{wa1}{stop}
        \end{tikzinline}}}
  \end{gathered}
\end{gather*}
It is immediate from the definition that $\sem{x\GETS r\SEMI z\GETS r}$
contains pomset candidates such as:
\begin{align*}
  \hbox{\begin{tikzinline}[node distance=.5em]
      \event{wx}{r{=}0\mid\DW{x}{0}}{}
      \event{wz}{r{=}0\mid\DW{z}{0}}{right=of wx}
    \end{tikzinline}}  
  &&
  \hbox{\begin{tikzinline}[node distance=.5em]
      \event{wx}{r{=}1\mid\DW{x}{1}}{}
      \event{wz}{r{=}1\mid\DW{z}{1}}{right=of wx}
    \end{tikzinline}}  
  &&
  \hbox{\begin{tikzinline}[node distance=.5em]
      \event{wx}{r{=}0\mid\DW{x}{0}}{}
      \event{wz}{r{=}1\mid\DW{z}{1}}{right=of wx}
    \end{tikzinline}}  
\end{align*}
Consistency (Def.~\ref{def:mmpomset}) rules out the rightmost pomset, since
the conjunction of preconditions is unsatisfiable.
No order is required between the writes.
Combining the middle pomset with $(r{=}0\mid \DW{x}{1})$,
the conditional
\begin{math}
  \sem{\IF{r}\THEN x\GETS r\SEMI z\GETS r \ELSE x\GETS1 \FI}
\end{math}
contains:
\begin{gather*}
  \hbox{\begin{tikzinline}[node distance=1em]
      \event{wx}{r{=}1\lor r{=}0\mid\DW{x}{1}}{}
      \event{wz}{r{=}1\mid\DW{z}{1}}{right=of wx}
    \end{tikzinline}}  
\end{gather*}

When prefixing ``$r\GETS y$,'' we first substitute $[y/r]$, resulting in:
\begin{gather*}
  \hbox{\begin{tikzinline}[node distance=1em]
      \event{wx}{y{=}1\lor y{=}0\mid\DW{x}{1}}{}
      \event{wz}{y{=}1\mid\DW{z}{1}}{right=of wx}
    \end{tikzinline}}  
\end{gather*}
Adding the read action,
\begin{math}
  \sem{r\GETS y\SEMI \IF{r}\THEN x\GETS r\SEMI z\GETS r \ELSE x\GETS1 \FI} 
\end{math}
contains:
\begin{gather*}
  \hbox{\begin{tikzinline}[node distance=1em]
      \event{wx}{(y{=}1\lor y{=}0)\land(1{=}1\lor 1{=}0)\mid\DW{x}{1}}{}
      \event{wz}{(y{=}1) \land (1{=}1)\mid\DW{z}{1}}{right=of wx}
      \event{ry}{\DR{y}{1}}{left=of wx}
    \end{tikzinline}}
\end{gather*}
The second conjunct in each event is required by \ref{4c}. \ref{4c} also prevents inconsistent reads such as:
\begin{gather*}
  \hbox{\begin{tikzinline}[node distance=1em]
      \event{wx}{(y{=}1\lor y{=}0) \land(2{=}1\lor 2{=}0)\mid\DW{x}{1}}{}
      \event{wz}{(y{=}1) \land (1{=}1)\mid\DW{z}{1}}{right=of wx}
      \event{ry}{\DR{y}{2}}{left=of wx}
    \end{tikzinline}}
\end{gather*}

\ref{4c} also allows predicates to weaken, in which case \ref{5a} requires order:
\begin{gather*}
  \smash[t]{\hbox{\begin{tikzinline}[node distance=1em]
        \event{wx}{(y{=}1\lor y{=}0)\land(1{=}1\lor 1{=}0)\mid\DW{x}{1}}{}
        \event{wz}{(1{=}1) \land (1{=}1)\mid\DW{z}{1}}{right=of wx}
        \event{ry}{\DR{y}{1}}{left=of wx}
        \po[out=10,in=170]{ry}{wz}
      \end{tikzinline}}}
\end{gather*}
Adding the write,
\begin{math}
  \sem{y\GETS 0\SEMI r\GETS y\SEMI \IF{r}\THEN x\GETS r\SEMI z\GETS r \ELSE x\GETS1 \FI} 
\end{math}
substitutes $[0/y]$, resulting in: 
\begin{gather*}
  \smash[t]{\hbox{\begin{tikzinline}[node distance=1em]
        \event{wx}{(0{=}1\lor 0{=}0)\land(1{=}1\lor 1{=}0)\mid\DW{x}{1}}{}
        \event{wz}{(1{=}1) \land (1{=}1)\mid\DW{z}{1}}{right=of wx}
        \event{ry}{\DR{y}{1}}{left=of wx}
        \event{wy}{0{=}0\mid\DW{y}{0}}{left=of ry}
        \po[out=10,in=170]{ry}{wz}
        \wk{wy}{ry}
      \end{tikzinline}}}
\end{gather*}
Simplifying the tautologies, we have:
\begin{gather*}
  \smash[t]{\hbox{\begin{tikzinline}[node distance=1em]
        \event{wx}{\DW{x}{1}}{}
        \event{wz}{\DW{z}{1}}{right=of wx}
        \event{ry}{\DR{y}{1}}{left=of wx}
        \event{wy}{\DW{y}{0}}{left=of ry}
        \po[out=25,in=155]{ry}{wz}
        \wk{wy}{ry}
      \end{tikzinline}}}
\end{gather*}

As discussed in the introduction of \textsection\ref{sec:intro},  this
reasoning fails in \ref{OOTA3}--\ref{OOTA4}.

For \ref{OOTA3}, $\sem{x\GETS r\SEMI z\GETS r}$ contains:
\begin{align*}
  \hbox{\begin{tikzinline}[node distance=1em]
      \event{wx}{r{=}1\mid\DW{x}{1}}{}
      \event{wz}{r{=}1\mid\DW{z}{1}}{right=of wx}
    \end{tikzinline}}  
\end{align*}
It is only possible to satisfy the precondition $(r{=}1)$
using \ref{4c} when prefixing $(\DR{y}{1})$. This forces a dependency from
read to write via \ref{5a}.  \ref{cyc} is similar.

For \ref{OOTA!}, %which writes $2$ in the \texttt{else} branch, for example, 
\begin{math}
  \sem{\IF{r}\THEN x\GETS r\SEMI z\GETS r \ELSE x\GETS2 \FI}
\end{math}
contains:
\begin{align*}
  \hbox{\begin{tikzinline}[node distance=1em]
      \event{wx}{r{=}1\mid\DW{x}{1}}{}
      \event{wz}{r{=}1\mid\DW{z}{1}}{right=of wx}
    \end{tikzinline}}  
  &&
  \hbox{\begin{tikzinline}[node distance=1em]
      \event{wx}{r{=}2\lor r{=}0\mid\DW{x}{2}}{}
      \event{wz}{r{=}2\mid\DW{z}{2}}{right=of wx}
    \end{tikzinline}}  
\end{align*}
Coalescing is possible when writing $2$, as on the right above, but it is not
possible when writing $1$, as on the left, since $(\DW{x}{1})$ and
$(\DW{x}{2})$ are different actions.
When writing $1$, this is no different than \ref{OOTA3}, and thus \ref{5a}
forces a dependency from the read of $y$ to the writes.

For \ref{OOTA4}, %the semantics of 
\begin{math}
  \sem{\IF{b}\THEN x\GETS r\SEMI z\GETS r \ELSE x\GETS1 \FI}
\end{math}
contains:
\begin{align*}
  \hbox{\begin{tikzinline}[node distance=1em]
      \event{wx}{\smash{\bigl((r{=}1\land b{\neq}0)\lor b{=}0\bigr)\land \bigl((r{=}1\land 1{\neq}0)\lor 1{=}0\bigr)}\mid\DW{x}{1}}{}
      \event{wz}{(r{=}1\land b{\neq}0)\land (r{=}1\land 1{\neq}0)\mid\DW{z}{1}}{right=of wx}
      \event{rb}{\DR{b}{1}}{left=of wx}
    \end{tikzinline}}  
\end{align*}
In an execution that reads $b{=}1$, \ref{5a} again forces a dependency
from the read of $y$ to the writes.

\subsection{Fulfillment, Local Declarations, and Top-Level Pomsets}
\label{sec:fulfillment}
As in our previous work \cite{2019-sp}, at the point that $\aLoc$ is bound,
we require that every read of $\aLoc$ be \emph{fulfilled}.  Fulfillment plays
the role that \emph{reads-from} ($\rrf$) and \emph{coherence} ($\rco$) play
in other relaxed memory models.  Unlike the acyclicity requirements involving
$\rrf$ and $\rco$ in other models, however, fulfillment is defined
\emph{compositionally}---see example \ref{blocker}, below.

\begin{definition}
  \label{def:rf}
  Two actions \emph{conflict} if one writes a location and the other
  either reads or writes the same location.
  We say $\bEv$ \emph{fulfills $\aEv$} (on $\aLoc$) if 
  \labeltextsc[F1]{(F1)}{rf1} $\bEv$ \externally writes $\aVal$ to $\aLoc$, 
  \labeltextsc[F2]{(F2)}{rf2} $\aEv$ \externally reads $\aVal$ from $\aLoc$,
  \labeltextsc[F3]{(F3)}{rf3} $\bEv \lt \aEv$, and
  \labeltextsc[F4]{(F4)}{rf4} for every conflicting write $\cEv$, either $\cEv
  \gtN \bEv$ or $\aEv \gtN \cEv$. %, or $\fthrd(\aEv)=\fthrd(\cEv)$ and $\aEv \leloc \cEv$.
\end{definition}

\ref{rf3} requires that a write $\bEv$ is ordered before any read $\aEv$
that it fulfills; this order is typically called \emph{reads from}. 
\ref{rf4} requires that any conflicting write $\cEv$ is ordered before $\bEv$
or after $\aEv$; this order is typically called \emph{extended coherence}.
For readability, we draw the order required by \ref{rf3} using bold green
arrows and the order required by \ref{rf4} using dashed red arrows.
As an example, consider:
\begin{gather*}
  \taglabel{Co1}
  \begin{gathered}
    x\GETS 1
    \PAR
    x\GETS2
    \PAR
    x\GETS 3
    \PAR
    x\GETS 4
    \PAR
    x\GETS 5
    \PAR
    r\GETS x\SEMI r\GETS x \SEMI r\GETS x \SEMI r\GETS x \SEMI r\GETS x
    \\
    \hbox{\begin{tikzinline}[node distance=.7em and 2em]
        \event{wx1}{\DW{x}{1}}{}
        \event{rx1a}{\DR{x}{1}}{above right=of wx1}
        \event{rx1b}{\DR{x}{1}}{above=of rx1a}
        \event{wx2}{\DW{x}{2}}{below right=of rx1a}
        \event{rx2a}{\DR{x}{2}}{above right=of wx2}
        \event{rx2b}{\DR{x}{2}}{above=of rx2a}
        \event{wx3}{\DW{x}{3}}{above right=of rx2a}            
        \event{wx4}{\DW{x}{4}}{right=of rx2a}
        \event{wx5}{\DW{x}{5}}{below right=of wx4}
        \event{rx5}{\DR{x}{5}}{above right=of wx5}
        \rf{wx1}{rx1a}
        \rf{wx1}{rx1b}
        \rf{wx2}{rx2a}
        \rf{wx2}{rx2b}
        \wk{rx1a}{wx2}
        \wk{rx1b}{wx2}
        \wk{rx2a}{wx3}
        \wk{rx2a}{wx4}
        \wk{rx2b}{wx3}
        \wk{rx2b}{wx4}
        \wk{wx3}{wx5}
        \wk{wx4}{wx5}
        \rf{wx5}{rx5}
      \end{tikzinline}}
  \end{gathered}
\end{gather*}
A write is \emph{relevant} if it is read from.  In order to fulfill all of
the reads on $x$ in the example, we pick a total order on the relevant
writes: in this case, $(\DW{x}{1})\le (\DW{x}{2})\le(\DW{x}{5})$.  The reads
slot between these, immediately after their fulfilling write.  Reads are not
necessarily ordered with respect to each other, even if they come from the
same thread, as do the reads here.  Irrelevant writes also float relative to
each other, as do $(\DW{x}{3})$ and $(\DW{x}{4})$.  But irrelevant writes
must be ordered with respect to relevant writes and reads.  The resulting
order is somewhat weaker than traditional extended coherence ($\reco$), which
requires a total order on writes, regardless of whether they are relevant.
We discuss coherence further in \textsection\ref{sec:litmus}.

In order to be $\aLoc$-closed, a pomset must be ``done'' with $\aLoc$, in
both the concurrent and the sequential semantics.  The concurrent semantics
requires that all reads of $\aLoc$ be fulfilled.  The sequential semantics
requires that all formulae be \emph{independent of $\aLoc$} (Def.~\ref{def:independent}):
$\forall\aVal.\;\aForm \vDash \aForm[\aVal/\aLoc] \vDash \aForm$.

\begin{definition}
  \label{def:top}
  A pomset is \emph{$\aLoc$-closed} if every \external read on $\aLoc$ is fulfilled,
  and every formula is independent of $\aLoc$.
  Let $\aPS\in(\nu\aLoc\!\DOT\!\aPSS)$ when $\aPS\in\aPSS$
  and $\aPS$ is $\aLoc$-closed.  
  We define:
  \begin{align*}
    \sem[\aThrd]{\VAR\aLoc\SEMI \aCmd} & \eqdef
    \nu \aLoc \DOT \sem[\aThrd]{\aCmd}  
  \end{align*}
  A pomset is \emph{top-level} if it is $\aLoc$-closed for every location
  $\aLoc$.  
\end{definition}
Together with \ref{rf4}, the definition of $x$-closed disallows the following execution:
\begin{gather}
  \taglabel{blocker}
  \begin{gathered}
    \VAR x\SEMI (x\GETS 1 \SEMI y_1^\mRA\GETS1
    \PAR
    \IF{z_1}\THEN x\GETS2\FI\SEMI y_2^\mRA\GETS1
    \PAR
    r\GETS z_2^\mRA\SEMI s\GETS x)
    \\[-.5ex]
    \hbox{\begin{tikzinline}[node distance=1em]
        \event{a}{\DW{x}{1}}{}
        \event{b}{\DW[\mRA]{y_1}{1}}{right=of a}
        \event{c}{\DR{z_1}{1}}{right=1.8em of b}
        \event{d}{\DW{x}{2}}{right=of c}
        \event{e}{\DW[\mRA]{y_2}{1}}{right=of d}
        \event{f}{\DR[\mRA]{z_2}{1}}{right=1.8em of e}
        \event{g}{\DR{x}{1}}{right=of f}
        \sync{a}{b}
        \po{c}{d}
        \sync{d}{e}
        \sync{f}{g}
        \rf[out=8,in=172]{a}{g}
      \end{tikzinline}}
  \end{gathered}
\end{gather}
In order to close $x$, we must choose whether $(\DW{x}{2})$ is
preceding $(\DW{x}{2}\xwk\DW{x}{1})$ or following
$(\DR{x}{1}\xwk\DW{x}{2})$.  This prevents $(\DW{x}{2})$ from blocking the
read after parallel composition.  For example, 
if \ref{blocker} were placed in the context
\begin{math}
  z_1\GETS y_1
  \PAR
  z_2\GETS y_2
  \PAR
  \hole{},
\end{math}
we would have:
\begin{gather*}
  \hbox{\begin{tikzinline}[node distance=.8em]
      \event{a}{\DW{x}{1}}{}
      \event{b}{\DW[\mRA]{y_1}{1}}{right=of a}
      \event{w}{\DR{y_1}{1}}{right=1.8em of b}
      \event{x}{\DW{z_1}{1}}{right=of w}
      \event{c}{\DR{z_1}{1}}{right=1.8em of x}
      \event{d}{\DW{x}{2}}{right=of c}
      \event{e}{\DW[\mRA]{y_2}{1}}{right=of d}
      \event{y}{\DR{y_2}{1}}{right=1.8em of e}
      \event{z}{\DW{z_2}{1}}{right=of y}
      \event{f}{\DR[\mRA]{z_2}{1}}{right=1.8em of z}
      \event{g}{\DR{x}{1}}{right=of f}
      \sync{a}{b}
      \po{c}{d}
      \sync{d}{e}
      \sync{f}{g}
      \po{w}{x}
      \rf{b}{w}
      \rf{x}{c}
      \po{y}{z}
      \rf{e}{y}
      \rf{z}{f}
    \end{tikzinline}}
\end{gather*}
This violates the conventional, weaker statement of \ref{rf4}: there is no
conflicting write $\cEv$ such that $\bEv<\cEv<\aEv$.  By requiring order on
$(\DW{x}{2})$ we forbid this blocker and
validate scope extrusion (\textsection\ref{sec:valid}).

\endinput

\begin{comment}
  https://preshing.com/20131125/acquire-and-release-fences-dont-work-the-way-youd-expect/

  Cannot encode R/A actions with actions+fences...

  A release operation prevents preceding memory operations from being delayed
  past it (a;Rel =/=> Rel;a)
  
  A release fence prevents preceding memory operations from being delayed past
  subsequent writes (a;FR;w =/=> w;a;FR)

  An acquire operation prevents subsequent memory operations from being advanced
  before it (Acq;a =/=> a;Acq)

  An acquire fence prevents subsequent memory operations from being advanced
  before prior reads (r;FA;a =/=> FA;a;r)

  https://www.modernescpp.com/index.php/fences-as-memory-barriers

  StoreLoad: Full fence allows a store before to be reordered with respect to a
  load after (wx;F;ry) ===> (ry;F;wx)

  StoreLoad+LoadLoad: Release fence also allows (rx;FR;ry) ===> (ry;FR;rx)

  StoreLoad+StoreStore: Acquire fence also allows (wx;FR;wy) ===> (wy;FR;wx)

  LoadStore: No fence allows a prior load to reorder w.r.t. a subsequent store
  (rx;FR;wy) =/=> (wy;FR;rx)

  https://preshing.com/20120710/memory-barriers-are-like-source-control-operations/
  Good news is that a fullFence does it.

  Bizarrely, it seems this is not supported in C++... You have to go to assembly.
\end{comment}
\begin{comment}
  \footnote{We only consider executions where register state is empty in
    forked threads.  Given requirement~\ref{pre-acquire}, a sufficient condition is that parallel
    composition is always preceded by an acquire fence, as in programs of the
    form:
    \begin{displaymath}
      \VAR\vec{\aLoc}\SEMI
      \vec{\aLoc}\GETS\vec{0}\SEMI
      \vec{\bLoc}\GETS\vec{0}\SEMI
      \FENCE\SEMI
      (\aCmd^1 \PAR \cdots \PAR \aCmd^n)
    \end{displaymath}
    where $\aCmd^1$, \ldots, $\aCmd^n$ do not include $\PAR$.  To avoid clutter
    in drawings, we often drop the explicit fence.}.
\end{comment}

\begin{align*}
  \begin{gathered}
    r\GETS x\SEMI y\GETS r
    \\    
    \hbox{\small $(\DR{x}{1}) \;\prefixsc\;$}
    \hbox{\begin{tikzinline}[node distance=1.5em]
        \event{wy1}{x=1\mid\DW{y}{1}}{}
      \end{tikzinline}}
  \end{gathered}
  &&
  \begin{gathered}
    \IF{r<0}\THEN y\GETS1 \FI
    \\
    \hbox{\small $(\DR{x}{1}) \;\prefixsc\;$}
    \hbox{\begin{tikzinline}[node distance=1.5em]
        \event{wy1}{x<0\mid\DW{y}{1}}{}
      \end{tikzinline}}
  \end{gathered}
\end{align*}
First consider the else-branch of the conditional: the semantics of
``${\IF{\lnot r}\THEN x\GETS1 \FI}$'' contains $(r{=}0 \mid \DW{x}{1})$,
indicating the control dependency.  The then-branch is more complex: the
semantics of ``${\IF{r}\THEN x\GETS r \FI}$'' contains
$(r{\neq}0 \land r{=}1 \mid \DW{x}{1})$ indicating both a control and a data
dependency.  This can be simplified to $(r{=}1 \mid \DW{x}{1})$.  Combining
the two branches of the conditional, we have
$(r{=}0 \lor r{=}1 \mid \DW{x}{1})$.  Prepending $r\GETS y$ substitutes
$[y/r]$, resulting in $(y{=}0 \lor y{=}1 \mid \DW{x}{1})$.  Prepending the
initializer $y\GETS 0$ substitutes $[0/y]$, resulting in
$(0{=}0 \lor 0{=}1 \mid \DW{x}{1})$.  This is a tautology, which we write as
$(\DW{x}{1})$.   We repeat this calculation in \textsection\ref{sec:pop},
after giving the formal definitions.

For the new event, \ref{4a} requires that the precondition in $\aPS'$ implies
$\aForm$.
For old writes, \ref{4b} requires that the new precondition in
$\aPS'$ must imply the old precondition in $\aPS$.
For reads, \ref{4c} allows some preconditions to weaken and requires
others to strengthen.  Recall the pomsets given previously \eqref{ex-1} for $y\GETS r$ and
$\IF{r<0}\THEN y\GETS1 \FI$. % from the beginning of this section:
Prepending $\aReg\GETS\aLoc$ first causes the substitution $[x/r]$:
\begin{align*}
  \begin{gathered}
    r\GETS x\SEMI y\GETS r
    \\    
    \hbox{\small $(\DR{x}{1}) \;\prefixsc\;$}
    \hbox{\begin{tikzinline}[node distance=1.5em]
        \event{wy1}{x=1\mid\DW{y}{1}}{}
      \end{tikzinline}}
  \end{gathered}
  &&
  \begin{gathered}
    r\GETS x\SEMI\IF{r<0}\THEN y\GETS1 \FI
    \\
    \hbox{\small $(\DR{x}{1}) \;\prefixsc\;$}
    \hbox{\begin{tikzinline}[node distance=1.5em]
        \event{wy1}{x<0\mid\DW{y}{1}}{}
      \end{tikzinline}}
  \end{gathered}
\end{align*}
\ref{4c} then the substitutes the chosen value $[1/x]$:
\begin{align}
  \label{ex-2}
  \begin{gathered}
    \hbox{\begin{tikzinline}[node distance=1.5em]
        \event{wy1}{1=1\mid\DW{y}{1}}{}
        \event{rx1}{\DR{x}{1}}{left=of wy1}
        \po{rx1}{wy1}
      \end{tikzinline}}
  \end{gathered}
  &&
  \begin{gathered}
    \hbox{\begin{tikzinline}[node distance=1.5em]
        \nonevent{wy1}{1<0\mid\DW{y}{1}}{}
        \event{rx1}{\DR{x}{1}}{left=of wy1}
      \end{tikzinline}}
  \end{gathered}
\end{align}
On the right, $(\DW{y}{1})$ has become impossible; this is no longer a pomset
due to \emph{inconsistency}.  On the left, it has become
causally dependent on the read.  By prefixing a read event, the precondition
$x=1$ has moved from the sequential realm of Hoare logic to the concurrent
memory model.  Rather than a precondition that must be \emph{satisfied}, the
resulting pomset has a read event that must be \emph{fulfilled}.

Whereas a write action introduces a precondition---satisfied
sequentially---on the value to be written; a read introduces a pomset
requirement---fulfilled concurrently---on the value to be read.  Since reads
of different values do not have disjoint preconditions, it is important that
a read introduce at most one event per pomset.  Thus, we use $\cup_\aVal$ to
combine pomsets for different read values, rather than $\parallel_\aVal$.

\ref{5a} imposes order from read to write when weakening the
precondition of the write via \ref{4c}, as on the left hand side of
\eqref{ex-2}.
\ref{4c} \emph{allows} a precondition to weaken, but does not \emph{require} it.
\ref{5a} only requires order when the precondition weakens.
Thus, no order is required in: % the following execution:
\begin{gather*}
  \aCmd=(r\GETS x\SEMI\IF{r\geq0}\THEN y\GETS1 \FI)
  \\
  \nonumber
  \hbox{\begin{tikzinline}[node distance=1.5em]
      \event{wy1}{x{\geq}0\mid\DW{y}{1}}{}
      \event{rx1}{\DR{x}{1}}{left=of wy1}
    \end{tikzinline}}
\end{gather*}
The existence of this pomset is justified
by the Hoare triple $\hoare{x\geq0}{\aCmd}{y=1}$.  It is not justified by the value of the read
action.  

Nonetheless, \ref{4c} requires that the value of the read action must
be \emph{consistent} with subsequent formulae, via
$\labelingForm(\aEv)[\aVal/\aLoc]$.  In this example, the pomset becomes inconsistent
if ${-}2$ is read for $x$:
\begin{gather*}
  \nonumber
  \hbox{\begin{tikzinline}[node distance=1.5em]
      \nonevent{wy1}{{-2}{\geq}0\mid\DW{y}{1}}{}
      \event{rx1}{\DR{x}{{-2}}}{left=of wy1}
    \end{tikzinline}}
\end{gather*}
This holds for all preceding reads, unless the precondition is discharged by
introducing order.  Thus, $\sem{s\GETS x\SEMI \aCmd}$ contains the
pomset on the left below, but not that inconsistent ``pomset'' on the right:
\begin{align*}
  \hbox{\begin{tikzinline}[node distance=.3em]
      \event{wy1}{1{\geq}0\mid\DW{y}{1}}{}
      \event{rx1}{\DR{x}{1}}{left=1em of wy1}
      \event{rx2}{\DR{x}{{-}2}}{left= of rx1}
      \po{rx1}{wy1}
    \end{tikzinline}}
  &&
  \hbox{\begin{tikzinline}[node distance=.3em]
      \nonevent{wy1}{{-}2{\geq}0\mid\DW{y}{1}}{}
      \event{rx1}{\DR{x}{1}}{left=of wy1}
      \event{rx2}{\DR{x}{{-}2}}{left= of rx1}
    \end{tikzinline}}
\end{align*}

Like \ref{4c}, which substitutes $[\aVal/\aLoc]$ during a read, Candidate
\ref{cand:ord} substitutes $[\aExp/\aLoc]$ during a write.  Like \ref{4c}, this
affects subsequent preconditions, either allowing them to weaken, or
requiring them to strengthen.  For write prefixing, however, there is no rule
corresponding to  \ref{5a}. Unlike a read event, order is \emph{not} imposed from
a write event to the subsequent events whose precondition it weakens:
\begin{align*}
  \begin{gathered}
    x \GETS 0\SEMI \aCmd
    \\[-1ex]
    \hbox{\begin{tikzinline}[node distance=1.5em]
        \event{wy1}{0{\geq}0\mid\DW{y}{1}}{}
        \event{rx1}{\DR{x}{1}}{left=.3em of wy1}
        \event{wx0}{\DW{x}{0}}{left=of rx1}
        \wk{wx0}{rx1}
      \end{tikzinline}}
  \end{gathered}
  &&
  \begin{gathered}
    x \GETS {-}2\SEMI\aCmd
    \\[-1ex]
    \hbox{\begin{tikzinline}[node distance=1.5em]
        \nonevent{wy1}{{-}2{\geq}0\mid\DW{y}{1}}{}
        \event{rx1}{\DR{x}{1}}{left=.3em of wy1}
        \event{wx0}{\DW{x}{{-}2}}{left=of rx1}
        \wk{wx0}{rx1}
      \end{tikzinline}}
  \end{gathered}
\end{align*}

