\section{Case Analysis, Access Elimination and Read Introduction}
\label{sec:refine}

The previous section shows the simplicity and beauty of pomsets with
preconditions as a model of relaxed memory.  In this section we look at some
of the complications and ugliness.  

We consider the following optimizations on relaxed access: case analysis
\eqref{CA}, dead store elimination \eqref{DS}, store forwarding \eqref{SF},
read elimination \eqref{RE}, and irrelevant read introduction \eqref{RI}.  We
do not attempt to validate rewrites that eliminate $\mRA$/$\mSC$
accesses, beyond those already given.

\begin{definition}
  \label{def:cover}
  Extend the definition of prefixing (Cand.~\ref{def:prefix}) to require:
  \begin{itemize}
  \item[{\labeltextsc[P6]{(P6)}{6}}]
    if $\bEv$ is a release, $\aEv_1$ is an acquire, $\aEv_1\le\aEv_2$, then $\labelingForm(\aEv_2)$
    is location independent.
  \end{itemize}  

  
  Let $\aPS\in(\relfilt[]{\aLoc} \aPSS)$
  when %be the set $\aPSS'\subseteq\aPSS$ such that
  $\aPS\in\aPSS$ and for every release $\aEv\in\Event$, there is some
  $\bEv\in\Event$ that writes $\aLoc$ such that $\bEv \le\aEv$.

  Let $\aPS'\in(\Rdis{\aLoc}{\aVal}\aPSS)$ %be the set $\aPSS'$ where $\aPS'\in\aPSS'$
  when there is $\aPS\in\aPSS$ such that $\Event' = \Event$,
  ${\le'} = {\le}$, $\labelingAct' = \labelingAct$, and either
  $\labelingForm'(\aEv)$ implies $\labelingForm(\aEv)$ or $\aEv$ is
  $\le$-minimal\footnote{$\aEv$ is \emph{$\le$-minimal} if there is no $\bEv$
    such that $\bEv\le\aEv$.} and
  $\labelingAct(\aEv)=\DR[\mRLXquiet]{\aLoc}{\aVal}$.
  \begin{align*}
    \sem{\aReg\GETS\aLoc^\mRLX\SEMI \aCmd} &\eqdef\;\mathhl{
      \sem{\aCmd}[\aLoc/\aReg]}\;
    \cup
    \textstyle\bigcup_\aVal\;
    (\DR[\mRLX]\aLoc\aVal) \prefix \;\mathhl{\Rdis{\aLoc}{\aVal}}\;\,
    \sem{\aCmd} [\aLoc/\aReg]
    \\
    \sem{\aLoc^\mRLX\GETS\aExp\SEMI \aCmd} & \eqdef
    \;\mathhl{%\Wdis{\aLoc}{\aExp}\bigl(
      \relfilt[]{\aLoc} \sem{\aCmd}[\aExp/\aLoc]
    }\;
    \cup
    \textstyle\bigcup_\aVal\; (\aExp=\aVal \mid \DW[\mRLX]\aLoc\aVal) \prefix \sem{\aCmd}[\aExp/\aLoc]
    \\
    \begin{aligned}
      \sem{\aReg\GETS\aLoc^\amode\SEMI \aCmd}
      \\
      \sem{\aLoc^\amode\GETS\aExp\SEMI \aCmd}
    \end{aligned}
    &
    \begin{aligned}
      &\eqdef \textstyle\bigcup_\aVal\;
      (\DRmode\aLoc\aVal) \prefix \sem{\aCmd} [\aLoc/\aReg]
      &&\textif \amode\neq\mRLX
      \\
      &\eqdef
      \textstyle\bigcup_\aVal\; (\aExp=\aVal \mid \DWmode\aLoc\aVal)
      \prefix \sem{\aCmd}[\aExp/\aLoc]
      &&\textif \amode\neq\mRLX      
    \end{aligned}
  \end{align*}
\end{definition}
There are four changes in the definition: To validate read elimination,
we include $\sem{\aCmd}[\aLoc/\aReg]$.  To ensure that read elimination does
not allow stale reads, we require \ref{6}.  To validate write elimination, we
include $\relfilt[]{\aLoc} \sem{\aCmd}[\aExp/\aLoc]$. To validate case analysis, we apply
$\Rdis{\aLoc}{\aVal}$ before prefixing a read.  

We close this section with a read-enriched semantics that validates
irrelevant read introduction.

\myparagraph{Read Elimination and Store Forwarding}

In our work on microarchitecture \citep{2019-sp}, read actions could be
observed using cache effects.  Candidate \ref{cand:ord} maintains this
perspective---for example, it distinguishes $\sem{r\GETS x}$ and
$\sem{\SKIP}$ % since
even though there is no context in the language of this paper that can
distinguish these programs.  If one accepts that these programs should be equated
at an architectural level, then one would expect the semantics to validate
read elimination \eqref{RE} and store forwarding \eqref{SF}.
\begin{align*}
  \taglabel{RE}
  \sem{\aReg  \GETS \aLoc\SEMI\aCmd} & \supseteq
  \sem{\aCmd}  
  &&\textif \aReg\not\in\free(\aCmd)&&\hbox{}
  \\
  \taglabel{SF}
  \sem{\aLoc^\amode \GETS \aExp \SEMI \aReg  \GETS \aLoc\SEMI\aCmd} &\supseteq 
  \sem{\aLoc^\amode \GETS \aExp \SEMI \aReg  \GETS \aExp\SEMI\aCmd}  
\end{align*}
These optimizations are validated by Definition \ref{def:cover}, since
$\sem{\aReg\GETS\aLoc^\mRLXquiet\SEMI \aCmd}\supseteq
\sem{\aCmd}[\aLoc/\aReg]$.  The proof of \ref{SF} also appeals to the
definition of write and the definition of register assignment.

Let us revisit the internal read examples from \textsection\ref{sec:litmus}.
With read elimination, the read action $(\DR{y}{1})$ can be elided in
\ref{Internal2}; regardless, the substitution into the write of $z$ is the
same.  On a more troubling note, the read action $(\DR{x}{1})$ can be also
elided in \ref{Internal1}, potentially converting this non-execution into a
valid execution, violating \drfsc{}.  The addition of  \ref{6} to
the definition of prefixing prevents this outcome.  When computing
$\sem{x\GETS1 \SEMI a^\mRA\GETS1 \SEMI \IF{b^\mRA}\THEN y\GETS x \FI}$,
\ref{6} prevents prefixing $(\DWRel{a}{1})$ in front of:
\begin{gather*}
  \hbox{\begin{tikzinline}[node distance=1.5em]
      \event{a6}{\DRAcq{b}{1}}{}
      \event{a7}{\DR{x}{1}}{right=of a6}
      \sync{a6}{a7}
      \event{a8}{x=1\mid\DW{y}{1}}{right=of a7}
      \graypo{a7}{a8}
      \sync[out=10,in=170]{a6}{a8}
    \end{tikzinline}}
\end{gather*}
In order to satisfy  \ref{6}, the precondition of $(\DW{y}{1})$
must be location independent.

\myparagraph{Dead Store Elimination}
Dead store elimination \eqref{DS} is symmetric to redundant load elimination.
\begin{align*}
  \taglabel{DS}
  \sem{\aLoc \GETS \aExp \SEMI \aLoc  \GETS \bExp\SEMI\aCmd} &\supseteq 
  \sem{\aLoc \GETS \bExp\SEMI\aCmd}    
\end{align*}
The rewrite is less general than \ref{RE} because general store elimination
is unsound.  For example, ``${x\GETS 0}$'' and ``${x\GETS 0\SEMI x\GETS 1}$''
can be distinguished by the context ``$\hole{}\PAR z\GETS x$''.

Using $\relfilt[]{\aLoc}$, \ref{DS} is validated by Definition
\ref{def:cover}.  A write may only be removed if it is \emph{covered} by a
following write.  This restriction is sufficient to prevent bad executions
such as:
\begin{gather*}
  x\GETS 1\SEMI
  x\GETS 2\SEMI
  y^\mRA\GETS 1
  \PAR
  \aReg\GETS y^\mRA \SEMI \bReg\GETS x
  \\[-1ex]
  \hbox{\begin{tikzinline}[node distance=1.5em]
      \event{a1}{\DW{x}{1}}{}
      \internal{a2}{\DW{x}{2}}{right=of a1}
      \graywk{a1}{a2}
      \event{a3}{\DWRel{y}{1}}{right=of a2}
      \graypo{a2}{a3}
      \event{b1}{\DRAcq{y}{1}}{right=2em of a3}
      \rf{a3}{b1}
      \event{b2}{\DR{x}{1}}{right=of b1}
      \sync{b1}{b2}
      \rf[out=10,in=170]{a1}{b2}
      \sync[out=-10,in=-170]{a1}{a3}
    \end{tikzinline}}
\end{gather*}
In this diagram, we have included a ``non-event''---dashed border---to mark
the eliminated write.  In general, there may
need to be many following writes, one for each subsequent release.  

\myparagraph{Case Analysis}
Definition \ref{def:cover} satisfies \emph{disjunction closure}.
\begin{definition}
  \label{def:dis}
  We say that $\aPS$ is a \emph{disjunct of $\aPS'$ and its downset $\aPS''$} when
  $\Event=\Event' \supseteq\Event''$, % \supseteq \{ \bEv \in \Event' \mid \exists\aEv\in\Event''.\; \bEv\le\aEv\}$,
  ${\le}={\le'}\supseteq{\le''}$,
  $\labelingAct=\labelingAct' \supseteq\labelingAct''$, 
  $\labelingForm(\aEv)$ implies
  $\labelingForm'(\aEv)\lor \labelingForm''(\aEv)$ if $\aEv\in\Event''$, and
  $\labelingForm(\aEv)$ implies
  $\labelingForm'(\aEv)$ otherwise.

  We say that $\aPSS$ is \emph{disjunction closed} if
  $\aPS\in\aPSS$ whenever there are $\{\aPS',\,\aPS''\}\subseteq \aPSS$
  such that $\aPS$ is a disjunct of $\aPS'$ and downset $\aPS''$.
\end{definition}
Disjunction closure is sufficient to establish case analysis
\eqref{CA}:
\begin{align*}
  \taglabel{CA}
  \sem{\aCmd} &\supseteq
  \sem{\IF{\aExp}\THEN\aCmd\ELSE\aCmd\FI} 
\end{align*}
The definition disjunction closure requires that
$\aPS''$ is a downset of
$\aPS'$, whereas the definition of disjunction makes no such requirement.
This requirement is implied by causal strengthening: once you take an event
that has been chosen from one side of the conditional---of the form
$\aExp\land\ldots$---then all subsequent events must satisfy $\aExp$.

Candidate \ref{cand:ord} is not disjunction closed.
For example, 
consider the two sides of the composition defined by the conditional, where
$\cmdR = \aReg\GETS x \SEMI \IF{\aExp}\THEN \bReg\GETS x \FI$.
\begin{align*}
  \begin{gathered}
    \IF{\bExp}\THEN \cmdR \FI
    \\
    \hbox{\begin{tikzinline}[node distance=2em]
        \eventl{d}{a}{\bExp \mid\DR{x}{0}}{}
        \eventl{e}{b}{\bExp \land \aExp\mid\DR{x}{0}}{right=of a}
      \end{tikzinline}}
  \end{gathered}
  &&
  \begin{gathered}
    \IF{\lnot \bExp}\THEN \cmdR \FI
    \\
    \hbox{\begin{tikzinline}[node distance=2em]
        \eventl{e}{a}{\lnot \bExp \mid\DR{x}{0}}{}
        \eventl{d}{b}{\lnot \bExp \land \aExp\mid\DR{x}{0}}{right=of a}
      \end{tikzinline}}
  \end{gathered}
\end{align*}
Because the reads are unordered, they can be confused when coalescing, resulting in:
\begin{align*}
  \begin{gathered}
    \IF{\bExp}\THEN \cmdR \ELSE \cmdR \FI
    \\
    \hbox{\begin{tikzinline}[node distance=2em]
        \eventl{d}{a}{\bExp\lor (\lnot \bExp \land \aExp) \mid\DR{x}{0}}{}
        \eventl{e}{b}{(\bExp \land \aExp)\lor \lnot \bExp \mid\DR{x}{0}}{right=of a}
      \end{tikzinline}}    
  \end{gathered}
\end{align*}
which is:
\begin{align*}
  \begin{gathered}
    \hbox{\begin{tikzinline}[node distance=2em]
        \eventl{d}{a}{\bExp\lor \aExp\mid\DR{x}{0}}{}
        \eventl{e}{b}{\lnot \bExp\lor \aExp\mid\DR{x}{0}}{right=of a}
      \end{tikzinline}}
  \end{gathered}
\end{align*}
But this pomset does not occur in $\sem{\cmdR}$.  
Our solution is to weaken the preconditions on reads using $\Rdis{\aLoc}{\aVal}$ so that both
$\sem{\cmdR}$ and $\sem{\IF{\bExp}\THEN \cmdR \ELSE \cmdR \FI}$ include: 
\begin{align*}
  \begin{gathered}
    \hbox{\begin{tikzinline}[node distance=2em]
        \eventl{d}{a}{\DR{x}{0}}{}
        \eventl{e}{b}{\DR{x}{0}}{right=of a}
      \end{tikzinline}}
  \end{gathered}
\end{align*}
Note that the precondition on the reads are weaker than one would expect.
This is not a problem for reads, since they must also be fulfilled---allowing
more reads \emph{increases} the obligations of fulfillment.  The same
solution would not work for writes---as we discussed at the end of
\textsection\ref{sec:props}, allowing more writes is simply unsound.
Fortunately, this problem does not occur when prefixing a write in front of
another write, due to the order required by \ref{5b}.

If \ref{5b} is strengthened to include read-read coherence, then disjunction
closure holds without $\Rdis{\aLoc}{\aVal}$.  In this case, however,
\ref{CSE} fails.  This compromise may be reasonable for C11 atomics, which
are meant to be used sparingly.  It is less attractive for %relaxed access in
safe languages, like Java.

\myparagraph{Irrelevant Read Introduction}
A compiler may introduce reads in order to lift code.  Consider the
following example \cite[\textsection1.4.5]{SevcikThesis}:
\begin{align*}
  \sem{\IF{\aReg} \THEN \bReg \GETS \aLoc \SEMI \bLoc \GETS \bReg \FI}
  &\not\supseteq
  \sem{\bReg \GETS \aLoc \SEMI \IF{\aReg} \THEN \bLoc \GETS \bReg \FI}
\end{align*}
The right-hand program is derived from the left by introducing an irrelevant
read in the else-branch, then moving the common code out of the
conditional.  Definition \ref{def:cover} does \emph{not} validate this rewrite.

Read introduction is only valid ``modulo irrelevant reads.'' We capture this
idea using \emph{read saturation}.  Read saturation allows us to add actions
of the form $(\DR{x}{v})$ to the left-hand side, validating the inclusion.

Let $\aPS'\in\readc(\aPSS)$ %be the set $\aPSS'$ where $\aPS'\in\aPSS'$
when $\exists\aPS\in\aPSS$ and $\exists D$ such that $\Event'= \Event\uplus D$,
${\le'}\supseteq{\le}$,
${\labeling'} \supseteq{\labeling}$, and
$\forall \bEv\in D.\;\exists\aLoc.\;\exists\aVal$.
$\labelingAct'(\bEv)=(\DR[\mRLXquiet]{\aLoc}{\aVal})$.
Note that if $\aPSS\supseteq\aPSS'$, then
$\readc(\aPSS)\supseteq \readc(\aPSS')$.

Read introduction \eqref{RI} is valid under the saturated semantics.
\begin{align*}
  \taglabel{RI}
  \readc\sem{\aCmd} & \supseteq
  \readc\sem{\aReg  \GETS \aLoc\SEMI\aCmd}  
  &&\textif \aReg\not\in\free(\aCmd)&&\hbox{}  
\end{align*}

With \ref{RI}, the model satisfies all of the transformations of
\citet[\textsection 5.3-4]{SevcikThesis} except redundant write after read
elimination (see \textsection\ref{sec:limits}) and reordering with external
actions, which we do not model.

