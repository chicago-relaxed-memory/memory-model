\section{Weak Ideas}
Goal is to capture \ppc, not \armseven.
See \textsection\ref{sec:ppc:arm7}, below.
So cannot use IMM out of the box.

Introduce \emph{weak order} $\wle$\footnote{Note we can \emph{not} require
\begin{itemize}
\item if $\bEv\;({\le}\cup{\wle})\;\aEv$ then $\labelingSub(\bEv)$ subsumes
  $\labelingSub(\aEv)$.
\end{itemize}
This does not hold, for example, in $\sem{x\GETS1\SEMI x\GETS2}$.}.

\begin{definition}[2.1]
  A \emph{(memory order) pomset} is a tuple
  $(\Event, {\le}, {\wle}, \labeling,{\xrmw})$: 
  \begin{itemize}
  \item $\Event$ is a set of \emph{states},
  \item ${\le}\subseteq (\Event\times\Event)$ and ${\wle}\subseteq (\Event\times\Event)$ are partial orders, 
  \item $\labeling: \Event \fun (\Formulae\times\Act)$ is a \emph{labeling},
    from which we derive functions $\labelingForm:\Event\fun\Formulae$ and $\labelingAct:\Event\fun\Act$,
  % \item $\labeling: \Event \fun (\Formulae\times\Act\times\Sub)$ is a \emph{labeling},
  %   from which we derive functions $\labelingForm:\Event\fun\Formulae$, $\labelingAct:\Event\fun\Act$, and  $\labelingSub:\Event\fun\Sub$,
  \item if $\bEv\;({\le}\cup{\wle})\;\aEv$ then $\labelingForm(\aEv)$ implies $\labelingForm(\bEv)$, and
  % \item $\bigwedge\{\labelingForm(\aEv_i)\mid\forall i,j.\; \aEv_i\xpox\aEv_j \textor \aEv_j\xpox\aEv_i\}$ is satisfiable.
  \item $\bigwedge_{\aEv}\labelingForm(\aEv)$ is satisfiable.
  \end{itemize}
  Additional stuff:
  \begin{itemize}
  \item if $\bEv\;({\le}\sequence{\wle})\;\aEv$ then $\bEv\neq\aEv$, and
  % \item if $\bEv\le\aEv$ and $\bEv$ and $\aEv$ conflict, then $\bEv\wle\aEv$,
  %\item if $\bEv\wle\aEv$ and $\bEv$ and $\aEv$ are SC, then $\bEv\le\aEv$,
  \item if $\bEv\;({\le}\sequence{\wle}\sequence{\le})\;\aEv$, $\bEv$ is SC,
    and $\aEv$ is SC, then $\bEv\le\aEv$.
  \end{itemize}
  RMW:
  \begin{itemize}
  \item If $\bEv\xrmw\aEv$, then $\bEv\le\aEv$.
  \item If $\cEv$, $\aEv$ write the same $x$, $\cEv\wle \aEv$ and $\bEv \xrmw \aEv$ then  $\cEv\wle \bEv$.
  \item If $\cEv$, $\aEv$ write the same $x$, $\bEv\wle \cEv$ and $\bEv \xrmw \aEv$ then  $\aEv\wle \cEv$.
  \end{itemize}  
\end{definition}
Update the definitions to use $\wle$ instead of $\le$ in two places:
\begin{itemize}
\item the last item defining fulfillment, and
\item item 5b defining prefixing.
\end{itemize}

\begin{definition}[2.4]
  We say $\bEv$ \emph{fulfills $\aEv$ on $\aLoc$} if 
  \begin{itemize}    
  \item $\bEv$ writes
    $\aVal$ to $\aLoc$, 
  \item $\aEv$ reads $\aVal$ from $\aLoc$,
  \item
    $\bEv \le \aEv$, and
  \item
    if $\cEv$ writes to $\aLoc$ then either $\cEv \wle \bEv$ or $\aEv \wle \cEv$.
  \end{itemize}
\end{definition}

\begin{definition}[2.10]
  \label{def:prefix}
Let $(\aForm \mid \aAct) \prefix \aPSS$ be the set $\PRE(\aPSS')$ where
$\aPS'\in\aPSS'$ when
there is some $\aPS\in\aPSS$ that satisfies items 1-4 of
Definition 2.8 %\ref{def:pre-sc}
such that:
\begin{enumerate}
\item[5a.]  if $\aEv$ writes then either $\bEv\lt'\aEv$ or
  $\labelingForm'(\aEv)$ implies $\labelingForm(\aEv)$,
\item[5b.] if $\bEv$ and $\aEv$ are actions in conflict, then $\bEv\wlt'\aEv$,
\item[5c.] if $\bEv$ is an acquire or $\aEv$ is a release, then $\bEv \lt' \aEv$, 
\item[5d.] if $\bEv$ is an SC write and $\aEv$ is an SC read, then $\bEv \lt' \aEv$,
\item[5e.] if $\bEv$ reads, and $\aEv$ is an acquiring fence, then
  $\bEv \lt' \aEv$, and
\item[5f.] if $\bEv$ is a releasing fence, and $\aEv$ writes, then
  $\bEv \lt' \aEv$.
\end{enumerate}
\end{definition}


Weak order is only required to relate actions on the same location.  In
augmentation minimal pomsets, it is a subset of $\reco$ (only relates writes
that are read).  The irreflexivity requirement in the definition is thus
comparable to requiring that ${\le}\cup{\wleaLoc}$ is a partial
order, for every $\aLoc$.  It is \emph{not} the case that ${\le}\cup{\wle}$
is a partial order.

Note that we have a kind of semi-transitivity here, but only per variable.
\begin{itemize}
\item If $\cEv\leaLoc\bEv\wleaLoc\aEv$ then $\cEv\wleaLoc\aEv$.
\item If $\cEv\wleaLoc\bEv\leaLoc\aEv$ then $\cEv\wleaLoc\aEv$.
\end{itemize}
With the requirements of fulfillment, we have that $\bEv\le\aEv$ implies
$\bEv\wle\aEv$ when the actions conflict---there is a caveat for unread
writes, where no order is forced.

Here are some examples of the main text.  To better visualize, we use
different arrowheads for strong and weak order.  We use a single color for
strong order, and separate colors for each variable in weak order.

% \tikzstyle{po} = [-latex,color=black,]
% \tikzstyle{rf} = [-latex,color=black,]

Here is fencing behavior mixing with release/acquire:
\begin{gather*}  
  x\GETS1\SEMI
  %\FENCE^{\modeREL}\SEMI
  y\REL\GETS1
  \PAR
  r\GETS y\ACQ\SEMI
  %\FENCE^{\modeACQ}\SEMI
  s\GETS x
  \\[-1ex]
  \hbox{\begin{tikzinline}[node distance=1em]
      \event{a1}{\DW{x}{1}}{}
      % \event{a2}{\DFS{\modeREL}}{right=of a1}
      % \po{a1}{a2}
      \event{a3}{\DWRel{y}{1}}{right=of a1}
      \po{a1}{a3}
      \event{b1}{\DRAcq{y}{1}}{right=3em of a3}
      % \event{b2}{\DFS{\modeACQ}}{right=of b1}
      % \po{b1}{b2}
      \event{b3}{\DR{x}{0}}{right=of b1}
      \po{b1}{b3}
      \rf{a3}{b1}
      \wkx[out=-170,in=-10]{b3}{a1}
    \end{tikzinline}}
\end{gather*}
\begin{gather*}  
  x\GETS1\SEMI
  \FENCE^{\modeREL}\SEMI
  y\GETS1
  \PAR
  r\GETS y\ACQ\SEMI
  %\FENCE^{\modeACQ}\SEMI
  s\GETS x
  \\[-1ex]
  \hbox{\begin{tikzinline}[node distance=1em]
      \event{a1}{\DW{x}{1}}{}
      \event{a2}{\DFS{\modeREL}}{right=of a1}
      \po{a1}{a2}
      \event{a3}{\DW{y}{1}}{right=of a2}
      \po{a2}{a3}
      \event{b1}{\DR{y}{1}}{right=3em of a3}
      % \event{b2}{\DFS{\modeACQ}}{right=of b1}
      % \po{b1}{b2}
      \event{b3}{\DR{x}{0}}{right=of b1}
      \po{b1}{b3}
      \rf{a3}{b1}
      \wkx[out=-170,in=-10]{b3}{a1}
    \end{tikzinline}}
\end{gather*}
\begin{gather*}  
  x\GETS1\SEMI
  y\REL\GETS1
  \PAR
  r\GETS y\SEMI
  \FENCE^{\modeACQ}\SEMI
  s\GETS x
  \\[-1ex]
  \hbox{\begin{tikzinline}[node distance=1em]
      \event{a1}{\DW{x}{1}}{}
      % \event{a2}{\DFS{\modeREL}}{right=of a1}
      % \po{a1}{a2}
      \event{a3}{\DWRel{y}{1}}{right=of a1}
      \po{a1}{a3}
      \event{b1}{\DR{y}{1}}{right=3em of a3}
      \event{b2}{\DFS{\modeACQ}}{right=of b1}
      \po{b1}{b2}
      \event{b3}{\DR{x}{0}}{right=of b2}
      \po{b2}{b3}
      \rf{a3}{b1}
      \wkx[out=-170,in=-10]{b3}{a1}
    \end{tikzinline}}
\end{gather*}
\begin{gather*}  
  x\GETS1\SEMI
  \FENCE^{\modeREL}\SEMI
  y\GETS1
  \PAR
  r\GETS y\SEMI
  \FENCE^{\modeACQ}\SEMI
  s\GETS x
  \\[-1ex]
  \hbox{\begin{tikzinline}[node distance=1em]
      \event{a1}{\DW{x}{1}}{}
      \event{a2}{\DFS{\modeREL}}{right=of a1}
      \po{a1}{a2}
      \event{a3}{\DW{y}{1}}{right=of a2}
      \po{a2}{a3}
      \event{b1}{\DR{y}{1}}{right=3em of a3}
      \event{b2}{\DFS{\modeACQ}}{right=of b1}
      \po{b1}{b2}
      \event{b3}{\DR{x}{0}}{right=of b2}
      \po{b2}{b3}
      \rf{a3}{b1}
      \wkx[out=-170,in=-10]{b3}{a1}
    \end{tikzinline}}
\end{gather*}

Coherence C does not allow:
\begin{gather*}
  x\GETS1\SEMI x\GETS 2
  \PAR
  y\GETS x \SEMI z\GETS x
  \\[-1ex]
  \hbox{\begin{tikzinline}[node distance=1em]
      \event{a}{\DW{x}{1}}{}
      \event{b}{\DW{x}{2}}{right=of a}
      \wkx{a}{b}
      \event{c}{\DR{x}{2}}{right=2em of b}
      \event{d}{\DW{y}{2}}{right=of c}
      \po{c}{d}
      \event{e}{\DR{x}{1}}{right=of d}
      \event{f}{\DW{z}{1}}{right=of e}
      \po{e}{f}
      \rf{b}{c}
      \rf[out=10,in=170]{a}{e}
    \end{tikzinline}}
\end{gather*}

Coherence Java allows:
\begin{gather*}
  x\GETS1\SEMI y^\modeRA\GETS 1
  \PAR
  x\GETS 2\SEMI
  r\GETS y^\modeSC \SEMI 
  r\GETS x \SEMI 
  r\GETS x
  \\[-1ex]
  \hbox{\begin{tikzinline}[node distance=1em]
      \event{a}{\DW{x}{1}}{}
      \event{b}{\DW[\modeRA]{y}{1}}{right=of a}
      \po{a}{b}
      \event{c}{\DW{x}{2}}{below=of a}
      \event{d}{\DR[\modeSC]{y}{1}}{right=of c}
      \po{c}{d}
      \event{e}{\DR{x}{2}}{right=of d}
      \po{d}{e}
      \event{f}{\DR{x}{1}}{right=of e}
      \po[out=10,in=170]{d}{f}
      \rf{b}{d}
      %\rf[out=-10,in=-170]{c}{e}
      \wkx{a}{c}
      %\rf[out=15,in=150]{a}{f}
      \wkx[out=-165,in=-15]{f}{c}
    \end{tikzinline}}
\end{gather*}
Store buffering
\begin{gather*}
  x\GETS0\SEMI
  y\GETS0\SEMI
  (
  x\REL\GETS1\SEMI\aReg\GETS y
  \PAR
  y\REL\GETS1\SEMI \aReg\GETS x)
  \\
  \hbox{\begin{tikzinline}[node distance=1em]
      \event{wx0}{\DW{x}{0}}{}
      \event{wy0}{\DW{y}{0}}{below=of wx0}
      \event{wx}{\DWRel{x}{1}}{right=of wx0}
      \event{wy}{\DWRel{y}{1}}{right=of wy0}
      \event{ry}{\DR{y}{0}}{right=of wx}
      \event{rx}{\DR{x}{0}}{right=of wy}
      \wkx{wx0}{wx}
      \wky{wy0}{wy}
      \po{wy}{rx}
      \po{wx}{ry}
      \rf{wy0}{ry}
      \rf{wx0}{rx}
      \wky{ry}{wy}
      \wkx{rx}{wx}
      % \po{rx}{wy}
    \end{tikzinline}}
\end{gather*}

IRIW
\begin{gather*}
  \renewcommand{\arraycolsep}{1pt}
  \begin{array}{ccccc}
    &x\GETS0\SEMI x\GETS 1
    &\PAR&
    r\GETS x\ACQ \SEMI s\GETS y
    %\IF{x}\THEN r\GETS y \FI
    \\
    \PAR
    &y\GETS0\SEMI y\GETS 1
    &\PAR&
    r\GETS y\ACQ \SEMI s\GETS x
    %\IF{y}\THEN s\GETS x \FI
  \end{array}
  \\
  \hbox{\begin{tikzinline}[baseline=-10pt,node distance=.5em and 1em]
  \event{wx0}{\DW{x}{0}}{}
  \event{wx1}{\DW{x}{1}}{right=of wx0}
  \event{wy0}{\DW{y}{0}}{below=2ex of wx0}
  \event{wy1}{\DW{y}{1}}{right=of wy0}
  \event{ry1}{\DRAcq{y}{1}}{right=2.5em of wy1}
  \event{rx0}{\DR{x}{0}}{right=of ry1}
  \event{rx1}{\DRAcq{x}{1}}{right=2.5 em of wx1}
  \event{ry0}{\DR{y}{0}}{right=of rx1}
  \wkx{wx0}{wx1}
  \wky{wy0}{wy1}
  \rf{wx1}{rx1}
  \rf[bend left]{wy0}{ry0}
  \rf{wy1}{ry1}
  \rf[bend right]{wx0}{rx0}
  \wkx{rx0}{wx1}
  \wky{ry0}{wy1}
  \po{rx1}{ry0}
  \po{ry1}{rx0}
    \end{tikzinline}}
\end{gather*}

Three variable \mca\ example Allowed.
\begin{gather*}
  \hbox{\small$\IF{x}\THEN y\GETS0 \FI \SEMI y\GETS1
  {\PAR}
  \IF{y}\THEN z\GETS0 \FI \SEMI z\GETS1
  {\PAR}
  \IF{z}\THEN x\GETS0 \FI \SEMI x\GETS1
  $}
  \\[-.5ex]
  \hbox{\begin{tikzinlinesmall}[node distance=1em]
  \event{a1}{\DR{x}{1}}{}
  \event{a2}{\DW{y}{0}}{right=of a1}
  \po{a1}{a2}
  \event{a3}{\DW{y}{1}}{right=of a2}
  \wky{a2}{a3}
  \event{b1}{\DR{y}{1}}{right=of a3}
  \event{b2}{\DW{z}{0}}{right=of b1}
  \po{b1}{b2}
  \event{b3}{\DW{z}{1}}{right=of b2}
  \wkz{b2}{b3}
  \event{c1}{\DR{z}{1}}{right=of b3}
  \event{c2}{\DW{x}{0}}{right=of c1}
  \po{c1}{c2}
  \event{c3}{\DW{x}{1}}{right=of c2}
  \wkx{c2}{c3}
  \rf{a3}{b1}
  \rf{b3}{c1}
  \rf[out=173,in=7]{c3}{a1}  
    \end{tikzinlinesmall}}
\end{gather*}
Two variable \mca\ example Allowed.
\begin{gather*}
  \hbox{\small$\IF{x}\THEN y\GETS0 \FI \SEMI y\GETS1
  {\PAR}
  \IF{y}\THEN x\GETS0 \FI \SEMI x\GETS1
  $}
  \\[-.5ex]
  \hbox{\begin{tikzinlinesmall}[node distance=1em]
  \event{a1}{\DR{x}{1}}{}
  \event{a2}{\DW{y}{0}}{right=of a1}
  \po{a1}{a2}
  \event{a3}{\DW{y}{1}}{right=of a2}
  \wky{a2}{a3}
  \event{b1}{\DR{y}{1}}{right=of a3}
  \event{b2}{\DW{x}{0}}{right=of b1}
  \po{b1}{b2}
  \event{b3}{\DW{x}{1}}{right=of b2}
  \wkx{b2}{b3}
  \rf{a3}{b1}
  \rf[out=173,in=7]{b3}{a1}  
    \end{tikzinlinesmall}}
\end{gather*}

\cite[Fig.~5]{DBLP:conf/pldi/LahavVKHD17}
\begin{gather*}
    x\GETS1
    \PAR
    r\GETS x\SEMI   
    \FENCE^{\modeSC}\SEMI
    r\GETS y  
    \PAR
    y\GETS 1 \SEMI
    \FENCE^{\modeSC}\SEMI
    r\GETS x  
    \\[-.1ex]
  \hbox{\begin{tikzinline}[node distance=1em]
  \event{a1}{\DW{x}{1}}{}
  \event{b1}{\DR{x}{1}}{right=3em of a1}
  \event{b2}{\DFS{\modeSC}}{right=of b1}
  \po{b1}{b2}
  \event{b3}{\DR{y}{0}}{right=of b2}
  \po{b2}{b3}
  \event{c1}{\DW{y}{1}}{right=3em of b3}
  \event{c2}{\DFS{\modeSC}}{right=of c1}
  \po{c1}{c2}
  \event{c3}{\DR{x}{0}}{right=of c2}
  \po{c2}{c3}
  \wky{b3}{c1}
  \rf{a1}{b1}
  \wkx[out=-170,in=-10]{c3}{a1}
  \po[in=170,out=10]{b2}{c2}
    \end{tikzinline}}
\end{gather*}

\cite[Fig.~6]{DBLP:conf/pldi/LahavVKHD17}
\begin{gather*}
x\GETS1\SEMI   
    z\REL\GETS1\SEMI   
    \PAR
    r\ACQ\GETS z\SEMI   
    \FENCE^{\modeSC}\SEMI
    r\GETS y  
    \PAR
    y\GETS 1 \SEMI
    \FENCE^{\modeSC}\SEMI
    r\GETS x  
    \\[-.1ex]
  \hbox{\begin{tikzinline}[node distance=.7em]
  \event{a1}{\DW{x}{1}}{}
  \event{a2}{\DWRel{z}{1}}{right=of a1}
  \po{a1}{a2}
  \event{b1}{\DRAcq{z}{1}}{right=2em of a2}
  \event{b2}{\DFS{\modeSC}}{right=of b1}
  \po{b1}{b2}
  \event{b3}{\DR{y}{0}}{right=of b2}
  \po{b2}{b3}
  \event{c1}{\DW{y}{1}}{right=2em of b3}
  \event{c2}{\DFS{\modeSC}}{right=of c1}
  \po{c1}{c2}
  \event{c3}{\DR{x}{0}}{right=of c2}
  \po{c2}{c3}
  \wky{b3}{c1}
  \rf{a2}{b1}
  \wkx[out=-170,in=-10]{c3}{a1}
  \po[in=170,out=10]{b2}{c2}
    \end{tikzinline}}
\end{gather*}

PSC is part of AR
\cite[Ex.~3.9]{DBLP:journals/pacmpl/PodkopaevLV19}:
\begin{gather*}
  r\GETS y\SEMI
  \FENCE^{\modeSC}\SEMI
  r\GETS z
  \PAR
  z\GETS 1\SEMI
  \FENCE^{\modeSC}\SEMI
  r\GETS x
  \PAR
  \IF{x{\neq}0}\THEN y\GETS1 \FI
  \\
  \hbox{\begin{tikzinline}[node distance=.5em and 1em]
      \event{a1}{\DR{y}{1}}{}
      \event{a2}{\DFS{\modeSC}}{right=of a1}
      \event{a3}{\DR{z}{0}}{right=of a2}
      \po{a1}{a2}
      \po{a2}{a3}
      \event{b1}{\DW{z}{1}}{right=2em of a3}
      \event{b2}{\DFS{\modeSC}}{right=of b1}
      \event{b3}{\DR{x}{1}}{right=of b2}
      \po{b1}{b2}
      \po{b2}{b3}
      \event{c1}{\DR{x}{1}}{right=2em of b3}
      \event{c2}{\DW{y}{1}}{right=of c1}
      \po{c1}{c2}
      \wkz{a3}{b1}
      \rf{b3}{c1}
      \rf[out=-170,in=-10]{c2}{a1}
   \end{tikzinline}}
\end{gather*}
\cite[Ex.~3.2]{DBLP:journals/pacmpl/PodkopaevLV19}:
\begin{gather*}
  \aLoc\GETS0\SEMI\bReg\GETS \FADD^{\modeRLX,\modeRLX}(\aLoc)
  \PAR
  x\GETS 2\SEMI s\GETS x
  \\
  \hbox{\begin{tikzinline}[node distance=1em]
  \event{a2}{\DR{x}{0}}{}
  \event{a1}{\DW{x}{0}}{left=of a2}
  \rf{a1}{a2}
  \event{a3}{\DW{x}{2}}{right=of a2}
  \wkx{a2}{a3}
  \event{b2}{\DW{x}{1}}{right=of a3}
  \event{b3}{\DR{x}{1}}{right=of b2}
  \rmw[out=-15,in=-165]{a2}[below]{b2}
  \wkx{a3}{b2}
  \rf{b2}{b3}
    \end{tikzinline}}
\end{gather*}

\cite[Ex.~3.10]{DBLP:journals/pacmpl/PodkopaevLV19} (Arm allows):
\begin{gather*}
  r \GETS y\SEMI
  z \GETS r
  \PAR
  r\GETS z\SEMI
  x\GETS 0\SEMI
  s\GETS \FADD^{\modeRLX,\modeRA}(x) \SEMI
  y\GETS s{+}1
  \\[-1ex]
  \hbox{\begin{tikzinline}[node distance=1em]
  \event{a1}{\DR{y}{1}}{}
  \event{a2}{\DW{z}{1}}{right=of a1}
  \po{a1}{a2}
  \event{b1}{\DR{z}{1}}{right=3em of a2}
  \rf{a2}{b1}
  \event{b2}{\DW{x}{0}}{right=of b1}
  \event{b3}{\DR{x}{0}}{right=of b2}
  \rf{b2}{b3}
  \event{b4}{\DWRel{x}{1}}{right=2em of b3}
  \rmw{b3}{b4}
  \event{b5}{\DW{y}{1}}{right=of b4}
  \po[out=-15,in=-165]{b1}{b4}
  \po[out=-20,in=-160]{b3}{b5}
  \rf[out=170,in=10]{b5}{a1}
    \end{tikzinline}}
\end{gather*}


Detour Example 
\cite[Ex.~3.7]{DBLP:journals/pacmpl/PodkopaevLV19}:
\begin{gather*}
  x\GETS z-1\SEMI
  y\GETS x
  \PAR
  x\GETS1
  \PAR
  z\GETS y
  \\
  \hbox{\begin{tikzinline}[node distance=.5em and 1em]
      \event{b1}{\DR{z}{1}}{}
      \event{b2}{\DW{x}{0}}{right=of b1}
      \po{b1}{b2}
      \event{b3}{\DR{x}{1}}{right=of b2}
      \event{b4}{\DW{y}{1}}{right=of b3}
      \po{b3}{b4}
      \event{c1}{\DR{y}{1}}{right=2em of b4}
      \event{c2}{\DW{z}{1}}{right=of c1}
      \po{c1}{c2}
      \event{a1}{\DW{x}{1}}{below right=2ex and -2ex of b2}
      \rf{b4}{c1}
      \rf{a1}{b3}
      \wkx{b2}{a1} 
      \rf[out=170,in=10]{c2}{b1}
   \end{tikzinline}}
\end{gather*}

\subsection{Another fencing example}

\cite[\textsection{}D]{DBLP:journals/pacmpl/PodkopaevLV19}:
The following execution graph is not consistent in the promise-free
declarative model of [Kang et al. 2017]. Nevertheless, its mapping to POWER
(obtained by simply replacing Fsc with Fsync) is POWER-consistent and ${\rpox}\cup {\rrf}$
is acyclic (so it is Strong-POWER-consistent). Note that, using promises, the
promising semantics allows this behavior.
\begin{gather*}  
  r\GETS z\SEMI
  \FENCE^{\modeSC}\SEMI
  x\GETS1
  \PAR
  x\GETS 2\SEMI
  \FENCE^{\modeSC}\SEMI
  y\GETS 1
  \PAR
  r\GETS y\SEMI
  z\GETS 1
  \\[-1ex]
  \hbox{\begin{tikzinline}[node distance=1em]
      \event{a1}{\DR{z}{1}}{}
      \event{a2}{\DFS{\modeSC}}{right=of a1}
      \po{a1}{a2}
      \event{a3}{\DW{x}{1}}{right=of a2}
      \po{a2}{a3}
      \event{b1}{\DW{x}{2}}{right=3em of a3}
      \event{b2}{\DFS{\modeSC}}{right=of b1}
      \po{b1}{b2}
      \event{b3}{\DW{y}{1}}{right=of b2}
      \po{b2}{b3}
      \event{c1}{\DR{y}{1}}{right=3em of b3}
      \event{c2}{\DW{z}{1}}{right=of c1}
      %\po{c1}{c2}
      \wkx{a3}{b1}
      \rf{b3}{c1}
      \rf[out=170,in=10]{c2}{a1}
      \po[out=-10,in=-170]{a2}{b2}
    \end{tikzinline}}
\end{gather*}
Allowed behavior on POWER...
Is there a dependency in the last thread?
If so, this is a problem.

\cite[\textsection{}8]{DBLP:journals/pacmpl/PodkopaevLV19}:
To establish the correctness of compilation of the promising semantics to
POWER, Kang et al. [2017] followed the approach of Lahav and Vafeiadis
[2016]. This approach reduces compilation correctness to POWER to (i) the
correctness of compilation to the POWER model strengthened with ${\rpox}\cup {\rrf}$
acyclicity; and (ii) the soundness of local reorderings of memory
accesses. To establish (i), Kang et al. [2017] wrongly argued that the
strengthened POWER-consistency of mapped promise-free execution graphs imply
the promise-free consistency of the source execution graphs. This is not the
case due to SC fences, which have relatively strong semantics in the
promise-free declarative model (see [Podkopaev et al. 2018, Appendix D] for a
counter example). Nevertheless, our proof shows that the compilation claim of
Kang et al. [2017] is correct.



\subsection{Power versus ARM7}
\label{sec:ppc:arm7}
\cite[\textsection5]{DBLP:conf/fm/LahavV16}: Characterizing ppo of power:
\begin{gather*}
  \tag{PPO lower}
  \mathsf{[RU]};({\rdeps}\cup{\rpoloc})^+;\mathsf{[WU]} \subseteq {\rppo}
  \\
  \tag{PPO upper}
  {\rppo}\cap{{\rpox}\imm} \subseteq ({\rdeps}\cup{\rpoloc})^+
\end{gather*}
$R\imm$ denotes the relation consisting of all immediate R-edges,
i.e., pairs $(a,b) \in R$ such that for every $c$, $(c,b) \in R$ implies $(c,a) \in R^?$, and 
$(a, c) \in R$ implies $(b, c) \in R^?$.

\cite[After example 3.6]{DBLP:journals/pacmpl/PodkopaevLV19}:
Note that we do not include fri in ppo since it is not preserved in ARMv7
[Alglave et al. 2014] (unlike in x86-TSO, POWER, and ARMv8). Thus, as ARMv7
(as well as the Flowing and POP models of ARM in [Flur et al. 2016]), IMM
allows the weak behavior from [Lahav and Vafeiadis 2016, §6].

\cite[\textsection6]{DBLP:conf/fm/LahavV16}:
Consider the program in Fig. 4.
\begin{gather*}
  r\GETS x\SEMI
  x\GETS 1
  \PAR
  y\GETS x
  \PAR
  x\GETS y
  \\
  \hbox{\begin{tikzinline}[node distance=.5em and 1em]
      \event{a1}{\DR{x}{1}}{}
      \event{a2}{\DW{x}{1}}{right=of a1}
      \wkx{a1}{a2}
      \event{b1}{\DR{x}{1}}{right=2em of a2}
      \event{b2}{\DW{y}{1}}{right=of b1}
      \po{b1}{b2}
      \event{c1}{\DR{y}{1}}{right=2em of b2}
      \event{c2}{\DW{x}{1}}{right=of c1}
      \po{c1}{c2}
      \rf{a2}{b1}
      \rf{b2}{c1}
      \rf[out=-170,in=-10]{c2}{a1}
   \end{tikzinline}}
\end{gather*}
Note that no reorderings or eliminations can
be applied to this program. In the second and the third threads, reordering
is forbidden because of the dependency between the load and the subsequent
store. On the first thread, there is no dependency, but since the load and
the store access the same location, their reordering is generally unsound, as
it allows the load to read from the (originally subsequent) store. Moreover,
this program cannot return a = 1 under a (po $\cup$ rf)-acyclic model, because the
only instance of the constant 1 in the program occurs after the load of x in
the first thread. Nevertheless, this behavior is allowed under both the
axiomatic ARMv7 model of Alglave et al. [4] and the ARMv8 Flowing and POP
models of Flur et al. [12]

The axiomatic ARMv7 model [4] is the same as the Power model presented in
Section 5, with the only difference being the definition of ppo (preserved
program order). In particular, this model does not satisfy (ppo-lower-bound)
because
\begin{displaymath}
  \mathsf{[RU]};{\rpoloc};\mathsf{[WU]} \not\subseteq {\rppo}.
\end{displaymath}
Hence, the first thread’s program order in the example above is not included
in ppo, and there is no happens-before cycle. For the same reason, our proof
for Power does not carry over to ARM.

In the ARMv8 Flowing model [12], consider the topology where the first two
threads share a queue and the third thread is separate. The following
execution is possible: (1) the first thread issues a load request from x and
immediately commits the x := 1 store; (2) the second thread then issues a
load request from x, which gets satisfied by the x := 1 store, and then (3)
issues a store to y := 1; (4) the store to y gets reordered with the
x-accesses, and flows to the third thread; (5) the third thread then loads y
= 1, and also issues a store x := 1, which flows to the memory; (6) the load
of x flows to the next level and gets satisfied by the x := 1 store of the
third thread; and (7) finally the x := 1 store of the first thread also flows
to the next level. The POP model is strictly weaker than the Flowing model,
and thus also allows this outcome.


\subsection{Fences and RMW}
\cite[Remark 2, After example 3.1]{DBLP:journals/pacmpl/PodkopaevLV19}: Aim:
allow the splitting of release writes and RMWs into release fences followed
by relaxed operations.  In RC11 [Lahav et al. 2017], as well as in C/C++11
[Batty et al. 2011], this rather intuitive transformation, as we found out,
is actually unsound.
\begin{gather*}
  y\GETS 1\SEMI
  x\REL\GETS 1
  \PAR
  r\GETS \FADD^{\modeRA,\modeRA}(x) \SEMI  %1
  x\GETS 3
  \PAR
  r\GETS x\ACQ \SEMI %3
  s\GETS y %0
  \\
  \hbox{\begin{tikzinline}[node distance=.5em and 1em]
      \event{a1}{\DW{y}{1}}{}
      \event{a2}{\DWRel{x}{1}}{right=of a1}
      \po{a1}{a2}
      \event{b1}{\DRAcq{x}{1}}{right=2em of a2}
      \event{b2}{\DWRel{x}{2}}{right=1.5em of b1}
      \rmw{b1}{b2}
      \event{b3}{\DW{x}{3}}{right=of b2}
      \wkx{b2}{b3}
      \event{c1}{\DRAcq{x}{3}}{right=2em of b3}
      \event{c2}{\DR{y}{0}}{right=of c1}
      \po{c1}{c2}
      \rf{b3}{c1}
      \rf{a2}{b1}
      \wky[out=170,in=10]{c2}{a1}
      \po[out=-15,in=-165]{b1}{b3}
   \end{tikzinline}}
\end{gather*}
(R)C11 disallows the annotated behavior, due in particular to the release sequence formed from the
release exclusive write to x in the second thread to its subsequent relaxed write. However, if we
split the increment to fencerel; a := FADDacq,rlx(x, 1) (which intuitively may seem stronger), the
release sequence will no longer exist, and the annotated behavior will be allowed. IMM overcomes
this problem by strengthening sw in a way that ensures a synchronization edge for the transformed
program as well
\begin{gather*}
  y\GETS 1\SEMI
  x\REL\GETS 1
  \PAR
  \FENCE^{\modeREL}\SEMI
  r\GETS \FADD^{\modeRA,\modeRLX}(x) \SEMI  %1
  x\GETS 3
  \PAR
  r\GETS x\ACQ \SEMI %3
  s\GETS y %0
  \\
  \hbox{\begin{tikzinline}[node distance=.5em and 1em]
      \event{a1}{\DW{y}{1}}{}
      \event{a2}{\DWRel{x}{1}}{right=of a1}
      \po{a1}{a2}
      \event{b0}{\DFS{\modeREL}}{right=2em of a2}
      \event{b1}{\DRAcq{x}{1}}{right=of b0}
      \event{b2}{\DW{x}{2}}{right=1.5em of b1}
      \rmw{b1}{b2}
      \event{b3}{\DW{x}{3}}{right=of b2}
      \wkx{b2}{b3}
      \event{c1}{\DRAcq{x}{3}}{right=2em of b3}
      \event{c2}{\DR{y}{0}}{right=of c1}
      \po{c1}{c2}
      \rf{b3}{c1}
      \rf[out=15,in=165]{a2}{b1}
      \po[out=-15,in=-165]{b0}{b2}
      \wky[out=170,in=10]{c2}{a1}
      \po[out=-15,in=-165]{b1}{b3}
      \po[out=-20,in=-160]{b0}{b3}
   \end{tikzinline}}
\end{gather*}

We seem to disallow both of those already.

In the case of a relaxed read in the RMW, the outcome is allowed in both
cases:
\begin{gather*}
  y\GETS 1\SEMI
  x\REL\GETS 1
  \PAR
  r\GETS \FADD^{\modeRLX,\modeRA}(x) \SEMI  %1
  x\GETS 3
  \PAR
  r\GETS x\ACQ \SEMI %3
  s\GETS y %0
  \\
  \hbox{\begin{tikzinline}[node distance=.5em and 1em]
      \event{a1}{\DW{y}{1}}{}
      \event{a2}{\DWRel{x}{1}}{right=of a1}
      \po{a1}{a2}
      \event{b1}{\DR{x}{1}}{right=2em of a2}
      \event{b2}{\DWRel{x}{2}}{right=1.5em of b1}
      \rmw{b1}{b2}
      \event{b3}{\DW{x}{3}}{right=of b2}
      \wkx{b2}{b3}
      \event{c1}{\DRAcq{x}{3}}{right=2em of b3}
      \event{c2}{\DR{y}{0}}{right=of c1}
      \po{c1}{c2}
      \rf{b3}{c1}
      \rf{a2}{b1}
      \wky[out=170,in=10]{c2}{a1}
      %\po[out=-15,in=-165]{b1}{b3}
      \wkx[out=-15,in=-165]{b1}{b3}
   \end{tikzinline}}
\end{gather*}
\begin{gather*}
  y\GETS 1\SEMI
  x\REL\GETS 1
  \PAR
  \FENCE^{\modeREL}\SEMI
  r\GETS \FADD^{\modeRLX,\modeRLX}(x) \SEMI  %1
  x\GETS 3
  \PAR
  r\GETS x\ACQ \SEMI %3
  s\GETS y %0
  \\
  \hbox{\begin{tikzinline}[node distance=.5em and 1em]
      \event{a1}{\DW{y}{1}}{}
      \event{a2}{\DWRel{x}{1}}{right=of a1}
      \po{a1}{a2}
      \event{b0}{\DFS{\modeREL}}{right=2em of a2}
      \event{b1}{\DR{x}{1}}{right=of b0}
      \event{b2}{\DW{x}{2}}{right=1.5em of b1}
      \rmw{b1}{b2}
      \event{b3}{\DW{x}{3}}{right=of b2}
      \wkx{b2}{b3}
      \event{c1}{\DRAcq{x}{3}}{right=2em of b3}
      \event{c2}{\DR{y}{0}}{right=of c1}
      \po{c1}{c2}
      \rf{b3}{c1}
      \rf[out=15,in=165]{a2}{b1}
      \po[out=-15,in=-165]{b0}{b2}
      \wky[out=170,in=10]{c2}{a1}
      %\po[out=-15,in=-165]{b1}{b3}
      \po[out=-20,in=-160]{b0}{b3}
      \wkx[out=-15,in=-165]{b1}{b3}
   \end{tikzinline}}
\end{gather*}
