\section{Sequential Composition}
\label{sec:semicolon}
We provide an alternative semantics that supports full sequential
composition, building $\sem{\aCmd\SEMI\bCmd}$ from $\sem{\aCmd}$ and
$\sem{\bCmd}$.  To simplify the definitions\footnote{This restriction can be
  removed as follows. Separate registers into two categories: those used for
  direct assignment ($\aReg\GETS\aExp$) and those used for reads
  ($\aReg\GETS \REF{\cExp}^{\amode}$).  When constructing a pomset, define
  \emph{program order} $(\xpox)$ in the obvious way.  Let $\bEv$ be
  \emph{visible} if it reads into some $\aReg$ and there is no $\aEv$ that
  reads into $\aReg$ such that $\bEv\xpox\aEv$.  Let $\SUBDRS{\aPS}$ be
  derived from visible reads (rather than all reads).  In item 5a of
  Definition \ref{def:semi:seq}, only consider events, $\bEv$, that are
  visible reads.}, we assume, without loss of generality
\cite{Rosen:1988:GVN:73560.73562}, that each register is assigned at most
once syntactically.  Since we exclude loops and functions, this trivially
ensures that each register is assigned at most once per pomset.

We refactor the syntax:
\begin{align*}
  \aCmd,\,\bCmd
  \BNFDEF& \SKIP
  \mkern-2mu\BNFSEP\mkern-2mu \FENCE^{\fmode}
  \mkern-2mu\BNFSEP\mkern-2mu \aReg\GETS\aExp
  \mkern-2mu\BNFSEP\mkern-2mu \aReg\GETS \REF{\cExp}^{\amode} 
  \mkern-2mu\BNFSEP\mkern-2mu \REF{\cExp}^{\amode}\GETS\aExp
  \\[-.5ex]
  \BNFSEP&\aCmd \PAR \bCmd
  \mkern-2mu\BNFSEP\mkern-2mu\aCmd \SEMI \bCmd
  \mkern-2mu\BNFSEP\mkern-2mu \VAR\aLoc\SEMI \aCmd
  \mkern-2mu\BNFSEP\mkern-2mu \IF{\aExp} \THEN \aCmd \ELSE \bCmd \FI
\end{align*}
\paragraph{Explicit Substitutions.}
Let $\aEExp$ range over \emph{extended expressions}, which may include memory
locations.  We introduce explicit substitutions over extended expressions,
following the conventions of \citet{DBLP:conf/icalp/RitterP97}:
\begin{gather*}
  \aLocReg\BNFDEF \aLoc \BNFSEP \aReg
  \qquad\quad
  \aSub,\,\bSub %,\, \SUBDRS{\dEvs} 
  \BNFDEF \SUBEMP \BNFSEP \SUBPAR{\aSub}{\aEExp/\aLocReg}
  \BNFSEP \aSub\SUBSEQ\aSub'
\end{gather*}
$\SUBEMP$ is the identity substitution.  We write
$\SUBPAR{\SUBEMP}{\aEExp/\aLocReg}$ as $\SUB{\aEExp/\aLocReg}$.

Application is written $\aSub\SUBAPP\aForm$.  We only apply substitutions to
formulae---which do not bind locations or registers.  The definition is
homomorphic over the syntax of formulae. For the basis, 
\begin{math}
  \SUBPAR{\aSub}{\aEExp/\bLocReg}\SUBAPP\aLocReg
\end{math}
is $\aEExp$ if $\bLocReg=\aLocReg$ and is $\aSub \SUBAPP\aLocReg$ otherwise.

Sequencing is defined so that
\begin{math}
  \aSub\SUBSEQ\SUB{\aEExp/\aLocReg}
  \allowbreak= 
  \SUBPAR{\aSub}{\aSub\SUBAPP\aEExp/\aLocReg}
\end{math}
and
\begin{math}
  (\beforeSub\SUBSEQ\afterSub)\SUBAPP\aForm = \beforeSub\SUBAPP(\afterSub\SUBAPP\aForm).
\end{math}

We say that $\aSub$ \emph{subsumes} $\bSub$ if for every $\aLocReg$, either
$\bSub\SUBAPP\aLocReg=\aSub\SUBAPP\aLocReg$ or $\bSub\SUBAPP\aLocReg=\aLocReg$.
For example, every substitution subsumes $\SUBEMP$.

We say that $\aSub$ is \emph{independent of $\aLoc$} if $\aSub\SUBAPP\aLoc=\aLoc$.

Let $\Sub$ be the set of all (explicit) substitutions.

\paragraph{Substitutions in the Model.}\

Change Definition~\ref{def:mmpomset}, of \emph{memory model pomset}, so that
$\labeling: \Event \fun (\Formulae\times\Act\times\Sub)$, from which we
derive the additional function $\labelingSub:\Event\fun\Sub$.

We write triples in $(\Formulae\times\Act\times\Sub)$ as
$(\aForm \mid \aAct\mid \aSub)$, eliding $\aForm$ when it is a tautology,
eliding $\aAct$ when it is the termination action, and eliding $\aSub$ when
it is the identity substitution.

As a notational convenience, let $\labelingSub(\aPS)$ return the substitution
of the termination event in $\aPS$, if one exists.  

We ignore substitutions, except on read events and termination events.  In
the definition of sequential composition, read substitutions are used to
calculate dependencies, as in item 5a of Definition \ref{def:prefix}.  The
terminal substitutions in $\aPSS^1$ are composed with the substitutions in
$\aPSS^2$ when calculating $(\aPSS^1\sequence\aPSS^2)$, and symmetrically.

Extend Definition \ref{def:closure:properties}, of relations between pomsets,
to include subsumption: $\aPS'$ \emph{subsumes} $\aPS$ if
$\Event'=\Event$, ${\le'}={\le}$, $\labelingForm'=\labelingForm$, and
$\labelingAct'=\labelingAct$ and $\labelingSub'(\aEv)$ subsumes
$\labelingSub(\aEv)$.

The semantics of programs is closed w.r.t.~\emph{reverse subsumption}: if
$\aPS\in\sem{\aCmd}$ and $\aPS$ is subsumed by $\aPS'$, then
$\aPS'\in\sem{\aCmd}$.

Subsumption is dual to implication: Strong\-er preconditions impose a greater
burden on the preceding code; stronger substitutions can better mitigate this
burden in following code.  In examples, we only show executions that are
implication and augmentation minimal; similarly, we only show executions that
are subsumption-maximal.

Change the partial function $\rreads$, from the data model, so that
$\rreads:\Act\fun(\Reg \times \Loc \times \Val)$.  When
$\rreads(\aAct) = (\aReg,\aLoc,\aVal)$, we say that $\aAct$ \emph{reads}
$\aVal$ \emph{from} $\aLoc$ \emph{into} $\aReg$.

The semantics satisfies the following invariant: If $\aPS$ is
subsumption\hyp{}maximal and $\labelingSub(\aPS)$ is defined, then for every
$\aEv$ that reads $\aLoc$ into $\aReg$, there are $\beforeSub$ and
$\afterSub$ such that
$\labelingSub(\aPS) = (\beforeSub\SUBSEQ\afterSub)$ and
$\labelingSub(\aEv)= (\beforeSub\SUBSEQ\SUB{\aLoc/\aReg}\SUBSEQ\afterSub)$.

Let $\SUBDRS{\aPS}$ be the substitution of values for registers that is
derived from the reads of $\aPS$ as follows:
$\SUBDRS{\aPS}\SUBAPP\aReg=\aVal$ if some event in $\aPS$ reads $\aVal$ into
$\aReg$, and $\SUBDRS{\aPS}\SUBAPP\aReg=\aReg$ otherwise.

\paragraph{Semantics of the Example Language.}\

As concrete syntax for read actions, we write
$(\DRreg[\amode]{\aReg}{\aLoc}{\aVal})$.

Change Definition \ref{def:rf}, of \emph{$\aLoc$-closed}, to require that
every substitution is independent of $\aLoc$.

Change Definition \ref{def:par}, of $\aPS' \in (\aPSS^1 \parallel \aPSS^2)$,
to additionally require that for all $\aEv\in\Event'$, either:
\begin{align*}
  &\labelingSub'(\aEv) \text{ is subsumed by both } \labelingSub^1(\aEv) \textand \labelingSub^2(\aEv),\\[-1ex]
  &\labelingSub'(\aEv) \text{ is subsumed by } \labelingSub^1(\aEv) \textand \aEv \not\in \Event^2,\; \textor\\[-1ex]
  &\labelingSub'(\aEv) \text{ is subsumed by } \labelingSub^2(\aEv) \textand \aEv \not\in \Event^1.
\end{align*}

Parallel composition, conditional and location binding are otherwise
unchanged from \textsection\ref{sec:model}.

Note that only shared register state is preserved by parallel composition.
For example,
\begin{math}
  (r\GETS 1\PAR r\GETS 1)\SEMI x\GETS r
\end{math}
can write $1$ to $x$, but
\begin{math}
  (r\GETS 1\PAR \SKIP)\SEMI x\GETS r
\end{math}
cannot.

To simplify the base cases, we use a literal notation for pomsets and define
$\CLOSE{\aPS}$ to be the smallest set that includes $\aPS$ and is closed
w.r.t.~prefixing, implication, augmentation, and reverse subsumption.
% : Let
% $\aPS''\in\CLOSE{\aPS}$ if there is $\aPS'\in\PRE(\aPS)$ such (1) $\aPS''$
% implies $\aPS'$, (2) $\aPS''$ is an augmentation of $\aPS'$ and (3) $\aPS''$
% is subsumed by $\aPS'$.
\begingroup
\allowdisplaybreaks
\begin{gather*}
  \begin{aligned}
  \sem{\SKIP} & \eqdef
  \CLOSE{\TIKZ{\final{f}{}{}}}
  \\  
  \sem{\aReg\GETS\aExp} & \eqdef
  \CLOSE{\TIKZ{\final{f}{\SUB{\aExp/\aReg}}{}}}
  \\
  \sem{\FENCE^{\fmode}} & =
  \CLOSE{\TIKZ{
      \event{a}{\DFS{\fmode}}{}
      \final{f}{}{right=of a}
      \po{a}{f}
    }} 
  \\
  \sem{\aReg\GETS\REF{\cExp}^\amode} & =
  \textstyle\bigcup_{\cVal,\aVal}\;
  \CLOSE{\TIKZ{
      \event{a}{\cExp=\cVal\mid\DRreg[\amode]{\aReg}{\REF{\cVal}}{\aVal}\mid \SUB{\REF{\cVal}/\aReg}}{}
      \final{f}{}{right=of a}
      \po{a}{f}
    }}
                                \\
  \sem{\REF{\cExp}^\amode\GETS\aExp} & =
  \;\;\textstyle\parallel_{\cVal,\aVal}\;
  \CLOSE{\TIKZ{
      \event{a}{\cExp=\cVal\land\aExp=\aVal \mid \DW[\amode]{\REF{\cVal}}{\aVal}}{}
      \final{f}{\SUB{\aExp/\REF{\cVal}}}{right=of a}
      \po{a}{f}
    }}
                                \\
  \sem{\aCmd \SEMI \bCmd} &= \sem{\aCmd} \sequence \sem{\bCmd}
    \end{aligned}
  \end{gather*}
\endgroup

Whereas a write introduces the substitution $\SUB{\aExp/\REF{\cVal}}$ on
the terminal event, a read introduces the substitution
$\SUB{\REF{\cVal}/\aReg}$ on the read event itself.

The most significant challenge is defining sequential composition.
Unfortunately, \emph{disjunction} and \emph{prefix weakening} (Definition
\ref{def:closure:properties}) do not come easily.

Recall \eqref{alanAddress} from \textsection\ref{sec:variants}:
\begin{math}
  \sem{a[r] \GETS 0\SEMI a[0]\GETS \BANG r}.
\end{math}
In the semantics, an event from the first statement can coalesce with an
event from the second.  Thus when computing
\begin{math}
  \sem{a[r] \GETS 0} \sequence \sem{a[0]\GETS \BANG r},
\end{math}
we must coalesce events with incompatible preconditions ($r{=}0$, $r{=}1$)
that occur on different sides of the sequencing operator.  This makes a
direct definition difficult.

Instead of a direct definition, we first construct the sequential composition
\emph{without} coalescing events, then close the resulting set of pomsets to
ensure the required properties.

There is also a challenge dealing with redundant write elimination:
\begin{math}
  \sem{x\GETS 1\SEMI x\GETS 2} 
\end{math}
should contain a pomset that includes only
\begin{math}
  (\DW{x}{2}).
\end{math}
We achieve this using the same strategy: closing after the construction.

Let $\cEv$ be an \emph{unused write} in $\aPS$ when it is a relaxed write to
some $\aLoc$ such that (1) $\cEv$ fulfills no reads, (2) there is some
$\bEv\gt\cEv$ that writes $\aLoc$, and (3) for every release $\aEv\gt\cEv$
there is some $\aEv\gt\bEv\ge\cEv$ that writes $\aLoc$.

Finally, there is a challenge in calculating the preconditions for the events
of $\aPSS^2$ in $(\aPSS^1\sequence\aPSS^2)$ when terminal event of $\aPSS^1$ is
missing, and symmetrically.  Again, we use the same strategy: We compute
sequential composition using completed executions, then prefix close.

\begin{definition}
  \label{def:semi:seq}
  Let $\DISJUNCT(\aPSS)$ be the least set that
  includes $\aPSS$ and that is closed w.r.t.~disjunction, prefixing, prefix weakening, and
  unused write removal.                  

  Let $(\aPSS^1 \sequence \aPSS^2)$ be the set
  \begin{math}
    \DISJUNCT(\aPSS')
  \end{math}
  where $\aPS'\in\aPSS'$ when there are $\aPS^1 \in \aPSS^1$,
  and $\aPS^2\in\aPSS^2$
  such that the following hold.

Let $\bEv$ range over $\Event^1$.  Let $\aEv$ range over $\Event^2$.  
      
\begin{enumerate}
\item[1.] $\Event' = \Event^2 \uplus \{\bEv\in\Event^1\mid\bEv\text{ is not a termination}\}$,
\item[2.] ${\le'}\supseteq{\le^1}\cup{\le^2}$, 
\item[3a.] $\labelingAct'(\bEv) = \labelingAct^1(\bEv)$,
\item[3b.] $\labelingForm'(\bEv)$ implies $\labelingForm^1(\bEv)$,
\item[3c.] $\labelingSub'(\bEv)$ is subsumed by $\labelingSub^1(\bEv)\SUBSEQ \labelingSub(\aPS^2)$,
\item[4a1.] $\labelingAct'(\aEv) = \labelingAct^2(\aEv)$,
\item[4a2.] $\labelingSub'(\aEv)$ is subsumed by $\labelingSub(\aPS^1)\SUBSEQ\labelingSub^2(\aEv)$,
\item[4bc.] $\labelingForm'(\aEv)$ implies
  $\SUBDRS{\aPS^1} \SUBAPP \bigl(\labelingSub(\aPS^1)\SUBAPP\labelingForm^2(\aEv)\bigr)$, 
\item[5a.] if $\bEv$ is a read and $\aEv$ is a write,
  then either $\bEv\le'\aEv$ or $\labelingForm'(\aEv)$ implies $\SUBDRS{\aPS^1}\SUBAPP \bigl(\labelingSub^1(\bEv)\SUBAPP\labelingForm^2(\aEv)\bigr)$,
\item[5b-f.] as before (see \textsection\ref{sec:model}-\ref{sec:variants}). %generalizing 5e: if $\aEv\in\Sub$ then $\dEv\le'\aEv$,
\end{enumerate}
\end{definition}

The item numbers are chosen to match those of the corresponding clauses in
\textsection\ref{sec:model}.  

Items 4b and 4c collapse into a single item here.
In 4bc, note that the domain of $\labelingSub(\aPS^1)$ is disjoint from the domain
of $\SUBDRS{\aPS^1}$, although registers in the domain of $\labelingSub(\aPS^1)$ may
appear in the expressions in the codomain of $\SUBDRS{\aPS^1}$.

Item 5 is morally unchanged.  
In 5a, recall that
$\labelingSub(\aPS) = (\beforeSub\SUBSEQ\afterSub)$ and
$\labelingSub(\aEv)= (\beforeSub\SUBSEQ\SUB{\aLoc/\aReg}\SUBSEQ\afterSub)$.
Note that
$\SUBDRS{\aPS^1}\SUBSEQ \labelingSub^1(\bEv)$ is insensitive to the value assigned to $\aReg$ by
$\SUBDRS{\aPS^1}$.

As a simple example, consider the following:
\begingroup
\allowdisplaybreaks
\begin{gather*}
  \begin{gathered}
    r\GETS y
    \\[-1ex]
    \hbox{\begin{tikzinline}[node distance=.2em]
      \event{a}{\DRreg{r}{y}{1} \mid \SUB{y/r}}{}
      \final{f}{}{below=of a}
      \end{tikzinline}}
  \end{gathered}
  \qquad
  \begin{gathered}
    x\GETS r
    \\[-1ex]
    \hbox{\begin{tikzinline}[node distance=.2em]
      \event{b}{r\EQ1 \mid \DW{x}{1}}{}
      \final{f}{r\EQ1 \mid \SUB{r/x}}{below=of b}
      \end{tikzinline}}
  \end{gathered}
  \qquad
  \begin{gathered}
    s\GETS x
    \\[-1ex]
    \hbox{\begin{tikzinline}[node distance=.2em]
      \event{c}{\DFR{}}{}
      \final{f}{\SUB{x/s}}{below=of c}
      \end{tikzinline}}
  \end{gathered}
  \qquad
  \begin{gathered}
    z\GETS s
    \\[-1ex]
    \hbox{\begin{tikzinline}[node distance=.2em]
      \event{d}{\DW{z}{1}}{}
      \final{f}{\SUB{s/z}}{below=of d}
      \end{tikzinline}}
  \end{gathered}
  \\
  \begin{gathered}
    r\GETS y\SEMI x\GETS r \SEMI s\GETS x \SEMI z\GETS s
    \\[-1ex]
    \hbox{\begin{tikzinline}[node distance=1em]
        \event{a}{\DRreg{r}{y}{1}\mid \SUB{y/r, r/x, x/s, s/z}}{}
        \event{b}{\DW{x}{1}}{right=of a}
        \event{c}{\DFR{}}{right=of b}
        \event{d}{\DW{z}{1}}{right=of c}
        \final{f}{\SUB{r/x, x/s, s/z}}{right=of d}
        \po{a}{b}
        \po[out=-10,in=-160]{a}{d}
      \end{tikzinline}}
  \end{gathered}
\end{gather*}

To see the need for parallel substitution of all register values via $\SUBDRS{\aPS^1}$ in 4bc, consider that the
precondition of $(\DW{y}{1})$ must be $\FALSE$ after composing the following:
\begin{gather*}
  \begin{gathered}
    r\GETS x\SEMI s\GETS z
    \\[-1ex]
    \hbox{\begin{tikzinline}[node distance=.5em]
      \event{a}{\DRreg{r}{x}{1} \mid \SUB{x/r}}{}
      \event{b}{\DRreg{s}{z}{2} \mid \SUB{z/s}}{right=of a}
      \final{f}{}{right=of b}
      \end{tikzinline}}
  \end{gathered}
  \qquad
  \begin{gathered}
     \IF{r{=}s}\THEN y\GETS1\FI
    \\[-1ex]
    \hbox{\begin{tikzinline}[node distance=.5em]
      \event{c}{r{=}s\mid\DW{y}{1}}{}
      \final{f}{r{=}s\mid\SUB{1/y}}{right=of c}
      \end{tikzinline}}
  \end{gathered}
  \\
  \begin{gathered}
    r\GETS x\SEMI s\GETS z \SEMI \IF{r{=}s}\THEN y\GETS1\FI
    \\[-1ex]
    \hbox{\begin{tikzinline}[node distance=.5em]
      \event{a}{\DRreg{r}{x}{1} \mid \SUB{x/r}}{}
      \event{b}{\DRreg{s}{z}{2} \mid \SUB{z/s}}{right=of a}
      \nonevent{c}{1{=}2\mid\DW{y}{1}}{right=of b}
      \nonfinal{f}{1{=}2\mid\SUB{1/y}}{right=of c}
      \end{tikzinline}}
  \end{gathered}
\end{gather*}
This pomset candidate does not satisfy the \emph{compatibility} requirement
of Definition \ref{def:mmpomset}.  However, note that
$(\IF{r{=}s}\THEN y\GETS1\FI)$ is shorthand for
$(\IF{r{=}s}\THEN y\GETS1\ELSE\SKIP\FI)$, and thus we do have a valid pomset
for this composition, even with this choice of read values:
\begin{gather*}
  \begin{gathered}
    r\GETS x\SEMI s\GETS z
    \\[-1ex]
    \hbox{\begin{tikzinline}[node distance=.5em]
      \event{a}{\DRreg{r}{x}{1} \mid \SUB{x/r}}{}
      \event{b}{\DRreg{s}{z}{2} \mid \SUB{z/s}}{right=of a}
      \final{f}{}{right=of b}
      \end{tikzinline}}
  \end{gathered}
  \qquad
  \begin{gathered}
     \IF{r{=}s}\THEN y\GETS1\FI
    \\[-1ex]
    \hbox{\begin{tikzinline}[node distance=.5em]
      \final{f}{r{\neq}s}{}
      \end{tikzinline}}
  \end{gathered}
  \\
  \begin{gathered}
    r\GETS x\SEMI s\GETS z \SEMI \IF{r{=}s}\THEN y\GETS1\FI
    \\[-1ex]
    \hbox{\begin{tikzinline}[node distance=.5em]
      \event{a}{\DRreg{r}{x}{1} \mid \SUB{x/r}}{}
      \event{b}{\DRreg{s}{z}{2} \mid \SUB{z/s}}{right=of a}
      \final{f}{1{\neq}2}{right=of b}
      \end{tikzinline}}
  \end{gathered}
\end{gather*}

The see the need for substitutions on read actions, used in 5a, consider the
following, where
$\bSub=\SUB{0/x,\allowbreak x/r,\allowbreak {-}2/x}$:
\begin{gather*}
  \begin{gathered}
    x\GETS 0\SEMI r\GETS x\SEMI x\GETS{-2} 
    \\[-1ex]
    \hbox{\begin{tikzinline}[node distance=1em]
      \event{a}{\DW{x}{0}}{}
      \event{b}{\DRreg{r}{x}{1} \mid \bSub}{right=of a}
      \event{c}{\DW{x}{{-}2}}{right=of b}
      \wk{a}{b}
      \wk{b}{c}
      \final{f}{\SUB{{-2}/x}}{right=.5em of c}
      \end{tikzinline}}
  \end{gathered}
  \quad
  \begin{gathered}
    \IF{r{\geq}0}\THEN y\GETS1\FI
    \\[-1ex]
    \hbox{\begin{tikzinline}[node distance=1em]
      \event{d}{r{\geq}0\mid\DW{y}{1}}{}
      \final{f}{r{\geq}0\mid\SUB{1/y}}{right=.5em of d}
      \end{tikzinline}}
  \end{gathered}
  \\
  \begin{gathered}
    x\GETS 0\SEMI r\GETS x\SEMI x\GETS{-2} \SEMI \IF{r{\geq}0}\THEN y\GETS1\FI
    \\[-1ex]
    \hbox{\begin{tikzinline}[node distance=1em]
      \event{a}{\DW{x}{0}}{}
      \event{b}{\DRreg{r}{x}{1} \mid \SUBPAR{\bSub}{1/y}}{right=of a}
      \event{c}{\DW{x}{{-}2}}{right=of b}
      \wk{a}{b}
      \wk{b}{c}
      \event{d}{0{\geq}0\mid\DW{y}{1}}{right=of c}
      \final{f}{0{\geq}0\mid\SUB{{-2}/x,1/y}}{right=of d}
      \end{tikzinline}}
  \end{gathered}
\end{gather*}
\endgroup

The semantics validates expected equations, such as
\begin{math}
  \sem{\aCmd_1\SEMI\aCmd_2}\sequence\sem{\aCmd_3} =
  \sem{\aCmd_1}\sequence\sem{\aCmd_2\SEMI\aCmd_3},
\end{math}
\begin{math}
  \sem{\IF{\aExp} \THEN \aCmd \SEMI\bCmd_1\ELSE \aCmd\SEMI\bCmd_2\FI} =
  \sem{\aCmd}\sequence\sem{\IF{\aExp} \THEN \bCmd_1 \ELSE \bCmd_2\FI},
\end{math}
and
\begin{math}
  \sem{\IF{\aExp} \THEN \aCmd_1 \SEMI\bCmd\ELSE\allowbreak \aCmd_2\SEMI\bCmd\FI} =
  \sem{\IF{\aExp} \THEN \aCmd_1 \ELSE \aCmd_2\FI}\sequence\sem{\bCmd}.
\end{math}

The semantics is equivalent to that of the main text.

Let $\killS\aPSS$ be the set $\aPSS'\subseteq\aPSS$ where $\aPS'\in\aPSS'$
if $\labelingSub(\aEv')=\SUBEMP$, for every $\aEv'\in\Event'$.

\begin{proposition}
  \label{thm:seq}
  Let $\semold{\aCmd}$ be the semantics of \textsection\ref{sec:model},
  adopting the read actions of this section---the read substitution
  being $\SUBEMP$.
\begin{displaymath}
  \semold{\aCmd} = \killS\sem{\aCmd}
\end{displaymath}
\end{proposition}
 
