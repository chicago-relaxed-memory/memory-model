\section{Logic}
\label{sec:logic}

\newcommand{\closed}{\textsf{closed}}
\newcommand{\pLTL}{\textsf{PLTL}}
\newcommand{\once}{\Diamond^{-1}}
\newcommand{\always}{\Box^{-1}}
\newcommand{\afo}{\phi}
\newcommand{\bfo}{\psi}
\newcommand{\mods}{\textsf{Models}}

In this section, we develop sufficient logical infrastructure to prove that
our semantics disallows thin air executions.  We present a variant of the
TAR-pit example from \S\ref{sec:relaxed-memory} which poses difficulties
under many speculative semantics.

We adapt past linear temporal logic (\pLTL)
\cite{Lichtenstein:1985:GP:648065.747612} to pomsets by dropping the previous
instant operator and adopting strict versions of the temporal operators.
The atoms of our logic are write and read events.
% \begin{displaymath}
%   \afo \QUAD::=\QUAD
%   \DR{\aLoc}{\aVal}
%   \mid
%   \DW\aLoc\aVal
%   \afo \wedge\bfo
%   \mid \lnot \afo
%   \once\afo
%   \mid \always\afo
% \end{displaymath}
%\begin{definition} %[Satisfaction]
  Given an pomset $\aPS$ and event $\aEv$, define:
  \begin{displaymath}
    \begin{array}{lrl}
      \aPS,\aEv &\models& \DW{\aLoc}{\aVal}, \text{ if } \labeling(\aEv) =  (\TRUE, \DW{\aLoc}{\aVal}) \\
      \aPS,\aEv &\models& \DR{\aLoc}{\aVal}, \text{ if } \labeling(\aEv) =  (\TRUE, \DR{\aLoc}{\aVal}) \\
      \aPS,\aEv &\models& \afo\land\bfo, \text{ if } \aPS,\aEv \models  \afo \text{ and } \aPS,\aEv \models  \bfo \\
      \aPS,\aEv &\models& \TRUE\\
      \aPS,\aEv &\models& \lnot\afo, \text{ if } \aPS,\aEv \not\models \afo \\
      %\aPS,\aEv &\models& \once\afo, \text{if } (\exists \bEv \le \aEv, \bEv\not=\aEv)  \aPS,\bEv \models \afo \\
      \aPS,\aEv &\models& \always\afo, \text{ if } (\forall \bEv \le \aEv,\,  \bEv\not=\aEv)\; \aPS,\bEv \models \afo
    \end{array} 
  \end{displaymath}
  Define $\aPS \models \afo$ if
  $(\forall \aEv \in \Event) \;\aPS,\aEv \models\afo$ and $\aPSS\models \afo$
  if $(\forall \aPS \in \aPSS)\; \aPS \models\afo$.
%\end{definition}

Let $\once\afo$ be defined as $\lnot(\always\lnot\afo)$. 
In addition, let $\FALSE$, $\lor$ and $\Rightarrow$ be defined in the
standard way.
% $\afo\lor\bfo$ for $\lnot(\lnot \afo \land \lnot \bfo)$,
% and $\afo \Rightarrow \bfo$ for $\lnot \afo \lor \bfo$.

The past operators do not include the current instant, and thus 
they do \emph{not} satisfy the rule
\begin{math}
  \always\afo\Rightarrow\once\afo.
\end{math}
However, they do satisfy:
% \begin{align*}  
%   \frac{\aPS \models \afo \Rightarrow\once{\afo}}{\aPS \models \lnot \afo}\text{(Coinduction)}
%   &&
%   \frac{\aPS \models \always\afo \Rightarrow\afo}{\aPS \models \afo}\text{(Induction)}
% \end{align*}
\begin{gather*}
  \tag{Coinduction}
  (\afo \Rightarrow\once{\afo}) \Rightarrow\lnot \afo
  \\
  \tag{Induction}
  (\always\afo \Rightarrow\afo) \Rightarrow\afo
\end{gather*}
% \begin{description}
% \item[Coinduction.]
%   \begin{math}
%     (\afo \Rightarrow\once{\afo}) \Rightarrow\lnot \afo
%   \end{math}
% \item[Induction.] 
%   \begin{math}
%     (\always\afo \Rightarrow\afo) \Rightarrow\afo
%   \end{math}
% \end{description}
Note that $\aPS \models \afo \land \always\afo$ whenever $\aPS \models \afo$.

We now present two proof rules.  The first rule captures the semantics of
local variables.  Define
\begin{math}
  \closed(\aLoc) = (\DR{\aLoc}{\aVal} \Rightarrow \once \DW{\aLoc}{\aVal}).
\end{math}
Although this definition does not mention intervening writes, it is
sufficient for our example.  It is straightforward to establish that
following rule is sound:
\begin{displaymath}
  \tag{Closing $\aLoc$}
  \frac{
    \afo \text{ is independent of } \aLoc
    \qquad
    \aPS \models \closed(\aLoc) \Rightarrow \afo
  }{
    \nu \aLoc \st \aPS \models \afo
  }
\end{displaymath}

The second rule describes composition, in the style of Abadi and
Lamport~\cite{Abadi:1993:CS:151646.151649}.  To simplify the presentation, we
consider the special case with a single invariant.
% We view the
% composition result as capturing key aspects of no-ThinAirRead, as will become
% clearer in the examples below.
In order to state the theorem, we generalize the satisfaction relation to
include environment assumptions.  Let
\begin{math}
  \mods{(\afo)} = \{ \aPSS \mid \aPSS \models \afo \}
\end{math}
be the set of pomsets that satisfy $\afo$.  We say that $\afo$ is prefix
closed if $\mods{(\afo)}$ is prefix-closed\footnote{$\aPS'$ is a prefix of
  $\aPS$ if $\Event'\subseteq\Event$, $\aEv\in\Event'$ and $ \bEv\le\aEv$
  imply $\bEv\in\Event'$, and $(\labeling',\leq',\gtN')$ coincide with
  $(\labeling,\leq,\gtN)$ for elements of $\Event'$.}.
  % when $\aEv,\,\bEv\in\Event'$: $\labeling'(\aEv)=\labeling(\aEv)$,
  % $\aEv\leq'\bEv$ iff $\aEv\leq\bEv$, and $\aEv\gtN'\bEv$ iff
  % $\aEv\gtN\bEv$. 
\begin{noenv}
  Define
  \begin{math}
    \afo, \aPSS \models \bfo  \text{ if } \mods{(\afo)} \parallel \aPSS \models \bfo.
  \end{math}
\end{noenv}
\begin{proposition}%[Composition]
  Let $\afo$ be prefix-closed.  Let $\aPSS_1, \aPSS_2$ be
  augmentation-closed.%\footnote{$\aPS'$ is an augmentation of $\aPS$ if
 %   $\Event'=\Event$, $\aEv\le\bEv$ implies $\aEv\le'\bEv$, $\aEv\gtN\bEv$
 %   implies $\aEv\gtN'\bEv$, and
 %   % $\labeling'(\aEv)=\labeling(\aEv)$
 %   if $\labeling(\aEv) = (\bForm \mid \bAct)$ then
 %   $\labeling'(\aEv) = (\bForm' \mid \bAct)$ where $\bForm'$ implies
 %   $\bForm$.}
  Then:
  \begin{displaymath}
    \tag{Composition}
    \frac{
      \afo, \aPSS_1 \models\afo
      \qquad
      \afo, \aPSS_2 \models\afo
    }{\aPSS_1 \parallel \aPSS_2 \models \afo}
  \end{displaymath}
\end{proposition}
\begin{proof}[Proof sketch]
  We will show that all prefixes in the prefix closures of
  $\aPSS_1 \parallel \aPSS_2$ satisfy the required property.  Proof proceeds
  by induction on prefixes of $\aPS \in \aPSS_1 \parallel \aPSS_2$.

  The case for empty prefix  follows from assumption that  $\afo$ is prefix closed.  

  For the inductive case, consider %$\aPS$ in the prefix closure of $\aPSS_1 \parallel \aPSS_2$, i.e.
  $\aPS \in \aPS_1 \parallel \aPS_2$ where
  $\aPS_i \in \aPSS_i$.  Since $\aPSS_1$ and $\aPSS_2$ are augmentation
  closed, we can assume that the restriction of $\aPS$ to the events of
  $\aPS_i$ coincides with $\aPS_i$, for $i=1,2$.
  %
  Consider a prefix $\aPS'$ derived by removing a maximal element $\aEv$ from
  $\aPS$.  Suppose $\aEv$ comes from $\aPS_1$ (the other case is
  symmetric). Since $\aPS_2$ is a prefix of $\aPS'$ and $\aPS' \models \afo$
  by induction hypothesis, we deduce that $\aPS_2 \models \afo$.
  % Thus, $\aPS_2 \in \mods{(\afo)}$.
  Since $\aPS_1 \in \aPSS_1$, by assumption $\afo, \aPSS_1 \models\afo$ we
  deduce that $\aPS \models \afo$.
\end{proof}

We now turn the conditional TAR-pit program, which is a variant of \cite[Figure 8]{DBLP:journals/toplas/Lochbihler13}:
\begin{multline*}
  \VAR x\GETS0\SEMI \VAR y\GETS0\SEMI \VAR z\GETS0\SEMI  %\\[-.5ex]
  (
    y\GETS x
  \PAR
    \IF(z)\THEN x\GETS1 \ELSE x\GETS y\SEMI a\GETS y \FI
  \PAR
    z\GETS1
  )
\end{multline*}
This program is allowed to write $1$ to $a$ under many speculative
memory models
\cite{Manson:2005:JMM:1047659.1040336,DBLP:conf/esop/JagadeesanPR10,DBLP:conf/popl/KangHLVD17},
even though the read of $1$ from $y$ in the else branch of the second
thread arises out of thin air.   In contrast, we prove the formula
\begin{math}
  \lnot\once(\DW{a}{1})
\end{math}
holds for the models of this program in our semantics.  We start with the following invariant,
which holds for each of the three threads, and thus, by composition, for the
aggregate program:
\begin{align*}
  &[\once(\DW{y}{1}) \Rightarrow \once(\DR{x}{1})]
  \land\\[-.5ex]
  &[\once(\DW{a}{1}) \Rightarrow (\once(\DR{y}{1}) \land \always(\DW{x}{1} \Rightarrow \once(\DR{y}{1})))]
\end{align*}
Closing $y$, we have,
\begin{math}
  \once(\DR{y}{1}) \Rightarrow \once(\DW{y}{1}) % \Rightarrow \once(\DR{x}{1})
\end{math}
which we substitute into the left conjunct to get:
\begin{displaymath}
  \once(\DR{y}{1}) \Rightarrow \once(\DR{x}{1})
\end{displaymath}
which in turn we substitute into the right conjunct to get:
\begin{displaymath}
  \once(\DW{a}{1}) \Rightarrow (\once(\DR{x}{1}) \land \always(\DW{x}{1} \Rightarrow \once(\DR{x}{1})))
\end{displaymath}
Closing $x$, we can replace $\once(\DR{x}{1})$ with $\once(\DW{x}{1})$:
\begin{displaymath}
  \once(\DW{a}{1}) \Rightarrow (\once(\DW{x}{1}) \land \always(\DW{x}{1} \Rightarrow \once(\DW{x}{1})))
\end{displaymath}
Applying coinduction to the right conjunct, we have:
\begin{displaymath}
  \once(\DW{a}{1}) \Rightarrow (\once(\DW{x}{1}) \land \always(\lnot \DW{x}{1}))
\end{displaymath}
Simplifying, we have, as required:  
\begin{displaymath}
  \once(\DW{a}{1}) \Rightarrow \FALSE
\end{displaymath}


\subsection{RNG example}
\begin{verbatim}
y=x+1; a=y || x=y
prove a!=2

Wyv_1 /\ Wyv_2 => v_1 == v_2 (and maybe v_1==0 \/ v_2==0)
Wx1 => <>-1 Ry1
Wy1 => <>-1 Rx1

\end{verbatim}

% Local Variables:
% mode: latex
% TeX-master: "paper"
% End:
