\section{Understanding ``Thin Air Reads'' using Temporal Logic}
\label{sec:logic}

A significant challenge for a software memory model is to relax order enough
to allow efficient implementation without admitting anomalous
behaviors---called \emph{out of thin air} (\oota) in the literature
\cite{vacuous,DBLP:conf/esop/BattyMNPS15,BoehmOOTA}.  The most famous example is:
\begin{align*}
  \label{OOTA3}\tag{\textsc{oota2}}
  y\GETS0\SEMI 
  y\GETS x
  \!\PAR\!
  x\GETS0\SEMI
  r\GETS y\SEMI x\GETS r  
  &&
  %\nonumber
  \smash{\hbox{\begin{tikzinline}[node distance=1.2em]
  \event{rx}{\DR{x}{1}}{}
  \event{wy}{\DW{y}{1}}{right=of rx}
  \po{rx}{wy}
  \event{y0}{\DW{y}{0}}{left=of rx}
  \event{x0}{\DW{x}{0}}{right=2em of wy}
  \event{ry}{\DR{y}{1}}{right=2em of x0}
  \event{wx}{\DW{x}{1}}{right=of ry}
  \po{ry}{wx}
  \rf[out=10,in=170]{wy}{ry}
  \rf[out=170,in=10]{wx}{rx}
  \wk[out=-15,in=-165]{y0}{wy}
  \wk[out=-15,in=-165]{x0}{wx}
    \end{tikzinline}}}
\end{align*}
Although Java does not allow \oota{} behaviors of \ref{OOTA3},
\citet{DBLP:journals/toplas/Lochbihler13} showed that it does allow \oota\
behaviors of \ref{OOTA1}, from \textsection\ref{sec:intro}.
\citet{DBLP:conf/lics/JeffreyR16} described a logic that rules out \ref{OOTA3} but not \ref{OOTA1}.  In this section, we  provide a more accurate test of \oota{} behaviors  by enhancing their logic with temporal features.

On first read, we suggest that readers skip to the examples and the
discussion that follows, coming back to the details of the logic as
necessary.  Example~\ref{ex:thin} discusses the canonical \oota{} example
\ref{OOTA3}; the analysis is trivial and well-known
\cite{DBLP:journals/lmcs/JeffreyR19, DBLP:conf/popl/KangHLVD17}.
Example~\ref{ex:lochb} is more interesting.  It discusses a variant of
\citeauthor{DBLP:journals/toplas/Lochbihler13}'s example \ref{OOTA1},
from the introduction.
% In this case, the violation is a subtler
% temporal property.  We develop a logic sufficient to prove that our semantics
% disallows \oota\ on \eqref{lochbihler}.  
The logic given here is not meant to be definitive; on page
\pageref{page:logic:limits}, we discuss \oota{} examples that require
non-trivial extensions
\cite{DBLP:conf/esop/SvendsenPDLV18,DBLP:journals/pacmpl/ChakrabortyV19}.

We adapt past linear temporal logic (\pLTL)
\cite{Lichtenstein:1985:GP:648065.747612} to pomsets by dropping the previous
instant operator and adopting strict versions of the temporal operators.
The atoms of our logic are write and read events.
% \begin{displaymath}
%   \afo \QUAD::=\QUAD
%   \DR{\aLoc}{\aVal}
%   \mid
%   \DW\aLoc\aVal
%   \afo \wedge\bfo
%   \mid \lnot \afo
%   \once\afo
%   \mid \always\afo
% \end{displaymath}
%\begin{definition} %[Satisfaction]
Given a pomset $\aPS$ and event $\aEv$, define:\footnote{Let $\FALSE$, $\lor$,
  $\Rightarrow$ and $\once$ as usual;
  for example,
  $\once\afo = \lnot(\always\lnot\afo)$.}
\begin{displaymath}
  \renewcommand{\arraycolsep}{.2ex}
    \begin{array}{lrll}
      \aPS,\aEv &\models& \DW{\aLoc}{\aVal} &\text{ if } \labelingAct(\aEv) = \DW{\aLoc}{\aVal} \text{ and } \TRUE \text{ implies } \labelingForm(\aEv) \\
      \aPS,\aEv &\models& \DR{\aLoc}{\aVal} &\text{ if } \labelingAct(\aEv) = \DR{\aLoc}{\aVal} \text{ and } \TRUE \text{ implies } \labelingForm(\aEv) \\
      \aPS,\aEv &\models& \afo\land\bfo &\text{ if } \aPS,\aEv \models  \afo \text{ and } \aPS,\aEv \models  \bfo \\
      \aPS,\aEv &\models& \TRUE\\
      \aPS,\aEv &\models& \lnot\afo &\text{ if } \aPS,\aEv \not\models \afo \\
      \aPS,\aEv &\models& \always\afo &\text{ if } \forall \bEv \lt \aEv.\; \aPS,\bEv \models \afo\\
      \aPS,\aEv &\models& \once\afo &\text{ if } \exists \bEv \lt \aEv.\;  \aPS,\bEv \models \afo 
    \end{array} 
  \end{displaymath}

  % \begin{definition}
  Let $\aPS \models \afo$ if
  $\aPS,\aEv \models\afo$, for all $\aEv \in \Event$.

  Let $\aPSS\models \afo$
  if $\aPS \models\afo$, for all $\aPS \in \aPSS$.
  
Let
  \begin{math}
    \afo, \aPSS \models \bfo  \text{ if } \{ \aPS \mid \aPS \models \afo \} \parallel \aPSS \models \bfo.
  \end{math}
%\end{definition}

  Let $\afo$ be \emph{downclosed} when
  $\{ \aPS \mid \aPS \models \afo \}$ is.

% Thus, $\aPS\models \afo \land \always\afo$ whenever $\aPS \models
  % \afo$. This fact relies on the use of universal quantification in the definition.

% We define other connectives as standard:
% $\once\afo = \lnot(\always\lnot\afo)$,
% %$\FALSE = \lnot(\TRUE)$
% $\afo\lor\bfo = \lnot(\lnot(\afo)\land\lnot(\bfo))$, and
% $\afo\Rightarrow\bfo = \lnot(\afo) \lor\ \bfo$.
% \begin{displaymath}
% \begin{array}{lrl}
% \once\afo &=& \lnot(\always\lnot\afo) \\
% \FALSE &=& \lnot(\TRUE) \\
% \afo\lor\bfo &=& \lnot(\lnot(\afo)\land\lnot(\bfo)) \\
% \afo\Rightarrow\bfo &=& \lnot(\afo) \lor\ \bfo
% \end{array}
% \end{displaymath}
%Let $$ be defined as $$. 
%In addition, let $\FALSE$, $\lor$ and $$ be defined in the
%standard way.
% $\afo\lor\bfo$ for $\lnot(\lnot \afo \land \lnot \bfo)$,
% and $\afo \Rightarrow \bfo$ for $\lnot \afo \lor \bfo$.
  The past operators do not include the current instant, and so
  do \emph{not} satisfy
  $(\always\afo\Rightarrow\once\afo)$.\footnote{The order-minimal elements always validate
    $\always\afo$ and invalidate
    $\once\afo$.}
  However, the following hold:
% \begin{align*}  
%   \frac{\aPS \models \afo \Rightarrow\once{\afo}}{\aPS \models \lnot \afo}\text{(Coinduction)}
%   &&
%   \frac{\aPS \models \always\afo \Rightarrow\afo}{\aPS \models \afo}\text{(Induction)}
% \end{align*}
% \begin{lemma}
% Given an pomset $\aPS$.  
\begin{align*}
  \tag{Induction}
  \aPS \models& (\always\afo \Rightarrow\afo) \Rightarrow\afo
  \\[-1ex]
  \tag{Coinduction}
  \aPS \models& (\afo \Rightarrow\once{\afo}) \Rightarrow\lnot \afo
  \\[-1ex]
  \tag{Weakening}
  \aPS \models& (\afo \Rightarrow\once{\bfo}) \Rightarrow (\once\afo \Rightarrow\once{\bfo})
\end{align*}
% \end{lemma}
% \begin{proof}
% We prove that any node in a pomset satisfies these formulas.  
%The proof for both rules proceeds by induction on the length of the maximal path from a root to a node. 
%\end{proof}

% \begin{description}
% \item[Coinduction.]
%   \begin{math}
%     (\afo \Rightarrow\once{\afo}) \Rightarrow\lnot \afo
%   \end{math}
% \item[Induction.] 
%   \begin{math}
%     (\always\afo \Rightarrow\afo) \Rightarrow\afo
%   \end{math}
% \end{description}


%We now present two proof rules for programs. 

%\paragraph*{Proof rules for programs}
We present two proof rules. 
The first provides a logical view of \emph{$\aLoc$-closure} (Definition~\ref{def:rf}):
%The soundness proof is straightforward.
% \begin{math}
%   \closed(\aLoc) = (\DR{\aLoc}{\aVal} \Rightarrow \once \DW{\aLoc}{\aVal}).
% \end{math}
% Although this definition does not mention intervening writes, it is
% sufficient for our example.  
\begin{displaymath}
  %\tag{Closing $\aLoc$}
  \frac{
    \afo \text{ is independent of } \aLoc
    \qquad
    %\aPS \models \closed(\aLoc) \Rightarrow \afo
    \aPS \models (\DR{\aLoc}{\aVal} \Rightarrow \once \DW{\aLoc}{\aVal}) \Rightarrow \afo
  }{
    \nu \aLoc \DOT \aPS \models \afo
  }
\end{displaymath}
%It is straightforward to establish that this rule is sound.
% Although it does
% not mention intervening writes, the rule is sufficient for our examples.

The second rule describes concurrent composition, in the style of~\citet{Abadi:1993:CS:151646.151649}.  To simplify the presentation, we
consider the special case with a single invariant.
% We view the
% composition result as capturing key aspects of no-ThinAirRead, as will become
% clearer in the examples below.
% In order to state the theorem, we generalize the satisfaction relation to
% include environment assumptions.

\begin{proposition}%[Composition]
  Let $\afo$ be downclosed.  Let $\aPSS_1, \aPSS_2$ be
  augmentation\hyp{}closed. %\footnote{$\aPS'$ is an augmentation of $\aPS$ if
 %   $\Event'=\Event$, $\aEv\le\bEv$ implies $\aEv\le'\bEv$, $\aEv\gtN\bEv$
 %   implies $\aEv\gtN'\bEv$, and
 %   % $\labeling'(\aEv)=\labeling(\aEv)$
 %   if $\labeling(\aEv) = (\bForm \mid \bAct)$ then
 %   $\labeling'(\aEv) = (\bForm' \mid \bAct)$ where $\bForm'$ implies
 %   $\bForm$.}
  Then:
  \begin{displaymath}
    %\tag{Composition}
    \frac{
      \afo, \aPSS_1 \models\afo
      \qquad
      \afo, \aPSS_2 \models\afo
    }{\aPSS_1 \parallel \aPSS_2 \models \afo}
  \end{displaymath}
\end{proposition}
\begin{proof}[Proof sketch]
  We will show that all downsets in the downset closures of
  $\aPSS_1 \parallel \aPSS_2$ satisfy the required property.  Proof proceeds
  by induction on downsets of $\aPS \in \aPSS_1 \parallel \aPSS_2$.
  %
  The case for empty downset  follows from assumption that  $\afo$ is downset closed.  
  %
  For the inductive case, consider %$\aPS$ in the downset closure of $\aPSS_1 \parallel \aPSS_2$, i.e.
  $\aPS \in \aPS_1 \parallel \aPS_2$ where
  $\aPS_i \in \aPSS_i$.  Since $\aPSS_1$ and $\aPSS_2$ are augmentation
  closed, we can assume that the restriction of $\aPS$ to the events of
  $\aPS_i$ coincides with $\aPS_i$, for $i=1,2$.
  %
  Consider a downset $\aPS'$ derived by removing a maximal element $\aEv$ from
  $\aPS$.  Suppose $\aEv$ comes from $\aPS_1$ (the other case is
  symmetric). Since $\aPS_2$ is a downset of $\aPS'$ and $\aPS' \models \afo$
  by induction hypothesis, we deduce that $\aPS_2 \models \afo$.
  % Thus, $\aPS_2 \in \mods{(\afo)}$.
  Since $\aPS_1 \in \aPSS_1$, by assumption $\afo, \aPSS_1 \models\afo$ we
  deduce that $\aPS \models \afo$.
\end{proof}

% The logic is defined with respect to downclosed sets, but $\sem{\aCmd}$
% includes only completed pomsets.  For reasoning in the logic, we downclose
% the semantics, considering pomsets that may not be completed.  Let
% $\semdown{\aCmd}=\{\aPS'\mid\aPS'$ is a downset of some
% $\aPS \in \sem{\aCmd}\}$.

\begin{example}
\label{ex:thin}
With all variables initialized to $0$, we show that \ref{OOTA3}
satisfies
\begin{math}
  \lnot\DW{x}{1}.
\end{math}

We start with the invariant:
\begin{displaymath}
  [\DW{x}{1}\Rightarrow\once\DR{y}{1}]
  \land
  [\DW{y}{1}\Rightarrow\once\DR{x}{1}]
\end{displaymath}
This invariant holds for each thread; thus, it holds for the
aggregate program by composition.  Closing $y$ yields
\begin{math}
  \DR{y}{1} \Rightarrow \once\DW{y}{1}.
\end{math}
Weakening the right conjunct: % yields
\begin{math}
  \once\DW{y}{1}\Rightarrow\once\DR{x}{1}.
\end{math}
Chaining these together: %yields
\begin{math}
  \DR{y}{1} \Rightarrow \once\DR{x}{1}.
\end{math}
Weakening:  %yields
\begin{math}
  \once\DR{y}{1} \allowbreak\Rightarrow \once\DR{x}{1}. 
\end{math}
Chaining into the left conjunct:  %yields
\begin{math}
  \DW{x}{1} \Rightarrow \once\DR{x}{1}. 
\end{math}
Closing $x$, 
% \begin{math}
%   \DR{x}{1} \Rightarrow \once\DW{x}{1}.
% \end{math}
weakening, 
% \begin{math}
%   \once\DR{x}{1} \Rightarrow \once\DW{x}{1}.
% \end{math}
then chaining: %, yields
\begin{math}
  \DW{x}{1} \Rightarrow \once\DW{x}{1}. 
\end{math}
By coinduction, 
\begin{math}
  \lnot\DW{x}{1}.
\end{math}
%as required.
\end{example}

The same reasoning can be applied to the control flow variant of \ref{OOTA3}
\cite[CYC]{DBLP:conf/popl/VafeiadisBCMN15}:
\begin{math}
  %\tag{\textsc{cyc}}\label{CYC}
  \IF{x}\THEN y\GETS 1 \FI \!\PAR\! \IF{y}\THEN x\GETS 1 \FI.
  % &&
  % %\nonumber
  % \hbox{\begin{tikzinline}[node distance=1.5em]
  % \event{rx}{\DR{x}{1}}{}
  % \event{wy}{\DW{y}{1}}{right=of rx}
  % \po{rx}{wy}
  % \event{ry}{\DR{y}{1}}{right=2em of wy}
  % \event{wx}{\DW{x}{1}}{right=of ry}
  % \po{ry}{wx}
  % \rf{wy}{ry}
  % \rf[out=170,in=10]{wx}{rx}
  %   \end{tikzinline}}
\end{math}
The program is data-race-free. Thus, allowing an execution that writes $1$
would violate \drfsc{}.

\begin{example}
  \label{ex:lochb}
  Because our language lacks object creation, we cannot consider \ref{OOTA1}
  directly.  The essential temporal property of \ref{OOTA1} is ``allocation
  at type \texttt{C} is preceded by a read of $1$ for $z$.''  The following
  variant retains this structure: ``any write to memory location $a$ is
  preceded by a read of $1$ for $z$.''  We elide initialization.
% A more general principle, in the spirit of~\citet{Abadi:1993:CS:151646.151649} can be proved.  We chose the simple case of temporal invariants to illustrate the idea in a simple form.  Even this simple version has interesting consequences. 
\begin{align*}
  \tag{\textsc{oota4}}\label{OOTA4}
  %   Z=1;
  % ||
  %   a=X; // 1
  %   Y=a;
  % ||
  %   b=Z; // 0
  %   if(b){
  %     X=1
  %   } else {
  %     c=Y; // 1
  %     X=c;
  %     W=c;
  %   }
  %\VAR  x\GETS0\SEMI \VAR  y\GETS0\SEMI \VAR  z\GETS0\SEMI
  z\GETS1
  \PAR
    y\GETS x
  \PAR
    \IF{z}\THEN x\GETS1 \ELSE r\GETS y \SEMI x\GETS r \SEMI a\GETS r \FI
  \end{align*}
  As a warmup, note that attempting to write $1$ to $a$ results
  in a cycle:
\begin{tikzdisplay}[node distance=1.5em]
  \event{rx}{\DR{x}{1}}{}
  \event{wz1}{\DW{z}{1}}{left=2em of rx}
  \event{wy}{\DW{y}{1}}{right=of rx}
  \po{rx}{wy}
  \event{rz}{\DR{z}{0}}{right=2em of wy}
  \event{ry}{\DR{y}{1}}{right=of rz}
  \event{wx}{\DW{x}{1}}{right=of ry}
  \event{wa}{\DW{a}{1}}{right=of wx}
  \po{ry}{wx}
  \rf[out=15,in=165]{wy}{ry}
  \rf[out=-170,in=-10]{wx}{rx}
  \po[out=-10,in=-170]{rz}{wa}
  \po[out=15,in=165]{ry}{wa}
\end{tikzdisplay}
% \begin{tikzdisplay}[node distance=1.5em]
%   \event{wy0}{\DW{y}{0}}{}
%   \event{rx}{\DR{x}{1}}{right=4.5em of wy0}
%   \event{wy}{\DW{y}{1}}{right=of rx}
%   \po{rx}{wy}
%   \wk[bend left]{wy0}{wy}
%   \event{wx0}{\DW{x}{0}}{below=of wy0}
%   \event{rz}{\DR{z}{0}}{right=of wx0}
%   \event{ry}{\DR{y}{1}}{right=of rz}
%   \event{wx}{\DW{x}{1}}{right=of ry}
%   \event{ry1}{\DR{y}{1}}{right=of wx}
%   \event{wa}{\DW{a}{1}}{right=of ry1}
%   \rf{wy}{ry1}
%   \po{ry}{wx}
%   \wk[bend right]{wx0}{wx}
%   \rf{wy}{ry}
%   \rf{wx}{rx}
%   \event{wz0}{\DW{z}{0}}{below=of wx0}
%   \event{wz1}{\DW{z}{1}}{right=of wz0}
%   \rf{wz0}{rz}
%   \wk{wz0}{wz1}
%   \po{ry1}{wa}
%   \po[bend right]{rz}{wa}
% \end{tikzdisplay}

We prove the formula
\begin{math}
  \lnot\DW{a}{1},
\end{math}
starting with invariant:
% which holds for each of the three threads, and thus, by composition, for the
% aggregate program:
\begin{scope}
\small
\begin{align*}
  [\once\DW{y}{1} \Rightarrow \once\DR{x}{1}]
  \land
  [\notonce\DW{a}{1} \Rightarrow (\once\DR{y}{1} \land \always(\DW{x}{1} \Rightarrow \once\DR{y}{1}))]
\end{align*}
\end{scope}
Closing $y$ and chaining into the left conjunct:
% \begin{math}
%   \once\DR{y}{1} \Rightarrow \once\DW{y}{1}. % \Rightarrow \once\DR{x}{1}
% \end{math}
% Chaining this implication on the left:
\begin{math}
  \once\DR{y}{1} \Rightarrow \once\DR{x}{1}.
\end{math}
% We can weaken this to:
% \begin{math}
%   \once\DR{y}{1} \Rightarrow \once\DR{x}{1}. % \Rightarrow \once\DR{x}{1}
% \end{math}
Chaining into the right conjunct:
\begin{displaymath}
  \notonce\DW{a}{1} \Rightarrow (\once\DR{x}{1} \land \always(\DW{x}{1} \Rightarrow \once\DR{x}{1}))
\end{displaymath}
Closing $x$:
% \begin{math}
%   \once\DR{x}{1} \Rightarrow \once\DW{x}{1}.
% \end{math}
%  Weakening and chaining again:
%we can replace $\once\DR{x}{1}$ with $\once\DW{x}{1}$:
\begin{math}
  \notonce\DW{a}{1} \Rightarrow (\once\DW{x}{1} \land \always(\DW{x}{1} \Rightarrow \once\DW{x}{1}).
\end{math}
Applying coinduction to the right conjunct:
\begin{displaymath}
  \notonce\DW{a}{1} \Rightarrow (\once\DW{x}{1} \land \always(\lnot \DW{x}{1}))
\end{displaymath}
Simplifying:
\begin{math}
  \notonce\DW{a}{1} \Rightarrow \FALSE,
\end{math}
as required.
\end{example}


\begin{comment}
  \color{red} Need to sort this out.
  Alan proposes:
\begin{verbatim}
     (W y 2) => <>(R x 1)
     (W y 1) => <>(R x 0)
     (W x 1) => <>(R y 1)
   <>(W x 1) => not(<>(W x 2))  --- which should be???  <>(W x 0) => not(<>(W x 1))
\end{verbatim}

2020/09/30: This seems to go bad because of initialization...
The formula
\begin{verbatim}
<>Wx0 => not(<>Wx1)
\end{verbatim}
does not hold for
\begin{verbatim}
x=0; x=y
\end{verbatim}

2020/09/10:  I am worried about the compositionality of this predicate:
\begin{verbatim}
I think
   <>(W x 0 => not(<>(W x 1)))
holds for 
   x=0; r=y 
and
   x=1
but not
   x=0; r=y || x=1
as shown by the execution
   Wx1 < Wx0 < Ry0
\end{verbatim}
  
It is impossible to fulfill $(\DR{y}{1})$ in the following
\cite[RNG]{DBLP:conf/esop/SvendsenPDLV18}:
\begin{align*}
  \taglabel{OOTA5}
    ( y\GETS x+1
    \PAR
    x\GETS y ) && \hbox{\begin{tikzinline}[node distance=1.5em]
        \event{rx}{\DR{x}{1}}{}
        \event{wy}{\DW{y}{2}}{right=of rx}
        \po{rx}{wy}
        \event{ry}{\DR{y}{1}}{right=3em of wy}
        \event{wx}{\DW{x}{1}}{right=of ry}
        \po{ry}{wx}
        \rf[out=170,in=10]{wx}{rx}
      \end{tikzinline}}
\end{align*}
The proof proceeds as before, starting with the following invariant:
\begin{gather*}
  [\DW{y}{2} \Rightarrow \once\DR{x}{1}] \land
  [\once\DW{x}{1} \Rightarrow \once\DR{y}{1}] \land
  [\once\DW{y}{1} \Rightarrow \once\DR{x}{0}] \land
  [\once\DW{x}{0} \Rightarrow \lnot(\once\DW{x}{1})]
\end{gather*}
\begin{verbatim}
  Wy2 => <>Rx1  /\  <>Wx1 => <>Ry1  /\  <>Wy1 => <>Rx0  
close x and y                                          
  Wy2 => <>Wx1  /\  <>Wx1 => <>Wy1  /\  <>Wy1 => <>Wx0  
chain
  Wy1 => <>Wx0  
chain with <>Wx0 => not(<>Wx1)
\end{verbatim}
\end{comment}

% Many examples are superficially similar, but in fact have fewer dependencies.
% A referee for a previous version of this paper expected that the following example is
% ``the same'':
% \begin{gather*}
%   \tag{OOTA?}\label{OOTA?}
%     y\GETS x
%   \PAR
%     \IF{y}\THEN r\GETS y\SEMI x\GETS r\SEMI a\GETS r \ELSE x\GETS1 \FI
%   \\
%   \hbox{\begin{tikzinline}[node distance=1.5em]
%   \event{rx}{\DR{x}{1}}{}
%   \event{wy}{\DW{y}{1}}{right=of rx}
%   \po{rx}{wy}
%   \event{ry}{\DR{y}{1}}{right=2em of wy}
%   \event{wx}{\DW{x}{1}}{right=of ry}
%   \event{wa}{\DW{a}{1}}{right=of wx}
%   \rf[out=-15,in=-165]{wy}{ry}
%   \rf[out=170,in=10]{wx}{rx}
%   \po[out=-15,in=-165]{ry}{wa}
%     \end{tikzinline}}
% \end{gather*}
% In this execution, $\DW{x}{1}$ is independent of $\DR{y}{1}$, thus there is no
% \oota{} behavior.

Many examples are superficially similar, but in fact have fewer dependencies,
such as \eqref{OOTA?} from \textsection\ref{sec:intro}.

\citeauthor{BoehmOOTA}'s [\citeyear{BoehmOOTA}] \ref{RFUB} example presents
another potential form of \oota{} behavior, in the context of compiler
optimization.  Our analysis shows that there is no \oota{} behavior in
\ref{RFUB}, instead \citeauthor{BoehmOOTA}'s analysis has a false dependency:
%\citet{BoehmOOTA} \labeltext{considers}{page:rfub} the following programs:
\begin{gather*}
  \tag{\textsc{rfub}}\label{RFUB}
  \sem{r\GETS y\SEMI x\GETS r}
  \not\supseteq
  \sem{r\GETS y\SEMI \IF{r \NOTEQ 1} \THEN z\GETS 1\SEMI r\GETS 1\FI \SEMI x\GETS r}
\end{gather*}
The left command is half of \ref{OOTA3}. %, from \textsection\ref{sec:logic}.
The right command is dubbed \rfub{}, for \emph{Register assignment From an
  Unexecuted Branch}.  \citeauthor{BoehmOOTA} observes that in the context
$x\GETS y \PAR \hole{}$, these programs have different behaviors.  Yet the
\oota{} example on the left never writes $1$.  Why should the unexecuted
branch change that?  As it turns out, both branches of the conditional in
\ref{RFUB} can execute, since the write to $x$ is independent of the read
from $y$.  It useful to considering the Hoare logic formulas satisfied by the
two threads above: we have $\hoare{\TRUE}{\text{\rfub}}{x=1}$, but not
$\hoare{\TRUE}{\text{\oota}}{x=1}$.  The change in the thread from
\ref{OOTA3} to \ref{RFUB} is not a valid refinement under Hoare logic; as a
result, it is expected that \ref{RFUB} may have additional behaviors.

Understanding \oota{} behavior is notoriously difficult, even for the
greatest minds in the field!  % We believe that \emph{logic} is the only tool
% that can cut the horrible knot that semanticists have tied themselves in.
% Preconditions provide a \emph{natural} solution to working out these
% dependencies.
This example shows the wisdom of using existing tools, such as preconditions
and Hoare logic, to model new problems, such as relaxed memory.  We don't
need to abandon established ideas; we only need to adapt them.
% On page \pageref{page:rfub}, we discuss \citeauthor{BoehmOOTA}'s
% [\citeyear{BoehmOOTA}] \ref{RFUB} example, which presents another potential
% form of \oota{} behavior, in the context of compiler optimization.  Our
% analysis shows that there is no \oota{} behavior in \ref{RFUB}, instead
% \citeauthor{BoehmOOTA}'s analysis has a false dependency.

% Understanding \oota{} behavior is notoriously difficult, even for the
% greatest minds in the field!  We believe that \emph{logic} is the only tool
% that can cut the horrible knot that semanticists have tied themselves in.
% Preconditions provide a \emph{natural} solution to working out these
% dependencies.

% \endinput

% \paragraph{Load buffering and thin air.}
% The program
% \begin{math}
%   %x\GETS0\SEMI y\GETS0\SEMI
%   (y\GETS x \PAR \bReg\GETS y\SEMI x\GETS1)
% \end{math}
% has top level executions that result in the final outcome $x = y = 1$, such as:
% \begin{tikzdisplay}[node distance=1.5em]
%   % \event{wx0}{\DW{x}{0}}{}
%   % \event{wy0}{\DW{y}{0}}{below=wx0}
%   \event{rx}{\DR{x}{1}}{}
%   \event{wy}{\DW{y}{1}}{right=of rx}
%   \po{rx}{wy}
%   \event{ry}{\DR{y}{1}}{right=3em of wy}
%   \event{wx}{\DW{x}{1}}{right=of ry}
%   \rf{wy}{ry}
%   \rf[out=170,in=10]{wx}{rx}
%   %\po{rx}{wy}
% \end{tikzdisplay}
% In \textsection\ref{sec:logic} we provide machinery to prove that this
% outcome is impossible if there is order from read to write in both
% threads.  This order can be achieved by replacing the second thread
% \begin{math}
%   (\bReg\GETS y\SEMI x\GETS1)
% \end{math}
% with 
% \begin{math}
%   (\bReg\GETS y\ACQ\SEMI x\GETS1)
% \end{math}
% or
% \begin{math}
%   (\IF{y}\THEN x\GETS 1\FI)
% \end{math}
% or
% \begin{math}
%   (x\GETS y).
% \end{math}

% A more interesting example is the following variant of \eqref{types}:
% \begin{displaymath}
%   %\label{OOTA4}
%   % x\GETS0\SEMI
%   %y\GETS0\SEMI   
%   (
%     y\GETS x
%   \PAR
%     \IF{z}\THEN x\GETS1 \ELSE x\GETS y\SEMI a\GETS y \FI
%   \PAR
%     z\GETS0\SEMI z\GETS1
%   )
% \end{displaymath}
% This program is allowed to write $1$ to $a$ under many speculative
% memory models
% \cite{Manson:2005:JMM:1047659.1040336,DBLP:conf/esop/JagadeesanPR10,DBLP:conf/popl/KangHLVD17},
% even though the read of $1$ from $y$ in the else branch of the second
% thread arises out of thin air.   \citet{DBLP:journals/toplas/Lochbihler13}
% argues that such executions compromise type safety unless object allocation
% partitions memory by type.
% In our model, the attempted execution is:
% \begin{tikzdisplay}[node distance=1.5em]
%   \event{rx}{\DR{x}{1}}{}
%   \event{wy}{\DW{y}{1}}{below=of rx}
%   \po{rx}{wy}
%   \event{ry}{\DR{y}{1}}{right=of rx}
%   \event{wx}{\DW{x}{1}}{below=of ry}
%   \po{ry}{wx}
%   \rf{wy}{ry}
%   \rf{wx}{rx}
%   \event{rz}{\DR{z}{0}}{right=of ry}
%   \event{wz0}{\DW{z}{0}}{right=of rz}
%   \rf{wz0}{rz}
%   \event{wz1}{\DW{z}{1}}{right=of wz0}
%   \wk{wz0}{wz1}
%   \event{ry1}{\DR{y}{1}}{below=of rz}
%   \rf[bend right]{wy}{ry1}
%   \event{wa}{\DW{a}{1}}{right=of ry1}
%   \po{ry1}{wa}
%   \po{rz}{wa}
% \end{tikzdisplay}
% This is forbidden by the evident cycle.


% \begin{verbatim}



% y=x+1; a=y || x=y
% prove a!=2

% Wyv_1 /\ Wyv_2 => v_1 == v_2 (and maybe v_1==0 \/ v_2==0)
% Wx1 => <>-1 Ry1
% Wy1 => <>-1 Rx1
% \end{verbatim}

% Local Variables:
% mode: latex
% TeX-master: "paper"
% End:
