\section{Efficient Implementation on ARMv8}
\label{sec:arm}

In this section, we consider the fragment of our language without
restriction.  For simplicity, we allow release and acquire synchronization
but ban fences.  We assume that all memory locations are initialized to $0$
and parallel-composition occurs only at top level.  We take the set of memory
locations to be finite.  In other words, we assume that programs have the
form
\begin{displaymath}
  {\aLoc_1}\GETS{0}\SEMI
  \cdots\SEMI
  {\aLoc_m}\GETS{0}\SEMI
  (\aCmd^1 \PAR \cdots \PAR \aCmd^n)
\end{displaymath}
where $\aCmd^1$, \ldots, $\aCmd^n$ do not include composition, restriction or
fence operations.

Our language can be translated to ARM using \textsf{ldr} for relaxed read,
\textsf{ldar} for acquiring read, \textsf{str} for relaxed write, and
\textsf{stlr} for releasing write.  Relative to the ARM specification, we
have removed loops and read-modify-write (RMW) operations, in addition to
fences.

We show that any ARM-consistent execution graph for this sublanguage can be
considered an execution of our semantics.  

% Syntactically, we drop the superscript \textsf{rlx} on relaxed reads and
% writes; in addition, we use structured conditionals rather than the more
% general \textsf{goto}.  We refer to this sublanguage as $\muIMM$.
% (Because the source language lacks RMW operations, the ``is
% exclusive'' flag on every read will be \textsf{not-ex} and the RMW mode on
% every write will be \textsf{normal}.)

Due to space limitations, we do not include a full description ARM
consistency.  Interested readers should see \textsection\ref{sec:arm:proof}
for further details.  Here we mention only a few details.

An ARM execution graph $G$ defines many relations, including program order
($\rpox$), reads-from ($\rrfx$), coherence ($\rco$) and several dependency
orders.  From these are derived:
\begin{itemize}
\item ${\rpoloc}$, which is the subrelation of $\rpox$ that only relates
  actions on the same location,
\item ${\rob}$, which is required to be acyclic (\ref{external}), and
\item $\reco$, with the requirement that ${\rpoloc}\cup{\reco}$ be acyclic (\ref{sc-per-loc}).
\end{itemize}
% Let $G$ be an execution graph satisfying the ARM consistency requirements.

Given an execution graph $G$, we say that $\aEv$ is an \emph{internal read} if
$\aEv\in\fcodom({\rpox}\cap {\rrfx})$.

From $G$ we construct a candidate pomset $\aPS$ as follows:
\begin{itemize}
\item $\Event= \textsf{E}$,
\item $\labelingAct(\aEv)=\tau \mathsf{lab}(e)$, if $\aEv$ is a relaxed
  internal read, 
\item $\labelingAct(\aEv)=\mathsf{lab}(e)$, if $\aEv$ is not a relaxed
  internal read,
\item $\labelingForm(\aEv)=\TRUE$,
\item ${\le} = {\rob}^?$, where $?$ denotes reflexive closure, and
\item ${\gtN} = ({\rob} \cup {\reco})^*$, where $*$ denotes reflexive and transitive closure.
\end{itemize}

\begin{theorem}
  If $G$ is ARM consistent, the constructed candidate satisfying the
  requirements for a top-level memory model pomset.
\end{theorem}

The proof uses the following facts about ARM-consistent execution graphs, 
which we establish in \textsection\ref{sec:arm:proof}.  Suppose $\aEv$,
$\bEv$ and $\aEv$ are distinct events.
\begin{itemize}
\item If $\cEv\xob\bEv\xeco\aEv$ then $\cEv\xob\aEv$.
\item If $\cEv\xeco\bEv\xob\aEv$ then $\cEv\xob\aEv$.
\item If $\cEv\xob\bEv$ then $\lnot(\bEv\xeco\cEv)$.
\end{itemize}

\subsection{Proof for ARMv8}
\label{sec:arm:proof}

Given a relation $R$, $R^?$ denotes reflexive closure, $R^+$ denotes
transitive closure and $R^*$ denotes reflexive and transitive closure.  Given
relations $R$ and $S$, $R;S$ denotes composition.


The ARMv8 model is described using the following relations.
\begin{itemize}
\item $\IDR$, $\IDW$, $\IDAcq$, $\IDRel$: identity on reads, writes, acquires
  and releases.
% \item $\IDR$ identity on reads
% \item $\IDW$: identity on writes
% \item $\IDAcq$: identity on acquires
% \item $\IDRel$: identity on releases
\item $\IDLoc$: relates any two events that touch the same location.
\item $\rpox$: program order.
\item $\rdata$, $\rctrl$, $\raddr$: data, control and address dependencies.
\item $\rrfx$: reads-from. $\rrfx^{-1}$ relates each read to a matching write
  on the same location.
\item $\rco$: coherence, which is a total order on the writes to a single
  location.
\item ${\rfr}\eqdef{\rco};\rrfx^{-1}$: from-read, which relates reads to
  subsequent writes.
\end{itemize}
For any relation, the cross-thread subrelation is denoted by appending $e$;
the intra-thread subrelation is denoted by appending $i$.  For example,
${\rrfe}\eqdef{\rrfx}\setminus{\rpox}$ and ${\rrfi}\eqdef{\rrfx}\cap{\rpox}$.
The subrelation restriction attention to actions on the same location is
given by appending $\mathsf{loc}$.  For example, ${\rpoloc}\eqdef{\rpox}\cap{\IDLoc}$.

The ARMv8 model defines the following relations.
In our presentation, we have elided rules concerning fences and RMW operations.
\begin{align*}
  \tag{Extended coherence}
  {\reco} &\eqdef {\rrf} \cup {\rfr} \cup {\rco}
  \\
  \tag{Observed externally}
  {\robs} &\eqdef \smash{
    {\rrfe} \cup {\rfre} \cup {\rcoe}
  }
  \\
  \tag{Dependency order}
  {\rdob} &\eqdef\smash{
    ({\raddr}\cup{\rdata}); {\rrfi}^?
    \cup ({\rctrl}\cup{\rdata}); {\IDW}; {\rcoi}^?
    \cup {\raddr}; {\rpox}; {\IDW}
  }
  \\
  \tag{Barrier order}
  {\rbob} &\eqdef\smash{
    {\IDAcq}; {\rpox}
    \cup {\rpox};{\IDRel}; {\rcoi}^?
  }
  \\
  \tag{Acyclic order}
  {\rob} &\eqdef\smash{
    ({\robs} \cup {\rdob} \cup {\rbob})^+
  }
\end{align*}
\begin{definition}
  An RMW-free and fence-free execution is \emph{ARM-consistent} if
  \begin{align*}&
    \tag{\textsc{$\rrfx$-completeness}}\label{rf-comp}
    \fcodom(\rrfx)=\fdom(\rreads)
    \\[-1ex]&
    \tag{\textsc{$\rco$-totality}}\label{co-tot}
    \text{For every location $\aLoc$, $\rco$ totally orders the writes of $\aLoc$}  
    \\[-1ex]&
    \tag{\textsc{sc-per-loc}}\label{sc-per-loc}
    {\rpoloc} \cup {\rrfx} \cup {\rfr} \cup {\rco}\;\text{is acyclic}
    \\[-1ex]&
    \tag{\textsc{external}}\label{external}
    {\rob}\;\text{is acyclic}
  \end{align*}
\end{definition}
Use these to refer to the rules in text:
\ref{rf-comp} 
\ref{co-tot}
\ref{sc-per-loc}
\ref{external}


Given an execution graph $G$, we say that $\aEv$ is an \emph{internal read} if
$\aEv\in\fcodom(\mathsf{po}\cap \mathsf{rf})$.

From $G$ we construct a candidate pomset $\aPS$ as follows:
\begin{itemize}
\item $\Event= \textsf{E}$,
\item $\labelingAct(\aEv)=\tau \mathsf{lab}(e)$, if $\aEv$ is a relaxed
  internal read, 
\item $\labelingAct(\aEv)=\mathsf{lab}(e)$, if $\aEv$ is not a relaxed
  internal read,
\item $\labelingForm(\aEv)=\TRUE$,
\item ${\le} = {\rob}$, and
\item ${\gtN} = ({\rob} \cup {\reco})^*$
\end{itemize}

We show that $\aPS$ is a top-level pomset, reasoning as follows.

 First, we establish the criteria for a 3-valued pomset (Definition~\ref{def:3valued}).
\begin{itemize}
\item ${\le}$ is a partial order.  This holds since $G.{\rar}$ is acyclic.
\item If $\bEv \le \aEv$ then $\bEv \gtN \aEv$.  By construction.
\item If $\bEv \le \aEv$ and $\aEv \gtN \bEv$ then $\bEv = \aEv$.  ????
\item If $\cEv \le \bEv \gtN \aEv$ or $\cEv \gtN \bEv \le \aEv$ then
  $\cEv \gtN \aEv$. By construction.
\end{itemize}

Next, we establish the criteria for a 3-valued pomset with preconditions (Definition~\ref{def:3pre}).
\begin{itemize}
\item $\labelingForm(\aEv)$ implies $\labelingForm(\bEv)$ whenever
  $\bEv\le\aEv$.   Trivial, since every formula is $\TRUE$.
\item $\aPS$ is $\aLoc$-coherent; that is, when restricted to events that
  read or write $\aLoc$, $\gtN$ forms a partial order.
\end{itemize}

Finally, we establish the criteria for a top-level pomset
(Definition~\ref{def:x-closed}).
\begin{itemize}
\item $\aEv$ is location independent. Trivial, since every formula is $\TRUE$.
\item If $\aEv$ reads $\aVal$ from $\aLoc$, then there is some $\bEv$ such that
  \begin{itemize}
  \item $\bEv \lt \aEv$,  
  \item $\bEv$ writes $\aVal$ to $\aLoc$, and
  \item if $\cEv$ writes to $\aLoc$
    then either $\cEv \gtN \bEv$ or $\aEv \gtN \cEv$.
  \end{itemize}    
\end{itemize}



\begin{comment}
Operation        Implementation
Relaxed read     ldr                     
Relaxed write    str             
Acquiring read   ldar            
Releasing write  stlr
Fence            dmb.sy
\end{comment}

\begin{comment}
ob does not contradict eco

ob does not contradict (co cap po):

Suppose that wx1 po wx2 then it cannot be that wx2 ob wx1.
We know that wx1 co wx2 by SC-PER-LOC

% Case 1. w1 is read externally, then we have
%   wx1 rfe r
% and
%   r fre w2
% so
%   wx1 obs+ wx2
% which contradicts EXTERNAL

% Case 2. wx1 is not read externally.
We show this by contradiction
Assume
  wx1 co wx2
and
  wx2 ob wx1

Note that
  po supseteq dob cup aob cup bob
So in order to get order into wx1, we must have
  wx2 (ob?; obs; ob?; obs; ob?) wx1

Note that we cannot have dob or bob into wx1 after obs, since then we would
also have it into wx2, creating a cycle in EXTERNAL.  This holds because both
dob and bob are closed on the right w.r.t. coi

So it must be that 
  wx2 (ob?; obs; ob?; wx0; coe) wx1, 
in which case we also have wx0 coe wx2, contradicting EXTERNAL
or 
  wx2 (ob?; obs; ob?; rx0; fre) wx1
in which case we also have rx0 fre wx2, contradicting EXTERNAL




Internal reads do not need to respect ob:
Arm allows the following:

  Ra1 -ctrl-> Wx1 -rfi-> Rx1 ---> Wb1    if(a){x=1}; b=x
   |                               |
  Wa1 <-------------------------- Rb1    a=b


Suppose that wx1 po rx2 and rx2 is read externally.
Then it cannot be that rx2 ob wx1.

Case 1: if wx1 co wx2, then we have wx1 coe wx2 rfe rx2, contradicting EXTERNAL
Case 2: if wx2 co wx1, then we have rx2 fr wx1, contradicting SC-PER-LOC



Suppose that rx1 po wx2 and rx1 is read externally.
Then it cannot be that wx2 ob rx1.

Case 1: if wx2 co wx1, then wx2 co wx1 rf rx1 po wx2, contradicting SC-PER-LOC 
Case 2: if wx1 co wx2, for a contradiction, suppose wx2 ob rx1.
then we need another thread involved to get order from wx2 to rx1.
To get order into the read, there are several options:
- use cross thread read, then dob; but dob does not include reads in it's domain.
  An attempt to do this is something like:

              Wx1                 x=1
               |
  Ra2 -ctrl-> Rx1 - - -> Wx2      if(a){r=x}; x=2
   |                      |
  Wa2 <----------------- Rx2      a=x

  But the ctrl dependency is not included in ob between reads.
- use cross thread read then barrier, but then you contradict EXTERNAL
- create and ob edge from Rx2 to Wx1.
  An attempt to do this is, 

  Wx1 <-------------- Ra1       
   |                   |        But cannot get Wx2 --> Wa1 without a barrier
  Rx1 - - -> Wx2 ---> Wa1       

  Wx1 <----- Rx2                         
   |          |                 contradicts SC-PER-LOC 
  Rx1 - - -> Wx2                         


Other examples to type in:
Allowed:
Rx1 -> Wy0  Wy1
Ry1 -> Wz0  Wz1
Rz1 -> Wx0  Wx1

Forbidden:
Rx1 -> Wy0 Wy1
Ry1 -> Wx0 Wx1

\end{comment}



\begin{comment}
\citet{DBLP:journals/pacmpl/PodkopaevLV19} define the \emph{Intermediate
  Memory Model (IMM)} and provide efficient implementations of the IMM into
several processor architectures, including TSO, ARMv8 and Power.

In this section, we show that any execution allowed by a sublanguage of the
IMM is also allowed by our semantics.  The sublanguage we consider bans
loops, read-modify-write (RMW) operations, and fences.  In addition, we take
the set of memory locations, $\Loc$, to be finite.  Syntactically, we drop
the superscript \textsf{rlx} on relaxed reads and writes; in addition, we use
structured conditionals rather than the more general \textsf{goto}.  We refer
to this sublanguage as $\muIMM$.

$\muIMM$ programs sit in the restriction-free fragment of our language, where
all memory locations are initialized to $0$ and parallel-composition occurs
only at top level.  In other words, $\muIMM$ programs have the form
\begin{displaymath}
  {\aLoc_1}\GETS{0}\SEMI
  \cdots\SEMI
  {\aLoc_m}\GETS{0}\SEMI
  (\aCmd^1 \PAR \cdots \PAR \aCmd^n)
\end{displaymath}
where $\aCmd^1$, \ldots, $\aCmd^n$ do not include either composition or
restriction.

Due to space limitations, we do not include a full description of the IMM.
The broad strokes of the argument given here should be clear, but interested
readers will need to refer to \citep{DBLP:journals/pacmpl/PodkopaevLV19} for
details.
\end{comment}
