\section{Properties of the Basic Model}
\label{sec:props}

It is amazing how much Candidate \ref{cand:ord} ``gets right'' out of the
box, including ``internal reads'', which greatly complicate other models
\cite{DBLP:journals/pacmpl/PulteFDFSS18}.  Before refining the model, we walk
through several litmus tests, valid rewrites and invalid rewrites.

\subsection{Litmus Tests}
\label{sec:litmus}



%Pugh, not 19/20 which have join.  Not 16 on coherence.

\citet{PughWebsite} developed a set of litmus tests for the java memory
model.  Our model gives the expected result for all but cases 16, 19 and 20'
(unrolling loops): we discuss \ref{TC16} below; TC19 and TC20 involve a
thread join operation, which is not expressible in our languages.  Our model
also agrees with the \oota{} examples in \citet[\textsection
4]{DBLP:conf/esop/BattyMNPS15} and the ``surprising and controversial
behaviors'' of \citet[\textsection 8]{Manson:2005:JMM:1047659.1040336}.

\paragraph{Buffering}

Consider the \emph{store buffering} and \emph{load buffering} litmus \labeltext[SB]{tests}{SB}:
\begin{gather*}
  \taglabel{SB/LB}
  \begin{gathered}
  x\GETS0\SEMI
  y\GETS0\SEMI
  (
  x\GETS1\SEMI\aReg\GETS y
  \PAR
  y\GETS1\SEMI \aReg\GETS x)
  \\
  \hbox{\begin{tikzinline}[node distance=1em and 2em]
      \event{wx0}{\DW{x}{0}}{}
      \event{wy0}{\DW{y}{0}}{below=of wx0}
      \event{wx}{\DW{x}{1}}{right=of wx0}
      \event{wy}{\DW{y}{1}}{right=of wy0}
      \event{ry}{\DR{y}{0}}{right=of wx}
      \event{rx}{\DR{x}{0}}{right=of wy}
      \wk{wx0}{wx}
      \wk{wy0}{wy}
      \rf{wy0}{ry}
      \rf{wx0}{rx}
      \wk[out=-150,in=15]{ry}{wy}
      \wk[out=150,in=-15]{rx}{wx}
      % \po{rx}{wy}
      % \pox{wy}[below]{rx}
      % \pox{wx}{ry}
    \end{tikzinline}}
\end{gathered}
\qquad
  \begin{gathered}
  \aReg\GETS y\SEMI x\GETS1
  \PAR
  \aReg\GETS x\SEMI y\GETS1
  \\
  \hbox{\begin{tikzinline}[node distance=1em and 2em]
      \event{ry}{\DR{y}{1}}{}
      \event{rx}{\DR{x}{1}}{below=of ry}
      \event{wx}{\DW{x}{1}}{right=of ry}
      \event{wy}{\DW{y}{1}}{right=of rx}
      \rf{wy}{ry}
      \rf{wx}{rx}
      % \pox{rx}[below]{wy}
      % \pox{ry}{wx}
      % \po{rx}{wy}
    \end{tikzinline}}
\end{gathered}
\end{gather*}
The desired outcomes are allowed, as shown.
% All of the cross-thread order is required by the definition of fulfillment.
% Candidate \ref{def:rw:sc1} disallows the executions, since it includes all of
% program order in pomset order, resulting in the cycles shown above.
% To allow these executions, we drop the order marked $\rpox$.


\paragraph{Publication and SC Access}

Items \ref{5b} and \ref{5c} ensure correct publication.  For
example, they disallow the following candidate execution, which sees a stale
value for $x$:
\vspace{-1ex}
\begin{gather}
  \taglabel{Pub1}
  x\GETS0\SEMI %y\GETS0\SEMI
  x\GETS 1\SEMI y \REL\GETS1 \PAR r\GETS y\ACQ\SEMI s\GETS x
  \\[-1ex]
  \nonumber
  \hbox{\begin{tikzinline}[node distance=1em]
      \event{wx0}{\DW{x}{0}}{}
      \event{wx1}{\DW{x}{1}}{right=of wx0}
      \event{wy1}{\DWRel{y}{1}}{right=of wx1}
      \event{ry1}{\DRAcq{y}{1}}{right=2.5em of wy1}
      \event{rx0}{\DR{x}{0}}{right=of ry1}
      \sync{wx1}{wy1}
      \sync{ry1}{rx0}
      \rf{wy1}{ry1}
      \rf[out=10,in=170]{wx0}{rx0}
      \wk{wx0}{wx1}
    \end{tikzinline}}
\end{gather}
By Definition \ref{def:rf}, $(\DR x0)$ is \emph{unfulfilled} in this
pomset. It fails the last requirement of the definition, since
$(\DW x0) \gtN (\DW x1) \lt (\DR x0)$.  In order to satisfy this requirement,
$(\DR x0)$ must be ordered before $(\DW x1)$, but this creates a cycle.

%This example is also disallowed if we use $\modeSC$ rather than $\modeRA$.
Item \ref{5d} ensures that program order between SC operations is always
preserved.  Combined with the requirements for fulfillment, this is
sufficient to establish that programs with only SC access have only SC
executions; for example, the executions of \ref{SB/LB} are banned when the
actions of the two threads are all $\modeSC$.  It is also immediate that SC
actions can be totally ordered, using any linearization of pomset order.
Just as SC access in \armeight{} is simplified by \mca, it is simplified here
by the global pomset order.

SC access is not as strict as volatile access in Java.  For example, our
model allows the following, which is disallowed by
\citet[\textsection8.2]{Dolan:2018:BDR:3192366.3192421}:
\begin{gather*}
  \taglabel{SC1}
  \begin{gathered}
  r\GETS y\SEMI x^\modeSC\GETS1\SEMI s\GETS x
  \PAR
  % y\GETS0\SEMI
  x^\modeSC\GETS2 \SEMI y\GETS1
  \\
  \hbox{\begin{tikzinline}[node distance=1em]
      \event{a}{\DR{y}{1}}{}
      \event{b}{\DW[\modeSC]{x}{1}}{right=of a}
      \sync{a}{b}
      \event{bb}{\DR{x}{2}}{right=of b}
      \wk{b}{bb}
      % \event{c}{\DW{y}{0}}{right=2em of bb}
      % \event{d}{\DW[\modeSC]{x}{2}}{right=of c}
      % \po{c}{d}
      \event{d}{\DW[\modeSC]{x}{2}}{right=2em of bb}
      \event{e}{\DW{y}{1}}{right=of d}
      %\rf[out=170,in=10]{d}{bb}
      \rf{d}{bb}
      \rf[out=-170,in=-10]{e}{a}
      \wk[in=165,out=15]{b}{d}
    \end{tikzinline}}
\end{gathered}
\end{gather*}
Note that there is no order from $(\DW[\modeSC]{x}{2})$ to $(\DW{y}{1})$.

For further comparison to Java volatiles and
\citet{Dolan:2018:BDR:3192366.3192421}, see the end of
\textsection\ref{sec:sc}.



\paragraph{Coherence}

Our model of coherence does not correspond to either Java or C11.  We have
chosen the model to validate \ref{CSE} (unlike C11 relaxed atomics) and the
local \drfsc{} theorem (unlike Java).

% \todo{There is a tension between read-read coherence and CSE.  Java favors CSE,
% whereas C11 prefers read-read coherence.  
% \ref{Co2} is not arbitrary, but needed to validate CSE.
% At Co3, point out that this stops races from the past from sneaking in.
% Unordered actions can be scheduled freely.
% As a result, our model of coherence is similar to that of
% \citet{Dolan:2018:BDR:3192366.3192421}.}

Since reads are not ordered by \ref{5b},
we {allow} the following unintuitive behavior. C11 includes read-read
coherence between relaxed atomics in order to forbid this:
\begin{gather*}
  \taglabel{Co2}
  \begin{gathered}
  x\GETS1\SEMI x\GETS 2
  \PAR
  y\GETS x \SEMI z\GETS x
  \\[-1ex]
  \hbox{\begin{tikzinline}[node distance=1em]
      \event{a}{\DW{x}{1}}{}
      \event{b}{\DW{x}{2}}{right=of a}
      \wk{a}{b}
      \event{c}{\DR{x}{2}}{right=2em of b}
      \event{d}{\DW{y}{2}}{right=of c}
      \po{c}{d}
      \event{e}{\DR{x}{1}}{right=of d}
      \event{f}{\DW{z}{1}}{right=of e}
      \po{e}{f}
      \rf{b}{c}
      \rf[out=10,in=170]{a}{e}
    \end{tikzinline}}
\end{gathered}
\end{gather*}
Here, the reader sees $2$ then $1$, although they are written in the reverse
order.
This behavior is allowed by Java in order to validate \ref{CSE}.

However, our model is more coherent than Java, which permits the following:
\begin{gather*}
  \taglabel{TC16}
  \begin{gathered}
  r\GETS x\SEMI x\GETS 1
  \PAR
  s\GETS x\SEMI x\GETS 2
  \\[-1ex]
  \hbox{\begin{tikzinline}[node distance=1em]
      \event{a1}{\DR{x}{2}}{}
      \event{a2}{\DW{x}{1}}{right=of a1}
      \wk{a1}{a2}
      \event{b1}{\DR{x}{1}}{right=2em of a2}
      \event{b2}{\DW{x}{2}}{right=of b1}
      \wk{b1}{b2}
      \rf{a2}{b1}
      \rf[out=-165,in=-15]{b2}{a1}
    \end{tikzinline}}
\end{gathered}
\end{gather*}
We also forbid the \labeltext{following}{page:coherence2}, which Java allows:
\begin{gather*}
  \taglabel{Co3}
  \begin{gathered}
  x\GETS1\SEMI y^\modeRA\GETS 1
  \PAR
  x\GETS2\SEMI z^\modeRA\GETS 1
  \PAR
  r\GETS y^\modeRA \SEMI 
  r\GETS z^\modeRA \SEMI 
  r\GETS x \SEMI 
  r\GETS x
  \\[-1ex]
  \hbox{\begin{tikzinline}[node distance=1em]
      \event{a1}{\DW{x}{1}}{}
      \event{a2}{\DW[\modeRA]{y}{1}}{right=of a1}
      \sync{a1}{a2}
      \event{b1}{\DW{x}{2}}{below=of a1}
      \event{b2}{\DW[\modeRA]{\,z}{1}}{right=of b1}
      \sync{b1}{b2}
      \event{c1}{\DR[\modeRA]{y}{1}}{right=2em of a2}
      \event{c2}{\DR[\modeRA]{\,z}{1}}{right=2em of b2}
      \event{c3}{\DR{x}{2}}{right=of c1}
      \event{c4}{\DR{x}{1}}{right=of c2}
      \sync{c1}{c3}
      \sync{c1}{c4}
      \sync{c2}{c3}
      \sync{c2}{c4}
      \rf{a2}{c1}
      \rf{b2}{c2}
      \wk{a1}{b1}
      \wk[out=-165,in=-15]{c4}{b1}
    \end{tikzinline}}
\end{gathered}
\end{gather*}
We disallow the outcome due to the cycle: the order from $(\DR{x}{1})$ to
$(\DW{x}{2})$ is required to fulfill $(\DR{x}{1})$. %by fulfillment.
Such an outcome would allow racing writes to be visible, even after a full
synchronization, invalidating local reasoning about data races
(\textsection\ref{sec:sc}).

% \begin{gather*}
%   x\GETS1\SEMI y^\modeRA\GETS 1
%   \PAR
%   x\GETS 2\SEMI
%   r\GETS y^\modeSC \SEMI 
%   r\GETS x \SEMI 
%   r\GETS x
%   \\[-1ex]
%   \hbox{\begin{tikzinline}[node distance=1em]
%       \event{a}{\DW{x}{1}}{}
%       \event{b}{\DW[\modeRA]{y}{1}}{right=of a}
%       \sync{a}{b}
%       \event{c}{\DW{x}{2}}{below=of a}
%       \event{d}{\DR[\modeSC]{y}{1}}{right=of c}
%       \po{c}{d}
%       \event{e}{\DR{x}{2}}{right=of d}
%       \po{d}{e}
%       \event{f}{\DR{x}{1}}{right=of e}
%       \po[out=10,in=170]{d}{f}
%       \rf{b}{d}
%       %\rf[out=-10,in=-170]{c}{e}
%       \wk{a}{c}
%       %\rf[out=15,in=150]{a}{f}
%       \wk[out=-165,in=-15]{f}{c}
%     \end{tikzinline}}
% \end{gather*}

% \todo{We will also clarify that the "expected result" for TC16 is not the same as it is in Java. We disallow this execution due to the cycle
% (r:=x; x:=1) || (s:=x; x:=2)
% Rx2 -co-> Wx1 -rf-> Rx1 -co-> Wx2 -rf-> Rx2}

\paragraph{Internal Reads}
The JMM causality test cases \citep{PughWebsite} are justified via
compiler analysis, possibly in collusion with the scheduler: If every 
observed value can be shown to satisfy a precondition, then the precondition
can be dropped.  For
example, \ref{TC1} determines that the following top-level execution should be
allowed, as it is in our model:
\begin{gather*}
  \taglabel{TC1}
  \begin{gathered}
  x\GETS 0 \SEMI
  (r\GETS x\SEMI\IF{r\geq0}\THEN y\GETS1 \FI
  \PAR
  x\GETS y)
  \\
  \hbox{\begin{tikzinline}[node distance=1em and 2em]
  \event{wy1}{0\geq0\mid\DW{y}{1}}{}
  \event{rx1}{\DR{x}{1}}{left=of wy1}
  \event{wx0}{\DW{x}{0}}{below left=-.2em and 2em of rx1}
  \event{ry1}{\DR{y}{1}}{below=of rx1}
  \event{wx1}{\DW{x}{1}}{right=of ry1}
  \po{ry1}{wx1}
  \rf[out=140,in=-20]{wx1}{rx1}
  \rf{wy1}{ry1}
  \wk[out=10,in=155]{wx0}{wx1}
  \wk{wx0}{rx1}
    \end{tikzinline}}
\end{gathered}
\end{gather*}
In this example, $(\DW{x}{0})$ ``fulfills'' the read of $x$ that is used in
the guard of the conditional.  This is possible when prefixing $(\DR{x}{1})$
performs the substitution $[x/r]$, but does not weaken the resulting
precondition $(x\geq0\mid\DW{y}{1})$.  Subsequently prefixing $(\DW{x}{0})$
substitutes $[0/x]$, resulting in the tautological precondition
$(0\geq0\mid\DW{y}{1})$.  Note that the execution does not have an action
$(\DR{x}{0})$.

This execution is an example of an \emph{internal read}, using \armeight{}
terminology \cite{DBLP:journals/pacmpl/PulteFDFSS18}. Unlike
\citep{DBLP:conf/lics/JeffreyR16}, our semantics is robust with respect to
the introduction of concurrent writes, as in \ref{TC9}:
\begin{gather*}
  \taglabel{TC9}
  \begin{gathered}
  x\GETS 0 \SEMI
  (r\GETS x\SEMI\IF{r\geq0}\THEN y\GETS1 \FI
  \PAR
  x\GETS y
  \PAR
  x\GETS {-2})
  \\
  \hbox{\begin{tikzinline}[node distance=1em and 2em]
  \event{wy1}{0\geq0\mid\DW{y}{1}}{}
  \event{rx1}{\DR{x}{1}}{left=of wy1}
  \event{wx0}{\DW{x}{0}}{below left=-.2em and 2em of rx1}
  \event{ry1}{\DR{y}{1}}{below=of rx1}
  \event{wx1}{\DW{x}{1}}{right=of ry1}
  \event{wx2}{\DW{x}{{-2}}}{below right=-.2em and 2em of wy1}
  \po{ry1}{wx1}
  \rf[out=140,in=-20]{wx1}{rx1}
  \rf{wy1}{ry1}
  \wk[out=10,in=155]{wx0}{wx1}
  \wk{wx0}{rx1}
  \wk{wx1}{wx2}
    \end{tikzinline}}
\end{gathered}
\end{gather*}

The reasoning for \ref{TC2} is similar, but in this case no value is necessary to
satisfy the precondition:
\begin{gather*}
  \taglabel{TC2}
  \begin{gathered}
  r\GETS x\SEMI
  s\GETS x\SEMI
  \IF{r{=}s}\THEN y\GETS 1\FI
  \PAR
  x\GETS y
  \\[-1ex]
  \nonumber
  \hbox{\begin{tikzinline}[node distance=.5em and 1em]
  \event{a1}{\DR{x}{1}}{}
  \event{a2}{\DR{x}{1}}{right=of a1}
  \event{a3}{x{=}x\mid\DW{y}{1}}{right=of a2}
  % \po{a2}{a3}
  % \po[out=15,in=165]{a1}{a3}
  \event{b1}{\DR{y}{1}}{right=2em of a3}
  \event{b2}{\DW{x}{1}}{right=of b1}
  \rf{a3}{b1}
  \po{b1}{b2}
  \rf[out=162,in=15]{b2}{a2}
  \rf[out=162,in=15]{b2}{a1}
    \end{tikzinline}}
\end{gathered}
\end{gather*}
Note that in the prefix
\begin{math}
  \sem{s\GETS x\SEMI
  \IF{r{=}s}\THEN y\GETS 1\FI},
\end{math}
the precondition on $(\DW{y}{1})$ must imply $r=1 \land r=x$.  The first conjunct is
imposed by \ref{4b}, the second by \ref{5a}.  Thus the two reads must see the same value.

Write substitution only effects subsequent reads, and a read action always
creates an event that must be fulfilled.  In combination, these ensure that an
\emph{internal read} cannot ignore a blocking write.  In the
following execution candidate, there is no order from $(\DR{x}{1})$ to
$(\DW{y}{1})$, potentially allowing the program to write a stale value.
However, $(\DR{x}{1})$ cannot be fulfilled, causing the execution candidate
to be disallowed:
\begin{gather*}
  \taglabel{Internal1}
  \begin{gathered}
  x\GETS1 \SEMI
  a\REL\GETS1 \SEMI
  \IF{b\ACQ}\THEN  y\GETS x \FI
  \PAR
  \IF{a\ACQ}\THEN  x\GETS 2\SEMI b\REL\GETS1 \FI
  \\
  \hbox{\begin{tikzinline}[node distance=.8em and 1em]
  \event{a1}{\DW{x}{1}}{}
  \event{a2}{\DWRel{a}{1}}{right=of a1}
  \sync{a1}{a2}
  \event{b3}{\DRAcq{a}{1}}{below=of a2}
  \rf{a2}{b3}
  \event{b4}{\DW{x}{2}}{right=of b3}
  \sync{b3}{b4}
  \event{b5}{\DWRel{b}{1}}{right=of b4}
  \sync{b4}{b5}
  \event{a6}{\DRAcq{b}{1}}{above=of b5}
  \rf{b5}{a6}
  \event{a7}{\DR{x}{1}}{right=of a6}
  \sync{a6}{a7}
  \event{a8}{\DW{y}{1}}{right=of a7}
  \graypo{a7}{a8}
  \sync[out=10,in=170]{a6}{a8}
    \end{tikzinline}}
\end{gathered}
\end{gather*}
The execution becomes inconsistent if we change $(\DR{x}{1})$ to
$(\DR{x}{2})$, resulting in
$(2{=}1\mid\DW{y}{1})$.

Internal reads are notoriously difficult to get right.  Consider \cite[Ex 3.6]{DBLP:journals/pacmpl/PodkopaevLV19}:
\begin{gather*}
  \taglabel{Internal2}
  \begin{gathered}
  \aReg\GETS x\SEMI
  y\REL\GETS 1\SEMI
  \bReg\GETS y\SEMI
  z\GETS \bReg
  %z\GETS y
  \PAR
  x\GETS z
  \\
  \nonumber
  \hbox{\begin{tikzinline}[node distance=.5em and 1em]
  \event{a1}{\DR{x}{1}}{}
  \event{a2}{\DWRel{y}{1}}{right=of a1}
  \sync{a1}{a2}
  \event{a3}{\DR{y}{1}}{right=of a2}
  \event{a4}{1{=}1\mid\DW{z}{1}}{right=of a3}
  \rf{a2}{a3}
  %\po{a3}{a4}
  \event{b1}{\DR{z}{1}}{right=2em of a4}
  \event{b2}{\DW{x}{1}}{right=of b1}
  %\po[out=10,in=170]{a2}{a4}
  \po{b1}{b2}
  \rf{a4}{b1}
  \rf[out=170,in=10]{b2}{a1}
    \end{tikzinline}}
\end{gathered}
\end{gather*}
This behavior is allowed in our model, as it is in \armeight.
%Revisiting \eqref{Internal2}, %the example,
Note that $\sem{z\GETS \bReg}$ includes $(\bReg{=}1\mid \DW{z}{1})$.
Prepending a read,
% of Definition~\ref{def:rw:local},
$\sem{\bReg\GETS y \SEMI z\GETS \bReg}$ may update the precondition to
$(y{=}1\mid \DW{z}{1})$ without introducing order.
%Crucially, there is no explicit read event.
Further prepending
$(\DWRel{y}{1})$ results in $(1{=}1\mid \DW{z}{1})$.

Our model drops order into actions that depend on a read that can be
fulfilled \emph{internally}, by a prefixed write.  This is natural
consequence of substitution.  The \armeight{} model has to jump through
some hoops to ensure that internal reads are handled correctly.  \armeight{}
takes the symmetric approach: rather than dropping order \emph{out of} an
internal read, \armeight{} drops the order \emph{into} it.  This difference
complicates the proof for \armeight{} (\textsection\ref{sec:arm}).

% performs the substitution $(y=1\mid \DW{z}{1})[1/y]$ resulting in a
% tautology.
% By removing the read event, the outcome is
% allowed by our semantics, enabling efficient compilation to \armeight:
% \begin{gather}
%   \label{inlining}
%   % \aReg\GETS x\SEMI
%   % y\REL\GETS 1\SEMI
%   % z\GETS y
%   % \PAR
%   % x\GETS z
%   % \\
%   \hbox{\begin{tikzinline}[node distance=.5em and 1em]
%   \event{a1}{\DR{x}{1}}{}
%   \event{a2}{\DWRel{y}{1}}{right=of a1}
%   \po{a1}{a2}
%   \internal{a3}{\DR{y}{1}}{right=of a2}
%   \event{a4}{\DW{z}{1}}{right=of a3}
%   %\rf{a2}{a3}
%   \graypo{a3}{a4}
%   \event{b1}{\DR{z}{1}}{right=of a4}
%   \event{b2}{\DW{x}{1}}{right=of b1}
%   %\po[out=10,in=170]{a2}{a4}
%   \po{b1}{b2}
%   \rf{a4}{b1}
%   \rf[out=170,in=10]{b2}{a1}
%     \end{tikzinline}}
% \end{gather}




\paragraph{MCA} We present a few examples that are hallmarks of \mca{}
architectures.  
In \ref{MCA1}, $(\DW{x}{1})$ is delivered to the first second thread, but not
the third. \ref{MCA2} is an example of \emph{write subsumption}
\cite[\textsection 3]{DBLP:journals/pacmpl/PulteFDFSS18}.
% \begin{gather*}
%   \renewcommand{\arraycolsep}{1pt}
%   \hbox{\small
%     $\begin{array}{ccccc}
%     &x\GETS0\SEMI x\GETS 1
%     &\PAR&
%     r\GETS x \SEMI s\GETS y
%     %\IF{x}\THEN r\GETS y \FI
%     \\
%     \PAR
%     &y\GETS0\SEMI y\GETS 1
%     &\PAR&
%     r\GETS y \SEMI s\GETS x
%     %\IF{y}\THEN s\GETS x \FI
%   \end{array}$}
%   \quad
%   \hbox{\begin{tikzinlinesmall}[node distance=.5em and 1em]
%   \event{wx0}{\DW{x}{0}}{}
%   \event{wx1}{\DW{x}{1}}{right=of wx0}
%   \event{wy0}{\DW{y}{0}}{below=2ex of wx0}
%   \event{wy1}{\DW{y}{1}}{right=of wy0}
%   \event{ry1}{\DR{y}{1}}{right=2.5em of wy1}
%   \event{rx0}{\DR{x}{0}}{right=of ry1}
%   \event{rx1}{\DR{x}{1}}{right=2.5 em of wx1}
%   \event{ry0}{\DR{y}{0}}{right=of rx1}
%   \wk{wx0}{wx1}
%   \wk{wy0}{wy1}
%   \rf{wx1}{rx1}
%   \rf[bend left]{wy0}{ry0}
%   \rf{wy1}{ry1}
%   \rf[bend right]{wx0}{rx0}
%   \wk{rx0}{wx1}
%   \wk{ry0}{wy1}
%     \end{tikzinlinesmall}}
% \end{gather*}
\begin{gather*}
  \taglabel{MCA1}
  \begin{gathered}
  x\GETS0\SEMI x\GETS 1
  \PAR
  y\GETS x
  \PAR
  r\GETS y\ACQ \SEMI s\GETS x
  \\[-1.5ex]
  \hbox{\begin{tikzinline}[node distance=1em]
  \event{wx0}{\DW{x}{0}}{}
  \event{wx1}{\DW{x}{1}}{right=of wx0}
  \wk{wx0}{wx1}
  \event{rx1}{\DR{x}{1}}{right=3em of wx1}
  \event{wy1}{\DW{y}{1}}{right=of rx1}
  \po{rx1}{wy1}
  \event{ry1}{\DRAcq{y}{1}}{right=3em of wy1}
  \event{rx0}{\DR{x}{0}}{right=of ry1}
  \rf{wx1}{rx1}
  \rf{wy1}{ry1}
  \sync{ry1}{rx0}
  \wk[out=170,in=10]{rx0}{wx1}
  \end{tikzinline}}
  \end{gathered}
  \\[2ex]
  % \taglabel{MCA1}
  % %\\[-2.5ex]
  % \renewcommand{\arraycolsep}{1pt}
  % \begin{array}{ccccc}
  %   &x\GETS0\SEMI x\GETS 1
  %   &\PAR&
  %   r\GETS x\ACQ \SEMI s\GETS y
  %   %\IF{x}\THEN r\GETS y \FI
  %   \\
  %   \PAR
  %   &y\GETS0\SEMI y\GETS 1
  %   &\PAR&
  %   r\GETS y\ACQ \SEMI s\GETS x
  %   %\IF{y}\THEN s\GETS x \FI
  % \end{array}
  % \quad
  % \smash{\hbox{\begin{tikzinlinesmall}[baseline=-10pt,node distance=.5em and 1em]
  % \event{wx0}{\DW{x}{0}}{}
  % \event{wx1}{\DW{x}{1}}{right=of wx0}
  % \event{wy0}{\DW{y}{0}}{below=2ex of wx0}
  % \event{wy1}{\DW{y}{1}}{right=of wy0}
  % \event{ry1}{\DRAcq{y}{1}}{right=2.5em of wy1}
  % \event{rx0}{\DR{x}{0}}{right=of ry1}
  % \event{rx1}{\DRAcq{x}{1}}{right=2.5 em of wx1}
  % \event{ry0}{\DR{y}{0}}{right=of rx1}
  % \wk{wx0}{wx1}
  % \wk{wy0}{wy1}
  % \rf{wx1}{rx1}
  % \rf[bend left]{wy0}{ry0}
  % \rf{wy1}{ry1}
  % \rf[bend right]{wx0}{rx0}
  % \wk{rx0}{wx1}
  % \wk{ry0}{wy1}
  % \sync{rx1}{ry0}
  % \sync{ry1}{rx0}
  %   \end{tikzinlinesmall}}}
  %\\[2ex]
  \taglabel{MCA2}
  \begin{gathered}
  \IF{z}\THEN x\GETS0 \FI \SEMI x\GETS1
  {\PAR}
  \IF{y}\THEN z\GETS0 \FI \SEMI z\GETS1
  {\PAR}
  \IF{x}\THEN y\GETS0 \FI \SEMI y\GETS1
  \\[-1.5ex]
  \hbox{\begin{tikzinline}[node distance=1em]
  \event{a1}{\DR{z}{1}}{}
  \event{a2}{\DW{x}{0}}{right=of a1}
  \po{a1}{a2}
  \event{a3}{\DW{x}{1}}{right=of a2}
  \wk{a2}{a3}
  \event{b1}{\DR{x}{1}}{right=3em of a3}
  \event{b2}{\DW{y}{0}}{right=of b1}
  \po{b1}{b2}
  \event{b3}{\DW{y}{1}}{right=of b2}
  \wk{b2}{b3}
  \event{c1}{\DR{y}{1}}{right=3em of b3}
  \event{c2}{\DW{z}{0}}{right=of c1}
  \po{c1}{c2}
  \event{c3}{\DW{z}{1}}{right=of c2}
  \wk{c2}{c3}
  \rf{a3}{b1}
  \rf{b3}{c1}
  \rf[out=173,in=7]{c3}{a1}  
    \end{tikzinline}}
\end{gathered}
\end{gather*}
% The analysis follows from a few simple principles.
% In this variant of \iriw\ (Independent Reads of Independent Writes), order 
% is imposed by \emph{coherence} (between the writes), \emph{fulfillment}
% (between read and write), and \emph{fencing} (from acquiring read to relaxed read).
These candidate executions are invalid, due to cycles.
% Note that if you enforce the dependency between the reads, then the execution
% is disallowed, because the is a cycle on weak edges restricted to $x$:
% \begin{math}
%   (\DW{x}{1})\le (\DR{y}{0}) \gtN (\DW{y}{1})
% \end{math}
% therefore
% \begin{math}
%   (\DW{x}{1}) \gtN  (\DW{y}{1});
% \end{math}
% then
% \begin{math}
%   (\DW{y}{1})\le (\DR{y}{0}) \gtN (\DW{y}{1})
% \end{math}
% therefore
% \begin{math}
%   (\DW{x}{1}) \gtN  (\DW{y}{1});
% \end{math}
% It is also impossible for all threads to read $1$ in the following, due to
% \emph{control dependencies}, \emph{coherence}, and \emph{fulfillment}.
If $y\ACQ$ is changed to $y\RLX$ in \ref{MCA1}, then there would be no order
from $(\DR[\modeRLX]{y}{1})$ to $(\DR{x}{0})$, and the execution would be
allowed.  Since read-read dependencies do not appear in pomset order, the
execution would still be allowed if a control or address dependency were to
be introduced between the reads. See \textsection\ref{sec:limits} for further
discussion.


% \citet{PughWebsite} developed a set of twenty {causality test cases} in the
% process of revising the Java Memory Model (JMM)
% \cite{Manson:2005:JMM:1047659.1040336}.
% %Using hand calculation, we have confirmed that
% Our model gives the desired result for all twenty cases,
% unrolling loops as necessary.
% % confirmed that our model gives the desired result these test cases, with the
% % following caveats: to model TC14 and TC15 we must unroll loops; the behaviors
% % of TC1 and TC8 are only allowed in our model if the inferred range of
% % variables is included as a global assumption in the logic. 
% Our model also gives the desired results for the examples of
% \citet{PughWebsite}, \citet[\textsection 5.3]{SevcikThesis},
% \citet[\textsection 4]{DBLP:conf/esop/BattyMNPS15}.  Our model agrees with
% ``surprising and controversial behaviors'' of
% \citet[\textsection 8]{Manson:2005:JMM:1047659.1040336}.  We elide the
% details.
% \textsection\ref{sec:examples} develops some of these examples.

% \subsection{Par-closure and Case Analysis}
% Par closure.
% \begin{definition}
%   \label{def-par-closed}
%   $\aPSS$ is \emph{par-closed} if $\aPS^i\!\in\aPSS$ implies $\aPS^1\!{\PAR}\aPS^2\!\in\aPSS$\!\!.
% \end{definition}

% \begin{definition}
%   Let $\aPS'\in\sem{\aReg\GETS\aLoc^\amode\SEMI \aCmd}$ if
%   \begin{math}
%     \aPS'= %(\iDRmode{\aLoc}{}) \prefix
%     \sem{\aCmd}[\aLoc/\aReg]
%   \end{math}
%   or
%   \begin{math}
%     \aPS'=\parallel_{k\in K}(\aForm_k \mid \DRmode\aLoc{\aVal_k}) \prefix \aPS_k
%   \end{math}
%   where $\aPS\in\sem{\aCmd}[\aLoc/\aReg]$ and $\aForm_k$ are
%   disjoint\footnote{For $i$, $j$ in $K$: if $\aForm_i \land \aForm_j$ is
%     satisfiable then $i=j$.}.
% \begin{align*}
%     \sem{\aReg\GETS\aLoc^\amode\SEMI \aCmd} & =
%     \textstyle\bigcup_K\; \parallel_{k\in K}(\aForm_k \mid \DRmode\aLoc{\aVal_k}) \prefix \aPS_k
%     \\[-.5ex] & \mkern2mu\cup
%     %(\iDRmode{\aLoc}{}) \prefix
%     \text{if $\amode\mathbin\neq\modeRLX$ then }\text{$\emptyset$ else }
%     \sem{\aCmd}[\aLoc/\aReg]
%     %\text{ else $\emptyset$}
%     \\
%     \sem{\aLoc^\amode\GETS\aExp\SEMI \aCmd} & \eqdef
%     \parallel_\aVal\; (\aExp=\aVal \mid \DWmode\aLoc\aVal) \prefix \sem{\aCmd}[\aExp/\aLoc]
%     \\[-.5ex] & \mkern2mu\cup
%     %(\iDWmode{\aLoc}{}) \prefix
%     \text{if $\amode\mathbin\neq\modeRLX$ then }\text{$\emptyset$ else }
%     (\relfilt{\aLoc} \sem{\aCmd}[\aExp/\aLoc])
%     %\text{ else $\emptyset$}
% \end{align*}
% \end{definition}

\subsection{Valid and Invalid Rewrites}
\label{sec:valid}

When $\sem{\aCmd} \supseteq \sem{\aCmd'}$, we say that $\aCmd'$ is a
\emph{valid transformation} of $\aCmd$.  In this subsection, we show the
validity of specific optimizations.  
Let $\free(\aCmd)$ be the set of locations and registers that occur in $\aCmd$.

% To enable reasoning about program fragments, transformation validity must be
% preserved by \emph{contexts}.  In \textsection\ref{sec:model}, we defined the
% semantics by prefixing one action at a time.  This helps to make the
% semantics understandable, but it also creates impoverished contexts.

% To allow for richer contexts, we appeal to the alternate presentation of the
% language given in \textsection\ref{sec:semicolon}.  We refactor the syntax
% of commands and define contexts:
% \begin{align*}
%   \aCmd,\,\bCmd
%   \BNFDEF& \SKIP
%   \mkern-2mu\BNFSEP\mkern-2mu \FENCE^{\fmode}
%   \mkern-2mu\BNFSEP\mkern-2mu \aReg\GETS\aExp
%   % \mkern-2mu\BNFSEP\mkern-2mu \aReg\GETS \aLoc^{\amode} 
%   % \mkern-2mu\BNFSEP\mkern-2mu \aLoc^{\amode}\GETS\aExp
%   \mkern-2mu\BNFSEP\mkern-2mu \aReg\GETS \REF{\cExp}^{\amode} 
%   \mkern-2mu\BNFSEP\mkern-2mu \REF{\cExp}^{\amode}\GETS\aExp
%   \\[-.5ex]
%   \BNFSEP&\aCmd \PAR \bCmd
%   \mkern-2mu\BNFSEP\mkern-2mu\aCmd \SEMI \bCmd
%   \mkern-2mu\BNFSEP\mkern-2mu \VAR\aLoc\SEMI \aCmd
%   \mkern-2mu\BNFSEP\mkern-2mu \IF{\aExp} \THEN \aCmd \ELSE \bCmd \FI
%   \\
%   \aCtxt,\,\bCtxt
%   \BNFDEF& \hole{}
%   \mkern-2mu\BNFSEP\mkern-2mu \aCtxt \PAR \bCmd
%   \mkern-2mu\BNFSEP\mkern-2mu \aCmd \PAR \bCtxt
%   \mkern-2mu\BNFSEP\mkern-2mu \aCtxt \SEMI \bCmd
%   \mkern-2mu\BNFSEP\mkern-2mu \aCmd \SEMI \bCtxt
%   \mkern-2mu\BNFSEP\mkern-2mu \VAR\aLoc\SEMI \aCtxt
%   \\[-.5ex]
%   \BNFSEP& \IF{\aExp} \THEN \aCtxt \ELSE \bCmd \FI
%   \mkern-2mu\BNFSEP\mkern-2mu \IF{\aExp} \THEN \aCmd \ELSE \bCtxt \FI
% \end{align*}


% \begin{lemma}%\label{freadssatcomp}[Compositionality of $\freadsat$]
%   Let $\bCtxt$ be a context
%   % \footnote{%
%   %   The results of this section hold for contexts of the example language
%   %   given in \textsection\ref{sec:model} and extended in \textsection\ref{sec:variants}.
%   %   % \begin{math}
%   %   %   \begin{array}[t]{rcl}
%   %   %     \aCtxt,\,\bCtxt
%   %   %     &\BNFDEF& \hole{}
%   %   %     \BNFSEP \aReg\GETS\aExp\SEMI \aCtxt
%   %   %     \BNFSEP \aReg\GETS\aLoc^{\amode}\SEMI \aCtxt 
%   %   %     \BNFSEP \aLoc^{\amode}\GETS\aExp\SEMI \aCtxt
%   %   %     \\[-.5ex]
%   %   %     &\BNFSEP&\aCtxt \PAR \bCmd
%   %   %     \BNFSEP \aCmd \PAR \bCtxt
%   %   %     \BNFSEP \VAR\aLoc\SEMI \aCtxt
%   %   %     \\[-.5ex]
%   %   %     &\BNFSEP& \IF{\aExp} \THEN \aCtxt \ELSE \bCmd \FI
%   %   %     \BNFSEP \IF{\aExp} \THEN \aCmd \ELSE \bCtxt \FI
%   %   %   \end{array}
%   %   % \end{math}
%   %   The results of this section hold for contexts of the example language
%   %   given in \textsection\ref{sec:model} and extended in \textsection\ref{sec:variants}.
%   %   The results also hold for the more general contexts of
%   %   \textsection\ref{sec:semicolon}, which includes full sequential composition:
%   %   \begin{displaymath}
%   %     \begin{array}[t]{rcl}
%   %       \aCtxt,\,\bCtxt
%   %       &\BNFDEF& \hole{}
%   %       \BNFSEP \aCtxt \SEMI \bCmd
%   %       \BNFSEP \aCmd \SEMI \bCtxt
%   %       \BNFSEP \aCtxt \PAR \bCmd
%   %       \BNFSEP \aCmd \PAR \bCtxt
%   %       \BNFSEP 
%   %       \\[-.5ex]
%   %       &\BNFSEP& \VAR\aLoc\SEMI \aCtxt
%   %       \BNFSEP \IF{\aExp} \THEN \aCtxt \ELSE \bCmd \FI
%   %       \BNFSEP \IF{\aExp} \THEN \aCmd \ELSE \bCtxt \FI
%   %     \end{array}
%   %   \end{displaymath}}
% and $\readc\sem{\aCmd} \supseteq \readc\sem{\aCmd'}$:
% \begin{displaymath}
%   \readc\sem{\bCtxt\hole{\aCmd}} \supseteq \readc\sem{\bCtxt\hole{\aCmd'}}
% \end{displaymath}
% \end{lemma}

%

The semantics validates several peephole optimizations.
\begin{align*}
  \taglabel{RL}
  \sem{\aReg \GETS \aLoc \SEMI \bReg  \GETS \aLoc\SEMI\aCmd} &\supseteq 
  \sem{\aReg \GETS \aLoc \SEMI \bReg  \GETS \aReg\SEMI\aCmd}
  \\
  \taglabel{WW}
  \sem{\aLoc \GETS \aExp \SEMI \bLoc  \GETS \bExp\SEMI\aCmd} &=
  \sem{\bLoc  \GETS \bExp\SEMI \aLoc \GETS \aExp\SEMI\aCmd} &&\text{if } \aLoc\neq\bLoc
  \\
  \taglabel{RW}
  \sem{\aReg \GETS \aLoc \SEMI \bLoc  \GETS \bExp\SEMI\aCmd} &=
  \sem{\bLoc  \GETS \bExp\SEMI \aReg \GETS \aLoc\SEMI\aCmd} &&\text{if } \disjoint{{\free(\aReg \GETS \aLoc)}}{{\free(\bLoc \GETS \bExp)}}
  \\
  \taglabel{RR}
  \sem{\aReg \GETS \aLoc \SEMI \bReg  \GETS \bLoc\SEMI\aCmd} &=
  \sem{\bReg  \GETS \bLoc\SEMI \aReg \GETS \aLoc\SEMI\aCmd} &&\text{if } \aReg\neq\bReg
\end{align*}
% \ref{WW}, \ref{RW} and \ref{RR} require that two sides of the semicolon
% have disjoint ids; for example, \ref{RW} requires $\disjoint{{\free(\aReg
%     \GETS \aLoc)}}{{\free(\bLoc \GETS \bExp)}}$. 
% \ref{RR} requires either $\aReg\neq\bReg$ or
%   $\aLoc=\bLoc$.  \ref{WW} and \ref{RW} require that two sides of the
% semicolon have disjoint ids; for example, \ref{RW} requires
% $\disjoint{{\free(\aReg \GETS \aLoc)}}{{\free(\bLoc \GETS \bExp)}}$.
Redundant load elimination \eqref{RL} follows
from item \ref{1} in the definition of prefixing, taking $\bEv\in\Event$.
The independent reorderings (\ref{WW}, \ref{RW} and \ref{RR})
follow from item \ref{item:5} in the definition of prefixing, since no order
is imposed.  Item \ref{item:5} also validates roach-motel reorderings.  For
example, the rules for relaxed writes are as follows:
  \begin{align*}
    \taglabel{AcqW} 
    \sem{x \GETS \aExp \SEMI\aReg \GETS y\ACQ \SEMI\aCmd} &\supseteq
    \sem{\aReg \GETS y\ACQ  \SEMI x\GETS \aExp \SEMI \aCmd} 
    &&\textif x\neq y&&\hbox{}
    \\
    \taglabel{RelW}
    \sem{y\REL \GETS \bExp \SEMI x \GETS \aExp \SEMI \aCmd} &\supseteq
    \sem{x \GETS \aExp \SEMI y\REL \GETS \bExp \SEMI \aCmd}
    &&\textif x\neq y&&\hbox{}
  \end{align*}

  % Reads on the same location are ordered in our model; thus, read optimizations
% are limited by the power of aliasing analysis
% \cite[\textsection2.3]{DBLP:conf/java/Pugh99}.  If $\aLoc\neq\bLoc$, then:
  Since item \ref{5b} does not impose order between reads of the same
  location, \ref{RR} can allow the possibility that $\aLoc=\bLoc$.  As a
  result, read optimizations are not limited by the power of aliasing
  analysis.  By composing \ref{RR} and \ref{RL}, we validate \ref{CSE}:
\begin{align*}
  \taglabel{CSE}
  \sem{r_1\GETS \aLoc \SEMI
  s\GETS \bLoc \SEMI  
  r_2\GETS \aLoc\SEMI\aCmd}
  \supseteq
  \sem{r_1\GETS \aLoc \SEMI     
    r_2\GETS r_1\SEMI
    s\GETS \bLoc \SEMI\aCmd}
    &&\textif \aReg_2\neq\bReg&&\hbox{}
\end{align*}
%However, this fails if $\aLoc=\bLoc$.


% \begin{displaymathsmall}
%   \sem{r_1\GETS \REF{\aExp} \SEMI
%   s\GETS \REF{\bExp} \SEMI  
%   r_2\GETS \REF{\aExp}}
%   \supseteq
%   \sem{r_1\GETS \REF{\aExp} \SEMI
%   s\GETS \REF{\bExp} \SEMI  
%   r_2\GETS r_1}
% \end{displaymathsmall}
% This holds regardless of whether $\aExp=\bExp$.


% By induction on the length of the pomsets in $\aCmd$, we can use the
% reorderings to establish, more generally, that when $\aCmd$ and $\bCmd$ are
% assignment sequences and $\disjoint{\free(\aCmd)}{\free(\bCmd)}$:
% \begin{gather*}
%   \sem{\aCmd \SEMI \bCmd} = \sem{\bCmd \SEMI \aCmd} 
% \end{gather*}
% Appealing directly to the semantics, we can establish general properties for
% redundant load \ref{RLp}, store forwarding \ref{SFp}, and roach-motel
% \ref{A}, \ref{R}:
% Let $\aCmd$ be
% synchronization-free, with disjoint ids as before:
% \begin{align*}
%   % \taglabelp{RL}
%   % \sem{\aReg \GETS \aLoc  \SEMI \bReg \GETS \aLoc  \SEMI \aCmd} &\supseteq
%   % \sem{\aReg \GETS \aLoc \SEMI \aCmd[\aReg/\bReg]}
%   % \\
%   % \taglabelp{SF} 
%   % \sem{\aLoc \GETS \aExp \SEMI \bReg \GETS \aLoc \SEMI \aCmd} &\supseteq 
%   % \sem{\aLoc \GETS \aExp \SEMI \aCmd[\aExp/\bReg]}  
%   % \\
%   \taglabel{A}
%   \sem{\aCmd \SEMI \bReg \GETS \aLoc \ACQ} &\supseteq
%   \sem{\bReg \GETS \aLoc \ACQ\SEMI  \aCmd}
%   \\
%   \taglabel{R}
%   \sem{\aLoc \REL \GETS \aExp \SEMI \aCmd } &\supseteq
%   \sem{\aCmd \SEMI \aLoc \REL \GETS \aExp }
% \end{align*}
% Roach-motel reorderings that increase the scope of synchronization are valid.
% \begin{lemma}%[Reorderings]
%   Suppose $\cExp\neq\dExp$ is a tautology.  Then the following reorderings hold.
%   \begin{align*}
%     \tag{R-Acq}
%     \sem{\aReg \GETS \REF{\cExp} \SEMI \bReg \GETS \REF{\dExp}\ACQ \SEMI \aCmd} &\supseteq
%     \sem{\bReg \GETS \REF{\dExp}\ACQ\SEMI \aReg \GETS \REF{\cExp} \SEMI \aCmd}
% &&\textif \aReg\notin\free(\dExp) \textand \bReg\notin\free(\cExp)
%     \\
%     \tag{W-Acq} 
%     \sem{\REF{\dExp} \GETS \bExp \SEMI\aReg \GETS \REF{\cExp}\ACQ \SEMI\aCmd} &\supseteq
%     \sem{\aReg \GETS \REF{\cExp}\ACQ  \SEMI \REF{\dExp}\GETS \bExp \SEMI \aCmd} 
%     &&\textif \aReg\notin\free(\dExp) \cup\free(\bExp)
%     \\
% \tag{Rel-R} 
% \sem{\REF{\dExp}\REL \GETS \bExp \SEMI\aReg \GETS \REF{\cExp} \SEMI \aCmd} &\supseteq
%     \sem{ \aReg \GETS \REF{\cExp} \SEMI \REF{\dExp}\REL \GETS \bExp \SEMI \aCmd}
%    &&\textif \aReg\notin\free(\dExp) \cup\free(\bExp)
%     \\
%     \tag{Rel-W}
%     \sem{\REF{\cExp}\REL \GETS \aExp \SEMI \REF{\dExp} \GETS \bExp \SEMI \aCmd} &\supseteq
%     \sem{\REF{\dExp} \GETS \bExp \SEMI \REF{\cExp}\REL \GETS \aExp \SEMI \aCmd}
%   \end{align*}
% \begin{proof}
% The proof follows from noticing that the pomsets in the semantics of the right hand sides are augmentations of a pomset on the left hand side.  
% \end{proof}
% \end{lemma}

% Again, we suppose that $\aCmd$ is an assignment sequence and disjoint
% identifiers.
% \ref{A} and \ref{R} require
% disjoint names, as in \ref{WW} and
% \ref{RW}. In addition, \ref{A} and \ref{R} require that $\aCmd$ is
% synchronization-free.


Many laws hold for the conditional, such as dead code elimination \eqref{DC}
and code lifting \eqref{CL}:
\begin{align*}
  \taglabel{DC}
  \sem{\IF{\aExp}\THEN\aCmd\ELSE\bCmd\FI} &=
  \sem{\aCmd}
  &&\textif \aExp \text{ is a tautology}
  \\
  \taglabel{CL}
  \sem{\IF{\aExp}\THEN\aCmd\ELSE\aCmd\FI} &\supseteq
  \sem{\aCmd}
\end{align*}
Code lifting also applies to program fragments inside a conditional.  For example:
\begin{align*}
  \sem{\IF{\aExp}\THEN x\GETS \bExp \SEMI\aCmd\ELSE x\GETS \bExp \SEMI\bCmd\FI} &\supseteq
  \sem{x\GETS \bExp \SEMI \IF{\aExp}\THEN\aCmd\ELSE\bCmd\FI}
  % \sem{\IF{r{=}1}\THEN x\GETS r\SEMI x\GETS 1 \ELSE x\GETS 1\FI}
  % \supseteq
  % \sem{x\GETS 1 \SEMI \IF{r{=}1}\THEN x\GETS r\FI}
\end{align*}
We discuss the inverse of \ref{CL} in \textsection\ref{sec:refine}.
% \begin{align*}
%   \begin{gathered}
%     \IF{r=1}\THEN x\GETS r\SEMI x\GETS 1 \ELSE x\GETS 1\FI
%     \\
%     \hbox{\begin{tikzinline}[node distance=1em]
%         \event{a}{s \mathbin{\mid}\DR{x}{0}}{}
%         \event{b}{s \land r\mathbin{\mid}\DR{x}{0}}{right=of a}
%       \end{tikzinline}}
%   \end{gathered}
%   &&
%   \begin{gathered}
%     \IF{\lnot s}\THEN \cmdR \FI
%     \\
%     \hbox{\begin{tikzinline}[node distance=1em]
%         \event{a}{\lnot s \mathbin{\mid}\DR{x}{0}}{}
%         \event{b}{\lnot s \land r\mathbin{\mid}\DR{x}{0}}{right=of a}
%       \end{tikzinline}}
%   \end{gathered}
% \end{align*}

% The fact that $\sem{\aCmd}\supseteq \sem{\IF{\aExp}\THEN\aCmd\ELSE\aCmd\FI}$
% follows because the semantics is closed with respect to disjunction:
% \begin{definition}
%   \label{def:dis1}
%   We say that $\aPS^0$ is a \emph{disjunct of $\aPS^1$ and $\aPS^2$} when
%   $\Event^0=\Event^1=\Event^2$, ${\le^0}={\le^1}={\le^2}$,
%   $\labelingAct^0=\labelingAct^1 =\labelingAct^2$, and
%   $\labelingForm^0(\aEv)$ implies 
%   $\labelingForm^1(\aEv)\lor \labelingForm^2(\aEv)$.
%   % We say that $\aPSS$ is \emph{disjunction closed} if
%   % $\aPS^0\in\aPSS$ whenever there are $\{\aPS^1,\,\aPS^2\}\subseteq \aPSS$
%   % such that $\aPS^0$ is a disjunct of $\aPS^1$ and $\aPS^2$.
% \end{definition}
% In proving disjunction closure, the difficult case is prefixing.

% \ref{DC} requires that $\aExp$ be a tautology.

% Suppose $\cExp\neq\dExp$ is a tautology.  Then the following 
%reorderings hold.
%  \begin{align*}
% \begin{align*}
%    \tag{R-R}
%    \sem{\aReg \GETS \REF{\cExp} \SEMI \bReg \GETS \REF{\dExp} 
%\SEMI \aCmd} &=
%    \sem{\bReg \GETS \REF{\dExp} \SEMI \aReg \GETS \REF{\cExp} 
%\SEMI \aCmd}
%    &&\textif \aReg\notin\free(\dExp) \textand \bReg\notin\free(\cExp)
 %   \\
 %   \tag{R-W}
 %   \sem{\aReg \GETS \REF{\cExp} \SEMI \REF{\dExp} \GETS \bExp 
%\SEMI \aCmd} &=
%    \sem{\REF{\dExp} \GETS \bExp \SEMI \aReg \GETS \REF{\cExp} 
%\SEMI \aCmd}
%    &&\textif \aReg\notin\free(\dExp) \cup\free(\bExp)
%    \\
 %   \tag{W-W}
 %   \sem{\REF{\cExp} \GETS \aExp \SEMI \REF{\dExp} \GETS \bExp \SEMI 
%\aCmd} &=
 %   \sem{\REF{\dExp} \GETS \bExp \SEMI \REF{\cExp} \GETS \aExp \SEMI 
%\aCmd}
%  \end{align*}
% \begin{proof}
% The semantics of sequential composition $\aCmd \SEMI \bCmd$, where 
% the only enforced $\lt$ relationships come from conflict on locations or 
%release or acquire actions.    The roach-motel reorderings that increase 
%the scope of synchronization are valid because the pomsets in the 
%semantics of the right hand sides are augmentations of a pomset on the 
%left hand side. 
%\end{proof}
%\end{lemma}


% \begin{lemma}%[Reorderings]
% Suppose $\cExp\neq\dExp$ is a tautology.  Then the following 
%reorderings hold.
 % \begin{align*}
%    \tag{R-Acq}
 %   \sem{\aReg \GETS \REF{\cExp} \SEMI \bReg \GETS \REF{\dExp}\ACQ 
%\SEMI \aCmd} &\supseteq
 %   \sem{\bReg \GETS \REF{\dExp}\ACQ\SEMI \aReg \GETS \REF{\cExp} 
%\SEMI \aCmd}
% &&\textif \aReg\notin\free(\dExp) \textand \bReg\notin\free(\cExp)
 %   \\
 %   \tag{W-Acq} 
 %   \sem{\REF{\dExp} \GETS \bExp \SEMI\aReg \GETS \REF{\cExp}\ACQ 
%\SEMI\aCmd} &\supseteq
%    \sem{\aReg \GETS \REF{\cExp}\ACQ  \SEMI \REF{\dExp}\GETS \bExp 
%\SEMI \aCmd} 
%    &&\textif \aReg\notin\free(\dExp) \cup\free(\bExp)
 %   \\
% \tag{Rel-R} 
%\sem{\REF{\dExp}\REL \GETS \bExp \SEMI\aReg \GETS \REF{\cExp} 
%\SEMI \aCmd} &\supseteq
 %   \sem{ \aReg \GETS \REF{\cExp} \SEMI \REF{\dExp}\REL \GETS \bExp 
%\SEMI \aCmd}
%   &&\textif \aReg\notin\free(\dExp) \cup\free(\bExp)
 %   \\
 %   \tag{Rel-W}
 %   \sem{\REF{\cExp}\REL \GETS \aExp \SEMI \REF{\dExp} \GETS \bExp %\SEMI \aCmd} &\supseteq
 %   \sem{\REF{\dExp} \GETS \bExp \SEMI \REF{\cExp}\REL \GETS \aExp 
%\SEMI \aCmd}
%  \end{align*}
%\begin{proof}
%The proof follows from noticing that the pomsets in the semantics of the 
%right hand sides are augmentations of a pomset on the left hand side.  
%\end{proof}
%\end{lemma}

% \paragraph*{Compiler optimizations.} Reordering and peephole optimizations
% can be combined to describe common compiler optimizations.  We illustrate
% using common subexpression elimination,
% following~\citet{Dolan:2018:BDR:3192366.3192421}:
% Consider the command 
% \begin{math}
%   (\aReg \GETS \aLoc *2  \SEMI \aCmd \SEMI \bReg \GETS \aLoc * 2)
% \end{math}
% where $\aCmd$ is independent of $\aReg$.  Reordering yields
% \begin{math}
%   (\aCmd \SEMI \aReg \GETS \aLoc *2  \SEMI  \bReg \GETS \aLoc * 2),
% \end{math}
% followed by redundant load to yield
% \begin{math}
%   (\aCmd \SEMI \aReg \GETS \aLoc * 2 \SEMI  \bReg \GETS \aReg).
% \end{math}

% Similarly, the treatment of loop-invariant code motion, dead-store
% elimination and constant propagation
% from~\citet{Dolan:2018:BDR:3192366.3192421} follow.

% Since our model is more generous about permitted reorderings, we 
% permit optimizations that they forbid.  Consider:
%\begin{math}
 % (\aReg \GETS \aLoc \SEMI \bLoc \GETS \cLoc  \SEMI \aLoc \GETS 
%\aReg).
%\end{math}
%Reordering, permitted by us, but forbidden by them, yields
%\begin{math}
%  (\aReg \GETS \aLoc \SEMI \aLoc \GETS \aReg \SEMI \bLoc \GETS 
%\cLoc),
%\end{math}
%followed by the valid elimination of redundant load
%\begin{math}
%  (\aReg \GETS \aLoc \SEMI \aLoc \GETS \aReg \SEMI \bLoc \GETS %
%\cLoc).
%\end{math}

As expected, %sequential and
parallel composition commutes with conditionals and declarations, and
conditionals and declarations commute with each other.  For example,
we have \emph{scope extrusion}~\cite{Milner:1999:CMS:329902}:
\begin{align*}
  \taglabel{SE}
  \sem{\aCmd\PAR \VAR\aLoc\SEMI\bCmd} &=
  \sem{\VAR\aLoc\SEMI(\aCmd\PAR\bCmd)}
  &&\text{if } \aLoc\not\in\free(\aCmd)
\end{align*}
As we noted in \cite[\textsection B]{2019-sp}, scope extrusion fails if
fulfillment requirement \ref{rf4}---for every conflicting write $\cEv$,
either $\cEv \gtN \bEv$ or $\aEv \gtN \cEv$---is replaced by an
existential---there is no conflicting write $\cEv$ such that
$\bEv<\cEv<\aEv$\footnote{Such conflicting writes are called \emph{blockers}.
  The universal quantification ensures that parallel threads cannot turn a
  non-blocker into a blocker.  In that paper, we advocated the use of
  \emph{three-valued pomsets}, which use strong order for \ref{rf3} and weak
  order for \ref{rf4}.  The pomsets studied in this paper are far simpler.}.


%$\nu\aLoc\DOT\aPSS\neq \nu\bLoc\DOT\aPSS[\bLoc/\aLoc]$ when $\bLoc\neq\aLoc$.
% $\nu\aLoc\DOT(\aPSS\parallel\nu\aLoc\DOT\bPSS)$ is generally
% not the same as
% $\nu\aLoc\DOT(\aPSS\parallel(\nu\bLoc\DOT\bPSS[\bLoc/\aLoc]))$.
%\todo{Scope extrusion fails for some definitions of fulfillment, see S\& P}
%For example,  % We show sequential scope extrusion \ref{SSE}, which concerns
% sequential composition and location binding:
% \begin{math}
%   % \taglabel{PSE}
%   \sem{\aCmd\PAR \VAR\aLoc\SEMI\bCmd}\allowbreak =
%   \sem{\VAR\aLoc\SEMI(\aCmd\PAR\bCmd)}
%   % \\
%   % \taglabel{SSE}
%   % \sem{\aCmd\SEMI \VAR\aLoc\SEMI\bCmd}\allowbreak &=
%   % \sem{\VAR\aLoc\SEMI(\aCmd\SEMI\bCmd)}
%   % \\
%   % \taglabel{CSE}
%   % \sem{\IF{\aExp}\THEN\aCmd\ELSE \VAR\aLoc\SEMI\bCmd\FI}\allowbreak &=
%   % \sem{\VAR\aLoc\SEMI \IF{\aExp}\THEN\aCmd\ELSE\bCmd\FI}
% \end{math}
% if $\aLoc$ does not appear in $\aCmd$.

%\ref{SSE} requires that $\aLoc$ does not appear in $\aCmd$.

\paragraph{Invalid Rewrites}
%\label{sec:invalid}

The definition of location binding does not validate renaming of locations:
if $\aLoc\neq\bLoc$ then
$\sem{\VAR\bLoc\SEMI\aCmd}\neq\sem{\VAR\aLoc\SEMI\aCmd[\aLoc/\bLoc]}$, even
if $\aCmd$ does not mention~$\aLoc$.  This is consistent with support for
address calculation, which is required by realistic memory allocators.

% \labeltext{As}{page:inlining} a side note,
\ref{Internal2} shows that---like most relaxed models---our model
fails to validate \emph{thread inlining}: The execution above is impossible
for
\begin{math}
  \aReg\GETS x\SEMI
  y\REL\GETS 1\PAR
  \bReg\GETS y\SEMI
  z\GETS \bReg
  \PAR
  x\GETS z.
\end{math}
The write in the first thread cannot discharge the precondition in the
second.

% We discussed the invalidity of variable renaming on page
% \pageref{page:extrusion} and thread inlining on page
% \pageref{page:inlining}.

Some rewrites are invalid in a concurrent setting, such as
relevant read introduction:
\begin{displaymath}
  %\tag{Read-Intro-Invalid}
  \sem{\aReg \GETS \aLoc \SEMI \IF{\aReg {\neq} \aReg} \THEN \cLoc \GETS 1 \FI}
  \not\supseteq
  \sem{\aReg \GETS \aLoc \SEMI \bReg \GETS \aLoc  \SEMI \IF{\aReg {\neq}\bReg} \THEN \cLoc \GETS 1 \FI}
\end{displaymath}
Observationally, these are distinguished by the context %
\begin{math}
  \hole{} \PAR x\GETS1\PAR x\GETS2.
\end{math}

% But even \emph{irrelevant} read introduction is invalid.
% Consider the following example \cite[\textsection1.4.5]{SevcikThesis}:
% \begin{align*}
%   \sem{\IF{\aReg} \THEN \bReg \GETS \aLoc \SEMI \bLoc \GETS \bReg \FI}
%   &\not\supseteq
%   \sem{\bReg \GETS \aLoc \SEMI \IF{\aReg} \THEN \bLoc \GETS \bReg \FI}
% \end{align*}
% This inequation holds even though there is no context in our
% language that distinguishes these programs.

Write introduction is also invalid, even when duplicating an existing write:
\begin{displaymath}
  \sem{\aLoc \GETS 1} 
  \not\supseteq
  \sem{\aLoc \GETS 1 \SEMI \aLoc \GETS 1}
\end{displaymath}
These are distinguished by the context:
\begin{math}
  \hole{} \PAR
  r\GETS x \SEMI
  x\GETS2 \SEMI
  s\GETS x\SEMI
  \IF{\aReg {=} \bReg} \THEN \cLoc \GETS 1 \FI
\end{math}

% General write elimination is invalid, even for non-racy programs:
% \begin{displaymath}
%   \sem{\aLoc\GETS 0 \SEMI\aLoc\GETS 1 \SEMI\bLoc^\modeRA\GETS 1} 
%   \not\supseteq
%   \sem{\aLoc\GETS 0 \SEMI\bLoc^\modeRA\GETS 1}
% \end{displaymath}
% These are distinguished by the context:
% \begin{math}
%   \hole{} \PAR
%   \IF{\bLoc^\modeRA}\THEN \aReg \GETS \aLoc \FI
% \end{math}


% But even \emph{irrelevant} read introduction is invalid:
% $\sem{\SKIP}\not\supseteq\sem{r\GETS x}\not\supseteq\sem{r\GETS x\SEMI r\GETS x}$.  These inequations hold even though there is no context in our
% language that distinguishes these programs.
% % With weaker notions of coherence
% % \cite{Manson:2005:JMM:1047659.1040336}, these
% % commands are indistinguishable.




% // p and q might be aliased
% int i = p
% // concurrent write to p.x by another thread
% int j = q 
% int k = p

% Since p and q only might be aliased, but are not definitely aliased, then the
% use of q cannot be optimized away (if it were known that p and q pointed to
% the same object, then it would be legal to replace the assignments to j and k
% with assignments of the value of i). Consider the case where p and q are in
% fact aliased, and another thread writes to the memory location for p/q
% between the first use of p and the use of q; the use of q will see the
% new value. It will be illegal for the second use of p (stored into k) to
% get the same value as was stored into i. However, a fairly standard compiler
% optimization would involve eliminating the getfield for k and replacing it
% with a reuse of the value stored into i. Un- fortunately, that optimization
% is illegal in any language that requires Coherence.

% One way to think of it is that since a read of a memory location may cause
% the thread to become aware of a write by another thread, it must be treated
% in the compiler as a possible write.



