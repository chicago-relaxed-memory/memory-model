\section{Efficient Implementation on ARM8}
\label{sec:arm}

We show that our semantics compiles efficiently to \armeight{}
\cite{deacon-git,DBLP:journals/pacmpl/PulteFDFSS18} using the translation
strategy of \citet{DBLP:journals/pacmpl/PodkopaevLV19}, which was extended to
SC access by \citet[\textsection5]{imm-sc}: Relaxed access is implemented using
\texttt{ldr}/\texttt{str}, non-relaxed access using \texttt{ldar}/\texttt{stlr},
acquire and other fences using
\texttt{dmb}.\texttt{ld}/\texttt{dmb}.\texttt{sy}.


We consider the fragment of our language where concurrent composition occurs
only at top level and there are no local declarations of the form
$(\VAR\aLoc\SEMI \aCmd)$. We show that any \emph{consistent} \armeight{}
execution graph for this sublanguage can be considered a top-level execution
of our semantics. The key step is constructing the order for the derived
pomset candidate.  We would like to take ${\gtN} = ({\rob} \cup {\reco})^*$,
where $\rob$ is the \armeight{} acyclicity relation, and ${\reco}$ is the
\armeight{} extended coherence order.  But this does not quite work.

The definition is complicated by \armeight's \emph{internal reads}, manifest
in ${\rrfi}$, which relates reads to writes that are fulfilled by the same
thread.  \armeight{} drops $\rob$-order \emph{into} an internal read.  As
discussed in \textsection\ref{sec:litmus}, however, our semantics drops pomset
order \emph{out of} an internal read.  To accommodate this, we drop these
dependencies from the \armeight{} \emph{dependency order before} ($\rdob$)
relation.
%
The relation ${\rdobi}$ is defined from ${\rdob}$ by restricting the order
into and out of a read that is in the codomain of the $\rrfi$ relation. More
formally, let $\bEv\xdobi\aEv$ when $\bEv\xdob\aEv$ and
$\bEv\notin\fcodom(\rrfi), \aEv \notin\fcodom(\rrfi)$.
%
% Let $\bEv\xdobi\aEv$ when $\bEv\xdob\aEv$ and
% $\bEv\notin\fcodom(\rrfi)$.
%
Let $\robi$ be defined as for $\rob$, simply replacing $\rdob$ with $\rdobi$.

% The definition is complicated by \armeight's \emph{internal reads}, manifest
% in ${\rrfi}$, which relates reads to writes that are fulfilled by the same
% thread.  \armeight{} drops $\rob$-order \emph{into} an internal read.  As
% discussed in \textsection\ref{sec:model}, however, our semantics drops pomset
% order \emph{out of} an internal read.  To accommodate this, 
% we define ${\robi}$ from ${\rob}$ by restricting the order into and out of an
% read that is in the codomain of the $\rrfi$ relation.  Formally, we drop
% these dependencies from the \armeight{} \emph{dependency order before}
% ($\rdob$) relation: let $\bEv\xdobi\aEv$ when $\bEv\xdob\aEv$ and
% $\bEv\notin\fcodom(\rrfi), \aEv \notin\fcodom(\rrfi)$.  Then $\robi$ be
% defined as for $\rob$, simply replacing $\rdob$ with $\rdobi$.


For pomset order, we then take ${\gtN}=({\robi}\cup{\reco})^*$.

%We prove the following theorem in \textsection\ref{sec:arm:proof}.
\begin{theorem}
  For any consistent \armeight{} execution graph, the constructed candidate
  is a top-level memory model pomset.
\end{theorem}

The proof for compilation into \tso\ is very similar.  The necessary
properties hold for \tso, where $\rob$ is replaced by (the transitive closure
of) the \tso\ propagation relation \citep{alglave}.

It is worth noting that efficient
compilation is not possible for the earlier Flowing and Pop model
\cite{DBLP:conf/popl/FlurGPSSMDS16}, referenced in
\cite[Fig.~4]{DBLP:conf/fm/LahavV16}, which allows the following:
\begin{gather*}
  %\taglabel{flowing/pop}
  \begin{gathered}
  r\GETS x\SEMI x\GETS 1
  \PAR
  y\GETS x 
  \PAR
  x\GETS y 
  \\[-1.2ex]
  \hbox{\begin{tikzinline}[node distance=1.5em]
      \event{a}{\DR{x}{1}}{}
      \event{b}{d:\DW{x}{1}}{right=of a}
      \wk{a}{b}
      \event{c}{\DR{x}{1}}{right=3em of b}
      \event{d}{\DW{y}{1}}{right=of c}
      \po{c}{d}
      \event{e}{\DR{y}{1}}{right=3em of d}
      \event{f}{e:\DW{x}{1}}{right=of e}
      \po{e}{f}
      \rf{b}{c}
      \rf{d}{e}
      \rf[out=172,in=8]{f}{a}
    \end{tikzinline}}
\end{gathered}
  \taglabel{MCA3}
\end{gather*}
This type of ``big detour'' \cite{alglave} is outlawed by
\armeight.\footnote{There is either a cycle
  $\DR{x}{1}\xpoloc d\xcoe e \xrfex \DR{x}{1}$ % SC-pER-LOC
  or % in ob
  $d\xrfex \DR{x}{1}\xdata \DW{y}{1}\xrfex \DR{y}{1} \xdata e \xcoe d$.}


% weakestmo:
% \begin{verbatim}
% Rsc ldar
% Wsc stlr
% \end{verbatim}
% Comparison to C11 from
% \cite[\textsection8.2]{Dolan:2018:BDR:3192366.3192421}:
% \begin{verbatim}
% volatile read:   dmb ld; ldar R, [x]
% volatile write:  L: ldaxr; stlxr; cbnz L; dmb st
% \end{verbatim}
% The second difference is that our atomic writes have stronger semantics,
% which is why we use atomic exchanges instead of stlr on ARMv8. Consider the
% following, using an atomic (SC atomic) location x and a nonatomic (relaxed)
% location y
% \begin{gather*}
%   r\GETS y\SEMI x^\mSC\GETS1\SEMI s\GETS x
%   \PAR
%   % y\GETS0\SEMI
%   x^\mSC\GETS2 \SEMI y\GETS1
%   \\
%   \hbox{\begin{tikzinline}[node distance=1.5em]
%       \event{a}{\DR{y}{1}}{}
%       \event{b}{\DW[\mSC]{x}{1}}{right=of a}
%       \po{a}{b}
%       \event{bb}{\DR{x}{2}}{right=of b}
%       \po{b}{bb}
%       % \event{c}{\DW{y}{0}}{right=2em of bb}
%       % \event{d}{\DW[\mSC]{x}{2}}{right=of c}
%       % \po{c}{d}
%       \event{d}{\DW[\mSC]{x}{2}}{right=2em of bb}
%       \event{e}{\DW{y}{1}}{right=of d}
%       \po{d}{e}
%       %\rf[out=170,in=10]{d}{bb}
%       \rf{d}{bb}
%       \rf[out=-170,in=-10]{e}{a}
%       \wk[in=165,out=15]{b}{d}
%     \end{tikzinline}}
% \end{gather*}
% In our model, if x = 2 afterwards, then r = 0. This is clear from the
% operational semantics: the step r = y must precede x = 1,which must precede x
% = 2 and y = 1. However, in C++ the outcome x = 2 and r = 1 is possible. In C++, SC
% atomic events are totally ordered, so r = y must happen-before x = 1, which
% must precede in the SC ordering x = 2, which must happen-before y =
% 1. However, these two orderings do not compose, and in particular r = y does
% not happen-before y = 1, and it is permissible for r = y to read-from y = 1.

% This behaviour cannot be explained operationally without either allowing
% reads to read from future writes, or allowing atomic locations to contain
% multiple or incoherent values, so it is not permitted in our simple
% operational model. However, this means that we must choose an alternative
% compilation scheme on ARMv8 and similar architectures.


\section{Local Data Race Freedom and Sequential Consistency}
\label{sec:sc}

We adapt \citeauthor{Dolan:2018:BDR:3192366.3192421}'s
[\citeyear{Dolan:2018:BDR:3192366.3192421}] notion of \emph{Local Data Race
  Freedom (LDRF)} to our setting.

The result requires that locations are properly initialized.  We assume a
sufficient condition: that programs have the form
``$\aLoc_1\GETS\aVal_1\SEMI \cdots \aLoc_n\GETS\aVal_n\SEMI\aCmd$'' where
every location mentioned in $\aCmd$ is some $\aLoc_i$.

We make two further restrictions to simplify the exposition.  To simplify the
definition of \emph{happens-before}, we ban fences and \RMWs.  To simplify
the proof, we assume there are no local declarations of the form
$(\VAR\aLoc\SEMI \aCmd)$.
% We do not believe that
% either of these restrictions is important.

To state the theorem, we require several technical definitions.  The reader
unfamiliar with \citep{Dolan:2018:BDR:3192366.3192421} may prefer to skip to
the examples in the proof sketch, referring back as needed.

\noparagraph{Definitions}
\paragraph{Data Race}
Data races are defined using \emph{program} order $(\rpox)$, not
\emph{pomset} order $(\le)$. %, and thus is stable with respect to augmentation.
In \ref{SB}, for example, $(\DR{x}{0})$ has an $x$-race with $(\DW{x}{1})$,
but not $(\DW{x}{0})$, which is $\rpox$-before it.

It is obvious how to enhance the semantics of prefixing and most other
operators to define $\rpox$.  When combining pomsets using the conditional,
the obvious definition may result in cycles, since $\rpox$-ordered reads may
coalesce---see the discussion of \ref{CA} in \textsection\ref{sec:refine}.  In
this case we include a separate pomset for each way of breaking these cycles.

Because we ignore the features of
\textsection\ref{sec:variants}, we can adopt the simplest definition of
\emph{synchronizes\hyp{}with}~($\rsw$): Let $\bEv\xsw\aEv$ exactly when
$\bEv$ fulfills $\aEv$, $\bEv$ is a release, $\aEv$ is an acquire, and
$\lnot(\bEv\xpox\aEv)$.

Let ${\rhb}=({\rpox}\cup{\rsw})^+$ be the \emph{happens-before} relation.  In
\ref{Pub1}, for example, $(\DW{x}{1})$ happens-before $(\DR{x}{0})$, but this
fails if either $\mRA$ access is relaxed.

Let $L\subseteq\Loc$ be a set of locations.  We say that $\bEv$ \emph{has an
  $L$-race with} $\aEv$ (notation $\bEv\lrace{L}\aEv$) when they \emph{conflict}
(Def.~\ref{def:rf}) at
some location in $L$ and at least one is relaxed, but are unordered by $\rhb$: neither $\bEv\xhb\aEv$ nor
$\aEv\xhb\bEv$.  
% A pomset has a \emph{data race} if there are conflicting events that are
% unordered by $\rhb$.
% We say that $\aPS'$ has an \emph{$L$-race after $\aPS$}
% if $(\exists\{\bEv,\aEv\}\subseteq(\Event'\setminus\Event):\;\bEv\lrace{L}\aEv)$.


\paragraph{Generators}
We say that $\aPS'$ \emph{generates} $\aPS$ if either
$\aPS$ augments $\aPS'$ or $\aPS$ implies $\aPS'$.  For example, the
unordered pomset $(\DR{x}{1})$ $(\DW{y}{1})$ generates the ordered pomset
$(\DR{x}{1})\xpo(\aReg=1\mid\DW{y}{1})$.

We say that $\aPS$ is a \emph{generation-minimal} in $\aPSS$ if $\aPS\in\aPSS$ and
there is no $\aPS\neq\aPS'\in\aPSS$ that generates $\aPS$.

% Let $\PRE{\aPSS}=\{\aPS'\mid\aPS'$ is a downset of some $\aPS \in \aPSS\}$.

Let $\semmin{\aCmd}=\{\aPS\in\sem{\aCmd} \mid \aPS$ is \emph{top-level}
(Def.~\ref{def:top}) and generation-minimal in $\sem{\aCmd}\}$.

\paragraph{Extensions}
% In order to relate pomsets with syntax, we enrich commands with program
% counter labels (\pcs), and we enrich actions with sets of \pcs.  We extend
% the semantics to record these labels.  We take the union of labels when
% coalescing actions via composition or prefixing.
% Program counter labels make the notion of downset sensitive to syntax.

% % In order to relate pomsets with syntax, we enrich both actions and commands
% % with program counter labels (\pcs), extending the semantics of the basic
% % model in the obvious way.  For the refined model of
% % \textsection\ref{sec:refine}, we must handle disjunctions.  

We say that $\aPS'$ \emph{$\aCmd$-extends} $\aPS$ if %$\aPS\in\semmin{\aCmd}$,
$\aPS\neq\aPS'\in\semmin{\aCmd}$ and $\aPS$ is a downset of $\aPS'$.
%$\aPS\in\PRE{\aPS'}$.
%Let $\POST{\aPS'}{\aPSS}$ be the set of extensions to $\aPS'$ in $\aPSS$.

\paragraph{Similarity}
We say that \emph{$\aPS'$ is $\aEv$-similar to $\aPS$} if they differ at most
in (1) pomset order adjacent to $\aEv$ and (2) the value associated with
event $\aEv$, if it is a
read.  % We say they are \emph{similar} if they are $\aEv$-similar for some $\aEv$.
Formally: $\Event'=\Event$, $\labelingForm'=\labelingForm$,
${\le'}\restrict{\Event\setminus\{\aEv\}}={\le}\restrict{\Event\setminus\{\aEv\}}$,
if $\aEv$ is not a read then $\labelingAct'=\labelingAct$, and if $\aEv$ is a
read then
$\labelingAct'\restrict{\Event\setminus\{\aEv\}}=\labelingAct\restrict{\Event\setminus\{\aEv\}}$
and $\labelingAct'(\aEv) = \labelingAct(\aEv)[\aVal'/\aVal]$, for some
$\aVal'$, $\aVal$.



\paragraph{Stability}
We say that $\aPS$ is \emph{$L$-stable in $\aCmd$} if
(1) $\aPS\in\semmin{\aCmd}$, 
(2) $\aPS$ is $\rpox$-convex (nothing missing in program order), and
(3) there is no $\aCmd$-extension of $\aPS$ with a \emph{crossing} $L$-race:
that is, there is no $\bEv\in\Event$, no $\aPS'$ $\aCmd$-extending
$\aPS$, and no $\aEv\in\Event'\setminus\Event$ such that $\bEv\lrace{L}\aEv$.
The empty pomset is $L$-stable.

\paragraph{Sequentiality}
Let ${\pole{L}}={\lt_L}\cup{\rpox}$, where $\lt_L$ is the restriction of $\lt$ to events that access locations in $L$.
We say that $\aPS'$ is \emph{$L$-sequential after $\aPS$} if 
$\aPS'$ is $\rpox$-convex and %
$\pole{L}$ is acyclic in $\Event'\setminus\Event$.
% We say that $\aPS'\in\semmin{\aCmd}$ is \emph{$L$-sequential in $\aCmd$ after $\aPS$} if 
% (1) $\aPS$ is $L$-stable in $\aCmd$, %
% (2) $\aPS'$ $\aCmd$-extends $\aPS$, %
% (3) $\aPS'$ is $\rpox$-convex, and %
% (4) $\pole{L}$ is acyclic in $\aPS'\setminus\aPS$.

% We say that $\bEv$ \emph{$\pole{L}$-intervenes in $\aPS'$} if there are
% $\{\cEv,\aEv\}\subseteq\Event'$ such that $\cEv\pole{L}\bEv\pole{L}\aEv$.

%For the remainder of this section, fix a command $\aCmd$.




% We say that a pomset is $\textit{SC}$ if it is top-level and $(\rpox\cup\le)$
% is acyclic.  %; note that all top-level executions of Candidate~\ref{cand:sc} are SC.


% We say that $\aPS$ is \emph{$L$-unstable in $\aCmd$} if
% $\aPS\in\semmin{\aCmd}$ and either %
% (1) $\aPS$ is not $\rpox$-convex (something missing in program order), or %
% (2) there is a $\aCmd$-extension of $\aPS$ with a \emph{crossing} $L$-race:
% that is, there is some $\bEv\in\Event$, some $\aPS'$ $\aCmd$-extending
% $\aPS$, and some $\aEv\in\Event'\setminus\Event$ such that $\bEv\lrace{L}\aEv$.

% We say that $\aPS$ is \emph{$L$-stable in $\aCmd$} if $\aPS\in\semmin{\aCmd}$
% and $\aPS$ is not $L$-unstable in $\aCmd$.

% We say that $\aPS$ is \emph{$L$-unstable in $\aCmd$} if
% $\aPS\in\semmin{\aCmd}$ and either %
% (1) $\aPS$ is not $\pole{L}$-convex\footnote{That is, there is some $\aPS'$ $\aCmd$-extending
% $\aPS$, some $\{\cEv,\aEv\}\subseteq\Event$ and some
% $\bEv\in\Event'\setminus\Event$ such that $\cEv\pole{L}\bEv\pole{L}\aEv$.}, or %
% (2) there is a $\aCmd$-extension of $\aPS$ with a \emph{crossing} $L$-race\footnote{That is, there is some $\aPS'$ $\aCmd$-extending
% $\aPS$, some $\bEv\in\Event$, and some $\aEv\in\Event'\setminus\Event$ such that $\bEv\lrace{L}\aEv$.}.


\noparagraph{Theorem and Proof Sketch}
\begin{theorem}
  Let $\aPS$ be $L$-stable in $\aCmd$.  Let $\aPS'$ be a $\aCmd$-extension of
  $\aPS$ that is $L$-sequential after $\aPS$.  Let $\aPS''$ be a
  $\aCmd$-extension of $\aPS'$ that is $\rpox$-convex, such that no subset of
  $\Event''$ satisfies these criteria.
  %
  Then either (1) $\aPS''$ is $L$-sequential after $\aPS$ or (2) there is
  some $\aCmd$-extension $\aPS'''$ of $\aPS'$ and some
  $\aEv\in(\Event''\setminus\Event')$ such that (a) $\aPS'''$ is
  $\aEv$-similar to $\aPS''$, (b) $\aPS'''$ is $L$-sequential after $\aPS$,
  and (c) $\bEv\lrace{L}\aEv$, for some $\bEv\in(\Event''\setminus\Event)$.
\end{theorem}
The theorem provides an inductive characterization of \emph{Sequential
  Consistency for Local
  Data-Race Freedom (SC-LDRF)}: Any extension of a $L$-stable pomset is either
$L$-sequential, or is $\aEv$-similar to a $L$-sequential extension that
includes a race involving $\aEv$.
\begin{proof}[Proof Sketch]
  In order to develop a technique to find $\aPS'''$ from $\aPS''$, we analyze
  pomset order in generation-minimal top-level pomsets.  First, we note that
  $\le_*$ (the transitive reduction $\le$) can be decomposed into three
  disjoint relations.  Let ${\rppo}=({\le_*}\cap{\rpox})$ denote
  \emph{preserved} program order, as required by prefixing (Def.~\ref{def:prefix}).  The other two relations are cross-thread subsets of
  $({\le_*}\setminus{\rpox})$, as required by fulfillment (Def.~\ref{def:rf}): $\rrfe$ orders writes before reads, satisfying fulfillment
  requirement \ref{rf3}; $\rxw$ orders read and write accesses before writes,
  satisfying requirement \ref{rf4}. ({Within a thread, \ref{rf3} and
    \ref{rf4} follow from prefixing requirement \ref{5b}, which is included
    in ${\rppo}$.})
    % Then ${\le_*}={\rppo}\cup{\rrfe}\cup{\rxw}$.

    Using this decomposition, we can show the following.
    \begin{lemma}
      Suppose $\aPS''\in\semmin{\aCmd}$ has a read $\aEv$ that is maximal in
      $({\rppo}\cup{\rrfe})$ and such that every $\rpox$-following read is
      also $\le$-following ($\aEv\xpox\bEv$ implies $\aEv\le\bEv$, for every
      read $\bEv$).  Further, suppose there is an $\aEv$-similar $\aPS'''$
      that satisfies the requirements of fulfillment.  Then
      $\aPS'''\in\semmin{\aCmd}$.
    \end{lemma}
    The proof of the lemma follows an inductive construction of
    $\semmin{\aCmd}$, starting from a large set with little order, and
    pruning the set as order is added: We begin with all pomsets generated by
    the semantics without imposing the requirements of fulfillment (including
    only $\rppo$).  We then prune reads which cannot be fulfilled, starting
    with those that are minimally ordered.  This proof is simplified by
    precluding local declarations.

    We can prove a similar result for $({\rpox}\cup{\rrfe})$-maximal read
    and write accesses.

    Turning to the proof of the theorem, if $\aPS''$ is $L$-sequential after
    $\aPS$, then the result follows from (1).  Otherwise, there must be a
    $\pole{L}$ cycle in $\aPS''$ involving all of the actions in
    $(\Event''\setminus\Event')$: If there were no such cycle, then $\aPS''$
    would be $L$-sequential; if there were elements outside the cycle, then
    there would be a subset of $\Event''$ that satisfies these criteria.

    If there is a $({\rpox}\cup{\rrfe})$-maximal access, we select one of
    these as $\aEv$.  If $\aEv$ is a write, we reverse the outgoing order in
    $\rxw$; the ability to reverse this order witnesses the race.  If $\aEv$
    is a read, we switch its fulfilling write to a ``newer'' one, updating
    $\rxw$; the ability to switch witnesses the race.  For
    example, for $\aPS''$ on the left below, we choose the $\aPS'''$ on the
    right;  $\aEv$ is the read of $x$, which races with $(\DW{x}{1})$.  % Program order
    % goes from left to right, with the left thread above the right one.% ; we
    % only show program order explicitly when it is relevant to the example.
    \begin{gather*}
      x\GETS 0 \SEMI y\GETS 0 \SEMI  (x \GETS 1  \SEMI y \GETS 1
      \PAR
      \IF{y}\THEN \aReg \GETS x \FI)
      \\[-.5ex]
      \hbox{\begin{tikzinline}[node distance=1.5em and 2em]
          \event{wy0}{\DW{y}{0}}{}
          \event{wx0}{\DW{x}{0}}{below=of wy0}
          \event{wx1}{\DW{x}{1}}{right=3em of wy0}
          \event{wy1}{\DW{y}{1}}{right=of wx1}
          \event{ry1}{\DR{y}{1}}{below=of wx1}
          \event{rx}{\DR{x}{0}}{below=of wy1}
          \rf[bend right]{wx0}{rx}
          \rf{wy1}{ry1}
          \wk[bend left]{wy0}{wy1}
          \pox{wx1}{wy1}
          \pox{ry1}[below]{rx}
          \wk{rx}{wx1}
          \node(ix)[left=of wx0]{};
          \node(iy)[left=of wy0]{};
          \bgoval[yellow!50]{(ix)(iy)}{P}
          \bgoval[pink!50]{(wx0)(wy0)}{P'\setminus P}
          \bgoval[green!10]{(ry1)(wx1)(rx)(wy1)}{P''\setminus P'}
          \pox{wx0}{wy0}
          \pox{wy0}{wx1}
          \pox{wy0}[below]{ry1}
        \end{tikzinline}}
      \qquad
      \hbox{\begin{tikzinline}[node distance=1.5em and 2em]
          \event{wy0}{\DW{y}{0}}{}
          \event{wx0}{\DW{x}{0}}{below=of wy0}
          \event{wx1}{\DW{x}{1}}{right=3em of wy0}
          \event{wy1}{\DW{y}{1}}{right=of wx1}
          \event{ry1}{\DR{y}{1}}{below=of wx1}
          \event{rx}{\DR{x}{1}}{below=of wy1}
          \rf{wx1}{rx}
          \rf{wy1}{ry1}
          \wk[bend left]{wy0}{wy1}
          \pox{wx1}{wy1}
          \pox{ry1}[below]{rx}
          \wk{wx0}{wx1}
          \node(ix)[left=of wx0]{};
          \node(iy)[left=of wy0]{};
          \bgoval[yellow!50]{(ix)(iy)}{P}
          \bgoval[pink!50]{(wx0)(wy0)}{P'\setminus P}
          \bgoval[green!10]{(ry1)(wx1)(rx)(wy1)}{P'''\setminus P'}
          \pox{wx0}{wy0}
          \pox{wy0}{wx1}
          \pox{wy0}[below]{ry1}
        \end{tikzinline}}
    \end{gather*}    
    It is important that $\aEv$ be $({\rpox}\cup{\rrfe})$-maximal, not just
    $({\rppo}\cup{\rrfe})$-maximal.  The latter criterion would allow us to
    choose $\aEv$ to be the read of $y$, but then there would be no
    $\aEv$-similar pomset: if an execution reads $0$ for $y$ then there is no
    read of $x$, due to the conditional.

    If there is no $({\rpox}\cup{\rrfe})$-maximal access, then all
    cross-thread order must be from $\rrfe$.  In this case, we select a
    $({\rppo}\cup{\rrfe})$-maximal read, switching its fulfilling write to an
    ``older'' one.  As an example, consider the following; once again,
    $\aEv$ is the read of $x$, which races with $(\DW{x}{1})$.
    \begin{gather*}
      x\GETS 0 \SEMI y\GETS 0 \SEMI (\aReg \GETS x  \SEMI y \GETS 1
      \PAR
      \bReg \GETS y \SEMI x \GETS \bReg)
      \\[-.5ex]
      \hbox{\begin{tikzinline}[node distance=1.5em and 2em]
          \event{wx0}{\DW{y}{0}}{}
          \event{ry}{\DR{x}{1}}{right=3em of wx0}
          \event{wx1}{\DW{y}{1}}{right=of ry}
          \event{wy0}{\DW{x}{0}}{below=of wx0}
          \event{rx1}{\DR{y}{1}}{right=3em of wy0}
          \event{wy1}{\DW{x}{1}}{right=of rx1}
          \rf{wx1}{rx1}
          \rf{wy1}{ry}
          \po{rx1}{wy1}
          \pox{ry}{wx1}
          \wk[bend left]{wx0}{wx1}
          \wk[bend right]{wy0}{wy1}
          % \wk{wx0}{rx1}
          % \wk{wy0}{ry}
          \node(ix)[left=of wx0]{};
          \node(iy)[left=of wy0]{};
          \bgoval[yellow!50]{(ix)(iy)}{P}
          \bgoval[pink!50]{(wx0)(wy0)}{P'\setminus P}
          \bgoval[green!10]{(ry)(wx1)(rx1)(wy1)}{P''\setminus P'}
          \pox{wy0}{wx0}
          \pox{wx0}{ry}
          \pox{wx0}[below]{rx1}
        \end{tikzinline}}
      \qquad
      \hbox{\begin{tikzinline}[node distance=1.5em and 2em]
          \event{wx0}{\DW{y}{0}}{}
          \event{ry}{\DR{x}{0}}{right=3em of wx0}
          \event{wx1}{\DW{y}{1}}{right=of ry}
          \event{wy0}{\DW{x}{0}}{below=of wx0}
          \event{rx1}{\DR{y}{1}}{right=3em of wy0}
          \event{wy1}{\DW{x}{1}}{right=of rx1}
          \pox{ry}{wx1}
          \wk[bend left]{wx0}{wx1}
          \rf{wx1}{rx1}
          \rf{wy0}{ry}
          \po{rx1}{wy1}
          \wk{ry}{wy1}
          \node(ix)[left=of wx0]{};
          \node(iy)[left=of wy0]{};
          \bgoval[yellow!50]{(ix)(iy)}{P}
          \bgoval[pink!50]{(wx0)(wy0)}{P'\setminus P}
          \bgoval[green!10]{(ry)(wx1)(rx1)(wy1)}{P'''\setminus P'}
          \pox{wy0}{wx0}
          \pox{wx0}{ry}
          \pox{wx0}[below]{rx1}
        \end{tikzinline}}
    \end{gather*}
    This example requires $(\DW{x}{0})$.  Proper initialization ensures the
    existence of such ``older'' writes.
\end{proof}

    % {$\pole{L}$-covers}\footnote{$\aEv$ $\pole{L}$-covers $\aPS'$ if
    %   $\not\exists\bEv\in(\Event''\setminus\Event'):\;\bEv\pole{L}\aEv$.}
    % $\aPS'$ or

%     Suppose that the $\le$-extremal read is fulfilled by a stale write in $\aPS'$.  In
%     this case, the read is minimal in $\le$, and we break the cycle by reading an ``newer'' value from $\aPS''$.
%     (We include a fence in the following example to break the symmetry.)
%     \begin{gather*}
%       (y\GETS 0 \SEMI   x \GETS 1  \SEMI \aReg \GETS y)
%       \PAR
%       (x\GETS 0 \SEMI  y \GETS 1  \SEMI \FENCE^{\mSC} \SEMI  \bReg \GETS x)
%       \\
%       \hbox{\begin{tikzinline}[node distance=1.5em and 2em]
%           \event{wx0}{\DW{x}{0}}{}
%           \event{wx1}{\DW{x}{1}}{right=3em of wx0}
%           \event{wy0}{\DW{y}{0}}{below=of wx0}
%           \event{wy1}{\DW{y}{1}}{right=3em of wy0}
%           \event{f}{\DFS{\mSC}}{right=of wy1}
%           \event{rx0}{\DR{x}{0}}{right=of f}
%           \event{ry0}{\DR{y}{0}}{above=of rx0}
%           \sync{wy1}{f}
%           \sync{f}{rx0}
%           \rf{wx0}{rx0}
%           \rf{wy0}{ry0}
%           \wk{rx0}{wx1}
%           \wk{ry0}{wy1}
%           \wk{wx0}{wx1}
%           \wk{wy0}{wy1}
%           \pox{wx1}{ry0}
%           \node(ix)[left=of wx0]{};
%           \node(iy)[left=of wy0]{};
%           \bgoval[yellow!50]{(ix)(iy)}{P}
%           \bgoval[pink!50]{(wx0)(wy0)}{P'\setminus P}
%           \bgoval[green!10]{(wx1)(wy1)(f)(rx0)(ry0)}{P''\setminus P'}
%         \end{tikzinline}}
%       \qquad
%       \hbox{\begin{tikzinline}[node distance=1.5em and 2em]
%           \event{wx0}{\DW{x}{0}}{}
%           \event{wx1}{\DW{x}{1}}{right=3em of wx0}
%           \event{wy0}{\DW{y}{0}}{below=of wx0}
%           \event{wy1}{\DW{y}{1}}{right=3em of wy0}
%           \event{f}{\DFS{\mSC}}{right=of wy1}
%           \event{rx0}{\DR{x}{0}}{right=of f}
%           \event{ry0}{\DR{y}{1}}{above=of rx0}
%           \pox{wx1}{ry0}
%           \sync{wy1}{f}
%           \sync{f}{rx0}
%           \rf{wx0}{rx0}
%           \rf{wy1}{ry0}
%           \wk{wx0}{wx1}
%           \wk{wy0}{wy1}
%           \wk{rx0}{wx1}
%           \node(ix)[left=of wx0]{};
%           \node(iy)[left=of wy0]{};
%           \bgoval[yellow!50]{(ix)(iy)}{P}
%           \bgoval[pink!50]{(wx0)(wy0)}{P'\setminus P}
%           \bgoval[green!10]{(wx1)(wy1)(f)(rx0)(ry0)}{P'''\setminus P'}
%         \end{tikzinline}}
%     \end{gather*}

% From the semantics, we deduce that
%     there must be at least one read in this cycle that is extremal
%     w.r.t.~pomset order.  From the semantics, we find the required $\aPS'''$
%     by choosing to read a different value.
% We prove the following theorem in \textsection\ref{drfproof}.
% \begin{theorem}
%   Let $\aPS$ be a generator for $\aCmd$.
%   (a) If $\aPS$ does not have a data race, then $\aPS \in \semsc{\aCmd}$.
%   (b) If $\aPS$ has a data race, then there is some
%     $\aPS'\in \semsc{\aCmd}$ that also has a data race.
% \end{theorem}


% \section{DRF notes}
%\subsection{Future read elimination: WWR$\rightarrow$WRW}

% Example:
% \begin{gather*}
%   (x\GETS 0 \SEMI \aReg \GETS z \SEMI x \GETS 1)
%   \PAR
%   (y\GETS 0 \SEMI \bReg \GETS x \SEMI y \GETS \bReg)
%   \PAR
%   (z\GETS 0 \SEMI \cReg \GETS y \SEMI z \GETS \cReg)
% \\
% \hbox{\begin{tikzinline}[node distance=1.5em and 2em]
% \event{wx0}{\DW{x}{0}}{}
% \event{rz1}{\DR{z}{1}}{right=of wx0}
% \event{wx1}{\DW{x}{1}}{right=of rz1}
% \event{wy0}{\DW{y}{0}}{below=of wx0}
% \event{rx1}{\DR{x}{1}}{right=of wy0}
% \event{wy1}{\DW{y}{1}}{right=of rx1}
% \event{wz0}{\DW{z}{0}}{below=of wy0}
% \event{ry1}{\DR{y}{1}}{right=of wz0}
% \event{wz1}{\DW{z}{1}}{right=of ry1}
% \rf{wx1}{rx1}
% \rf{wz1}{rz1}
% \rf{wy1}{ry1}
% \po{rx1}{wy1}
% \po{ry1}{wz1}
% \wk[bend left]{wx0}{wx1}
% \wk[bend left]{wy0}{wy1}
% \wk[bend right]{wz0}{wz1}
% \end{tikzinline}}
% \qquad
% \hbox{\begin{tikzinline}[node distance=1.5em and 2em]
% \event{wx0}{\DW{x}{0}}{}
% \event{rz1}{\DR{z}{0}}{right=of wx0}
% \event{wx1}{\DW{x}{1}}{right=of rz1}
% \event{wy0}{\DW{y}{0}}{below=of wx0}
% \event{rx1}{\DR{x}{1}}{right=of wy0}
% \event{wy1}{\DW{y}{1}}{right=of rx1}
% \event{wz0}{\DW{z}{0}}{below=of wy0}
% \event{ry1}{\DR{y}{1}}{right=of wz0}
% \event{wz1}{\DW{z}{1}}{right=of ry1}
% \rf{wx1}{rx1}
% \wk{rz1}{wz1}
% \rf{wz0}{rz1}
% \rf{wy1}{ry1}
% \po{rx1}{wy1}
% \po{ry1}{wz1}
% \wk[bend left]{wx0}{wx1}
% \wk[bend left]{wy0}{wy1}
% \end{tikzinline}}
% \end{gather*}
%Example:




% Not possible:
% \begin{gather*}
%   (x\GETS 0 \SEMI \aReg \GETS y  \SEMI x \GETS \aReg)
%   \PAR
%   (y\GETS 0 \SEMI \bReg \GETS x \SEMI y \GETS \bReg)
% \\
% \hbox{\begin{tikzinline}[node distance=1.5em and 2em]
% \event{wx0}{\DW{x}{0}}{}
% \event{ry1}{\DR{y}{1}}{right=of wx0}
% \event{wx1}{\DW{x}{1}}{right=of ry1}
% \event{wy0}{\DW{y}{0}}{below=of wx0}
% \event{rx1}{\DR{x}{1}}{right=of wy0}
% \event{wy1}{\DW{y}{1}}{right=of rx1}
% \rf{wx1}{rx1}
% \rf{wy1}{ry1}
% \po{rx1}{wy1}
% \po{ry1}{wx1}
% \wk[bend left]{wx0}{wx1}
% \wk[bend right]{wy0}{wy1}
% \end{tikzinline}}
% \end{gather*}

% \begin{gather*}
% (y\GETS 0 \SEMI   x \GETS 1  \SEMI \aReg \GETS y)
% \PAR (x\GETS 0 \SEMI  y \GETS 1  \SEMI  \bReg \GETS x)
% \\
% \hbox{\begin{tikzinline}[node distance=1.5em and 2em]
% \event{wx0}{\DW{x}{0}}{}
% \event{wx1}{\DW{x}{1}}{right=of wx0}
% \event{ry0}{\DR{y}{0}}{right=of wx1}
% \event{wy0}{\DW{y}{0}}{below=of wx0}
% \event{wy1}{\DW{y}{1}}{right=of wy0}
% \event{rx0}{\DR{x}{0}}{right=of wy1}
% \rf[bend right]{wx0}{rx0}
% \rf[bend left]{wy0}{ry0}
% \wk{rx0}{wx1}
% \wk{ry0}{wy1}
% % \wk{wx0}{wx1}
% % \wk{wy0}{wy1}
% \end{tikzinline}}
% \qquad
% \hbox{\begin{tikzinline}[node distance=1.5em and 2em]
% \event{wx0}{\DW{x}{0}}{}
% \event{wx1}{\DW{x}{1}}{right=of wx0}
% \event{ry0}{\DR{y}{1}}{right=of wx1}
% \event{wy0}{\DW{y}{0}}{below=of wx0}
% \event{wy1}{\DW{y}{1}}{right=of wy0}
% \event{rx0}{\DR{x}{1}}{right=of wy1}
% \rf{wx1}{rx0}
% \rf{wy1}{ry0}
% \wk{wx0}{wx1}
% \wk{wy0}{wy1}
% \end{tikzinline}}
% \end{gather*}


% \begin{gather*}
% (y\GETS 0 \SEMI \aReg \GETS y \SEMI  x \GETS 1 )
% \PAR (x\GETS 0 \SEMI  \bReg \GETS x \SEMI  y \GETS \bReg + 1)
% \\
% \hbox{\begin{tikzinline}[node distance=1.5em and 2em]
% \event{wx0}{\DW{x}{0}}{}
% \event{ry0}{\DR{y}{0}}{right=of wx0}
% \event{wx1}{\DW{x}{1}}{right=of ry0}
% \event{wy0}{\DW{y}{0}}{below=of wx0}
% \event{rx0}{\DR{x}{0}}{right=of wy0}
% \event{wy1}{\DW{y}{1}}{right=of rx0}
% \po{rx0}{wy1}
% \rf{wx0}{rx0}
% \rf{wy0}{ry0}
% \wk{rx0}{wx1}
% \wk{ry0}{wy1}
% % \wk[bend left]{wx0}{wx1}
% % \wk[bend right]{wy0}{wy1}
% \end{tikzinline}}
% \qquad
% \hbox{\begin{tikzinline}[node distance=1.5em and 2em]
% \event{wx0}{\DW{x}{0}}{}
% \event{ry0}{\DR{y}{1}}{right=of wx0}
% \event{wx1}{\DW{x}{1}}{right=of ry0}
% \event{wy0}{\DW{y}{0}}{below=of wx0}
% \event{rx0}{\DR{x}{0}}{right=of wy0}
% \event{wy1}{\DW{y}{1}}{right=of rx0}
% \po{rx0}{wy1}
% \rf{wx0}{rx0}
% \rf{wy1}{ry0}
% \wk{rx0}{wx1}
% %\wk[bend left]{wx0}{wx1}
% \wk[bend right]{wy0}{wy1}
% \end{tikzinline}}
% \end{gather*}

% \begin{gather*}
% (y\GETS 0 \SEMI   x \GETS 1  \SEMI \aReg \GETS y)
% \PAR (x\GETS 0 \SEMI  \bReg \GETS x \SEMI  y \GETS \bReg + 1)
% \\
% \hbox{\begin{tikzinline}[node distance=1.5em and 2em]
% \event{wx0}{\DW{x}{0}}{}
% \event{wx1}{\DW{x}{1}}{right=of wx0}
% \event{ry0}{\DR{y}{0}}{right=of wx1}
% \event{wy0}{\DW{y}{0}}{below=of wx0}
% \event{rx0}{\DR{x}{0}}{right=of wy0}
% \event{wy1}{\DW{y}{1}}{right=of rx0}
% \po{rx0}{wy1}
% \rf{wx0}{rx0}
% \rf{wy0}{ry0}
% \wk{rx0}{wx1}
% \wk{ry0}{wy1}
% % \wk[bend left]{wx0}{wx1}
% % \wk[bend right]{wy0}{wy1}
% \end{tikzinline}}
% \qquad
% \hbox{\begin{tikzinline}[node distance=1.5em and 2em]
% \event{wx0}{\DW{x}{0}}{}
% \event{wx1}{\DW{x}{1}}{right=of wx0}
% \event{ry0}{\DR{y}{1}}{right=of wx1}
% \event{wy0}{\DW{y}{0}}{below=of wx0}
% \event{rx0}{\DR{x}{0}}{right=of wy0}
% \event{wy1}{\DW{y}{1}}{right=of rx0}
% \po{rx0}{wy1}
% \rf{wx0}{rx0}
% \rf{wy1}{ry0}
% \wk{rx0}{wx1}
% %\wk[bend left]{wx0}{wx1}
% \wk[bend right]{wy0}{wy1}
% \end{tikzinline}}
% \end{gather*}



\noparagraph{Mixed Races}
The premises of the theorem allow us to avoid the complications caused by ``mixed races'' in
\cite{DBLP:conf/ppopp/DongolJR19}.  In the left pomset below, $\aPS''$ is not
an extension of $\aPS'$, since $\aPS'$ is not a downset of $\aPS''$.  
When considering this pomset, we must perform the decomposition on the right.
\begin{gather*}
  (x\GETS 0 \SEMI   x^\mRA \GETS 1)
  \PAR
  (\aReg\GETS x^\mRA)
  \\[-2ex]
  \hbox{\begin{tikzinline}[node distance=1.5em and 2em]
      \event{wx0}{\DW{x}{0}}{}
      \event{wx1}{\DWRel{x}{1}}{right=of wx0}
      \event{rx}{\DRAcq{x}{0}}{below=of wx0}
      \rf{wx0}{rx}
      \pox{wx0}{wx1}
      \wk{rx}{wx1}
      \node(ix)[left=of wx0]{};
      \bgoval[yellow!50]{(ix)}{P}
      \bgoval[pink!50]{(wx0)(wx1)}{P'\setminus P}
      \bgovalright[green!10]{(rx)}{P''\setminus P'}
    \end{tikzinline}}
  \qquad
  \qquad
  \qquad
  \hbox{\begin{tikzinline}[node distance=1.5em and 2em]
      \event{wx0}{\DW{x}{0}}{}
      \event{wx1}{\DWRel{x}{1}}{right=of wx0}
      \event{rx}{\DRAcq{x}{0}}{below=of wx0}
      \rf{wx0}{rx}
      \pox{wx0}{wx1}
      \wk{rx}{wx1}
      \node(ix)[left=of wx0]{};
      \bgoval[yellow!50]{(ix)}{P}
      \bgoval[pink!50]{(wx0)(rx)}{P'\setminus P}
      \bgoval[green!10]{(wx1)}{P''\setminus P'}
    \end{tikzinline}}
\end{gather*}
This affects the inductive order in which we move across pomsets, but does
not affect the set of pomsets that are considered.  This simplification is
enabled by denotational reasoning.
% and a precise characterization of
% dependency.

\noparagraph{Comparison to Java}
In our language, past races are always resolved at a stable point, as in
\ref{Co3}.  As another example, consider the following, which is disallowed
here, but allowed by Java \cite[Ex.~2]{Dolan:2018:BDR:3192366.3192421}.  We
include an SC fence here to mimic the behavior of volatiles in the JMM.
\begin{gather*}
  \taglabel{past}
  \begin{gathered}
  (x\GETS 1 \SEMI   y^\mRA \GETS 1)
  \PAR
  (x\GETS 2 \SEMI \FENCE^\mSC \SEMI \IF{y^\mRA}\THEN r\GETS x \SEMI s\GETS x\FI)
  \\[-2ex]
      \hbox{\begin{tikzinline}[node distance=1.2em]
          \event{wx1}{\DW{x}{1}}{}
          \event{wy1}{\DWRel{y}{1}}{right=of wx0}
          \sync{wx1}{wy1}
          \event{wx2}{\DW{x}{2}}{right=3em of wy1}
          \event{f}{\DFS{\mSC}}{right=of wx2}
          \sync{wx2}{f}
          \event{ry1}{\DR[\mRA]{y}{1}}{right=of f}
          \sync{f}{ry1}
          \event{rx1}{\DR{x}{1}}{right=3em of ry1}
          \event{rx2}{\DR{x}{2}}{right=of rx1}
          \sync{ry1}{rx1}
          \sync[out=15,in=165]{ry1}{rx2}
          \rf[out=10,in=170]{wy1}{ry1}
          \wk[out=10,in=170]{wx1}{wx2}
          \wk[out=-170,in=-10]{rx1}{wx2}
          \bgellipsesmaller[yellow!50]{(wy1)(f)}{}
        \end{tikzinline}}
  % \hbox{\begin{tikzinline}[node distance=1.2em]
  %     \event{wx1}{\DW{x}{1}}{}
  %     \event{wy1}{\DWRel{y}{1}}{right=of wx0}
  %     \sync{wx1}{wy1}
  %     \event{wx2}{\DW{x}{2}}{right=3em of wy1}
  %     \event{f}{\DFS{\mSC}}{right=of wx2}
  %     \sync{wx2}{f}
  %     \event{ry1}{\DR[\mRA]{y}{1}}{right=of f}
  %     \sync{f}{ry1}
  %     \event{rx1}{\DR{x}{1}}{right=5em of ry1}
  %     \event{rx2}{\DR{x}{2}}{right=of rx1}
  %     \sync{ry1}{rx1}
  %     \sync[out=15,in=165]{ry1}{rx2}
  %     \rf[out=10,in=170]{wy1}{ry1}
  %     \wk[out=10,in=170]{wx1}{wx2}
  %     \wk[out=-170,in=-10]{rx1}{wx2}
  %     \bgoval[yellow!50]{(wx1)(ry1)}{}
  %   \end{tikzinline}}
  \end{gathered}
\end{gather*}
The highlighted events are $L$-stable.  The order from $(\DR{x}{1})$ to
$(\DW{x}{2})$ is required by fulfillment, causing the cycle.  If the fence is
removed, there would be no order from $(\DW{x}{2})$ to
$(\DRAcq{y}{1})$, the highlighted events would no longer be $L$-stable, and
the execution would be allowed.  This more relaxed notion of ``past'' is not
expressible using \citeauthor{Dolan:2018:BDR:3192366.3192421}'s
synchronization primitives.

The notion of ``future'' is also richer here.  Consider \cite[Ex.~3]{Dolan:2018:BDR:3192366.3192421}:
\begin{gather*}
  \taglabel{future}
  \begin{gathered}
  (r\GETS 1 \SEMI \REF{r}\GETS 42\SEMI s\GETS \REF{r}\SEMI  x^\mRA\GETS r)
  \PAR
  (r\GETS x \SEMI \REF{r}\GETS 7)
  \\[-1ex]
  \hbox{\begin{tikzinline}[node distance=1.2em]
      \event{a1}{\DW{\REF{1}}{42}}{}
      \event{a2}{\DR{\REF{1}}{7}}{right=of a1}
      \wk{a1}{a2}
      % \event{a3}{\DFS{\mREL}}{right=of a2}
      % \sync{a2}{a3}
      \event{a4}{\DWRel{x}{1}}{right=of a2}
      \sync{a2}{a4}
      \event{b1}{\DR{x}{1}}{right=3em of a4}
      \event{b2}{\DW{\REF{1}}{7}}{right=of b1}
      \po{b1}{b2}
      \rf{a4}{b1}
      \wk[out=170,in=10]{b2}{a2}
    \end{tikzinline}}
  \end{gathered}
\end{gather*}
There is no interesting stable point here.  The execution is disallowed
because of a read from the causal future.  If we changed $x^\mRA$ to
$x^\mRLX$, then there would be no order from $(\DR{\REF{1}}{7})$ to
$(\DW[\mRLX]{x}{1})$, and the execution would be allowed.  The distinction
between ``causal future'' and ``temporal future'' is not expressible in
\citeauthor{Dolan:2018:BDR:3192366.3192421}'s operational semantics.
% $L$-stability allows the theorem to encompass histories that are not entirely
% $L$-sequential:
% \begin{gather*}
%   (\aReg\GETS y\SEMI x^\mRA\GETS1
%   \PAR
%   \aReg\GETS x\SEMI y^\mRA\GETS1)
%   \PAR
%   \big(\aReg\GETS x^\mRA \SEMI \aReg\GETS y^\mRA \SEMI
%   (\aReg\GETS y\SEMI x\GETS2
%   \PAR
%   \aReg\GETS x\SEMI y\GETS2)\big)
% \\
% \hbox{\begin{tikzinline}[node distance=1.5em and 2em]
%     \event{ry}{\DR{y}{1}}{}
%     \event{rx}{\DR{x}{1}}{below=of ry}
%     \event{wx}{\DWRel{x}{1}}{right=of ry}
%     \event{wy}{\DWRel{y}{1}}{right=of rx}
%     \rf{wy}{ry}
%     \rf{wx}{rx}
%     \pox{rx}[below]{wy}
%     \pox{ry}{wx}
%     \event{ax}{\DRAcq{x}{1}}{right=3em of wx}
%     \event{ay}{\DRAcq{x}{1}}{below=of ax}
%     \rf{wx}{ax}
%     \rf{wy}{ay}
%     \node(ix)[right=3em of ax]{$\mathstrut$};
%     \node(iy)[below=of ix]{$\mathstrut$};
%     \event{ry2}{\DR{y}{2}}{right=6em of ax}
%     \event{rx2}{\DR{x}{2}}{below=of ry2}
%     \event{wx2}{\DW{x}{2}}{right=of ry2}
%     \event{wy2}{\DW{y}{2}}{right=of rx2}
%     \sync{ax}{ry2}
%     \sync{ax}{rx2}
%     \sync{ay}{ry2}
%     \sync{ay}{rx2}
%     \rf{wy2}{ry2}
%     \rf{wx2}{rx2}
%     \pox{rx2}[below]{wy2}
%     \pox{ry2}{wx2}
%     \bgoval[yellow!50]{(ry)(rx)(ax)(ay)}{P}
%     \bgoval[pink!50]{(ix)(iy)}{P'\setminus P}
%     \bgoval[green!10]{(ry2)(rx2)(wx2)(wy2)}{P''\setminus P'}
% \end{tikzinline}}
% \end{gather*}
% In this execution, the $L$-stable point $P$ is not $L$-sequential.
% Nonetheless, the theorem guarantees that any races in $P''$ can be found
% sequentially in some $P'''$ --- in this case, by changing either read of $2$
% to $1$.  Further, the theorem guarantees that the race found in $P'''$ is
% between two actions outside of $P$ --- in this case, between the new read of
% $1$ and the write of $2$.

Our definition of $L$-sequentiality does not quite correspond to SC
executions, since actions may be elided by read/write elimination
(\textsection\ref{sec:refine}).  However, for any properly initialized
$L$-sequential pomset that uses elimination, there is larger $L$-sequential
pomset that does not use elimination. This can be shown inductively---in the
inductive step, writes that are introduced can be ignored by existing reads,
and reads that are introduced can be fulfilled, for some value, by some
preceding write.

\endinput

\subsection{Proof chat}
  \textcolor{red}{WIP...}

  
  For such programs, we define up with a procedure that inductively generates
  all the pomsets in $\semmin{\aCmd}$.  It works very much like an opsem, but
  on $\rppo$-order, rather than full program order.

  In a minimal pomset, cross-thread order is only introduced by the
  requirement of fulfillment.

  Here's the procedure.  We are building pomset $\aPS$, with set $\aPSS'$ of
  possible extensions.  
  
  \begin{itemize}
  \item Initially, take $\aPSS'$ to be all of the pomsets generated by
    semantics, without imposing fulfillment requirements (so they just have
    $\rppo$).    
  \item Initially take $\aPS$ to include the initial non-conflicting writes.
    These must be the same in all executions.
  \end{itemize}
  Procedure:
  \begin{itemize}
  \item Pick a set of extensions, each of which includes a single additional
    event\footnote{There is only one, since language is deterministic}.  We
    either one write extension $\aPS'$ or a set of read extensions
    $\{\aPS'_\aVal\mid\aVal\in\Val\}$\footnote{The semantics guarantees we have
      an extension for every $\aVal$.}.  For reads, each pomset adds an event
    reading the given value.  The syntactic marker on the new read event is
    the same across all events.
  \item Compute $\aPS''$. Update $\aPS$ to $\aPS''$ and repeat.
    \begin{itemize}
    \item For a write extension, $\aPS''$ augments $\aPS'$ with $\rfre$-order
      into the new event from any conflicting reads in $\aPS$.
    \item For a read extension $\{\aPS'_\aVal\mid\aVal\in\Val\}$, we pick a
      $\aPS'_\aVal$ that can be fulfilled.  $\aPS''$ augments $\aPS'_\aVal$
      with $\rrfe$-order into the new event from the fulfilling write.
      $\aPS''$ also augments $\aPS'_\aVal$ $\rcoe$-order if needed.
    \end{itemize}
  \end{itemize}
  Claims:
  \begin{itemize}
  \item You can always get to a termination event, no matter which choices
    you make.
  \item Every execution in $\semmin{\aCmd}$ can be generated this way.
  \end{itemize}

  By induction on this procedure, we can show that: If there is a cycle in
  $\pole{L}$, then we can always find an execution that includes all of the
  events in the cycle except one read, which has a new value; further the
  order in the new execution is preserved, except for edges that are adjacent
  to the read.
  \begin{itemize}
  \item The only way to introduce a cycle is to choose to put things in order
    that contradicts $\rpox$. 
  \end{itemize}
  




% Let $P$ be a prefix.  Let $P'$ be an extension of $P$. Then, there exists $P''$ :
% \begin{itemize}
% \item $P''$ extends $P$  
% \item $P''$ is a prefix of $P'$ 
% \item $P''\setminus P$ is $L$-sequential, and
% \end{itemize}
% such that one of the following holds
% \begin{itemize}
% \item 
%   $P'' = P'$ , or
% \item
%   $P''$ has a race involving an event in $P''\setminus P$, or
% \item
%   there is a read event $e$ in $P$ such that :
%   \begin{itemize}
%   \item $e$ is in a race with some event in $P''$ and
%   \item there is an $e'$, reading same location as $e$, such that:
%     $e \cup P''$ is $L$-sequential
%   \end{itemize}
% \end{itemize}
           
