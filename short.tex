\section{Efficient Implementation on ARMv8}
\label{sec:arm}

We consider the fragment of our language where concurrent composition occurs
only at top level and there are no location declarations.  Using the
translation strategy of \citet{DBLP:journals/pacmpl/PodkopaevLV19}, we show
that any \emph{consistent} \armeight{} execution graph for this sublanguage
can be considered a top-level execution of our semantics.  Consistency is
defined by \citet{DBLP:journals/pacmpl/PulteFDFSS18}.  The key step is
constructing the order for the derived pomset candidate.  We would like to
take ${\gtN} = ({\rob} \cup {\reco})^*$, where $\rob$ is the \armeight{}
acyclicity relation, and ${\reco}$ is the \armeight{} extended coherence
order, as discussed after Definition~\ref{def:rf}.  But this does not quite
work.

The definition is complicated by \armeight's \emph{internal reads}, manifest
in ${\rrfi}$, which relates reads to writes that are fulfilled by the same
thread.  \armeight{} drops $\rob$-order \emph{into} an internal read.  As
discussed in \textsection\ref{sec:model}, however, our semantics drops pomset
order \emph{out of} an internal read.  To accommodate this, we drop these
dependencies from the \armeight{} \emph{dependency order before} ($\rdob$)
relation.
%
The relation ${\rdobi}$ is defined from ${\rdob}$ by restricting the order
into and out of a read that is in the codomain of the $\rrfi$ relation. More
formally, let $\bEv\xdobi\aEv$ when $\bEv\xdob\aEv$ and
$\bEv\notin\fcodom(\rrfi), \aEv \notin\fcodom(\rrfi)$.
%
% Let $\bEv\xdobi\aEv$ when $\bEv\xdob\aEv$ and
% $\bEv\notin\fcodom(\rrfi)$.
%
Let $\robi$ be defined as for $\rob$, simply replacing $\rdob$ with $\rdobi$.

% The definition is complicated by \armeight's \emph{internal reads}, manifest
% in ${\rrfi}$, which relates reads to writes that are fulfilled by the same
% thread.  \armeight{} drops $\rob$-order \emph{into} an internal read.  As
% discussed in \textsection\ref{sec:model}, however, our semantics drops pomset
% order \emph{out of} an internal read.  To accommodate this, 
% we define ${\robi}$ from ${\rob}$ by restricting the order into and out of an
% read that is in the codomain of the $\rrfi$ relation.  Formally, we drop
% these dependencies from the \armeight{} \emph{dependency order before}
% ($\rdob$) relation: let $\bEv\xdobi\aEv$ when $\bEv\xdob\aEv$ and
% $\bEv\notin\fcodom(\rrfi), \aEv \notin\fcodom(\rrfi)$.  Then $\robi$ be
% defined as for $\rob$, simply replacing $\rdob$ with $\rdobi$.


For pomset order, we then take ${\gtN}=({\robi}\cup{\reco})^*$.

We prove the following theorem in \textsection\ref{sec:arm:proof}.
\begin{theorem}
  For any consistent \armeight{} execution graph, the constructed candidate
  is a top-level memory model pomset.
\end{theorem}

The proof for compilation into \tso\ is very similar.  The necessary
properties hold for \tso, where $\rob$ is replaced by (the transitive closure
of) the \tso\ propagation relation \citep{alglave}.

% weakestmo:
% \begin{verbatim}
% Rsc ldar
% Wsc stlr
% \end{verbatim}
% Comparison to C11 from
% \cite[\textsection8.2]{Dolan:2018:BDR:3192366.3192421}:
% \begin{verbatim}
% volatile read:   dmb ld; ldar R, [x]
% volatile write:  L: ldaxr; stlxr; cbnz L; dmb st
% \end{verbatim}
% The second difference is that our atomic writes have stronger semantics,
% which is why we use atomic exchanges instead of stlr on ARMv8. Consider the
% following, using an atomic (SC atomic) location x and a nonatomic (relaxed)
% location y
% \begin{gather*}
%   r\GETS y\SEMI x^\modeSC\GETS1\SEMI s\GETS x
%   \PAR
%   % y\GETS0\SEMI
%   x^\modeSC\GETS2 \SEMI y\GETS1
%   \\
%   \hbox{\begin{tikzinline}[node distance=1em]
%       \event{a}{\DR{y}{1}}{}
%       \event{b}{\DW[\modeSC]{x}{1}}{right=of a}
%       \po{a}{b}
%       \event{bb}{\DR{x}{2}}{right=of b}
%       \po{b}{bb}
%       % \event{c}{\DW{y}{0}}{right=2em of bb}
%       % \event{d}{\DW[\modeSC]{x}{2}}{right=of c}
%       % \po{c}{d}
%       \event{d}{\DW[\modeSC]{x}{2}}{right=2em of bb}
%       \event{e}{\DW{y}{1}}{right=of d}
%       \po{d}{e}
%       %\rf[out=170,in=10]{d}{bb}
%       \rf{d}{bb}
%       \rf[out=-170,in=-10]{e}{a}
%       \wk[in=165,out=15]{b}{d}
%     \end{tikzinline}}
% \end{gather*}
% In our model, if x = 2 afterwards, then r = 0. This is clear from the
% operational semantics: the step r = y must precede x = 1,which must precede x
% = 2 and y = 1. However, in C++ the outcome x = 2 and r = 1 is possible. In C++, SC
% atomic events are totally ordered, so r = y must happen-before x = 1, which
% must precede in the SC ordering x = 2, which must happen-before y =
% 1. However, these two orderings do not compose, and in particular r = y does
% not happen-before y = 1, and it is permissible for r = y to read-from y = 1.

% This behaviour cannot be explained operationally without either allowing
% reads to read from future writes, or allowing atomic locations to contain
% multiple or incoherent values, so it is not permitted in our simple
% operational model. However, this means that we must choose an alternative
% compilation scheme on ARMv8 and similar architectures.


\section{Local Data Race Freedom and Sequential Consistency}
\label{sec:sc}

We adapt \citeauthor{Dolan:2018:BDR:3192366.3192421}'s
[\citeyear{Dolan:2018:BDR:3192366.3192421}] notion of \emph{Local Data Race
  Freedom (LDRF)} to our setting.

When constructing a pomset, define \emph{program order} $(\rpox)$ in the
obvious way.  As usual\footnote{To allow this simple definition of $\rsw$, we
  consider the base language of \textsection\ref{sec:model}, which does not
  include fences or \RMW{}s.}, we say that $\bEv$ \emph{synchronizes with}
$\aEv$ (notation $\bEv\xsw\aEv$) exactly when $\bEv$ fulfills $\aEv$, $\bEv$
is a release, $\aEv$ is an acquire, and $\lnot(\bEv\xpox\aEv)$.  Let
${\rhb}=({\rpox}\cup{\rsw})^+$ be the \emph{happens-before} relation.  In
\ref{Pub1}, for example, $(\DW{x}{1})$ happens-before $(\DR{x}{0})$, but
this fails if either $\modeRA$ access is relaxed.



Let $L\subseteq\Loc$ be a set of locations.  We say that $\bEv$ \emph{has an
  $L$-race with} $\aEv$ (notation $\bEv\lrace{L}\aEv$) when they conflict at
some location in $L$, but are unordered by $\rhb$: neither $\bEv\xhb\aEv$ nor
$\aEv\xhb\bEv$.  
% A pomset has a \emph{data race} if there are conflicting events that are
% unordered by $\rhb$.
The definition of $L$-race uses \emph{program} order, not \emph{pomset}
order, and thus is stable with respect to augmentation.
In \ref{SB}, for example,
$(\DR{x}{0})$ has an $x$-race with $(\DW{x}{1})$, but not $(\DW{x}{0})$,
which is $\rpox$-before it.

To state the theorem, we require several technical definitions.  The reader
unfamiliar with \citep{Dolan:2018:BDR:3192366.3192421} may prefer to skip to
the examples that follow the theorem statement, coming back to the
definitions as needed.

% We say that a pomset is $\textit{SC}$ if it is top-level and $(\rpox\cup\le)$
% is acyclic.  %; note that all top-level executions of Candidate~\ref{cand:sc} are SC.

We say that $\aPS'$ \emph{generates} $\aPS$ if either
$\aPS$ augments $\aPS'$ or $\aPS$ implies $\aPS'$.  For example, the
unordered pomset $(\DR{x}{1})$ $(\DW{y}{1})$ generates the ordered pomset
$(\DR{x}{1})\xpo(\aReg=1\mid\DW{y}{1})$.

We say that $\aPS$ is a \emph{generation-minimal} in $\aPSS$ if $\aPS\in\aPSS$ and
there is no $\aPS\neq\aPS'\in\aPSS$ that generates $\aPS$.

Let $\PRE{\aPSS}=\{\aPS'\mid\aPS'$ is a downset of some $\aPS \in \aPSS\}$.

Let $\semmin{\aCmd}=\PRE{}\{\aPS\in\sem{\aCmd} \mid \aPS$ is top-level and
  generation-minimal in $\sem{\aCmd}\}$.

%For the remainder of this section, fix a command $\aCmd$.

We say that $\aPS'$ \emph{$\aCmd$-extends} $\aPS$ if $\aPS\in\semmin{\aCmd}$,
$\aPS'\in\semmin{\aCmd}$, and $\aPS\in\PRE{\aPS'}$.
%Let $\POST{\aPS'}{\aPSS}$ be the set of extensions to $\aPS'$ in $\aPSS$.

We say that $\aPS$ is \emph{$L$-unstable in $\aCmd$} if
$\aPS\in\semmin{\aCmd}$ and either %
(1) $\aPS$ is not $\rpox$-convex (something missing in program order), or %
(2) there is a $\aCmd$-extension of $\aPS$ with a \emph{crossing} $L$-race:
that is, there is some $\bEv\in\Event$, some $\aPS'$ $\aCmd$-extending
$\aPS$, and some $\aEv\in\Event'\setminus\Event$ such that $\bEv\lrace{L}\aEv$.

We say that $\aPS$ is \emph{$L$-stable in $\aCmd$} if $\aPS\in\semmin{\aCmd}$
and $\aPS$ is not $L$-unstable in $\aCmd$.

Note that the empty pomset is $L$-stable.

Let ${\pole{L}}={\lt}\cup{\rpox_L}$, where $\rpox_L$ is the restriction of $\rpox$ to events that read or write locations in $L$.

We say that $\aPS'\in\semmin{\aCmd}$ is \emph{$L$-sequential in $\aCmd$} after $\aPS$ if 
(1) $\aPS$ is $L$-stable in $\aCmd$, %
(2) $\aPS'$ $\aCmd$-extends $\aPS$, %
(3) $\aPS'$ is $\rpox$-convex, and %
(4) $\pole{L}$ is acyclic in $\aPS'\setminus\aPS$.

\begin{theorem}
  Suppose $\aPS$ is $L$-stable in $\aCmd$, $\aPS'$ is an $L$-sequential $\aCmd$-extension of
  $\aPS$, and $\aPS''$ is a nonempty $\aCmd$-extension of $\aPS'$.  Then either
  \begin{enumerate}
  \item \label{xx1}  $\exists\aEv\in(\Event''\setminus\Event')$ that
    \emph{$L$-covers} $\aPS'$:
    $(\not\exists\bEv\in(\Event''\setminus\Event'):\;\bEv\pole{L}\aEv)$ or
  \item  \label{xx2}  $\exists \aPS'''$: an $L$-sequential $\aCmd$-extension of $\aPS'$ with an
    \emph{$L$-race after $\aPS$}:
    $(\exists\{\bEv,\aEv\}\subseteq(\Event'''\setminus\Event):\;\bEv\lrace{L}\aEv)$.
  \end{enumerate}
  \begin{proof}[Proof Sketch]
    Suppose there is no $\aEv$ satisfying \eqref{xx1}.  Then we must find
    $\aPS'''$ satisfying \eqref{xx2}.  From \eqref{xx1}, we deduce that there
    is a minimal $\pole{L}$ cycle in $\aPS''$.  From the semantics, we deduce
    that there must be at least one read in this cycle that is minimal
    w.r.t.~pomset order.  From the semantics, we find the required $\aPS'''$
    by choosing to read a different value.
  \end{proof}
\end{theorem}
% We prove the following theorem in \textsection\ref{drfproof}.
% \begin{theorem}
%   Let $\aPS$ be a generator for $\aCmd$.
%   (a) If $\aPS$ does not have a data race, then $\aPS \in \semsc{\aCmd}$.
%   (b) If $\aPS$ has a data race, then there is some
%     $\aPS'\in \semsc{\aCmd}$ that also has a data race.
% \end{theorem}


% \section{DRF notes}
%\subsection{Future read elimination: WWR$\rightarrow$WRW}

% Example:
% \begin{gather*}
%   (x\GETS 0 \SEMI \aReg \GETS z \SEMI x \GETS 1)
%   \PAR
%   (y\GETS 0 \SEMI \bReg \GETS x \SEMI y \GETS \bReg)
%   \PAR
%   (z\GETS 0 \SEMI \cReg \GETS y \SEMI z \GETS \cReg)
% \\
% \hbox{\begin{tikzinline}[node distance=1em and 2em]
% \event{wx0}{\DW{x}{0}}{}
% \event{rz1}{\DR{z}{1}}{right=of wx0}
% \event{wx1}{\DW{x}{1}}{right=of rz1}
% \event{wy0}{\DW{y}{0}}{below=of wx0}
% \event{rx1}{\DR{x}{1}}{right=of wy0}
% \event{wy1}{\DW{y}{1}}{right=of rx1}
% \event{wz0}{\DW{z}{0}}{below=of wy0}
% \event{ry1}{\DR{y}{1}}{right=of wz0}
% \event{wz1}{\DW{z}{1}}{right=of ry1}
% \rf{wx1}{rx1}
% \rf{wz1}{rz1}
% \rf{wy1}{ry1}
% \po{rx1}{wy1}
% \po{ry1}{wz1}
% \wk[bend left]{wx0}{wx1}
% \wk[bend left]{wy0}{wy1}
% \wk[bend right]{wz0}{wz1}
% \end{tikzinline}}
% \qquad
% \hbox{\begin{tikzinline}[node distance=1em and 2em]
% \event{wx0}{\DW{x}{0}}{}
% \event{rz1}{\DR{z}{0}}{right=of wx0}
% \event{wx1}{\DW{x}{1}}{right=of rz1}
% \event{wy0}{\DW{y}{0}}{below=of wx0}
% \event{rx1}{\DR{x}{1}}{right=of wy0}
% \event{wy1}{\DW{y}{1}}{right=of rx1}
% \event{wz0}{\DW{z}{0}}{below=of wy0}
% \event{ry1}{\DR{y}{1}}{right=of wz0}
% \event{wz1}{\DW{z}{1}}{right=of ry1}
% \rf{wx1}{rx1}
% \wk{rz1}{wz1}
% \rf{wz0}{rz1}
% \rf{wy1}{ry1}
% \po{rx1}{wy1}
% \po{ry1}{wz1}
% \wk[bend left]{wx0}{wx1}
% \wk[bend left]{wy0}{wy1}
% \end{tikzinline}}
% \end{gather*}
%Example:

We discuss the required properties of the semantics by example, taking
$L=\{x,y\}$ and $\aPS=\emptyset$.

Suppose that the $\le$-minimal read is fulfilled by a write in $\aPS''$.  In
this case, we break the cycle by reading an ``older'' value from $\aPS'$.
For example, we switch from $\aPS''$ on the left to below to $\aPS'''$ on the
right.  Program order goes from left to right, with the left thread above the
right one; we only show program order explicitly when it is relevant to the
example.
\begin{gather*}
  (x\GETS 0 \SEMI \aReg \GETS y  \SEMI x \GETS 1)
  \PAR
  (y\GETS 0 \SEMI \bReg \GETS x \SEMI y \GETS \bReg)
\\
\hbox{\begin{tikzinline}[node distance=1em and 2em]
\event{wx0}{\DW{x}{0}}{}
\event{ry1}{\DR{y}{1}}{right=3em of wx0}
\event{wx1}{\DW{x}{1}}{right=of ry1}
\event{wy0}{\DW{y}{0}}{below=of wx0}
\event{rx1}{\DR{x}{1}}{right=3em of wy0}
\event{wy1}{\DW{y}{1}}{right=of rx1}
\rf{wx1}{rx1}
\rf{wy1}{ry1}
\po{rx1}{wy1}
\pox{ry1}{wx1}
\wk[bend left]{wx0}{wx1}
\wk[bend right]{wy0}{wy1}
\node(ix)[left=of wx0]{};
\node(iy)[left=of wy0]{};
\bgoval[yellow!50]{(ix)(iy)}{P}
\bgoval[pink!50]{(wx0)(wy0)}{P'\setminus P}
\bgoval[green!10]{(ry1)(wx1)(rx1)(wy1)}{P''\setminus P'}
\end{tikzinline}}
\qquad
\hbox{\begin{tikzinline}[node distance=1em and 2em]
\event{wx0}{\DW{x}{0}}{}
\event{ry1}{\DR{y}{0}}{right=3em of wx0}
\event{wx1}{\DW{x}{1}}{right=of ry1}
\event{wy0}{\DW{y}{0}}{below=of wx0}
\event{rx1}{\DR{x}{1}}{right=3em of wy0}
\event{wy1}{\DW{y}{1}}{right=of rx1}
\pox{ry1}{wx1}
\wk[bend left]{wx0}{wx1}
\rf{wx1}{rx1}
\rf{wy0}{ry1}
\po{rx1}{wy1}
\wk{ry1}{wy1}
\node(ix)[left=of wx0]{};
\node(iy)[left=of wy0]{};
\bgoval[yellow!50]{(ix)(iy)}{P}
\bgoval[pink!50]{(wx0)(wy0)}{P'\setminus P}
\bgoval[green!10]{(ry1)(wx1)(rx1)(wy1)}{P'''\setminus P'}
\end{tikzinline}}
\end{gather*}
% Not possible:
% \begin{gather*}
%   (x\GETS 0 \SEMI \aReg \GETS y  \SEMI x \GETS \aReg)
%   \PAR
%   (y\GETS 0 \SEMI \bReg \GETS x \SEMI y \GETS \bReg)
% \\
% \hbox{\begin{tikzinline}[node distance=1em and 2em]
% \event{wx0}{\DW{x}{0}}{}
% \event{ry1}{\DR{y}{1}}{right=of wx0}
% \event{wx1}{\DW{x}{1}}{right=of ry1}
% \event{wy0}{\DW{y}{0}}{below=of wx0}
% \event{rx1}{\DR{x}{1}}{right=of wy0}
% \event{wy1}{\DW{y}{1}}{right=of rx1}
% \rf{wx1}{rx1}
% \rf{wy1}{ry1}
% \po{rx1}{wy1}
% \po{ry1}{wx1}
% \wk[bend left]{wx0}{wx1}
% \wk[bend right]{wy0}{wy1}
% \end{tikzinline}}
% \end{gather*}

% \begin{gather*}
% (y\GETS 0 \SEMI   x \GETS 1  \SEMI \aReg \GETS y)
% \PAR (x\GETS 0 \SEMI  y \GETS 1  \SEMI  \bReg \GETS x)
% \\
% \hbox{\begin{tikzinline}[node distance=1em and 2em]
% \event{wx0}{\DW{x}{0}}{}
% \event{wx1}{\DW{x}{1}}{right=of wx0}
% \event{ry0}{\DR{y}{0}}{right=of wx1}
% \event{wy0}{\DW{y}{0}}{below=of wx0}
% \event{wy1}{\DW{y}{1}}{right=of wy0}
% \event{rx0}{\DR{x}{0}}{right=of wy1}
% \rf[bend right]{wx0}{rx0}
% \rf[bend left]{wy0}{ry0}
% \wk{rx0}{wx1}
% \wk{ry0}{wy1}
% % \wk{wx0}{wx1}
% % \wk{wy0}{wy1}
% \end{tikzinline}}
% \qquad
% \hbox{\begin{tikzinline}[node distance=1em and 2em]
% \event{wx0}{\DW{x}{0}}{}
% \event{wx1}{\DW{x}{1}}{right=of wx0}
% \event{ry0}{\DR{y}{1}}{right=of wx1}
% \event{wy0}{\DW{y}{0}}{below=of wx0}
% \event{wy1}{\DW{y}{1}}{right=of wy0}
% \event{rx0}{\DR{x}{1}}{right=of wy1}
% \rf{wx1}{rx0}
% \rf{wy1}{ry0}
% \wk{wx0}{wx1}
% \wk{wy0}{wy1}
% \end{tikzinline}}
% \end{gather*}


% \begin{gather*}
% (y\GETS 0 \SEMI \aReg \GETS y \SEMI  x \GETS 1 )
% \PAR (x\GETS 0 \SEMI  \bReg \GETS x \SEMI  y \GETS \bReg + 1)
% \\
% \hbox{\begin{tikzinline}[node distance=1em and 2em]
% \event{wx0}{\DW{x}{0}}{}
% \event{ry0}{\DR{y}{0}}{right=of wx0}
% \event{wx1}{\DW{x}{1}}{right=of ry0}
% \event{wy0}{\DW{y}{0}}{below=of wx0}
% \event{rx0}{\DR{x}{0}}{right=of wy0}
% \event{wy1}{\DW{y}{1}}{right=of rx0}
% \po{rx0}{wy1}
% \rf{wx0}{rx0}
% \rf{wy0}{ry0}
% \wk{rx0}{wx1}
% \wk{ry0}{wy1}
% % \wk[bend left]{wx0}{wx1}
% % \wk[bend right]{wy0}{wy1}
% \end{tikzinline}}
% \qquad
% \hbox{\begin{tikzinline}[node distance=1em and 2em]
% \event{wx0}{\DW{x}{0}}{}
% \event{ry0}{\DR{y}{1}}{right=of wx0}
% \event{wx1}{\DW{x}{1}}{right=of ry0}
% \event{wy0}{\DW{y}{0}}{below=of wx0}
% \event{rx0}{\DR{x}{0}}{right=of wy0}
% \event{wy1}{\DW{y}{1}}{right=of rx0}
% \po{rx0}{wy1}
% \rf{wx0}{rx0}
% \rf{wy1}{ry0}
% \wk{rx0}{wx1}
% %\wk[bend left]{wx0}{wx1}
% \wk[bend right]{wy0}{wy1}
% \end{tikzinline}}
% \end{gather*}

% \begin{gather*}
% (y\GETS 0 \SEMI   x \GETS 1  \SEMI \aReg \GETS y)
% \PAR (x\GETS 0 \SEMI  \bReg \GETS x \SEMI  y \GETS \bReg + 1)
% \\
% \hbox{\begin{tikzinline}[node distance=1em and 2em]
% \event{wx0}{\DW{x}{0}}{}
% \event{wx1}{\DW{x}{1}}{right=of wx0}
% \event{ry0}{\DR{y}{0}}{right=of wx1}
% \event{wy0}{\DW{y}{0}}{below=of wx0}
% \event{rx0}{\DR{x}{0}}{right=of wy0}
% \event{wy1}{\DW{y}{1}}{right=of rx0}
% \po{rx0}{wy1}
% \rf{wx0}{rx0}
% \rf{wy0}{ry0}
% \wk{rx0}{wx1}
% \wk{ry0}{wy1}
% % \wk[bend left]{wx0}{wx1}
% % \wk[bend right]{wy0}{wy1}
% \end{tikzinline}}
% \qquad
% \hbox{\begin{tikzinline}[node distance=1em and 2em]
% \event{wx0}{\DW{x}{0}}{}
% \event{wx1}{\DW{x}{1}}{right=of wx0}
% \event{ry0}{\DR{y}{1}}{right=of wx1}
% \event{wy0}{\DW{y}{0}}{below=of wx0}
% \event{rx0}{\DR{x}{0}}{right=of wy0}
% \event{wy1}{\DW{y}{1}}{right=of rx0}
% \po{rx0}{wy1}
% \rf{wx0}{rx0}
% \rf{wy1}{ry0}
% \wk{rx0}{wx1}
% %\wk[bend left]{wx0}{wx1}
% \wk[bend right]{wy0}{wy1}
% \end{tikzinline}}
% \end{gather*}


Suppose that the $\le$-minimal read is fulfilled by a stale write in $\aPS'$.  In
this case, we break the cycle by reading an ``newer'' value from $\aPS''$.
(We include a fence in the following example to break the symmetry.)
\begin{gather*}
(y\GETS 0 \SEMI   x \GETS 1  \SEMI \aReg \GETS y)
\PAR
(x\GETS 0 \SEMI  y \GETS 1  \SEMI \FENCE^{\modeSC} \SEMI  \bReg \GETS x)
\\
\hbox{\begin{tikzinline}[node distance=1em and 2em]
\event{wx0}{\DW{x}{0}}{}
\event{wx1}{\DW{x}{1}}{right=3em of wx0}
\event{wy0}{\DW{y}{0}}{below=of wx0}
\event{wy1}{\DW{y}{1}}{right=3em of wy0}
\event{f}{\DFS{\modeSC}}{right=of wy1}
\event{rx0}{\DR{x}{0}}{right=of f}
\event{ry0}{\DR{y}{0}}{above=of rx0}
\sync{wy1}{f}
\sync{f}{rx0}
\rf{wx0}{rx0}
\rf{wy0}{ry0}
\wk{rx0}{wx1}
\wk{ry0}{wy1}
\wk{wx0}{wx1}
\wk{wy0}{wy1}
\pox{wx1}{ry0}
\node(ix)[left=of wx0]{};
\node(iy)[left=of wy0]{};
\bgoval[yellow!50]{(ix)(iy)}{P}
\bgoval[pink!50]{(wx0)(wy0)}{P'\setminus P}
\bgoval[green!10]{(wx1)(wy1)(f)(rx0)(ry0)}{P''\setminus P'}
\end{tikzinline}}
\qquad
\hbox{\begin{tikzinline}[node distance=1em and 2em]
\event{wx0}{\DW{x}{0}}{}
\event{wx1}{\DW{x}{1}}{right=3em of wx0}
\event{wy0}{\DW{y}{0}}{below=of wx0}
\event{wy1}{\DW{y}{1}}{right=3em of wy0}
\event{f}{\DFS{\modeSC}}{right=of wy1}
\event{rx0}{\DR{x}{0}}{right=of f}
\event{ry0}{\DR{y}{1}}{above=of rx0}
\pox{wx1}{ry0}
\sync{wy1}{f}
\sync{f}{rx0}
\rf{wx0}{rx0}
\rf{wy1}{ry0}
\wk{wx0}{wx1}
\wk{wy0}{wy1}
\wk{rx0}{wx1}
\node(ix)[left=of wx0]{};
\node(iy)[left=of wy0]{};
\bgoval[yellow!50]{(ix)(iy)}{P}
\bgoval[pink!50]{(wx0)(wy0)}{P'\setminus P}
\bgoval[green!10]{(wx1)(wy1)(f)(rx0)(ry0)}{P'''\setminus P'}
\end{tikzinline}}
\end{gather*}

Because we define $L$-stability using $\pole{L}$ rather than $\rpox$, this
result does not suffer from the complications caused by ``mixed races'' in
\citep{DBLP:conf/ppopp/DongolJR19}.  In particular, the $\aPS'$ chosen on the
left below is \emph{not} $L$-sequential, because it is not $\pole{L}$-convex.
\begin{gather*}
(x\GETS 0 \SEMI   x\REL \GETS 1)
\PAR
(\aReg\GETS x\ACQ)
\\
\hbox{\begin{tikzinline}[node distance=1em and 2em]
\event{wx0}{\DW{x}{0}}{}
\event{wx1}{\DWRel{x}{1}}{right=of wx0}
\event{rx}{\DRAcq{x}{0}}{below=of wx0}
\rf{wx0}{rx}
\pox{wx0}{wx1}
\wk{rx}{wx1}
\node(ix)[left=of wx0]{};
\bgoval[yellow!50]{(ix)}{P}
\bgoval[pink!50]{(wx0)(wx1)}{P'\setminus P}
\bgovalright[green!10]{(rx)}{P''\setminus P'}
\end{tikzinline}}
\qquad
\qquad
\hbox{\begin{tikzinline}[node distance=1em and 2em]
\event{wx0}{\DW{x}{0}}{}
\event{wx1}{\DWRel{x}{1}}{right=of wx0}
\event{rx}{\DRAcq{x}{0}}{below=of wx0}
\rf{wx0}{rx}
\pox{wx0}{wx1}
\wk{rx}{wx1}
\node(ix)[left=of wx0]{};
\bgoval[yellow!50]{(ix)}{P}
\bgoval[pink!50]{(wx0)}{P'\setminus P}
\bgovalright[green!10]{(rx)}{P''\setminus P'}
\end{tikzinline}}
\end{gather*}
To satisfy the premise of the theorem, we must choose $\aPS'$ as on the
right.  This simplification is enabled by denotational reasoning and a
precise characterization of the dependency.


%Write swap (WW$\rightarrow$WW):
