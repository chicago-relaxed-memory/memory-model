\section{Efficient Implementation on ARMv8}
\label{sec:arm}

We consider the fragment of our language where concurrent composition occurs
only at top level and there are no location declarations.  Using the
translation strategy of \citet{DBLP:journals/pacmpl/PodkopaevLV19}, we show
that any \emph{consistent} \armeight{} execution graph for this sublanguage
can be considered a top-level execution of our semantics.  Consistency is
defined by \citet{DBLP:journals/pacmpl/PulteFDFSS18}.  The key step is
constructing the order for the derived pomset candidate.  We would like to
take ${\gtN} = ({\rob} \cup {\reco})^*$, where $\rob$ is the \armeight{}
acyclicity relation, and ${\reco}$ is the \armeight{} extended coherence
order, as discussed after Definition~\ref{def:rf}.  But this does not quite
work.

The definition is complicated by \armeight's \emph{internal reads}, manifest
in ${\rrfi}$, which relates reads to writes that are fulfilled by the same
thread.  \armeight{} drops $\rob$-order \emph{into} an internal read.  As
discussed in \textsection\ref{sec:model}, however, our semantics drops pomset
order \emph{out of} an internal read.  To accommodate this, we drop these
dependencies from the \armeight{} \emph{dependency order before} ($\rdob$)
relation.
%
The relation ${\rdobi}$ is defined from ${\rdob}$ by restricting the order
into and out of a read that is in the codomain of the $\rrfi$ relation. More
formally, let $\bEv\xdobi\aEv$ when $\bEv\xdob\aEv$ and
$\bEv\notin\fcodom(\rrfi), \aEv \notin\fcodom(\rrfi)$.
%
% Let $\bEv\xdobi\aEv$ when $\bEv\xdob\aEv$ and
% $\bEv\notin\fcodom(\rrfi)$.
%
Let $\robi$ be defined as for $\rob$, simply replacing $\rdob$ with $\rdobi$.

% The definition is complicated by \armeight's \emph{internal reads}, manifest
% in ${\rrfi}$, which relates reads to writes that are fulfilled by the same
% thread.  \armeight{} drops $\rob$-order \emph{into} an internal read.  As
% discussed in \textsection\ref{sec:model}, however, our semantics drops pomset
% order \emph{out of} an internal read.  To accommodate this, 
% we define ${\robi}$ from ${\rob}$ by restricting the order into and out of an
% read that is in the codomain of the $\rrfi$ relation.  Formally, we drop
% these dependencies from the \armeight{} \emph{dependency order before}
% ($\rdob$) relation: let $\bEv\xdobi\aEv$ when $\bEv\xdob\aEv$ and
% $\bEv\notin\fcodom(\rrfi), \aEv \notin\fcodom(\rrfi)$.  Then $\robi$ be
% defined as for $\rob$, simply replacing $\rdob$ with $\rdobi$.


For pomset order, we then take ${\gtN}=({\robi}\cup{\reco})^*$.

We prove the following theorem in \textsection\ref{sec:arm:proof}.
\begin{theorem}
  For any consistent \armeight{} execution graph, the constructed candidate
  is a top-level memory model pomset.
\end{theorem}

The proof for compilation into \tso\ is very similar.  The necessary
properties hold for \tso, where $\rob$ is replaced by (the transitive closure
of) the \tso\ propagation relation \citep{alglave}.

% weakestmo:
% \begin{verbatim}
% Rsc ldar
% Wsc stlr
% \end{verbatim}
% Comparison to C11 from
% \cite[\textsection8.2]{Dolan:2018:BDR:3192366.3192421}:
% \begin{verbatim}
% volatile read:   dmb ld; ldar R, [x]
% volatile write:  L: ldaxr; stlxr; cbnz L; dmb st
% \end{verbatim}
% The second difference is that our atomic writes have stronger semantics,
% which is why we use atomic exchanges instead of stlr on ARMv8. Consider the
% following, using an atomic (SC atomic) location x and a nonatomic (relaxed)
% location y
% \begin{gather*}
%   r\GETS y\SEMI x^\modeSC\GETS1\SEMI s\GETS x
%   \PAR
%   % y\GETS0\SEMI
%   x^\modeSC\GETS2 \SEMI y\GETS1
%   \\
%   \hbox{\begin{tikzinline}[node distance=1em]
%       \event{a}{\DR{y}{1}}{}
%       \event{b}{\DW[\modeSC]{x}{1}}{right=of a}
%       \po{a}{b}
%       \event{bb}{\DR{x}{2}}{right=of b}
%       \po{b}{bb}
%       % \event{c}{\DW{y}{0}}{right=2em of bb}
%       % \event{d}{\DW[\modeSC]{x}{2}}{right=of c}
%       % \po{c}{d}
%       \event{d}{\DW[\modeSC]{x}{2}}{right=2em of bb}
%       \event{e}{\DW{y}{1}}{right=of d}
%       \po{d}{e}
%       %\rf[out=170,in=10]{d}{bb}
%       \rf{d}{bb}
%       \rf[out=-170,in=-10]{e}{a}
%       \wk[in=165,out=15]{b}{d}
%     \end{tikzinline}}
% \end{gather*}
% In our model, if x = 2 afterwards, then r = 0. This is clear from the
% operational semantics: the step r = y must precede x = 1,which must precede x
% = 2 and y = 1. However, in C++ the outcome x = 2 and r = 1 is possible. In C++, SC
% atomic events are totally ordered, so r = y must happen-before x = 1, which
% must precede in the SC ordering x = 2, which must happen-before y =
% 1. However, these two orderings do not compose, and in particular r = y does
% not happen-before y = 1, and it is permissible for r = y to read-from y = 1.

% This behaviour cannot be explained operationally without either allowing
% reads to read from future writes, or allowing atomic locations to contain
% multiple or incoherent values, so it is not permitted in our simple
% operational model. However, this means that we must choose an alternative
% compilation scheme on ARMv8 and similar architectures.


\section{Local Data Race Freedom and Sequential Consistency}
\label{sec:sc}

We adapt \citeauthor{Dolan:2018:BDR:3192366.3192421}'s
[\citeyear{Dolan:2018:BDR:3192366.3192421}] notion of \emph{Local Data Race
  Freedom (LDRF)} to our setting.

To simplify the result, we use the base language from
\textsection\ref{sec:model} and assume all locations are top-level; that is,
there are no local declarations of the form $(\VAR\aLoc\SEMI \aCmd)$.  We
also assume all locations are properly initialized; a sufficient requirement
is that all programs have the form
``$\aLoc_1\GETS\aVal_1\SEMI \cdots \aLoc_n\GETS\aVal_n\SEMI\aCmd$''
such that every location mentioned in $\aCmd$ is some $\aLoc_i$.

To state the theorem, we require several technical definitions.  The reader
unfamiliar with \citep{Dolan:2018:BDR:3192366.3192421} may prefer to skip to
the examples in the proof sketch, referring back as needed.

\paragraph{Data Race}
Data races are defined using \emph{program} order, not \emph{pomset}
order. %, and thus is stable with respect to augmentation.
In \ref{SB}, for example,
$(\DR{x}{0})$ has an $x$-race with $(\DW{x}{1})$, but not $(\DW{x}{0})$,
which is $\rpox$-before it.

When constructing a pomset, define \emph{program order} $(\rpox)$ in the
obvious way.  As usual\nofootnote{To allow this simple definition of $\rsw$, we
  consider the base language of \textsection\ref{sec:model}, which does not
  include fences or \RMW{}s.}, we say that $\bEv$ \emph{synchronizes with}
$\aEv$ (notation $\bEv\xsw\aEv$) exactly when $\bEv$ fulfills $\aEv$, $\bEv$
is a release, $\aEv$ is an acquire, and $\lnot(\bEv\xpox\aEv)$.  Let
${\rhb}=({\rpox}\cup{\rsw})^+$ be the \emph{happens-before} relation.  In
\ref{Pub1}, for example, $(\DW{x}{1})$ happens-before $(\DR{x}{0})$, but
this fails if either $\modeRA$ access is relaxed.

Let $L\subseteq\Loc$ be a set of locations.  We say that $\bEv$ \emph{has an
  $L$-race with} $\aEv$ (notation $\bEv\lrace{L}\aEv$) when they conflict at
some location in $L$, but are unordered by $\rhb$: neither $\bEv\xhb\aEv$ nor
$\aEv\xhb\bEv$.  
% A pomset has a \emph{data race} if there are conflicting events that are
% unordered by $\rhb$.
% We say that $\aPS'$ has an \emph{$L$-race after $\aPS$}
% if $(\exists\{\bEv,\aEv\}\subseteq(\Event'\setminus\Event):\;\bEv\lrace{L}\aEv)$.


\paragraph{Generators}
We say that $\aPS'$ \emph{generates} $\aPS$ if either
$\aPS$ augments $\aPS'$ or $\aPS$ implies $\aPS'$.  For example, the
unordered pomset $(\DR{x}{1})$ $(\DW{y}{1})$ generates the ordered pomset
$(\DR{x}{1})\xpo(\aReg=1\mid\DW{y}{1})$.

We say that $\aPS$ is a \emph{generation-minimal} in $\aPSS$ if $\aPS\in\aPSS$ and
there is no $\aPS\neq\aPS'\in\aPSS$ that generates $\aPS$.

% Let $\PRE{\aPSS}=\{\aPS'\mid\aPS'$ is a downset of some $\aPS \in \aPSS\}$.

Let $\semmin{\aCmd}=\{\aPS\in\sem{\aCmd} \mid \aPS$ is top-level and
  generation-minimal in $\sem{\aCmd}\}$.

\paragraph{Extensions}
In order to relate pomsets with syntax, we enrich both actions and commands
with program counter labels, extending the semantics in the obvious way.
Program counter labels make the notion of downset sensitive to syntax.

We say that $\aPS'$ \emph{$\aCmd$-extends} $\aPS$ if %$\aPS\in\semmin{\aCmd}$,
$\aPS\neq\aPS'\in\semmin{\aCmd}$ and $\aPS$ is a downset of $\aPS'$.
%$\aPS\in\PRE{\aPS'}$.
%Let $\POST{\aPS'}{\aPSS}$ be the set of extensions to $\aPS'$ in $\aPSS$.

\paragraph{Similarity}
We say that \emph{$\aPS'$ is $\aEv$-similar to $\aPS$} if they differ at most
in (1) the value associated with read event $\aEv$ and (2) pomset order
adjacent to $\aEv$.  % We say they are \emph{similar} if they are $\aEv$-similar for some $\aEv$.
Formally:
$\Event'=\Event$,
$\labelingForm'=\labelingForm$,
${\le'}\restrict{\Event\setminus\{\aEv\}}={\le}\restrict{\Event\setminus\{\aEv\}}$,
$\labelingAct'\restrict{\Event\setminus\{\aEv\}}=\labelingAct\restrict{\Event\setminus\{\aEv\}}$,
and $\labelingAct'(\aEv) =
\labelingAct(\aEv)[\aVal'/\aVal]$ are reads, for some $\aVal'$, $\aVal$.



\paragraph{Stability}
We say that $\aPS$ is \emph{$L$-stable in $\aCmd$} if
(1) $\aPS\in\semmin{\aCmd}$, 
(2) $\aPS$ is $\rpox$-convex (nothing missing in program order), and
(3) there is no $\aCmd$-extension of $\aPS$ with a \emph{crossing} $L$-race:
that is, there is no $\bEv\in\Event$, no $\aPS'$ $\aCmd$-extending
$\aPS$, and no $\aEv\in\Event'\setminus\Event$ such that $\bEv\lrace{L}\aEv$.
The empty pomset is $L$-stable.

\paragraph{Sequentiality}
Let ${\pole{L}}={\lt_L}\cup{\rpox}$, where $\lt_L$ is the restriction of $\lt$ to events that access locations in $L$.
We say that $\aPS'$ is \emph{$L$-sequential after $\aPS$} if 
$\aPS'$ is $\rpox$-convex and %
$\pole{L}$ is acyclic in $\Event'\setminus\Event$.
% We say that $\aPS'\in\semmin{\aCmd}$ is \emph{$L$-sequential in $\aCmd$ after $\aPS$} if 
% (1) $\aPS$ is $L$-stable in $\aCmd$, %
% (2) $\aPS'$ $\aCmd$-extends $\aPS$, %
% (3) $\aPS'$ is $\rpox$-convex, and %
% (4) $\pole{L}$ is acyclic in $\aPS'\setminus\aPS$.

% We say that $\bEv$ \emph{$\pole{L}$-intervenes in $\aPS'$} if there are
% $\{\cEv,\aEv\}\subseteq\Event'$ such that $\cEv\pole{L}\bEv\pole{L}\aEv$.

%For the remainder of this section, fix a command $\aCmd$.




% We say that a pomset is $\textit{SC}$ if it is top-level and $(\rpox\cup\le)$
% is acyclic.  %; note that all top-level executions of Candidate~\ref{cand:sc} are SC.


% We say that $\aPS$ is \emph{$L$-unstable in $\aCmd$} if
% $\aPS\in\semmin{\aCmd}$ and either %
% (1) $\aPS$ is not $\rpox$-convex (something missing in program order), or %
% (2) there is a $\aCmd$-extension of $\aPS$ with a \emph{crossing} $L$-race:
% that is, there is some $\bEv\in\Event$, some $\aPS'$ $\aCmd$-extending
% $\aPS$, and some $\aEv\in\Event'\setminus\Event$ such that $\bEv\lrace{L}\aEv$.

% We say that $\aPS$ is \emph{$L$-stable in $\aCmd$} if $\aPS\in\semmin{\aCmd}$
% and $\aPS$ is not $L$-unstable in $\aCmd$.

% We say that $\aPS$ is \emph{$L$-unstable in $\aCmd$} if
% $\aPS\in\semmin{\aCmd}$ and either %
% (1) $\aPS$ is not $\pole{L}$-convex\footnote{That is, there is some $\aPS'$ $\aCmd$-extending
% $\aPS$, some $\{\cEv,\aEv\}\subseteq\Event$ and some
% $\bEv\in\Event'\setminus\Event$ such that $\cEv\pole{L}\bEv\pole{L}\aEv$.}, or %
% (2) there is a $\aCmd$-extension of $\aPS$ with a \emph{crossing} $L$-race\footnote{That is, there is some $\aPS'$ $\aCmd$-extending
% $\aPS$, some $\bEv\in\Event$, and some $\aEv\in\Event'\setminus\Event$ such that $\bEv\lrace{L}\aEv$.}.


\begin{theorem}
  Let $\aPS$ be $L$-stable in $\aCmd$.  Let $\aPS'$ be a $\aCmd$-extension of
  $\aPS$ that is $L$-sequential after $\aPS$.  Let $\aPS''$ be a
  $\aCmd$-extension of $\aPS'$ that is $\rpox$-convex, such that no subset of
  $\Event''$ satisfies these criteria.
  %
  Then either (1) $\aPS''$ is $L$-sequential after $\aPS$ or (2) there is
  some $\aCmd$-extension $\aPS'''$ of $\aPS'$ and some
  $\aEv\in(\Event''\setminus\Event')$ such that (a) $\aPS'''$ is
  $\aEv$-similar to $\aPS''$, (b) $\aPS'''$ is $L$-sequential after $\aPS$,
  and (c) $\bEv\lrace{L}\aEv$, for some $\bEv\in(\Event''\setminus\Event)$.
\end{theorem}
The theorem provides an inductive characterization of \emph{Local
  Data-Race Freedom (LDRF)}: Any extension of a $L$-stable pomset is either
$L$-sequential, or is $\aEv$-similar to a $L$-sequential extension that
includes a race involving $\aEv$.
\begin{proof}[Proof Sketch]
    In order to develop a technique to find $\aPS'''$ from $\aPS''$, we
    analyze pomset order in generation-minimal top-level pomsets.  First, we
    not that $\le_*$ (the transitive reduction $\le$) can be decomposed into
    three disjoint relations.  Let ${\rppo}=({\le_*}\cap{\rpox})$ denote
    \emph{preserved} program order, as required by prefixing (Definition
    \ref{def:prefix}).  The other two relations are cross-thread\footnote{Within a
      thread, items 3 and 4 follow from prefixing requirement \ref{5b}, which
      is included in ${\rppo}$.} subsets of $({\le_*}\setminus{\rpox})$, as
    required by fulfillment (Definition \ref{def:rf}): $\rrfe$ orders writes
    before reads, as required by item 3 of the definition; $\rxw$ orders
    accesses before writes, as required by item 4.
    % Then ${\le_*}={\rppo}\cup{\rrfe}\cup{\rxw}$.

    Using this decomposition, we can show the following.
    \begin{lemma}
      Suppose $\aPS''\in\semmin{\aCmd}$ has a read $\aEv$ that is maximal in
      $({\rppo}\cup{\rrfe})$ and such that every $\rpox$-following read is
      also $\le$-following ($\aEv\xpox\bEv$ implies $\aEv\le\bEv$, for every
      read $\bEv$).  Further, suppose there is an $\aEv$-similar $\aPS'''$
      that satisfies the requirements of fulfillment.  Then
      $\aPS'''\in\semmin{\aCmd}$.
    \end{lemma}
    The proof of the lemma follows an inductive construction of
    $\semmin{\aCmd}$, starting from a larger set and pruning: We begin with
    all pomsets generated by the semantics without imposing the requirements
    of fulfillment (including only $\rppo$).  We then prune reads which
    cannot be fulfilled, starting with those that are minimally ordered.
    This proof is simplified by precluding local declarations.  

    Turning to the proof of the theorem, if $\aPS''$ is $L$-sequential after
    $\aPS$, then the result follows from (1).  Otherwise, there must be a
    $\pole{L}$ cycle in $\aPS''$ involving all of the actions in
    $(\Event''\setminus\Event')$: If there were no such cycle, then $\aPS''$
    would be $L$-sequential; if there were elements outside the cycle, then
    there would be a subset of $\Event''$ that satisfies these criteria.

    If there is a $({\rpox}\cup{\rrfe})$-maximal access, we select one of
    these as $\aEv$.  If $\aEv$ is a write, we reverse the outgoing order in
    $\rxw$.  If $\aEv$ is a read, we switch its fulfilling write to a
    ``newer'' one, updating $\rxw$.  For example, for $\aPS''$ on the left
    below, we choose the $\aPS'''$ on the right: $\aEv$ reads $x$ and
    races with $(\DW{x}{1})$.  % Program order
    % goes from left to right, with the left thread above the right one.% ; we
    % only show program order explicitly when it is relevant to the example.
    \begin{gather*}
      x\GETS 0 \SEMI y\GETS 0 \SEMI  (x \GETS 1  \SEMI y \GETS 1
      \PAR
      \IF{y}\THEN \aReg \GETS x \FI
      \\
      \hbox{\begin{tikzinline}[node distance=1em and 2em]
          \event{wy0}{\DW{y}{0}}{}
          \event{wx0}{\DW{x}{0}}{below=of wy0}
          \event{wx1}{\DW{x}{1}}{right=3em of wy0}
          \event{wy1}{\DW{y}{1}}{right=of wx1}
          \event{ry1}{\DR{y}{1}}{below=of wx1}
          \event{rx}{\DR{x}{0}}{below=of wy1}
          \rf[bend right]{wx0}{rx}
          \rf{wy1}{ry1}
          \wk[bend left]{wy0}{wy1}
          \pox{wx1}{wy1}
          \pox{ry1}[below]{rx}
          \wk{rx}{wx1}
          \node(ix)[left=of wx0]{};
          \node(iy)[left=of wy0]{};
          \bgoval[yellow!50]{(ix)(iy)}{P}
          \bgoval[pink!50]{(wx0)(wy0)}{P'\setminus P}
          \bgoval[green!10]{(ry1)(wx1)(rx)(wy1)}{P'''\setminus P'}
          \pox{wx0}{wy0}
          \pox{wy0}{wx1}
          \pox{wy0}[below]{ry1}
        \end{tikzinline}}
      \qquad
      \hbox{\begin{tikzinline}[node distance=1em and 2em]
          \event{wy0}{\DW{y}{0}}{}
          \event{wx0}{\DW{x}{0}}{below=of wy0}
          \event{wx1}{\DW{x}{1}}{right=3em of wy0}
          \event{wy1}{\DW{y}{1}}{right=of wx1}
          \event{ry1}{\DR{y}{1}}{below=of wx1}
          \event{rx}{\DR{x}{1}}{below=of wy1}
          \rf{wx1}{rx}
          \rf{wy1}{ry1}
          \wk[bend left]{wy0}{wy1}
          \pox{wx1}{wy1}
          \pox{ry1}[below]{rx}
          \wk{wx0}{wx1}
          \node(ix)[left=of wx0]{};
          \node(iy)[left=of wy0]{};
          \bgoval[yellow!50]{(ix)(iy)}{P}
          \bgoval[pink!50]{(wx0)(wy0)}{P'\setminus P}
          \bgoval[green!10]{(ry1)(wx1)(rx)(wy1)}{P'''\setminus P'}
          \pox{wx0}{wy0}
          \pox{wy0}{wx1}
          \pox{wy0}[below]{ry1}
        \end{tikzinline}}
    \end{gather*}    
    Note that it is important that $({\rpox}\cup{\rrfe})$-maximal, not just
    $({\rppo}\cup{\rrfe})$-maximal.  The latter criterion also applies to the
    read of $y$, but there is no similar pomset that reads $0$ for $y$.

    If there is no $({\rpox}\cup{\rrfe})$-maximal access, then all
    cross-thread order must be from $\rrfe$.  In this case, we select a
    $({\rppo}\cup{\rrfe})$-maximal read, switching its fulfilling write to an
    ``older'' one.  As an example, consider the following; once again,
    $\aEv$ reads $x$ and races with $(\DW{x}{1})$.
    \begin{gather*}
      x\GETS 0 \SEMI y\GETS 0 \SEMI (\aReg \GETS x  \SEMI y \GETS 1
      \PAR
      \bReg \GETS y \SEMI x \GETS \bReg)
      \\
      \hbox{\begin{tikzinline}[node distance=1em and 2em]
          \event{wx0}{\DW{y}{0}}{}
          \event{ry}{\DR{x}{1}}{right=3em of wx0}
          \event{wx1}{\DW{y}{1}}{right=of ry}
          \event{wy0}{\DW{x}{0}}{below=of wx0}
          \event{rx1}{\DR{y}{1}}{right=3em of wy0}
          \event{wy1}{\DW{x}{1}}{right=of rx1}
          \rf{wx1}{rx1}
          \rf{wy1}{ry}
          \po{rx1}{wy1}
          \pox{ry}{wx1}
          \wk[bend left]{wx0}{wx1}
          \wk[bend right]{wy0}{wy1}
          % \wk{wx0}{rx1}
          % \wk{wy0}{ry}
          \node(ix)[left=of wx0]{};
          \node(iy)[left=of wy0]{};
          \bgoval[yellow!50]{(ix)(iy)}{P}
          \bgoval[pink!50]{(wx0)(wy0)}{P'\setminus P}
          \bgoval[green!10]{(ry)(wx1)(rx1)(wy1)}{P''\setminus P'}
          \pox{wy0}{wx0}
          \pox{wx0}{ry}
          \pox{wx0}[below]{rx1}
        \end{tikzinline}}
      \qquad
      \hbox{\begin{tikzinline}[node distance=1em and 2em]
          \event{wx0}{\DW{y}{0}}{}
          \event{ry}{\DR{x}{0}}{right=3em of wx0}
          \event{wx1}{\DW{y}{1}}{right=of ry}
          \event{wy0}{\DW{x}{0}}{below=of wx0}
          \event{rx1}{\DR{y}{1}}{right=3em of wy0}
          \event{wy1}{\DW{x}{1}}{right=of rx1}
          \pox{ry}{wx1}
          \wk[bend left]{wx0}{wx1}
          \rf{wx1}{rx1}
          \rf{wy0}{ry}
          \po{rx1}{wy1}
          \wk{ry}{wy1}
          \node(ix)[left=of wx0]{};
          \node(iy)[left=of wy0]{};
          \bgoval[yellow!50]{(ix)(iy)}{P}
          \bgoval[pink!50]{(wx0)(wy0)}{P'\setminus P}
          \bgoval[green!10]{(ry)(wx1)(rx1)(wy1)}{P'''\setminus P'}
          \pox{wy0}{wx0}
          \pox{wx0}{ry}
          \pox{wx0}[below]{rx1}
        \end{tikzinline}}
    \end{gather*}
\end{proof}

    % {$\pole{L}$-covers}\footnote{$\aEv$ $\pole{L}$-covers $\aPS'$ if
    %   $\not\exists\bEv\in(\Event''\setminus\Event'):\;\bEv\pole{L}\aEv$.}
    % $\aPS'$ or

%     Suppose that the $\le$-extremal read is fulfilled by a stale write in $\aPS'$.  In
%     this case, the read is minimal in $\le$, and we break the cycle by reading an ``newer'' value from $\aPS''$.
%     (We include a fence in the following example to break the symmetry.)
%     \begin{gather*}
%       (y\GETS 0 \SEMI   x \GETS 1  \SEMI \aReg \GETS y)
%       \PAR
%       (x\GETS 0 \SEMI  y \GETS 1  \SEMI \FENCE^{\modeSC} \SEMI  \bReg \GETS x)
%       \\
%       \hbox{\begin{tikzinline}[node distance=1em and 2em]
%           \event{wx0}{\DW{x}{0}}{}
%           \event{wx1}{\DW{x}{1}}{right=3em of wx0}
%           \event{wy0}{\DW{y}{0}}{below=of wx0}
%           \event{wy1}{\DW{y}{1}}{right=3em of wy0}
%           \event{f}{\DFS{\modeSC}}{right=of wy1}
%           \event{rx0}{\DR{x}{0}}{right=of f}
%           \event{ry0}{\DR{y}{0}}{above=of rx0}
%           \sync{wy1}{f}
%           \sync{f}{rx0}
%           \rf{wx0}{rx0}
%           \rf{wy0}{ry0}
%           \wk{rx0}{wx1}
%           \wk{ry0}{wy1}
%           \wk{wx0}{wx1}
%           \wk{wy0}{wy1}
%           \pox{wx1}{ry0}
%           \node(ix)[left=of wx0]{};
%           \node(iy)[left=of wy0]{};
%           \bgoval[yellow!50]{(ix)(iy)}{P}
%           \bgoval[pink!50]{(wx0)(wy0)}{P'\setminus P}
%           \bgoval[green!10]{(wx1)(wy1)(f)(rx0)(ry0)}{P''\setminus P'}
%         \end{tikzinline}}
%       \qquad
%       \hbox{\begin{tikzinline}[node distance=1em and 2em]
%           \event{wx0}{\DW{x}{0}}{}
%           \event{wx1}{\DW{x}{1}}{right=3em of wx0}
%           \event{wy0}{\DW{y}{0}}{below=of wx0}
%           \event{wy1}{\DW{y}{1}}{right=3em of wy0}
%           \event{f}{\DFS{\modeSC}}{right=of wy1}
%           \event{rx0}{\DR{x}{0}}{right=of f}
%           \event{ry0}{\DR{y}{1}}{above=of rx0}
%           \pox{wx1}{ry0}
%           \sync{wy1}{f}
%           \sync{f}{rx0}
%           \rf{wx0}{rx0}
%           \rf{wy1}{ry0}
%           \wk{wx0}{wx1}
%           \wk{wy0}{wy1}
%           \wk{rx0}{wx1}
%           \node(ix)[left=of wx0]{};
%           \node(iy)[left=of wy0]{};
%           \bgoval[yellow!50]{(ix)(iy)}{P}
%           \bgoval[pink!50]{(wx0)(wy0)}{P'\setminus P}
%           \bgoval[green!10]{(wx1)(wy1)(f)(rx0)(ry0)}{P'''\setminus P'}
%         \end{tikzinline}}
%     \end{gather*}

% From the semantics, we deduce that
%     there must be at least one read in this cycle that is extremal
%     w.r.t.~pomset order.  From the semantics, we find the required $\aPS'''$
%     by choosing to read a different value.
% We prove the following theorem in \textsection\ref{drfproof}.
% \begin{theorem}
%   Let $\aPS$ be a generator for $\aCmd$.
%   (a) If $\aPS$ does not have a data race, then $\aPS \in \semsc{\aCmd}$.
%   (b) If $\aPS$ has a data race, then there is some
%     $\aPS'\in \semsc{\aCmd}$ that also has a data race.
% \end{theorem}


% \section{DRF notes}
%\subsection{Future read elimination: WWR$\rightarrow$WRW}

% Example:
% \begin{gather*}
%   (x\GETS 0 \SEMI \aReg \GETS z \SEMI x \GETS 1)
%   \PAR
%   (y\GETS 0 \SEMI \bReg \GETS x \SEMI y \GETS \bReg)
%   \PAR
%   (z\GETS 0 \SEMI \cReg \GETS y \SEMI z \GETS \cReg)
% \\
% \hbox{\begin{tikzinline}[node distance=1em and 2em]
% \event{wx0}{\DW{x}{0}}{}
% \event{rz1}{\DR{z}{1}}{right=of wx0}
% \event{wx1}{\DW{x}{1}}{right=of rz1}
% \event{wy0}{\DW{y}{0}}{below=of wx0}
% \event{rx1}{\DR{x}{1}}{right=of wy0}
% \event{wy1}{\DW{y}{1}}{right=of rx1}
% \event{wz0}{\DW{z}{0}}{below=of wy0}
% \event{ry1}{\DR{y}{1}}{right=of wz0}
% \event{wz1}{\DW{z}{1}}{right=of ry1}
% \rf{wx1}{rx1}
% \rf{wz1}{rz1}
% \rf{wy1}{ry1}
% \po{rx1}{wy1}
% \po{ry1}{wz1}
% \wk[bend left]{wx0}{wx1}
% \wk[bend left]{wy0}{wy1}
% \wk[bend right]{wz0}{wz1}
% \end{tikzinline}}
% \qquad
% \hbox{\begin{tikzinline}[node distance=1em and 2em]
% \event{wx0}{\DW{x}{0}}{}
% \event{rz1}{\DR{z}{0}}{right=of wx0}
% \event{wx1}{\DW{x}{1}}{right=of rz1}
% \event{wy0}{\DW{y}{0}}{below=of wx0}
% \event{rx1}{\DR{x}{1}}{right=of wy0}
% \event{wy1}{\DW{y}{1}}{right=of rx1}
% \event{wz0}{\DW{z}{0}}{below=of wy0}
% \event{ry1}{\DR{y}{1}}{right=of wz0}
% \event{wz1}{\DW{z}{1}}{right=of ry1}
% \rf{wx1}{rx1}
% \wk{rz1}{wz1}
% \rf{wz0}{rz1}
% \rf{wy1}{ry1}
% \po{rx1}{wy1}
% \po{ry1}{wz1}
% \wk[bend left]{wx0}{wx1}
% \wk[bend left]{wy0}{wy1}
% \end{tikzinline}}
% \end{gather*}
%Example:




% Not possible:
% \begin{gather*}
%   (x\GETS 0 \SEMI \aReg \GETS y  \SEMI x \GETS \aReg)
%   \PAR
%   (y\GETS 0 \SEMI \bReg \GETS x \SEMI y \GETS \bReg)
% \\
% \hbox{\begin{tikzinline}[node distance=1em and 2em]
% \event{wx0}{\DW{x}{0}}{}
% \event{ry1}{\DR{y}{1}}{right=of wx0}
% \event{wx1}{\DW{x}{1}}{right=of ry1}
% \event{wy0}{\DW{y}{0}}{below=of wx0}
% \event{rx1}{\DR{x}{1}}{right=of wy0}
% \event{wy1}{\DW{y}{1}}{right=of rx1}
% \rf{wx1}{rx1}
% \rf{wy1}{ry1}
% \po{rx1}{wy1}
% \po{ry1}{wx1}
% \wk[bend left]{wx0}{wx1}
% \wk[bend right]{wy0}{wy1}
% \end{tikzinline}}
% \end{gather*}

% \begin{gather*}
% (y\GETS 0 \SEMI   x \GETS 1  \SEMI \aReg \GETS y)
% \PAR (x\GETS 0 \SEMI  y \GETS 1  \SEMI  \bReg \GETS x)
% \\
% \hbox{\begin{tikzinline}[node distance=1em and 2em]
% \event{wx0}{\DW{x}{0}}{}
% \event{wx1}{\DW{x}{1}}{right=of wx0}
% \event{ry0}{\DR{y}{0}}{right=of wx1}
% \event{wy0}{\DW{y}{0}}{below=of wx0}
% \event{wy1}{\DW{y}{1}}{right=of wy0}
% \event{rx0}{\DR{x}{0}}{right=of wy1}
% \rf[bend right]{wx0}{rx0}
% \rf[bend left]{wy0}{ry0}
% \wk{rx0}{wx1}
% \wk{ry0}{wy1}
% % \wk{wx0}{wx1}
% % \wk{wy0}{wy1}
% \end{tikzinline}}
% \qquad
% \hbox{\begin{tikzinline}[node distance=1em and 2em]
% \event{wx0}{\DW{x}{0}}{}
% \event{wx1}{\DW{x}{1}}{right=of wx0}
% \event{ry0}{\DR{y}{1}}{right=of wx1}
% \event{wy0}{\DW{y}{0}}{below=of wx0}
% \event{wy1}{\DW{y}{1}}{right=of wy0}
% \event{rx0}{\DR{x}{1}}{right=of wy1}
% \rf{wx1}{rx0}
% \rf{wy1}{ry0}
% \wk{wx0}{wx1}
% \wk{wy0}{wy1}
% \end{tikzinline}}
% \end{gather*}


% \begin{gather*}
% (y\GETS 0 \SEMI \aReg \GETS y \SEMI  x \GETS 1 )
% \PAR (x\GETS 0 \SEMI  \bReg \GETS x \SEMI  y \GETS \bReg + 1)
% \\
% \hbox{\begin{tikzinline}[node distance=1em and 2em]
% \event{wx0}{\DW{x}{0}}{}
% \event{ry0}{\DR{y}{0}}{right=of wx0}
% \event{wx1}{\DW{x}{1}}{right=of ry0}
% \event{wy0}{\DW{y}{0}}{below=of wx0}
% \event{rx0}{\DR{x}{0}}{right=of wy0}
% \event{wy1}{\DW{y}{1}}{right=of rx0}
% \po{rx0}{wy1}
% \rf{wx0}{rx0}
% \rf{wy0}{ry0}
% \wk{rx0}{wx1}
% \wk{ry0}{wy1}
% % \wk[bend left]{wx0}{wx1}
% % \wk[bend right]{wy0}{wy1}
% \end{tikzinline}}
% \qquad
% \hbox{\begin{tikzinline}[node distance=1em and 2em]
% \event{wx0}{\DW{x}{0}}{}
% \event{ry0}{\DR{y}{1}}{right=of wx0}
% \event{wx1}{\DW{x}{1}}{right=of ry0}
% \event{wy0}{\DW{y}{0}}{below=of wx0}
% \event{rx0}{\DR{x}{0}}{right=of wy0}
% \event{wy1}{\DW{y}{1}}{right=of rx0}
% \po{rx0}{wy1}
% \rf{wx0}{rx0}
% \rf{wy1}{ry0}
% \wk{rx0}{wx1}
% %\wk[bend left]{wx0}{wx1}
% \wk[bend right]{wy0}{wy1}
% \end{tikzinline}}
% \end{gather*}

% \begin{gather*}
% (y\GETS 0 \SEMI   x \GETS 1  \SEMI \aReg \GETS y)
% \PAR (x\GETS 0 \SEMI  \bReg \GETS x \SEMI  y \GETS \bReg + 1)
% \\
% \hbox{\begin{tikzinline}[node distance=1em and 2em]
% \event{wx0}{\DW{x}{0}}{}
% \event{wx1}{\DW{x}{1}}{right=of wx0}
% \event{ry0}{\DR{y}{0}}{right=of wx1}
% \event{wy0}{\DW{y}{0}}{below=of wx0}
% \event{rx0}{\DR{x}{0}}{right=of wy0}
% \event{wy1}{\DW{y}{1}}{right=of rx0}
% \po{rx0}{wy1}
% \rf{wx0}{rx0}
% \rf{wy0}{ry0}
% \wk{rx0}{wx1}
% \wk{ry0}{wy1}
% % \wk[bend left]{wx0}{wx1}
% % \wk[bend right]{wy0}{wy1}
% \end{tikzinline}}
% \qquad
% \hbox{\begin{tikzinline}[node distance=1em and 2em]
% \event{wx0}{\DW{x}{0}}{}
% \event{wx1}{\DW{x}{1}}{right=of wx0}
% \event{ry0}{\DR{y}{1}}{right=of wx1}
% \event{wy0}{\DW{y}{0}}{below=of wx0}
% \event{rx0}{\DR{x}{0}}{right=of wy0}
% \event{wy1}{\DW{y}{1}}{right=of rx0}
% \po{rx0}{wy1}
% \rf{wx0}{rx0}
% \rf{wy1}{ry0}
% \wk{rx0}{wx1}
% %\wk[bend left]{wx0}{wx1}
% \wk[bend right]{wy0}{wy1}
% \end{tikzinline}}
% \end{gather*}



The premises of the theorem allow us to avoid the complications caused by ``mixed races'' in
\cite{DBLP:conf/ppopp/DongolJR19}.  In the left pomset below, $\aPS''$ is not
an extension of $\aPS'$, since $\aPS'$ is not a downset of $\aPS''$.  
When considering this pomset, we must perform the decomposition on the right.
\begin{gather*}
  (x\GETS 0 \SEMI   x\REL \GETS 1)
  \PAR
  (\aReg\GETS x\ACQ)
  \\
  \hbox{\begin{tikzinline}[node distance=1em and 2em]
      \event{wx0}{\DW{x}{0}}{}
      \event{wx1}{\DWRel{x}{1}}{right=of wx0}
      \event{rx}{\DRAcq{x}{0}}{below=of wx0}
      \rf{wx0}{rx}
      \pox{wx0}{wx1}
      \wk{rx}{wx1}
      \node(ix)[left=of wx0]{};
      \bgoval[yellow!50]{(ix)}{P}
      \bgoval[pink!50]{(wx0)(wx1)}{P'\setminus P}
      \bgovalright[green!10]{(rx)}{P''\setminus P'}
    \end{tikzinline}}
  \qquad
  \qquad
  \hbox{\begin{tikzinline}[node distance=1em and 2em]
      \event{wx0}{\DW{x}{0}}{}
      \event{wx1}{\DWRel{x}{1}}{right=of wx0}
      \event{rx}{\DRAcq{x}{0}}{below=of wx0}
      \rf{wx0}{rx}
      \pox{wx0}{wx1}
      \wk{rx}{wx1}
      \node(ix)[left=of wx0]{};
      \bgoval[yellow!50]{(ix)}{P}
      \bgoval[pink!50]{(wx0)(rx)}{P'\setminus P}
      \bgoval[green!10]{(wx1)}{P''\setminus P'}
    \end{tikzinline}}
\end{gather*}
This affects the inductive order in which we move across pomsets, but does
not affect the set of pomsets that are considered.  This simplification is
enabled by denotational reasoning and a precise characterization of
dependency.

%Write swap (WW$\rightarrow$WW):

$L$-stability allows the theorem to encompass histories that are not entirely
$L$-sequential:
\begin{gather*}
  (\aReg\GETS y\SEMI x\REL\GETS1
  \PAR
  \aReg\GETS x\SEMI y\REL\GETS1)
  \PAR
  \big(\aReg\GETS x\ACQ \SEMI \aReg\GETS y\ACQ \SEMI
  (\aReg\GETS y\SEMI x\GETS2
  \PAR
  \aReg\GETS x\SEMI y\GETS2)\big)
\\
\hbox{\begin{tikzinline}[node distance=1em and 2em]
    \event{ry}{\DR{y}{1}}{}
    \event{rx}{\DR{x}{1}}{below=of ry}
    \event{wx}{\DWRel{x}{1}}{right=of ry}
    \event{wy}{\DWRel{y}{1}}{right=of rx}
    \rf{wy}{ry}
    \rf{wx}{rx}
    \pox{rx}[below]{wy}
    \pox{ry}{wx}
    \event{ax}{\DRAcq{x}{1}}{right=3em of wx}
    \event{ay}{\DRAcq{x}{1}}{below=of ax}
    \rf{wx}{ax}
    \rf{wy}{ay}
    \node(ix)[right=3em of ax]{$\mathstrut$};
    \node(iy)[below=of ix]{$\mathstrut$};
    \event{ry2}{\DR{y}{2}}{right=6em of ax}
    \event{rx2}{\DR{x}{2}}{below=of ry2}
    \event{wx2}{\DW{x}{2}}{right=of ry2}
    \event{wy2}{\DW{y}{2}}{right=of rx2}
    \sync{ax}{ry2}
    \sync{ax}{rx2}
    \sync{ay}{ry2}
    \sync{ay}{rx2}
    \rf{wy2}{ry2}
    \rf{wx2}{rx2}
    \pox{rx2}[below]{wy2}
    \pox{ry2}{wx2}
    \bgoval[yellow!50]{(ry)(rx)(ax)(ay)}{P}
    \bgoval[pink!50]{(ix)(iy)}{P'\setminus P}
    \bgoval[green!10]{(ry2)(rx2)(wx2)(wy2)}{P''\setminus P'}
\end{tikzinline}}
\end{gather*}
In this execution, the $L$-stable point $P$ is not $L$-sequential.
Nonetheless, the theorem guarantees that any races in $P''$ can be found
sequentially in some $P'''$.  Further, the theorem guarantees that the race
found in $P'''$ does not involve any actions in the stable past $P$.

\endinput

\subsection{Proof chat}
  \textcolor{red}{WIP...}

  
  For such programs, we define up with a procedure that inductively generates
  all the pomsets in $\semmin{\aCmd}$.  It works very much like an opsem, but
  on $\rppo$-order, rather than full program order.

  In a minimal pomset, cross-thread order is only introduced by the
  requirement of fulfillment.

  Here's the procedure.  We are building pomset $\aPS$, with set $\aPSS'$ of
  possible extensions.  
  
  \begin{itemize}
  \item Initially, take $\aPSS'$ to be all of the pomsets generated by
    semantics, without imposing fulfillment requirements (so they just have
    $\rppo$).    
  \item Initially take $\aPS$ to include the initial non-conflicting writes.
    These must be the same in all executions.
  \end{itemize}
  Procedure:
  \begin{itemize}
  \item Pick a set of extensions, each of which includes a single additional
    event\footnote{There is only one, since language is deterministic}.  We
    either one write extension $\aPS'$ or a set of read extensions
    $\{\aPS'_\aVal\mid\aVal\in\Val\}$\footnote{The semantics guarantees we have
      an extension for every $\aVal$.}.  For reads, each pomset adds an event
    reading the given value.  The syntactic marker on the new read event is
    the same across all events.
  \item Compute $\aPS''$. Update $\aPS$ to $\aPS''$ and repeat.
    \begin{itemize}
    \item For a write extension, $\aPS''$ augments $\aPS'$ with $\rfre$-order
      into the new event from any conflicting reads in $\aPS$.
    \item For a read extension $\{\aPS'_\aVal\mid\aVal\in\Val\}$, we pick a
      $\aPS'_\aVal$ that can be fulfilled.  $\aPS''$ augments $\aPS'_\aVal$
      with $\rrfe$-order into the new event from the fulfilling write.
      $\aPS''$ also augments $\aPS'_\aVal$ $\rcoe$-order if needed.
    \end{itemize}
  \end{itemize}
  Claims:
  \begin{itemize}
  \item You can always get to a termination event, no matter which choices
    you make.
  \item Every execution in $\semmin{\aCmd}$ can be generated this way.
  \end{itemize}

  By induction on this procedure, we can show that: If there is a cycle in
  $\pole{L}$, then we can always find an execution that includes all of the
  events in the cycle except one read, which has a new value; further the
  order in the new execution is preserved, except for edges that are adjacent
  to the read.
  \begin{itemize}
  \item The only way to introduce a cycle is to choose to put things in order
    that contradicts $\rpox$. 
  \end{itemize}
  




% Let $P$ be a prefix.  Let $P'$ be an extension of $P$. Then, there exists $P''$ :
% \begin{itemize}
% \item $P''$ extends $P$  
% \item $P''$ is a prefix of $P'$ 
% \item $P''\setminus P$ is $L$-sequential, and
% \end{itemize}
% such that one of the following holds
% \begin{itemize}
% \item 
%   $P'' = P'$ , or
% \item
%   $P''$ has a race involving an event in $P''\setminus P$, or
% \item
%   there is a read event $e$ in $P$ such that :
%   \begin{itemize}
%   \item $e$ is in a race with some event in $P''$ and
%   \item there is an $e'$, reading same location as $e$, such that:
%     $e \cup P''$ is $L$-sequential
%   \end{itemize}
% \end{itemize}
           