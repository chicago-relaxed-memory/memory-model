\section{Examples}
\label{sec:examples}

\subsection{Blockers}
\label{app:blockers}

Unfortunately by itself, this is not enough. The problem is the final
clause saying that there does not exist an $\aLoc$-\emph{blocking}
event $\cEv$ between $\bEv$ and $\aEv$. Unfortunately, concurrency can
turn events that were not $\aLoc$-blockers into an $\aLoc$-blocker,
\emph{even if the new thread does not mention $\aLoc$.}
We give an example to show this in \refapp{blockers}.
This is a problem in that it means the preliminary model violates
\emph{scope extrusion}~\cite{Milner:1999:CMS:329902},
in that we can find programs $\aCmd$ and $\bCmd$ such that
$\sem{\VAR\aLoc\SEMI(\aCmd\PAR\bCmd)}$ is not the same as
$\sem{(\VAR\aLoc\SEMI\aCmd)\PAR\bCmd}$, even if $\bCmd$ does not mention~$\aLoc$.

Recall the preliminary definition of reads-from in \S\ref{sec:pomsets}, which
defined an $\aLoc$-blocker to be and event $\cEv$ that writes to $\aLoc$ such that
$\bEv \lt \cEv \lt \aEv$.  Were we to adopt this definition, then concurrent
threads could turn events that were not $\aLoc$-blockers into an
$\aLoc$-blocker, even if the new thread does not mention $\aLoc$.

To see this, consider the program
\begin{math}
  (
  \aLoc\GETS1\SEMI
  \bLoc\GETS\aLoc
  \PAR
  \aLoc\GETS\cLoc+1\SEMI
  \bLoc\GETS\aLoc
  \PAR
  \IF{z=2}\THEN\aReg\GETS\aLoc\FI
  )
\end{math}
with execution:
\[\begin{tikzpicture}[node distance=1em]
  \event{wx1}{\DW{\aLoc}{1}}{}
  \event{rz1}{\DR{\cLoc}{1}}{right=of wx1}
  \event{wx2}{\DW{\aLoc}{2}}{right=of rz1}
  \event{rz2}{\DR{\cLoc}{2}}{right=of wx2}
  \event{rx1}{\DR{\aLoc}{1}}{right=of rz2}
  \event{rx1a}{\DR{\aLoc}{1}}{below=of wx1}
  \event{wy1}{\DW{\bLoc}{1}}{below=of rx1a}
  \event{rx2a}{\DR{\aLoc}{2}}{below=of wx2}
  \event{wy2}{\DW{\bLoc}{2}}{below=of rx2a}
  \rf{wx1}{rx1a}
  \po{rx1a}{wy1}
  \rf{wx2}{rx2a}
  \po{rx2a}{wy2}
  \po{rz1}{wx2}
  \po{rz2}{rx1}
  \rf[out=20,in=160]{wx1}{rx1}
\end{tikzpicture}\]
and the program
\begin{math}
  (
  \cLoc\GETS\bLoc\SEMI
  \cLoc\GETS\bLoc
  )
\end{math}
with execution:
\[\begin{tikzpicture}[node distance=1em]
  \event{ry1}{\DR{\bLoc}{1}}{}
  \event{wz1}{\DW{\cLoc}{1}}{right=of ry1}
  \event{ry2}{\DR{\bLoc}{2}}{right=of wz1}
  \event{wz2}{\DW{\cLoc}{2}}{right=of ry2}
  \po{ry1}{wz1}
  \po{ry2}{wz2}
\end{tikzpicture}\]
If these are placed in parallel, then a possible execution is:
\[\begin{tikzpicture}[node distance=1em]
  \event{wx1}{\DW{\aLoc}{1}}{}
  \event{rz1}{\DR{\cLoc}{1}}{right=of wx1}
  \event{wx2}{\DW{\aLoc}{2}}{right=of rz1}
  \event{rz2}{\DR{\cLoc}{2}}{right=of wx2}
  \event{rx1}{\DR{\aLoc}{1}}{right=of rz2}
  \event{rx1a}{\DR{\aLoc}{1}}{below=of wx1}
  \event{wy1}{\DW{\bLoc}{1}}{below=of rx1a}
  \event{rx2a}{\DR{\aLoc}{2}}{below=of wx2}
  \event{wy2}{\DW{\bLoc}{2}}{below=of rx2a}
  \rf{wx1}{rx1a}
  \po{rx1a}{wy1}
  \rf{wx2}{rx2a}
  \po{rx2a}{wy2}
  \po{rz1}{wx2}
  \po{rz2}{rx1}
  \event{ry1}{\DR{\bLoc}{1}}{below=of wy1}
  \event{wz1}{\DW{\cLoc}{1}}{right=of ry1}
  \event{ry2}{\DR{\bLoc}{2}}{below=of wy2}
  \event{wz2}{\DW{\cLoc}{2}}{right=of ry2}
  \po{ry1}{wz1}
  \po{ry2}{wz2}
  \rf{wy1}{ry1}
  \rf{wz1}{rz1}
  \rf{wy2}{ry2}
  \rf{wz2}{rz2}
\end{tikzpicture}\]
and now the $(\DW{\aLoc}{2})$ event is an $\aLoc$-blocker,
so $(\DR{\aLoc}{1})$ cannot
read from $(\DW{\aLoc}{1})$.

In the final definition of reads-from in \S\ref{sec:pomsets} we
ruled out $\aLoc$-blockers by requiring that any
event $\cEv$ that writes to $\aLoc$ has
either $\cEv \gtN \bEv$ or $\aEv \gtN \cEv$.
With this definition, in order for $(\DR{\aLoc}{1})$ to read from
$(\DW{\aLoc}{1})$, we either need $(\DW{\aLoc}{2}) \gtN (\DW{\aLoc}{1})$
or $(\DR{\aLoc}{1}) \gtN (\DW{\aLoc}{2})$, for example:
\[\begin{tikzpicture}[node distance=1em]
  \event{wx1}{\DW{\aLoc}{1}}{}
  \event{rz1}{\DR{\cLoc}{1}}{right=of wx1}
  \event{wx2}{\DW{\aLoc}{2}}{right=of rz1}
  \event{rz2}{\DR{\cLoc}{2}}{right=of wx2}
  \event{rx1}{\DR{\aLoc}{1}}{right=of rz2}
  \event{rx1a}{\DR{\aLoc}{1}}{below=of wx1}
  \event{wy1}{\DW{\bLoc}{1}}{below=of rx1a}
  \event{rx2a}{\DR{\aLoc}{2}}{below=of wx2}
  \event{wy2}{\DW{\bLoc}{2}}{below=of rx2a}
  \rf{wx1}{rx1a}
  \po{rx1a}{wy1}
  \rf{wx2}{rx2a}
  \po{rx2a}{wy2}
  \po{rz1}{wx2}
  \po{rz2}{rx1}
  \rf[out=20,in=160]{wx1}{rx1}
  \wk[out=-150,in=-30]{rx1}{wx2}
  \wk{wy1}{wy2}
\end{tikzpicture}\]
then putting this in parallel as before results in:
\[\begin{tikzpicture}[node distance=1em]
  \event{wx1}{\DW{\aLoc}{1}}{}
  \event{rz1}{\DR{\cLoc}{1}}{right=of wx1}
  \event{wx2}{\DW{\aLoc}{2}}{right=of rz1}
  \event{rz2}{\DR{\cLoc}{2}}{right=of wx2}
  \event{rx1}{\DR{\aLoc}{1}}{right=of rz2}
  \event{rx1a}{\DR{\aLoc}{1}}{below=of wx1}
  \event{wy1}{\DW{\bLoc}{1}}{below=of rx1a}
  \event{rx2a}{\DR{\aLoc}{2}}{below=of wx2}
  \event{wy2}{\DW{\bLoc}{2}}{below=of rx2a}
  \rf{wx1}{rx1a}
  \po{rx1a}{wy1}
  \rf{wx2}{rx2a}
  \po{rx2a}{wy2}
  \po{rz1}{wx2}
  \po{rz2}{rx1}
  \rf[out=20,in=160]{wx1}{rx1}
  \wk[out=-150,in=-30]{rx1}{wx2}
  \wk{wy1}{wy2}
  \event{ry1}{\DR{\bLoc}{1}}{below=of wy1}
  \event{wz1}{\DW{\cLoc}{1}}{right=of ry1}
  \event{ry2}{\DR{\bLoc}{2}}{below=of wy2}
  \event{wz2}{\DW{\cLoc}{2}}{right=of ry2}
  \po{ry1}{wz1}
  \po{ry2}{wz2}
  \rf{wy1}{ry1}
  \rf{wz1}{rz1}
  \rf{wy2}{ry2}
  \rf{wz2}{rz2}
  \wk[out=30,in=150]{wz1}{wz2}
\end{tikzpicture}\]
but this is \emph{not} a valid 3-valued pomset,
since $(\DW{\aLoc}{2}) \lt (\DR{\aLoc}{1})$ but also $(\DR{\aLoc}{1}) \gtN (\DW{\aLoc}{2})$,
which is a contradiction.


%\section{Understanding ``Out Of Thin Air'' using Temporal Logic}
\label{sec:logic}

A significant challenge for a software memory model is to relax order enough
to allow efficient implementation without admitting anomalous
behaviors---called \emph{out of thin air} (\oota) in the literature
\cite{vacuous,DBLP:conf/esop/BattyMNPS15,BoehmOOTA}.  The most famous example
is \ref{OOTA3} from \textsection\ref{sec:intro}.  Here we inline
initialization in order to fit the format of our proof rules:
\begin{align*}
  \label{OOTA3}\tag{\textsc{oota2}}
  y\GETS0\SEMI 
  y\GETS x
  \!\PAR\!
  x\GETS0\SEMI
  r\GETS y\SEMI x\GETS r  
  &&
  %\nonumber
  \smash{\hbox{\begin{tikzinline}[node distance=1.2em]
  \event{rx}{\DR{x}{1}}{}
  \event{wy}{\DW{y}{1}}{right=of rx}
  \po{rx}{wy}
  \event{y0}{\DW{y}{0}}{left=of rx}
  \event{x0}{\DW{x}{0}}{right=2em of wy}
  \event{ry}{\DR{y}{1}}{right=2em of x0}
  \event{wx}{\DW{x}{1}}{right=of ry}
  \po{ry}{wx}
  \rf[out=10,in=170]{wy}{ry}
  \rf[out=170,in=10]{wx}{rx}
  \wk[out=-15,in=-165]{y0}{wy}
  \wk[out=-15,in=-165]{x0}{wx}
    \end{tikzinline}}}
\end{align*}
Although Java does not allow \oota{} behaviors of \ref{OOTA3},
\citet{DBLP:journals/toplas/Lochbihler13} showed that it does allow \oota\
behaviors of \ref{OOTA1}, from \textsection\ref{sec:intro}.  In
\cite{DBLP:conf/lics/JeffreyR16}, we described a logic that rules out
\ref{OOTA3} but not \ref{OOTA1}.  In this section, we provide a more accurate
test of \oota{} behaviors by enhancing our previous logic with temporal features.

On first read, we suggest that readers skip to the examples and the
discussion that follows, coming back to the details of the logic as
necessary.  Example~\ref{ex:thin} discusses the canonical \oota{} example
\ref{OOTA3}; the analysis is trivial and well-known
\cite{DBLP:conf/lics/JeffreyR16, DBLP:conf/popl/KangHLVD17}.
Example~\ref{ex:lochb} is more interesting.  It discusses a variant of
\citeauthor{DBLP:journals/toplas/Lochbihler13}'s example \ref{OOTA1},
from the introduction.
% In this case, the violation is a subtler
% temporal property.  We develop a logic sufficient to prove that our semantics
% disallows \oota\ on \eqref{lochbihler}.  

The logic given here is not meant to be definitive; on page
\pageref{page:logic:limits}, we discuss \oota{} examples that require
non-trivial extensions
\cite{DBLP:conf/esop/SvendsenPDLV18,DBLP:journals/pacmpl/ChakrabortyV19}.

\noparagraph{Definitions}
We adapt past linear temporal logic (\pLTL)
\cite{Lichtenstein:1985:GP:648065.747612} to pomsets by dropping the previous
instant operator and adopting strict versions of the temporal operators.
The atoms of our logic are write and read events.
% \begin{displaymath}
%   \afo \QUAD::=\QUAD
%   \DR{\aLoc}{\aVal}
%   \mid
%   \DW\aLoc\aVal
%   \afo \wedge\bfo
%   \mid \lnot \afo
%   \once\afo
%   \mid \always\afo
% \end{displaymath}
%\begin{definition} %[Satisfaction]
Given a pomset $\aPS$ and event $\aEv$, define:\nofootnote{Let $\FALSE$, $\lor$,
  $\Rightarrow$ and $\once$ as usual;
  for example,
  $\once\afo = \lnot(\always\lnot\afo)$.}
\begin{displaymath}
  \renewcommand{\arraycolsep}{.2ex}
    \begin{array}{lrll}
      \aPS,\aEv &\models& \DW{\aLoc}{\aVal} &\text{ if } \labelingAct(\aEv) = \DW{\aLoc}{\aVal} \text{ and } \TRUE \text{ implies } \labelingForm(\aEv) \\
      \aPS,\aEv &\models& \DR{\aLoc}{\aVal} &\text{ if } \labelingAct(\aEv) = \DR{\aLoc}{\aVal} \text{ and } \TRUE \text{ implies } \labelingForm(\aEv) \\
      \aPS,\aEv &\models& \afo\land\bfo &\text{ if } \aPS,\aEv \models  \afo \text{ and } \aPS,\aEv \models  \bfo \\
      \aPS,\aEv &\models& \TRUE\\
      \aPS,\aEv &\models& \lnot\afo &\text{ if } \aPS,\aEv \not\models \afo \\
      \aPS,\aEv &\models& \always\afo &\text{ if } \forall \bEv \lt \aEv.\; \aPS,\bEv \models \afo\\
      \aPS,\aEv &\models& \once\afo &\text{ if } \exists \bEv \lt \aEv.\;  \aPS,\bEv \models \afo 
    \end{array} 
  \end{displaymath}

  Define $\FALSE$, $\lor$, and $\Rightarrow$ as usual.

  % \begin{definition}
  Let $\aPS \models \afo$ if
  $\aPS,\aEv \models\afo$, for all $\aEv \in \Event$.

  Let $\aPSS\models \afo$
  if $\aPS \models\afo$, for all $\aPS \in \aPSS$.
  
Let
  \begin{math}
    \afo, \aPSS \models \bfo  \text{ if } \{ \aPS \mid \aPS \models \afo \} \parallel \aPSS \models \bfo.
  \end{math}
%\end{definition}

  Let $\afo$ be \emph{downclosed} when
  $\{ \aPS \mid \aPS \models \afo \}$ is.

% Thus, $\aPS\models \afo \land \always\afo$ whenever $\aPS \models
  % \afo$. This fact relies on the use of universal quantification in the definition.

% We define other connectives as standard:
% $\once\afo = \lnot(\always\lnot\afo)$,
% %$\FALSE = \lnot(\TRUE)$
% $\afo\lor\bfo = \lnot(\lnot(\afo)\land\lnot(\bfo))$, and
% $\afo\Rightarrow\bfo = \lnot(\afo) \lor\ \bfo$.
% \begin{displaymath}
% \begin{array}{lrl}
% \once\afo &=& \lnot(\always\lnot\afo) \\
% \FALSE &=& \lnot(\TRUE) \\
% \afo\lor\bfo &=& \lnot(\lnot(\afo)\land\lnot(\bfo)) \\
% \afo\Rightarrow\bfo &=& \lnot(\afo) \lor\ \bfo
% \end{array}
% \end{displaymath}
%Let $$ be defined as $$. 
%In addition, let $\FALSE$, $\lor$ and $$ be defined in the
%standard way.
% $\afo\lor\bfo$ for $\lnot(\lnot \afo \land \lnot \bfo)$,
% and $\afo \Rightarrow \bfo$ for $\lnot \afo \lor \bfo$.
  The past operators do not include the current instant, and so
  do \emph{not} satisfy
  $(\always\afo\Rightarrow\once\afo)$. The order-minimal elements always validate
    $\always\afo$ and invalidate
    $\once\afo$.
  However, we can prove the following:
% \begin{align*}  
%   \frac{\aPS \models \afo \Rightarrow\once{\afo}}{\aPS \models \lnot \afo}\text{(Coinduction)}
%   &&
%   \frac{\aPS \models \always\afo \Rightarrow\afo}{\aPS \models \afo}\text{(Induction)}
% \end{align*}
% \begin{lemma}
% Given an pomset $\aPS$.  
\begin{align*}
  \tag{Induction}
  \aPS \models& (\always\afo \Rightarrow\afo) \Rightarrow\afo
  \\[-1ex]
  \tag{Coinduction}
  \aPS \models& (\afo \Rightarrow\once{\afo}) \Rightarrow\lnot \afo
  \\[-1ex]
  \tag{Weakening}
  \aPS \models& (\afo \Rightarrow\once{\bfo}) \Rightarrow (\once\afo \Rightarrow\once{\bfo})
\end{align*}
% \end{lemma}
% \begin{proof}
% We prove that any node in a pomset satisfies these formulas.  
%The proof for both rules proceeds by induction on the length of the maximal path from a root to a node. 
%\end{proof}

% \begin{description}
% \item[Coinduction.]
%   \begin{math}
%     (\afo \Rightarrow\once{\afo}) \Rightarrow\lnot \afo
%   \end{math}
% \item[Induction.] 
%   \begin{math}
%     (\always\afo \Rightarrow\afo) \Rightarrow\afo
%   \end{math}
% \end{description}


%We now present two proof rules for programs. 

%\paragraph*{Proof rules for programs}
We present two additional proof rules. 
The first provides a logical view of \emph{$\aLoc$-closure} (Def.~\ref{def:rf}):
%The soundness proof is straightforward.
% \begin{math}
%   \closed(\aLoc) = (\DR{\aLoc}{\aVal} \Rightarrow \once \DW{\aLoc}{\aVal}).
% \end{math}
% Although this definition does not mention intervening writes, it is
% sufficient for our example.  
\begin{displaymath}
  %\tag{Closing $\aLoc$}
  \frac{
    \afo \text{ is independent of } \aLoc
    \qquad
    %\aPS \models \closed(\aLoc) \Rightarrow \afo
    \aPS \models (\DR{\aLoc}{\aVal} \Rightarrow \once \DW{\aLoc}{\aVal}) \Rightarrow \afo
  }{
    \nu \aLoc \DOT \aPS \models \afo
  }
\end{displaymath}
%It is straightforward to establish that this rule is sound.
% Although it does
% not mention intervening writes, the rule is sufficient for our examples.

The second rule describes concurrent composition, in the style of~\citet{Abadi:1993:CS:151646.151649}.  To simplify the presentation, we
consider the special case with a single invariant.
% We view the
% composition result as capturing key aspects of no-ThinAirRead, as will become
% clearer in the examples below.
% In order to state the theorem, we generalize the satisfaction relation to
% include environment assumptions.

\begin{proposition}%[Composition]
  Let $\afo$ be downclosed.  Let $\aPSS_1, \aPSS_2$ be
  augmentation\hyp{}closed. %\footnote{$\aPS'$ is an augmentation of $\aPS$ if
 %   $\Event'=\Event$, $\aEv\le\bEv$ implies $\aEv\le'\bEv$, $\aEv\gtN\bEv$
 %   implies $\aEv\gtN'\bEv$, and
 %   % $\labeling'(\aEv)=\labeling(\aEv)$
 %   if $\labeling(\aEv) = (\bForm \mid \bAct)$ then
 %   $\labeling'(\aEv) = (\bForm' \mid \bAct)$ where $\bForm'$ implies
 %   $\bForm$.}
  Then:
  \begin{displaymath}
    %\tag{Composition}
    \frac{
      \afo, \aPSS_1 \models\afo
      \qquad
      \afo, \aPSS_2 \models\afo
    }{\aPSS_1 \parallel \aPSS_2 \models \afo}
  \end{displaymath}
\end{proposition}
\begin{proof}[Proof sketch]
  We will show that all downsets in the downset closures of
  $\aPSS_1 \parallel \aPSS_2$ satisfy the required property.  Proof proceeds
  by induction on downsets of $\aPS \in \aPSS_1 \parallel \aPSS_2$.
  %
  The case for empty downset  follows from assumption that  $\afo$ is downset closed.  
  %
  For the inductive case, consider %$\aPS$ in the downset closure of $\aPSS_1 \parallel \aPSS_2$, i.e.
  $\aPS \in \aPS_1 \parallel \aPS_2$ where
  $\aPS_i \in \aPSS_i$.  Since $\aPSS_1$ and $\aPSS_2$ are augmentation
  closed, we can assume that the restriction of $\aPS$ to the events of
  $\aPS_i$ coincides with $\aPS_i$, for $i=1,2$.
  %
  Consider a downset $\aPS'$ derived by removing a maximal element $\aEv$ from
  $\aPS$.  Suppose $\aEv$ comes from $\aPS_1$ (the other case is
  symmetric). Since $\aPS_2$ is a downset of $\aPS'$ and $\aPS' \models \afo$
  by induction hypothesis, we deduce that $\aPS_2 \models \afo$.
  % Thus, $\aPS_2 \in \mods{(\afo)}$.
  Since $\aPS_1 \in \aPSS_1$, by assumption $\afo, \aPSS_1 \models\afo$ we
  deduce that $\aPS \models \afo$.
\end{proof}

% The logic is defined with respect to downclosed sets, but $\sem{\aCmd}$
% includes only completed pomsets.  For reasoning in the logic, we downclose
% the semantics, considering pomsets that may not be completed.  Let
% $\semdown{\aCmd}=\{\aPS'\mid\aPS'$ is a downset of some
% $\aPS \in \sem{\aCmd}\}$.

\begin{example}
\label{ex:thin}
\noparagraph{Basic Examples}
With all variables initialized to $0$, we show that \ref{OOTA3}
satisfies
\begin{math}
  \lnot\DW{x}{1}.
\end{math}

We start with the invariant:
\begin{displaymath}
  [\DW{x}{1}\Rightarrow\once\DR{y}{1}]
  \land
  [\DW{y}{1}\Rightarrow\once\DR{x}{1}]
\end{displaymath}
This invariant holds for each thread; thus, it holds for the
aggregate program by composition.  Closing $y$ yields
\begin{math}
  \DR{y}{1} \Rightarrow \once\DW{y}{1}.
\end{math}
Weakening the right conjunct: % yields
\begin{math}
  \once\DW{y}{1}\Rightarrow\once\DR{x}{1}.
\end{math}
Chaining these together: %yields
\begin{math}
  \DR{y}{1} \Rightarrow \once\DR{x}{1}.
\end{math}
Weakening:  %yields
\begin{math}
  \once\DR{y}{1} \allowbreak\Rightarrow \once\DR{x}{1}. 
\end{math}
Chaining into the left conjunct:  %yields
\begin{math}
  \DW{x}{1} \Rightarrow \once\DR{x}{1}. 
\end{math}
Closing $x$, 
% \begin{math}
%   \DR{x}{1} \Rightarrow \once\DW{x}{1}.
% \end{math}
weakening, 
% \begin{math}
%   \once\DR{x}{1} \Rightarrow \once\DW{x}{1}.
% \end{math}
then chaining: %, yields
\begin{math}
  \DW{x}{1} \Rightarrow \once\DW{x}{1}. 
\end{math}
By coinduction, 
\begin{math}
  \lnot\DW{x}{1}.
\end{math}
%as required.
\end{example}

The same reasoning can be applied to the control flow variant of \ref{OOTA3}
\cite[CYC]{DBLP:conf/popl/VafeiadisBCMN15}:
\begin{math}
  %\tag{\textsc{cyc}}\label{CYC}
  \IF{x}\THEN y\GETS 1 \FI \!\PAR\! \IF{y}\THEN x\GETS 1 \FI.
  % &&
  % %\nonumber
  % \hbox{\begin{tikzinline}[node distance=1.5em]
  % \event{rx}{\DR{x}{1}}{}
  % \event{wy}{\DW{y}{1}}{right=of rx}
  % \po{rx}{wy}
  % \event{ry}{\DR{y}{1}}{right=2em of wy}
  % \event{wx}{\DW{x}{1}}{right=of ry}
  % \po{ry}{wx}
  % \rf{wy}{ry}
  % \rf[out=170,in=10]{wx}{rx}
  %   \end{tikzinline}}
\end{math}
The program is data-race-free. Thus, allowing an execution that writes $1$
would violate \drfsc{}.

\begin{example}
  \label{ex:lochb}
  \noparagraph{Lochbihler's Example} The essential temporal property of
  \ref{OOTA1} is ``allocation at type \texttt{C} is preceded by a read of $1$
  from $z$.''  Because our language lacks object creation, we cannot consider
  \ref{OOTA1} directly.  Instead we study \ref{OOTA4}, which has the same
  temporal structure: ``a write of $1$ to location $a$ is preceded by a read
  of $1$ from $z$.''
% A more general principle, in the spirit of~\citet{Abadi:1993:CS:151646.151649} can be proved.  We chose the simple case of temporal invariants to illustrate the idea in a simple form.  Even this simple version has interesting consequences. 
\begin{gather*}
  \tag{\textsc{oota4}}\label{OOTA4}
  %   Z=1;
  % ||
  %   a=X; // 1
  %   Y=a;
  % ||
  %   b=Z; // 0
  %   if(b){
  %     X=1
  %   } else {
  %     c=Y; // 1
  %     X=c;
  %     W=c;
  %   }
  %     \VAR  x\GETS0\SEMI \VAR  y\GETS0\SEMI \VAR  z\GETS0\SEMI
  \begin{gathered}
  z\GETS0\SEMI
    z\GETS1
  \PAR
  y\GETS0\SEMI
    y\GETS x
  \PAR
  x\GETS0\SEMI
  a\GETS0\SEMI
  \IF{z}\THEN x\GETS1 \ELSE r\GETS y \SEMI x\GETS r \SEMI a\GETS r \FI
  \\
\hbox{\begin{tikzinline}[node distance=1.2em]
  \event{wz0}{\DW{z}{0}}{}
  \event{wz1}{\DW{z}{1}}{right=of wz0}
  \wk{wz0}{wz1}
  \event{wy0}{\DW{y}{0}}{right=2em of wz1}
  \event{rx}{\DR{x}{1}}{right=of wy0}
  \event{wy}{\DW{y}{1}}{right=of rx}
  \po{rx}{wy}
  \event{wx0}{\DW{x}{0}}{right=2em of wy}
  \event{wa0}{\DW{a}{0}}{right=of wx0}
  \event{rz}{\DR{z}{0}}{right=of wa0}
  \event{ry}{\DR{y}{1}}{right=of rz}
  \event{wx}{\DW{x}{1}}{right=of ry}
  \event{wa}{\DW{a}{1}}{right=of wx}
  \po{ry}{wx}
  \rf[out=15,in=165]{wy}{ry}
  \rf[out=-170,in=-10]{wx}{rx}
  \po[out=-18,in=-162]{rz}{wa}
  \po[out=25,in=155]{ry}{wa}
  \wk[out=25,in=155]{wy0}{wy}
  \wk[out=15,in=165]{wx0}{wx}
  \wk[out=15,in=165]{wa0}{wa}
  \rf[out=-10,in=-170]{wz0}{rz}
\end{tikzinline}}
% \hbox{\begin{tikzinline}[node distance=1.5em]
%   \event{rx}{\DR{x}{1}}{}
%   \event{wz1}{\DW{z}{1}}{left=3em of rx}
%   \event{wy}{\DW{y}{1}}{right=of rx}
%   \po{rx}{wy}
%   \event{rz}{\DR{z}{0}}{right=3em of wy}
%   \event{ry}{\DR{y}{1}}{right=of rz}
%   \event{wx}{\DW{x}{1}}{right=of ry}
%   \event{wa}{\DW{a}{1}}{right=of wx}
%   \po{ry}{wx}
%   \rf[out=15,in=165]{wy}{ry}
%   \rf[out=-170,in=-10]{wx}{rx}
%   \po[out=-10,in=-170]{rz}{wa}
%   \po[out=15,in=165]{ry}{wa}
% \end{tikzinline}}
  \end{gathered}  
  \end{gather*}
  Attempting to write $1$ to $a$ results in the cycle shown above.  Thus, the
  outcome is disallowed by our semantics.  It was also disallowed by our
  event structures model, although the logic given there is insufficient to
  establish this fact \citep[\textsection9]{DBLP:journals/lmcs/JeffreyR19}.
  The outcome is \emph{allowed} by
  \citep{DBLP:conf/popl/KangHLVD17,DBLP:conf/esop/JagadeesanPR10,DBLP:journals/pacmpl/ChakrabortyV19}.
% \begin{tikzdisplay}[node distance=1.5em]
%   \event{wy0}{\DW{y}{0}}{}
%   \event{rx}{\DR{x}{1}}{right=4.5em of wy0}
%   \event{wy}{\DW{y}{1}}{right=of rx}
%   \po{rx}{wy}
%   \wk[bend left]{wy0}{wy}
%   \event{wx0}{\DW{x}{0}}{below=of wy0}
%   \event{rz}{\DR{z}{0}}{right=of wx0}
%   \event{ry}{\DR{y}{1}}{right=of rz}
%   \event{wx}{\DW{x}{1}}{right=of ry}
%   \event{ry1}{\DR{y}{1}}{right=of wx}
%   \event{wa}{\DW{a}{1}}{right=of ry1}
%   \rf{wy}{ry1}
%   \po{ry}{wx}
%   \wk[bend right]{wx0}{wx}
%   \rf{wy}{ry}
%   \rf{wx}{rx}
%   \event{wz0}{\DW{z}{0}}{below=of wx0}
%   \event{wz1}{\DW{z}{1}}{right=of wz0}
%   \rf{wz0}{rz}
%   \wk{wz0}{wz1}
%   \po{ry1}{wa}
%   \po[bend right]{rz}{wa}
% \end{tikzdisplay}

To establish that this outcome is disallowed here, we prove 
\begin{math}
  \lnot\DW{a}{1},
\end{math}
starting with invariant:
% which holds for each of the three threads, and thus, by composition, for the
% aggregate program:
\begin{scope}
\small
\begin{align*}
  [\once\DW{y}{1} \Rightarrow \once\DR{x}{1}]
  \land
  [\notonce\DW{a}{1} \Rightarrow (\once\DR{y}{1} \land \always(\DW{x}{1} \Rightarrow \once\DR{y}{1}))]
\end{align*}
\end{scope}
Closing $y$ and chaining into the left conjunct:
% \begin{math}
%   \once\DR{y}{1} \Rightarrow \once\DW{y}{1}. % \Rightarrow \once\DR{x}{1}
% \end{math}
% Chaining this implication on the left:
\begin{math}
  \once\DR{y}{1} \Rightarrow \once\DR{x}{1}.
\end{math}
% We can weaken this to:
% \begin{math}
%   \once\DR{y}{1} \Rightarrow \once\DR{x}{1}. % \Rightarrow \once\DR{x}{1}
% \end{math}
Chaining into the right conjunct:
\begin{displaymath}
  \notonce\DW{a}{1} \Rightarrow (\once\DR{x}{1} \land \always(\DW{x}{1} \Rightarrow \once\DR{x}{1}))
\end{displaymath}
Closing $x$:
% \begin{math}
%   \once\DR{x}{1} \Rightarrow \once\DW{x}{1}.
% \end{math}
%  Weakening and chaining again:
%we can replace $\once\DR{x}{1}$ with $\once\DW{x}{1}$:
\begin{math}
  \notonce\DW{a}{1} \Rightarrow (\once\DW{x}{1} \land \always(\DW{x}{1} \Rightarrow \once\DW{x}{1}).
\end{math}
Applying coinduction to the right conjunct:
\begin{displaymath}
  \notonce\DW{a}{1} \Rightarrow (\once\DW{x}{1} \land \always(\lnot \DW{x}{1}))
\end{displaymath}
Simplifying:
\begin{math}
  \notonce\DW{a}{1} \Rightarrow \FALSE,
\end{math}
as required.
\end{example}


\begin{comment}
  \color{red} Need to sort this out.
  Alan proposes:
\begin{verbatim}
     (W y 2) => <>(R x 1)
     (W y 1) => <>(R x 0)
     (W x 1) => <>(R y 1)
   <>(W x 1) => not(<>(W x 2))  --- which should be???  <>(W x 0) => not(<>(W x 1))
\end{verbatim}

2020/09/30: This seems to go bad because of initialization...
The formula
\begin{verbatim}
<>Wx0 => not(<>Wx1)
\end{verbatim}
does not hold for
\begin{verbatim}
x=0; x=y
\end{verbatim}

2020/09/10:  I am worried about the compositionality of this predicate:
\begin{verbatim}
I think
   <>(W x 0 => not(<>(W x 1)))
holds for 
   x=0; r=y 
and
   x=1
but not
   x=0; r=y || x=1
as shown by the execution
   Wx1 < Wx0 < Ry0
\end{verbatim}
  
It is impossible to fulfill $(\DR{y}{1})$ in the following
\cite[RNG]{DBLP:conf/esop/SvendsenPDLV18}:
\begin{align*}
  \taglabel{OOTA5}
    ( y\GETS x+1
    \PAR
    x\GETS y ) && \hbox{\begin{tikzinline}[node distance=1.5em]
        \event{rx}{\DR{x}{1}}{}
        \event{wy}{\DW{y}{2}}{right=of rx}
        \po{rx}{wy}
        \event{ry}{\DR{y}{1}}{right=3em of wy}
        \event{wx}{\DW{x}{1}}{right=of ry}
        \po{ry}{wx}
        \rf[out=170,in=10]{wx}{rx}
      \end{tikzinline}}
\end{align*}
The proof proceeds as before, starting with the following invariant:
\begin{gather*}
  [\DW{y}{2} \Rightarrow \once\DR{x}{1}] \land
  [\once\DW{x}{1} \Rightarrow \once\DR{y}{1}] \land
  [\once\DW{y}{1} \Rightarrow \once\DR{x}{0}] \land
  [\once\DW{x}{0} \Rightarrow \lnot(\once\DW{x}{1})]
\end{gather*}
\begin{verbatim}
  Wy2 => <>Rx1  /\  <>Wx1 => <>Ry1  /\  <>Wy1 => <>Rx0  
close x and y                                          
  Wy2 => <>Wx1  /\  <>Wx1 => <>Wy1  /\  <>Wy1 => <>Wx0  
chain
  Wy1 => <>Wx0  
chain with <>Wx0 => not(<>Wx1)
\end{verbatim}
\end{comment}

% Many examples are superficially similar, but in fact have fewer dependencies.
% A referee for a previous version of this paper expected that the following example is
% ``the same'':
% \begin{gather*}
%   \tag{OOTA?}\label{OOTA?}
%     y\GETS x
%   \PAR
%     \IF{y}\THEN r\GETS y\SEMI x\GETS r\SEMI a\GETS r \ELSE x\GETS1 \FI
%   \\
%   \hbox{\begin{tikzinline}[node distance=1.5em]
%   \event{rx}{\DR{x}{1}}{}
%   \event{wy}{\DW{y}{1}}{right=of rx}
%   \po{rx}{wy}
%   \event{ry}{\DR{y}{1}}{right=2em of wy}
%   \event{wx}{\DW{x}{1}}{right=of ry}
%   \event{wa}{\DW{a}{1}}{right=of wx}
%   \rf[out=-15,in=-165]{wy}{ry}
%   \rf[out=170,in=10]{wx}{rx}
%   \po[out=-15,in=-165]{ry}{wa}
%     \end{tikzinline}}
% \end{gather*}
% In this execution, $\DW{x}{1}$ is independent of $\DR{y}{1}$, thus there is no
% \oota{} behavior.

Many examples are superficially similar, but in fact have fewer dependencies,
such as \eqref{OOTA?} from \textsection\ref{sec:intro}.

\noparagraph{RFUB: Register assignment From an Unexecuted Branch}
\citeauthor{BoehmOOTA}'s [\citeyear{BoehmOOTA}] \ref{RFUB} example presents
another potential form of \oota{} behavior.
% , in the context of compiler
% optimization.
Our analysis shows that there is no \oota{} behavior in
\ref{RFUB}, only a false dependency:
%\citet{BoehmOOTA} \labeltext{considers}{page:rfub} the following programs:
\begin{gather*}
  \tag{\textsc{rfub}}\label{RFUB}
  \sem{r\GETS y\SEMI x\GETS r}
  \not\supseteq
  \sem{r\GETS y\SEMI \IF{r \NOTEQ 1} \THEN z\GETS 1\SEMI r\GETS 1\FI \SEMI x\GETS r}
\end{gather*}
The left command is half of \ref{OOTA3}. %, from \textsection\ref{sec:logic}.
The right command is dubbed \rfub{}, for \emph{Register assignment From an
  Unexecuted Branch}.  \citeauthor{BoehmOOTA} observes that in the context
$x\GETS y \PAR \hole{}$, these programs have different behaviors.  Yet the
\oota{} example on the left never writes $1$.  Why should the unexecuted
branch change that?  Because of the conditional, the write to $x$ in
\ref{RFUB} is independent of the read from $y$.  It useful to considering the
Hoare logic formulas satisfied by the two threads above: we have
$\hoare{\TRUE}{\text{\rfub}}{x=1}$, but not
$\hoare{\TRUE}{\text{\oota}}{x=1}$.  The change in the thread from
\ref{OOTA3} to \ref{RFUB} is not a valid refinement under Hoare logic; as a
result, it is expected that \ref{RFUB} may have additional behaviors.

Understanding \oota{} behavior is notoriously difficult, even for the
greatest minds in the field!  % We believe that \emph{logic} is the only tool
% that can cut the horrible knot that semanticists have tied themselves in.
% Preconditions provide a \emph{natural} solution to working out these
% dependencies.
This example shows the wisdom of using existing tools, such as preconditions
and Hoare logic, to model new problems, such as relaxed memory.
% We don't
% need to abandon established ideas; we only need to adapt them!
% On page \pageref{page:rfub}, we discuss \citeauthor{BoehmOOTA}'s
% [\citeyear{BoehmOOTA}] \ref{RFUB} example, which presents another potential
% form of \oota{} behavior, in the context of compiler optimization.  Our
% analysis shows that there is no \oota{} behavior in \ref{RFUB}, instead
% \citeauthor{BoehmOOTA}'s analysis has a false dependency.

% Understanding \oota{} behavior is notoriously difficult, even for the
% greatest minds in the field!  We believe that \emph{logic} is the only tool
% that can cut the horrible knot that semanticists have tied themselves in.
% Preconditions provide a \emph{natural} solution to working out these
% dependencies.

% \endinput

% \paragraph{Load buffering and thin air.}
% The program
% \begin{math}
%   %x\GETS0\SEMI y\GETS0\SEMI
%   (y\GETS x \PAR \bReg\GETS y\SEMI x\GETS1)
% \end{math}
% has top level executions that result in the final outcome $x = y = 1$, such as:
% \begin{tikzdisplay}[node distance=1.5em]
%   % \event{wx0}{\DW{x}{0}}{}
%   % \event{wy0}{\DW{y}{0}}{below=wx0}
%   \event{rx}{\DR{x}{1}}{}
%   \event{wy}{\DW{y}{1}}{right=of rx}
%   \po{rx}{wy}
%   \event{ry}{\DR{y}{1}}{right=3em of wy}
%   \event{wx}{\DW{x}{1}}{right=of ry}
%   \rf{wy}{ry}
%   \rf[out=170,in=10]{wx}{rx}
%   %\po{rx}{wy}
% \end{tikzdisplay}
% In \textsection\ref{sec:logic} we provide machinery to prove that this
% outcome is impossible if there is order from read to write in both
% threads.  This order can be achieved by replacing the second thread
% \begin{math}
%   (\bReg\GETS y\SEMI x\GETS1)
% \end{math}
% with 
% \begin{math}
%   (\bReg\GETS y\ACQ\SEMI x\GETS1)
% \end{math}
% or
% \begin{math}
%   (\IF{y}\THEN x\GETS 1\FI)
% \end{math}
% or
% \begin{math}
%   (x\GETS y).
% \end{math}

% A more interesting example is the following variant of \eqref{types}:
% \begin{displaymath}
%   %\label{OOTA4}
%   % x\GETS0\SEMI
%   %y\GETS0\SEMI   
%   (
%     y\GETS x
%   \PAR
%     \IF{z}\THEN x\GETS1 \ELSE x\GETS y\SEMI a\GETS y \FI
%   \PAR
%     z\GETS0\SEMI z\GETS1
%   )
% \end{displaymath}
% This program is allowed to write $1$ to $a$ under many speculative
% memory models
% \cite{Manson:2005:JMM:1047659.1040336,DBLP:conf/esop/JagadeesanPR10,DBLP:conf/popl/KangHLVD17},
% even though the read of $1$ from $y$ in the else branch of the second
% thread arises out of thin air.   \citet{DBLP:journals/toplas/Lochbihler13}
% argues that such executions compromise type safety unless object allocation
% partitions memory by type.
% In our model, the attempted execution is:
% \begin{tikzdisplay}[node distance=1.5em]
%   \event{rx}{\DR{x}{1}}{}
%   \event{wy}{\DW{y}{1}}{below=of rx}
%   \po{rx}{wy}
%   \event{ry}{\DR{y}{1}}{right=of rx}
%   \event{wx}{\DW{x}{1}}{below=of ry}
%   \po{ry}{wx}
%   \rf{wy}{ry}
%   \rf{wx}{rx}
%   \event{rz}{\DR{z}{0}}{right=of ry}
%   \event{wz0}{\DW{z}{0}}{right=of rz}
%   \rf{wz0}{rz}
%   \event{wz1}{\DW{z}{1}}{right=of wz0}
%   \wk{wz0}{wz1}
%   \event{ry1}{\DR{y}{1}}{below=of rz}
%   \rf[bend right]{wy}{ry1}
%   \event{wa}{\DW{a}{1}}{right=of ry1}
%   \po{ry1}{wa}
%   \po{rz}{wa}
% \end{tikzdisplay}
% This is forbidden by the evident cycle.


% \begin{verbatim}



% y=x+1; a=y || x=y
% prove a!=2

% Wyv_1 /\ Wyv_2 => v_1 == v_2 (and maybe v_1==0 \/ v_2==0)
% Wx1 => <>-1 Ry1
% Wy1 => <>-1 Rx1
% \end{verbatim}

% Local Variables:
% mode: latex
% TeX-master: "paper"
% End:



\subsection{Release/acquire synchronization}
\label{app:ra}

% In relaxed memory models, synchronization actions act as memory fences: that
% is, they are a barrier to reordering memory accesses.  In this section, we
% present a simple model of release/acquire fencing. In
% \S\ref{sec:transactions}, we show that this can be scaled up to a model of
% transactional memory.

% We assume there are sets $\Rel$ and $\Acq \subseteq\Act$.  We say that
% $\aAct$ is a \emph{release action} if $\aAct\in\Rel$ and $\aAct$ is an
% \emph{acquire action} if $\aAct\in\Acq$.
% In a pomset, a release event is one labeled with a release action,
% and an acquire event is one labeled by an acquire action.
% To give the semantics of fences, we add extra constraints
% to Definition~\ref{def:prefix} of prefixing %$\aAct\prefix\aPSS$
% (recalling that $\cEv$ is the %$\aAct$-labeled
% event being introduced):
% \begin{itemize}
% \item $\cEv \le \aEv$ whenever $\cEv$ is an acquire event or $\aEv$ is a release event, and
% \item if $\cEv$ is an acquire event then $\aEv$ is independent of $\aLoc$,
%   for every $\aLoc$.
% \end{itemize}
% The first constraint ensures that events are ordered before a release and
% after an acquire.  The second constraint ensures that thread-local reads do
% not cross acquire fences.

% We can develop a simple model of release/acquire synchronization using the
% following actions: % we will use
% % releasing writes and acquiring reads:
% \begin{itemize}
% \item $(\DWRel{\aLoc}{\aVal})$, a release action that writes $\aVal$ to $\aLoc$, and
% \item $(\DRAcq{\aLoc}{\aVal})$, an acquire action that reads $\aVal$ from $\aLoc$.
% \end{itemize}
% The semantics of programs with releasing write and acquiring read are similar
% to regular write and read, with $\DWRel\aLoc\aVal$ replacing
% $\DW\aLoc\aVal$ and $\DRAcq\aLoc\aVal$ replacing $\DR\aLoc\aVal$:
% \begin{eqnarray*}
%   \sem{\REL\aLoc\GETS\aExp\SEMI \aCmd} & = & \textstyle\bigcup_\aVal\; \bigl((\aExp=\aVal) \guard (\DWRel\aLoc\aVal) \prefix \sem{\aCmd}[\aExp/\aLoc]\bigr) \\
%   \sem{\ACQ\aReg\GETS\aLoc\SEMI \aCmd} & = & \textstyle\bigcup_\aVal\; (\DRAcq\aLoc\aVal) \prefix \sem{\aCmd}[\aLoc/\aReg]
% \end{eqnarray*}

To see the need for the first constraint on prefixing, consider the program:
\[
  \VAR x\GETS0\SEMI \VAR f\GETS0\SEMI
  (x\GETS 1\SEMI \REL f\GETS1 \PAR \ACQ r\GETS f; s\GETS x)
\]
This has an execution:
\[\begin{tikzpicture}[node distance=1em]
  \event{wx0}{\DW{x}{0}}{}
  \event{wf0}{\DW{f}{0}}{right=of wx0}
  \event{wx1}{\DW{x}{1}}{below=of wx0}
  \event{wf1}{\DWRel{f}{1}}{right=of wx1}
  \event{rf1}{\DRAcq{f}{1}}{right=2.5em of wf1}
  \event{rx1}{\DR{x}{1}}{right=of rf1}
  \po{wx0}{wf1}
  \po{wf0}{wf1}
  \po{wx1}{wf1}
  \po{rf1}{rx1}
  \rf{wf1}{rf1}
  \rf[out=20,in=160]{wx1}{rx1}
  \wk{wx0}{wx1}
\end{tikzpicture}\]
but \emph{not}:
\[\begin{tikzpicture}[node distance=1em]
  \event{wx0}{\DW{x}{0}}{}
  \event{wf0}{\DW{f}{0}}{right=of wx0}
  \event{wx1}{\DW{x}{1}}{below=of wx0}
  \event{wf1}{\DWRel{f}{1}}{right=of wx1}
  \event{rf1}{\DRAcq{f}{1}}{right=2.5em of wf1}
  \event{rx0}{\DR{x}{0}}{right=of rf1}
  \po{wx0}{wf1}
  \po{wf0}{wf1}
  \po{wx1}{wf1}
  \po{rf1}{rx1}
  \rf{wf1}{rf1}
  \rf[out=-20,in=160]{wx0}{rx0}
  \wk{wx0}{wx1}
\end{tikzpicture}\]
since $(\DW x0) \gtN (\DW x1) \lt (\DR x0)$, so this pomset does not satisfy the
requirements to be $x$-closed.
If we replace the release
with a plain write, then the outcome $(\DRAcq f1)$ and $(\DR x0)$ is possible:
\[\begin{tikzpicture}[node distance=1em]
  \event{wx0}{\DW{x}{0}}{}
  \event{wf0}{\DW{f}{0}}{right=of wx0}
  \event{wx1}{\DW{x}{1}}{below=of wx0}
  \event{wf1}{\DW{f}{1}}{right=of wx1}
  \event{rf1}{\DRAcq{f}{1}}{right=2.5em of wf1}
  \event{rx0}{\DR{x}{0}}{right=of rf1}
  \wk{wf0}{wf1}
  \po{rf1}{rx0}
  \rf{wf1}{rf1}
  \rf[out=-20,in=160]{wx0}{rx0}
  \wk{wx0}{wx1}
\end{tikzpicture}\]
since no order is required between $(\DW x1)$ and $(\DW f1)$.  
Symmetrically, if we replace the acquire of the original program
with a plain read, then the outcome $(\DR f1)$ and $(\DR x0)$ is possible.
% \begin{verbatim}
%   x := 0; rel f := 0; ||
%   acq r := f; if (r == 0) { x := x+1; rel f := 1; } ||
%   acq s := f; if (r == 1) { x := x+1; rel f := 2; }
% \end{verbatim}
% This has an execution:
% \[\begin{tikzpicture}[node distance=1em]
%   \event{wx0}{\DW{x}{0}}{}
%   \event{wf0}{\DWRel{f}{0}}{below=of wx0}
%   \event{rf0}{\DRAcq{f}{0}}{right=2.5 em of wx0}
%   \event{rx0}{\DR{x}{0}}{below=of rf0}
%   \event{wx1}{\DW{x}{1}}{below=of rx0}
%   \event{wf1}{\DWRel{f}{1}}{below=of wx1}
%   \event{rf1}{\DRAcq{f}{1}}{right=2.5 em of rf0}
%   \event{rx1}{\DR{x}{1}}{below=of rf1}
%   \event{wx2}{\DW{x}{2}}{below=of rx1}
%   \event{wf2}{\DWRel{f}{2}}{below=of wx2}
%   \po{wx0}{wf0}
%   \po{rf0}{rx0}
%   \po{rx0}{wx1}
%   \po{wx1}{wf1}
%   \po{rf1}{rx1}
%   \po{rx1}{wx2}
%   \po{wx2}{wf2}
%   \rf{wf0}{rf0}
%   \rf{wx0}{rx0}
%   \rf{wf1}{rf1}
%   \rf{wx1}{rx1}
% \end{tikzpicture}\]
% but \emph{not}:
% \[\begin{tikzpicture}[node distance=1em]
%   \event{wx0}{\DW{x}{0}}{}
%   \event{wf0}{\DWRel{f}{0}}{below=of wx0}
%   \event{rf0}{\DRAcq{f}{0}}{right=2.5 em of wx0}
%   \event{rx0}{\DR{x}{0}}{below=of rf0}
%   \event{wx1}{\DW{x}{1}}{below=of rx0}
%   \event{wf1}{\DWRel{f}{1}}{below=of wx1}
%   \event{rf1}{\DRAcq{f}{1}}{right=2.5 em of rf0}
%   \event{rx0b}{\DR{x}{0}}{below=of rf1}
%   \event{wx1b}{\DW{x}{1}}{below=of rx0b}
%   \event{wf2}{\DWRel{f}{2}}{below=of wx1b}
%   \po{wx0}{wf0}
%   \po{rf0}{rx0}
%   \po{rx0}{wx1}
%   \po{wx1}{wf1}
%   \po{rf1}{rx0b}
%   \po{rx0b}{wx1b}
%   \po{wx1b}{wf2}
%   \rf{wf0}{rf0}
%   \rf{wx0}{rx0}
%   \rf{wf1}{rf1}
%   \rf{wx0}{rx0b}
% \end{tikzpicture}\]
% since $(\DW x0) \lt (\DW x1) \lt (\DR x0)$, so this pomset does not satisfy the
% requirements to be an rf-pomset.

% The notion rf-pomset is sufficient to capture hardware models and
% release/acquire access in C++, where reads-from implies happens-before
% \cite{alglave}.  To model C++ relaxed access, it
% would be necessary to use a more general notion of rf-pomset, where
% $(\bEv,\aLoc,\aEv) \in \RF$ does not necessarily imply $\bEv \lt \aEv$, instead
% requiring that $(\mathord\lt \cup \mathord\RF)$ be acyclic.

%% To see the need for the second constraint on prefixing, consider the program:
%% \begin{displaymath}
%%   (
%%   x\GETS1\SEMI
%%   \REL f\GETS 1\SEMI
%%   \ACQ r\GETS f\SEMI
%%   y\GETS x
%%   )
%%   \PAR
%%   (
%%   \ACQ s\GETS f\SEMI
%%   x\GETS2\SEMI
%%   \REL f\GETS 2\SEMI
%%   )
%% \end{displaymath}
%% whose semantics includes execution:
%% \begin{displaymath}
%% \begin{tikzpicture}[node distance=1em]
%%   \event{wx1}{\DW{x}{1}}{}
%%   \event{wf1}{\DWRel{f}{1}}{right=of wx1}
%%   \event{rf1}{\DRAcq{f}{2}}{below=of wf1}
%%   \event{wx2}{\DW{x}{2}}{right=of rf1}
%%   \event{wf2}{\DWRel{f}{1}}{right=of wx2}
%%   \event{rf2}{\DRAcq{f}{2}}{above=of wf2}
%%   \event{wy1}{\DW{y}{1}}{right=of rf2}
%%   \po{wx1}{wf1}
%%   \rf{wf1}{rf1}
%%   \po{rf1}{wx2}
%%   \po{wx2}{wf2}
%%   \rf{wf2}{rf2}
%%   \po{rf2}{wy1}
%% \end{tikzpicture}
%% \end{displaymath}
%% This execution exists because
%% \begin{math}
%%   \sem{y\GETS x}
%% \end{math}
%% includes
%% \begin{math}
%%   (x=1\mid \DW{y}{1})
%% \end{math}
%% and the precondition $x=1$ is fulfilled by the preceding write $x\GETS1$.  In
%% implementation term, this execution is reading $1$ from $x$ in a ``stale
%% cache.''  The alternative execution that attempts to read $1$ from the $x$ in
%% ``main memory,'' has an explicit $(\DR{x}{1})$ between $(\DRAcq{f}{2})$ and
%% $(\DW{y}{1})$, and thus will fail to be $x$-closed.

%% To prevent thread-local writes from crossing release/acquire pairs, we
%% require that pomsets in the semantics of acquire have no free locations.
%% This corresponds to the idea that acquires flush the read cache, and
%% therefore reads must reload values from main memory after an acquire.

% In addition, we must change the semantics of write from
% \S\ref{sec:sets-of-pomsets} to ensure that an action is generated for every
% write that might be published by a subsequent release action.
% Formally, $\sem{\aLoc\GETS\aExp\SEMI \aCmd}$ only includes pomsets
% from $\sem{\aCmd}[\aExp/\aLoc]$ that contain a write to
% $\aLoc$ that is not preceded by a release.


\subsection{Relaxed memory}
\label{sec:relaxed-memory}

% In \S\ref{sec:info-flow-attack} we presented an information flow attack
% on relaxed memory, similar to Spectre in that it relies on speculative
% evaluation. Unlike Spectre it does not depend on timing attacks,
% but instead is based on the sensitivity of relaxed memory to data
% dependencies. % For this reason, we present a simple model of relaxed
% % memory, which is strong enough to capture this attack.

Our model includes concurrent memory accesses, which can introduce concurrent
reads-from. 
Since we are allowing events to be partially ordered, this gives a simple
model of relaxed memory.  For example an independent read independent write
(IRIW) example is:
\[\begin{array}{l}
  x\GETS0\SEMI x\GETS x+1
  \PAR
  y\GETS0\SEMI y\GETS y+1
\\{}
  \PAR
  r_1\GETS x\SEMI r_2\GETS y
  \PAR
  s_1\GETS y\SEMI s_2\GETS x
\end{array}\]
which includes the execution:
\[\begin{tikzpicture}[node distance=1em]
  \event{wx0}{\DW{x}{0}}{}
  \event{wx1}{\DW{x}{1}}{right=of wx0}
  \event{wy0}{\DW{y}{0}}{right=2.5em of wx1}
  \event{wy1}{\DW{y}{1}}{right=of wy0}
  \event{ry1}{\DR{y}{1}}{below=4ex of wx0}
  \event{rx0}{\DR{x}{0}}{right=of ry1}
  \event{rx1}{\DR{x}{1}}{right=2.5 em of rx0}
  \event{ry0}{\DR{y}{0}}{right=of rx1}
  \wk{wx0}{wx1}
  \wk{wy0}{wy1}
  \rf{wx1}{rx1}
  \rf{wy0}{ry0}
  \rf[out=210,in=30]{wy1}{ry1}
  \rf{wx0}{rx0}
  \wk{rx0}{wx1}
  \wk{ry0}{wy1}
\end{tikzpicture}\]
This model does not introduce thin-air reads (TAR).
For example the TAR pit
\((
  x\GETS y \PAR y \GETS x
)\)
fails to produce a value for $x$ from thin air
since this produces a cycle in $\le$, as shown on the left below:
\begin{align*}
\begin{tikzpicture}[node distance=1em]
  \event{ry42}{\DR{y}{42}}{}
  \event{wx42}{\DW{x}{42}}{below=of ry42}
  \event{rx42}{\DR{x}{42}}{right=2.5em of ry42}
  \event{wy42}{\DW{y}{42}}{below=of rx42}
  \po{ry42}{wx42}
  \po{rx42}{wy42}
  \rf{wx42}{rx42}
  \rf{wy42}{ry42}
\end{tikzpicture}
&&
\begin{tikzpicture}[node distance=1em]
  \event{ry1}{\DR{y}{1}}{}
  \event{wx1}{\DW{x}{1}}{below=of ry1}
  \event{rx1}{\DR{x}{1}}{right=2.5em of ry1}
  \event{wy1}{\DW{y}{1}}{below=of rx1}
  \po{ry1}{wx1}
  \rf{wx1}{rx1}
  \rf{wy1}{ry1}
\end{tikzpicture}
\end{align*}
This cycle can be broken by removing a dependency. For example
\((
  x\GETS y \PAR r\GETS x\SEMI y \GETS r+1-r
)\)
has the execution on the right above.
% \[\begin{tikzpicture}[node distance=1em]
%   \event{ry1}{\DR{y}{1}}{}
%   \event{wx1}{\DW{x}{1}}{below=of ry1}
%   \event{rx1}{\DR{x}{1}}{right=2.5em of ry1}
%   \event{wy1}{\DW{y}{1}}{below=of rx1}
%   \po{ry1}{wx1}
%   \rf{wx1}{rx1}
%   \rf{wy1}{ry1}
% \end{tikzpicture}\]
Note that $(\DR x1) \not\le (\DW y1)$, so this does not introduce a cycle.

Although it is not the primary focus of this paper, our model may be an
attractive model of relaxed memory.  Many prior models either permit
thin-air executions that our model forbids or forbid desirable executions
that our model permits.
%% In \S\ref{sec:logic}, we develop a logic which allows us to prove that our
%% semantics forbids thin air examples that are permitted by prior speculative
%% models
%% \cite{Manson:2005:JMM:1047659.1040336,Jagadeesan:2010:GOS:2175486.2175503,DBLP:conf/popl/KangHLVD17}.
% Our model passes all of the causality test cases
% \cite{PughWebsite}.
%% Significantly, this
%% includes test case 9, which is forbidden by \cite{DBLP:conf/lics/JeffreyR16},
%% one of the few models that disallows the thin air example from
%% \S\ref{sec:logic}.  We present this test case in the appendix, where we also
%% discuss the thread inlining examples from
%% \cite{Manson:2005:JMM:1047659.1040336}.

% In \refapp{logic}, we present a variant of the TAR-pit
% example %from \S\ref{sec:relaxed-memory}
% that is allowed under prior speculative semantics
% \cite{Manson:2005:JMM:1047659.1040336,Jagadeesan:2010:GOS:2175486.2175503,DBLP:conf/popl/KangHLVD17}.
% We develop a logic that allows us to prove that the problematic execution is
% forbidden in our model.  \citet{DBLP:conf/esop/BattyMNPS15} showed that the
% thin-air problem has no per-candidate-execution solution for C++.  This
% result does not apply to our model, which has a different notion of
% dependency.

% as the semantics of a conditional can depend on the semantics
% of both branches.

\citet{PughWebsite} developed a set of twenty {causality test cases} in the
process of revising the Java Memory Model (JMM)
\cite{Manson:2005:JMM:1047659.1040336}.  Using hand calculation, we have
confirmed that our model gives the desired result for all twenty cases,
unrolling loops as necessary.  Our model also gives the desired results for
all of the examples in \citet[\textsection 4]{DBLP:conf/esop/BattyMNPS15} and
all but one in \citet[\textsection 5.3]{SevcikThesis}:
redundant-write-after-read-elimination fails for any
sensible non-coherent semantics.  Our model agrees with the JMM on the
``surprising and controversial behaviors'' of \citet[\textsection
8]{Manson:2005:JMM:1047659.1040336}, and thus fails to validate thread
inlining.
% In \refapp{tc}, we discuss three of the causality test cases and the thread
% inlining example from \cite{Manson:2005:JMM:1047659.1040336}.%  In presenting the
% examples, we unroll loops, correct typos and simplify the code.  

% \subsection{Causality test cases}
% \label{app:tc}

% \citet{PughWebsite} developed a set of twenty {causality test cases} in the
% process of revising the Java Memory Model (JMM)
% \cite{Manson:2005:JMM:1047659.1040336}.  Using hand calculation, we have
% confirmed that our model gives the desired result for all twenty cases,
% unrolling loops as necessary.  Our model also gives the desired results for
% all of the examples in \citet[\textsection 4]{DBLP:conf/esop/BattyMNPS15} and
% all but one in \citet[\textsection 5.3]{SevcikThesis}:
% redundant-write-after-read-elimination fails for any
% sensible non-coherent semantics.  Our model agrees with the JMM on the
% ``surprising and controversial behaviors'' of \citet[\textsection
% 8]{Manson:2005:JMM:1047659.1040336}, and thus fails to validate thread
% inlining.

\subsection{Causality test cases}
\label{app:tc}

\citet{PughWebsite} developed a set of twenty {causality test cases} in the
process of revising the Java Memory Model (JMM)
\cite{Manson:2005:JMM:1047659.1040336}.  Using hand calculation, we have
confirmed that our model gives the desired result for all twenty cases,
unrolling loops as necessary.  Our model also gives the desired results for
all of the examples in \citet[\textsection 4]{DBLP:conf/esop/BattyMNPS15} and
all but one in \citet[\textsection 5.3]{SevcikThesis}:
redundant-write-after-read-elimination fails for any
sensible non-coherent semantics.  Our model agrees with the JMM on the
``surprising and controversial behaviors'' of \citet[\textsection
8]{Manson:2005:JMM:1047659.1040336}, and thus fails to validate thread
inlining.

In this section, we discuss three of the causality test cases and the thread
inlining from \cite{Manson:2005:JMM:1047659.1040336}.  In presenting the
examples, we unroll loops, correct typos and simplify the code.  

\subsubsection{Causality test case 8}

Test case 8 asks whether:
\begin{displaymath}
  \VAR x\GETS 0\SEMI
  \VAR y\GETS 0\SEMI
  (\IF{x<2}\THEN y\GETS 1\FI 
  \PAR
  x\GETS y)
\end{displaymath}
may read $1$ for both $x$ and $y$.  This behavior is allowed, since
``interthread analysis could determine that $x$ and $y$ are always either $0$
or $1$.''  This breaks the dependency between the read of $x$ and the write
to $y$ in the first thread, allowing the write to be moved earlier.

The semantics of TC8 includes
\[\begin{tikzpicture}[node distance=1em]
  \event{ix}{\DW{x}{0}}{}
  \event{iy}{\DW{y}{0}}{right=of ix}
  \event{rx1}{\DR{x}{1}}{right=2.1em of iy}
  \event{wy1}{\DW{y}{1}}{right=of rx1}
  \event{ry1}{\DR{y}{1}}{right=2.1em of wy1}
  \event{wx1}{\DW{x}{1}}{right=of ry1}
  \po{ry1}{wx1}
  \po[out=30,in=150]{ix}{rx1}
  \rf[in=-25,out=-160]{wx1}{rx1}
  \rf[out=20,in=160]{wy1}{ry1}
  \wk[out=-25,in=-150]{ix}{wx1}
  \wk[out=25,in=155]{iy}{wy1}
\end{tikzpicture}\]
Where we require $(\DW{x}{0})\lt(\DR{x}{1})$ but not $(\DR{x}{1})\lt(\DW{y}{1})$.
To see why this execution exists, consider the left thread with syntax sugar
removed:
\begin{displaymath}
  r\GETS x\SEMI \IF{r<2}\THEN y\GETS 1\FI
\end{displaymath}
\begin{math}
  \sem{\IF{r<2}\THEN y\GETS 1\FI}
\end{math}
includes
\begin{math}
  (r<2\mid\DW{y}{1}).
\end{math}
% \[\begin{tikzpicture}[node distance=1em]
%   \event{wy1}{r<2\mid\DW{y}{1}}{}
% \end{tikzpicture}\]
Thus, by Figure~\ref{fig:programs}, 
\begin{math}
  \sem{r\GETS x\SEMI \IF{r<2}\THEN y\GETS 1\FI}
\end{math}
includes
\begin{math}
  (\DR{x}{1}) \prefix (r<2\mid\DW{y}{1})[x/r]
\end{math}
which simplifies to
\begin{math}
  (\DR{x}{1}) \prefix (x<2\mid\DW{y}{1}),
\end{math}
which, by Definition~\ref{def:prefix}, includes:
\[\begin{tikzpicture}[node distance=1em,baselinecenter]
    \event{rx1}{\DR{x}{1}}{}
    \event{wy1}{x<2\mid\DW{y}{1}}{right=of rx1}
  \end{tikzpicture}\]
Here we have used the \textsc{[non-ordering read]} clause of Definition~\ref{def:prefix}:
``$\bForm'$ implies $\bForm[\aVal/\aLoc] \land \bForm$, if $\aAct$ reads $\aVal$ from $\aLoc$,''
where $a=(\DR{x}{1})$,  $\bForm=\bForm'=(x<2)$.  We can use this case since
$x<2$ implies $1<2\land x<2$.

Prefixing with $(\DW{x}{0})$ allows us to discharge the assumption $x<2$,
arriving at:
\[\begin{tikzpicture}[node distance=1em,baselinecenter]
    \event{ix}{\DW{x}{0}}{}
    \event{rx1}{\DR{x}{1}}{right=2.5 em of ix}
    \event{wy1}{\DW{y}{1}}{right=of rx1}
    \po{ix}{rx1}
  \end{tikzpicture}\]
Here we have used the \textsc{[ordering read]}
clause of \ref{def:prefix}:
``$\bForm'$ implies $\bForm[\aVal/\aLoc]$, if $\aAct$ reads $\aVal$ from $\aLoc$ and $\cEv\lt'\aEv$,''
where $a=(\DW{x}{0})$,  $\bForm=(x<2)$ and $\bForm'=\TRUE$.  As long as
require
\begin{math}
  (\DW{x}{0})\lt
  (\DR{x}{1}),
\end{math}
we can use this case since $\TRUE$ implies $0<2$.

\subsubsection{Causality test case 9}

Test case 9 asks whether:
\begin{displaymath}
  \VAR x\GETS 0\SEMI
  \VAR y\GETS 0\SEMI
  (\IF{x<2}\THEN y\GETS 1\FI 
  \PAR
  x\GETS y
  \PAR
  y\GETS 2\SEMI)
\end{displaymath}
may read $1$ for both $x$ and $y$.  This behavior is also allowed.  This is
``similar to test case $8$, except that $x$ is not always $0$ or
$1$. However, a compiler might determine that the read of $x$ by thread $1$
will never see the write by thread $3$ (perhaps because thread $3$ will be
scheduled after thread $1$)''

Reasoning as for test case 8, the semantics of test case 9 includes:
\[\begin{tikzpicture}[node distance=1em]
  \event{ix}{\DW{x}{0}}{}
  \event{iy}{\DW{y}{0}}{right=of ix}
  \event{rx1}{\DR{x}{1}}{right=2.2 em of iy}
  \event{wy1}{\DW{y}{1}}{right=of rx1}
  \event{ry1}{\DR{y}{1}}{right=2.2em of wy1}
  \event{wx1}{\DW{x}{1}}{right=of ry1}
  \event{wx2}{\DW{x}{2}}{below=3ex of $(ix)!0.5!(iy)$}%{right=2.5em of wx1}
  \po{ry1}{wx1}
  \po[out=30,in=150]{ix}{rx1}
  \rf[in=-25,out=-160]{wx1}{rx1}
  \rf[out=20,in=160]{wy1}{ry1}
  \wk[out=-25,in=-150]{ix}{wx1}
  \wk[out=25,in=155]{iy}{wy1}
  \wk{ix}{wx2}
  % \wk[out=-25,in=-150]{ix}{wx2}
\end{tikzpicture}\]

Thus, with respect to the introduction of new threads, our model appears to
be more robust than the event structures semantics of
\cite{DBLP:conf/lics/JeffreyR16}, which fails on this test case.

\subsubsection{Causality test case 14}

Test case 14 asks whether:
\begin{multline*}
  \VAR a\GETS 0\SEMI
  \VAR b\GETS 0\SEMI
  \VAR y\GETS 0\SEMI\\[-.5ex]
  (\IF{a}\THEN b\GETS 1\ELSE y\GETS 1\FI 
  \PAR\\[-.5ex]
  \WHILE(y+b==0) \THEN\SKIP\FI\; a\GETS1)
\end{multline*}
may read $1$ for $a$ and $b$, yet $0$ for $y$.  Here $a$ and $b$ are regular
variables and $y$ is volatile, which is equivalent to release/acquire in this
example.  This behavior is also disallowed, since ``in all sequentially
consistent executions, [the read of $a$ gets $0$] and the program is
correctly synchronized. Since the program is correctly synchronized in all SC
executions, no non-SC behaviors are allowed.''

Unrolling the loop once, we have:
\begin{multline*}
  \VAR a\GETS 0\SEMI
  \VAR b\GETS 0\SEMI
  \VAR y\GETS 0\SEMI\\[-.5ex]
  (\IF{a}\THEN b\GETS 1\ELSE y\GETS 1\FI 
  \PAR\\[-.5ex]
  \IF{y\lor b}\THEN a\GETS 1\FI)
\end{multline*}
We argue that any execution with $(\DR{a}{1})$, $(\DR{b}{1})$, and
$(\DR{y}{0})$ must be cyclic.  The closure requirements require that
\begin{math}
  (\DW{a}{1})\lt(\DR{a}{1})
  \;\text{and}\;
  (\DR{b}{1})\lt(\DR{b}{1}).
\end{math}
Ignoring initialization, least ordered execution that includes all of these
actions is:
\[\begin{tikzpicture}[node distance=1em]
  \event{ra1}{\DR{a}{1}}{}
  \event{wb1}{\DW{b}{1}}{below=of ra1}
  \nonevent{wy1}{\DW{y}{1}}{left=of wb1}
  \event{rb1}{\DR{b}{1}}{right=4.5em of ra1}
  \event{ry0}{\DR{y}{0}}{right=of rb1}
  \event{wa1}{\DW{a}{1}}{below=of rb1}
  \po{ra1}{wb1}
  \po{rb1}{wa1}
  \rf{wa1}{ra1}
  \rf{wb1}{rb1}
\end{tikzpicture}\]
where the read of $a$ is ordering for $(\DW{b}{1})$ but
not $(\DW{y}{1})$, and the read of $b$ is ordering for $(\DW{a}{1})$ but the
read of $y$ is not.  $(\DW{y}{1})$ is crossed out, since its
precondition must imply $(\lnot a)[1/a]$, which is equivalent to $\FALSE$.
To avoid order from $(\DR{y}{0})$ to $(\DW{a}{1})$, we
have strengthened the predicate on $(\DW{a}{1})$ from $(y\lor b)$ to
$(y=0\land b=1)$.  Note that we cannot use this trick symmetrically to remove
the order from $(\DR{b}{1})$ to $(\DW{a}{1})$, since $b=1$ does not follow
from the initialization of $b$.


\subsubsection{Thread inlining}

One property one could ask of a model of shared memory is thread
inlining: any execution of $\sem{P\SEMI Q}$ is an execution of $\sem{P
  \PAR Q}$. This is \emph{not} a goal of our model, and indeed is not
satisfied, due to the different semantics of concurrent and sequential
memory accesses. We demonstrate this by considering an example from
the Java Memory Model~\cite{Manson:2005:JMM:1047659.1040336}, which shows that Java does not
satisfy thread inlining either.

The lack of thread inlining is related to the different dependency
relations introduced by sequential and concurrent access.
Recall from \S\ref{sec:sequential-memory} that the program
\verb`(x := 0; y := x+1;)` has execution:
\[\begin{tikzpicture}[node distance=1em]
  \event{wx0}{\DW{x}{0}}{}
  \event{wy1}{\DW{y}{1}}{right=of wx0}
\end{tikzpicture}\]
but that \verb`(x := 1; || y := x+1;)` has:
\[\begin{tikzpicture}[node distance=1em]
  \event{wx1}{\DW{x}{1}}{}
  \event{rx1}{\DR{x}{1}}{right=2.5em of wx1}
  \event{wy2}{\DW{y}{2}}{right=of rx1}
  \rf{wx1}{rx1}
  \po{rx1}{wy2}
\end{tikzpicture}\]
That is, in the sequential case there is no dependency from the
write of $x$ to the write of $y$, but in the concurrent case there
is such a dependency.

This can be used to construct a counter-example to thread inlining, based on~\cite[Ex~11]{Manson:2005:JMM:1047659.1040336}:
\begin{verbatim}
  x := 0; if (x == 1) { z := 1; } else { x := 1; } || y := x; || x := y;
\end{verbatim}
This has no execution containing $(\DW z1)$. Any attempt to build such an execution
results in a cycle:
\[\begin{tikzpicture}[node distance=1em]
  \event{rx1a}{\DR{x}{1}}{}
  \event{wz1}{\DW{z}{1}}{right=of rx1a}
  \nonevent{wx1a}{\DW{x}{1}}{right=of wz1}
  \event{rx1b}{\DR{x}{1}}{below=of wx1a}%{right=2.5em of wx1a}
  \event{wy1}{\DW{y}{1}}{right=of rx1b}
  \event{ry1}{\DR{y}{1}}{right=2.5em of wy1}
  \event{wx1b}{\DW{x}{1}}{right=of ry1}
  \po{rx1a}{wz1}
  \po[out=25, in=150]{rx1a}{wx1a}
  \po{rx1b}{wy1}
  \po{ry1}{wx1b}
  \rf{wy1}{ry1}
  \rf[out=160, in=30]{wx1b}{rx1a}
  \rf[out=160, in=30]{wx1b}{rx1b}
\end{tikzpicture}\]
Inlining the thread \verb|(y := x)| gives~\cite[Ex~12]{Manson:2005:JMM:1047659.1040336}:
\begin{verbatim}
  x := 0; if (x == 1) { z := 1; } else { x := 1; } y := x; || x := y;
\end{verbatim}
with execution:
\[\begin{tikzpicture}[node distance=1em]
  \event{rx1a}{\DR{x}{1}}{}
  \event{wz1}{\DW{z}{1}}{right=of rx1a}
  \nonevent{wx1a}{\DW{x}{1}}{right=of wz1}
  \event{wy1}{\DW{y}{1}}{right=of wx1a}
  \event{ry1}{\DR{y}{1}}{right=2.5em of wy1}
  \event{wx1b}{\DW{x}{1}}{right=of ry1}
  \po{rx1a}{wz1}
  \po[out=25, in=150]{rx1a}{wx1a}
  \po{ry1}{wx1b}
  \rf{wy1}{ry1}
  \rf[out=160, in=30]{wx1b}{rx1a}
\end{tikzpicture}\]
To see why this execution exists, consider the program fragment:
\begin{verbatim}
  if (x == 1) { z := 1; } else { x := 1; } y := x;
\end{verbatim}
Removing the syntax sugar, this is:
\begin{verbatim}
  r1 := x; if (r1 == 1) {
    z := 1; r2 := x; y := r2; skip
  } else {
    x := 1; r3 := x; y := r3; skip
  }
\end{verbatim}
Now, $\sem{z := 1\SEMI r_2 := x\SEMI y := r_2\SEMI \SKIP}$
includes pomset:
\[\begin{tikzpicture}[node distance=1em]
  \event{wz1}{r_1=1 \mid \DW{z}{1}}{}
  \event{wy1}{r_1=x=1 \mid \DW{y}{1}}{right=of wz1}
\end{tikzpicture}\]
and $\sem{x := 1\SEMI r_3 := x\SEMI y := r_3\SEMI \SKIP}$
includes pomset:
\[\begin{tikzpicture}[node distance=1em]
  \event{wx1a}{r_1\neq 1 \mid \DW{x}{1}}{}
  \event{wy1}{r_1\neq 1 \mid \DW{y}{1}}{right=of wx1a}
\end{tikzpicture}\]
so  $\sem{\IF{r_1 = 1} \THEN z := 1\SEMI r_2 := x\SEMI y := r_2\SEMI \SKIP \ELSE x := 1\SEMI r_3 := x\SEMI y := r_3\SEMI \SKIP \FI}$ includes:
\[\begin{tikzpicture}[node distance=1em]
  \event{wz1}{r_1=1 \mid \DW{z}{1}}{}
  \event{wx1a}{r_1\neq1 \mid \DW{x}{1}}{right=of wz1}
  \event{wy1}{(r_1=x=1) \lor (r_1\neq1) \mid \DW{y}{1}}{below=3ex of $(wz1)!0.5!(wx1a)$}
\end{tikzpicture}\]
which means $\sem{\IF{r_1 = 1} \THEN z := 1\SEMI r_2 := x\SEMI y := r_2\SEMI \SKIP \ELSE x := 1\SEMI r_3 := x\SEMI y := r_3\SEMI \SKIP \FI}[x/r_1]$ includes:
\[\begin{tikzpicture}[node distance=1em]
  \event{wz1}{x=1 \mid \DW{z}{1}}{}
  \event{wx1a}{x\neq1 \mid \DW{x}{1}}{right=of wz1}
  \event{wy1}{(x=x=1) \lor (x\neq1)) \mid \DW{y}{1}}{below=3ex of $(wz1)!0.5!(wx1a)$}%{right=of wx1a}
\end{tikzpicture}\]
Now $(x=x=1) \lor (x\neq1)$ is a tautology, so this is just:
\[\begin{tikzpicture}[node distance=1em]
  \event{wz1}{x=1 \mid \DW{z}{1}}{}
  \event{wx1a}{x\neq1 \mid \DW{x}{1}}{right=of wz1}
  \event{wy1}{\DW{y}{1}}{right=of wx1a}
\end{tikzpicture}\]
and so $\sem{r_1 \GETS x\SEMI \IF{r_1 = 1} \THEN z := 1\SEMI r_2 := x\SEMI y := r_2\SEMI \SKIP \ELSE x := 1\SEMI r_3 := x\SEMI y := r_3\SEMI \SKIP \FI}$ includes:
\[\begin{tikzpicture}[node distance=1em]
  \event{rx1a}{\DR{x}{1}}{}
  \event{wz1}{1=1 \mid \DW{z}{1}}{right=of rx1a}
  \event{wx1a}{1\neq1 \mid \DW{x}{1}}{right=of wz1}
  \event{wy1}{\DW{y}{1}}{right=of wx1a}
  \po{rx1a}{wz1}
  \po[out=25, in=150]{rx1a}{wx1a}
\end{tikzpicture}\]
which simplifies to:
\[\begin{tikzpicture}[node distance=1em]
  \event{rx1a}{\DR{x}{1}}{}
  \event{wz1}{\DW{z}{1}}{right=of rx1a}
  \nonevent{wx1a}{\DW{x}{1}}{right=of wz1}
  \event{wy1}{\DW{y}{1}}{right=of wx1a}
  \po{rx1a}{wz1}
  \po[out=25, in=150]{rx1a}{wx1a}
\end{tikzpicture}\]
as required. The rest of the example is straightforward, and shows that our semantics
agrees with the JMM in not supporting thread inlining.



% \subsection{Word tearing}

% \todo{Remove this section, since it's not needed for transactions?}

% In \S\ref{sec:transactions}, we shall be considering transactional memory,
% and in \S\ref{sec:transactions} show that we can model a simplified version
% of an information flow attack on transactions. In order to model transactions,
% we need to consider actions that can write many memory locations at once,
% since this is part of the semantics of commitment. To lead up to this, we first
% consider a simpler scenario of many-location writes and reads, which is word
% tearing.

% In word tearing, a program contains a write instruction with data larger
% than the hardware word size, for example copying a byte array, or assigning
% a 64-bit float on a 32-bit architecture. For example, consider the program:
% \begin{verbatim}
%   (x := [0, 0];) || (x := [1, 1];) || (r := x;)
% \end{verbatim}
% This has executions in which the read of $x$ only reads from one of the writes,
% for example:
% \[\begin{tikzpicture}[node distance=1em]
%   \event{wx00}{\DW{x}{[0,0]}}{}
%   \event{wx11}{\DW{x}{[1,1]}}{right=2.5em of wx00}
%   \event{rx00}{\DR{x}{[0,0]}}{right=2.5em of wx11}
%   \rf[out=20, in=160]{wx00}{rx00}
% \end{tikzpicture}\]
% but also has executions in which the read of $x$ reads from both writes,
% for example:
% \[\begin{tikzpicture}[node distance=1em]
%   \event{wx00}{\DW{x}{[0,0]}}{}
%   \event{wx11}{\DW{x}{[1,1]}}{right=2.5em of wx00}
%   \event{rx01}{\DR{x}{[0,1]}}{right=2.5em of wx11}
%   \rfx[out=20, in=160]{wx00}{x[0]}{rx01}
%   \rfx[out=-20, in=-160]{wx11}{x[1]}{rx01}
% \end{tikzpicture}\]
% Word tearing can occur, for example, in Java extended floating point~\cite{jmm},
% LLVM 64-bit instructions on 32-bit hardware~\cite{llvm}, or in
% JavaScript SharedArrayBuffers~\cite{js-sab}.

% \newcommand{\rfControl}[4][]{\draw[rf,#1](#2) .. controls (#3) .. (#4);}
% \[\begin{tikzpicture}[node distance=1em]
%   \event{wx0}{\DW{x}{0}}{}
%   \event{wx1}{\DW{x}{1}}{right=of wx0}
%   \event{wy0}{\DW{y}{0}}{right=2.5em of wx1}
%   \event{wy1}{\DW{y}{1}}{right=of wy0}
%   \event{rx1}{\DR{x}{1}}{right=2.5 em of wy1}
%   \event{ry0}{\DR{y}{0}}{right=of rx1}
%   \event{ry1}{\DR{y}{1}}{right=2.5 em of ry0}
%   \event{rx0}{\DR{x}{0}}{right=of ry1}
%   \rf[out=20,in=160]{wx1}{rx1}
%   \rf[out=20,in=160]{wy0}{ry0}
%   \rf[out=340,in=200]{wy1}{ry1}
%   \coordinate (a) [below=of wy1];
%   \rfControl[out=340,in=200]{wx0}{a}{rx0}
%   \wk{wx0}{wx1}
%   \wk{wy0}{wy1}
%   \po{rx1}{ry0}
%   \po{ry1}{rx0}
% \end{tikzpicture}\]


% Batty section 4:
% \cite[\S4]{DBLP:conf/esop/BattyMNPS15},
% Example LB+ctrldata+ctrl-double (language must allow)
% r1=loadrlx(x) //reads 42
% if (r1 == 42)
%   storerlx(y,r1)

% r2=loadrlx(y) //reads 42
% if (r2 == 42)
%   storerlx (x,42)
% else
% storerlx (x,42)

% a:RRLX x=42 sb,dd,cd
% c:RRLX y=42 sb,cd
%   This is forbidden on hardware if compiled naively, as the architectures respect read-to-write control dependencies, but in practice compilers will collapse con- ditionals like that of the second thread, removing the control dependencies from the read of y to the writes of x and making the code identical to the previous example. As that example is allowed and observable on hardware (and we pre- sume that it would be impractical to outlaw such optimisation for C or C++), the language must also allow this execution. But this execution has a cycle in the union of reads-from and dependency, so we cannot simply exclude all those.
% Then one might hope for some other adaptation of the C/C++11 model, but the following example shows at least that there is no per-candidate-execution solution.
% Example LB+ctrldata+ctrl-single (language can and should forbid)
% r1=loadrlx(x) //reads 42 if (r1 == 42)
% storerlx (y,r1) r2=loadrlx (y) //reads 42 if (r2 == 42)
% a:RRLX x=42 sb,dd,cd
% rf
% b:WRLX y=42
% c:RRLX y=42 sb,cd
% rf
% d:WRLX x=42
% rf rf
% b:WRLX y=42 d:WRLX x=42
%   storerlx (x,42)

% Local Variables:
% mode: latex
% TeX-master: "paper"
% End: