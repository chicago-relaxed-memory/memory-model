\section{Sequential Composition}
\label{sec:semicolon}
We provide an alternative semantics that supports full sequential
composition, building $\sem{\aCmd\SEMI\bCmd}$ from $\sem{\aCmd}$ and
$\sem{\bCmd}$.  To simplify the definitions, we assume that each register is
assigned at most once syntactically
\cite{Rosen:1988:GVN:73560.73562}\nofootnote{One can remove this restriction
  by defining conflict (Definition \ref{def:prefix}) to include actions that
  read into the same register; this includes order between reads in item 5b.
  In addition, let $\READS(\aPS)=\{\aEv\mid\aEv$ is a read into $\aReg$ and
  there is no $\aEv$ that reads into $\aReg$ such that $\dEv\le\aEv\}$.  For
  Theorem \ref{thm:seq} to hold, register conflict must then be included in
  $\semold{}$.}.  Since we exclude loops, this trivially ensures that each
register is assigned at most once semantically.

We refactor the syntax:
\begin{align*}
  \aCmd,\,\bCmd
  \BNFDEF& \SKIP
  \mkern-2mu\BNFSEP\mkern-2mu \FENCE^{\fmode}
  \mkern-2mu\BNFSEP\mkern-2mu \aReg\GETS\aExp
  % \mkern-2mu\BNFSEP\mkern-2mu \aReg\GETS \aLoc^{\amode} 
  % \mkern-2mu\BNFSEP\mkern-2mu \aLoc^{\amode}\GETS\aExp
  \mkern-2mu\BNFSEP\mkern-2mu \aReg\GETS \REF{\cExp}^{\amode} 
  \mkern-2mu\BNFSEP\mkern-2mu \REF{\cExp}^{\amode}\GETS\aExp
  \\[-.5ex]
  \BNFSEP&\aCmd \PAR \bCmd
  \mkern-2mu\BNFSEP\mkern-2mu\aCmd \SEMI \bCmd
  \mkern-2mu\BNFSEP\mkern-2mu \VAR\aLoc\SEMI \aCmd
  \mkern-2mu\BNFSEP\mkern-2mu \IF{\aExp} \THEN \aCmd \ELSE \bCmd \FI
\end{align*}
% Without loss of generality,
% To keep the presentation as simple as possible, we include neither address
% calculation nor \RMW{}s.  These extensions are straightforward, but
% notationally cumbersome.

\paragraph{Explicit Substitutions.}
Let $\aEExp$ range over \emph{extended expressions}, which may include memory
locations.  We introduce explicit substitutions over extended expressions,
following the conventions of \citet{DBLP:conf/icalp/RitterP97}:
\begin{gather*}
  \aLocReg\BNFDEF \aLoc \BNFSEP \aReg
  \qquad\quad
  \aSub,\,\bSub,\, \SUBDRS{\dEvs} %\beforeSub,\,\afterSub
  \BNFDEF \SUBEMP \BNFSEP \SUBPAR{\aSub}{\aEExp/\aLocReg}
  \BNFSEP \aSub\SUBSEQ\aSub'
\end{gather*}
$\SUBEMP$ is the identity substitution.  We write
$\SUBPAR{\SUBEMP}{\aEExp/\aLocReg}$ as $\SUB{\aEExp/\aLocReg}$.

Application is written $\aSub\SUBAPP\aForm$.  We only apply substitutions to
formulae---which do not bind locations or registers.  The definition is
homomorphic over the syntax of formulae. For the basis, 
\begin{math}
  \SUBPAR{\aSub}{\aEExp/\bLocReg}\SUBAPP\aLocReg
\end{math}
is $\aEExp$ if $\bLocReg=\aLocReg$ and is $\aSub \SUBAPP\aLocReg$ otherwise.

%and $\SUBPAR{\SUBPAR{\aSub}{\bEExp/\aLocReg}}{\aEExp/\aLocReg}$ as $\SUBPAR{\aSub}{\aEExp/\aLocReg}$.

Sequencing is defined so that
% \begin{math}
%   \aSub\SUBSEQ\SUBPAR{\bSub}{\aEExp/\aLocReg}
%   = 
%   \SUBPAR{\aSub\SUBSEQ\bSub}{\aSub\SUBAPP\aEExp/\aLocReg}
% \end{math}
\begin{math}
  \aSub\SUBSEQ\SUB{\aEExp/\aLocReg}
  \allowbreak= 
  \SUBPAR{\aSub}{\aSub\SUBAPP\aEExp/\aLocReg}
\end{math}
and
\begin{math}
  (\beforeSub\SUBSEQ\afterSub)\SUBAPP\aForm = \beforeSub\SUBAPP(\afterSub\SUBAPP\aForm).
\end{math}

We say that $\aSub$ \emph{subsumes} $\bSub$ if for every $\aLocReg$, either
$\bSub\SUBAPP\aLocReg=\aSub\SUBAPP\aLocReg$ or $\bSub\SUBAPP\aLocReg=\aLocReg$.
For example, every substitution subsumes $\SUBEMP$.
% Goal:
% \begin{math}
%   (\aForm\aSub)\bSub =
%   \aForm(\aSub;\bSub)
% \end{math}
% Pure substitution: $\fdom(\aSub)$ disjoint $\fcodom(\aSub)$.
% Pure substitutions are idempotent.
% $\aSub$ and $\bSub$ are composable if $\fdom(\aSub)$ disjoint $\fcodom(\bSub)$
% \begin{displaymath}
%   (\aSub;\bSub)(x) =
%   \begin{cases}
%     \aSub(\bSub(x)) & \text{if } x \in \fdom(\bSub)\\
%     \aSub(x) & \text{otherwise}
%   \end{cases}
% \end{displaymath}

\paragraph{Substitutions in the Data Model.}
Let $\Sub$ be the set of all (explicit) substitutions.  We include
substitutions as actions: $\Sub\subseteq\Act$.  All substitution actions are
\emph{termination} actions.  For comparison with the semantics of
\textsection\ref{sec:model}, we identify $\SUBEMP$ and $\DSTOP$.

In the context of a
pomset $\aPS$, we use $\Sub$ also to represent the set of substitution
events:
\begin{math}
  \{ \aEv\in\Event \mid \labelingAct(\aEv) \in \Sub \}.
\end{math}
% Alternatively, we
% could extend the Definition \ref{def:mmpomset} to include an optional
% substitution:
Recall that our pomsets contain at most one termination event, which is ordered
after every other event.  
In examples, we typically drop order into
substitution actions, instead drawing them as accepting states.

We modify read actions both to name the register that was written
and also to include a substitution:
$\DRreg[\amode]{\aReg}{\bSub}{\aLoc}{\aVal}$.
We say $(\DRreg[\amode]{\aReg}{\bSub}{\aLoc}{\aVal})$ reads \emph{into}
$\aReg$.

We say that $\aAct$ \emph{substitutes} $\aSub$ if either
$\aAct=\aSub$, $\aAct=(\DRreg[\amode]{\aReg}{\aSub}{\aLoc}{\aVal})$, or
$\aSub=\SUBEMP$ and $\aAct$ is neither a termination nor a read.

We modify Definition \ref{def:rf} to require that when $\aPS$ is
\emph{$\aLoc$-closed}, every substitution in $\aPS$ is the identity on
$\aLoc$: If $\aEv$ substitutes $\aSub$ then $\aSub\SUBAPP\aLoc=\aLoc$.

We lift the notion of subsumption from explicit substitutions to actions and
pomsets in the obvious way: $\aAct$ subsumes $\bAct$ if 
%either (1) $\aAct=\bAct$, (2) $\aAct=\aSub$, $\bAct=\bSub$ and $\aSub$
%subsumes $\bSub$, or (3)
$\aAct$ substitutes $\aSub$,
$\bAct$ substitutes $\bSub$, and $\aSub$ subsumes
$\bSub$.
$\aPS'$ \emph{subsumes} $\aPS$ if $\Event'=\Event$, ${\le'}={\le}$,
$\labelingForm'=\labelingForm$, and $\labelingAct'(\aEv)$
subsumes $\labelingAct(\aEv)$. 

The semantics of programs is closed w.r.t.~\emph{reverse subsumption}: if
$\aPS\in\sem{\aCmd}$ and $\aPS$ is subsumed by $\aPS'$, then
$\aPS'\in\sem{\aCmd}$.

Subsumption is dual to implication (Definition \ref{def:closure:properties}):
Strong\-er preconditions impose a greater burden on the preceding code;
stronger substitutions can better mitigate this burden in the code that
follows.  In examples, we only show executions that are implication and
augmentation minimal; similarly, we only show executions that are
subsumption-maximal.

Suppose $\aPS$ is a completed, subsumption-maximal pomset.  Then
$\labelingAct(\Event\cap\Sub)$ is defined, and all reads satisfy the invariant:
if $\labelingAct(\aEv)=(\DRreg[\amode]{\aReg}{\bSub}{\aLoc}{\aVal})$ then there are
$\beforeSub$ and $\afterSub$ such that
$\bSub=\beforeSub\SUBSEQ\SUB{\aLoc/\aReg}\SUBSEQ\afterSub$ and
$(\beforeSub\SUBSEQ\afterSub) = \labelingAct(\Event\cap\Sub)$.



We define notation to lift sequential composition of substitutions into
read actions:
\begin{itemize}
\item Let $\aAct$ \emph{before} $\aPS$ be $\aAct$ when
  $\disjoint{\Event}{\Sub}$ or $\aAct$ is not a read, and
  $\DRreg[\amode]{\aReg}{(\bSub\SUBSEQ\aSub)}{\aLoc}{\aVal}$ when
  $\aSub=\labelingAct(\Event\cap\Sub)$ and
  $\aAct=\DRreg[\amode]{\aReg}{\bSub}{\aLoc}{\aVal}$.
\item 
  Let $\aAct$ \emph{after} $\aPS$ be $\aAct$ when $\disjoint{\Event}{\Sub}$
  or $\aAct$ is not a read, and
  $\DRreg[\amode]{\aReg}{(\aSub\SUBSEQ\bSub)}{\aLoc}{\aVal}$ when
  $\aSub=\labelingAct(\Event\cap\Sub)$ and
  $\aAct=\DRreg[\amode]{\aReg}{\bSub}{\aLoc}{\aVal}$.
\end{itemize}

\paragraph{Semantics of the Example Language.}

We elide the definitions of parallel composition, conditional and location
binding, which can be imported directly from
\textsection\ref{sec:model}---using the modified definition of
$\aLoc$-closure discussed above.

Note that when coalescing reads or termination events using \!$\PAR$\!,
the substitutions must be identical.  Using reverse subsumption closure, this
is always possible; further it guarantees that the resulting substitution is subsumed by
the substitutions on either side.  Thus, register state is only preserved
after composition if it is shared on both sides.
For example,
\begin{math}
  (r\GETS 1\PAR r\GETS 1)\SEMI x\GETS r
\end{math}
can write $1$ to $x$, but
\begin{math}
  (r\GETS 1\PAR \SKIP)\SEMI x\GETS r
\end{math}
cannot.

To simplify the base cases, we use a literal notation for pomsets and define
$\CLOSE{\aPS}$ to be the smallest set that includes $\aPS$ and is closed
w.r.t.~prefixing, implication, augmentation, and reverse subsumption: Let
$\aPS''\in\CLOSE{\aPS}$ if there is $\aPS'\in\PRE{\aPS}$ such (1) $\aPS''$
implies $\aPS'$, (2) $\aPS''$ is an augmentation of $\aPS'$ and (3) $\aPS''$
is subsumed by $\aPS'$.
\begingroup
\allowdisplaybreaks
\begin{gather*}
  \begin{aligned}
  \sem{\SKIP} & \eqdef
  \CLOSE{\TIKZ{\final{f}{\SUBEMP}{}}}
  \\  
  \sem{\aReg\GETS\aExp} & \eqdef
  \CLOSE{\TIKZ{\final{f}{\SUB{\aExp/\aReg}}{}}}
  \\
  \sem{\FENCE^{\fmode}} & =
  \CLOSE{\TIKZ{
      \event{a}{\DFS{\fmode}}{}
      \final{f}{\SUBEMP}{right=of a}
      \po{a}{f}
    }} 
  \\
  \sem{\aReg\GETS\REF{\cExp}^\amode} & =
  \textstyle\bigcup_{\cVal,\aVal}\;
  \CLOSE{\TIKZ{
      \event{a}{\cExp=\cVal\mid\DRreg[\amode]{\aReg}{\SUB{\REF{\cVal}/\aReg}}{\REF{\cVal}}{\aVal}}{}
      \final{f}{\SUBEMP}{right=of a}
      \po{a}{f}
    }}
  % \\
  % \sem{\aReg\GETS\aLoc^\amode} & =
  % \textstyle\bigcup_\aVal\;
  % \CLOSE{\TIKZ{
  %     \event{a}{\DRreg[\amode]{\aReg}{\SUB{\aLoc/\aReg}}{\aLoc}{\aVal}}{}
  %     \final{f}{\SUBEMP}{right=of a}
  %     \po{a}{f}
  %   }}
  % \\[-.5ex] &
  % \mkern2mu\cup
  % \CLOSE{\TIKZ{
  %     \event{a}{\DFR{\amode}}{}
  %     \final{f}{\SUB{\aLoc/\aReg}}{right=of a}
  %     \po{a}{f}
  %   }}
  \\
  \sem{\REF{\cExp}^\amode\GETS\aExp} & =
  \;\;\textstyle\parallel_{\cVal,\aVal}\;
  \CLOSE{\TIKZ{
      \event{a}{\cExp=\cVal\land\aExp=\aVal \mid \DW[\amode]{\REF{\cVal}}{\aVal}}{}
      \final{f}{\SUB{\aExp/\REF{\cVal}}}{right=of a}
      \po{a}{f}
    }}
  % \\
  % \sem{\aLoc^\amode\GETS\aExp} & =
  % \;\;\textstyle\parallel_\aVal
  % \CLOSE{\TIKZ{
  %     \event{a}{\aExp=\aVal \mid \DW[\amode]{\aLoc}{\aVal}}{}
  %     \final{f}{\aExp=\aVal \mid \SUB{\aExp/\aLoc}}{right=of a}
  %     \po{a}{f}
  %   }}
  % \\[-.5ex] &
  % \mkern2mu\cup
  % \CLOSE{\TIKZ{
  %     \event{a}{\DFW[\aLoc]{\amode}}{}
  %     \final{f}{\SUB{\aExp/\aLoc}}{right=of a}
  %     \po{a}{f}
  %   }}
  \\
  \sem{\aCmd \SEMI \bCmd} &= \sem{\aCmd} \sequence \sem{\bCmd}
  % \\
  % \sem{\aCmd \PAR \bCmd} &= \sem{\aCmd} \parallel\sem{\bCmd}
\end{aligned}
  % \\
  % \begin{aligned}
  %   \sem{\aCmd \PAR \bCmd} &= \sem{\aCmd} \parallel \killS\sem{\bCmd}
  %   &
  %   \killS(\aPS)&=\{ \aEv\in\Event \mid \labelingAct(\aEv)\notin\Sub \}
  %   \\
  %   \sem{\aCmd \SEMI \bCmd} &= \sem{\aCmd} \sequence \sem{\bCmd}
  %   &
  %   \killS(\aPSS)&=\{\aPS\restrict{\killS(\aPS)} \mid \aPS\in\aPSS \}
  % \end{aligned}
\end{gather*}
\endgroup

The most significant challenge is to define semantic sequencing.  The
definition is complex, not only because of the bookkeeping required by
explicit substitutions.  Unfortunately, \emph{disjunction} and \emph{prefix
  weakening} (Definition \ref{def:closure:properties}) do not come easily.

Recall \eqref{alanAddress} from \textsection\ref{sec:variants}:
\begin{math}
  \sem{a[r] \GETS 0\SEMI a[0]\GETS \BANG r}.
\end{math}
In the semantics, an event from the first statement can coalesce with an
event from the second.  Thus when computing
\begin{math}
  \sem{a[r] \GETS 0} \sequence \sem{a[0]\GETS \BANG r},
\end{math}
we must coalesce events with incompatible preconditions ($r{=}0$, $r{=}1$)
that occur on different sides of the sequencing operator.  This makes a
direct definition difficult.

Instead of a direct definition, we first construct the sequential composition
\emph{without} coalescing events, then close the resulting set of pomsets to
ensure the required properties.

There is also a challenge dealing with redundant write elimination:
\begin{math}
  \sem{x\GETS 1\SEMI x^\mREL\GETS 2} 
\end{math}
should contain a pomset that includes only
\begin{math}
  (\DW[\mRA]{x}{2}).
\end{math}
We achieve this using the same strategy: closing after the construction.

Let $\cEv$ be an \emph{unused write} in $\aPS$ when it is a relaxed write to
some $\aLoc$ such that (1) $\cEv$ fulfills no reads, (2) there is some
$\bEv\gt\cEv$ that writes $\aLoc$, and (3) for every release $\aEv\gt\cEv$
there is some $\aEv\gt\bEv\ge\cEv$ that writes $\aLoc$.

Let $\DISJUNCT(\aPSS)$ be the least set that
includes $\aPSS$ and that is closed w.r.t.~disjunction, prefix weakening, and
unused write removal.

% Let $\DISJUNCT(\aPSS)$ be the set $\aPSS''$
% where $\aPS''\in\aPSS''$ if there are $\aPS'\in\PRE{\aPS''}$ and
% $\aPS^i\in\aPSS$ such $\Event' = \Event^i$, ${\le'}={\le^i}$,
% $\labelingAct'=\labelingAct^i$, and $\labelingForm''(\aEv)$ implies
% $\labelingForm'(\aEv)\cup\bigvee_i \labelingForm^i(\aEv)$.

In the following definition, $\aPSS'$ is defined using the subset of
$\aPSS^1$ that have an accepting state: $\labelingAct(\Event^1\cap\Sub)$ is only defined for
$\{\aPS^1\in\aPSS^1\mid \notdisjoint{\Event^1}{\Sub}\}$.  To ensure prefix
closure, we include $\aPSS^1$ before applying $\DISJUNCT$.
% A simple inductive proof shows that for any pomset in $\aPS\in\sem{\aCmd}$, there is
% extension $\aPS'\in\sem{\aCmd}$ with an accepting state, such that $\aPS\in\PRE{\aPS'}$.

% We give the definition as two cases: (1) We include all the
% pomsets in $\aPSS^1$\!\!.  (2) We include composed pomsets for every $\aPS^1$
% with an accepting state.

% \begin{gather*}
%   \begin{aligned}
%     \killS(\aPS)&=\{ \aEv\in\Event \mid \labelingAct(\aEv)\notin\Sub \}
%     &
%     \killS(\aPSS)&=\{\aPS\restrict{\killS(\aPS)} \mid \aPS\in\aPSS \}
%   \end{aligned}
% \end{gather*}

% Relative to the previous definitions, the base cases are handled as follows:
% \begin{itemize}
% \item $\sem{\SKIP}$ introduces the identity substitution,
% \item $\sem{\aLoc\GETS\aExp}$ introduces $\SUB{\aExp/\aLoc}$,
% \item $\sem{\aReg\GETS\aExp}$ introduces $\SUB{\aExp/\aReg}$,
% \item $\sem{\aReg\GETS\aLoc}$, local rule, introduces $\SUB{\aLoc/\aReg}$,
% \item $\sem{\aReg\GETS\aLoc}$, nonlocal rule, introduces
%   $(\DR{\aLoc}{\aVal}) \lt \SUB{\aVal/\aReg}$.
% \end{itemize}
% Base cases:
% \begin{align*}
%   \sem{\SKIP}
%   =&
%   \TIKZ{\final{f}{}{}} 
%   \\
%   \sem{\aLoc\GETS\aExp}
%   =&
%   \textstyle\bigcup_\aVal\; \TIKZ{\event{a}{(\aExp=\aVal\mid\DW\aLoc\aVal)}{}\final{f}{\aExp/\aLoc}{right=of a}}
%   \\
%   \sem{\aReg\GETS\aLoc}
%   =&
%   \TIKZ{\final{f}{\aLoc/\aReg}{}}
%   \cup
%   \textstyle\bigcup_\aVal\; \TIKZ{\event{a}{(\DR\aLoc\aVal)}{}\final{f}{\aLoc/\aReg}{right=of a}\po{a}{f}}
% \end{align*}


% Here's the def for prefixing:
% \begin{enumerate}
% \item[1.] $\Event' = \Event \uplus \{\bEv\}$,
% \item[2.] ${\le'}\supseteq{\le}$,
% \item[3a.] $\labelingAct'(\bEv) = \aAct$,
% \item[3b.] $\labelingForm'(\bEv)$ implies $\aForm$,
% \item[4a.] $\labelingAct'(\aEv) = \labelingAct(\aEv)$,
% \item[4b.] if $\bEv$ \externally reads $\aVal$ from $\aLoc$ then
%   $\labelingForm'(\aEv)$ implies $\labelingForm(\aEv)[\aVal/\aLoc]$,
% \item[4c.] if $\bEv$ does not \externally read then $\labelingForm'(\aEv)$
%   implies $\labelingForm(\aEv)$, 
% \item[5a.] if $\labelingForm'(\aEv)$ does not imply $\labelingForm(\aEv)$ and
%   $\aEv$ writes, then $\bEv\lt'\aEv$,
% \item[5b.] if $\bEv$ and $\aEv$ are \external actions in conflict, then
%   $\bEv\lt'\aEv$,
% \item[5c.] if $\bEv$ is an acquire or $\aEv$ is a release, then
%   $\bEv \lt' \aEv$,
% \item[5d.] if $\bEv$ is an SC write and $\aEv$ is an SC read, then
%   $\bEv \lt' \aEv$, 
% \item[5e.] if $\bEv$ reads and $\labelingAct(\aEv)=\DFS{\mACQ}$, then
%   $\aEv \lt' \bEv$,
% \item[5f.] if $\labelingAct(\bEv)=\DFS{\mREL}$ and $\aEv$ writes, then
%   $\bEv \lt' \aEv$, and
% \item[6.] if $\bEv$ is a release, $\aEv_1$ is an acquire, $\aEv_1\le\aEv_2$, then $\labelingForm(\aEv_2)$
%   is location independent.
% \end{enumerate}


% \begin{definition}
%   \label{def:semi:seq}
%   % Let $\READS(\aPS)$ be the set $\dEvs\subseteq\Event$ where
%   % $\aEv\in\dEvs$ when $\aEv$ is a read of some $\aLoc$, and
%   % there is no $\dEv>\aEv$ that reads from $\aLoc$.

%   % Let $\dEvs$ be a \emph{disjoint read set} of $\aPS$  when 
%   % $\dEvs\subseteq\Event$,
%   % all $\aEv\in\dEvs$ are reads, and
%   % if $\aEv\in\dEvs$ reads from $\aLoc$ then
%   % (1) if $\dEv\in\dEvs$ reads from $\aLoc$ then $\dEv=\aEv$, and
%   % (2) if $\dEv\in\Event$ reads from $\aLoc$ then $\aEv\not\le\dEv$.

%   Let $\dEvs$ be a \emph{maximal disjoint read set} of $\aPS$  when 
%   (1) $\dEvs\subseteq\Event$,
%   (2) all $\aEv\in\dEvs$ are reads,
%   (3) if $\aEv\in\Event$ reads from $\aLoc$ then some $\dEv\in\dEvs$ reads from $\aLoc$, and
%   (4) if $\aEv\in\dEvs$ reads from $\aLoc$ then
%   (4a) if $\dEv\in\dEvs$ reads from $\aLoc$ then $\dEv=\aEv$, and
%   (4b) if $\dEv\in\Event$ reads from $\aLoc$ then $\aEv\not\le\dEv$.

%   Let $\drs(\aPS)$ be the set of maximal disjoint read sets of $\aPS$.

%   Let $\SUBDRS{\dEvs}$ be the substitution such
%   that $\SUBDRS{\dEvs}\SUBAPP\aLoc$ is $\aVal$ if some
%   $\dEv\in\SUBDRS{\dEvs}$ reads $\aVal$ from $\aLoc$, and is $\aLoc$
%   otherwise---where $\dEvs\in\drs(\aPS)$.
  
%   Let $\DISJUNCT(\aPSS)$ be the set $\aPSS''$ where $\aPS''\in\aPSS''$ if
%   there are $\aPS'\in\PRE{\aPS''}$ and $\aPS^i\in\aPSS$ such
%   $\Event' = \Event^i$, ${\le'}={\le^i}$, $\labelingAct'=\labelingAct^i$, and
%   $\labelingForm'(\aEv)$ implies $\bigvee_i \labelingForm^i(\aEv)$.
  
%   % Let $\aPS''$ be a \emph{disjunct} of $\aPS^i$ if there is some
%   % $\aPS'\in\PRE{\aPS''}$ such that $\Event' = \Event^i$, ${\le'}={\le^i}$,
%   % $\labelingAct'=\labelingAct^i$, and $\labelingForm'(\aEv)$ implies
%   % $\bigvee_i \labelingForm^i(\aEv)$.

%   Let $(\aPSS^1 \sequence \aPSS^2)$ be the set
%   $\{\aPS^1\in\aPSS^1\mid \disjoint{\Event^1}{\Sub}\}\cup \DISJUNCT(\aPSS')$
%   where $\aPS'\in\aPSS'$ when there are $\aPS^1 \in \aPSS^1$,
%   $\cEv\in\Event^1\cap\Sub$,
%   $\aSub=\labelingAct^1(\cEv)$, 
%   and $\aPS^2\in\aPSS^2$
%   such that the following hold.
%   Let $\bEv$ range over $\Event^1\setminus\Sub$, and $\aEv$ range over $\Event^2$.  
%   % For each $\aEv$, let $B_\aEv\subseteq\Event^1$ be a set of
%   % reads such that no two events read the same variable, and let $\aSub_\aEv$ be
%   % the associated substitution.
% \begin{enumerate}
% \item[1.] $\Event' = \Event^1\setminus\Sub \uplus \Event^2$,
% \item[2.] ${\le'}\supseteq{\le^1}\cup{\le^2}$, 
% %\item[2b.] if $\bEv\le\cEv$ then $\bEv\le'\aEv$, for $\aEv\in\Event^2\cap\Sub$,
% \item[3a.] $\labelingAct'(\bEv) = \labelingAct^1(\bEv)$,
% \item[3b.] $\labelingForm'(\bEv)$ implies $\labelingForm^1(\bEv)$,
% \item[4a1.] $\labelingAct'(\aEv) = \labelingAct^2(\aEv)$,  for $\aEv\in\Event^2\setminus\Sub$,
% \item[4a2.] $\labelingAct'(\aEv) =  \aSub\SUBSEQ\labelingAct^2(\aEv)$,  for $\aEv\in\Event^2\cap\Sub$, 
% % \item[4b.] if $\dEv\in\READS(\aPS^1)$ reads $\aVal$ from $\aLoc$, there is
% %   some $\bSub$ such that
% %   $\labelingForm'(\aEv)$ implies
% %   $\SUBPAR{\bSub}{\aVal/\aLoc}\SUBAPP(\aSub\SUBAPP\labelingAct^2(\aEv))$,
% \item[4bc.] if
%   $\dEvs\in\drs(\aPS^1)$ then $\labelingForm'(\aEv)$ implies
%   $\SUBDRS{\dEvs} \SUBAPP (\aSub\SUBAPP\labelingForm^2(\aEv))$, 
% % \item[4c.] 
% %   %if $\READS(\aPS^1)=\emptyset$
% %   if $\emptyset\in\drs(\aPS^1)$
% %   then $\labelingForm'(\aEv)$ implies
% %   $\labelingForm(\aEv)$,   
% \item[5a.] if $\aEv$ writes, there is some $\dEvs\in\READS(\aPS^1)$
%   such that when $\dEv\in\dEvs$, either $\dEv\le'\aEv$ or
%   $\dEv$ reads $\aLoc$ and $\labelingForm'(\aEv)$ implies  
%   $\SUBPAR{\SUBDRS{\dEvs}}{\aLoc/\aLoc} \SUBAPP (\aSub\SUBAPP\labelingForm^2(\aEv))$,
% % \item[----]  
% %  for every $\dEvs\in\drs(\aPS^1)$, either
% %   \begin{itemize}
% %   \item[4c]
% %     $\labelingForm'(\aEv)$ implies $\labelingForm(\aEv)$
% %     if $\labelingForm'(\aEv)$ implies
% %     $\aSub\SUBAPP\labelingForm^2(\aEv)$ and $\aEv$ writes,
% %     then for some $\dEvs\in\drs(\aPS^1)$ and every $\dEv\in \dEvs$,
% %     $\dEv\ORDER\aEv$, or
% %   \item[5a] if $\labelingForm'(\aEv)$ implies
% %     $\aSub\SUBAPP\labelingForm^2(\aEv)$ and $\aEv$ writes,
% %     then for some $\dEvs\in\drs(\aPS^1)$ and every $\dEv\in \dEvs$,
% %     $\dEv\ORDER\aEv$,    
% %   \end{itemize}
% \item[5b-f.] as before,
% % \item[5b.] if $\bEv$ and $\aEv$ are \external actions in conflict, then
% %   $\bEv\ORDER\aEv$,
% % \item[5c.] if $\bEv$ is an acquire or $\aEv$ is a release, then
% %   $\bEv \ORDER \aEv$,
% % \item[5d.] if $\bEv$ is an SC write and $\aEv$ is an SC read, then
% %   $\bEv \ORDER \aEv$, 
% % \item[5e.] if $\bEv$ reads and $\labelingAct(\aEv)=\DFS{\mACQ}$, then
% %   $\aEv \ORDER \bEv$,
% % \item[5f.] if $\labelingAct(\bEv)=\DFS{\mREL}$ and $\aEv$ writes, then
% %   $\bEv \ORDER \aEv$, 
% \item[6a.] if $\bEv$ is a release, $\dEv\in\Event'$ is an acquire,
%   $\bEv\le'\dEv\le'\aEv$, then $\labelingForm(\aEv)$ is location independent, and
% \item[6b.] if $\labelingAct(\bEv)=\DFW[\aLoc]{\amode}$, $\bEv\le\aEv$, and
%   $\aEv$ is a release that does not write $\aLoc$, then  some
%   $\dEv$ writes $\aLoc$ and $\bEv\le'\dEv\le'\aEv$. %such that $\dEv$ %(explicitly)
% \end{enumerate}
% \end{definition}


\begin{definition}
  \label{def:semi:seq}
  
%Let $\READS(\aPS)=\{\aEv\mid\aEv \text{ is a read}\}$.
%For $\dEvs\subseteq \READS(\aPS)$, let $\SUBDRS{\dEvs}$ be the derived

  % if
%   $\labelingAct^1(\bEv)
%   =\DRreg[\amode]{\aReg}{\beforeSub}{\aLoc}{\aVal}$, where if
%   $\aEv\in\Event^2\cap\Sub$ then $\aSub=\labelingAct^2(\aEv)$ otherwise $\aSub=\SUBEMP$,

% Let $\SUBDRS{\aPS}$ be a substitution of values for registers that is derived
% from the reads of $\aPS$ as follows: $\SUBDRS{\aPS}\SUBAPP\aReg$ is $\aVal$
% if some $\aEv$ reads $\aVal$ into $\aReg$, and is $\aReg$ otherwise.

  % Let $\dEvs$ be a \emph{disjoint read set} of $\aPS$  when 
  % $\dEvs\subseteq\Event$,
  % all $\aEv\in\dEvs$ are reads, and
  % if $\aEv\in\dEvs$ reads from $\aLoc$ then
  % (1) if $\dEv\in\dEvs$ reads from $\aLoc$ then $\dEv=\aEv$, and
  % (2) if $\dEv\in\Event$ reads from $\aLoc$ then $\aEv\not\le\dEv$.

  % Let $\dEvs$ be a \emph{maximal disjoint read set} of $\aPS$  when 
  % (1) $\dEvs\subseteq\Event$,
  % (2) all $\aEv\in\dEvs$ are reads,
  % (3) if $\aEv\in\Event$ reads from $\aLoc$ then some $\dEv\in\dEvs$ reads from $\aLoc$, and
  % (4) if $\aEv\in\dEvs$ reads from $\aLoc$ then
  % (4a) if $\dEv\in\dEvs$ reads from $\aLoc$ then $\dEv=\aEv$, and
  % (4b) if $\dEv\in\Event$ reads from $\aLoc$ then $\aEv\not\le\dEv$.

  % Let $\drs(\aPS)$ be the set of maximal disjoint read sets of $\aPS$.

  % Let $\SUBDRS{\dEvs}$ be the substitution such
  % that $\SUBDRS{\dEvs}\SUBAPP\aLoc$ is $\aVal$ if some
  % $\dEv\in\SUBDRS{\dEvs}$ reads $\aVal$ from $\aLoc$, and is $\aLoc$
  % otherwise---where $\dEvs\in\drs(\aPS)$.
  
  % Let $\aPS''$ be a \emph{disjunct} of $\aPS^i$ if there is some
  % $\aPS'\in\PRE{\aPS''}$ such that $\Event' = \Event^i$, ${\le'}={\le^i}$,
  % $\labelingAct'=\labelingAct^i$, and $\labelingForm'(\aEv)$ implies
  % $\bigvee_i \labelingForm^i(\aEv)$.

  Let $(\aPSS^1 \sequence \aPSS^2)$ be the set
  \begin{math}
    %\{\aPS^1{\in}\aPSS^1\mid \disjointsmall{\Event^1}{\Sub}\}
    \DISJUNCT(\aPSS^1
    \cup
    \aPSS')
  \end{math}
  where $\aPS'\in\aPSS'$ when there are $\aPS^1 \in \aPSS^1$,
  % $\cEv\in\Event^1\cap\Sub$,
  % $\aSubC=\labelingAct^1(\cEv)$, 
  $\aSubC=\labelingAct(\Event^1\cap\Sub)$, 
  %$\dEvs=\READS(\aPS^1)$,
  and $\aPS^2\in\aPSS^2$
  such that the following hold.

Let $\SUBDRS{\aPS}$ be a substitution of values for registers that is derived
from the reads of $\aPS^1$ as follows: $\SUBDRS{\aPS}\SUBAPP\aReg$ is $\aVal$
if some $\bEv\in\Event^1$ reads $\aVal$ into $\aReg$, and is $\aReg$ otherwise.

Let $\bEv$ range over $\Event^1\setminus\Sub$, and $\aEv$ range over $\Event^2$.  

  % For each $\aEv$, let $B_\aEv\subseteq\Event^1$ be a set of
  % reads such that no two events read the same variable, and let $\aSub_\aEv$ be
  % the associated substitution.

\begin{enumerate}
\item[1.] $\Event' = \Event^1\setminus\Sub \uplus \Event^2$,
\item[2.] ${\le'}\supseteq{\le^1}\cup{\le^2}$, 
\item[3a.] $\labelingAct'(\bEv) = \labelingAct^1(\bEv)$ before $\aPS^2$,
% \item[3a1.] $\labelingAct'(\bEv) = \labelingAct^1(\bEv)$ %, for $\bEv\in\Event^1\setminus\READS(\aPS^1)$,
% \item[3a2.] $\labelingAct'(\bEv) = \labelingAct^1(\bEv)
%   =\DRreg[\amode]{\aReg}{\beforeSub}{\aLoc}{\aVal}$,
%   if
%   $\labelingAct^1(\bEv)
%   =\DRreg[\amode]{\aReg}{\beforeSub}{\aLoc}{\aVal}$, where if
%   $\aEv\in\Event^2\cap\Sub$ then $\aSub=\labelingAct^2(\aEv)$ otherwise $\aSub=\SUBEMP$,
\item[3b.] $\labelingForm'(\bEv)$ implies $\labelingForm^1(\bEv)$,
\item[4a1.] $\labelingAct'(\aEv) = \labelingAct^2(\aEv)$  after $\aPS^1$,  for $\aEv\in\Event^2\setminus\Sub$,
\item[4a2.] $\labelingAct'(\aEv) = \aSubC\SUBSEQ\labelingAct^2(\aEv)$,  for $\aEv\in\Event^2\cap\Sub$, 
\item[4bc.] $\labelingForm'(\aEv)$ implies
  $(\SUBDRS{\aPS^1} \SUBSEQ \aSubC)\SUBAPP\labelingForm^2(\aEv)$, 
\item[5a.] if $\aEv$ writes and
  $\bEv=\DRreg[\amode]{\aReg}{\bSub}{\aLoc}{\aVal}$
  then either $\bEv\le'\aEv$ or
  $\labelingForm'(\aEv)$ implies  
  $(\SUBDRS{\aPS^1}\SUBSEQ \bSub)\SUBAPP\labelingForm^2(\aEv)$,
% \item[old5a.] if $\aEv$ writes 
%   and $\dEv\in\dEvs$, then either $\dEv\le'\aEv$ or
%   $\dEv$ reads from $\aLoc$ into $\aReg$ and $\labelingForm'(\aEv)$ implies  
%   $\SUBPAR{\SUBDRS{\dEvs}}{\aLoc/\aReg} \SUBAPP (\aSubC\SUBAPP\labelingForm^2(\aEv))$,
\item[5b-f.] as before (see \textsection\ref{sec:model}-\ref{sec:variants}). %generalizing 5e: if $\aEv\in\Sub$ then $\dEv\le'\aEv$,
% \item[6.] if $\bEv$ is a release, $\dEv\in\Event'$ is an acquire,
%   $\bEv\le'\dEv\le'\aEv$, then $\labelingForm(\aEv)$ is location independent, and
% \item[6b.] if $\labelingAct(\bEv)=\DFW[\aLoc]{\amode}$, $\bEv\le\aEv$, and
%   $\aEv$ is a release that does not write $\aLoc$, then  some
%   $\dEv$ writes $\aLoc$ and $\bEv\le'\dEv\le'\aEv$. %such that $\dEv$ %(explicitly)
% \item[1.] $\Event' = \Event^1\setminus\Sub \uplus \Event^2$,
% \item[2.] ${\le'}\supseteq{\le^1}\cup{\le^2}$, 
% \item[3a.] $\labelingAct'(\bEv) = \labelingAct^1(\bEv)$,
% \item[3b.] $\labelingForm'(\bEv)$ implies $\labelingForm^1(\bEv)$,
% \item[4a1.] $\labelingAct'(\aEv) = \labelingAct^2(\aEv)$,  for $\aEv\in\Event^2\setminus\Sub$,
% \item[4a2.] $\labelingAct'(\aEv) =  \aSubC\SUBSEQ\labelingAct^2(\aEv)$,  for $\aEv\in\Event^2\cap\Sub$, 
% \item[4bc.] $\labelingForm'(\aEv)$ implies
%   $(\SUBDRS{\dEvs} \SUBSEQ \aSubC)\SUBAPP\labelingForm^2(\aEv)$, 
% \item[5a.] if $\aEv$ writes 
%   and $\dEv\in\dEvs$, then either $\dEv\le'\aEv$ or
%   $\dEv$ reads from $\aLoc$ into $\aReg$ and $\labelingForm'(\aEv)$ implies  
%   $\SUBPAR{\SUBDRS{\dEvs}}{\aLoc/\aReg} \SUBAPP (\aSubC\SUBAPP\labelingForm^2(\aEv))$,
% \item[5b-f.] as before,
% \item[6a.] if $\bEv$ is a release, $\dEv\in\Event'$ is an acquire,
%   $\bEv\le'\dEv\le'\aEv$, then $\labelingForm(\aEv)$ is location independent, and
% \item[6b.] if $\labelingAct(\bEv)=\DFW[\aLoc]{\amode}$, $\bEv\le\aEv$, and
%   $\aEv$ is a release that does not write $\aLoc$, then  some
%   $\dEv$ writes $\aLoc$ and $\bEv\le'\dEv\le'\aEv$. %such that $\dEv$ %(explicitly)
\end{enumerate}
\end{definition}


% \begin{enumerate}
% \item[4b.] if $\bEv$ \externally reads $\aVal$ from $\aLoc$ then
%   $\labelingForm'(\aEv)$ implies $\labelingForm(\aEv)[\aVal/\aLoc]$,
% \item[4c.] if $\bEv$ does not \externally read then
%   $\labelingForm'(\aEv)$ implies $\labelingForm(\aEv)$, and
% \item[5a.] if %$\bEv$ \externally reads and
%   $\labelingForm'(\aEv)$ does not imply $\labelingForm(\aEv)$ and $\aEv$ writes, then
%   $\bEv\lt'\aEv$,
% \end{enumerate}

The item numbers are chosen to match those of the corresponding clauses in
\textsection\ref{sec:model}.  Item 5 is morally unchanged.  We only add
addition 5g, which requires that the accepting state be final (and therefore
excluded from any prefix).
Item 4 is more complicated here:
% Item 2b ensures that order into the substitution $\cEv\in\Event^1$ is
% preserved into any substitution in $\Event'$.
In item 4a, the label of the accepting state is calculated differently from
other states. Instead, items 4b and 4c collapse into a single item here.
% Item 6b is necessary to ensure the effect of $\relfilt{}$ in the local write
% rule\footnote{Because of this, we believe it is necessary to include silent
% actions for local writes when giving a semantics for sequential
% composition.}.
The requirement for a final write is imposed at top level---see below.

In item 4bc, note that the domain of $\aSubC$ is disjoint from the domain
of $\SUBDRS{\aPS^1}$, although registers in the domain of $\aSubC$ may
appear in the expressions in the codomain of $\SUBDRS{\aPS^1}$.

In item 5a, recall that $\bSub=\beforeSub\SUBSEQ\SUB{\aLoc/\aReg}\SUBSEQ\afterSub$ and
$(\beforeSub\SUBSEQ\afterSub) = \aSub$.
Note that
$\SUBDRS{\aPS^1}\SUBSEQ \bSub$ is insensitive to the value assigned to $\aReg$ by
$\SUBDRS{\aPS^1}$.

As a simple example, consider the following:
\begin{gather*}
  \begin{gathered}
    r\GETS y
    \\[-1ex]
    \hbox{\begin{tikzinline}[node distance=.2em]
      \event{a}{\DRreg{r}{\SUB{y/r}}{y}{1}}{}
      \final{f}{\SUBEMP}{below=of a}
      \end{tikzinline}}
  \end{gathered}
  \qquad
  \begin{gathered}
    x\GETS r
    \\[-1ex]
    \hbox{\begin{tikzinline}[node distance=.2em]
      \event{b}{r\EQ1 \mid \DW{x}{1}}{}
      \final{f}{r\EQ1 \mid \SUB{r/x}}{below=of b}
      \end{tikzinline}}
  \end{gathered}
  \qquad
  \begin{gathered}
    s\GETS x
    \\[-1ex]
    \hbox{\begin{tikzinline}[node distance=.2em]
      \event{c}{\DFR{}}{}
      \final{f}{\SUB{x/s}}{below=of c}
      \end{tikzinline}}
  \end{gathered}
  \qquad
  \begin{gathered}
    z\GETS s
    \\[-1ex]
    \hbox{\begin{tikzinline}[node distance=.2em]
      \event{d}{\DW{z}{1}}{}
      \final{f}{\SUB{s/z}}{below=of d}
      \end{tikzinline}}
  \end{gathered}
  \\
  % \begin{gathered}
  %   r\GETS y\SEMI x\GETS r
  %   \\[-1ex]
  %   \hbox{\begin{tikzinline}[node distance=1em]
  %       \event{a}{\DRreg{r}{}{y}{1}}{}
  %       \event{b}{\DW{x}{1}}{right=of a}
  %       \final{f}{\SUB{y/r, 1/x}}{right=of b}
  %       \po{a}{b}
  %     \end{tikzinline}}
  % \end{gathered}
  % \qquad
  % \begin{gathered}
  %   s\GETS x
  %   \\[-1ex]
  %   \hbox{\begin{tikzinline}[node distance=1em]
  %     \event{c}{\DFR{}}{}
  %     \final{f}{\SUB{x/s}}{right=of c}
  %     \po{c}{f}
  %     \end{tikzinline}}
  % \end{gathered}
  % \\
  \begin{gathered}
    r\GETS y\SEMI x\GETS r \SEMI s\GETS x \SEMI z\GETS s
    \\[-1ex]
    \hbox{\begin{tikzinline}[node distance=1em]
        \event{a}{\DRreg{r}{\SUB{y/r, r/x, x/s, s/z}}{y}{1}}{}
        \event{b}{\DW{x}{1}}{right=of a}
        \event{c}{\DFR{}}{right=of b}
        \event{d}{\DW{z}{1}}{right=of c}
        \final{f}{\SUB{r/x, x/s, s/z}}{right=of d}
        \po{a}{b}
        \po[out=-10,in=-160]{a}{d}
      \end{tikzinline}}
  \end{gathered}
\end{gather*}

To see the need for parallel substitution of all register values via $\SUBDRS{}$ in 4bc, consider that the
precondition of $\DW{y}{1}$ must be $\FALSE$ after composing:
\begin{gather*}
  \begin{gathered}
    r\GETS x\SEMI s\GETS z
    \\[-1ex]
    \hbox{\begin{tikzinline}[node distance=1em]
      \event{a}{\DRreg{r}{\SUB{x/r}}{x}{1}}{}
      \event{b}{\DRreg{s}{\SUB{z/s}}{z}{2}}{right=of a}
      \final{f}{\SUBEMP}{right=of b}
      \end{tikzinline}}
  \end{gathered}
  \qquad
  \begin{gathered}
     \IF{r{=}s}\THEN y\GETS1\FI
    \\[-1ex]
    \hbox{\begin{tikzinline}[node distance=1em]
      \event{c}{r{=}s\mid\DW{y}{1}}{}
      \final{f}{r{=}s\mid\SUB{1/y}}{right=of c}
      \end{tikzinline}}
  \end{gathered}
  \\
  \begin{gathered}
    r\GETS x\SEMI s\GETS z \SEMI \IF{r{=}s}\THEN y\GETS1\FI
    \\[-1ex]
    \hbox{\begin{tikzinline}[node distance=1em]
      \event{a}{\DRreg{r}{\SUB{x/r}}{x}{1}}{}
      \event{b}{\DRreg{s}{\SUB{z/s}}{z}{2}}{right=of a}
      \nonevent{c}{1{=}2\mid\DW{y}{1}}{right=of b}
      \nonfinal{f}{1{=}2\mid\SUB{1/y}}{right=of c}
      \end{tikzinline}}
  \end{gathered}
\end{gather*}
This pomset candidate does not satisfy the \emph{compatibility} requirement
of Definition \ref{def:mmpomset}.  Note, however, that
$\IF{r{=}s}\THEN y\GETS1\FI$ is shorthand for
$\IF{r{=}s}\THEN y\GETS1\ELSE\SKIP\FI$, and thus we do have a valid pomset
for this composition, even with this choice of read values:
\begin{gather*}
  \begin{gathered}
    r\GETS x\SEMI s\GETS z
    \\[-1ex]
    \hbox{\begin{tikzinline}[node distance=1em]
      \event{a}{\DRreg{r}{\SUB{x/r}}{x}{1}}{}
      \event{b}{\DRreg{s}{\SUB{z/s}}{z}{2}}{right=of a}
      \final{f}{\SUBEMP}{right=of b}
      \end{tikzinline}}
  \end{gathered}
  \qquad
  \begin{gathered}
     \IF{r{=}s}\THEN y\GETS1\FI
    \\[-1ex]
    \hbox{\begin{tikzinline}[node distance=1em]
      \final{f}{r{\neq}s\mid\SUBEMP}{}
      \end{tikzinline}}
  \end{gathered}
  \\
  \begin{gathered}
    r\GETS x\SEMI s\GETS z \SEMI \IF{r{=}s}\THEN y\GETS1\FI
    \\[-1ex]
    \hbox{\begin{tikzinline}[node distance=1em]
      \event{a}{\DRreg{r}{\SUB{x/r}}{x}{1}}{}
      \event{b}{\DRreg{s}{\SUB{z/s}}{z}{2}}{right=of a}
      \final{f}{1{\neq}2\mid\SUBEMP}{right=of b}
      \end{tikzinline}}
  \end{gathered}
\end{gather*}


The see the need for substitutions on read actions, used in 5a, consider the
following, where
$\bSub=\SUB{0/x,\allowbreak x/r,\allowbreak {-}2/x}$:
\begin{gather*}
  \begin{gathered}
    x\GETS 0\SEMI r\GETS x\SEMI x\GETS{-2} 
    \\[-1ex]
    \hbox{\begin{tikzinline}[node distance=1em]
      \event{a}{\DW{x}{0}}{}
      \event{b}{\DRreg{r}{\bSub}{x}{1}}{right=of a}
      \event{c}{\DW{x}{{-}2}}{right=of b}
      \wk{a}{b}
      \wk{b}{c}
      \final{f}{\SUB{{-2}/x}}{right=of c}
      \end{tikzinline}}
  \end{gathered}
  \quad
  \begin{gathered}
    \IF{r{\geq}0}\THEN y\GETS1\FI
    \\[-1ex]
    \hbox{\begin{tikzinline}[node distance=1em]
      \event{d}{r{\geq}0\mid\DW{y}{1}}{}
      \final{f}{r{\geq}0\mid\SUB{1/y}}{right=of d}
      \end{tikzinline}}
  \end{gathered}
  \\
  \begin{gathered}
    x\GETS 0\SEMI r\GETS x\SEMI x\GETS{-2} \SEMI \IF{r{\geq}0}\THEN y\GETS1\FI
    \\[-1ex]
    \hbox{\begin{tikzinline}[node distance=1em]
      \event{a}{\DW{x}{0}}{}
      \event{b}{\DRreg{r}{\SUBPAR{\bSub}{1/y}}{x}{1}}{right=of a}
      \event{c}{\DW{x}{{-}2}}{right=of b}
      \wk{a}{b}
      \wk{b}{c}
      \event{d}{0{\geq}0\mid\DW{y}{1}}{right=of c}
      \final{f}{0{\geq}0\mid\SUB{{-2}/x,1/y}}{right=of d}
      \end{tikzinline}}
  \end{gathered}
\end{gather*}


The semantics validates expected equations, such as
\begin{math}
  \sem{\aCmd_1\SEMI\aCmd_2}\sequence\sem{\aCmd_3} =
  \sem{\aCmd_1}\sequence\sem{\aCmd_2\SEMI\aCmd_3},
\end{math}
\begin{math}
  \sem{\IF{\aExp} \THEN \aCmd \SEMI\bCmd_1\ELSE \aCmd\SEMI\bCmd_2\FI} =
  \sem{\aCmd}\sequence\sem{\IF{\aExp} \THEN \bCmd_1 \ELSE \bCmd_2\FI},
\end{math}
and
\begin{math}
  \sem{\IF{\aExp} \THEN \aCmd_1 \SEMI\bCmd\ELSE\allowbreak \aCmd_2\SEMI\bCmd\FI} =
  \sem{\IF{\aExp} \THEN \aCmd_1 \ELSE \aCmd_2\FI}\sequence\sem{\bCmd}.
\end{math}


The semantics is equivalent to that of the main text.

Let $\killS\aPSS$ be the set $\aPSS'\subseteq\aPSS$ where $\aPS'\in\aPSS'$
every $\aEv'\in\Event'$ substitutes $\SUBEMP$.

\begin{proposition}
  \label{thm:seq}
  Let $\semold{\aCmd}$ be the semantics of \textsection\ref{sec:model},
  adopting the read actions of this section---the read substitution
  being $\SUBEMP$.
\begin{displaymath}
  \semold{\aCmd} = \killS\sem{\aCmd}
\end{displaymath}
% \begin{align*}
%   \semold{\aCmd} &= \{ \aPS\in\sem{\aCmd} \mid \disjoint{\Event}{\Sub}\}
%   \\
%   \sem{\aCmd_1\SEMI\aCmd_2}\sequence\sem{\aCmd_3} &= \sem{\aCmd_1}\sequence\sem{\aCmd_2\SEMI\aCmd_3}
%   \\
%   \hbox{\small$\sem{\IF{\aExp} \THEN \aCmd_1 \ELSE \aCmd_2\FI}\sequence\sem{\bCmd}$} &= 
%   \hbox{\small$\sem{\IF{\aExp} \THEN \aCmd_1 \SEMI\bCmd\ELSE \aCmd_2\SEMI\bCmd\FI}$}
% \end{align*}
\end{proposition}





% \begin{comment}
% Plan: 
%   a.  Define pom1; pom2  for pomsets
%   b.  [| C1 ; C2 |] = cup { pom1; pom2 | pom1 in C1, pom2 in C2}

% Def:
%    pom1; pom2
%              cup_L  L prefix pom2 
%              where L is a  linearization of pom1
% \end{comment}

% \section{Variations}

% \citet{2019-sp} define \emph{3-valued pomsets with preconditions} to model
% security flaws that arise from speculative evaluation in computer
% microarchitecture (such as Spectre \cite{DBLP:journals/corr/abs-1801-01203}).
% \begin{definition}
%   %\label{def:3valued}
%   A \emph{3-valued pomset with preconditions} is a tuple
%   $(\Event, {\le}, {\gtN}, \labeling)$, such that
%   \begin{itemize}
%   \item $\Event$ is a set of \emph{states},
%   \item $\labeling: \Event \fun (\Formulae\times\Act)$ is a \emph{labeling},
%   \item ${\le} \subseteq (\Event\times\Event)$ is a partial order, and
%   \item ${\gtN} \subseteq (\Event\times\Event)$ such that:
%     \begin{itemize}
%     \item\label{5a} if $\bEv \le \aEv$ then $\bEv \gtN \aEv$, \hfill
%       (Inclusion)
%     \item\label{5b} if $\bEv \le \aEv$ and $\aEv \gtN \bEv$ then
%       $\bEv = \aEv$, and \hfill (Consistency)
%     \item\label{5c} if $\cEv \le \bEv \gtN \aEv$ or $\cEv \gtN \bEv \le \aEv$
%       then $\cEv \gtN \aEv$.  \hfill (Semi-transitivity)
%     \end{itemize}
%   \end{itemize}

%   A \emph{(memory model) pomset} is a 3-valued pomset with preconditions,
%   such that
%   \begin{itemize}
%   \item if $\bEv\le\aEv$ then $\labelingForm(\aEv)$ implies
%     $\labelingForm(\bEv)$, and \hfill (Causal-strengthening)
%   \end{itemize}
% \end{definition}
% The axioms for $\gtN$ are adapted
% from \citet[A1--A3]{DBLP:journals/dc/Lamport86}.  

% \begin{definition}
%   A pomset \emph{coherent} if, when restricted to events that read or write
%   any single location $\aLoc$, $\gtN$ forms a partial order.
% \end{definition}

% \begin{figure}
% \begin{eqnarray*}
%   \sem{\SKIP}
%   & = & \TIKZ{\final{f}{}{}} 
%   \\
%   \sem{\aLoc\GETS\aExp}
%   & = & \textstyle\bigcup_\aVal\; \TIKZ{\event{a}{(\aExp=\aVal\mid\DW\aLoc\aVal)}{}\final{f}{\aExp/\aLoc}{right=of a}}
%   \\
%   \sem{\aReg\GETS\aLoc}
%   & = &
%   \TIKZ{\final{f}{\aLoc/\aReg}{}}
%   \cup
%   \textstyle\bigcup_\aVal\; \TIKZ{\event{a}{(\DR\aLoc\aVal)}{}\final{f}{\aLoc/\aReg}{right=of a}\po{a}{f}}
%   \\
%   \sem{\IF{\aExp} \THEN \aCmd \ELSE \bCmd \FI}
%   & = & \bigl((\aExp \neq 0) \guard \sem{\aCmd}\bigr) \parallel \bigl((\aExp=0) \guard \sem{\bCmd}\bigr)
%   \\
%   \sem{\aCmd \SEQ \bCmd}
%   & = & \sem{\aCmd} \sequence \sem{\bCmd}
%   \\
%   \sem{\aCmd \PAR \bCmd}
%   & = & \sem{\aCmd}\fork \parallel \sem{\bCmd}
%   \\
%   \sem{\VAR\aLoc\SEMI \aCmd}
%   & = & \nu \aLoc \DOT \sem{\aCmd}
% \end{eqnarray*}
% \caption{Semantics of a concurrent shared-memory language}
% \label{fig:semi:programs}
% \end{figure}

% \section{Sequential Composition}
% \label{sec:semi:model}
% \subsection{Data models}
% \label{sec:semi:preliminaries}

% A \emph{data model} consists of:
% \begin{itemize}
% \item a set of \emph{substitutions} $\Sub$, ranged over by
%   $\aSub$ and $\bSub$,
% \end{itemize}
% % Let $\LR=\Loc\cup\Reg$, be the set of \emph{locations}, ranged over by
% % $\aLR$.  Let $\Sub=\LR\partialfun\Exp$ be the set of \emph{substitutions},
% % ranged over by $\aSub$ and $\bSub$.
% Let $\aActSub$ and $\bActSub$ range over $(\Act\cup\Sub)$.
% % The empty substitution is written as $[\,]$.

% We require that data models satisfy the following:
% \begin{itemize}
% \item substitutions include at least $[\aLoc/\aReg]$ and $[\aExp/\aLoc]$,
% \item substitutions are closed under composition,
% \end{itemize}
% % By composition, it follows that substitutions must include $[\aExp/\aReg]$
% % which is equal to $[\aLoc/\aReg][\aExp/\aLoc]$.

% \subsection{Modal pomsets}
% \label{sec:semi:pomsets}

% We fix the alphabet $\Alphabet=(\Formulae\times(\Act\cup\Sub))$.  With this
% alphabet, the labeling of a pomset determines two disjoint sets of events.
% We refer to those that map to actions $\Sub$ as \emph{final}, and those that
% map to substitutions $\Act$ as \emph{nonfinal}.
% When we need to distinguish these, we let the set of events
% $\Event=\EAct\uplus\ESub$, where $\EAct$ denotes the nonfinal events 
% and $\ESub$ denotes the final events.

% We write pairs in $(\Formulae\times(\Act\cup\Sub))$ as $(\aForm \mid \aActSub)$.
% \subsection{Semantics of programs}
% \label{sec:semi:semantics}

% Semantics in Figure \ref{fig:semi:programs}

% We give the semantics using combinators over sets of pomsets, defined below.
% Using $\aPSS$ to range over sets of pomsets, these are:
% \begin{itemize}
% \item \emph{forking} $\aPSS\fork$, which removes final states from
%   $\aPSS$, 
% \item \emph{sequencing} $\aPSS_1\sequence\aPSS_2$, which prepends
%   $\aPSS_1$ to $\aPSS_2$, calculating dependencies between the two.
% \end{itemize}

% Forking is %and substitution each perform
% a simple transformation on each pomset
% in a set of pomsets.

% \begin{definition}
% Let $\aPSS\fork$ be the set $\aPSS'$ where $\aPS'\in\aPSS'$ whenever
% there is $\aPS\in\aPSS$ such that:
% $\Event' = \EAct$,
% $\labeling'\subseteq\labeling$,
% $\aEv\le'\bEv$ whenever $\aEv\le\bEv$, and
% $\aEv\gtN'\bEv$ whenever $\aEv\gtN\bEv$.
% \end{definition}

% \subsection{Sequencing}
% \begin{definition}
%   \label{def:semi:seq}
%   Let $\aPS' \in (\aPSS_1 \sequence \aPSS_2)$ whenever there are
%   $\aPS_1 \in \aPSS_1$ and $\aPS_{\cEv}\in\aPSS_2$, for every
%   $\cEv\in\ESub_1$, such that:
% \begin{itemize}
% \item $\Event' = \EAct_1 \cup \bigcup_{\cEv}\Event_{\cEv}$,
% \item if $\bEv \le_1 \aEv$ or $\bEv\le_{\cEv} \aEv$ then $\bEv \le' \aEv$,
% \item if $\bEv \gtN_1 \aEv$ or $\bEv\gtN_{\cEv} \aEv$ then $\bEv \gtN' \aEv$,
% \item if $\bEv\in\EAct_1$ then $\labeling'(\bEv) = \labeling_1(\bEv)$, and
% \item if $\labeling_1(\bEv) = (\dontcare \mid \aAct)$ and
%   $\labeling_{\cEv}(\aEv) = (\bForm \mid \bAct)$ then:
%   \begin{itemize}
%   \item if $\aAct$ is an acquire or $\bAct$ is a release then $\bEv \lt' \aEv$,
%   \item if $\aAct$ is an acquire then $\bForm$ is independent of every $\bLoc$,
%   \item either $\aAct$ and $\bAct$ are provably separable, or both are reads, or
%     $\bEv \gtN' \aEv$, and
%   \item if $\aAct$ and $\bAct$ conflict
%     then $\bEv \gtN' \aEv$, and
%   \end{itemize}
% \item 
%   if
%   $\labeling_1(\cEv) = (\dontcare \mid \aSub)$, and
%   $\labeling_{\cEv}(\aEv) = (\bForm \mid \bActSub)$ then
%   $\labeling'(\aEv) = (\bForm' \mid \bActSub\aSub)$ where:
%   \begin{itemize}
%   \item $\bForm'$ implies
%     $\lor\{\aForm\mid\bEv\in\ESub_1 \land \aEv\in\aPS\!_{\bEv}
%     \land \labeling_1(\bEv)=(\aForm{\mid}\dontcare)\}$,
%   \item $\bForm'$ implies $\bForm\aSub$, and
%   \item either $\bForm'$ implies $\bForm$ or $\bEv\lt_1\cEv$ implies $\bEv\lt'\aEv$.
%     %$\bEv\in\EAct_1$ such that.
%   \end{itemize}
% \end{itemize}
% \end{definition}


% \begin{comment}
% The first constraint ensures that events are ordered before a release and
% after an acquire.  The second constraint ensures that thread-local reads do
% not cross acquire fences.
 
% The second constraint prevents bad executions like the following:
%    x=1; rel; acq; if (x) {y=1};  ||  acq; x=0; rel; 
% where the second thread is interleaved between the rel and acq of the first.

% Note that you cannot require that $\bForm'$ is independent of every $\bLoc$
% because then it's not augment closed.
% \end{comment}


% % To see that we need $[\aExp/\aLoc]$ in the rule for write, rather than $[\aVal/\aLoc]$
% % consider example:
% % \begin{verbatim}
% % r=y; if (r) {x=r} else {x=r}; s=x; if (r==s) {z=1}
% % \end{verbatim}
% % or simplified:
% % \begin{verbatim}
% % r=y;x=r;s=x; if(s==r){z=1}
% % \end{verbatim}
% % If you read 37 for $y$, then the predicate on \texttt{Wz1} before the
% % read is either $r=r$ or $v=r$, where $v=37$, for example.  In one case you
% % get a dependency and in the other you do not.



% Local Variables:
% mode: latex
% TeX-master: "paper"
% End:
