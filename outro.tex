\section{Other Related Work}\label{sec:related}
We survey related work not discussed previously.
% \textsection\ref{sec:intro}-\ref{sec:logic} and \textsection\ref{sec:outro}.

A memory consistency model for a shared-memory multiprocessor defines the
values that a read may return.  For a survey of hardware models, see
\citep{AlglaveThesis}. For software models, see
\citep{DBLP:journals/toplas/Lochbihler13,DBLP:phd/ethos/Batty15}.  For an
attempt to bridge the two, see \citep{DBLP:journals/pacmpl/PodkopaevLV19}.

In our previous work~\cite{2019-sp}, we introduced the notion of pomsets with preconditions.
This was to study \emph{micro\hyp{}architecture}---specifically, speculative execution.
We have presented an \emph{architectural} model, leading to some formal
differences between the models.  Chief among these: we now require pomsets
to satisfy \emph{consistency} requirement, which simplifies the model,
though it can no longer reason about executions where speculative
behaviour can become visible, for example by cache timing attacks.
There are many less fundamental differences.  For example,
the previous use of \emph{three-valued pomsets} means that we allowed \ref{MCA2},
but disallowed its two-location variant.  This odd behavior stems from
\emph{semi\hyp{}transitivity}, inherited from
\cite{DBLP:journals/dc/Lamport86}.

Both \cite{2019-sp} and this paper are inspired by \citeauthor{Batty17}'s
[\citeyear{Batty17}] call for a compositional approach to the semantics of
relaxed memory concurrency.  
% in relaxed 
% outlines the program of trying to
% understand relaxed dependency modularly, inspiring \citet{2019-sp} and
% \citet{DBLP:conf/esop/PaviottiCPWOB20}.  % ; impressively, the latter underlies a
% recent proposal to update the ISO-C11 standard.
%(http://www.open-std.org/jtc1/sc22/wg21/docs/papers/2019/p1780r0.html).

Our model shares important structural elements with that of
\citet{DBLP:conf/esop/PaviottiCPWOB20}, who provide a fix for the thin-air
problem in C11.  Like us, they use true concurrency semantics to identify
(in)dependence in an execution and thus calculate the preserved program order
explicitly.  Our definition of parallel composition allows events to
coalesce, taking preconditions via disjunction---this is mirrored by
\citeauthor{DBLP:conf/esop/PaviottiCPWOB20}'s definition of coproduct.  Their
condition on $\leq^{\mathcal{X}}$ (discussed in their \textsection6.3) is
mirrored by our requirement that events that coalesce must have the same
downset.  Nonetheless, the papers have different goals, leading to different
outcomes.  For example, their model compiles efficiently to non-MCA
architectures; our model clearly does not!  Conversely, our model provides an
intrinsic characterization optimizations, such as redundant read elimination,
which only hold in their model up to observational refinement.
% In contrast to our LDRF
% theorem, they prove an internal version of DRF, ie that race free programs
% have SC behavior, without necessarily settling the question of discovering
% races by SC executions.

% \citeauthor{DBLP:conf/esop/PaviottiCPWOB20} develop a comprehensive array of
% tools to deal with the testing and validation of weak memory models.  In
% future work, we hope to adapt this suite to our setting.



True concurrency techniques have been applied to relaxed memory by
\citet{DBLP:conf/esop/CenciarelliKS07}, \citet{Castellan},
\citet{Pichon-Pharabod:2016:CSR:2837614.2837616}, and
\citet{DBLP:conf/cgo/ChakrabortyV17}.
% See
% \citet{DBLP:journals/lmcs/JeffreyR19} for a discussion.
Partial order models have been developed by \citet{brookes} and
\citet{DBLP:journals/lmcs/KavanaghB19}.

There is also a rich literature on the use of transformations over SC
executions to model relaxed memory:
%
\citet{Saraswat:2007:TMM:1229428.1229469}
aimed to describe a \jmm-like model this way.  \drfsc\ holds, but \citet{SP} 
discovered that it permits \oota\ behavior.
%
\citet{DBLP:conf/popl/DemangeLZJPV13} developed \textsc{bmm}, which permits
reordering of a relaxed write with a following relaxed read.
% The paper develops an axiomatic semantics and studies effects on compilation.  
\textsc{bmm} is designed as a restriction of the \jmm\ that compiles efficiently to
\tso.  It requires fencing on other architectures.

\citet{DBLP:conf/fm/LahavV16} characterized \tso\ as being derived by
considering Write-Read (\textsf{WR}) reordering and Read-After-Write
(\textsf{RAW}) elimination.  They also showed that the release acquires of
C11 are less expressive than considering \textsf{WR} and \textsf{RAW}
together with thread-inlining.  Our paper is inspired by their implicit
challenge: ``Some memory models can be defined via transformations.  But
there is more to weak memory than transformations.''

% \paragraph{Modeling the Invalidation of Load Buffering.} 

% In this paragraph, we argue that our paper is relevant even to the reader
% who desires a semantics that invalidates ``load buffering,'' or ``reads from the future.''
% There are two related lines of research that suggest to prohibit the
% reordering of loads with following stores.
% \begin{itemize}
% \item \citet{Dolan:2018:BDR:3192366.3192421} imposed this restriction in
%   order to ``bound data races in space and time.''
%   Consider the following program:
%   % In the following program $r,s$ are registers, and $x,y,z,u,v,w$
%   % are shared variables.
%   % \begin{align*}
%   %   & (u \GETS 0 \SEMI y \GETS 1 \SEMI v\REL  \GETS 1 \SEMI z \GETS 2 \SEMI r \GETS u \SEMI y \GETS 3) \\[-.5ex]
%   %   \PAR & (y \GETS 2 \SEMI  \IF{v\ACQ} \THEN w \GETS 1 + y -y \SEMI u \GETS 1 \SEMI x \GETS 1  \FI \SEMI s \GETS u \SEMI  y \GETS 2)
%   % \end{align*}
% \begin{displaymatharrayrl}
%   &x\GETS1\SEMI y\GETS1\SEMI v\REL\GETS1\SEMI z\GETS1\SEMI
%   \IF{w}\THEN x\GETS2\FI
%   \\\PAR
%   &y\GETS2\SEMI \aReg{}\GETS v\ACQ\SEMI z\GETS2\,\SEMI
%   \IF{\aReg{}}\THEN w\GETS{y-y+x}\FI
% \end{displaymatharrayrl}
% \begin{displaymath}
%     \begin{tikzpicture}[node distance=1.2em and .8em,baselinecenter]
%     \event{a1}{\DW{x}{1}}{}
%     \event{a2}{\DW{y}{1}}{right=of a1}
%     \event{a3}{\DWRel{v}{1}}{right=of a2}
%     \event{a4}{\DW{z}{1}}{right=of a3}
%     \pox{a1}{a2}
%     \pox{a2}{a3}
%     \pox{a3}{a4}
%     \event{b1}{\DW{y}{2}}{below=of a2}
%     \event{b2}{\DRAcq{v}{1}}{below=of a3}
%     \event{b3}{\DW{z}{2}}{right=of b2}
%     %\co{b1}[right]{a2}
%     \event{b4}{\DR{y}{?}}{right=of b3}
%     \event{b5}{\DR{y}{?}}{right=of b4}
%     %\co{b3}[right]{a4}
%     \event{b6}{\DR{x}{1}}{right=of b5}
%     \pox{b1}{b2}
%     \pox{b2}{b3}
%     \pox{b3}{b4}
%     \pox{b4}{b5}
%     \pox{b5}{b6}
%     \rf{a3}{b2}
%     \event{b7}{\DW{w}{1}}{right=of b6}
%     \event{a5}{\DR{w}{1}}{above=of b7}
%     \event{a6}{\DW{x}{2}}{right=of a5}
%     \pox{a4}{a5}
%     \pox{a5}{a6}
%     \pox{b6}{b7}
%     \rf{b7}{a5}
%     \rf[out=-18,in=162]{a1}{b6}
%     %\race{b7}{a5}
%     %\race{a6}{b6}
%   \end{tikzpicture}
% \end{displaymath}
% In their model, one can deduce that $w$ must be assigned $1$, despite the
% past race on $y$, concurrent race on $z$ and future race on $x$.
% % In their model, one can reason sequentially about the  assignment to $w$
% % and deduce that $x=1$ implies $z= 1$ despite the concurrent races on $u$, and
% % the past and future races on $y$, where past and future are stated with
% % respect to the assignment to $w$.

% \item In the C11 family of models, eg. see
%   \citep{Boehm:2014:OGA:2618128.2618134,DBLP:conf/pldi/LahavVKHD17,DBLP:conf/oopsla/VafeiadisN13},
%   load buffering is forbidden to avoid \oota. The elegant discussion in
%   \citet{BoehmOOTA} elucidates the issues via a series of ``\oota'' like
%   examples.  In particular, to rule out example~\ref{rfub}.
% \end{itemize}

% Our model can easily capture models that forbid load buffering.  It
% necessitates two changes.
% \begin{itemize}
% \item Modify the semantic rule for read to remove the possibility of an
%   internal read:
%   \begin{align*}
%     \sem{\aReg\GETS\REF{\cExp}\SEMI \aCmd} & =
%     \textstyle\bigcup_{\aLoc=\REF{\bVal}}\; ({\cExp}=\bVal) \guard \textstyle\bigcup_\aVal\; (\DR\aLoc\aVal) \prefix \sem{\aCmd}[\aLoc/\aReg] 
%   \end{align*}
% \item Modify item \ref{pre-read}b in the definition of prefixing to require
%   order from a read to subsequent write.  Item \ref{pre-read}b$'$ becomes: if
%   $\aEv$ is a write then $\cEv\lt'\aEv$.  In \ref{pre-read}b$'$ we removed
%   the ``or  $\labelingForm'(\aEv)$ implies $\labelingForm(\aEv)$.''
% \end{itemize}

% While \citet{Dolan:2018:BDR:3192366.3192421} already describes DRF and the
% compilation to \armeight, our approach yields the following new insights.
% \begin{itemize}
% \item Compositional treatment of temporal invariants
%   from \textsection\ref{sec:logic} holds mutatis mutandis, since all
%   executions of the restricted model are already accounted for in the general
%   model.

% \item With regards to single-threaded optimizations in
%   \textsection\ref{sec:opt}, our approach provides different methods to prove
%   optimizations, which complement the extant methods of
%   \citet{Dolan:2018:BDR:3192366.3192421} with data sensitivity; for example,
%   our methods provides a simple proof of the transformation of
%   $\IF{\aExp}\THEN x \GETS 1 \SEMI \aCmd \ELSE x\GETS 1 \SEMI \bCmd\FI$ to
%   $x \GETS 1 \SEMI \IF{\aExp}\THEN \aCmd \ELSE \bCmd\FI$.

% \end{itemize}

%\paragraph{Relationship to Models of Speculation.}

\section{Limitations}
\label{sec:limits}

% Our \drfsc{} theorem does not address fences or \RMW{}s.  \emph{On the upside,
% we have proven a \emph{local} \drfsc{} theorem, which is stronger
% than the standard \drfsc{} result.}
%\todo{assume fixed sized access}

 Our work has several limitations, each of which provides an opportunity for
future research.  % In several cases, these limitations have allowed us to
% \emph{surpass prior work in significant ways}.

We treat loops via unrolling: loops introduce
complexities\allowbreak{}---such as liveness and continuity---that are
orthogonal to the main topic of the paper. We also do not include types,
allocation, garbage collection, etc.  This is the norm for work on relaxed
memory: The problem is already hard enough. % \emph{On the upside, we do allow
  % thread creation via $\!\!\PAR\!\!$, which is unusual.}

Except for roach motel (\ref{RelW}, \ref{AcqW}), we have not attempted to validate
transformations that involve synchronizations and fences
\cite{DBLP:conf/popl/VafeiadisBCMN15}.  For example, the model does not
support elimination of redundant synchronization accesses.  % Accommodating
% such optimizations will likely require changes to the definition of
% $\textsf{cover}$ (page \pageref{def:cover}) and $\textsf{read}$ (page
% \pageref{page:readsat}).
% \emph{On the upside, 
  % Corollary~\ref{seqcompleteness} is the first abstract characterization of
  % valid transformations for relaxed access.}



\labeltext{The}{page:logic:limits} logic we presented in \textsection\ref{sec:logic} is only strong enough
to prove a few examples.  \citet{DBLP:conf/esop/SvendsenPDLV18} presented a
different a logic, capable of showing that the following program cannot write
2:
\begin{align*}
  \taglabel{OOTA5}
    ( y\GETS x+1
    \PAR
    x\GETS y ) && \hbox{\begin{tikzinline}[node distance=1em]
        \event{rx}{\DR{x}{1}}{}
        \event{wy}{\DW{y}{2}}{right=of rx}
        \po{rx}{wy}
        \event{ry}{\DR{y}{1}}{right=3em of wy}
        \event{wx}{\DW{x}{1}}{right=of ry}
        \po{ry}{wx}
        % \rf{wy}{ry}
        \rf[out=170,in=10]{wx}{rx}
      \end{tikzinline}}
\end{align*}
The attempted execution is \emph{not} allowed by our semantics since there is no write
to fulfill $(\DR{y}{1})$.  Proving this requires the ability to reason about
values.

% \labeltext{The}{page:logic:limits} logic we presented in \textsection\ref{sec:logic} is only strong enough
% to prove a few examples.
As another example,
consider the following, from \citet[Fig.~3]{DBLP:journals/pacmpl/ChakrabortyV19}.
\begin{gather*}
  \taglabel{OOTA6}
  \begin{gathered}
  x\GETS 2\SEMI
  \IF{x\NOTEQ2}\THEN y\GETS 1 \FI
  \PAR
  x\GETS 1\SEMI
  r\GETS x\SEMI
  \IF{y}\THEN x\GETS 3 \FI
  \\\nonumber
  \hbox{\begin{tikzinline}[node distance=1em]
  \event{wx2}{\DW{x}{2}}{}
  \event{rx3}{\DR{x}{3}}{right=of wx2}
  \event{wy1}{\DW{y}{1}}{right=of rx3}
  \po{rx3}{wy1}
  \event{wx1}{\DW{x}{1}}{right=2em of wy1}
  \event{rx2}{\DR{x}{2}}{right=of wx1}
  \wk{wx1}{rx2}
  \event{ry1}{\DR{y}{1}}{right=of rx2}
  \event{wx3}{\DW{x}{3}}{right=of ry1}
  \po{ry1}{wx3}
  \wk[in=165,out=15]{rx2}{wx3}
  \rf[in=-170,out=-10]{wy1}{ry1}
  \rf[in=170,out=10]{wx2}{rx2}
  \rf[out=-170,in=-10]{wx3}{rx3}
  \wk[out=-170,in=-10]{wx1}{wx2}
  \wk{wx2}{rx3}
    \end{tikzinline}}
\end{gathered}
\end{gather*}
The attempted execution is \emph{not} allowed by our semantics, due to the
evident cycle.  Surprisingly, this outcome is allowed by the promising
semantics \cite{DBLP:conf/popl/KangHLVD17}.
\citeauthor{DBLP:journals/pacmpl/ChakrabortyV19} developed \weakestmo{} to
address this example.  Intuitively, it is not possible for the left thread to
read $3$ for $x$ when the right thread reads $2$.  Proving this may require a
logic with modalities to deal with intervening writes and coherence.
% \emph{On the upside, our logic is powerful enough to show that our semantics
%   disallows \oota{} executions of \ref{OOTA4} --- \weakestmo{} and the
%   promising semantics both permit \oota{} executions of \ref{OOTA4} (see \textsection\ref{sec:promising}).}

% Our results have not been formally verified, but only because we do not have
% the resources to do it ourselves.  It is worth noting that the current
% emphasis on formal verification has the unintended effect of making research
% more conservative: Large libraries are built, and large institutions with
% teams of graduate students build them.

Our model realizes \emph{multi-copy atomicity} (\mca).  Thus it will not
compile efficiently to non-\mca{} architectures, such as \ppc\ and \armseven.
To do so, one cannot include the order required by \ref{rf4} in pomset order.
It may be sufficient to use a weaker second order.  \citet{2019-sp} attempted
to define such a weaker order, but their model is anomalous on
\ref{MCA2}---as we discussed in \textsection\ref{sec:related}.  % \emph{On the
  % upside, ours is the first language-level model of \mca{}.  The use of
  % preconditions does not depend on \mca, so there is hope to extend our
  % approach to dependency calculations to non-\mca{} architectures.}

In the discussion of \ref{MCA1}, we noted that our model does not enforce
order between reads dues to address and control dependencies.  This has
implications for Java's final field semantics.
\begin{gather*}
  \begin{gathered}
  (r\GETS 1 \SEMI \REF{r}\GETS 0 \SEMI \REF{r}\GETS 1\SEMI  x^\modeRA\GETS r)
  \PAR
  (r\GETS x^\modeRA \SEMI s \GETS \REF{r})
  \\[-1ex]
  \hbox{\begin{tikzinline}[node distance=1em and 2em]
      \event{a1}{\DW{\REF{1}}{0}}{}
      \event{a2}{\DW{\REF{1}}{1}}{right=of a1}
      \wk{a1}{a2}
      \event{a4}{\DWRel{x}{1}}{right=of a2}
      \sync{a2}{a4}
      \event{b1}{\DR{x}{1}}{right=2em of a4}
      \event{b2}{\DR{\REF{1}}{0}}{right=of b1}
      \sync{b1}{b2}
      \rf{a4}{b1}
      \wk[out=170,in=10]{b2}{a2}
    \end{tikzinline}}
  \end{gathered}
\end{gather*}
If we changed $x^\modeRA$ to $x^\modeRLX$, then there would be no order from
$(\DR[\modeRA]{x}{1})$ to $(\DR{\REF{1}}{0})$, and the execution would be
allowed.  It may be desirable to distinguish address dependencies from other
dependencies---doing so would likely require two separate preconditions for
each event.
% .  without
% distinguishing address dependency from other things, or keeping thread ids in
% actions.  ARM requires that we do not order writes that only have a control
% dependency or that are fulfilled by an external writes.

% In the concurrent there is an inherent tension between ``simplicity'' and
% compiler optimizations.  Where these are in tension, we have opted to prefer
% the compiler.  In particular, we do not enforce address and control
% dependencies between reads.  This can cause counterintuitive outcomes, such
% as the following.

% Final fields example without address dependency:
% r=new C; r.f=1; Fence1; x=r   ||   s1=x; Fence2; if (s1) {s2=s1.f}
% r=100; [r]=1; Fence1; x=r     ||   s1=x; Fence2; if (s1) {s2=[s1]}
% without Fence2 we allow the outcome s1=100, s2=0

% \section{Conclusions}
% \label{sec:outro}

% We have defined a relaxed memory model, using standard tools from mathematics,
% and demonstrated that it satisfies the fundamental criterion for
% software memory models, namely ``the memory model must strike a balance
% between ease-of-use for programmers and implementation flexibility for system
% designers.''
% We have made two substantial contributions:

% In \textsection\ref{sec:logic}, we proposed a cyclical proof rule for
% parallel composition of temporal properties in the style of
% \citet{Abadi:1993:CS:151646.151649}.  This provides an objectively
% falsifiable alternative to the notion of \emph{out of thin air} (\oota)
% execution, which has been famously difficult to prove or disprove
% \cite{DBLP:conf/esop/BattyMNPS15,BoehmOOTA}.

% In \textsection\ref{sec:opt}, we formalized a connection between Hoare
% logic and the sequential fragment of our language, allowing us to make a
% general statement about transformations.  Thus, we present a solution to a
% problem that has been open since \citet{DBLP:conf/esop/CenciarelliKS07}
% discovered reordering of independent statements is invalid in the JMM
% \cite{Manson:2005:JMM:1047659.1040336}.

% These contributions rest on a semantics that supports a simple, executable
% prescription of \emph{essential dependencies}, calculated using logic and
% manifest in pomset order.  This provides a
% clear interface between the language user and the compiler writer, and a
% clear bridge from the compiler writer to computer architecture.

% We presented a denotational semantics using pomsets with preconditions.  The
% model may also be expressible in the operational framework of
% \citet{DBLP:conf/esop/FerreiraFS10}, which parameterizes a standard SC
% operational semantics with respect to a set of program transformations.




\endinput

\section{Conclusions}
We have defined a relaxed memory model, using standard tools from the
semantics and demonstrated that it satisfies the fundamental criterion for
software memory models, namely ``the memory model must strike a balance
between ease-of-use for programmers and implementation flexibility for system
designers.''

One key lesson we draw is that any software relaxed memory model should
support the cyclical composition proof rule of
\citet{Abadi:1993:CS:151646.151649} for temporal properties.  We posit that all
the benefits of \oota, such as typing guarantees, can be derived from this
principle. 

We conclude with a suggestion for the consideration of programming language
designers.  The semantics of a programming language should support a clear
and executable prescription of dependency in a program.  Such a specification
provides a nice interface between the language user and the compiler writer,
and a clear bridge from the compiler writer to architectures.

Our current paper is to be viewed as a step towards this goal.  We will
explore developing an executable specification based on these ideas in future
work.
 


\endinput
\section{SCRAPS}

Some \oota\ examples can be proven if we extend the logic with value
quantification---we leave the details to future work.  Consider
\cite[(\textsc{rng})]{DBLP:conf/esop/SvendsenPDLV18}:
\begin{gather*}
  % \VAR x\GETS0\SEMI \VAR y\GETS0\SEMI \VAR z\GETS0\SEMI %\\[-.5ex]
  \begin{aligned}
    ( y\GETS x+1
    \PAR
    x\GETS y ) && \hbox{\begin{tikzinline}[node distance=1em]
        \event{rx}{\DR{x}{1}}{}
        \event{wy}{\DW{y}{2}}{right=of rx}
        \po{rx}{wy}
        \event{ry}{\DR{y}{1}}{right=3em of wy}
        \event{wx}{\DW{x}{1}}{right=of ry}
        \po{ry}{wx}
        % \rf{wy}{ry}
        \rf[out=170,in=10]{wx}{rx}
      \end{tikzinline}}
  \end{aligned}
  \\
  \hbox{$
    \begin{gathered}
      [\forall u,v.\;(\once\DW{y}{u} \land \once\DW{y}{v}) \Rightarrow ((u \EQ v) \vee (u \EQ 0) \vee (v \EQ 0))]
      \\[-.5ex]
      \land [\DW{x}{1} \Rightarrow \once\DR{y}{1}]
      \land [\DW{y}{2} \Rightarrow \once\DR{x}{1}]
      % \land [\DW{z}{2} \Rightarrow \once\DR{y}{2}]
    \end{gathered}$}
\end{gather*}
% \begin{displaymath}
%   \VAR x\GETS0\SEMI \VAR y\GETS0\SEMI \VAR z\GETS0\SEMI  %\\[-.5ex]
%   (
%     y\GETS x+1\SEMI z \GETS y
%   \PAR
%     x\GETS y
%     %   )
% \end{displaymath}
% \begin{scope}
%   \small
%   \begin{gather*}
%     [(\once\DW{y}{v_1} \land \once\DW{y}{v_2}) \Rightarrow ((v_1 = v_2) \vee (v_1 = 0) \vee (v_2 = 0))] \land\\[-.5ex]
%     [\once\DW{x}{1} \Rightarrow \once\DR{y}{1}] \land [\once\DW{y}{2}
%     \Rightarrow \once\DR{x}{1}] \land [\once\DW{z}{2} \Rightarrow
%     \once\DR{y}{2}]
%   \end{gather*}
% \end{scope}
The attempted execution is disallowed since there is no write to fulfill
$\DR{y}{1}$.  Closing $x$ and $y$ and using the last two conjuncts in the
invariant above:
% last conjunct, we have
% \begin{math}
%   % \once\DW{z}{2} \Rightarrow \once\DR{y}{2}
%   % \once\DW{z}{2} \Rightarrow \once\DW{y}{2}
%   \DW{y}{2} \Rightarrow \once\DR{x}{1}.
% \end{math}
% Closing $x$, we then have
\begin{math}
  % \once\DW{z}{2} \Rightarrow \once\DW{x}{1}
  \DW{y}{2} \Rightarrow \once\DW{y}{1}.
\end{math}
From the first conjunct, we then derive
\begin{math}
  \DW{y}{2} \Rightarrow \FALSE,
\end{math}
as required.

Other \oota\ examples may require that we extend the logic to deal with
intervening writes or coherence. The following outcome is allowed by the
promising semantics \cite{DBLP:conf/popl/KangHLVD17}, but not in \weakestmo{}
\cite[Fig.~3]{DBLP:journals/pacmpl/ChakrabortyV19} nor in our semantics, due
to the cycle:
\begin{gather*}
  \label{weakestmo}
  x\GETS 2\SEMI
  \IF{x\NOTEQ2}\THEN y\GETS 1 \FI
  \PAR
  x\GETS 1\SEMI
  r\GETS x\SEMI
  \IF{y}\THEN x\GETS 3 \FI
  \\\nonumber
  \hbox{\begin{tikzinline}[node distance=1em]
  \event{wx2}{\DW{x}{2}}{}
  \event{rx3}{\DR{x}{3}}{right=of wx2}
  \event{wy1}{\DW{y}{1}}{right=of rx3}
  \po{rx3}{wy1}
  \event{wx1}{\DW{x}{1}}{right=2em of wy1}
  \event{rx2}{\DR{x}{2}}{right=of wx1}
  \wk{wx1}{rx2}
  \event{ry1}{\DR{y}{1}}{right=of rx2}
  \event{wx3}{\DW{x}{3}}{right=of ry1}
  \po{ry1}{wx3}
  \wk[in=165,out=15]{rx2}{wx3}
  \rf[in=-170,out=-10]{wy1}{ry1}
  \rf[in=170,out=10]{wx2}{rx2}
  \rf[out=-170,in=-10]{wx3}{rx3}
  \wk[out=-170,in=-10]{wx1}{wx2}
  \wk{wx2}{rx3}
    \end{tikzinline}}
\end{gather*}


\subsection{RFUB}
Let $\aCmd$ be the right thread in \ref{rfub}.
\begin{align*}
  \aCmd = r\GETS y\SEMI \IF{r \NOTEQ 1} \THEN z\GETS 1\SEMI r\GETS 1\FI  \SEMI x\GETS r 
\end{align*}
Then, $\hoare{\TRUE}{\aCmd}{x=1} $ and $\hoare{y \neq 1}{\bCmd}{z=1}$.  Thus, the traditional sequential semantics {\em does not} attribute any cause for the write of $1$ to $x$, as seen by the execution:
\begin{tikzdisplay}[node distance=1em]
 \event{ry}{\DR{y}{1}}{}
 \event{wx}{\DW{x}{1}}{right=of ry}
 \event{wa}{\DW{z}{1}}{below=of rx}
 \rf{ry}{wa}
\end{tikzdisplay}
Consider the
following, where all locations are initialized to $0$.  In \textsection\ref{sec:logic} we provide machinery to prove that \oota{} is
incapable of writing $1$.  The question is whether this should be possible
for \rfub, which changes \oota{} only to include a \emph{Register assignment
  From an Unexecuted Branch} \cite{BoehmOOTA}:?  
\begin{align*}
  \label{oota}  \tag{\textsc{oota}}
  y\GETS x
  \PAR&
  r\GETS y\SEMI
  x\GETS r
  \\
  \tag{\rfub1}
  \label{rfub}
  y\GETS x
  \PAR&
  r\GETS y\SEMI
  \IF{r \NOTEQ 1} \THEN z\GETS 1\SEMI r\GETS 1\FI  \SEMI x\GETS r 
\end{align*}
%and the following \emph{Out Of Thin Air} litmus test:
The change in the second
thread from \oota{} to \rfub{} is not a valid refinement under Hoare logic:
\rfub{} validates the triple $\hoare{\TRUE}{\aCmd}{x=1}$, but \oota{} does
not.  As a result, it is expected that \rfub{} may have additional behaviors.

The program writes $1$ to $x$ in both branches of the conditional.  Further,
the writes to $z$ and $x$ in the then-branch of the conditional are
independent.  Therefore, it is sensible for a compiler to hoist the write to
$x$ out of the conditional.  In our semantics, there is no dependency from
the read of $y$ to the write to $x$
The execution in question is:
\begin{displaymathsmall}
  \begin{tikzcenter}[node distance=1em]
    \event{a1}{\DR{x}{1}}{}
    \event{a2}{\DW{y}{1}}{right=of a1}
    \po{a1}{a2}
  \end{tikzcenter}
  \Bigm\|
  (\DR{y}{1})
  \prefix
  \left(
    \begin{tikzcenter}[node distance=.5em]
      \event{b1}{r\NOTEQ1\mid\DW{z}{1}}{}
      \event{b2}{\DW{x}{1}}{right=of b1}
    \end{tikzcenter}
    % \begin{tikzcenter}[node distance=.5em]
    %   \event{b1}{r\NOTEQ1\mid\DW{z}{1}}{}
    %   \event{b2}{r\NOTEQ1\mid\DW{x}{1}}{right=of b1}
    % \end{tikzcenter}
    % \biggm\|
    % \begin{tikzcenter}[node distance=1em]
    %   \event{c1}{r\EQ1\mid\DW{x}{1}}{}
    % \end{tikzcenter}
  \right)
\end{displaymathsmall}
With an internal read of $y$, we have:
\begin{displaymathsmall}
  \begin{tikzcenter}[node distance=1em]
    \event{a1}{\DR{x}{1}}{}
    \event{a2}{\DW{y}{1}}{right=of a1}
    \po{a1}{a2}
  \end{tikzcenter}
  \Bigm\|
    \begin{tikzcenter}[node distance=1em]
      \event{b1}{y\NOTEQ1\mid\DW{z}{1}}{}
      \event{b2}{\DW{x}{1}}{right=of b1}
    \end{tikzcenter}
\end{displaymathsmall}
but the precondition $y\NOTEQ1$ cannot be satisfied locally.
With an external read of $(\DR{y}{1})$, we can discharge the precondition, but in this
case, the predicate becomes $1\NOTEQ 1$. 
\begin{displaymathsmall}
  \begin{tikzcenter}[node distance=1em]
    \event{a1}{\DR{x}{1}}{}
    \event{a2}{\DW{y}{1}}{right=of a1}
    \po{a1}{a2}
  \end{tikzcenter}
  \Bigm\|  
    \begin{tikzcenter}[node distance=1em]
      \nonevent{b1}{1\NOTEQ1\mid\DW{z}{1}}{}
      \event{b2}{\DW{x}{1}}{right=of b1}
      \event{b0}{\DR{y}{1}}{left=of b1}
      \po{b0}{b1}
    \end{tikzcenter}
\end{displaymathsmall}


