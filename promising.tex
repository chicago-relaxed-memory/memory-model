\section{The Promising Semantics Allows OOTA1/OOTA4}
\label{sec:promising}

We have confirmed that
\citeauthor{DBLP:journals/toplas/Lochbihler13}'s
[\citeyear{DBLP:journals/toplas/Lochbihler13}], \ref{OOTA1}, is allowed by
the promising semantics.  Because the promising semantics does not have
object creation, we used the variant \ref{OOTA4}.

On Aril 3, 2018, we sent \ref{OOTA4} to \citet{DBLP:conf/esop/SvendsenPDLV18}.
We present the example here exactly as it was in our correspondence:
\begin{verbatim}
Thread s: 
  a:=x         // a==1
  y:=a
Thread t: 
  b:=z         // b==1
  if (b==1)
    c:=y       // c==1
    x:=c
  else
    x:=1
Thread u: 
  z:=1
\end{verbatim}

The transitions of the threads are as follows:
\begin{verbatim}
s0 --R(x,v)--> s1 --W(y,v)--> s2
t0 --R(z,0)--> t1 --W(x,1)--> t2
t0 --R(z,1)--> t3 --R(y,v)--> t4 --W(x,v)--> t5
u0 --W(z,1)--> u1
\end{verbatim}

To get the result, first execute \texttt{u} to get message \texttt{<z:1@1>}.
Then \texttt{t} promises \texttt{<x:1@1>}, which it can fulfill by reading \texttt{b/z==0}.
Then execute \texttt{s} to get message \texttt{<y:1@1>}.
Then execute \texttt{t}, reading \texttt{b/z==1} and \texttt{c/y==1} and fulfill the promise by writing \texttt{<x:1@1>}.

On Aril 3, 2018, \citeauthor{DBLP:conf/esop/SvendsenPDLV18} replied:
\begin{quotation}
  \emph{Yes, it is allowed in the way you described.}
\end{quotation}
On March 17, 2020, the authors confirmed that this did not change in Promising 2.0.

We sent the same example to \citet{DBLP:journals/pacmpl/ChakrabortyV19}.
On December 10, 2018, they replied:
\begin{quotation}
  \emph{Yes the following outcome is allowed in our semantics. The event structure and the extracted execution are in pic-2 (attached).}
\end{quotation}