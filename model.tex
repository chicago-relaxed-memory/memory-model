\section{The Model,  Informally}
\label{sec:model:intro}

\begin{comment}
https://preshing.com/20131125/acquire-and-release-fences-dont-work-the-way-youd-expect/

Cannot encode R/A actions with actions+fences...

A release operation prevents preceding memory operations from being delayed
past it (a;Rel =/=> Rel;a)
 
A release fence prevents preceding memory operations from being delayed past
subsequent writes (a;FR;w =/=> w;a;FR)

An acquire operation prevents subsequent memory operations from being advanced
before it (Acq;a =/=> a;Acq)

An acquire fence prevents subsequent memory operations from being advanced
before prior reads (r;FA;a =/=> FA;a;r)

https://www.modernescpp.com/index.php/fences-as-memory-barriers

StoreLoad: Full fence allows a store before to be reordered with respect to a
load after (wx;F;ry) ===> (ry;F;wx)

StoreLoad+LoadLoad: Release fence also allows (rx;FR;ry) ===> (ry;FR;rx)

StoreLoad+StoreStore: Acquire fence also allows (wx;FR;wy) ===> (wy;FR;wx)

LoadStore: No fence allows a prior load to reorder w.r.t. a subsequent store
(rx;FR;wy) =/=> (wy;FR;rx)

https://preshing.com/20120710/memory-barriers-are-like-source-control-operations/
Good news is that a fullFence does it.

Bizarrely, it seems this is not supported in C++... You have to go to assembly.
\end{comment}

DRF: we use weaker def of hb than normal and so have more races.
(in particular, Rel(x)1 ; Wx2 || Acq(x)2 is a race in our system)
(also we ignore fences for the purposes of defining a race)

OPT:  optimization requires that perform operation that removes
unsatisfiable preconditions.

\begin{definition}
  \label{def:3valued}
  A \emph{3-valued pomset} with alphabet $\Alphabet$ is tuple $(\Event,
  {\le}, {\gtN}, \labeling)$, such that 
  \begin{itemize}
  \item $\Event$ is a set of \emph{states},
  %\item $\ESub\subseteq\Event$ is a set of \emph{accepting states}, 
  \item $\labeling: \Event \fun \Alphabet$ is a \emph{labeling},
  \item ${\le} \subseteq (\Event\times\Event)$ is a partial order, and
    % \begin{enumerate}
    % \item $\aEv \le \aEv$,
    % \item if $\bEv \le \aEv$ and $\aEv \le \bEv$ then $\bEv = \aEv$,
    %   \\(this follows from 5a and 5b)
    % \item if $\cEv \le \bEv \le \aEv$ then $\cEv \le \aEv$, and
    % \end{enumerate}
  \item ${\gtN} \subseteq (\Event\times\Event)$ such that:
    \begin{itemize}
    \item\label{5a} if $\bEv \le \aEv$ then $\bEv \gtN \aEv$, \hfill (Inclusion)
    \item\label{5b} if $\bEv \le \aEv$ and $\aEv \gtN \bEv$ then $\bEv = \aEv$,  \hfill (Consistency)
    \item\label{5c} if $\cEv \le \bEv \gtN \aEv$ or $\cEv \gtN \bEv \le \aEv$ then $\cEv \gtN \aEv$.  \hfill (Semi-transitivity)
    \end{itemize}
\end{itemize}
\end{definition}
Note that $\gtN$ is reflexive.

In the next section, we define \emph{3-valued pomsets with preconditions},
which we refer to simply as \emph{pomsets}.  In this section we give an
informal introduction to these pomsets and to the semantics of programs as
sets of pomsets.

We fix the alphabet of pomsets to be $\Alphabet=(\Formulae\times\Act)$, where
$\Formulae$ is a set of logical formula and $\Act$ is a set of actions.  We
write pairs in $(\Formulae\times\Act)$ as $(\aForm \mid \aAct)$.  We elide
$\aForm$ when it is a tautology.

Actions have the form $(\DR{\aLoc}{\aVal})$, which reads value $\aVal$
from $\aLoc$, $(\DRAcq{\aLoc}{\aVal})$, which is an acquire that reads
$\aVal$ from $\aLoc$, $(\DW{\aLoc}{\aVal})$, which writes $\aVal$ to $\aLoc$,
and $(\DWRel{\aLoc}{\aVal})$, which is a release that writes $\aVal$ to
$\aLoc$.

Formulae are \emph{open}, in that occurrences of register names and memory
locations are subject to substitutions of the form $\aForm[\aLoc/\aReg]$ and
$\aForm[\bExp/\aLoc]$, where $\aLoc$ is a memory location, $\aReg$ is a
register and $\bExp$ is an memory-location-free expression.  Actions are not
subject to substitution.

\citet{2019-sp} model microarchitecture using pomsets in order to capture
security flaws, such as Spectre \cite{DBLP:journals/corr/abs-1801-01203}.  In
this paper, we study the extent to which pomsets can serve as a model of
relaxed memory architectures.  As a result, our use of pomsets is more
abstract than that of \citeauthor{2019-sp}.  For example, at top-level, we
ignore events whose preconditions are not tautologies; \citeauthor{2019-sp}
use such events to model the influence of ``unexecuted'' conditional code on
executed code, as found in Spectre.

% Two basic concepts derived from pomsets are \emph{coherence} and \emph{reads-from}.
% \begin{definition}
%   A pomset \emph{coherent} if, when restricted to events that read or write
%   any single location $\aLoc$, $\gtN$ forms a partial order.
%   % As we shall see below, for top-level pomsets we could equivalently require
%   % that $\gtN$ forms a total order.
% \end{definition}

\begin{definition}
  \label{def:rf}
  We say that $\aEv$ \emph{reads $\aLoc$ from} $\bEv$ if $\aEv$ reads $\aVal$
  from $\aLoc$, $\bEv$ writes $\aVal$ to $\aLoc$, 
  \begin{itemize}
  \item $\bEv \lt \aEv$, and
  \item if an event $\cEv$ writes to $\aLoc$ then either $\cEv \gtN \bEv$ or $\aEv \gtN \cEv$.
  \end{itemize}    
\end{definition}


We visualize a pomset as a graph where the nodes are drawn from $\Event$,
each node $\aEv$ is labeled with $\labeling(\aEv)$, and an edge
$\bEv \rightarrow \aEv$ corresponds to an ordering $\bEv\le\aEv$.  We
visualize $(\bEv \gtN \aEv)$ as a dashed arrow from $\bEv$ to $\aEv$.
% For example:
% \begin{tikzdisplay}[node distance=1em]
%   \event{rx1}{a}{}
%   \event{wy0}{b}{below right=of rx1}
%   \event{wy1}{c}{above right=of wy0}
%   \po{rx1}{wy0}
%   \po{rx1}{wy1}
%   \wk{wy0}{wy1}
% \end{tikzdisplay}
% is a visualization of the pomset where:
% \[\begin{array}{c}
%     E = \{ 0,1,2 \}
%     \quad
%     {\labeling} = \{(0,a),\,(1,b),\,(2,c)\}
%     \\
%     {\le} = \{(0,1),\,(0,2)\}\cup\{(0,0),\,(1,1),\,(2,2)\}
%     \quad
%     {\gtN} = {\le}\cup\{(2,3)\}
% \end{array}\]
% for example:
For example, the semantics of
\begin{math}
  x\GETS 0\SEMI
  x\GETS 1
  \PAR
  \IF{x}\THEN y\GETS 1\FI
\end{math}
includes the pomset:
\begin{tikzdisplay}[node distance=1em]
  \event{wx0}{\DW{x}{0}}{}
  \event{wx1}{\DW{x}{1}}{right=of wx0}
  \event{rx1}{\DR{x}{1}}{right=2.5 em of wx1}
  \event{wy1}{\DW{y}{1}}{right=of rx1}
  \rf{wx1}{rx1}
  \wk{wx0}{wx1}
  \po{rx1}{wy1}
\end{tikzdisplay}
We refer to edges introduced by $(\bEv \le \aEv)$ as \emph{strong edges} and
by $(\bEv \gtN \aEv)$ as \emph{weak edges}.
Reads-from is included in strong order, and we often highlight it in green, as above.


Need to forbid cycles in $\gtN$ per location:
\[
  \IF{x\EQ1}\THEN r\GETS x \FI
  \PAR
  x\GETS 1
  \PAR
  x\GETS2
  \PAR
  \IF{x\EQ2}\THEN s\GETS x \FI
\]
which includes the execution:
\begin{tikzdisplay}[node distance=1em]
  \event{wx1}{\DW{x}{1}}{}
  \event{wx2}{\DW{x}{2}}{below=of wx1}
  \event{rx1a}{\DR{x}{1}}{left=of wx1}
  \event{rx2a}{\DR{x}{2}}{left=of wx2}
  \event{rx1b}{\DR{x}{1}}{right=of wx1}
  \event{rx2b}{\DR{x}{2}}{right=of wx2}
  \po{rx1a}{rx2a}
  \po{rx2b}{rx1b}
  \rf{wx1}{rx1a}
  \rf{wx1}{rx1b}
  \rf{wx2}{rx2a}
  \rf{wx2}{rx2b}
  \wk{rx1a}{wx2}
  \wk{rx2a}{wx1}
  \wk{rx1b}{wx2}
  \wk{rx2b}{wx1}
\end{tikzdisplay}
This satisfies the requirements for $x$-closure, but is not coherent.

With the restrictions in $x$-closure, forbidding cycles forces an order
between any writes that are read from:
\[
  x\GETS 3
  \PAR
  \IF{x\EQ1}\THEN r\GETS x \FI
  \PAR
  x\GETS 1
  \PAR
  x\GETS2
\]
\begin{tikzdisplay}[node distance=1em]
  \event{wx1}{\DW{x}{1}}{}
  \event{wx2}{\DW{x}{2}}{below=of wx1}
  \event{rx1a}{\DR{x}{1}}{left=of wx1}
  \event{rx2a}{\DR{x}{2}}{left=of wx2}
  \po{rx1a}{rx2a}
  \rf{wx1}{rx1a}
  \rf{wx2}{rx2a}
  \wk{rx1a}{wx2}
  \wk{wx1}{wx2}
  \event{wx3}{\DW{x}{3}}{below left=-.2em and 1em of rx1a}
  \wk{rx1a}{wx3}
  \wk{rx2a}{wx3}
\end{tikzdisplay}


Across variables, however, cycles in $\gtN$ arise naturally in non-multicopy
atomic examples, such as IRIW.
\[\begin{array}{rl}
  &x\GETS0\SEMI x\GETS 1
  \PAR
  \IF{x}\THEN r\GETS y \FI
 \\{}
  \PAR&
  y\GETS0\SEMI y\GETS 1
  \PAR
  \IF{y}\THEN s\GETS x \FI
\end{array}\]
which includes the execution:
\begin{tikzdisplay}[node distance=1em]
  \event{wx0}{\DW{x}{0}}{}
  \event{wx1}{\DW{x}{1}}{right=of wx0}
  \event{wy0}{\DW{y}{0}}{below=4ex of wx0}
  \event{wy1}{\DW{y}{1}}{right=of wy0}
  \event{ry1}{\DR{y}{1}}{right=2.5em of wy1}
  \event{rx0}{\DR{x}{0}}{right=of ry1}
  \event{rx1}{\DR{x}{1}}{right=2.5 em of wx1}
  \event{ry0}{\DR{y}{0}}{right=of rx1}
  \wk{wx0}{wx1}
  \wk{wy0}{wy1}
  \rf{wx1}{rx1}
  \rf[bend left]{wy0}{ry0}
  \rf{wy1}{ry1}
  \rf[bend right]{wx0}{rx0}
  \wk{rx0}{wx1}
  \wk{ry0}{wy1}
  %\po{rx1}{ry0}
  %\po{ry1}{rx0}
\end{tikzdisplay}
Note that if you enforce the dependency between the reads, then the execution
is disallowed, because the is a cycle on weak edges restricted to $x$:
\begin{math}
  (\DW{x}{1})\le (\DR{y}{0}) \gtN (\DW{y}{1})
\end{math}
therefore
\begin{math}
  (\DW{x}{1}) \gtN  (\DW{y}{1});
\end{math}
then
\begin{math}
  (\DW{y}{1})\le (\DR{y}{0}) \gtN (\DW{y}{1})
\end{math}
therefore
\begin{math}
  (\DW{x}{1}) \gtN  (\DW{y}{1});
\end{math}
The model ensures that events are ordered before a release and
after an acquire.
As an example, consider the program:
\[
  x\GETS0\SEMI f\GETS0\SEMI x\GETS 1\SEMI f \REL\GETS1 \PAR r\GETS f\ACQ; s\GETS x
\]
This has a top-level execution:
\begin{tikzdisplay}[node distance=1em]
  \event{wx0}{\DW{x}{0}}{}
  \event{wf0}{\DW{f}{0}}{below=of wx0}
  \event{wx1}{\DW{x}{1}}{right=of wx0}
  \event{wf1}{\DWRel{f}{1}}{right=of wf0}
  \event{rf1}{\DRAcq{f}{1}}{right=2.5em of wf1}
  \event{rx1}{\DR{x}{1}}{above=of rf1}
  \po{wx0}{wf1}
  \po{wf0}{wf1}
  \po{wx1}{wf1}
  \po{rf1}{rx1}
  \rf{wf1}{rf1}
  \rf{wx1}{rx1}
  \wk{wx0}{wx1}
\end{tikzdisplay}
but \emph{not}:
\begin{tikzdisplay}[node distance=1em]
  \event{wx0}{\DW{x}{0}}{}
  \event{wf0}{\DW{f}{0}}{below=of wx0}
  \event{wx1}{\DW{x}{1}}{right=of wx0}
  \event{wf1}{\DWRel{f}{1}}{right=of wf0}
  \event{rf1}{\DRAcq{f}{1}}{right=2.5em of wf1}
  \event{rx0}{\DR{x}{0}}{above=of rf1}
  \po{wx0}{wf1}
  \po{wf0}{wf1}
  \po{wx1}{wf1}
  \po{rf1}{rx1}
  \rf{wf1}{rf1}
  \rf[bend left]{wx0}{rx0}
  \wk{wx0}{wx1}
\end{tikzdisplay}
since $(\DW x0) \gtN (\DW x1) \lt (\DR x0)$, so this pomset does not satisfy the
requirements to be $x$-closed.
If we replace the release
with a plain write, then the outcome $(\DRAcq f1)$ and $(\DR x0)$ is possible:
\begin{tikzdisplay}[node distance=1em]
  \event{wx0}{\DW{x}{0}}{}
  \event{wf0}{\DW{f}{0}}{below=of wx0}
  \event{wx1}{\DW{x}{1}}{right=of wx0}
  \event{wf1}{\DW{f}{1}}{right=of wf0}
  \event{rf1}{\DRAcq{f}{1}}{right=2.5em of wf1}
  \event{rx0}{\DR{x}{0}}{above=of rf1}
  \wk{wf0}{wf1}
  \po{rf1}{rx0}
  \rf{wf1}{rf1}
  \rf[bend left]{wx0}{rx0}
  \wk{wx0}{wx1}
\end{tikzdisplay}
since no order is required between $(\DW x1)$ and $(\DW f1)$.  
Symmetrically, if we replace the acquire of the original program
with a plain read, then the outcome $(\DR f1)$ and $(\DR x0)$ is possible.

\section{The Model,  Formally}
\label{sec:model}

Differences with \cite{2019-sp}:
\begin{itemize}
\item We only consider pomsets where $\lt$-greater elements have stronger
  preconditions (Definition~\ref{def:3pre}).
\item We only consider pomsets that are coherent (Definition~\ref{def:3pre}).
\item Prefixing only introduces required order from read to write in item~\ref{pre-implies} (Definition~\ref{def:prefix}).
\item We provide semantics for computed addresses of memory locations.
\item The semantics of relaxed read internal actions.
\item restriction introduces internal actions.
\item The semantics of parallel composition is asymmetric.  This ensures that
  in $\aCmd;(\bCmd_1\PAR\bCmd_2)$, $\bCmd_1$ may read internally from $\aCmd$,
  but all reads in $\bCmd_2$ must be explicit.
\end{itemize}


\subsection{Background: 3-valued pomsets}
\label{sec:pomsets}

Structures similar to 3-valued pomsets have come up in many guises, for example
rough sets~\cite{Pawlak1982} or ultrametrics over
$\{0,{}^1\!/_2,1\}$. They correspond to axioms A1--A3 of Lamport's
\emph{system executions}~\cite{DBLP:journals/dc/Lamport86}.
They are the notion of pomset given by interpreting
$\bEv\le\aEv$ in a 3-valued logic~\cite{Urquhart1986}. 




\subsection{Data models}
\label{sec:preliminaries}

A \emph{data model} consists of:
\begin{itemize}
\item a set of \emph{values} $\Val$, ranged over by
  $\aVal$, $\bVal$, $\dVal$ and $\cVal$,
\item a set of \emph{registers} $\Reg$, ranged over by
  $\aReg$ and $\bReg$,
\item a set of \emph{expressions} $\Exp$, ranged over by
  $\aExp$, $\bExp$, $\cExp$ and $\dExp$,
\item a set of \emph{memory locations} $\Loc$, ranged over by $\aLoc$ and
  $\bLoc$, 
\item a set of \emph{logical formulae} $\Formulae$, ranged over by
  $\aForm$ and $\bForm$, and
\item a set of \emph{actions} $\Act$, ranged over by $\aAct$ and $\bAct$.
\end{itemize}

Let $\aSub$ range over substitutions of the form
$\aForm[\aLoc/\aReg]$ or $\aForm[\bExp/\aLoc]$.

We require that data models satisfy the following:
\begin{itemize}
\item the sets of values, registers and memory locations are disjoint,
\item values include at least the constants $0$ and $1$,
\item expressions include at least registers and values,
\item memory locations have the form $\REF{\aVal}$,
\item expressions do \emph{not} include memory locations or the operator $\REF{\aExp}$,
\item formulae include at least $\TRUE$, $\FALSE$, and equalities of the form
  $(\aExp=\aVal)$ and $(\REF{\aExp}=\aLoc)$,
\item formulae are closed under negation, conjunction, disjunction, and
  substitution\footnote{Since formulae are closed under substitutions of the
    form $\aForm[\aLoc/\aReg]$, they must include equalities of the form
    $(\aEExp=\aVal)$ and $(\REF{\aEExp}=\aLoc)$, where $\aEExp$ is an
    \emph{extended expression} that includes memory locations.  We elide the
    details.  By composition of the closure conditions, formulae must also be
    closed under that substitutions of the form
    $\aForm[\aExp/\aReg]=\aForm[\aLoc/\aReg][\aExp/\aLoc]$.}, and
\item there is a relation $\vDash$ between formulae.
\end{itemize}

We say that $\aForm$ is \emph{independent of $\aLoc$} whenever
$\aForm \vDash \aForm[\aVal/\aLoc] \vDash \aForm$ for every $\aVal$, and that
$\aForm$ is \emph{dependent on $\aLoc$} otherwise.  We say that $\aForm$ is
\emph{location independent} if it is independent of every location.

We say that $\aForm$ \emph{implies} $\bForm$ whenever $\aForm\vDash\bForm$,
that $\aForm$ is a \emph{tautology} whenever $\TRUE\vDash\aForm$, that
$\aForm$ is \emph{unsatisfiable} whenever $\aForm\vDash\FALSE$.

For the actions of a data model, we require that
\begin{itemize}
\item there are partial functions $\rreads$ and
  $\rwrites: \Act \fun (\Loc \times \Val)$,
\item there are sets $\Rel$ and $\Acq \subseteq\Act$, and
\item there is a function $\finternalize: (\Loc\times\Act) \fun \Act$ that
  satisfies the restrictions given below.
\end{itemize}

We say that $\aAct$ \emph{reads} $\aVal$ \emph{from} $\aLoc$ whenever
$\rreads(\aAct) = (\aLoc,\aVal)$, and that $\aAct$ \emph{writes} $\aVal$
\emph{to} $\aLoc$ whenever $\rwrites(\aAct) = (\aLoc,\aVal)$.  Two actions
\emph{conflict} if at least one action writes a location and the other either
reads or writes the same location.  Actions that read or write are
\emph{external}, other actions are \emph{internal}.
% Actions in
% $\Ext=\fdom(\rreads)\cup\fdom(\rwrites)$ are \emph{external}, whereas those
% in $\Int=\Act\setminus\Ext$ are \emph{internal}.

We say that $\aAct$ is an \emph{acquire} if $\aAct\in\Acq$, and that $\aAct$
is a \emph{release} if $\aAct\in\Rel$.  Actions that acquire or release are
\emph{synchronizations}, other actions are \emph{relaxed}.
% We say that $\aAct$ is a
% \emph{synchronization} if it is either a release or an acquire.

We require that $\finternalize$ satisfy the following:
\begin{itemize}
\item the codomain of $\finternalize$ includes only internal actions, %$\fcodom(\finternalize)\subseteq\Int$,
\item $\finternalize(\aAct)$ is an acquire exactly when $\aAct$ is an acquire, and 
\item $\finternalize(\aAct)$ is a release exactly when $\aAct$ is a release.
\end{itemize}

As noted in \textsection\ref{sec:model:intro}, our example language includes acquiring
read $(\DRAcq{\aLoc}{\aVal})$, relaxed read $(\DR{\aLoc}{\aVal})$, releasing
write $(\DWRel{\aLoc}{\aVal})$, and relaxed write $(\DW{\aLoc}{\aVal})$.
For each external action, we also define a corresponding internal action
% which replaces the letter $\mathsf{R}$ or $\mathsf{W}$ with $\tau$.
denoted by prefixing $\tau$.
For example, $(\iDRAcq{\aLoc}{\aVal})$ is an acquiring internal action, which
neither reads nor writes. In pictures, we draw internal actions grayed out,
rather than using $\tau$.  For example, the ``read'' action is internal in:
\begin{tikzdisplay}[node distance=1em]
  \event{wx1}{\DW{x}{1}}{}
  \internal{rx1}{\DR{x}{1}}{below right=of rx1}
  \event{wy1}{\DW{y}{1}}{above right=of wy0}
  \po{wx1}{wy1}
\end{tikzdisplay}


% In examples, we use fence actions of the form $(\DF{\aF})$, where the annotation
% indicates that the fence is a release ($\FR$), an acquire ($\FA$) or both ($\FF$):
% \begin{displaymath}
%   \aF\BNFDEF\FR\BNFSEP\FA\BNFSEP\FF
% \end{displaymath}

\subsection{3-valued pomsets with preconditions}

Fix an alphabet $\Alphabet=(\Formulae\times\Act)$.
Define %$\labelingForm$ and $\labelingAct$ so that
$\labelingForm(\aEv)=\aForm$ and $\labelingAct(\aEv)=\aAct$ whenever
$\labelingForm(\aEv)=(\aForm\mid\aAct)$.

We lift terminology from logical formulae and actions to events. For example,
we say that $\aEv$ is unsatisfiable when $\labelingForm(\aEv)$ is unsatisfiable,
and that $\aEv$ is an acquire when $\labelingAct(\aEv)$ is an acquire.



In this paper, we are not investigating microarchitecture.  So we make the
global assumption formulae can only get stronger in dependent actions:
\begin{definition}
  \label{def:3pre}
  A \emph{Memory model pomset} is a 3-valued pomset such that
  \begin{itemize}
  \item $\gtN$ is acyclic, and
  \item where $\labelingForm(\aEv)$ implies $\labelingForm(\bEv)$ whenever
    $\bEv\le\aEv$.
  \end{itemize}
\end{definition}
In the remainder of the paper, we refer to memory model pomsets simply as
\emph{pomsets}.

At top-level, we expect the use of each variable to be \emph{closed}.
\begin{definition}
\label{def:x-closed}
  A pomset is \emph{$\aLoc$-closed} if, for every $\aEv\in\Event$:
  \begin{itemize}
  \item $\aEv$ is independent of $\aLoc$, and
  \item if $\aEv$ reads from $\aLoc$, then there is some $\bEv$ such that
    $\aEv$ reads $\aLoc$ from $\bEv$.
  \end{itemize}
  A pomset is \emph{top-level} if it is $\aLoc$-closed for every $\aLoc$.
\end{definition}

We give the semantics of programs as sets of pomsets.  Each pomset
$\aPS\in\sem{\aCmd}$ will represent a single execution of $\aCmd$.  We do not
expect $\sem{\aCmd}$ to be prefixed closed; thus, one may view each
$\aPS\in\sem{\aCmd}$ as a \emph{completed} execution\footnote{NOTE: because
  implication closed, any event can go false, and we kill everything after
  it, so that means we do get a kind of prefix closure.}.  However, we do
expect the sets of pomsets given by the semantics to be closed with respect
to \emph{isomorphism}, \emph{augmentation} and \emph{implication}.
\begin{definition}
  $\aPS'$ is an \emph{isomorphism} of $\aPS$ if there is a bijection
  $f:\Event\fun\Event'$ such that $\labeling(\aEv)=\labeling'(f(\aEv))$,
  $\aEv\le\bEv$ iff $f(\aEv)\le'f(\bEv)$, and $\aEv\gtN\bEv$ iff
  $f(\aEv)\gtN'f(\bEv)$.

  $\aPS'$ is an \emph{augmentation} of $\aPS$ if $\Event'=\Event$,
  ${\labeling'}={\labeling}$, ${\le'}\supseteq{\le}$, and
  ${\gtN'}\supseteq{\gtN}$.

  $\aPS'$ \emph{implies} $\aPS$ if $\Event'=\Event$, ${\le'}={\le}$,
  ${\gtN'}={\gtN}$, $\labelingAct'=\labelingAct$, and $\labelingForm'(\aEv)$
  implies $\labelingForm(\aEv)$ for all $\aEv\in\Event$.
\end{definition}
% \begin{definition}
%   $\aPS'$ is an \emph{augmentation} of $\aPS$ if $\Event'=\Event$, ${\labeling'}={\labeling}$,
%   ${\le}\subseteq{\le'}$, %$\aEv\le\bEv$ implies $\aEv\le'\bEv$,
%   and ${\gtN}\subseteq{\gtN'}$. %$\aEv\gtN\bEv$ implies $\aEv\gtN'\bEv$,
%   % $\labelingAct'=\labelingAct$, and  $\labelingForm'(\aEv)$ implies $\labelingForm(\aEv)$.
%   % if $\labeling(\aEv) = (\bForm \mid \bAct)$ then $\labeling'(\aEv) = (\bForm' \mid \bAct)$ where $\bForm'$ implies $\bForm$.
% \end{definition}


% Restriction also filters a set of pomsets; we have
% $(\nu\aLoc\st\aPSS)\subseteq\aPSS$.
% The definition requires that we define
% when a read is possible.

% \begin{definition}\label{def:rf}
%   In a pomset, $\aEv$ \emph{can read $\aLoc$ from} $\bEv$ whenever: 
%   \begin{itemize}
%   \item $\bEv \lt \aEv$,  
%   \item $\aEv$ implies $\bEv$,
%   \item $\bEv$ writes $\aVal$ to $\aLoc$,
%     and $\aEv$ reads $\aVal$ from $\aLoc$, and
%   \item if $\cEv$ writes to $\aLoc$
%     then either $\cEv \gtN \bEv$ or $\aEv \gtN \cEv$.
%   \end{itemize}
% \end{definition}

\subsection{Combinators}

We give the semantics using combinators over sets of pomsets, defined below.
Using $\aPSS$ to range over sets of pomsets, these are:
\begin{itemize}
\item \emph{substitution} $\aPSS\aSub$, which applies the substitution to
  every precondition,
\item \emph{restriction} $\nu\aLoc\st\aPSS$, which internalizes $\aLoc$ for
  pomsets that are $\aLoc$-closed,
% \item \emph{guarding} $\aForm\guard\aPSS$, which filters $\aPSS$,
%   keeping pomsets where all events imply $\aForm$,
% \item \emph{independency filtering} $\Loc\guard\aPSS$, which filters
%   $\aPSS$, keeping pomsets all events are independent of every location,
% \item \emph{write filtering} $\DW{\aLoc}{}\guard\aPSS$, which filters
%   $\aPSS$, keeping pomsets that have an initial write to $\aLoc$,
\item \emph{composition} $\aPSS^1\parallel\aPSS^2$, which unions pomsets, allowing events to be merged, and
\item \emph{prefixing} $\aAct\prefix\aPSS$, which adds an event with action
  $\aAct$ to pomsets in $\aPSS$, ordering $\aAct$ before any $\aEv$ whose predicate
  depends on the value read by $\aAct$.
\end{itemize}
These operations are similar to those from models of concurrency such
as~\cite{Brookes:1984:TCS:828.833}, but adapted here to the setting of
speculative evaluation.

We also define two filtering operations:
\begin{itemize}
\item \emph{guarding} $\aForm\guard\aPSS$, which
  keeps pomsets where all preconditions imply $\aForm$, and
\item \emph{independency filtering} $\Loc\guard\aPSS$, which keeps pomsets
  where all preconditions are location independent.
% \item \emph{write filtering} $\DW{\aLoc}{}\guard\aPSS$, which keeps pomsets
%   that have an initial write to $\aLoc$,
\end{itemize}

%% A write generates a write event that may be visible
%% to other threads.  A read may see a
%% thread-local value, or it may generate a read event that must be justified by
%% another thread.  In the latter case, occurrences of $\aReg$ are replaced with
%% $\aLoc$ (rather than $\aVal$) to ensure that dependencies are tracked
%% properly.  The subsequent substitution of $\aVal$ for $\aLoc$ occurs in
%% Definition~\ref{def:prefix} of prefixing.

% We have completed the formal definition of our model of speculative
% evaluation, and now turn to examples.

%\subsubsection{Substitution and Guarding} 

% Substitution updates the preconditions in a pomset, thus we expect the number
% of pomsets to be unchanged; in addition, the number of events in each of the
% pomsets is unchanged.

% Guarding and restriction filter a set of pomsets; we have
% $(\aForm\guard\aPSS)\subseteq\aPSS$ and
% $(\nu\aLoc\st\aPSS)\subseteq\aPSS$.

The definitions of substitution, restriction and the filtering operations %guarding and write filtering
are straightforward\footnote{We have chosen the definition of restriction for
  its simplicity.  It is worth noting, however, that our definition does not
  support renaming of variables.  In particular
  $(\nu\aLoc\st\aPSS\parallel(\nu\aLoc\st\bPSS))$ is generally not the same
  as $(\nu\aLoc\st\aPSS\parallel(\nu\bLoc\st\bPSS[\bLoc/\aLoc]))$.  To
  support renaming, $(\nu\aLoc\st\aPSS)$ would need to either remove or
  relabel events that mention $\aLoc$.}:
\begin{definition}
  %For a substitution $\aSub$, of the form $[\aLoc/\aReg]$ or $[\bExp/\aLoc]$,
  Let $\aPSS\aSub$ be the set $\aPSS'$ where $\aPS'\in\aPSS'$ whenever
there is $\aPS\in\aPSS$ such that:
$\Event' = \Event$,
${\le'} = {\le}$, 
${\gtN'} = {\gtN}$,
and
$\labeling'(\aEv) = (\bForm\aSub \mid \aAct)$ when $\labeling(\aEv) = (\bForm \mid \aAct)$.
% \begin{itemize}
% \item if $\labeling(\aEv) = (\bForm \mid \aAct)$ then $\labeling'(\aEv) =
%   (\bForm\aSub \mid \aAct)$, and
% \item if $\labeling(\aEv) = (\bForm \mid \aSub)$ then $\labeling'(\aEv) = (\bForm\bSub \mid \aSub\bSub)$.
%\end{itemize}

  % Let $(\nu\aLoc\st\aPSS)$ be the subset of $\aPSS$ such that $\aPS\in\aPSS$ whenever
  % and $\aPS$ is $\aLoc$-coherent and $\aLoc$-closed.

Let $(\nu\aLoc\st\aPSS)$ be the set $\aPSS'$ where $\aPS'\in\aPSS'$ whenever
there is $\aPS\in\aPSS$ such that $\aPS$ is $\aLoc$-closed and:
$\Event' = \Event$,
${\le'} = {\le}$, 
${\gtN'} = {\gtN}$,
and
$\labeling'(\aEv) = (\bForm \mid \finternalize(\aAct))$ when $\labeling(\aEv) = (\bForm \mid \aAct)$.

Let $(\aForm \guard \aPSS)$ be the subset of $\aPSS$ such that $\aPS\in\aPSS$ whenever
$\aForm$ implies $\labelingForm(\aEv)$, for every $\aEv\in\Event$. % if $\labelingAct(\aEv)$ writes.
% \begin{itemize}
% \item if $\labeling(\aEv) = (\bForm \mid \aActSub)$ then $\aForm$ implies $\bForm$.
% \end{itemize}

Let $(\Loc\guard \aPSS)$ be the subset of $\aPSS$ such that
$\aPS\in\aPSS$ whenever $\labelingForm(\aEv)$ is location independent, for every $\aEv\in\Event$.
% Let $(\DW{\aLoc}{} \guard \aPSS)$ be the subset of $\aPSS$ such that
% $\aPS\in\aPSS$ whenever there is some $\bEv$ that writes $\aLoc$ such that
% $\bEv\le\aEv$, for every release $\aEv\in\Event$.
\end{definition}
% Note that this liberalization allows reads more flexibility.  This is
% desirable in the language and architectural models, but not necessarily in
% microarchitectural models where reads are visible.


% \subsubsection{Restriction}
% \label{sec:restriction}


% We say that $\aPS' = \aPS\restrict{\Event'}$ when 
%  $\Event' \subseteq \Event$,
%  ${\labeling'} = {\labeling}\restrict{\Event'}$, 
%  ${\le'} = {\le}\restrict{\Event'}$, and
%  ${\gtN'} = {\gtN}\restrict{\Event'}$.

% \begin{definition}
%   Let $(\nu\aLoc\st\aPSS)$ be the subset of $\aPSS$ such that $\aPS\in\aPSS$ whenever
%   and $\aPS$ is $\aLoc$-coherent and $\aLoc$-closed.
%   % Let $(\nu\aLoc\st\aPSS)$ be the set $\aPSS'$ where $\aPS'\in\aPSS'$
%   % whenever there is $\aPS\in\aPSS$ such that $\aPS' = \aPS\restrict{\Event'}$
%   % and $\aPS'$ is $\aLoc$-coherent and $\aLoc$-closed.
%   % Let $(\nu\aLoc\st\aPSS)$ be the subset of $\aPSS$ such that $\aPS\in\aPSS$ whenever
%   % \begin{itemize}
%   % \item $\aEv$ is independent of $\aLoc$, and
%   % \item if $\aEv$ reads $\aLoc$, then there is some $\bEv$ such that $\aEv$ can read $\aLoc$ from $\bEv$.
%   % \end{itemize}
% \end{definition}
%This definition throws away useless writes.

%\subsubsection{Composition}
\begin{definition}
Let $\aPS' \in (\aPSS^1 \parallel \aPSS^2)$
whenever there are $\aPS^1 \in \aPSS^1$ and $\aPS^2 \in \aPSS^2$ such that:
\begin{itemize}
\item $\Event' = \Event^1 \cup \Event^2$,
\item ${\le'}\supseteq{\le^1}\cup{\le^2}$, %if $\aEv \le^1 \bEv$ or $\aEv \le^2 \bEv$ then $\aEv \le' \bEv$,
\item ${\gtN'}\supseteq{\gtN^1}\cup{\gtN^2}$, and %if $\aEv \gtN^1 \bEv$ or $\aEv \gtN^2 \bEv$ then $\aEv \gtN' \bEv$,
% \item if $\labeling'(\aEv) = (\aForm' \mid \aAct)$ then either:
%   \begin{itemize}
%   \item $\labeling^1(\aEv) = (\aForm^1 \mid \aAct)$ and $\labeling^2(\aEv) = (\aForm^2 \mid \aAct)$
%     and $\aForm'$ implies $\aForm^1 \lor \aForm^2$,
%   \item $\labeling^1(\aEv) = (\aForm^1 \mid \aAct)$ and $\aEv \not\in \Event^2$
%     and $\aForm'$ implies $\aForm^1$, or
%   \item $\labeling^2(\aEv) = (\aForm^2 \mid \aAct)$ and $\aEv \not\in \Event^1$
%     and $\aForm'$ implies $\aForm^2$.
%   \end{itemize}
\item either
  % \begin{gather*}
  %   \labelingAct'(\aEv) = \labelingAct^1(\aEv) = \labelingAct^2(\aEv) \textand \labelingForm'(\aEv) \textimplies \labelingForm^1(\aEv) \lor \labelingForm^2(\aEv),\\
  %   \aEv \not\in \Event^2,\; \labelingAct'(\aEv) = \labelingAct^1(\aEv) \textand \labelingForm'(\aEv) \textimplies \labelingForm^1(\aEv),\; \textor\\    
  %   \aEv \not\in \Event^1,\; \labelingAct'(\aEv) = \labelingAct^2(\aEv) \textand \labelingForm'(\aEv) \textimplies \labelingForm^2(\aEv).
  % \end{gather*}
  \begin{itemize}
  \item $\labelingAct'(\aEv) = \labelingAct^1(\aEv) = \labelingAct^2(\aEv)
    \textand \labelingForm'(\aEv) \textimplies \labelingForm^1(\aEv) \lor \labelingForm^2(\aEv)$,
  \item $\labelingAct'(\aEv) = \labelingAct^1(\aEv),\;\; \aEv \not\in \Event^2\,
    \textand \labelingForm'(\aEv) \textimplies \labelingForm^1(\aEv),\; \textor$
  \item $\labelingAct'(\aEv) = \labelingAct^2(\aEv),\;\; \aEv \not\in \Event^1\,
    \textand \labelingForm'(\aEv) \textimplies \labelingForm^2(\aEv)$.
  \end{itemize}
\end{itemize}
\end{definition}
Composition is used in giving the semantics for conditionals and concurrency.
$\aPSS^1 \parallel \aPSS^2$ contains the union of pomsets from $\aPSS^1$ and
$\aPSS^2$, allowing overlap as long as they agree on actions. For example, if
$\aPSS^1$ and $\aPSS^2$ contain:
\begin{tikzdisplay}[node distance=1em]
  \event{a}{\aForm \mid \aAct}{}
  \event{b}{\bForm^1 \mid \bAct}{right=of a}
  \po{a}{b}
  \event{b2}{\bForm^2 \mid \bAct}{right=6em of b}
  \event{c2}{\cForm \mid \cAct}{right=of b2}
  \wk{b2}{c2}
\end{tikzdisplay}
then $\aPSS^1 \parallel \aPSS^2$ contains:
\begin{tikzdisplay}[node distance=1em]
  \event{a}{\aForm \mid \aAct}{}
  \event{b}{\bForm^1 \lor \bForm^2 \mid \bAct}{right=of a}
  \event{c}{\cForm \mid \cAct}{right=of b}
  \po{a}{b}
  \wk{b}{c}
\end{tikzdisplay}

% We use $\aPSS^1 \parallel \aPSS^2$ in defining the semantics of conditionals
% and concurrency.
% It contains the union of pomsets from $\aPSS^1$ and $\aPSS^2$,
% allowing overlap as long as they agree on actions. For example, if
% $\aPSS^1$ and $\aPSS^2$ contain:
% \begin{tikzdisplay}[node distance=1em]
%   \event{a}{\aForm \mid \aAct}{}
%   \event{b}{\bForm^1 \mid \bAct}{right=of a}
%   \po{a}{b}
% \end{tikzpicture}\qquad\qquad\begin{tikzpicture}[node distance=1em]
%   \event{b}{\bForm^2 \mid \bAct}{}
%   \event{c}{\cForm \mid \cAct}{right=of b}
%   \wk{b}{c}
% \end{tikzdisplay}
% then $\aPSS^1 \parallel \aPSS^2$ contains:
% \begin{tikzdisplay}[node distance=1em]
%   \event{a}{\aForm \mid \aAct}{}
%   \event{b}{\bForm^1 \lor \bForm^2 \mid \bAct}{right=of a}
%   \event{c}{\cForm \mid \cAct}{right=of b}
%   \po{a}{b}
%   \wk{b}{c}
% \end{tikzdisplay}


%\subsubsection{Prefixing}
\begin{definition}
  \label{def:prefix}
Let $\aAct \prefix \aPSS$ be the set $\aPSS'$ where $\aPS'\in\aPSS'$ whenever
there is $\aPS\in\aPSS$ such that:
\begin{enumerate}
\item\label{pre-E} $\Event' = \Event \cup \{\cEv\}$,
\item\label{pre-le} ${\le'}\supseteq{\le}$, % if $\bEv \le \aEv$ then $\bEv \le' \aEv$,
\item\label{pre-gtN} ${\gtN'}\supseteq{\gtN}$, %if $\aEv \gtN \bEv$ then $\aEv \gtN' \bEv$,
% \item $\labelingAct'(\cEv) = \aAct$, 
% \item if $\labeling(\aEv) = (\bForm \mid \bAct)$ then $\labeling'(\aEv) =
%   (\bForm' \mid \bAct)$, where:
%   \begin{itemize}
%   \item if $\aAct$ is an acquire then $\bForm'$ is independent of every $\bLoc$,
%   \item if $\aAct$ does not read then $\bForm'$ implies $\bForm$,
%   \item if $\aAct$ reads then $\aVal$ from $\aLoc$ then
%     \begin{itemize}
%     \item $\bForm'$ implies $\bForm[\aVal/\aLoc]$, and
%     \item either $\bForm'$ implies $\bForm$ or $\cEv\lt'\aEv$, 
%     \end{itemize}
%   \end{itemize}
% \item if $\labelingAct(\aEv) = \bAct$ then:
%   \begin{itemize}
%   \item if $\aAct$ is an acquire or $\bAct$ is a release then $\cEv \lt' \aEv$, 
%   \item if $\aAct$ and $\bAct$ both touch the same location and one is a write,
%     then $\cEv \gtN' \aEv$, and
%   \end{itemize}
\item\label{pre-act} $\labelingAct'(\cEv) = \aAct$ and $\labelingAct'(\aEv) = \labelingAct(\aEv)$,
% \item\label{pre-implies} either $\labelingForm'(\aEv)$ implies $\labelingForm(\aEv)$ or
%   $\aAct$ is a read and if $\aEv$ is a write then $\cEv\lt'\aEv$,
% \item\label{pre-read} if $\aAct$ reads $\aVal$ from $\aLoc$ then
%    $\labelingForm'(\aEv)$ implies $\labelingForm(\aEv)[\aVal/\aLoc]$, %
\item\label{pre-write} if $\aAct$ is not a read, then $\labelingForm'(\aEv)$
  implies $\labelingForm(\aEv)$,
\item\label{pre-read} if $\aAct$ reads $\aVal$ from $\aLoc$ then
  \begin{enumerate}
  \item[(\ref{pre-read}a)] $\labelingForm'(\aEv)$ implies $\labelingForm(\aEv)[\aVal/\aLoc]$, and
  \item[(\ref{pre-read}b)] if $\aEv$ is a write then either $\cEv\lt'\aEv$
    or $\labelingForm'(\aEv)$ implies $\labelingForm(\aEv)$,
  \end{enumerate}
% \item if $\aAct$ does not read then $\labelingForm'(\aEv)$ implies $\labelingForm(\aEv)$,
% \item if $\aAct$ reads then either $\labelingForm'(\aEv)$ implies $\labelingForm(\aEv)$ or $\cEv\lt'\aEv$,  %
\item\label{pre-coherence} if $\aAct$ is a write that conflicts with $\labelingAct(\aEv)$ 
    then $\cEv \gtN' \aEv$,
\item\label{pre-sync} if $\aAct$ is an acquire or $\labelingAct(\aEv)$ is a release then $\cEv \lt' \aEv$, and
\item\label{pre-acquire} if $\aAct$ is an acquire then $\labelingForm(\aEv)$ is location independent.
% \item if $\aAct$ is a read but not a synchronization then either
%   $\labelingForm'(\cEv)$ is unsatisfiable or there is some $\aEv$ such
%   that $\labelingForm'(\aEv)$ does not imply $\labelingForm(\aEv)$.
% \item if $\labeling(\aEv) = (\bForm \mid \bAct)$ then $\labeling'(\aEv) =
%   (\bForm' \mid \bAct)$, where:
%   \begin{itemize}
%   \item if $\aAct$ is an acquire or $\bAct$ is a release then $\cEv \lt' \aEv$, 
%   \item if $\aAct$ is an acquire then $\bForm$ is independent of every $\bLoc$,
%   \item if $\aAct$ and $\bAct$ both touch the same location and one is a write,
%     then $\cEv \gtN' \aEv$, and
%   \item $\bForm'$ implies \(\left\{\begin{array}{l@{~}ll}
%     % \bForm[\aVal/\aLoc]                     & \mbox{if $\aAct$ reads $\aVal$ from $\aLoc$ and $\cEv\lt'\aEv$} & \textsc{[dependent read]} \\
%     % \bForm[\aVal/\aLoc] \text{ and } \bForm & \mbox{if $\aAct$ reads $\aVal$ from $\aLoc$}                  & \textsc{[independent read]} \\
%     % \bForm                                  & \mbox{otherwise}                                              & \textsc{[non-read]} \\        
%     \bForm[\aVal/\aLoc]                     \\\quad \mbox{if $\aAct$ reads $\aVal$ from $\aLoc$ and $\cEv\lt'\aEv$} \\\qquad \textsc{[dependent read]} \\[\jot]
%     \bForm[\aVal/\aLoc] \text{ and } \bForm \\\quad \mbox{if $\aAct$ reads $\aVal$ from $\aLoc$}                  \\\qquad \textsc{[independent read]} \\[\jot]
%     \bForm                                  \\\quad \mbox{otherwise}                                              \\\qquad \textsc{[non-read]} \\
%   \end{array}\right.\)
%   \end{itemize}
\end{enumerate}
\end{definition}
% The last condition ensures that useless reads are not included.
% Otherwise, $\labelingForm'(\cEv)$ is unconstrained.

% In order to keep augmentation closure, we need to keep the unsatisfiable
% elements in the set of pomsets.

Item~\ref{pre-read}b allows one to choose a weaker precondition for $\aEv$ in
$\aPS'$ when $\aAct$ is a read.  In this case, item~\ref{pre-read}a ensures
that the precondition is not \emph{too} weak.  In the case that one chooses
the weaker precondition, item~\ref{pre-read}b requires that there be an order
from the new event to $\aEv$ when $\aEv$ is a write.  No additional order is
required when $\aEv$ is a pure read or an internal action\footnote{The
  absence of order between reads is required for our compilation result,
  since the IMM does not enforce order between reads of a single thread.}.

``Similar to syntactic dependencies, \textsc{rfi} edges are guaranteed to be preserved
only on dependency paths from a read to a write, not otherwise.''
Our language does not preserve these dependencies, since compiler
optimizations may be able to remove them.
% \begin{displaymath}
%   r \GETS x \SEMI
%   \REF{r}\GETS 1\SEMI 
%   s\GETS \REF{r}\SEMI
%   y\GETS{s}
% \end{displaymath}
% In the calculation of the pomset, the precondition on $\DW{y}{1}$ goes from
% $s\EQ1$ to $\REF{r}\EQ1$


$\aAct\prefix\aPSS$ adds a new event $\cEv$ with action $\aAct$ to each
pomset in $\aPSS$.  As in the definition of parallel composition, the
definition allows the new event to overlap with events in $\aPSS$ as long as
they agree on the action.  Overlapping of synchronization events is
disallowed by item~\ref{pre-sync}.

If $\cEv$ writes to a location that is also written by some $\aEv$ in $\aPSS$,
item~\ref{pre-coherence} introduces weak order between them: $\cEv \gtN \aEv$.  This
ensures that these writes cannot be given the reverse order in an augmentation.

If $\cEv$ reads from a location that occurs in the predicate of $\aEv$, then
prefixing introduces order from $\cEv$ to $\aEv$.
whose predicate depends on $\aLoc$. 
For example, if $\aPSS$ contains %a pomset with only
\begin{tikzinline}[node distance=1em]
  \event{b}{y=1 \mid \DW{x}{1}}{}
  \event{c}{x>0 \mid \DW{z}{1}}{right=of b}
\end{tikzinline}
then $(\DR{x}{1})\prefix\aPSS$ contains:
\begin{displaymathsmall}
\begin{tikzcenter}[node distance=1em]
  \event{a}{\DR{x}{1}}{}
  \event{b}{y=1 \mid \DW{x}{1}}{above right=0em and 2em of a}
  \event{c}{1>0 \mid \DW{z}{1}}{below right=0em and 2em of a}
  \po{a}{c}
  \wk{a}{b}
\end{tikzcenter}
\qquad\text{and}\qquad
\begin{tikzcenter}[node distance=1em]
  \event{a2}{\DR{x}{1}}{right=4em of a}
  \event{b2}{y=1 \mid \DW{x}{1}}{above right=0em and 2em of a2}
  \event{c2}{x>0 \mid \DW{z}{1}}{below right=0em and 2em of a2}
  \wk{a2}{b2}
\end{tikzcenter}
\end{displaymathsmall}
In order to weaken the predicate on $(\DW{z}{1})$, item~\ref{pre-implies}
requires that we include the order from $(\DR{x}{1})$ to $(\DW{z}{1})$.
If the precondition on $(\DW{z}{1})$ in $\aPSS$ was $x<0$, then, by
item~\ref{pre-read}, all preconditions for $(\DW{z}{1})$ in
$(\DR{x}{1})\prefix\aPSS$ must be equivalent to $\FALSE$, regardless of
the ordering of the events.

% For example, if $\aAct$ and $\bAct$ write to the same location, $\aAct$ reads
% $\aVal$ from $\aLoc$, $\bForm$ is independent of $\aLoc$, and $\aPSS$
% contains:
% $\footnotesize\begin{tikzpicture}[baselinecenter,node distance=1em]
%   \event{b}{\bForm \mid \bAct}{}
%   \event{c}{\cForm \mid \cAct}{right=of b}
% \end{tikzpicture}$
% then $\aAct\prefix\aPSS$ contains:
% \begin{tikzdisplay}[node distance=1em]
%   \event{a}{\aForm \mid \aAct}{}
%   \event{b}{\bForm \mid \bAct}{right=of a}
%   \event{c}{\cForm[\vec\aVal/\vec\aLoc] \mid \cAct}{right=of b}
%   \po[out=25,in=155]{a}{c}
%   \wk{a}{b}
%   \po{b}{c}
% \end{tikzdisplay}

% We say $\aEv$ \emph{depends on} $\cEv$ if
% $\labeling(\aEv) = (\bForm \mid \dontcare)$,
% $\labeling(\cEv) = (\dontcare \mid \aSub)$,
% and $\bForm$ depends on $\aSub$.

% We say $\aEv$ \emph{conflicts with}  $\bEv$ if
% $\labeling(\aEv) = (\dontcare \mid \aAct)$,
% $\labeling(\cEv) = (\dontcare \mid \bAct)$,
% $\aAct$ and $\bAct$ touch the same location, and either
% $\aAct$ or $\bAct$ is a write.

Item~\ref{pre-acquire} filters the executions of $\aPSS$, ensuring that
thread-local reads do not cross acquire fences.  This prevents bad executions
like the following:
\begin{verbatim}
   x=1; rel; acq; if (x) {y=1};  ||  acq; x=0; rel; 
\end{verbatim}
where the second thread is interleaved between the rel and acq of the first.
In item~\ref{pre-acquire}, we do not require that $\bForm'$ is independent of
every $\bLoc$; were we to require this, the definition would not be augment closed.

Item~\ref{pre-sync} ensures that events are ordered before a release and
after an acquire.

We end this section with the following lemma, which is immediate from the definitions.
\begin{lemma}
  \label{lem:monotone}
  All combinators are monotone with respect to subset order.  For example, 
  $\aAct\prefix\aPSS \subseteq \aAct\prefix\aPSS'$ whenever
  $\aPSS\subseteq\aPSS'$.
%   Suppose
%   $\aPSS'\supseteq\aPSS$, $\aPSS_1'\supseteq\aPSS_1$ and
%   $\aPSS_2'\supseteq\aPSS_2$.  Then we have the following.
% \begin{itemize}
% \item $\aPSS'\aSub \supseteq \aPSS\aSub$,
% \item $\nu\aLoc\st\aPSS' \supseteq \nu\aLoc\st\aPSS$,
% \item $\aForm\guard\aPSS' \supseteq \aForm\guard\aPSS$,
% \item $\Loc\guard\aPSS' \supseteq \Loc\guard\aPSS$,
% \item $\DW{\aLoc}{}\guard\aPSS' \supseteq \DW{\aLoc}{}\guard\aPSS$,
% \item $\aPSS'_1\parallel\aPSS'_2 \supseteq \aPSS_1\parallel\aPSS_2$, and
% \item $\aAct\prefix\aPSS' \supseteq \aAct\prefix\aPSS$.
% \end{itemize}
\end{lemma}

\subsection{Semantics of programs}
\label{sec:semantics}

We consider a simple shared-memory concurrent language, with statements
defined as follows.

\begin{comment}
\footnote{We only consider executions where register state is empty in
  forked threads.  Given item~\ref{pre-acquire} of
  Definition~\ref{def:prefix}, a sufficient condition is that parallel
  composition is always preceded by an acquire fence, as in programs of the
  form:
  \begin{displaymath}
    \VAR\vec{\aLoc}\SEMI
    \vec{\aLoc}\GETS\vec{0}\SEMI
    \vec{\bLoc}\GETS\vec{0}\SEMI
    \FENCE\SEMI
    (\aCmd^1 \PAR \cdots \PAR \aCmd^n)
  \end{displaymath}
  where $\aCmd^1$, \ldots, $\aCmd^n$ do not include $\PAR$.  To avoid clutter
  in drawings, we often drop the explicit fence.}.
\end{comment}

% \begin{align*}
% \aCmd,\,\bCmd
% \BNFDEF& \SKIP \tag{No Operation}
% \\[-1ex]\BNFSEP& \FENCE\SEMI \aCmd \tag{Full fence}
% \\[-1ex]\BNFSEP& \REF{\cExp}\GETS\aExp\SEMI \aCmd \tag{Relaxed write to memory}
% \\[-1ex]\BNFSEP& \REF{\cExp}\REL\GETS\aExp\SEMI \aCmd \tag{Releasing write to memory}
% \\[-1ex]\BNFSEP& \aReg\GETS\REF{\cExp}\SEMI \aCmd \tag{Relaxed read from memory}
% \\[-1ex]\BNFSEP& \aReg\GETS\REF{\cExp}\ACQ\SEMI \aCmd
% \\[-1ex]\BNFSEP& \IF{\aExp} \THEN \aCmd \ELSE \bCmd \FI
% \\[-1ex]\BNFSEP& \aCmd \PAR \bCmd
% \\[-1ex]\BNFSEP& \VAR\aLoc\SEMI \aCmd
% \end{align*}
\begin{align*}
\aCmd,\,\bCmd
\BNFDEF& \SKIP
\BNFSEP \FENCE\SEMI \aCmd
\BNFSEP \aReg\GETS\aExp\SEMI \aCmd
\BNFSEP \aReg\GETS\REF{\cExp}\ACQ\SEMI \aCmd 
\BNFSEP \aReg\GETS\REF{\cExp}\SEMI \aCmd
\BNFSEP \REF{\cExp}\REL\GETS\aExp\SEMI \aCmd
\BNFSEP \REF{\cExp}\GETS\aExp\SEMI \aCmd
\\
\BNFSEP& \IF{\aExp} \THEN \aCmd \ELSE \bCmd \FI
\BNFSEP \aCmd \PAR \bCmd
\BNFSEP \VAR\aLoc\SEMI \aCmd
\end{align*}
We use common syntax sugar, such as \emph{extended expressions}, which include
memory locations.  For example, if the extended expression $\aEExp$ includes
a single occurrence of $\aLoc$, then $\bLoc\GETS\aEExp\SEMI \aCmd$ is
shorthand for $\aReg\GETS\aLoc\SEMI\bLoc\GETS\aEExp[\aReg/\aLoc]\SEMI \aCmd$.
Each occurrence of $\aLoc$ in an extended expression corresponds to an
independent read.  We also write
$\IF{\aExp} \THEN \aCmd^1 \ELSE \aCmd^2 \FI \SEMI \bCmd$ as shorthand for
$\IF{\aExp} \THEN \aCmd^1 \SEMI \bCmd\ELSE \aCmd^2 \SEMI \bCmd\FI$ and
$\VAR\aLoc\GETS\aExp\SEMI \aCmd$ as shorthand for
$\VAR\aLoc\SEMI\aLoc\GETS\aExp\SEMI \aCmd$.
% We write $\REL\aLoc\GETS\aExp\SEMI \aCmd$ as shorthand for $\FENCE_{\FR} \SEMI\aLoc\GETS\aExp\SEMI \aCmd$.
% We write $\ACQ\aReg\GETS\aLoc\SEMI \aCmd$ as shorthand for $\aReg\GETS\aLoc\SEMI \FENCE_{\FA}\SEMI \aCmd$.

% In Figure~\ref{fig:programs}, we give the semantics as sets of pomsets.  
% \begin{figure*}
The semantics of programs is as follows:
\allowdisplaybreaks
\begin{align*}
  \sem{\SKIP} & =
  \{ \emptyset \}
  \\
  \sem{\FENCE\SEMI \aCmd} & =
  (\DF{}) \prefix \sem{\aCmd}
  \\
  \sem{\aReg\GETS\aExp\SEMI \aCmd} & =
  \sem{\aCmd}[\aExp/\aReg] 
  \\  
  \sem{\aReg\GETS\REFAcq{\cExp}\SEMI \aCmd} & =
  \textstyle\bigcup_{\aLoc=\REF{\cVal}}\; ({\cExp}=\cVal) \guard \textstyle\bigcup_\aVal\; (\DRAcq\aLoc\aVal) \prefix \sem{\aCmd}[\aLoc/\aReg] 
  \\
  \sem{\aReg\GETS\REF{\cExp}\SEMI \aCmd} & =
  \textstyle\bigcup_{\aLoc=\REF{\cVal}}\; ({\cExp}=\cVal) \guard \textstyle\bigcup_\aVal\; (\DR\aLoc\aVal) \prefix \sem{\aCmd}[\aLoc/\aReg] 
  \\ & \mkern2mu\cup \textstyle\bigcup_{\aLoc=\REF{\cVal}}\; ({\cExp}=\cVal) \guard \mkern21mu\; (\iDR{\aLoc}{0}) \prefix \sem{\aCmd}[\aLoc/\aReg]
  \\
  \sem{\REFRel{\cExp}\GETS\aExp\SEMI \aCmd} & =
  \textstyle\bigcup_{\aLoc=\REF{\cVal}}\; ({\cExp}=\cVal) \guard \textstyle\bigcup_\aVal\;  (\aExp=\aVal) \guard\bigl((\DWRel\aLoc\aVal) \prefix \sem{\aCmd}\bigr)[\aExp/\aLoc] 
  \\
  \sem{\REF{\cExp}\GETS\aExp\SEMI \aCmd} & =
  \textstyle\bigcup_{\aLoc=\REF{\cVal}}\; ({\cExp}=\cVal) \guard \textstyle\bigcup_\aVal\;  (\aExp=\aVal) \guard\bigl((\DW\aLoc\aVal) \prefix \sem{\aCmd}\bigr)[\aExp/\aLoc]
  %\\ & \mkern2mu\cup \textstyle\bigcup_{\aLoc=\REF{\cVal}}\; ({\cExp}=\cVal) \guard \DW{\aLoc}{} \guard \sem{\aCmd}[\aExp/\aLoc]
  \\
  \sem{\IF{\aExp} \THEN \aCmd \ELSE \bCmd \FI} & =
  \bigl((\aExp \neq 0) \guard \sem{\aCmd}\bigr) \parallel \bigl((\aExp=0) \guard \sem{\bCmd}\bigr) 
  \\
  \sem{\aCmd \PAR \bCmd} & =
  \sem{\aCmd} \parallel \Loc \guard \sem{\bCmd} 
  \\
  \sem{\VAR\aLoc\SEMI \aCmd} & =
  \nu \aLoc \st \sem{\aCmd}
  % \sem{\aLoc\GETS\aExp\SEMI \aCmd} & = \textstyle\bigcup_\aVal\; \bigl((\aExp=\aVal) \guard (\DW\aLoc\aVal) \prefix \sem{\aCmd}\bigr)[\aExp/\aLoc] \\
  % \sem{\REL\aLoc\GETS\aExp\SEMI \aCmd} & = \textstyle\bigcup_\aVal\; \bigl((\aExp=\aVal) \guard (\DWRel\aLoc\aVal) \prefix \sem{\aCmd}\bigr)[\aExp/\aLoc] \\
  % \sem{\aReg\GETS\REF{\cExp}\SEMI \aCmd} & = \textstyle\bigcup_\aLoc\; (\REF{\cExp}=\aLoc) \guard (\sem{\aCmd}[\aLoc/\aReg] \cup \textstyle\bigcup_\aVal\; (\DR\aLoc\aVal) \prefix \sem{\aCmd}[\aLoc/\aReg]) \\
  % \sem{\aReg\GETS\aLoc\SEMI \aCmd} & =  \sem{\aCmd}[\aLoc/\aReg] \cup \textstyle\bigcup_\aVal\; (\DR\aLoc\aVal) \prefix \sem{\aCmd}[\aLoc/\aReg] \\
  % \sem{\ACQ\aReg\GETS\aLoc\SEMI \aCmd} & =  \textstyle\bigcup_\aVal\; (\DRAcq\aLoc\aVal) \prefix \sem{\aCmd}[\aLoc/\aReg] \\
% \caption{Semantics of a concurrent shared-memory language}
% \label{fig:programs}
% \end{figure*}
\end{align*}

The semantics of relaxed writes is the union of two sets.  The first set adds
a write action to each pomset in $\sem{\aCmd}$; the second does not.  In
discussion we refer to these respectively as \emph{explicit} and
\emph{implicit} writes.  The semantics of relaxed reads is similar.  We
collectively refer to these as explicit and implicit \emph{actions}.

Note that synchronizing actions are always explicit.

Note that the rule for write uses the substitution $[\aExp/\aLoc]$ with
precondition $\aExp=\aVal$, rather than using $[\aVal/\aLoc]$ directly.
To see the need for this, consider
$\sem{\IF{\bReg\EQ\aReg}\THEN \cLoc\GETS 1\FI}$,
which includes
\begin{tikzinline}[node distance=1em]
  \event{c}{\bReg=\aReg \mid \DW\cLoc1}{}
\end{tikzinline}.
Therefore
$\sem{\bReg\GETS\aLoc\SEMI\IF{\bReg\EQ\aReg}\THEN \cLoc\GETS 1\FI}$
includes
\begin{tikzinline}[node distance=1em]
  \event{c}{\aLoc=\aReg \mid \DW\cLoc1}{}
\end{tikzinline}
and
$\sem{\aLoc\GETS\aReg\SEMI\bReg\GETS\aLoc\SEMI\IF{\bReg\EQ\aReg}\THEN \cLoc\GETS 1\FI}$
includes
\begin{tikzinline}[node distance=1em]
  \event{c}{\aReg=\aReg \mid \DW\cLoc1}{}
\end{tikzinline}
which is independent of $\aReg$.
%
If we took the semantics of write to use $[\aVal/\aLoc]$, then we would end
up with pomsets of the form
\begin{tikzinline}[node distance=1em]
  \event{c}{\aVal=\aReg \mid \DW\cLoc1}{}
\end{tikzinline}
which depend on $\aReg$.

It is worth emphasizing that prefixing does not necessarily induce a
dependency, even for read actions where the read is used.  To see that this
is desirable, consider  the semantics of
$\bLoc\GETS0\SEMI\aReg\GETS\bLoc\SEMI\IF{\aReg\leq1}\THEN \aLoc\GETS 2\FI$.
To begin, not that 
$\sem{\IF{\aReg\leq1}\THEN \aLoc\GETS 2\FI}$ includes
\begin{tikzinline}[node distance=1em]
  \event{c}{\aReg\leq1 \mid \DW\aLoc2}{}
\end{tikzinline}
which depends on $\aReg$.
Then $\sem{\aReg\GETS\bLoc\SEMI\IF{\aReg\leq1}\THEN \aLoc\GETS 2\FI}$ includes
\begin{tikzdisplay}[node distance=1em]
    \event{b}{\DR\bLoc1}{}
    \event{c}{\bLoc\leq1 \mid \DW\aLoc2}{right=of b}
\end{tikzdisplay}
which has no order between the read and write.
By prefixing a write to $\bLoc$, $\sem{\bLoc\GETS0\SEMI\aReg\GETS\bLoc\SEMI\IF{\aReg\leq1}\THEN \aLoc\GETS
  2\FI}$ discharges the precondition of the write to $\aLoc$, giving
\begin{tikzinline}[node distance=1em]
    \event{a}{\DW\bLoc0}{}
    \event{b}{\DR\bLoc1}{right=of a}
    \event{c}{0\leq1 \mid \DW\aLoc2}{right=of b}
\end{tikzinline}
which is simply:
\begin{tikzdisplay}[node distance=1em]
    \event{a}{\DW\bLoc0}{}
    \event{b}{\DR\bLoc1}{right=of a}
    \event{c}{\DW\aLoc2}{right=of b}
\end{tikzdisplay}
Here the thread-local value of $\bLoc$ discharges the predicate.
Using the left-hand side of the read rule, the semantics of this program also includes
\begin{tikzinline}[node distance=1em]
    \event{a}{\DW\bLoc0}{}
    \event{c}{\DW\aLoc2}{right=of a}
\end{tikzinline}.

A variant which indicates the branch taken:
$\sem{\bLoc\GETS0\SEMI\aReg\GETS\bLoc\SEMI\IF{\aReg\leq1}\THEN
  \aLoc\GETS2\SEMI\cLoc\GETS\aReg\FI}$
includes
\begin{tikzdisplay}[node distance=1em]
    \event{a}{\DW\bLoc0}{}
    \event{b}{\DR\bLoc1}{right=of a}
    \event{c}{\DW\aLoc2}{right=of b}
    \event{d}{\DW\cLoc1}{right=of c}
    \po[bend left]{b}{d}
\end{tikzdisplay}
A program to that witnesses the independence of $\DR\bLoc1$ and $\DW\aLoc2$ is
\begin{math}
  \IF{\bLoc\EQ0}\THEN
    \IF{\aLoc\EQ2}\THEN
      \bLoc\GETS1\SEMI
      \IF{\cLoc\EQ1}\THEN\PASS\FI
    \FI
  \FI
\end{math}.
Putting these in parallel gives you:
\begin{tikzdisplay}[node distance=1em]
    \event{a}{\DW\bLoc0}{}
    \event{b}{\DR\bLoc1}{right=of a}
    \event{c}{\DW\aLoc2}{right=of b}
    \event{d}{\DW\cLoc1}{right=of c}
    \po[bend left]{b}{d}
    \event{a2}{\DR\bLoc0}{below=of a}
    \event{b2}{\DR\aLoc2}{right=of a2}
    \event{c2}{\DW\bLoc1}{right=of b2}
    \event{d2}{\DR\cLoc1}{right=of c2}
    \po{a2}{b2}
    \po{b2}{c2}
    \po[bend right]{b2}{d2}
    \rf{a}{a2}
    \rf{c}{b2}
    \rf{c2}{b}
    \rf{d}{d2}
\end{tikzdisplay}

% Local Variables:
% mode: latex
% TeX-master: "paper"
% End:
