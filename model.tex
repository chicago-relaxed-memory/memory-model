\section{An Introduction to the Model}
\label{sec:model:intro}

\begin{comment}
https://preshing.com/20131125/acquire-and-release-fences-dont-work-the-way-youd-expect/

Cannot encode R/A actions with actions+fences...

A release operation prevents preceding memory operations from being delayed
past it (a;Rel =/=> Rel;a)
 
A release fence prevents preceding memory operations from being delayed past
subsequent writes (a;FR;w =/=> w;a;FR)

An acquire operation prevents subsequent memory operations from being advanced
before it (Acq;a =/=> a;Acq)

An acquire fence prevents subsequent memory operations from being advanced
before prior reads (r;FA;a =/=> FA;a;r)

https://www.modernescpp.com/index.php/fences-as-memory-barriers

StoreLoad: Full fence allows a store before to be reordered with respect to a
load after (wx;F;ry) ===> (ry;F;wx)

StoreLoad+LoadLoad: Release fence also allows (rx;FR;ry) ===> (ry;FR;rx)

StoreLoad+StoreStore: Acquire fence also allows (wx;FR;wy) ===> (wy;FR;wx)

LoadStore: No fence allows a prior load to reorder w.r.t. a subsequent store
(rx;FR;wy) =/=> (wy;FR;rx)

https://preshing.com/20120710/memory-barriers-are-like-source-control-operations/
Good news is that a fullFence does it.

Bizarrely, it seems this is not supported in C++... You have to go to assembly.
\end{comment}

% \citet{2019-sp} define \emph{3-valued pomsets with preconditions} to model
% security flaws that arise from speculative evaluation in computer
% microarchitecture (such as Spectre \cite{DBLP:journals/corr/abs-1801-01203}).
% We build on their work to define a model of relaxed memory
% \emph{architecture}.  Interestingly, the model naturally captures multi-copy
% atomicity (\mca).

In this section, we define the model, provide some intuitions about the
semantics, and work through a series of illustrative examples.  We present
precise details of the semantics in \textsection\ref{sec:model}.

Let $\Act$ be a set of \emph{actions}, that includes actions of the form
$(\DR{\aLoc}{\aVal})$, which \emph{reads} value $\aVal$ from location
$\aLoc$, $(\DRAcq{\aLoc}{\aVal})$, which is an \emph{acquire} that reads
$\aVal$ from $\aLoc$, $(\DW{\aLoc}{\aVal})$, which \emph{writes} $\aVal$ to
$\aLoc$, and $(\DWRel{\aLoc}{\aVal})$, which is a \emph{release} that writes
$\aVal$ to $\aLoc$. %, and $(\DF{})$, which is a acquire-release \emph{fence}.
Actions that neither acquire nor release are \emph{relaxed}; other actions
are \emph{synchronizations}.

The actions listed above are \emph{external}.  Each external action has a
corresponding \emph{internal} action, denoted by prefixing $\tau$.  Internal
actions also read and write locations, just as external actions do.
% \footnote{Fences have a limited role in our
% discussion.  We inappropriately refer to them as synchronizations for
% simplicity.}.

Two actions \emph{conflict} if one writes a location and the other
either reads or writes the same location.

Let $\Formulae$ be a set of logical formulae, such as $(\aLoc=1)$ or
$(\aReg=\bReg+1)$, where $\aReg$ and $\bReg$ are register names.

Formulae are \emph{open}, in that occurrences of register names and memory
locations are subject to substitutions of the form $\aForm[\aLoc/\aReg]$ and
$\aForm[\bExp/\aLoc]$, where $\aLoc$ is a memory location, $\aReg$ is a
register and $\bExp$ is an memory-location-free expression.  Actions are not
subject to substitution.

Our model is based on \emph{partially ordered multisets} (\emph{pomsets})~\cite{GISCHER1988199}.
\begin{definition}
  \label{def:mmpomset}
  A \emph{(memory model) pomset} is a tuple
  $(\Event, {\le}, %{\gtN},
  \labeling)$, such that
  \begin{itemize}
  \item $\Event$ is a set of \emph{states},
  \item $\labeling: \Event \fun (\Formulae\times\Act)$ is a \emph{labeling},
    from which we derive $\labelingForm:\Event\fun\Formulae$ and $\labelingAct:\Event\fun\Act$,
    % define $\labelingForm(\aEv)=\aForm$ and $\labelingAct(\aEv)=\aAct$ whenever
    % $\labelingForm(\aEv)=(\aForm\mid\aAct)$,
  \item ${\le} \subseteq (\Event\times\Event)$ is a partial order, and
  % \item ${\gtN} \subseteq (\Event\times\Event)$ is a partial order,
  % \item ${\le} \subseteq {\gtN}$ is a partial order, and
  \item if $\bEv\le\aEv$ then $\labelingForm(\aEv)$ implies
    $\labelingForm(\bEv)$.
  \end{itemize}
\end{definition}
In the parlance of chapter 3.3 of~\citet{AlglaveThesis}, $\le$ is a ``global
happens-before'' relation.  The restriction of $\lt$ to conflicting writes is
called the coherence order, $\rco$.  

We refer to ``memory model pomsets'' as ``{pomsets}''.
We write pairs in $(\Formulae\times\Act)$ as $(\aForm \mid \aAct)$.  We elide
$\aForm$ when it is a tautology.
We write $\bEv\lt\aEv$ when $\bEv\le\aEv$ and $\bEv\neq\aEv$.

We give the semantics of a program as a set of pomsets, where each pomset
corresponds to a single execution.  We visualize a pomset as a directed graph where
the nodes are drawn from $\Event$, each node $\aEv$ is labeled with
$\labeling(\aEv)$, and order is drawn as an edge.
We elide events with unsatisfiable preconditions.
As discussed below, we visualize order using 
arrows that indicate the reason that the order arises.  Although we use
multiple arrow, we emphasize that they are all part of the same $\le$ relation.
% the orders are drawn with
% \citeauthor{DBLP:journals/dc/Lamport86}'s notation.  
% For example:
% \begin{tikzdisplay}[node distance=1em]
%   \event{rx1}{a}{}
%   \event{wy0}{b}{below right=of rx1}
%   \event{wy1}{c}{above right=of wy0}
%   \po{rx1}{wy0}
%   \po{rx1}{wy1}
%   \wk{wy0}{wy1}
% \end{tikzdisplay}
% is a visualization of the pomset where:
% \[\begin{array}{c}
%     E = \{ 0,1,2 \}
%     \quad
%     {\labeling} = \{(0,a),\,(1,b),\,(2,c)\}
%     \\
%     {\le} = \{(0,1),\,(0,2)\}\cup\{(0,0),\,(1,1),\,(2,2)\}
%     \quad
%     {\gtN} = {\le}\cup\{(2,3)\}
% \end{array}\]
% for example:

\paragraph{Dependencies.}
A write event $\aEv$ has two kinds of dependencies.  First, it may have an
external dependency on a read $\bEv$ from another thread, which is reflected
in the order $\bEv \lt \aEv$.  Second, it may have an internal dependency on
a preceding event in the same thread, which is reflected in the formula
$\labelingForm(\aEv)$.  In order for the write event to ``fire,'' both these
dependencies need to be satisfied.  The external dependency on a read is
satisfied by globally visible writes.  The internal dependency can only be
fulfilled by a preceding actions of the same thread.

\paragraph{Intrathread dependencies.}
Within a single thread, $\lt$ captures the notion of dependency,
indicating that the related events cannot appear in the opposite order.  

%It includes the dependency of writes on reads.
%the absence of synchronization, this relation always goes \emph{from 
%reads to writes within a thread}.  
%The formula decorating a write event is another kind of dependency: it is 
%to be viewed as a precondition necessary for the event to fire.  Thus, for 
%a write to fire
% It is used to calculate these single-threaded dependencies, using 
%semantic entailment. 

For example, semantics of $y\GETS \aReg$ contains the following pomset.  
\begin{tikzdisplaylabel}[node distance=1em]{m1}
  \event{wy1}{\aReg=1 \mid \DW{y}{1}}{}
\end{tikzdisplaylabel}
that can be viewed as expressing the Hoare triple $\hoare{\aReg =1}{y\GETS \aReg}{y=1}$. 
The Hoare triple $\hoare{x =1}{\aReg\GETS x\SEMI y\GETS \aReg}{y =1}$ expresses the effect of prefixing an assignment to $\aReg$.
The assignment to $\aReg$ causes a substitution in
the precondition, changing the label to $(x=1\mid \DW{y}{1})$.
The effect of the read is realized in the semantics in two different ways:
\begin{displaymathsmall}
\begin{tikzpicture}[baselinecenter,node distance=1em]
  \event{wy1}{x=1 \mid \DW{y}{1}}{}
  \internal{rx1}{x=1 \mid \DR{x}{1}}{left=of wy1}
  \graypo{rx1}{wy1}
\end{tikzpicture} 
\qquad\qquad{\text{\normalsize and}}\qquad\qquad
\begin{tikzpicture}[baselinecenter,node distance=1em]
  \event{rx1}{\DR{x}{1}}{}
  \event{wy1}{\DW{y}{1}}{right=of rx1}
  \po{rx1}{wy1}
\end{tikzpicture}
\end{displaymathsmall}

In the pomset on the left, the read is \emph{internal}, and therefore the
formula on the write actions must be discharged by a preceding action in the  same thread.   In this execution, we have included an \emph{internal action}, which we visualize in grey.  

In the pomset on the right, the read is \emph{external}.  An
external read may discharge the precondition on a event that is ordered
after: by introducing order $(\DR{x}{1})\xpo(x=1\mid\DW{y}{1})$, we may
substitute $1$ for $x$ in the precondition, resulting in $1=1$, which is a
tautology and therefore elided.

Consider prefixing a write to the above program:
\begin{math}
  x \GETS 1 \SEMI
  \aReg\GETS x\SEMI
  y\GETS \aReg
\end{math}.
In this special case when the value is read from a write of the same thread, the semantics includes:
\begin{tikzdisplay}[node distance=1em]
  \event{wx1}{\DW{x}{1}}{}
  \internal{rx1}{\DR{x}{1}}{right=of wx1}
  \event{wy1}{\DW{y}{1}}{right=of rx1}
  \graypo{rx1}{wy1}
\end{tikzdisplay}
as in \citep{2019-sp}.  This pomset is derived discharging the internal
dependency $x=1$ from the left pomset above in the semantics of
$\aReg\GETS x\SEMI y\GETS \aReg$.  This feature is reminiscent of the
\emph{read-from internal} $({\rrfi})$ in hardware memory models, where writes
from a thread are forwarded to a read in the same thread without global
publication.

The Hoare triple corresponding to the above pomset is
$\hoare{\TRUE}{x \GETS 1\SEMI \aReg \GETS x \SEMI y\GETS \aReg}{y =1\land x=1}$.
This triple reflects the store forwarding optimization in compilers.

As seen above, preconditions can be weakened as a result of prefixing.  As in
\citep{2019-sp}, weakening of preconditions\footnote{Formula $\psi$ is
  \emph{weaker} than a formula $\phi$ if $\phi$ implies $\psi$.} can also
happen by merging actions.
% For example, mirroring the fact that the weakest precondition of $y=1$ is true for the program
% $\IF{\aReg}\THEN y\GETS1 \ELSE y\GETS1\FI$, its semantics includes a pomset with the single
% action $(\DW{y}{1})$, with precondition $\TRUE$.
%
% In computing the dependencies for the conditional
% statement, we first compute the then-part to contain
% $(\aReg\neq0\mid\DW{y}{1})$ and the else-part to contain
% $(\aReg=0\mid\DW{y}{1})$.  Mirroring the proof rule for conditionals in Hoare logic,  since the actions are the same in these events,
% they can be merged into a single event whose precondition is the disjunction
% of the preconditions of the individual events.  Thus, the semantics of
% $\IF{\aReg}\THEN y\GETS1 \ELSE y\GETS1\FI$ includes
% $(\aReg\neq0\lor \aReg=0\mid\DW{y}{1})$.  Since the precondition is a
% tautology, we may elide it.  
Consider the program $\IF{\aReg}\THEN y\GETS1 \ELSE y\GETS1\FI$.  We derive:
\[
\frac{\hoare{\aReg=0}{y\GETS1}{y=1}, \ \ \  \hoare{\aReg\neq0}{y\GETS1}{y=1}}
{\hoare{\TRUE}{\IF{\aReg}\THEN y\GETS1 \ELSE y\GETS1\FI}{y=1}}
\]
This is reflected in our semantics by a pomset with a single action whose precondition is $\TRUE$
\begin{tikzdisplay}[node distance=1em]{}
  \event{wy1}{\DW{y}{1}}{}
\end{tikzdisplay}
% For example, mirroring the fact that the weakest precondition of $y=1$ 
%is true for the program
%$\IF{\aReg}\THEN y\GETS1 \ELSE y\GETS1\FI$, its semantics includes a 
%pomset with the single
%action $(\DW{y}{1})$, with precondition $\TRUE$.
that is derived as follows.  The two branches of the conditional contain pomsets as follows:
\begin{displaymathsmall}
\mbox{THEN:}\;\;
\begin{tikzcenter}[node distance=1em]
  \event{wy1}{\aReg=0\mid \DW{y}{1}}{}
\end{tikzcenter}
\qquad\qquad
\mbox{ELSE: }\;\;
\begin{tikzcenter}[node distance=1em]
  \event{wy1}{\aReg\neq0\mid \DW{y}{1}}{}
\end{tikzcenter}
\end{displaymathsmall}
In the semantics of the conditional, we take the (not necessarily disjoint) union of the events, that permits events with the same actions to be identified by combining their  preconditions with a disjunction.  In this case, this yields the event $(\aReg\neq0\lor \aReg=0\mid\DW{y}{1})$, permitting us to derive the desired pomset in the semantics of
$\IF{\aReg}\THEN y\GETS1 \ELSE y\GETS1\FI$.


Within a single thread, $\geN$ also orders \emph{conflicting} events.  We
refer order arising from conflict as ``weak'' and visualize it using a dashed
arrow ``$\xwk$''.  This is in contrast to the ``strong'' order arising from dependency,
which we visualize using a solid arrow ``$\xpo$''.  For example, the semantics of
\begin{math}
  x\GETS 0\SEMI
  x\GETS 1
\end{math}
includes the pomset:
\begin{tikzdisplaylabel}[node distance=1em]{m2}
  \event{wx0}{\DW{x}{0}}{}
  \event{wx1}{\DW{x}{1}}{right=of wx0}
  \wk{wx0}{wx1}
\end{tikzdisplaylabel}
The weak edge $(\DW{x}{0})\xwk(\DW{x}{1})$ is required since $x\GETS 0$
precedes $x\GETS 1$ in thread order.  This notion of conflict is stable under
substitution (since actions are not subject to substitution).


\paragraph{Interthread dependencies.}
Between threads, $\lt$ includes reads-from, which always goes from writes to reads across threads.   We consider reads-from to be a strong order, which we visualize
with a thick arrow ``$\xrf$''.  indicates that one thread has \emph{read-from}
another.  Since new threads may introduce order from
write to read, we define our semantics to be closed with respect to
augmentation of $\le$.

At top level, we expect that every read event is \emph{fulfilled} by a
matching write.  We define fulfillment %in terms of $\gtN$, rather than $\le$,
so that it is monotone with respect to addition of new threads.  
% This suffices since $\bEv\geN\aEv$ implies that we cannot have $\aEv\lt\bEv$ in
% any augmentation. % (In \textsection\ref{app:blockers} we show that the obvious
% definition of fulfillment in terms of $\le$ does not enjoy this property.)

% Two basic concepts derived from pomsets are \emph{coherence} and \emph{reads-from}.
% \begin{definition}
%   A pomset \emph{coherent} if, when restricted to events that read or write
%   any single location $\aLoc$, $\gtN$ forms a partial order.
%   % As we shall see below, for top-level pomsets we could equivalently require
%   % that $\gtN$ forms a total order.
% \end{definition}

\begin{definition}
  \label{def:rf}
  We say $\bEv$ \emph{fulfills $\aEv$ on $\aLoc$} if $\bEv$ externally writes
  $\aVal$ to $\aLoc$, $\aEv$ externally reads $\aVal$ from $\aLoc$,
  \begin{itemize}
  \item $\bEv \lt \aEv$, and
  \item if an event $\cEv$ writes to $\aLoc$ then either $\cEv \gtN \bEv$ or $\aEv \gtN \cEv$.
  \end{itemize}

  An event \emph{is fulfilled} if it is fulfilled on $\aLoc$ for every
  $\aLoc$ it reads externally.

  A pomset is \emph{top-level} if
  \begin{itemize}
  \item every event is fulfilled, and
  \item every precondition is either a tautology or is unsatisfiable.
  \end{itemize}
\end{definition}

For example, combining \eqref{m1} and \eqref{m2}, the semantics of
\begin{math}
  x\GETS 0\SEMI
  x\GETS 1
  \PAR
  \aReg\GETS x\SEMI
  y\GETS \aReg
\end{math}
includes:
\begin{tikzdisplaylabel}[node distance=1em]{m3}
  \event{wx0}{\DW{x}{0}}{}
  \event{wx1}{\DW{x}{1}}{right=of wx0}
  \event{rx1}{\DR{x}{1}}{right=2.5 em of wx1}
  \event{wy1}{\DW{y}{1}}{right=of rx1}
  \rf{wx1}{rx1}
  \wk{wx0}{wx1}
  \po{rx1}{wy1}
\end{tikzdisplaylabel}
We often highlight the required strong edge from a write to a read action that it
fulfills, as above.

Between threads, $\geN$ also defines a partial order on conflicting actions;
we consider these to be a weak order, visualized ``$\xwk$''.
Because of the requirements of fulfillment, any writes to $\aLoc$ that are
read must be totally ordered.  As an example consider the following program
and execution:
\begin{displaymath}
  x\GETS 3
  \PAR
  \IF{x\EQ1}\THEN r\GETS x \FI
  \PAR
  x\GETS 1
  \PAR
  x\GETS2
\end{displaymath}
\begin{tikzdisplay}[node distance=1em]
  \event{wx1}{\DW{x}{1}}{}
  \event{wx2}{\DW{x}{2}}{below=of wx1}
  \event{rx1a}{\DR{x}{1}}{left=of wx1}
  \event{rx2a}{\DR{x}{2}}{left=of wx2}
  \po{rx1a}{rx2a}
  \rf{wx1}{rx1a}
  \rf{wx2}{rx2a}
  \wk{rx1a}{wx2}
  \wk{wx1}{wx2}
  \event{wx3}{\DW{x}{3}}{below left=-.2em and 1em of rx1a}
  \wk{rx1a}{wx3}
  \wk{rx2a}{wx3}
\end{tikzdisplay}
The restriction of $\geN$ to conflicting events on a single location is called the \emph{extended coherence order}, $\reco$.  Since $\geN$ is a partial order, so is $\reco$.  

% Thus, $\geN$ can always be extended to a total orderwhich orders any 
% two conflicting operations in a pomset.  includes an ,   As a derived
% property, we have that executions are \emph{coherent}.
%
%\begin{definition}
 % A pomset \emph{coherent} if, when restricted to  $\gtN$ forms a partial 
%order.
%\end{definition}
%To see that this definition captures the idea of coherence, consider that 
%incoherent executions always result
%in cycles in weak order.  (We include acquiring reads in order to enforce
%read-to-read order, as discussed below.)  % (In the following example, we include fences to
% ensure order between the reads; we elide the fences in the drawing, since
% they are not essential to the point being made.)
% Because we are modeling architecture rather than microarchitecture, result,
% our use of pomsets is more abstract than that of \citeauthor{2019-sp}.  For
% example, at top-level, we ignore events whose preconditions are not
% tautologies; \citeauthor{2019-sp} use such events to model the influence of
% ``unexecuted'' conditional code on executed code, as found in Spectre.

%\begin{displaymath}
 % r\GETS x\ACQ \SEMI %\FENCE\SEMI
 % s\GETS x
%  \PAR
%  x\GETS 1
 % \PAR
 % x\GETS2
 % \PAR
 % r\GETS x\ACQ \SEMI %\FENCE\SEMI
 % s\GETS x
%\end{displaymath}
%\begin{tikzdisplay}[node distance=1em]
%  \event{wx1}{\DW{x}{1}}{}
%  \event{wx2}{\DW{x}{2}}{below=of wx1}
%  \event{rx1a}{\DRAcq{x}{1}}{left=of wx1}
%  \event{rx2a}{\DR{x}{2}}{left=of wx2}
%  \event{rx1b}{\DR{x}{1}}{right=of wx1}
 % \event{rx2b}{\DRAcq{x}{2}}{right=of wx2}
%  \po{rx1a}{rx2a}
%  \po{rx2b}{rx1b}
 % \rf{wx1}{rx1a}
 % \rf{wx1}{rx1b}
 % \rf{wx2}{rx2a}
%  \rf{wx2}{rx2b}
%%  \wk{rx1a}{wx2}
%%  \wk{rx2a}{wx1}
%  \wk{rx1b}{wx2}
%  \wk{rx2b}{wx1}
%\end{tikzdisplay}
%This satisfies the requirements for fulfillment, but is not coherent.

\paragraph{Synchronization}
The model ensures that events are ordered before a release and after an
acquire.  This is a strong order which is visualized as a solid arrow, like
order derived from dependency.  As an example, consider the program:
\begin{displaymath}
  x\GETS0\SEMI %y\GETS0\SEMI
  x\GETS 1\SEMI y \REL\GETS1 \PAR r\GETS y\ACQ; s\GETS x
\end{displaymath}
\text{which allows:}  \hfill
\begin{tikzinline}[node distance=1em]
  \event{wx0}{\DW{x}{0}}{}
  %\event{wy0}{\DW{y}{0}}{below=of wx0}
  \event{wx1}{\DW{x}{1}}{right=of wx0}
  \event{wy1}{\DWRel{y}{1}}{below=of wx1}
  \event{ry1}{\DRAcq{y}{1}}{right=2.5em of wy1}
  \event{rx1}{\DR{x}{1}}{above=of ry1}
  \po{wx0}{wy1}
  %\po{wy0}{wy1}
  \po{wx1}{wy1}
  \po{ry1}{rx1}
  \rf{wy1}{ry1}
  \rf{wx1}{rx1}
  \wk{wx0}{wx1}
\end{tikzinline}
\hfill\text{but \emph{not}:}\hfill
\begin{tikzinline}[node distance=1em, baseline={([yshift=-1.5ex]current bounding box.west)}]
  \event{wx0}{\DW{x}{0}}{}
  %\event{wy0}{\DW{y}{0}}{below=of wx0}
  \event{wx1}{\DW{x}{1}}{right=of wx0}
  \event{wy1}{\DWRel{y}{1}}{below=of wx1}
  \event{ry1}{\DRAcq{y}{1}}{right=2.5em of wy1}
  \event{rx0}{\DR{x}{0}}{above=of ry1}
  \po{wx0}{wy1}
  %\po{wy0}{wy1}
  \po{wx1}{wy1}
  \po{ry1}{rx1}
  \rf{wy1}{ry1}
  \rf[out=20,in=160]{wx0}{rx0}
  \wk{wx0}{wx1}
\end{tikzinline}
\hfill\hbox{}

\smallskip
\noindent
Since $(\DW x0) \gtN (\DW x1) \lt (\DR x0)$, this pomset does not satisfy the
requirements to fulfill the read.
If we replace the release
with a plain write, then the outcome $(\DRAcq y1)$ and $(\DR x0)$ is possible:
\begin{tikzdisplay}[node distance=1em]
  \event{wx0}{\DW{x}{0}}{}
  %\event{wy0}{\DW{y}{0}}{below=of wx0}
  \event{wx1}{\DW{x}{1}}{right=of wx0}
  \event{wy1}{\DW{y}{1}}{below=of wx1}
  \event{ry1}{\DRAcq{y}{1}}{right=2.5em of wy1}
  \event{rx0}{\DR{x}{0}}{above=of ry1}
  %\wk{wy0}{wy1}
  \po{ry1}{rx0}
  \rf{wy1}{ry1}
  \rf[out=20,in=160]{wx0}{rx0}
  \wk{wx0}{wx1}
  \wk{rx0}{wx1}
\end{tikzdisplay}
since no order is required between $(\DW x1)$ and $(\DW y1)$.  
Symmetrically, if we replace the acquire of the original program
with a plain read, then the outcome $(\DR y1)$ and $(\DR x0)$ is possible.

\paragraph{Impact of Internal reads}
The ability to remove of the program order from a write to a read of the same thread as discussed earlier causes great expressivity in terms of permitted executions.  To see this, consider the example 3.6 from
\cite{DBLP:journals/pacmpl/PodkopaevLV19}:
\begin{displaymath}
  \aReg\GETS x\SEMI
  y\REL\GETS 1\SEMI
  z\GETS y
  \PAR
  x\GETS z
\end{displaymath}

\begin{tikzdisplaylabel}[node distance=1em]{thread}
  \event{a1}{\DR{x}{1}}{}
  \event{a2}{\DWRel{y}{1}}{right=of a1}
  \po{a1}{a2}
  \internal{a3}{\DR{y}{1}}{right=of a2}
  \event{a4}{\DW{z}{1}}{right=of a3}
  %\graypo{a2}{a3}
  \graypo{a3}{a4}
  \event{b1}{\DR{z}{1}}{below=of a3}
  \event{b2}{\DW{x}{1}}{left=of b1}
  %\po[out=10,in=170]{a2}{a4}
  \po{b1}{b2}
  \rf{a4}{b1}
  \rf{b2}{a1}
\end{tikzdisplaylabel}
There is no order from $(\DWRel{y}{1})$ to $(\DW{z}{1})$.  This behavior is
allowed by \armeight.  Were the internal action made explicit, then the order
shown in gray would be imposed; this behavior would be disallowed due to the
evident cycle:
\begin{tikzdisplay}[node distance=1em]
  \event{a1}{\DR{x}{1}}{}
  \event{a2}{\DWRel{y}{1}}{right=of a1}
  \po{a1}{a2}
  \event{a3}{\DR{y}{1}}{right=of a2}
  \event{a4}{\DW{z}{1}}{right=of a3}
  \rf{a2}{a3}
  \po{a3}{a4}
  \event{b1}{\DR{z}{1}}{below=of a3}
  \event{b2}{\DW{x}{1}}{left=of b1}
  %\po[out=10,in=170]{a2}{a4}
  \po{b1}{b2}
  \rf{a4}{b1}
  \rf{b2}{a1}
\end{tikzdisplay}
The internal actions also facilitate the definitions required in the proof
of \drfsc\ (\textsection\ref{sec:sc}).  

The dependency calculation for $(\DW{z}{1})$ in this execution proceeds as
follows.  Desugaring, the thread is
\begin{math}
  (\aReg\GETS x\SEMI
  y\REL\GETS 1\SEMI
  \bReg\GETS y\SEMI
  z\GETS \bReg).
\end{math}
The semantics of $z\GETS \bReg$ includes a pomset with event
$(\bReg=1\mid \DW{z}{1})$. The precondition is discharged internally by the
write of $y=1$ without an externally visible dependency to an explicit read:
Prefixing $\bReg\GETS y$ transforms the precondition to $y=1$.  As in the
discussion of execution \eqref{m1}, the order from $(\DWRel{y}{1})$
allows the precondition $y=1$ to be weakened to $1=1$, which is a tautology.
Note that both write prefixing and read prefixing enable precondition
weakening.  Only read prefixing imposes order; there is no order from
the prefixed write to any subsequent action.

\paragraph*{Invalidation of thread inlining. }
The behavior in~\ref{thread} is not possible if we split the first
thread in two:
\begin{math}
  \aReg\GETS x\SEMI
  y\REL\GETS 1\PAR
  z\GETS y.
\end{math}
since that would make the read explicit, thus imposing the order
shown in gray.   This shows that, like the JMM, our model invalidates thread
inlining.

%Here is an example which uses a cycle in weak order rather than strong 
%order.
%We also use a fence (not drawn) rather than an acquire or release:
%\begin{displaymath}
 % y \GETS0 \SEMI %\FENCE\SEMI
 % x\REL\GETS1\SEMI a\GETS y
 % \PAR
 %% x \GETS0 \SEMI %\FENCE\SEMI
 % y\REL\GETS1\SEMI b\GETS x
%\end{displaymath}
%\begin{tikzdisplay}[node distance=1em]
 % \event{a1}{\DW{y}{0}}{}
 % \event{a2}{\DWRel{x}{1}}{right=of a1}
 % \po{a1}{a2}
 % \internal{a3}{\DR{y}{0}}{right=of a2}
 %% \graypo{a2}{a3}
%  \event{a4}{\DW{a}{0}}{right=of a3}
 % \graypo{a3}{a4}
 % \event{b1}{\DW{x}{0}}{below=of a1}
 % \event{b2}{\DWRel{y}{1}}{right=of b1}
 % \po{b1}{b2}
 % \internal{b3}{\DR{x}{0}}{right=of b2}
 % \graypo{b2}{b3}
 % \event{b4}{\DW{b}{0}}{right=of b3}
 % \graypo{b3}{b4}
 % \graywk{b3}{a2}
 % \graywk{a3}{b2}
%\end{tikzdisplay}
% \begin{displaymath}
%   x \GETS1 \SEMI \IF{y}\THEN \FENCE \SEMI z\GETS x\FI
%   \PAR
%   \IF{x}\THEN x\GETS2 \SEMI \FENCE\SEMI y \GETS1\FI
% \end{displaymath}
% \begin{tikzdisplay}[node distance=1em]
%   \event{a1}{\DW{x}{1}}{}
%   \event{a2}{\DR{y}{1}}{right=of a1}
%   \po{a1}{a2}
%   \internal{a3}{\DR{x}{1}}{right=of a2}
%   \event{a4}{\DW{z}{1}}{right=of a3}
%   \graypo{a2}{a3}
%   \graypo{a3}{a4}
%   \event{b1}{\DR{x}{1}}{below=of a1}
%   \rf{a1}{b1}
%   \event{b2}{\DW{x}{2}}{right=of b1}
%   \po{b1}{b2}
%   \event{b3}{\DW{y}{1}}{right=of b2}
%   \po{b2}{b3}
%   \rf{b3}{a2}
%   \graywk{a3}{b2}
% \end{tikzdisplay}

In defining \emph{data races} (and thus in the proof of \drfsc) it is
important to position internal reads relative to synchronization actions.
For example, the following execution is consider data-race free.
\begin{displaymath}
  x\GETS1\SEMI
  \aReg\GETS x \SEMI
  y\REL\GETS 1 \SEMI
  \PAR
  \bReg\GETS y\ACQ \SEMI
  x\GETS 2
\end{displaymath}
\begin{tikzdisplay}[node distance=1em]
  \event{a1}{\DW{x}{1}}{}
  \internal{a2}{\DR{x}{1}}{right=of a1}
  %\graypo{a1}{a2}
  \event{a3}{\DWRel{y}{1}}{right=of a2}
  \graypo{a2}{a3}
  \po[bend left]{a1}{a3}
  \event{b1}{\DRAcq{z}{1}}{right=3em of a3}
  \rf{a3}{b1}
  \event{b2}{\DW{x}{2}}{right=of b1}
  \po{b1}{b2}
\end{tikzdisplay}
However, if we commute the read of $x$ and the release of $y$, the resulting
program has a data race.
% \begin{displaymath}
%   x\GETS0\SEMI
%   x\GETS1\SEMI
%   \aReg\GETS x \SEMI
%   y\REL\GETS 1 \SEMI
%   z\GETS\aReg
%   \PAR
%   \bReg\GETS z\ACQ \SEMI
%   \cReg\GETS x
% \end{displaymath}
% \begin{tikzdisplay}[node distance=1em]
%   \event{a1}{\DW{x}{1}}{}
%   \event{a0}{\DW{x}{0}}{left=of a1}
%   \wk{a0}{a1}
%   \internal{a2}{\DR{x}{1}}{right=of a1}
%   \graypo{a1}{a2}
%    \event{a3}{\DWRel{y}{1}}{right=of a2}
%   \graypo{a2}{a3}
%   \po[bend left]{a1}{a3}
%   \event{a4}{\DW{z}{1}}{right=of a3}
%   \graypo[bend left]{a2}{a4}
%   \event{b1}{\DRAcq{z}{1}}{below=of a4}
%   \rf{a4}{b1}
%   \event{b2}{\DR{x}{0}}{left=8em of b1}
%   \po{b1}{b2}
%   \wk{b2}{a1}
%   \wk{a0}{b2}
% \end{tikzdisplay}

\paragraph{Multi-copy atomicity.}  Many of the standard litmus tests for \mca\ are
variants of \iriw\ (Independent Reads of Independent Writes), such as \iriw\
with control dependencies between the reads:
\begin{displaymatharray}{rl}
  &x\GETS0\SEMI x\GETS 1
  \PAR
  \IF{x}\THEN r\GETS y \FI
 \\{}
  \PAR&
  y\GETS0\SEMI y\GETS 1
  \PAR
  \IF{y}\THEN s\GETS x \FI
\end{displaymatharray}
\begin{tikzdisplay}[node distance=1em]
  \event{wx0}{\DW{x}{0}}{}
  \event{wx1}{\DW{x}{1}}{right=of wx0}
  \event{wy0}{\DW{y}{0}}{below=4ex of wx0}
  \event{wy1}{\DW{y}{1}}{right=of wy0}
  \event{ry1}{\DR{y}{1}}{right=2.5em of wy1}
  \event{rx0}{\DR{x}{0}}{right=of ry1}
  \event{rx1}{\DR{x}{1}}{right=2.5 em of wx1}
  \event{ry0}{\DR{y}{0}}{right=of rx1}
  \wk{wx0}{wx1}
  \wk{wy0}{wy1}
  \rf{wx1}{rx1}
  \rf[bend left]{wy0}{ry0}
  \rf{wy1}{ry1}
  \rf[bend right]{wx0}{rx0}
  \wk{rx0}{wx1}
  \wk{ry0}{wy1}
  %\po{rx1}{ry0}
  %\po{ry1}{rx0}
\end{tikzdisplay}
Since our semantics does not enforce control dependencies between reads, 
this execution is acyclic and is allowed.  If the first read in each thread is acquiring, there is order between the reads.
\begin{displaymatharray}{rl}
  &x\GETS0\SEMI x\GETS 1
  \PAR
  \IF{x\ACQ}\THEN r\GETS y \FI
 \\{}
  \PAR&
  y\GETS0\SEMI y\GETS 1
  \PAR
  \IF{y\ACQ}\THEN s\GETS x \FI
\end{displaymatharray}
The resulting execution is disallowed due to the  cycle.
\begin{tikzdisplay}[node distance=1em]
  \event{wx0}{\DW{x}{0}}{}
  \event{wx1}{\DW{x}{1}}{right=of wx0}
  \event{wy0}{\DW{y}{0}}{below=4ex of wx0}
  \event{wy1}{\DW{y}{1}}{right=of wy0}
  \event{ry1}{\DR{y\ACQ}{1}}{right=2.5em of wy1}
  \event{rx0}{\DR{x}{0}}{right=of ry1}
  \event{rx1}{\DR{x\ACQ}{1}}{right=2.5 em of wx1}
  \event{ry0}{\DR{y}{0}}{right=of rx1}
  \wk{wx0}{wx1}
  \wk{wy0}{wy1}
  \rf{wx1}{rx1}
  \rf[bend left]{wy0}{ry0}
  \rf{wy1}{ry1}
  \rf[bend right]{wx0}{rx0}
  \wk{rx0}{wx1}
  \wk{ry0}{wy1}
  \po{rx1}{ry0}
  \po{ry1}{rx0}
\end{tikzdisplay}

% Note that if you enforce the dependency between the reads, then the execution
% is disallowed, because the is a cycle on weak edges restricted to $x$:
% \begin{math}
%   (\DW{x}{1})\le (\DR{y}{0}) \gtN (\DW{y}{1})
% \end{math}
% therefore
% \begin{math}
%   (\DW{x}{1}) \gtN  (\DW{y}{1});
% \end{math}
% then
% \begin{math}
%   (\DW{y}{1})\le (\DR{y}{0}) \gtN (\DW{y}{1})
% \end{math}
% therefore
% \begin{math}
%   (\DW{x}{1}) \gtN  (\DW{y}{1});
% \end{math}

Another class of non-\mca\ behavior is characterized by the following litmus test.
\begin{displaymath}
  \IF{x}\THEN y\GETS0 \FI \SEMI y\GETS1
  \PAR
  \IF{y}\THEN z\GETS0 \FI \SEMI z\GETS1
  \PAR
  \IF{z}\THEN x\GETS0 \FI \SEMI x\GETS1
\end{displaymath}
\begin{tikzdisplay}[node distance=1em]
  \event{a1}{\DR{x}{1}}{}
  \event{a2}{\DW{y}{0}}{right=of a1}
  \po{a1}{a2}
  \event{a3}{\DW{y}{1}}{right=of a2}
  \wk{a2}{a3}
  \event{b1}{\DR{x}{1}}{right=3em of a3}
  \event{b2}{\DW{y}{0}}{right=of b1}
  \po{b1}{b2}
  \event{b3}{\DW{y}{1}}{right=of b2}
  \wk{b2}{b3}
  \event{c1}{\DR{x}{1}}{right=3em of b3}
  \event{c2}{\DW{y}{0}}{right=of c1}
  \po{c1}{c2}
  \event{c3}{\DW{y}{1}}{right=of c2}
  \wk{c2}{c3}
  \rf{a3}{b1}
  \rf{b3}{c1}
  \rf[out=170,in=10]{c3}{a1}  
\end{tikzdisplay}
This execution is disallowed due to the evident cycle.
%in weak order. 

% This execution is allowed by 3-valued pomsets with coherence 
%\citep{2019-sp}. Interestingly, the two thread/two variable variant of this 
%example is disallowed by 3-valued pomsets, due to the requirements of
%\citeauthor{DBLP:journals/dc/Lamport86}'s axiom A3, which requires that
%$\cEv \gtN \aEv$ if $\cEv \le \bEv \gtN \aEv$ or $\cEv \gtN \bEv \le \aEv%$. If one were to shift from 3-valued logic to an interval logic, then it %would make sense to include \citeauthor{DBLP:journals/dc/Lamport86}'s %axiom A4,which requires that $\bEv \lt \aEv$ if $\bEv \lt \bEv' \geN \aEv' %\lt \aEv$.In the semantics of programs, requiring A4 for 3-valued pomsets %withcoherence is equivalent to the assumption that $\gtN$ is a partial %order, as in the memory model pomsets used here.  Thus, the two %thread/two variable variant is also disallowed.

\paragraph{Load buffering and thin air.}
The program
\begin{math}
  %x\GETS0\SEMI y\GETS0\SEMI
  (y\GETS x \PAR \bReg\GETS y\SEMI x\GETS1)
\end{math}
has top level executions that result in the final outcome $x = y = 1$, such as:
\begin{tikzdisplay}[node distance=1em]
  % \event{wx0}{\DW{x}{0}}{}
  % \event{wy0}{\DW{y}{0}}{below=wx0}
  \event{rx}{\DR{x}{1}}{}
  \event{wy}{\DW{y}{1}}{below=of rx}
  \po{rx}{wy}
  \event{ry}{\DR{y}{1}}{right=of rx}
  \event{wx}{\DW{x}{1}}{below=of ry}
  \rf{wy}{ry}
  \rf{wx}{rx}
  %\po{rx}{wy}
\end{tikzdisplay}
In \textsection\ref{sec:logic} we provide machinery to prove that this
outcome is impossible if there is order from read to write in both
threads.  This order can be achieved by replacing the second thread
\begin{math}
  (\bReg\GETS y\SEMI x\GETS1)
\end{math}
with 
\begin{math}
  (\bReg\GETS y\ACQ\SEMI x\GETS1)
\end{math}
or
\begin{math}
  (\IF{y}\THEN x\GETS 1\FI)
\end{math}
or
\begin{math}
  (x\GETS y).
\end{math}

A more interesting example is the following variant of \eqref{types}:
\begin{equation}
  \label{alan}
  % x\GETS0\SEMI
  %y\GETS0\SEMI   
  (
    y\GETS x
  \PAR
    \IF{z}\THEN x\GETS1 \ELSE x\GETS y\SEMI a\GETS y \FI
  \PAR
    z\GETS0\SEMI z\GETS1
  )
\end{equation}
This program is allowed to write $1$ to $a$ under many speculative
memory models
\cite{Manson:2005:JMM:1047659.1040336,DBLP:conf/esop/JagadeesanPR10,DBLP:conf/popl/KangHLVD17},
even though the read of $1$ from $y$ in the else branch of the second
thread arises out of thin air.   \citet{DBLP:journals/toplas/Lochbihler13}
argues that such executions compromise type safety unless object allocation
partitions memory by type.
In our model, the attempted execution is:
\begin{tikzdisplay}[node distance=1em]
  \event{rx}{\DR{x}{1}}{}
  \event{wy}{\DW{y}{1}}{below=of rx}
  \po{rx}{wy}
  \event{ry}{\DR{y}{1}}{right=of rx}
  \event{wx}{\DW{x}{1}}{below=of ry}
  \po{ry}{wx}
  \rf{wy}{ry}
  \rf{wx}{rx}
  \event{rz}{\DR{z}{0}}{right=of ry}
  \event{wz0}{\DW{z}{0}}{right=of rz}
  \rf{wz0}{rz}
  \event{wz1}{\DW{z}{1}}{right=of wz0}
  \wk{wz0}{wz1}
  \event{ry1}{\DR{y}{1}}{below=of rz}
  \rf[bend right]{wy}{ry1}
  \event{wa}{\DW{a}{1}}{right=of ry1}
  \po{ry1}{wa}
  \po{rz}{wa}
\end{tikzdisplay}
This is forbidden by the evident cycle.

\paragraph{Pointers.}
Our language allows address calculations.  To model this, we allow locations
to have the form $\REF{n}$, where $n$ is a natural number.  Because we do not
enforce order between reads, there is some danger that address calculations
could allow thin air behavior.   To see how our model addresses this, assume
that we have the following locations and initial values:
\begin{align*}
  \REF{0}&=0  &
  \REF{1}&=2  &
  \REF{2}&=1  &
  x &=0 &
  y &=0 
\end{align*}
Consider the program
\begin{math}
  (x\GETS\REF{y} \PAR y\GETS\REF{x})
\end{math}
with attempted execution:
\begin{tikzdisplay}[node distance=1em]
  \event{a1}{\DR{y}{2}}{}
  \event{a2}{\DR{\REF{2}}{1}}{below=of a1}
  \po{a1}{a2}
  \event{a3}{\DW{x}{1}}{below=of a2}
  \po[out=210,in=150]{a1}{a3}
  \event{b1}{\DR{x}{1}}{right=3em of a1}
  \event{b2}{\DR{\REF{1}}{2}}{below=of b1}
  \po{b1}{b2}
  \event{b3}{\DW{y}{2}}{below=of b2}
  \po[out=-30,in=30]{b1}{b3}
  \rf{b3}{a1}
  \rf{a3}{b1}
\end{tikzdisplay}
Although there is no order enforced between the reads, the read-to-write
order induced by the semantics is sufficient to prohibit this thin-air
behavior.

The dependency calculation in this example is interesting.  Desugaring, the
first thread is $\aReg\GETS y\SEMI \bReg\GETS \REF{\aReg}\SEMI x\GETS \bReg$.
In isolation, the write action is $(\bReg=1\mid\DW{x}{1})$.  Following the
discussion of execution \eqref{m1}, $\bReg\GETS \REF{\aReg}$ causes the
precondition to become $\REF{\aReg}=1$; subsequently prefixing with
$(\aReg=2\mid\DR{\REF{2}}{1})$, this can be weakened to $1=1$ by the
substitution of $1$ for $\REF{2}$.  But the precondition is also constrained
by the last clause of the Definition~\ref{def:mmpomset}:
 if $\bEv\le\aEv$ then $\labelingForm(\aEv)$ implies $\labelingForm(\bEv)$.
%
Thus the we arrive at the label $(\aReg=2\mid\DW{x}{1})$.  Subsequently,
$\aReg\GETS y$ transforms the precondition to $y=2$ and prefixing $\DR{y}{2}$
allows this to be weakened to $2=2$.

% if $\bEv\lt\bEv'\geN\aEv'\lt\aEv$ then $\bEv\lt\aEv$.
%   \begin{displaymath}
%     y\GETS 1
%     \PAR
%     x\GETS 1\SEMI
%     x\GETS 2\SEMI
%     \FENCE\SEMI
%     \aReg\GETS y
%     \PAR
%     y\GETS 2\SEMI
%     \FENCE\SEMI
%     \bReg\GETS x
%   \end{displaymath}
% \begin{tikzdisplay}[node distance=1em]
%   \event{a1}{\DW{x}{1}}{}
%   \event{a2}{\DW{x}{2}}{right=of a1}
%   \po{a1}{a2}
%   %\event{a3}{\DF{}}{right=of a2}
%   %\po{a2}{a3}
%   \event{a4}{\DR{y}{1}}{right=of a2}
%   \po{a2}{a4}
%   %
%   \event{b1}{\DW{y}{2}}{below=of a4}
%   %\event{b2}{\DF{}}{left=of b1}
%   %\po{b1}{b2}
%   \event{b3}{\DR{x}{1}}{left=of b1}
%   \po{b1}{b3}
%   \wk{a4}{b1}
%   %
%   \event{c1}{\DW{y}{1}}{above=of a4}
%   \po{c1}{a4}
%   \rf{a1}{b3}
%   \wk{b3}{a2}
% \end{tikzdisplay}
% Ie, it does publication by antidependency...

\paragraph{Blockers.}
% It is tempting simplify the model to use only strong order.  This creates a
% difficulty in defining fulfillment (Definition~\ref{def:rf}).  It is clearly
% too strong to simply replace weak order with strong order in the definition.
% As a consequence of the fact that $\bEv$ fulfills $\aEv$ on $\aLoc$, we would
% have any that $\cEv$ writes $\aLoc$ must be strongly before $\bEv$ or
% strongly after $\aEv$.  This would require the introduction of strong order
% between events with no causal relationship.
Fulfillment is often defined in terms of the absence of \emph{blocker},
rather than the presence of order to every conflicting event.  In this
definition, the last bullet of Definition~\ref{def:rf} becomes:
\begin{itemize}
\item there is no $\bEv \lt \cEv \lt \aEv$ such that $\cEv$ writes to $\aLoc$.
\end{itemize}
This definition is not preserved by parallel composition.
To make the consequences clear, we include location declarations in our
language, written $\VAR\aLoc\SEMI \aCmd$.  Fulfillment with blockers violates
\emph{scope extrusion}~\cite{Milner:1999:CMS:329902}, in that we can find
programs $\aCmd$ and $\bCmd$ such that the semantics
${\VAR\aLoc\SEMI(\aCmd\PAR\bCmd)}$ is different from the semantics of
${(\VAR\aLoc\SEMI\aCmd)\PAR\bCmd}$, even if $\bCmd$ does not mention~$\aLoc$.
To see this, consider the program:
\begin{gather*}
  \aReg\GETS a\ACQ\SEMI
  x\GETS1 \SEMI
  b\GETS\REL 1
  \PAR
  \aReg\GETS c\ACQ\SEMI
  x\GETS2 \SEMI
  d\GETS\REL 1
  \PAR
  \aReg\GETS e\ACQ\SEMI
  \bReg\GETS x \SEMI
  f\GETS\REL 1
  \\[-2ex]
  \begin{minipage}{1.0\linewidth}
    \begin{tikzdisplay}[node distance=1em]
      \event{a1}{\DRAcq{a}{1}}{}
      \event{a2}{\DW{x}{1}}{right=of a1}
      \po{a1}{a2}
      \event{a3}{\DWRel{b}{1}}{right=of a2}
      \po{a2}{a3}
      \event{b1}{\DRAcq{c}{1}}{right=3em of a3}
      \event{b2}{\DW{x}{2}}{right=of b1}
      \po{b1}{b2}
      \event{b3}{\DWRel{d}{1}}{right=of b2}
      \po{b2}{b3}
      \event{c1}{\DRAcq{e}{1}}{right=3em of b3}
      \event{c2}{\DR{x}{1}}{right=of c1}
      \po{c1}{c2}
      \event{c3}{\DWRel{f}{1}}{right=of c2}
      \po{c2}{c3}
      \rf[out=10,in=170]{a2}{c2}
    \end{tikzdisplay}
  \end{minipage}
\end{gather*}
In this execution, the read is fulfilled by the first thread; the second
thread is not blocking.  However, when placed in parallel with
\begin{math}
  (c\GETS b\SEMI d\GETS e),
\end{math}
the second thread may become a blocker:
\begin{tikzdisplay}[node distance=1em]
  \event{a1}{\DRAcq{a}{1}}{}
  \event{a2}{\DW{x}{1}}{right=of a1}
  \po{a1}{a2}
  \event{a3}{\DWRel{b}{1}}{right=of a2}
  \po{a2}{a3}
  \event{b1}{\DRAcq{c}{1}}{right=3em of a3}
  \event{b2}{\DW{x}{2}}{right=of b1}
  \po{b1}{b2}
  \event{b3}{\DWRel{d}{1}}{right=of b2}
  \po{b2}{b3}
  \event{c1}{\DRAcq{e}{1}}{right=3em of b3}
  \event{c2}{\DR{x}{1}}{right=of c1}
  \po{c1}{c2}
  \event{c3}{\DWRel{f}{1}}{right=of c2}
  \po{c2}{c3}
  \rf[out=10,in=170]{a2}{c2}
  \event{d1}{\DR{b}{1}}{below=of a3}
  \rf{a3}{d1}
  \event{d2}{\DW{c}{1}}{below=of b1}
  \po{d1}{d2}
  \rf{d2}{b1}
  \event{e1}{\DR{d}{1}}{below=of b3}
  \rf{b3}{e1}
  \event{e2}{\DW{e}{1}}{below=of c1}
  \po{e1}{e2}
  \rf{e2}{c1}
\end{tikzdisplay}

% The use of weak order in the definition of fulfillment splits the difference
% between these extremes.
Our definition of fulfillment requires that the read be ordered with respect
to both writes:
\begin{tikzdisplay}[node distance=1em]
  \event{a1}{\DRAcq{a}{1}}{}
  \event{a2}{\DW{x}{1}}{right=of a1}
  \po{a1}{a2}
  \event{a3}{\DWRel{b}{1}}{right=of a2}
  \po{a2}{a3}
  \event{b1}{\DRAcq{c}{1}}{right=3em of a3}
  \event{b2}{\DW{x}{2}}{right=of b1}
  \po{b1}{b2}
  \event{b3}{\DWRel{d}{1}}{right=of b2}
  \po{b2}{b3}
  \event{c1}{\DRAcq{e}{1}}{right=3em of b3}
  \event{c2}{\DR{x}{1}}{right=of c1}
  \po{c1}{c2}
  \event{c3}{\DWRel{f}{1}}{right=of c2}
  \po{c2}{c3}
  \rf[out=10,in=170]{a2}{c2}
  \wk[out=200,in=-20]{c2}{b2}
\end{tikzdisplay}
This makes augmentation in the reverse order impossible.
% The weak order required between the second and third threads stops any
% parallel component from blocking the read.  Yet, this creates no causal
% relation between the threads.

% \textcolor{red}{TODO: Is there any example that makes use of this lack of
%   causality?  Or should we just get rid of weak order????}


\paragraph{Read from unexecuted branch (\rfub).}
\begin{comment}
RFUB Example 1
Thread 1:	                        
r1 = x.load(memory_order_relaxed);
y.store(r1, memory_order_relaxed);	

Thread 2:
bool assigned_42(false);
r1 = y.load(memory_order_relaxed);
if (r1 != 42) {
    assigned_42 = true;
    r1 = 42;
}
x.store(r1, memory_order_relaxed);
assert_not(assigned_42);
\end{comment}

\citet{BoehmOOTA} analyzes programs in which register state from an
unexecuted branch can effect the outcome of an execution.  
\citeauthor{BoehmOOTA}'s \rfub1 example can be written in our language as:
\begin{equation}
  \label{rfub}
  y\GETS x
  \PAR
  r\GETS y\SEMI
  \IF{r \NOTEQ 1} \THEN z\GETS 1\SEMI r\GETS 1\SEMI x\GETS r \ELSE x\GETS r \FI
\end{equation}
\citeauthor{BoehmOOTA}'s concern arises from the similarity of this program
with the \oota\ litmus test that replaces the second thread with
\begin{math}
  r\GETS y\SEMI
  x\GETS r.
\end{math}
Should an unexecuted conditional be allowed to change the outcome of the
program?  In our semantics the answer is ``yes''.  First note that the
program writes $1$ to $x$ in both branches of the conditional.  Further, the
writes to $z$ and $x$ in the then-branch of the conditional are independent.
Therefore, it is sensible for a compiler to hoist the write to $x$ out of the
conditional.

To analyze the example formally, we use combinators on pomsets that are
defined in the next section.  These are prefixing $(\prefix)$ and composition
$(\parallel)$.  We also include events with unsatisfiable conditions in our
drawings; we indicate unsatisfiability by crossing out the event.  The
execution in question is:
\begin{displaymathsmall}
  \begin{tikzcenter}[node distance=1em]
    \event{a1}{\DR{x}{1}}{}
    \event{a2}{\DW{y}{1}}{below=of a1}
    \po{a1}{a2}
  \end{tikzcenter}
  \Biggm\|
  \DR{y}{1}
  \prefix
  \left(
    \begin{tikzcenter}[node distance=1em]
      \event{b1}{r\NOTEQ1\mid\DW{z}{1}}{}
      \event{b2}{r\NOTEQ1\mid\DW{x}{1}}{below=of b1}
    \end{tikzcenter}
    \biggm\|
    \begin{tikzcenter}[node distance=1em]
      \event{c1}{r\EQ1\mid\DW{x}{1}}{}
    \end{tikzcenter}
  \right)
\end{displaymathsmall}
With an internal read of $y$, we have:
\begin{displaymathsmall}
  \begin{tikzcenter}[node distance=1em]
    \event{a1}{\DR{x}{1}}{}
    \event{a2}{\DW{y}{1}}{below=of a1}
    \po{a1}{a2}
  \end{tikzcenter}
  \Biggm\|
    \begin{tikzcenter}[node distance=1em]
      \event{b1}{y\NOTEQ1\mid\DW{z}{1}}{}
      \event{b2}{\DW{x}{1}}{below=of b1}
    \end{tikzcenter}
\end{displaymathsmall}
but the precondition $y\NOTEQ1$ cannot be satisfied locally.
With an external read of $(\DR{y}{1})$, we can discharge the precondition, but in this
case, the predicate becomes $1\NOTEQ 1$. 
\begin{displaymathsmall}
  \begin{tikzcenter}[node distance=1em]
    \event{a1}{\DR{x}{1}}{}
    \event{a2}{\DW{y}{1}}{below=of a1}
    \po{a1}{a2}
  \end{tikzcenter}
  \Bigm\|  
    \begin{tikzcenter}[node distance=1em]
      \nonevent{b1}{1\NOTEQ1\mid\DW{z}{1}}{}
      \event{b2}{\DW{x}{1}}{below=of b1}
      \event{b0}{\DR{y}{1}}{left=of b1}
      \po{b0}{b1}
    \end{tikzcenter}
\end{displaymathsmall}

\paragraph{Causality Test Cases. } 
\citet{PughWebsite} developed a set of twenty {causality test cases} in the
process of revising the Java Memory Model (JMM)
\cite{Manson:2005:JMM:1047659.1040336}.  Using hand calculation, we have
confirmed that our model gives the desired result for all twenty cases,
unrolling loops as necessary.  Our model also gives the desired results for
all of the examples in \citet[\textsection 4]{DBLP:conf/esop/BattyMNPS15} and
\citet[\textsection 5.3]{SevcikThesis}.  Our model agrees with the JMM on the
``surprising and controversial behaviors'' of \citet[\textsection
8]{Manson:2005:JMM:1047659.1040336}. \textsection\ref{sec:examples} develops some of these examples.

\section{The Semantics of a Simple Concurrent Language}
\label{sec:model}

In the previous section, we described our semantics informally.  In this
section we firm things up.  In \textsection\ref{sec:data}, we describe data
models.  In \textsection\ref{sec:sets} we formalize some definitions for sets
of pomsets.  In both cases, the details are tedious and mundane.  The
exciting bits are in \textsection\ref{sec:combinators}, where we describe
combinators for sets of pomsets, and \textsection\ref{sec:semantics}, where
we use the combinators to define the semantics of a simple concurrent
language.
 
\begin{comment}
Differences with \cite{2019-sp}:
\begin{itemize}
\item We only consider pomsets where $\lt$-greater elements have stronger
  preconditions (Definition~\ref{def:mmpomset}).
\item We only consider pomsets that are coherent (Definition~\ref{def:mmpomset}).
\item Prefixing only introduces required order from read to write in item~\ref{pre-read}b (Definition~\ref{def:prefix}).
\item Prefixing allows writes to weaken preconditions in item~\ref{pre-write} (Definition~\ref{def:prefix}).
\item We provide semantics for computed addresses of memory locations.
\item The semantics of relaxed read introduces internal actions.
\item Restriction introduces internal actions.
\item The semantics of parallel composition requires that the constituent
  pomsets be location independent.  This ensures that in
  $\aCmd;(\bCmd_1\PAR\bCmd_2)$, that there are no internal reads from any
  write in $\aCmd$ to any read in any $\bCmd_i$.
\end{itemize}
\end{comment}

% \subsection{Background: 3-valued pomsets}
% \label{sec:pomsets}

% Structures similar to 3-valued pomsets have come up in many guises, for example
% rough sets~\cite{Pawlak1982} or ultrametrics over
% $\{0,{}^1\!/_2,1\}$. They correspond to axioms A1--A3 of Lamport's
% \emph{system executions}~\cite{DBLP:journals/dc/Lamport86}.
% They are the notion of pomset given by interpreting
% $\bEv\le\aEv$ in a 3-valued logic~\cite{Urquhart1986}. 




\subsection{Data models}
\label{sec:data}

A \emph{data model} consists of:
\begin{itemize}
\item a set of \emph{values} $\Val$, ranged over by
  $\aVal$, $\bVal$, $\dVal$ and $\cVal$,
\item a set of \emph{registers} $\Reg$, ranged over by
  $\aReg$ and $\bReg$,
\item a set of \emph{expressions} $\Exp$, ranged over by
  $\aExp$, $\bExp$, $\cExp$ and $\dExp$,
\item a set of \emph{memory locations} $\Loc$, ranged over by $\aLoc$ and
  $\bLoc$, 
\item a set of \emph{logical formulae} $\Formulae$, ranged over by
  $\aForm$ and $\bForm$, and
\item a set of \emph{actions} $\Act$, ranged over by $\aAct$ and $\bAct$.
\end{itemize}

Let $\aSub$ range over substitutions of the form
$\aForm[\aLoc/\aReg]$ or $\aForm[\bExp/\aLoc]$.

We require that data models satisfy the following:
\begin{itemize}
\item the sets of values, registers and memory locations are disjoint,
\item values include at least the constants $0$ and $1$,
\item expressions include at least registers and values,
\item memory locations have the form $\REF{\aVal}$,
\item expressions do \emph{not} include memory locations or the operator $\REF{\aExp}$,
\item formulae include at least $\TRUE$, $\FALSE$, and equalities of the form
  $(\aExp=\aVal)$, % and $(\REF{\aExp}=\aLoc)$,
\item formulae are closed under negation, conjunction, disjunction, and
  substitution\footnote{Since formulae are closed under substitutions of the
    form $\aForm[\aLoc/\aReg]$, they must include equalities of the form
    $(\aEExp=\aVal)$ and $(\REF{\aEExp}=\aLoc)$, where $\aEExp$ is an
    \emph{extended expression} that includes memory locations.  We elide the
    details.  By composition of the closure conditions, formulae must also be
    closed under that substitutions of the form
    $\aForm[\aExp/\aReg]=\aForm[\aLoc/\aReg][\aExp/\aLoc]$.}, and
\item there is a relation $\vDash$ between formulae.
\end{itemize}

We say that $\aForm$ is \emph{independent of $\aLoc$} whenever
$\aForm \vDash \aForm[\aVal/\aLoc] \vDash \aForm$ for every $\aVal$, and that
$\aForm$ is \emph{dependent on $\aLoc$} otherwise.  We say that $\aForm$ is
\emph{location independent} if it is independent of every location.

We say that $\aForm$ \emph{implies} $\bForm$ whenever $\aForm\vDash\bForm$,
that $\aForm$ is a \emph{tautology} whenever $\TRUE\vDash\aForm$, that
$\aForm$ is \emph{unsatisfiable} whenever $\aForm\vDash\FALSE$.

For the actions of a data model, we require that
\begin{itemize}
\item there are partial functions $\rreads$ and
  $\rwrites: \Act \fun (\Loc \times \Val)$,
\item there are sets $\Rel$ and $\Acq \subseteq\Act$, and
\item there is a function $\finternalize: (\Loc\times\Act) \fun \Act$ that
  satisfies the restrictions given below.
\end{itemize}

We say that $\aAct$ \emph{reads} $\aVal$ \emph{from} $\aLoc$ whenever
$\rreads(\aAct) = (\aLoc,\aVal)$, and that $\aAct$ \emph{writes} $\aVal$
\emph{to} $\aLoc$ whenever $\rwrites(\aAct) = (\aLoc,\aVal)$.
% Actions that
% read or write are \emph{external}, other actions are \emph{internal}.
% Actions in
% $\Ext=\fdom(\rreads)\cup\fdom(\rwrites)$ are \emph{external}, whereas those
% in $\Int=\Act\setminus\Ext$ are \emph{internal}.

We say that $\aAct$ is an \emph{acquire} if $\aAct\in\Acq$, and that $\aAct$
is a \emph{release} if $\aAct\in\Rel$.  
% We say that $\aAct$ is a
% \emph{synchronization} if it is either a release or an acquire.

We require that $\finternalize$ satisfy the following:
\begin{itemize}
\item $\finternalize(\aAct)$ reads $\aVal$ from $\aLoc$ exactly when $\aAct$ reads $\aVal$ from $\aLoc$,
\item $\finternalize(\aAct)$ writes $\aVal$ from $\aLoc$ exactly when $\aAct$ writes $\aVal$ from $\aLoc$,
%\item the codomain of $\finternalize$ includes only internal actions, %$\fcodom(\finternalize)\subseteq\Int$,
\item $\finternalize(\aAct)$ is an acquire exactly when $\aAct$ is an acquire, and 
\item $\finternalize(\aAct)$ is a release exactly when $\aAct$ is a release.
\end{itemize}

As noted in \textsection\ref{sec:model:intro}, our example language includes acquiring
read $(\DRAcq{\aLoc}{\aVal})$, relaxed read $(\DR{\aLoc}{\aVal})$, releasing
write $(\DWRel{\aLoc}{\aVal})$, and relaxed write $(\DW{\aLoc}{\aVal})$.
For each external action, we also define a corresponding internal action
%which replaces the letter $\mathsf{R}$ or $\mathsf{W}$ with $\tau$.
denoted by prefixing $\tau$.
For example, $(\iDRAcq{\aLoc}{\aVal})$ is an acquiring internal action, which
neither reads nor writes. In pictures, we draw internal actions grayed out,
rather than using $\tau$.  % For example, the ``read'' action is internal in:
% \begin{tikzdisplay}[node distance=1em]
%   \event{wx1}{\DW{x}{1}}{}
%   \internal{rx1}{\DR{x}{1}}{below right=of rx1}
%   \event{wy1}{\DW{y}{1}}{above right=of wy0}
%   \po{wx1}{wy1}
% \end{tikzdisplay}

We also include acquire-release fences of the form $(\DF)$.

% In examples, we use fence actions of the form $(\DF{\aF})$, where the annotation
% indicates that the fence is a release ($\FR$), an acquire ($\FA$) or both ($\FF$):
% \begin{displaymath}
%   \aF\BNFDEF\FR\BNFSEP\FA\BNFSEP\FF
% \end{displaymath}

% \subsection{3-valued pomsets with preconditions}

% Fix an alphabet $\Alphabet=(\Formulae\times\Act)$.

\subsection{The semantic domain}
\label{sec:sets}
Recall Definition~\ref{def:mmpomset} of memory model pomsets.

We lift terminology from logical formulae and actions to events. For example,
we say that $\aEv$ is unsatisfiable when $\labelingForm(\aEv)$ is unsatisfiable,
and that $\aEv$ is an acquire when $\labelingAct(\aEv)$ is an acquire.

% In this paper, we are not investigating microarchitecture.  So we make the
% global assumption formulae can only get stronger in dependent actions:

We give the semantics of programs as sets of pomsets.  Each pomset
$\aPS\in\sem{\aCmd}$ will represent a single execution of $\aCmd$.

We expect the sets of pomsets given by the semantics to be closed with
respect to \emph{isomorphism}, \emph{augmentation} and \emph{implication}.
\begin{definition}
  $\aPS'$ is an \emph{isomorphism} of $\aPS$ if there is a bijection
  $f:\Event\fun\Event'$ such that $\labeling(\aEv)=\labeling'(f(\aEv))$, and
  $\aEv\le\bEv$ iff $f(\aEv)\le'f(\bEv)$. %, and $\aEv\gtN\bEv$ iff $f(\aEv)\gtN'f(\bEv)$.

  $\aPS'$ is an \emph{augmentation} of $\aPS$ if $\Event'=\Event$,
  ${\labeling'}={\labeling}$, and ${\le'}\supseteq{\le}$. %, and ${\gtN'}\supseteq{\gtN}$.

  $\aPS'$ \emph{implies} $\aPS$ if $\Event'=\Event$, ${\le'}={\le}$,
  %${\gtN'}={\gtN}$,
  $\labelingAct'=\labelingAct$, and $\labelingForm'(\aEv)$
  implies $\labelingForm(\aEv)$ for all $\aEv\in\Event$.
\end{definition}
Each
$\aPS\in\sem{\aCmd}$ as a \emph{completed} execution.  So, we do not expect $\sem{\aCmd}$ to be prefixed closed.  However, implication
closure in a memory-model pomset does give something similar: any event
$\aEv$ can be given an unsatisfiable precondition, which means that every
event ordered after $\aEv$ must also be unsatisfiable, as per
Definition~\ref{def:mmpomset}.  Since unsatisfiable events are ignored by our model, this
provides a kind of prefix closure.
% \begin{definition}
%   $\aPS'$ is an \emph{augmentation} of $\aPS$ if $\Event'=\Event$, ${\labeling'}={\labeling}$,
%   ${\le}\subseteq{\le'}$, %$\aEv\le\bEv$ implies $\aEv\le'\bEv$,
%   and ${\gtN}\subseteq{\gtN'}$. %$\aEv\gtN\bEv$ implies $\aEv\gtN'\bEv$,
%   % $\labelingAct'=\labelingAct$, and  $\labelingForm'(\aEv)$ implies $\labelingForm(\aEv)$.
%   % if $\labeling(\aEv) = (\bForm \mid \bAct)$ then $\labeling'(\aEv) = (\bForm' \mid \bAct)$ where $\bForm'$ implies $\bForm$.
% \end{definition}

% Restriction also filters a set of pomsets; we have
% $(\nu\aLoc\st\aPSS)\subseteq\aPSS$.
% The definition requires that we define
% when a read is possible.

% \begin{definition}\label{def:rf}
%   In a pomset, $\aEv$ \emph{can read $\aLoc$ from} $\bEv$ whenever: 
%   \begin{itemize}
%   \item $\bEv \lt \aEv$,  
%   \item $\aEv$ implies $\bEv$,
%   \item $\bEv$ writes $\aVal$ to $\aLoc$,
%     and $\aEv$ reads $\aVal$ from $\aLoc$, and
%   \item if $\cEv$ writes to $\aLoc$
%     then either $\cEv \gtN \bEv$ or $\aEv \gtN \cEv$.
%   \end{itemize}
% \end{definition}

\subsection{Combinators}
\label{sec:combinators}
We give the semantics using combinators over sets of pomsets, defined below.
Using $\aPSS$ to range over sets of pomsets, these are:
\begin{itemize}
\item \emph{substitution} $\aPSS\aSub$, which applies the substitution to
  every precondition,
\item \emph{restriction} $\nu\aLoc\st\aPSS$, which internalizes $\aLoc$ for
  pomsets that are $\aLoc$-closed,
% \item \emph{guarding} $\aForm\guard\aPSS$, which filters $\aPSS$,
%   keeping pomsets where all events imply $\aForm$,
% \item \emph{independency filtering} $\Loc\guard\aPSS$, which filters
%   $\aPSS$, keeping pomsets all events are independent of every location,
% \item \emph{write filtering} $\DW{\aLoc}{}\guard\aPSS$, which filters
%   $\aPSS$, keeping pomsets that have an initial write to $\aLoc$,
\item \emph{composition} $\aPSS^1\parallel\aPSS^2$, which unions pomsets, allowing events to be merged, and
\item \emph{prefixing} $\aAct\prefix\aPSS$, which adds an event with action
  $\aAct$ to pomsets in $\aPSS$, ordering $\aAct$ before any $\aEv$ whose predicate
  depends on the value read by $\aAct$.
\end{itemize}
These operations are similar to those from models of concurrency such
as~\cite{Brookes:1984:TCS:828.833}.

We also define two filtering operations:
\begin{itemize}
\item \emph{guarding} $\aForm\guard\aPSS$, which
  keeps pomsets where all preconditions imply $\aForm$, and
\item \emph{independency filtering} $\Loc\guard\aPSS$, which keeps pomsets
  where all preconditions are location independent.
% \item \emph{write filtering} $\DW{\aLoc}{}\guard\aPSS$, which keeps pomsets
%   that have an initial write to $\aLoc$.
\end{itemize}

%% A write generates a write event that may be visible
%% to other threads.  A read may see a
%% thread-local value, or it may generate a read event that must be justified by
%% another thread.  In the latter case, occurrences of $\aReg$ are replaced with
%% $\aLoc$ (rather than $\aVal$) to ensure that dependencies are tracked
%% properly.  The subsequent substitution of $\aVal$ for $\aLoc$ occurs in
%% Definition~\ref{def:prefix} of prefixing.

% We have completed the formal definition of our model of speculative
% evaluation, and now turn to examples.

%\subsubsection{Substitution and Guarding} 

% Substitution updates the preconditions in a pomset, thus we expect the number
% of pomsets to be unchanged; in addition, the number of events in each of the
% pomsets is unchanged.

% Guarding and restriction filter a set of pomsets; we have
% $(\aForm\guard\aPSS)\subseteq\aPSS$ and
% $(\nu\aLoc\st\aPSS)\subseteq\aPSS$.

The definitions of substitution, restriction and the filtering operations %guarding and write filtering
are straightforward\footnote{We have chosen the definition of restriction for
  its simplicity.  It is worth noting, however, that our definition does not
  support renaming of variables.  In particular
  $(\nu\aLoc\st\aPSS\parallel(\nu\aLoc\st\bPSS))$ is generally not the same
  as $(\nu\aLoc\st\aPSS\parallel(\nu\bLoc\st\bPSS[\bLoc/\aLoc]))$.  To
  support renaming, $(\nu\aLoc\st\aPSS)$ would need to either remove or
  relabel events that mention $\aLoc$.}:
\begin{definition}
  %For a substitution $\aSub$, of the form $[\aLoc/\aReg]$ or $[\bExp/\aLoc]$,
  Let $\aPSS\aSub$ be the set $\aPSS'$ where $\aPS'\in\aPSS'$ whenever
there is $\aPS\in\aPSS$ such that:
$\Event' = \Event$,
${\le'} = {\le}$, 
%${\gtN'} = {\gtN}$,
and
$\labeling'(\aEv) = (\bForm\aSub \mid \aAct)$ when $\labeling(\aEv) = (\bForm \mid \aAct)$.
% \begin{itemize}
% \item if $\labeling(\aEv) = (\bForm \mid \aAct)$ then $\labeling'(\aEv) =
%   (\bForm\aSub \mid \aAct)$, and
% \item if $\labeling(\aEv) = (\bForm \mid \aSub)$ then $\labeling'(\aEv) = (\bForm\bSub \mid \aSub\bSub)$.
%\end{itemize}

  % Let $(\nu\aLoc\st\aPSS)$ be the subset of $\aPSS$ such that $\aPS\in\aPSS$ whenever
  % and $\aPS$ is $\aLoc$-coherent and $\aLoc$-closed.

Let $(\nu\aLoc\st\aPSS)$ be the set $\aPSS'$ where $\aPS'\in\aPSS'$ whenever
there is $\aPS\in\aPSS$ such that $\aPS$ is $\aLoc$-closed and:
$\Event' = \Event$,
${\le'} = {\le}$, 
%${\gtN'} = {\gtN}$,
and
$\labeling'(\aEv) = (\bForm \mid \finternalize(\aAct))$ when $\labeling(\aEv) = (\bForm \mid \aAct)$.

Let $(\aForm \guard \aPSS)$ be the subset of $\aPSS$ such that $\aPS\in\aPSS$ whenever
$\aForm$ implies $\labelingForm(\aEv)$, for every $\aEv\in\Event$. % if $\labelingAct(\aEv)$ writes.
% \begin{itemize}
% \item if $\labeling(\aEv) = (\bForm \mid \aActSub)$ then $\aForm$ implies $\bForm$.
% \end{itemize}

Let $(\Loc\guard \aPSS)$ be the subset of $\aPSS$ such that
$\aPS\in\aPSS$ whenever $\labelingForm(\aEv)$ is location independent, for every $\aEv\in\Event$.
%
% Let $(\DW{\aLoc}{} \guard \aPSS)$ be the subset of $\aPSS$ such that
% $\aPS\in\aPSS$ whenever there is some $\bEv$ that writes $\aLoc$ such that
% $\bEv\le\aEv$, for every release $\aEv\in\Event$.
\end{definition}
% Note that this liberalization allows reads more flexibility.  This is
% desirable in the language and architectural models, but not necessarily in
% microarchitectural models where reads are visible.


% \subsubsection{Restriction}
% \label{sec:restriction}


% We say that $\aPS' = \aPS\restrict{\Event'}$ when 
%  $\Event' \subseteq \Event$,
%  ${\labeling'} = {\labeling}\restrict{\Event'}$, 
%  ${\le'} = {\le}\restrict{\Event'}$, and
%  ${\gtN'} = {\gtN}\restrict{\Event'}$.

% \begin{definition}
%   Let $(\nu\aLoc\st\aPSS)$ be the subset of $\aPSS$ such that $\aPS\in\aPSS$ whenever
%   and $\aPS$ is $\aLoc$-coherent and $\aLoc$-closed.
%   % Let $(\nu\aLoc\st\aPSS)$ be the set $\aPSS'$ where $\aPS'\in\aPSS'$
%   % whenever there is $\aPS\in\aPSS$ such that $\aPS' = \aPS\restrict{\Event'}$
%   % and $\aPS'$ is $\aLoc$-coherent and $\aLoc$-closed.
%   % Let $(\nu\aLoc\st\aPSS)$ be the subset of $\aPSS$ such that $\aPS\in\aPSS$ whenever
%   % \begin{itemize}
%   % \item $\aEv$ is independent of $\aLoc$, and
%   % \item if $\aEv$ reads $\aLoc$, then there is some $\bEv$ such that $\aEv$ can read $\aLoc$ from $\bEv$.
%   % \end{itemize}
% \end{definition}
%This definition throws away useless writes.

%\subsubsection{Composition}
\begin{definition}
Let $\aPS' \in (\aPSS^1 \parallel \aPSS^2)$
whenever there are $\aPS^1 \in \aPSS^1$ and $\aPS^2 \in \aPSS^2$ such that:
\begin{itemize}
\item $\Event' = \Event^1 \cup \Event^2$,
\item ${\le'}\supseteq{\le^1}\cup{\le^2}$, %if $\aEv \le^1 \bEv$ or $\aEv \le^2 \bEv$ then $\aEv \le' \bEv$,
%\item ${\gtN'}\supseteq{\gtN^1}\cup{\gtN^2}$, and %if $\aEv \gtN^1 \bEv$ or $\aEv \gtN^2 \bEv$ then $\aEv \gtN' \bEv$,
% \item if $\labeling'(\aEv) = (\aForm' \mid \aAct)$ then either:
%   \begin{itemize}
%   \item $\labeling^1(\aEv) = (\aForm^1 \mid \aAct)$ and $\labeling^2(\aEv) = (\aForm^2 \mid \aAct)$
%     and $\aForm'$ implies $\aForm^1 \lor \aForm^2$,
%   \item $\labeling^1(\aEv) = (\aForm^1 \mid \aAct)$ and $\aEv \not\in \Event^2$
%     and $\aForm'$ implies $\aForm^1$, or
%   \item $\labeling^2(\aEv) = (\aForm^2 \mid \aAct)$ and $\aEv \not\in \Event^1$
%     and $\aForm'$ implies $\aForm^2$.
%   \end{itemize}
\item either
  % \begin{gather*}
  %   \labelingAct'(\aEv) = \labelingAct^1(\aEv) = \labelingAct^2(\aEv) \textand \labelingForm'(\aEv) \textimplies \labelingForm^1(\aEv) \lor \labelingForm^2(\aEv),\\
  %   \aEv \not\in \Event^2,\; \labelingAct'(\aEv) = \labelingAct^1(\aEv) \textand \labelingForm'(\aEv) \textimplies \labelingForm^1(\aEv),\; \textor\\    
  %   \aEv \not\in \Event^1,\; \labelingAct'(\aEv) = \labelingAct^2(\aEv) \textand \labelingForm'(\aEv) \textimplies \labelingForm^2(\aEv).
  % \end{gather*}
  \begin{itemize}
  \item $\labelingAct'(\aEv) = \labelingAct^1(\aEv) = \labelingAct^2(\aEv)
    \textand \labelingForm'(\aEv) \textimplies \labelingForm^1(\aEv) \lor \labelingForm^2(\aEv)$,
  \item $\labelingAct'(\aEv) = \labelingAct^1(\aEv),\;\; \aEv \not\in \Event^2\,
    \textand \labelingForm'(\aEv) \textimplies \labelingForm^1(\aEv),\; \textor$
  \item $\labelingAct'(\aEv) = \labelingAct^2(\aEv),\;\; \aEv \not\in \Event^1\,
    \textand \labelingForm'(\aEv) \textimplies \labelingForm^2(\aEv)$.
  \end{itemize}
\end{itemize}
\end{definition}
Composition is used in giving the semantics for conditionals and concurrency.
$\aPSS^1 \parallel \aPSS^2$ contains the union of pomsets from $\aPSS^1$ and
$\aPSS^2$, allowing overlap as long as they agree on actions. For example, if
$\aPSS^1$ and $\aPSS^2$ contain:
\begin{tikzdisplay}[node distance=1em]
  \event{a}{\aForm \mid \aAct}{}
  \event{b}{\bForm^1 \mid \bAct}{right=of a}
  \po{a}{b}
  \event{b2}{\bForm^2 \mid \bAct}{right=6em of b}
  \event{c2}{\cForm \mid \cAct}{right=of b2}
  \wk{b2}{c2}
\end{tikzdisplay}
then $\aPSS^1 \parallel \aPSS^2$ contains:
\begin{tikzdisplay}[node distance=1em]
  \event{a}{\aForm \mid \aAct}{}
  \event{b}{\bForm^1 \lor \bForm^2 \mid \bAct}{right=of a}
  \event{c}{\cForm \mid \cAct}{right=of b}
  \po{a}{b}
  \wk{b}{c}
\end{tikzdisplay}

% We use $\aPSS^1 \parallel \aPSS^2$ in defining the semantics of conditionals
% and concurrency.
% It contains the union of pomsets from $\aPSS^1$ and $\aPSS^2$,
% allowing overlap as long as they agree on actions. For example, if
% $\aPSS^1$ and $\aPSS^2$ contain:
% \begin{tikzdisplay}[node distance=1em]
%   \event{a}{\aForm \mid \aAct}{}
%   \event{b}{\bForm^1 \mid \bAct}{right=of a}
%   \po{a}{b}
% \end{tikzpicture}\qquad\qquad\begin{tikzpicture}[node distance=1em]
%   \event{b}{\bForm^2 \mid \bAct}{}
%   \event{c}{\cForm \mid \cAct}{right=of b}
%   \wk{b}{c}
% \end{tikzdisplay}
% then $\aPSS^1 \parallel \aPSS^2$ contains:
% \begin{tikzdisplay}[node distance=1em]
%   \event{a}{\aForm \mid \aAct}{}
%   \event{b}{\bForm^1 \lor \bForm^2 \mid \bAct}{right=of a}
%   \event{c}{\cForm \mid \cAct}{right=of b}
%   \po{a}{b}
%   \wk{b}{c}
% \end{tikzdisplay}


%\subsubsection{Prefixing}
\begin{definition}
  \label{def:prefix}
Let $\aAct \prefix \aPSS$ be the set $\aPSS'$ where $\aPS'\in\aPSS'$ whenever
there is $\aPS\in\aPSS$ such that:
\begin{enumerate}
\item\label{pre-E} $\Event' = \Event \cup \{\cEv\}$,
\item\label{pre-le} ${\le'}\supseteq{\le}$, % if $\bEv \le \aEv$ then $\bEv \le' \aEv$,
%\item\label{pre-gtN} ${\gtN'}\supseteq{\gtN}$, %if $\aEv \gtN \bEv$ then $\aEv \gtN' \bEv$,
% \item $\labelingAct'(\cEv) = \aAct$, 
% \item if $\labeling(\aEv) = (\bForm \mid \bAct)$ then $\labeling'(\aEv) =
%   (\bForm' \mid \bAct)$, where:
%   \begin{itemize}
%   \item if $\aAct$ is an acquire then $\bForm'$ is independent of every $\bLoc$,
%   \item if $\aAct$ does not read then $\bForm'$ implies $\bForm$,
%   \item if $\aAct$ reads then $\aVal$ from $\aLoc$ then
%     \begin{itemize}
%     \item $\bForm'$ implies $\bForm[\aVal/\aLoc]$, and
%     \item either $\bForm'$ implies $\bForm$ or $\cEv\lt'\aEv$, 
%     \end{itemize}
%   \end{itemize}
% \item if $\labelingAct(\aEv) = \bAct$ then:
%   \begin{itemize}
%   \item if $\aAct$ is an acquire or $\bAct$ is a release then $\cEv \lt' \aEv$, 
%   \item if $\aAct$ and $\bAct$ both touch the same location and one is a write,
%     then $\cEv \gtN' \aEv$, and
%   \end{itemize}
\item\label{pre-act} $\labelingAct'(\cEv) = \aAct$ and $\labelingAct'(\aEv) = \labelingAct(\aEv)$,
% \item\label{pre-implies} either $\labelingForm'(\aEv)$ implies $\labelingForm(\aEv)$ or
%   $\aAct$ is a read and if $\aEv$ is a write then $\cEv\lt'\aEv$,
% \item\label{pre-read} if $\aAct$ reads $\aVal$ from $\aLoc$ then
%    $\labelingForm'(\aEv)$ implies $\labelingForm(\aEv)[\aVal/\aLoc]$, %
\item\label{pre-nowrite} if $\aAct$ is neither reads nor writes, then $\labelingForm'(\aEv)$
  implies $\labelingForm(\aEv)$,
\item\label{pre-write} if $\aAct$ writes $\aVal$ to $\aLoc$ then
  $\labelingForm'(\aEv)$ implies $\labelingForm(\aEv)[\aVal/\aLoc]$,
  % either
  % \begin{enumerate}
  % \item[(\ref{pre-write}a)] $\labelingForm'(\aEv)$ implies $\labelingForm(\aEv)[\aVal/\aLoc]$, or
  % \item[(\ref{pre-write}b)] $\labelingForm'(\aEv)$ implies $\labelingForm(\aEv)$,
  % \end{enumerate}
\item\label{pre-read} if $\aAct$ reads $\aVal$ from $\aLoc$ then both
  \begin{enumerate}
  \item[(\ref{pre-read}a)] $\labelingForm'(\aEv)$ implies $\labelingForm(\aEv)[\aVal/\aLoc]$, and
  \item[(\ref{pre-read}b)] if $\aEv$ is a write then either $\cEv\lt'\aEv$
    or $\labelingForm'(\aEv)$ implies $\labelingForm(\aEv)$,
  \end{enumerate}
% \item if $\aAct$ does not read then $\labelingForm'(\aEv)$ implies $\labelingForm(\aEv)$,
% \item if $\aAct$ reads then either $\labelingForm'(\aEv)$ implies $\labelingForm(\aEv)$ or $\cEv\lt'\aEv$,  %
\item\label{pre-coherence} if $\aAct$ is a write that conflicts with $\labelingAct(\aEv)$ 
    then $\cEv \gtN' \aEv$,
\item\label{pre-sync} if $\aAct$ is an acquire or $\labelingAct(\aEv)$ is a release then $\cEv \lt' \aEv$, and
\item\label{pre-acquire} if $\aAct$ is an acquire then $\labelingForm(\aEv)$ is location independent.
% \item if $\aAct$ is a read but not a synchronization then either
%   $\labelingForm'(\cEv)$ is unsatisfiable or there is some $\aEv$ such
%   that $\labelingForm'(\aEv)$ does not imply $\labelingForm(\aEv)$.
% \item if $\labeling(\aEv) = (\bForm \mid \bAct)$ then $\labeling'(\aEv) =
%   (\bForm' \mid \bAct)$, where:
%   \begin{itemize}
%   \item if $\aAct$ is an acquire or $\bAct$ is a release then $\cEv \lt' \aEv$, 
%   \item if $\aAct$ is an acquire then $\bForm$ is independent of every $\bLoc$,
%   \item if $\aAct$ and $\bAct$ both touch the same location and one is a write,
%     then $\cEv \gtN' \aEv$, and
%   \item $\bForm'$ implies \(\left\{\begin{array}{l@{~}ll}
%     % \bForm[\aVal/\aLoc]                     & \mbox{if $\aAct$ reads $\aVal$ from $\aLoc$ and $\cEv\lt'\aEv$} & \textsc{[dependent read]} \\
%     % \bForm[\aVal/\aLoc] \text{ and } \bForm & \mbox{if $\aAct$ reads $\aVal$ from $\aLoc$}                  & \textsc{[independent read]} \\
%     % \bForm                                  & \mbox{otherwise}                                              & \textsc{[non-read]} \\        
%     \bForm[\aVal/\aLoc]                     \\\quad \mbox{if $\aAct$ reads $\aVal$ from $\aLoc$ and $\cEv\lt'\aEv$} \\\qquad \textsc{[dependent read]} \\[\jot]
%     \bForm[\aVal/\aLoc] \text{ and } \bForm \\\quad \mbox{if $\aAct$ reads $\aVal$ from $\aLoc$}                  \\\qquad \textsc{[independent read]} \\[\jot]
%     \bForm                                  \\\quad \mbox{otherwise}                                              \\\qquad \textsc{[non-read]} \\
%   \end{array}\right.\)
%   \end{itemize}
\end{enumerate}
\end{definition}
% The last condition ensures that useless reads are not included.
% Otherwise, $\labelingForm'(\cEv)$ is unconstrained.

% In order to keep augmentation closure, we need to keep the unsatisfiable
% elements in the set of pomsets.



$\aAct\prefix\aPSS$ adds a new event $\cEv$ with action $\aAct$ to each
pomset in $\aPSS$.  As in the definition of parallel composition, the
definition allows the new event to overlap with events in $\aPSS$ as long as
they agree on the action.  Overlapping of synchronization events is
disallowed by item~\ref{pre-sync}.

If $\cEv$ writes to a location that is also written by some $\aEv$ in $\aPSS$,
item~\ref{pre-coherence} introduces order between them: $\cEv \gtN \aEv$.  This
ensures that these writes cannot be given the reverse order in an augmentation.

If $\cEv$ reads from a location that occurs in the predicate of $\aEv$, then
prefixing introduces order from $\cEv$ to $\aEv$.
whose predicate depends on $\aLoc$. 
For example, if $\aPSS$ contains %a pomset with only
\begin{tikzinline}[node distance=1em]
  \event{b}{y=1 \mid \DW{x}{1}}{}
  \event{c}{x>0 \mid \DW{z}{1}}{right=of b}
\end{tikzinline}
then $(\DR{x}{1})\prefix\aPSS$ contains:
\begin{displaymathsmall}
\begin{tikzcenter}[node distance=1em]
  \event{a}{\DR{x}{1}}{}
  \event{b}{y=1 \mid \DW{x}{1}}{above right=0em and 2em of a}
  \event{c}{1>0 \mid \DW{z}{1}}{below right=0em and 2em of a}
  \po{a}{c}
  \wk{a}{b}
\end{tikzcenter}
\qquad\text{and}\qquad
\begin{tikzcenter}[node distance=1em]
  \event{a2}{\DR{x}{1}}{right=4em of a}
  \event{b2}{y=1 \mid \DW{x}{1}}{above right=0em and 2em of a2}
  \event{c2}{x>0 \mid \DW{z}{1}}{below right=0em and 2em of a2}
  \wk{a2}{b2}
\end{tikzcenter}
\end{displaymathsmall}
In order to weaken the predicate on $(\DW{z}{1})$, item~\ref{pre-read}b
requires that we include the order from $(\DR{x}{1})$ to $(\DW{z}{1})$.
If the precondition on $(\DW{z}{1})$ in $\aPSS$ was $x<0$, then, by
item~\ref{pre-read}, all preconditions for $(\DW{z}{1})$ in
$(\DR{x}{1})\prefix\aPSS$ must be equivalent to $\FALSE$, regardless of
the ordering of the events.

% For example, if $\aAct$ and $\bAct$ write to the same location, $\aAct$ reads
% $\aVal$ from $\aLoc$, $\bForm$ is independent of $\aLoc$, and $\aPSS$
% contains:
% $\footnotesize\begin{tikzpicture}[baselinecenter,node distance=1em]
%   \event{b}{\bForm \mid \bAct}{}
%   \event{c}{\cForm \mid \cAct}{right=of b}
% \end{tikzpicture}$
% then $\aAct\prefix\aPSS$ contains:
% \begin{tikzdisplay}[node distance=1em]
%   \event{a}{\aForm \mid \aAct}{}
%   \event{b}{\bForm \mid \bAct}{right=of a}
%   \event{c}{\cForm[\vec\aVal/\vec\aLoc] \mid \cAct}{right=of b}
%   \po[out=25,in=155]{a}{c}
%   \wk{a}{b}
%   \po{b}{c}
% \end{tikzdisplay}

% We say $\aEv$ \emph{depends on} $\cEv$ if
% $\labeling(\aEv) = (\bForm \mid \dontcare)$,
% $\labeling(\cEv) = (\dontcare \mid \aSub)$,
% and $\bForm$ depends on $\aSub$.

% We say $\aEv$ \emph{conflicts with}  $\bEv$ if
% $\labeling(\aEv) = (\dontcare \mid \aAct)$,
% $\labeling(\cEv) = (\dontcare \mid \bAct)$,
% $\aAct$ and $\bAct$ touch the same location, and either
% $\aAct$ or $\bAct$ is a write.


Item~\ref{pre-sync} ensures that events are ordered before a release and
after an acquire.

Item~\ref{pre-acquire} filters the executions of $\aPSS$, ensuring that
thread-local reads do not cross acquire actions.  This prevents bad executions like the following, which violate \drfsc. 
\begin{displaymath}
  x\GETS1 \SEMI
  a\REL\GETS1 \SEMI
  \IF{b\ACQ}\THEN  \aReg\GETS x\SEMI y\GETS\aReg \FI
  \PAR
  \IF{a\ACQ}\THEN  x\GETS 2\SEMI b\REL\GETS1 \FI
\end{displaymath}
\begin{tikzdisplay}[node distance=1em]
  \event{a1}{\DW{x}{1}}{}
  \event{a2}{\DWRel{a}{1}}{right=of a1}
  \po{a1}{a2}
  \event{b3}{\DRAcq{a}{1}}{below=of a2}
  \rf{a2}{b3}
  \event{b4}{\DW{x}{2}}{right=of b3}
  \po{b3}{b4}
  \event{b5}{\DWRel{b}{1}}{right=of b4}
  \po{b4}{b5}
  \event{a6}{\DRAcq{b}{1}}{above=of b5}
  \rf{b5}{a6}
  \internal{a7}{\DR{x}{1}}{right=of a6}
  \graypo{a6}{a7}
  \event{a8}{\DW{y}{1}}{right=of a7}
  \graypo{a7}{a8}
  %\po[out=20,in=160]{a6}{a8}
\end{tikzdisplay}
In item~\ref{pre-acquire}, we do not require that $\bForm'$ is independent of
every $\bLoc$; were we to require this, the definition would not be augment closed.

The following lemma is immediate from the definitions.
\begin{lemma}
  \label{lem:monotone}
  All combinators are monotone with respect to subset order.  
% For example,  $\aAct\prefix\aPSS \subseteq \aAct\prefix\aPSS'$ 
% whenever $\aPSS\subseteq\aPSS'$.
%   Suppose
%   $\aPSS'\supseteq\aPSS$, $\aPSS_1'\supseteq\aPSS_1$ and
%   $\aPSS_2'\supseteq\aPSS_2$.  Then we have the following.
% \begin{itemize}
% \item $\aPSS'\aSub \supseteq \aPSS\aSub$,
% \item $\nu\aLoc\st\aPSS' \supseteq \nu\aLoc\st\aPSS$,
% \item $\aForm\guard\aPSS' \supseteq \aForm\guard\aPSS$,
% \item $\Loc\guard\aPSS' \supseteq \Loc\guard\aPSS$,
% \item $\DW{\aLoc}{}\guard\aPSS' \supseteq \DW{\aLoc}{}\guard\aPSS$,
% \item $\aPSS'_1\parallel\aPSS'_2 \supseteq \aPSS_1\parallel\aPSS_2$, and
% \item $\aAct\prefix\aPSS' \supseteq \aAct\prefix\aPSS$.
% \end{itemize}
\end{lemma}

\subsection{Semantics of programs}
\label{sec:semantics}


\begin{comment}
\footnote{We only consider executions where register state is empty in
  forked threads.  Given item~\ref{pre-acquire} of
  Definition~\ref{def:prefix}, a sufficient condition is that parallel
  composition is always preceded by an acquire fence, as in programs of the
  form:
  \begin{displaymath}
    \VAR\vec{\aLoc}\SEMI
    \vec{\aLoc}\GETS\vec{0}\SEMI
    \vec{\bLoc}\GETS\vec{0}\SEMI
    \FENCE\SEMI
    (\aCmd^1 \PAR \cdots \PAR \aCmd^n)
  \end{displaymath}
  where $\aCmd^1$, \ldots, $\aCmd^n$ do not include $\PAR$.  To avoid clutter
  in drawings, we often drop the explicit fence.}.
\end{comment}
% \begin{align*}
% \aCmd,\,\bCmd
% \BNFDEF& \SKIP \tag{No Operation}
% \\[-1ex]\BNFSEP& \FENCE\SEMI \aCmd \tag{Full fence}
% \\[-1ex]\BNFSEP& \REF{\cExp}\GETS\aExp\SEMI \aCmd \tag{Relaxed write to memory}
% \\[-1ex]\BNFSEP& \REF{\cExp}\REL\GETS\aExp\SEMI \aCmd \tag{Releasing write to memory}
% \\[-1ex]\BNFSEP& \aReg\GETS\REF{\cExp}\SEMI \aCmd \tag{Relaxed read from memory}
% \\[-1ex]\BNFSEP& \aReg\GETS\REF{\cExp}\ACQ\SEMI \aCmd
% \\[-1ex]\BNFSEP& \IF{\aExp} \THEN \aCmd \ELSE \bCmd \FI
% \\[-1ex]\BNFSEP& \aCmd \PAR \bCmd
% \\[-1ex]\BNFSEP& \VAR\aLoc\SEMI \aCmd
% \end{align*}


We consider a simple shared-memory concurrent language, with statements
defined as follows.
\begin{align*}
\aCmd,\,\bCmd
\BNFDEF& \SKIP
\BNFSEP \aCmd \PAR \bCmd
\BNFSEP \VAR\aLoc\SEMI \aCmd
\BNFSEP \FENCE\SEMI \aCmd
\BNFSEP \IF{\aExp} \THEN \aCmd \ELSE \bCmd \FI
\\
\BNFSEP& \aReg\GETS\aExp\SEMI \aCmd
\BNFSEP \aReg\GETS\REF{\cExp}\ACQ\SEMI \aCmd 
\BNFSEP \aReg\GETS\REF{\cExp}\SEMI \aCmd
\BNFSEP \REF{\cExp}\REL\GETS\aExp\SEMI \aCmd
\BNFSEP \REF{\cExp}\GETS\aExp\SEMI \aCmd
\end{align*}
We use common syntax sugar, such as \emph{extended expressions}, which include
memory locations.  For example, if the extended expression $\aEExp$ includes
a single occurrence of $\aLoc$, then $\bLoc\GETS\aEExp\SEMI \aCmd$ is
shorthand for $\aReg\GETS\aLoc\SEMI\bLoc\GETS\aEExp[\aReg/\aLoc]\SEMI \aCmd$.
Each occurrence of $\aLoc$ in an extended expression corresponds to an
independent read.  We also write
$\IF{\aExp} \THEN \aCmd^1 \ELSE \aCmd^2 \FI \SEMI \bCmd$ as shorthand for
$\IF{\aExp} \THEN \aCmd^1 \SEMI \bCmd\ELSE \aCmd^2 \SEMI \bCmd\FI$ and
$\VAR\aLoc\GETS\aExp\SEMI \aCmd$ as shorthand for
$\VAR\aLoc\SEMI\aLoc\GETS\aExp\SEMI \aCmd$.
% We write $\REL\aLoc\GETS\aExp\SEMI \aCmd$ as shorthand for $\FENCE_{\FR} \SEMI\aLoc\GETS\aExp\SEMI \aCmd$.
% We write $\ACQ\aReg\GETS\aLoc\SEMI \aCmd$ as shorthand for $\aReg\GETS\aLoc\SEMI \FENCE_{\FA}\SEMI \aCmd$.

% In Figure~\ref{fig:programs}, we give the semantics as sets of pomsets.  
% \begin{figure*}
The semantics of programs is as follows\footnote{Fork parallelism is easily expressible in this
  framework:
  $\sem{\aCmd \LPAR \bCmd} = \sem{\aCmd} \parallel \Loc \guard
  \sem{\bCmd}$.}:
\allowdisplaybreaks
\begin{align*}
  \sem{\SKIP} & =
  \{ \emptyset \}
  \\
  \sem{\aCmd \PAR \bCmd} & =
  \Loc \guard \sem{\aCmd} \parallel \Loc \guard \sem{\bCmd} 
  \\
  \sem{\VAR\aLoc\SEMI \aCmd} & =
  \nu \aLoc \st \sem{\aCmd}
  \\
  \sem{\FENCE\SEMI \aCmd} & =
  (\DF{}) \prefix \sem{\aCmd}
  \\  
  \sem{\IF{\aExp} \THEN \aCmd \ELSE \bCmd \FI} & =
  \bigl((\aExp \neq 0) \guard \sem{\aCmd}\bigr) \parallel \bigl((\aExp=0) \guard \sem{\bCmd}\bigr) 
  \\
  \sem{\aReg\GETS\aExp\SEMI \aCmd} & =
  \sem{\aCmd}[\aExp/\aReg] 
  \\
  \sem{\aReg\GETS\REFAcq{\cExp}\SEMI \aCmd} & =
  \textstyle\bigcup_{\aLoc=\REF{\cVal}}\; ({\cExp}=\cVal) \guard \textstyle\bigcup_\aVal\; (\DRAcq\aLoc\aVal) \prefix \sem{\aCmd}[\aLoc/\aReg] 
  \\
  \sem{\aReg\GETS\REF{\cExp}\SEMI \aCmd} & =
  \textstyle\bigcup_{\aLoc=\REF{\cVal}}\; ({\cExp}=\cVal) \guard \textstyle\bigcup_\aVal\; (\DR\aLoc\aVal) \prefix \sem{\aCmd}[\aLoc/\aReg] 
  \\ & \mkern2mu\cup \textstyle\bigcup_{\aLoc=\REF{\cVal}}\; ({\cExp}=\cVal) \guard \textstyle\bigcup_\aVal\; ({\aLoc}=\aVal) \guard (\iDR{\aLoc}{\aVal}) \prefix \sem{\aCmd}[\aLoc/\aReg]
  \\
  \sem{\REFRel{\cExp}\GETS\aExp\SEMI \aCmd} & =
  \textstyle\bigcup_{\aLoc=\REF{\cVal}}\; ({\cExp}=\cVal) \guard \textstyle\bigcup_\aVal\;  (\aExp=\aVal) \guard\bigl((\DWRel\aLoc\aVal) \prefix \sem{\aCmd}\bigr)[\aExp/\aLoc] 
  \\
  \sem{\REF{\cExp}\GETS\aExp\SEMI \aCmd} & =
  \textstyle\bigcup_{\aLoc=\REF{\cVal}}\; ({\cExp}=\cVal) \guard \textstyle\bigcup_\aVal\;  (\aExp=\aVal) \guard\bigl((\DW\aLoc\aVal) \prefix \sem{\aCmd}\bigr)[\aExp/\aLoc]
  \\ & \mkern2mu\cup \textstyle\bigcup_{\aLoc=\REF{\cVal}}\; ({\cExp}=\cVal) \guard \textstyle\bigcup_\aVal\; ({\aExp}=\aVal) \guard \bigl((\iDW{\aLoc}{\aVal}) \prefix \sem{\aCmd}\bigr)[\aExp/\aLoc]
  %\\ & \mkern2mu\cup \textstyle\bigcup_{\aLoc=\REF{\cVal}}\; ({\cExp}=\cVal) \guard \DW{\aLoc}{} \guard \sem{\aCmd}[\aExp/\aLoc]
  % \sem{\aLoc\GETS\aExp\SEMI \aCmd} & = \textstyle\bigcup_\aVal\; \bigl((\aExp=\aVal) \guard (\DW\aLoc\aVal) \prefix \sem{\aCmd}\bigr)[\aExp/\aLoc] \\
  % \sem{\REL\aLoc\GETS\aExp\SEMI \aCmd} & = \textstyle\bigcup_\aVal\; \bigl((\aExp=\aVal) \guard (\DWRel\aLoc\aVal) \prefix \sem{\aCmd}\bigr)[\aExp/\aLoc] \\
  % \sem{\aReg\GETS\REF{\cExp}\SEMI \aCmd} & = \textstyle\bigcup_\aLoc\; (\REF{\cExp}=\aLoc) \guard (\sem{\aCmd}[\aLoc/\aReg] \cup \textstyle\bigcup_\aVal\; (\DR\aLoc\aVal) \prefix \sem{\aCmd}[\aLoc/\aReg]) \\
  % \sem{\aReg\GETS\aLoc\SEMI \aCmd} & =  \sem{\aCmd}[\aLoc/\aReg] \cup \textstyle\bigcup_\aVal\; (\DR\aLoc\aVal) \prefix \sem{\aCmd}[\aLoc/\aReg] \\
  % \sem{\ACQ\aReg\GETS\aLoc\SEMI \aCmd} & =  \textstyle\bigcup_\aVal\; (\DRAcq\aLoc\aVal) \prefix \sem{\aCmd}[\aLoc/\aReg] \\
% \caption{Semantics of a concurrent shared-memory language}
% \label{fig:programs}
% \end{figure*}
\end{align*}

The semantics of relaxed reads is the union of two sets.  The first set adds
a read action to each pomset in $\sem{\aCmd}$; the second adds an internal
action.  Whereas read actions have values that are important in the
semantics, the values of internal actions are ignored.  Internal reads are used in the definition of data races in \textsection\ref{sec:sc}.
% discussion we refer to these respectively as \emph{explicit} and
% \emph{implicit} writes.  The semantics of relaxed reads is similar.  We
% collectively refer to these as explicit and implicit \emph{actions}.
Acquiring reads are never internal.

The rule for relaxed writes is similar. The internal write rule enables dead store elimination.  The internal write can only be applied to pomsets that satisfy the filter $\DW{\aLoc}{}\guard\sem{\aCmd}[\aExp/\aLoc]$ that  requires some write to $\aLoc$ that precedes every release.  Thus, internal write is only allowed if there is a subsequent write that masks the internal write. 

The write rule uses the substitution $[\aExp/\aLoc]$ with
precondition $\aExp=\aVal$, rather than using $[\aVal/\aLoc]$ directly.
To see the need for this, consider
$\sem{\IF{\bReg\EQ\aReg}\THEN \cLoc\GETS 1\FI}$,
which includes
\begin{math}
(\bReg=\aReg \mid \DW\cLoc1)
\end{math}.
Therefore
$\sem{\bReg\GETS\aLoc\SEMI\IF{\bReg\EQ\aReg}\THEN \cLoc\GETS 1\FI}$
includes
\begin{math}
  (\aLoc=\aReg \mid \DW\cLoc1)
\end{math}
and
$\sem{\aLoc\GETS\aReg\SEMI\bReg\GETS\aLoc\SEMI\IF{\bReg\EQ\aReg}\THEN \cLoc\GETS 1\FI}$
includes
\begin{math}
  (\aReg=\aReg \mid \DW\cLoc1)
\end{math}
which is independent of $\aReg$.
%
If we took the semantics of write to use $[\aVal/\aLoc]$, then we would end
up with pomsets of the form
\begin{math}
  (\aVal=\aReg \mid \DW\cLoc1)
\end{math}
which depend on $\aReg$.

Prefixing does not necessarily induce a
dependency, even for read actions where the read is used.  To see that this
is desirable, consider  the semantics of
$\bLoc\GETS0\SEMI\aReg\GETS\bLoc\SEMI\IF{\aReg\leq1}\THEN \aLoc\GETS 2\FI$.
To begin, not that 
$\sem{\IF{\aReg\leq1}\THEN \aLoc\GETS 2\FI}$ includes
\begin{tikzinline}[node distance=1em]
  \event{c}{\aReg\leq1 \mid \DW\aLoc2}{}
\end{tikzinline}
which depends on $\aReg$.
Then $\sem{\aReg\GETS\bLoc\SEMI\IF{\aReg\leq1}\THEN \aLoc\GETS 2\FI}$ includes
\begin{tikzdisplay}[node distance=1em]
    \event{b}{\DR\bLoc1}{}
    \event{c}{\bLoc\leq1 \mid \DW\aLoc2}{right=of b}
\end{tikzdisplay}
with unordered read and write.
Prefixing with a write to $\bLoc$, $\sem{\bLoc\GETS0\SEMI\aReg\GETS\bLoc\SEMI\IF{\aReg\leq1}\THEN \aLoc\GETS
  2\FI}$ discharges the precondition of the write to $\aLoc$, yielding
\begin{tikzinline}[node distance=1em]
    \event{a}{\DW\bLoc0}{}
    \event{b}{\DR\bLoc1}{right=of a}
    \event{c}{0\leq1 \mid \DW\aLoc2}{right=of b}
\end{tikzinline}
which is simply:
\begin{tikzdisplay}[node distance=1em]
    \event{a}{\DW\bLoc0}{}
    \event{b}{\DR\bLoc1}{right=of a}
    \event{c}{\DW\aLoc2}{right=of b}
\end{tikzdisplay}
Here the thread-local value of $\bLoc$ discharges the predicate.
Using the left-hand side of the read rule, the semantics of this program also includes
\begin{tikzinline}[node distance=1em]
    \event{a}{\DW\bLoc0}{}
    \event{c}{\DW\aLoc2}{right=of a}
\end{tikzinline}.

A variant which indicates the branch taken:
$\sem{\bLoc\GETS0\SEMI\aReg\GETS\bLoc\SEMI\IF{\aReg\leq1}\THEN
  \aLoc\GETS2\SEMI\cLoc\GETS\aReg\FI}$
includes
\begin{tikzdisplay}[node distance=1em]
    \event{a}{\DW\bLoc0}{}
    \event{b}{\DR\bLoc1}{right=of a}
    \event{c}{\DW\aLoc2}{right=of b}
    \event{d}{\DW\cLoc1}{right=of c}
    \po[bend left]{b}{d}
\end{tikzdisplay}
A program to that witnesses the independence of $\DR\bLoc1$ and $\DW\aLoc2$ is
\begin{math}
  \IF{\bLoc\EQ0}\THEN
    \IF{\aLoc\EQ2}\THEN
      \bLoc\GETS1\SEMI
      \IF{\cLoc\EQ1}\THEN\PASS\FI
    \FI
  \FI
\end{math}.
Putting these in parallel gives you:
\begin{tikzdisplay}[node distance=1em]
    \event{a}{\DW\bLoc0}{}
    \event{b}{\DR\bLoc1}{right=of a}
    \event{c}{\DW\aLoc2}{right=of b}
    \event{d}{\DW\cLoc1}{right=of c}
    \po[bend left]{b}{d}
    \event{a2}{\DR\bLoc0}{below=of a}
    \event{b2}{\DR\aLoc2}{right=of a2}
    \event{c2}{\DW\bLoc1}{right=of b2}
    \event{d2}{\DR\cLoc1}{right=of c2}
    \po{a2}{b2}
    \po{b2}{c2}
    \po[bend right]{b2}{d2}
    \rf{a}{a2}
    \rf{c}{b2}
    \rf{c2}{b}
    \rf{d}{d2}
\end{tikzdisplay}

% Local Variables:
% mode: latex
% TeX-master: "paper"
% End:
