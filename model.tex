\section{The Model}
\label{sec:model:intro}

We define the model and give the semantics of a concurrent language.  We
layer the presentation, beginning with a simple language that supports only
read and write operations.  Later, we define extensions that incorporate
address computation, fences, and read-modify-write operations.  As is common
for work on relaxed memory, we treat loops via unrolling: loops introduce
complexities---such as liveness and continuity---that are orthogonal to the
main topic of the paper.
\begin{comment}
https://preshing.com/20131125/acquire-and-release-fences-dont-work-the-way-youd-expect/

Cannot encode R/A actions with actions+fences...

A release operation prevents preceding memory operations from being delayed
past it (a;Rel =/=> Rel;a)
 
A release fence prevents preceding memory operations from being delayed past
subsequent writes (a;FR;w =/=> w;a;FR)

An acquire operation prevents subsequent memory operations from being advanced
before it (Acq;a =/=> a;Acq)

An acquire fence prevents subsequent memory operations from being advanced
before prior reads (r;FA;a =/=> FA;a;r)

https://www.modernescpp.com/index.php/fences-as-memory-barriers

StoreLoad: Full fence allows a store before to be reordered with respect to a
load after (wx;F;ry) ===> (ry;F;wx)

StoreLoad+LoadLoad: Release fence also allows (rx;FR;ry) ===> (ry;FR;rx)

StoreLoad+StoreStore: Acquire fence also allows (wx;FR;wy) ===> (wy;FR;wx)

LoadStore: No fence allows a prior load to reorder w.r.t. a subsequent store
(rx;FR;wy) =/=> (wy;FR;rx)

https://preshing.com/20120710/memory-barriers-are-like-source-control-operations/
Good news is that a fullFence does it.

Bizarrely, it seems this is not supported in C++... You have to go to assembly.
\end{comment}

\paragraph{Data models.}
A \emph{data model} consists of:
\begin{itemize}
\item a set of \emph{values} $\Val$, ranged over by
  $\aVal$, $\bVal$, $\dVal$ and $\cVal$,
\item a set of \emph{registers} $\Reg$, ranged over by
  $\aReg$ and $\bReg$,
\item a set of \emph{expressions} $\Exp$, ranged over by
  $\aExp$, $\bExp$, $\cExp$ and $\dExp$,
\item a set of \emph{memory locations} $\Loc$, ranged over by $\aLoc$ and
  $\bLoc$, 
\item a set of \emph{actions} $\Act$, ranged over by $\aAct$ and $\bAct$, and
\item a set of \emph{logical formulae} $\Formulae$, ranged over by
  $\aForm$ and $\bForm$.
\end{itemize}

Let $\aSub$ range over substitutions of the form
$[\aLoc/\aReg]$ or $[\bExp/\aLoc]$.

We require that data models satisfy the following:
\begin{itemize}
\item values, registers, and memory locations are disjoint,
\item values include at least the constants $0$ and $1$,
\item expressions include at least registers and values,
%\item memory locations have the form $\REF{\aVal}$,
\item expressions do \emph{not} include memory locations, % or the operator $\REF{\aExp}$,
\item formulae include at least $\TRUE$, $\FALSE$, and equalities of the form
  $(\aExp=\aVal)$, % and $(\REF{\aExp}=\aLoc)$,
\item formulae are closed under negation, conjunction, disjunction, and
  substitution\footnote{Since formulae are closed under substitutions of the
    form $\aForm[\aLoc/\aReg]$, they must include equalities of the form
    $(\aEExp=\aVal)$ and $(\REF{\aEExp}=\aLoc)$, where $\aEExp$ is an
    \emph{extended expression} that includes memory locations.  We elide the
    details.  By composition of the closure conditions, formulae must also be
    closed under that substitutions of the form
    $\aForm[\aExp/\aReg]=\aForm[\aLoc/\aReg][\aExp/\aLoc]$.}, and
\item there is a relation $\vDash$ between formulae.
\end{itemize}

We use expressions as formulas, coercing $\aExp$ to $\aExp\neq0$.

For the actions of a data model, we require that
% \begin{itemize}
% \item
  there are partial functions $\rreads$ and
  $\rwrites: \Act \fun (\Loc \times \Val_\bot)$, and
%\item
  there are sets $\Acq$ and $\Rel$ and $\SC \subseteq\Act$ such that
  $\SC\cap\rreads\subseteq\Acq$ and
  $\SC\cap\rwrites\subseteq\Rel$. %, and
% \item there is a function $\finternalize: \Act \fun \Act$ that
%   satisfies the restrictions given below.
%\end{itemize}
%
We say that $\aAct$ \emph{reads} $\aVal$ \emph{from} $\aLoc$ when
$\rreads(\aAct) = (\aLoc,\aVal)$, and that $\aAct$ \emph{writes} $\aVal$
\emph{to} $\aLoc$ when $\rwrites(\aAct) = (\aLoc,\aVal)$.
We say that $\aAct$ \emph{reads from} $\aLoc$ when
$\rreads(\aAct) = (\aLoc,\aVal)$, and that $\aAct$ \emph{writes to}
$\aLoc$ when $\rwrites(\aAct) = (\aLoc,\aVal)$, for some $\aVal$. % (possibly $\bot$).
%
% Actions that read or write values are \emph{external},
% actions that read or write $\bot$ are \emph{internal}.
% % Actions in
% % $\Ext=\fdom(\rreads)\cup\fdom(\rwrites)$ are \emph{external}, whereas those
% % in $\Int=\Act\setminus\Ext$ are \emph{internal}.
%
We say that $\aAct$ is an \emph{acquire} if $\aAct\in\Acq$, that $\aAct$
is a \emph{release} if $\aAct\in\Rel$, and that $\aAct$ is \emph{SC} if $\aAct\in\SC$.  
% % We say that $\aAct$ is a
% % \emph{synchronization} if it is either a release or an acquire.

% The actions listed above are \emph{external}.  Each external action has a
% corresponding \emph{internal} action, denoted by prefixing $\tau$.  Internal
% actions also read and write locations, just as external actions do,
% but are not used to model communication between threads,
% so we do not record their value.
% \footnote{Fences have a limited role in our
% discussion.  We inappropriately refer to them as synchronizations for
% simplicity.}.


Logical formulae include equations over locations and registers, such
$(\aLoc=1)$ and $(\aReg=\bReg+1)$.  Formulae are \emph{open}, in that
occurrences of register names and memory locations are subject to
substitutions of the form $\aForm[\aLoc/\aReg]$ and $\aForm[\bExp/\aLoc]$,
where $\aLoc$ is a memory location, $\aReg$ is a register and $\bExp$ is an
memory-location-free expression.  Actions are not subject to substitution.


% % We require that $\finternalize$ satisfy the following:
% % \begin{itemize}
% % \item $\finternalize(\aAct)$ reads $\bot$ from $\aLoc$ exactly when $\aAct$ reads from $\aLoc$,
% % \item $\finternalize(\aAct)$ writes $\bot$ to $\aLoc$ exactly when $\aAct$ writes to $\aLoc$,
% % %\item the codomain of $\finternalize$ includes only internal actions, %$\fcodom(\finternalize)\subseteq\Int$,
% % \item $\finternalize(\aAct)$ is an acquire exactly when $\aAct$ is an acquire, 
% % \item $\finternalize(\aAct)$ is a release exactly when $\aAct$ is a release, and
% % \item $\finternalize(\aAct)$ is SC exactly when $\aAct$ is SC. 
% % \end{itemize}

% As noted in \textsection\ref{sec:model:intro}, our example language includes
% SC read $(\DRSC{\aLoc}{\aVal})$, acquiring
% read $(\DRAcq{\aLoc}{\aVal})$, relaxed read $(\DR{\aLoc}{\aVal})$, SC write
% $(\DWSC{\aLoc}{\aVal})$, releasing
% write $(\DWRel{\aLoc}{\aVal})$, and relaxed write $(\DW{\aLoc}{\aVal})$.
% For each external action, we also define a corresponding internal action
% %which replaces the letter $\mathsf{R}$ or $\mathsf{W}$ with $\tau$.
% $(\iDRSC{\aLoc}{\aVal})$,
% $(\iDRAcq{\aLoc}{\aVal})$,
% $(\iDR{\aLoc}{\aVal})$,
% $(\iDWSC{\aLoc}{\aVal})$,
% $(\iDWRel{\aLoc}{\aVal})$, and
% $(\iDW{\aLoc}{\aVal})$.
% In pictures, we draw internal actions grayed out,
% rather than using $\bot$.  % For example, the ``read'' action is internal in:
% % \begin{tikzdisplay}[node distance=1em]
% %   \event{wx1}{\DW{x}{1}}{}
% %   \internal{rx1}{\DR{x}{1}}{below right=of rx1}
% %   \event{wy1}{\DW{y}{1}}{above right=of wy0}
% %   \po{wx1}{wy1}
% % \end{tikzdisplay}

%We also include acquire-release fences of the form $(\DF)$.

% In examples, we use fence actions of the form $(\DF{\aF})$, where the annotation
% indicates that the fence is a release ($\FR$), an acquire ($\FA$) or both ($\FF$):
% \begin{displaymath}
%   \aF\BNFDEF\FR\BNFSEP\FA\BNFSEP\FF
% \end{displaymath}

% \subsection{3-valued pomsets with preconditions}

% Fix an alphabet $\Alphabet=(\Formulae\times\Act)$.

For the formulae of the data model,
we say that $\aForm$ is \emph{independent of $\aLoc$} when
$\aForm \vDash \aForm[\aVal/\aLoc] \vDash \aForm$ for every $\aVal$, and that
$\aForm$ is \emph{dependent on $\aLoc$} otherwise.  We say that $\aForm$ is
\emph{location independent} if it is independent of every location.
%
We say that $\aForm$ \emph{implies} $\bForm$ when $\aForm\vDash\bForm$,
that $\aForm$ is a \emph{tautology} when $\TRUE\vDash\aForm$, and that
$\aForm$ is \emph{unsatisfiable} when $\aForm\vDash\FALSE$.

\paragraph{Example Language.}
Our example language include actions of the form
$(\DR[\amode]{\aLoc}{\aVal})$, which \emph{reads} value $\aVal$ from location
$\aLoc$ and $(\DW[\amode]{\aLoc}{\aVal})$, which \emph{writes} $\aVal$ to
$\aLoc$.
% The \emph{mode} $\amode$ is either \emph{relaxed} ($\modeRLX$),
% \emph{release-acquire} ($\modeRA$) or \emph{sequentially-consistent} ($\modeSC$).
The \emph{mode} $\amode \BNFDEF \modeRLX \BNFSEP \modeRA \BNFSEP \modeSC$ is
either \emph{relaxed}, \emph{release-acquire}, or
\emph{sequentially-consistent}.
As expected, $\modeRA$ and $\modeSC$ reads are acquires; $\modeRA$ and
$\modeSC$ writes are releases.
%We write $\modeREL$ and $\modeACQ$ as synonyms for $\modeRA$.

We elide the $\modeRLX$-mode annotation in examples.

\begin{comment}
\footnote{We only consider executions where register state is empty in
  forked threads.  Given item~\ref{pre-acquire} of
  Definition~\ref{def:prefix}, a sufficient condition is that parallel
  composition is always preceded by an acquire fence, as in programs of the
  form:
  \begin{displaymath}
    \VAR\vec{\aLoc}\SEMI
    \vec{\aLoc}\GETS\vec{0}\SEMI
    \vec{\bLoc}\GETS\vec{0}\SEMI
    \FENCE\SEMI
    (\aCmd^1 \PAR \cdots \PAR \aCmd^n)
  \end{displaymath}
  where $\aCmd^1$, \ldots, $\aCmd^n$ do not include $\PAR$.  To avoid clutter
  in drawings, we often drop the explicit fence.}.
\end{comment}
% \begin{align*}
% \aCmd,\,\bCmd
% \BNFDEF& \SKIP \tag{No Operation}
% \\[-1ex]\BNFSEP& \FENCE\SEMI \aCmd \tag{Full fence}
% \\[-1ex]\BNFSEP& \REF{\cExp}\GETS\aExp\SEMI \aCmd \tag{Relaxed write to memory}
% \\[-1ex]\BNFSEP& \REF{\cExp}\REL\GETS\aExp\SEMI \aCmd \tag{Releasing write to memory}
% \\[-1ex]\BNFSEP& \aReg\GETS\REF{\cExp}\SEMI \aCmd \tag{Relaxed read from memory}
% \\[-1ex]\BNFSEP& \aReg\GETS\REF{\cExp}\ACQ\SEMI \aCmd
% \\[-1ex]\BNFSEP& \IF{\aExp} \THEN \aCmd \ELSE \bCmd \FI
% \\[-1ex]\BNFSEP& \aCmd \PAR \bCmd
% \\[-1ex]\BNFSEP& \VAR\aLoc\SEMI \aCmd
% \end{align*}


The syntax of statements is as follows.
\begin{align*}
  % \amode \BNFDEF& \modeRLX \BNFSEP \modeRA \BNFSEP \modeSC
  % \\
\aCmd,\,\bCmd
\BNFDEF& \SKIP
\BNFSEP \aReg\GETS\aExp\SEMI \aCmd
\BNFSEP \VAR\aLoc\SEMI \aCmd
\\
\BNFSEP&\aCmd \PAR \bCmd
\BNFSEP \IF{\aExp} \THEN \aCmd \ELSE \bCmd \FI
\\
\BNFSEP& \aReg\GETS\aLoc^{\amode}\SEMI \aCmd 
\BNFSEP \aLoc^{\amode}\GETS\aExp\SEMI \aCmd
\end{align*}



We use common syntax sugar, such as \emph{extended expressions}, which include
memory locations.  For example, if the extended expression $\aEExp$ includes
a single occurrence of $\aLoc$, then $\bLoc\GETS\aEExp\SEMI \aCmd$ is
shorthand for $\aReg\GETS\aLoc\SEMI\bLoc\GETS\aEExp[\aReg/\aLoc]\SEMI \aCmd$.
Each occurrence of $\aLoc$ in an extended expression corresponds to an
independent read.

We write
$\IF{\aExp} \THEN \aCmd \FI$ as shorthand for
$\IF{\aExp} \THEN \aCmd\ELSE \SKIP\FI$ and
$\IF{\aExp} \THEN \aCmd^1 \ELSE \aCmd^2 \FI \SEMI \bCmd$ as shorthand for
$\IF{\aExp} \THEN \aCmd^1 \SEMI \bCmd\ELSE \aCmd^2 \SEMI \bCmd\FI$.
% and
% $\VAR\aLoc\GETS\aExp\SEMI \aCmd$ as shorthand for
% $\VAR\aLoc\SEMI\aLoc\GETS\aExp\SEMI \aCmd$.
% We write $\REL\aLoc\GETS\aExp\SEMI \aCmd$ as shorthand for $\FENCE_{\FR} \SEMI\aLoc\GETS\aExp\SEMI \aCmd$.
% We write $\ACQ\aReg\GETS\aLoc\SEMI \aCmd$ as shorthand for $\aReg\GETS\aLoc\SEMI \FENCE_{\FA}\SEMI \aCmd$.

\paragraph{The semantic domain.}
Our model is based on \emph{partially ordered multisets} (\emph{pomsets})~\cite{GISCHER1988199}.
\begin{definition}
  \label{def:mmpomset}
  A \emph{(memory model) pomset} is a tuple
  $(\Event, {\le}, %{\gtN},
  \labeling)$, such that
  \begin{itemize}
  \item $\Event$ is a set of \emph{states},
  \item $\labeling: \Event \fun (\Formulae\times\Act)$ is a \emph{labeling},
    from which we derive functions $\labelingForm:\Event\fun\Formulae$ and $\labelingAct:\Event\fun\Act$,
    % define $\labelingForm(\aEv)=\aForm$ and $\labelingAct(\aEv)=\aAct$ when
    % $\labelingForm(\aEv)=(\aForm\mid\aAct)$,
  \item ${\le} \subseteq (\Event\times\Event)$ is a partial order, and
  % \item ${\gtN} \subseteq (\Event\times\Event)$ is a partial order,
  % \item ${\le} \subseteq {\gtN}$ is a partial order, and
  \item if $\bEv\le\aEv$ then $\labelingForm(\aEv)$ implies
    $\labelingForm(\bEv)$.
  \end{itemize}
\end{definition}
We refer to ``memory model pomsets'' as ``{pomsets}''.
We write pairs in $(\Formulae\times\Act)$ as $(\aForm \mid \aAct)$.  We elide
$\aForm$ when it is a tautology.
%
We lift terminology from logical formulae and actions to events. For example,
we say that $\aEv$ is unsatisfiable when $\labelingForm(\aEv)$ is unsatisfiable,
and that $\aEv$ is an acquire when $\labelingAct(\aEv)$ is an acquire.
We write $\bEv\lt\aEv$ when $\bEv\le\aEv$ and $\bEv\neq\aEv$.

% In this paper, we are not investigating microarchitecture.  So we make the
% global assumption formulae can only get stronger in dependent actions:

We give the semantics of programs as sets of pomsets.  
%
We expect sets of pomsets given by the semantics to be closed with
respect to \emph{isomorphism}, \emph{augmentation} and \emph{implication}.
\begin{definition}
  $\aPS'$ is an \emph{isomorphism} of $\aPS$ if there is a bijection
  $f:\Event\fun\Event'$ such that $\labeling(\aEv)=\labeling'(f(\aEv))$, and
  $\aEv\le\bEv$ iff $f(\aEv)\le'f(\bEv)$. %, and $\aEv\gtN\bEv$ iff $f(\aEv)\gtN'f(\bEv)$.

  $\aPS'$ is an \emph{augmentation} of $\aPS$ if $\Event'=\Event$,
  ${\labeling'}={\labeling}$, and ${\le'}\supseteq{\le}$. %, and ${\gtN'}\supseteq{\gtN}$.

  $\aPS'$ \emph{implies} $\aPS$ if $\Event'=\Event$, ${\le'}={\le}$,
  %${\gtN'}={\gtN}$,
  $\labelingAct'=\labelingAct$, and $\labelingForm'(\aEv)$
  implies $\labelingForm(\aEv)$ for all $\aEv\in\Event$.
\end{definition}
Each pomset $\aPS\in\sem{\aCmd}$ is a \emph{completed} execution of $\aCmd$;
we sometimes refer to pomsets as \emph{executions}.  Because they are
completed executions, we do not expect $\sem{\aCmd}$ to be prefixed closed.
However, implication closure in a memory-model pomset does give something
similar: any event $\aEv$ can be given an unsatisfiable precondition, which
means that every event ordered after $\aEv$ must also be unsatisfiable, as
per Definition~\ref{def:mmpomset}.  In many applications of the model,
unsatisfiable events are ignored, thus providing a kind of prefix closure.
% \begin{definition}
%   $\aPS'$ is an \emph{augmentation} of $\aPS$ if $\Event'=\Event$, ${\labeling'}={\labeling}$,
%   ${\le}\subseteq{\le'}$, %$\aEv\le\bEv$ implies $\aEv\le'\bEv$,
%   and ${\gtN}\subseteq{\gtN'}$. %$\aEv\gtN\bEv$ implies $\aEv\gtN'\bEv$,
%   % $\labelingAct'=\labelingAct$, and  $\labelingForm'(\aEv)$ implies $\labelingForm(\aEv)$.
%   % if $\labeling(\aEv) = (\bForm \mid \bAct)$ then $\labeling'(\aEv) = (\bForm' \mid \bAct)$ where $\bForm'$ implies $\bForm$.
% \end{definition}

% Restriction also filters a set of pomsets; we have
% $(\nu\aLoc\st\aPSS)\subseteq\aPSS$.
% The definition requires that we define
% when a read is possible.

% \begin{definition}\label{def:rf}
%   In a pomset, $\aEv$ \emph{can read $\aLoc$ from} $\bEv$ when: 
%   \begin{itemize}
%   \item $\bEv \lt \aEv$,  
%   \item $\aEv$ implies $\bEv$,
%   \item $\bEv$ writes $\aVal$ to $\aLoc$,
%     and $\aEv$ reads $\aVal$ from $\aLoc$, and
%   \item if $\cEv$ writes to $\aLoc$
%     then either $\cEv \gtN \bEv$ or $\aEv \gtN \cEv$.
%   \end{itemize}
% \end{definition}

% \citet{2019-sp} define \emph{3-valued pomsets with preconditions} to model
% security flaws that arise from speculative evaluation in computer
% microarchitecture (such as Spectre \cite{DBLP:journals/corr/abs-1801-01203}).
% We build on their work to define a model of relaxed memory
% \emph{architecture}.  Interestingly, the model naturally captures multi-copy
% atomicity (\mca).

% In this section, we define the model, provide some intuitions about the
% semantics, and work through a series of illustrative examples.  We present
% precise details of the semantics in \textsection\ref{sec:model}.

We visualize a pomset as a directed graph where the nodes are drawn from
$\Event$, each node $\aEv$ is labeled with $\labeling(\aEv)$, and order is
drawn as an edge.  We elide events with unsatisfiable preconditions.  In
examples, we draw pomsets that are augmentation-minimal and
implication-minimal.  For example, the semantics of
\begin{math}
  \VAR\aLoc\SEMI(
  x\GETS 0\SEMI
  x\GETS 1
  \PAR
  y\GETS x)
  %\aReg\GETS x\SEMI y\GETS \aReg)
\end{math}
includes:
\begin{tikzdisplay}[node distance=1em]
  \event{wx0}{\DW{x}{0}}{}
  \event{wx1}{\DW{x}{1}}{right=of wx0}
  \event{rx1}{\DR{x}{1}}{right=2.5 em of wx1}
  \event{wy1}{\DW{y}{1}}{right=of rx1}
  \rf{wx1}{rx1}
  \wk{wx0}{wx1}
  \po{rx1}{wy1}
\end{tikzdisplay}
We visualize order using arrows that indicate the reason that the order
arises.
$\DW{x}{1}\xrf\DR{x}{1}$ is a \emph{reads-from} requirement: the read of $x$
must be fulfilled by a matching write.
$\DW{x}{0}\xwk\DW{x}{1}$ is a \emph{coherence} requirement: the write of $1$
must follow the write to $0$, since these are in program order.
$\DR{x}{1}\xpo\DW{y}{1}$ is a \emph{dependency} requirement: the write to $y$
depends on the read of $x$.
Although we use multiple arrows, we emphasize that they are all part
of the same $\le$ relation.

The logical formulae associated with events are \emph{preconditions}.
The following programs gives rise to the pomset below them, capture data and control dependencies, respectively.
\begin{align*}
  \begin{gathered}
    y\GETS r
    \\
    \hbox{\begin{tikzinline}[node distance=1em]
        \event{wy1}{r=1\mid\DW{y}{1}}{}
      \end{tikzinline}}
  \end{gathered}
  &&
  \begin{gathered}
    \IF{r<0}\THEN y\GETS1 \FI
    \\
    \hbox{\begin{tikzinline}[node distance=1em]
        \event{wy1}{r<0\mid\DW{y}{1}}{}
      \end{tikzinline}}
  \end{gathered}
\end{align*}

In \textsection\ref{sec:opt} we show that pomsets preconditions are closely
tied to Hoare logic.  For example, the pomset above left is equivalent to the
Hoare triple $\hoare{\aReg =1}{y\GETS \aReg}{y=1}$.

Prepending an assignment to $\aReg$, the rules of Hoare logic derive the
triple $\hoare{x =1}{\aReg\GETS x\SEMI y\GETS \aReg}{y =1}$.  In our
semantics, the assignment to $\aReg$ causes a substitution in the
precondition: %, changing the label to $(x=1\mid \DW{y}{1})$.
% At top level, we expect preconditions to be either tautological or
% unsatisfiable---true or false.
\begin{align*}
  \begin{gathered}
    r\GETS x\SEMI y\GETS r
    \\
    \hbox{\begin{tikzinline}[node distance=1em]
        \event{wy1}{x=1\mid\DW{y}{1}}{}
        % \event{rx1}{\DR{x}{1}}{left=of wy1}
        % \po{rx1}{wy1}
      \end{tikzinline}}
  \end{gathered}
  &&
  \begin{gathered}
    r\GETS x\SEMI\IF{r<0}\THEN y\GETS1 \FI
    \\
    \hbox{\begin{tikzinline}[node distance=1em]
        \event{wy1}{x<0\mid\DW{y}{1}}{}
        % \event{rx1}{\DR{x}{1}}{left=of wy1}
        % \po{rx1}{wy1}
      \end{tikzinline}}
  \end{gathered}
\end{align*}
Our semantics may optionally prepend a read event.  By committing to a specific
value, the read event allows us to weaken the precondition of events that
follow in pomset order:
\begin{align*}
  \begin{gathered}
    r\GETS x\SEMI y\GETS r
    \\
    \hbox{\begin{tikzinline}[node distance=1em]
        \event{wy1}{1=1\mid\DW{y}{1}}{}
        \event{rx1}{\DR{x}{1}}{left=of wy1}
        \po{rx1}{wy1}
      \end{tikzinline}}
  \end{gathered}
  &&
  \begin{gathered}
    r\GETS x\SEMI\IF{r<0}\THEN y\GETS1 \FI
    \\
    \hbox{\begin{tikzinline}[node distance=1em]
        \event{wy1}{1<0\mid\DW{y}{1}}{}
        \event{rx1}{\DR{x}{1}}{left=of wy1}
        \po{rx1}{wy1}
      \end{tikzinline}}
  \end{gathered}
\end{align*}
We drop preconditions on tautological events and cross out unsatisfiable
ones (which may are ignored):
\begin{align*}
  \begin{gathered}
    % r\GETS x\SEMI y\GETS r
    % \\
    \hbox{\begin{tikzinline}[node distance=1em]
        \event{wy1}{\DW{y}{1}}{}
        \event{rx1}{\DR{x}{1}}{left=of wy1}
        \po{rx1}{wy1}
      \end{tikzinline}}
  \end{gathered}
  &&
  \begin{gathered}
    % r\GETS x\SEMI\IF{r<0}\THEN y\GETS1 \FI
    % \\
    \hbox{\begin{tikzinline}[node distance=1em]
        \nonevent{wy1}{\DW{y}{1}}{}
        \event{rx1}{\DR{x}{1}}{left=of wy1}
        \po{rx1}{wy1}
      \end{tikzinline}}
  \end{gathered}
\end{align*}
By prefixing a read event, the precondition $x=1$ has moved from the sequential realm of
Hoare logic to the concurrent memory model.  Rather than a precondition that
must be satisfied, the resulting pomset has a read event that must be
\emph{fulfilled} by a matching write, as described below.

\paragraph{Semantics of the Example Language.}
In the remainder of this section, we explain the semantics of our example
language.  By far the most complex operators are the prefixing
operators---read and write---which introduce new actions.  We first discuss
the other operators.
% the orders are drawn with
% \citeauthor{DBLP:journals/dc/Lamport86}'s notation.  
% For example:
% \begin{tikzdisplay}[node distance=1em]
%   \event{rx1}{a}{}
%   \event{wy0}{b}{below right=of rx1}
%   \event{wy1}{c}{above right=of wy0}
%   \po{rx1}{wy0}
%   \po{rx1}{wy1}
%   \wk{wy0}{wy1}
% \end{tikzdisplay}
% is a visualization of the pomset where:
% \[\begin{array}{c}
%     E = \{ 0,1,2 \}
%     \quad
%     {\labeling} = \{(0,a),\,(1,b),\,(2,c)\}
%     \\
%     {\le} = \{(0,1),\,(0,2)\}\cup\{(0,0),\,(1,1),\,(2,2)\}
%     \quad
%     {\gtN} = {\le}\cup\{(2,3)\}
% \end{array}\]
% for example:


The semantics of $\SKIP$ and register assignment are simple.
\begin{definition}
Let $\aPSS\aSub$ be the set $\aPSS'$ where $\aPS'\in\aPSS'$ when
there is $\aPS\in\aPSS$ such that:
$\Event' = \Event$,
${\le'} = {\le}$, 
%${\gtN'} = {\gtN}$,
and
$\labeling'(\aEv) = (\bForm\aSub \mid \aAct)$ when $\labeling(\aEv) = (\bForm \mid \aAct)$.
\begin{align*}
  \sem{\SKIP} & \eqdef
  \{ \emptyset \}
  \\  
  \sem{\aReg\GETS\aExp\SEMI \aCmd} & \eqdef
  \sem{\aCmd}[\aExp/\aReg] 
\end{align*}
\end{definition}

\paragraph{Fulfillment and Location Binding.}
At the point a location is bound, every read must be \emph{fulfilled} by a
matching write.

\begin{definition}
  \label{def:rf}
  We say $\bEv$ \emph{fulfills $\aEv$ on $\aLoc$} if $\bEv$ \externally writes
  $\aVal$ to $\aLoc$, $\aEv$ \externally reads $\aVal$ from $\aLoc$,
  \begin{itemize}
  \item
    $\bEv \lt \aEv$, and
  \item
    if $\cEv$ \externally writes to $\aLoc$ then either $\cEv \gtN \bEv$ or $\aEv \gtN \cEv$.
  \end{itemize}

  A pomset is $\aLoc$-closed if every \external read on $\aLoc$ is fulfilled,
  and every event is independent of $\aLoc$.

  Let $(\nu\aLoc\st\aPSS)$ be the subset of $\aPSS$ such that $\aPS\in\aPSS$
  when and $\aPS$ is $\aLoc$-closed.
\begin{align*}
  \sem{\VAR\aLoc\SEMI \aCmd} & \eqdef
  \nu \aLoc \st \sem{\aCmd}  
\end{align*}

  A pomset is \emph{top-level} if it is $\aLoc$-closed for every location $\aLoc$.
\end{definition}

Fulfillment imposes order between any two writes that are read, but not
necessarily between unread writes.  Consider the following program and
top-level execution:
\begin{displaymath}
  x\GETS 3
  \PAR
  x\GETS 4
  \PAR
  r\GETS x\SEMI r\GETS x 
  \PAR
  x\GETS 1
  \PAR
  x\GETS2
\end{displaymath}
\begin{tikzdisplay}[node distance=1em]
  \event{wx1}{\DW{x}{1}}{}
  \event{wx2}{\DW{x}{2}}{below=of wx1}
  \event{rx1a}{\DR{x}{1}}{left=of wx1}
  \event{rx2a}{\DR{x}{2}}{left=of wx2}
  %\po{rx1a}{rx2a}
  \rf{wx1}{rx1a}
  \rf{wx2}{rx2a}
  \wk{rx1a}{wx2}
  \wk{wx1}{wx2}
  %\event{wx3}{\DW{x}{3}}{below left=-.2em and 1em of rx1a}
  \event{wx3}{\DW{x}{3}}{left=of rx1a}
  \event{wx4}{\DW{x}{4}}{left=of rx2a}
  \wk{rx1a}{wx3}
  \wk{rx2a}{wx3}
  %\wk{rx1a}{wx4}
  \wk{rx2a}{wx4}
\end{tikzdisplay}

The restriction of pomset order to conflicting events is called the
\emph{extended coherence order}, $\reco$.  This relation can always be
extended to totally order all conflicting events, as is common in hardware
memory models.

The definition of fulfillment and restriction are chosen to validate
\emph{scope extrusion}~\cite{Milner:1999:CMS:329902}:
$\sem{\VAR\aLoc\SEMI(\aCmd\PAR\bCmd)}=\sem{(\VAR\aLoc\SEMI\aCmd)\PAR\bCmd}$,
when $\bCmd$ does not mention~$\aLoc$.  However, they do not validate
renaming of locations: if $\bLoc\neq\aLoc$ then
$\sem{\VAR\aLoc\SEMI\aCmd)}\neq\sem{(\VAR\bLoc\SEMI\aCmd[\bLoc/\aLoc])}$,
even if $\bLoc$ does not appear in $\aCmd$.  This is consistent with our
support of address calculation, which is required by realistic memory
allocators.
%$\nu\aLoc\st\aPSS\neq \nu\bLoc\st\aPSS[\bLoc/\aLoc]$ when $\bLoc\neq\aLoc$.
% $\nu\aLoc\st(\aPSS\parallel\nu\aLoc\st\bPSS)$ is generally
% not the same as
% $\nu\aLoc\st(\aPSS\parallel(\nu\bLoc\st\bPSS[\bLoc/\aLoc]))$.

\paragraph{Composition, Concurrency and Conditional.}

Composition is roughly union:
$(\aPS^1\cup\aPS^2) \in (\aPSS^1 \parallel \aPSS^2)$ when
$\aPS^1 \in \aPSS^1$ and $\aPS^2 \in \aPSS^2$.  Thus we have:
\begin{displaymath}
  \IF{r<0}\THEN y\GETS1\FI
  \PAR
  \IF{r\geq0}\THEN y\GETS1\FI
\end{displaymath}
\begin{tikzdisplay}[node distance=1em]
  \event{wx1}{r<0\mid\DW{x}{1}}{}
  \event{wx2}{r\geq0\mid\DW{x}{1}}{right=of wx1}
\end{tikzdisplay}
In addition, however, we allow that events with the same label may collapse,
taking the disjunction of their preconditions.  Thus, the semantics of this
program also includes:
% \begin{displaymath}
%   \IF{r<0}\THEN y\GETS1\FI
%   \PAR
%   \IF{r\geq0}\THEN y\GETS1\FI
% \end{displaymath}
\begin{tikzdisplay}[node distance=1em]
  \event{wx1}{r<0\lor r\geq0\mid\DW{x}{1}}{}
\end{tikzdisplay}
Collapsed events inherit order from both sides of the composition.
Thus, if
$\aPSS^1$ and $\aPSS^2$ contain:
\begin{tikzdisplay}[node distance=1em]
  \event{a}{\aForm \mid \aAct}{}
  \event{b}{\bForm^1 \mid \bAct}{right=of a}
  \po{a}{b}
  \event{b2}{\bForm^2 \mid \bAct}{right=3em of b}
  \event{c2}{\cForm \mid \cAct}{right=of b2}
  \po{b2}{c2}
\end{tikzdisplay}
then $\aPSS^1 \parallel \aPSS^2$ contains:
\begin{tikzdisplay}[node distance=1em]
  \event{a}{\aForm \mid \aAct}{}
  \event{b}{\bForm^1 \lor \bForm^2 \mid \bAct}{right=of a}
  \event{c}{\cForm \mid \cAct}{right=of b}
  \po{a}{b}
  \po{b}{c}
\end{tikzdisplay}

Concurrency and conditional execution are defined using composition,
validating the equation:
\begin{displaymath}
  \sem{\IF{\aExp} \THEN \aCmd \ELSE \bCmd \FI}
  =
  \sem{\IF{\aExp} \THEN \aCmd \FI \PAR \IF{\lnot\aExp} \THEN \bCmd \FI}
\end{displaymath}
\begin{definition}
Let $\aPS' \in (\aPSS^1 \parallel \aPSS^2)$
when there are $\aPS^1 \in \aPSS^1$ and $\aPS^2 \in \aPSS^2$ such that
% \begin{itemize}
% \item
  $\Event' = \Event^1 \cup \Event^2$,
%\item
  ${\le'}\supseteq{\le^1}\cup{\le^2}$, and
% if $\aEv \le^1 \bEv$ or $\aEv \le^2 \bEv$ then $\aEv \le' \bEv$,
%\item ${\gtN'}\supseteq{\gtN^1}\cup{\gtN^2}$, and %if $\aEv \gtN^1 \bEv$ or $\aEv \gtN^2 \bEv$ then $\aEv \gtN' \bEv$,
%\item
  either
  \begin{align*}
    \;\;&\labelingAct'(\aEv) = \labelingAct^1(\aEv) = \labelingAct^2(\aEv) \textand \labelingForm'(\aEv) \textimplies \labelingForm^1(\aEv) \lor \labelingForm^2(\aEv),\\[-1ex]
    &\;\;\aEv \not\in \Event^2,\; \labelingAct'(\aEv) = \labelingAct^1(\aEv) \textand \labelingForm'(\aEv) \textimplies \labelingForm^1(\aEv),\; \textor\\[-1ex]
    &\;\;\aEv \not\in \Event^1,\; \labelingAct'(\aEv) = \labelingAct^2(\aEv) \textand \labelingForm'(\aEv) \textimplies \labelingForm^2(\aEv).
  \end{align*}
  Let $(\aForm \guard \aPSS)$ be the subset of $\aPSS$ such that
  $\aPS\in\aPSS$ when $\aForm$ implies $\labelingForm(\aEv)$, for every
  $\aEv\in\Event$. % if $\labelingAct(\aEv)$ writes.
  % \begin{itemize}
  % \item $\labelingAct'(\aEv) = \labelingAct^1(\aEv) = \labelingAct^2(\aEv)
  %   \textand \labelingForm'(\aEv) \textimplies \labelingForm^1(\aEv) \lor \labelingForm^2(\aEv)$,
  % \item $\labelingAct'(\aEv) = \labelingAct^1(\aEv),\;\; \aEv \not\in \Event^2\,
  %   \textand \labelingForm'(\aEv) \textimplies \labelingForm^1(\aEv),\; \textor$
  % \item $\labelingAct'(\aEv) = \labelingAct^2(\aEv),\;\; \aEv \not\in \Event^1\,
  %   \textand \labelingForm'(\aEv) \textimplies \labelingForm^2(\aEv)$.
  % \end{itemize}
%\end{itemize}
\begin{align*}
  \sem{\aCmd \PAR \bCmd} & \eqdef
  %\Loc\guard \sem{\aCmd} \parallel \Loc\guard \sem{\bCmd} 
  \sem{\aCmd} \parallel \sem{\bCmd} 
  \\
  \sem{\IF{\aExp} \THEN \aCmd \ELSE \bCmd \FI} & \eqdef
  \bigl(\aExp \guard \sem{\aCmd}\bigr) \parallel \bigl(\lnot\aExp \guard \sem{\bCmd}\bigr) 
  % \bigl((\aExp \neq 0) \guard \sem{\aCmd}\bigr) \parallel \bigl((\aExp=0) \guard \sem{\bCmd}\bigr) 
\end{align*}
\end{definition}
Thus, the semantics of
 $\IF{\aReg}\THEN y\GETS1 \ELSE y\GETS1\FI$ 
includes:
\begin{tikzdisplay}[node distance=1em]
  \event{wy1}{\aReg\lor\lnot\aReg\mid \DW{y}{1}}{}
\end{tikzdisplay}
This weakening of the precondition of an event that occurs on both sides of a
conditional can be seen in the inference:
\begin{displaymath}
  \frac{
    \hoare{\aExp}{\aCmd}{\bForm}
    \quad
    \hoare{\lnot\aExp}{\bCmd}{\bForm}
  }{
    \hoare{\aExp\lor \lnot\aExp}{\IF{\aExp}\THEN \aCmd \ELSE \bCmd\FI}{\bForm}
  }
\end{displaymath}


\paragraph{SC Prefixing without Dependencies.}
Because the definition of prefixing is somewhat complex, we start with very
restrictive assumptions and then generalize.

For programs without data or control dependencies, the
simplest candidate definitions of read and write
result in top-level executions being sequentially consistent.
\begin{candidate}
  \label{def:rw:sc1}
  Let $(\aForm \mid \aAct) \prefixsc \aPSS$ be the set $\aPSS'$ where
  $\aPS'\in\aPSS'$ when there is $\aPS\in\aPSS$ such that $\aPS'$ adds a
  new event with the given label that precedes all of the events in $\aPS$.
  \begin{align*}
    % \tag{$\dagger$}\label{sc-read}
    \sem{\aReg\GETS\aLoc^\amode\SEMI \aCmd} & =
    \textstyle\bigcup_\aVal\; (\DRmode\aLoc\aVal) \prefixsc \sem{\aCmd}[\aLoc/\aReg] 
    \\
    % \tag{$\ddagger$}\label{sc-write}
    \sem{\aLoc^\amode\GETS\aExp\SEMI \aCmd} & =
    \textstyle\bigcup_\aVal\; (\aExp=\aVal \mid \DWmode\aLoc\aVal) \prefixsc \sem{\aCmd}[\aExp/\aLoc]
  \end{align*}
\end{candidate}
For now, ignore the substitutions, which we explain in the subsections below.

The definition ensures that program order is included in the pomset
order.  Due to the requirements of fulfillment, we also have that $\reco$ is
included in pomset order.  As a result, all executions are sequentially consistent. 
For example, consider the following candidate execution for the
\emph{store buffering} litmus test:
\begin{gather*}
  x\GETS0\SEMI x\GETS1\SEMI\aReg\GETS y
  \PAR
  y\GETS0\SEMI y\GETS1\SEMI \aReg\GETS x
  \\
  \tag{SB}\label{SB}
  \hbox{\begin{tikzinline}[node distance=1em]
      \event{wx0}{\DW{x}{0}}{}
      \event{wy0}{\DW{y}{0}}{below=of wx0}
      \event{wx}{\DW{x}{1}}{right=of wx0}
      \event{wy}{\DW{y}{1}}{right=of wy0}
      \event{ry}{\DR{y}{0}}{right=of wx}
      \event{rx}{\DR{x}{0}}{right=of wy}
      \wk{wx0}{wx}
      \wk{wy0}{wy}
      \po{wy}{rx}
      \po{wx}{ry}
      \rf{wy0}{ry}
      \rf{wx0}{rx}
      \wk{ry}{wy}
      \wk{rx}{wx}
      % \po{rx}{wy}
    \end{tikzinline}}
\end{gather*}
The order from the reads of 0 to the writes of 1 is required by the
definition of fulfillment.  This candidate execution is \emph{not} a pomset 
due to the resulting cycle, and thus disallowed.

\paragraph{SC Prefixing with Dependencies.}
For SC executions with dependencies, we must allow preconditions to weaken in
some cases, and require them to strengthen in others.  Consider the following
example.  From Candidate~\ref{def:rw:sc1}, $\sem{y\GETS r}$ contains
$(r\EQ1\mid\DW{y}{1})$.  When prefixing a read event in
$\sem{r\GETS x\SEMI y\GETS r}$, the precondition can weaken when $1$ is read
and must strengthen when $0$ is read:
\begin{displaymath}
  \hbox{\begin{tikzinline}[node distance=1em]
      \event{wy1}{1=1\mid\DW{y}{1}}{}
      \event{rx1}{\DR{x}{1}}{left=of wy1}
      \po{rx1}{wy1}
    \end{tikzinline}}
  \qquad \qquad
  \hbox{\begin{tikzinline}[node distance=1em]
      \nonevent{wy1}{0=1\mid\DW{y}{1}}{}
      \event{rx}{\DR{x}{0}}{left=of wy1}
      \po{rx}{wy1}
    \end{tikzinline}}
\end{displaymath}

We arrive at the following definition.
\begin{definition}
  \label{def:pre-sc}
Let $(\aForm \mid \aAct) \prefixsc \aPSS$ be the set $\aPSS'$ where $\aPS'\in\aPSS'$ when
there is $\aPS\in\aPSS$ such that:
\begin{enumerate}
\item\label{pre-E} $\Event' = \Event \uplus \{\cEv\}$,
\item\label{pre-le} ${\le'}\supseteq{\le}$, % if $\bEv \le \aEv$ then $\bEv \le' \aEv$,
\item[3a.] $\labelingAct'(\cEv) = \aAct$,
\item[3b.] $\labelingForm'(\cEv)$ implies $\aForm$,
\item[4a.] $\labelingAct'(\aEv) = \labelingAct(\aEv)$,
\item[4b.] if $\cEv$ \externally reads $\aVal$ from $\aLoc$ then
  $\labelingForm'(\aEv)$ implies $\labelingForm(\aEv)[\aVal/\aLoc]$,
\item[4c.] if $\cEv$ does not \externally read then
  $\labelingForm'(\aEv)$ implies $\labelingForm(\aEv)$, and
\item[5.] $\cEv \lt' \aEv$, 
\end{enumerate}
\end{definition}
Item 1 introduces the new event\nofootnote{Item 5 ensures that
  $\cEv\notin\Event$.  We relax item 5 in Definition \ref{def:prefix},
  allowing $\cEv\in\Event$.}.  Item 5 ensures that program order is included in
pomset order.  Item 2 ensures that no order is removed.  Item 3 describes the
label of the new event $\cEv$.  Item 4 describes the label of old events
$\aEv$.

We can now explain the substitution in the semantics of read in Candidate~\ref{def:rw:sc1}.
% (The substitution in the semantics of write \eqref{sc-write} is not used here; we explain
% it after introducing \emph{internal reads}, below.)
Returning to our example, to compute
$\sem{r\GETS x\SEMI y\GETS r}$, we first compute
$(r\EQ1\mid\DW{y}{1})[x/r]$, then perform prefixing.  By 4b, reading $1$
performs the further substitution of $[1/x]$, allowing the precondition to
weaken.  Likewise, reading $0$ performs $[0/x]$, requiring the precondition
to strengthen.

\paragraph{General Prefixing.}

In order to allow executions such as \eqref{SB}, we relax item 5 of
Definition \ref{def:pre-sc} so that only some program order is {preserved}:
\begin{definition}
  \label{def:prefix}
Two actions \emph{conflict} if one writes a location and the other
either reads or writes the same location.

Let $(\aForm \mid \aAct) \prefix \aPSS$ be the set $\aPSS'$ where $\aPS'\in\aPSS'$ when
there is $\aPS\in\aPSS$ that satisfies the items 1-4 of
Definition \ref{def:pre-sc} and:
\begin{enumerate}
\item[5a.] if %$\cEv$ \externally reads and
  $\labelingForm'(\aEv)$ does not imply $\labelingForm(\aEv)$ and $\aEv$ writes, then
  $\cEv\lt'\aEv$,
\item[5b.] if $\cEv$ and $\aEv$ are \external actions in conflict,
    then $\cEv \gtN' \aEv$,
\item[5c.] if $\cEv$ is an acquire or $\aEv$ is a release, then $\cEv \lt' \aEv$, 
\item[5d.] if $\cEv$ is an SC write and $\aEv$ is an SC read, then $\cEv \lt' \aEv$, and
\item[6.] if $\cEv$ is an acquire then $\labelingForm(\aEv)$
  is location independent.
\end{enumerate}
\end{definition}
\begin{candidate}
  Replace $\prefixsc$ in Candidate~\ref{def:rw:sc1} by $\prefix$.
\end{candidate}
Item 5a captures read to write dependency\footnote{When $\cEv$ is not a read,
  4c trivially implies 5a.}; it only requires order when the precondition of
the write is weakened by prefixing.  Item 5b captures the coherence
requirement on actions that touch the same location.  Item 5c imposes the
order required by acquire and release actions.  Item 5d imposes the
additional order required by SC actions\footnote{Recall that SC reads are
  acquires and SC writes are releases.}.  (Item 6 is not used here; we explain
it after introducing \emph{internal reads}, below.)

Item 5c ensures access ensure correct publication.  For
example, the following candidate execution is disallowed.
\begin{gather*}
    x\GETS0\SEMI %y\GETS0\SEMI
    x\GETS 1\SEMI y \REL\GETS1 \PAR r\GETS y\ACQ; s\GETS x
    \\[-1ex]
    \hbox{\begin{tikzinline}[node distance=1em]
        \event{wx0}{\DW{x}{0}}{}
        \event{wx1}{\DW{x}{1}}{right=of wx0}
        \event{wy1}{\DWRel{y}{1}}{right=of wx1}
        \event{ry1}{\DRAcq{y}{1}}{right=2.5em of wy1}
        \event{rx0}{\DR{x}{0}}{right=of ry1}
        \po{wx1}{wy1}
        \po{ry1}{rx0}
        \rf{wy1}{ry1}
        \rf[out=10,in=170]{wx0}{rx0}
        \wk{wx0}{wx1}
      \end{tikzinline}}
\end{gather*}
Since $(\DW x0) \gtN (\DW x1) \lt (\DR x0)$, this pomset does not satisfy the
requirements to fulfill the read.

%This example is also disallowed if we use $\modeSC$ rather than $\modeRA$.
Items 5d ensures that program order between SC operations is always
preserved.  Combined with the requirements for fulfillment, this is
sufficient to establish that programs with only SC access have only SC
executions; for example, execution \eqref{SB} is banned when all accesses are
$\modeSC$.  It is also immediate that SC actions can be totally ordered.
Just as SC access is simplified by \mca\ in \armeight, it is simplified by
the use of a single partial order in our model.
% \begin{gather*}
%   x\GETS0\SEMI x\GETS1\SEMI\aReg\GETS y
%   \PAR
%   y\GETS0\SEMI y\GETS1\SEMI \aReg\GETS x
%   \\
%   \hbox{\begin{tikzinline}[node distance=1em]
%       \event{wx0}{\DW{x}{0}}{}
%       \event{wy0}{\DW{y}{0}}{below=of wx0}
%       \event{wx}{\DW{x}{1}}{right=of wx0}
%       \event{wy}{\DW{y}{1}}{right=of wy0}
%       \event{ry}{\DR{y}{0}}{right=of wx}
%       \event{rx}{\DR{x}{0}}{right=of wy}
%       \wk{wx0}{wx}
%       \wk{wy0}{wy}
%       \po{wy}{rx}
%       \po{wx}{ry}
%       \rf{wy0}{ry}
%       \rf{wx0}{rx}
%       \wk{ry}{wy}
%       \wk{rx}{wx}
%       % \po{rx}{wy}
%     \end{tikzinline}}
% \end{gather*}

We relax program order on non-SC accesses in order to allow the outcome
of executions such as \eqref{SB}.
Order is relaxed between reads, between writes to different
locations, and from a read to an independent write:
\begin{gather*}
  \aReg\GETS x\SEMI \IF{\aReg}\THEN y\GETS1 \ELSE y\GETS1\FI
  \\
  \hbox{\begin{tikzinline}[node distance=1em]
      \event{wy1}{\DW{y}{1}}{}
      \event{rx1}{\DR{x}{1}}{left=of wy1}
    \end{tikzinline}}
\end{gather*}
Let $\aCmd$ be the program above.  The existence of this pomset is justified
by the triple
$\hoare{\TRUE}{\aCmd}{y=1}$.

Unordered actions can be scheduled freely.

% Program order is only imposed from read to write when the precondition of
% the write is weakened.  Thus we have:
% \begin{displaymath}
%   \aReg\GETS x\SEMI \IF{\aReg}\THEN y\GETS1 \ELSE y\GETS1\FI
% \end{displaymath}
% \begin{tikzdisplay}[node distance=1em]
%   \event{wy1}{\DW{y}{1}}{}
%   \event{rx1}{\DR{x}{1}}{left=of wy1}
% \end{tikzdisplay}
% Since no order is imposed, the two actions can be reordered.
% Let $\aCmd$ be the program above.  The existence of this pomset is justified
% by the triple
% $\hoare{\TRUE}{\aCmd}{y=1}$.

Item 5a imposes order from read to write when weakening the
precondition of the write via 4b:
\begin{gather*}
  r\GETS x\SEMI\IF{r\geq0}\THEN y\GETS1 \FI
  \\
  \hbox{\begin{tikzinline}[node distance=1em]
      \event{wy1}{1\geq0\mid\DW{y}{1}}{}
      \event{rx1}{\DR{x}{1}}{left=of wy1}
      \po{rx1}{wy1}
    \end{tikzinline}}
\end{gather*}


Item 4b \emph{allows} a precondition to weaken, but does not \emph{require} it.
No order is required in the following execution:
\begin{gather*}
  % r\GETS x\SEMI\IF{r\geq0}\THEN y\GETS1 \FI
  % \\
  \hbox{\begin{tikzinline}[node distance=1em]
      \event{wy1}{x\geq0\mid\DW{y}{1}}{}
      \event{rx1}{\DR{x}{1}}{left=of wy1}
      % \po{rx1}{wy1}
    \end{tikzinline}}
\end{gather*}
Let $\aCmd$ be the program above.  The existence of this pomset is justified
by the triple $\hoare{x=1}{\aCmd}{y=1}$, rather than by the value of the read
action.  Nonetheless, item 4b requires that the value of the read action must
be compatible with subsequent formulae.  In this example, the write
precondition must be unsatisfiable if a negative number is read:
\begin{gather*}
  % r\GETS x\SEMI\IF{r\geq0}\THEN y\GETS1 \FI
  % \\
  \hbox{\begin{tikzinline}[node distance=1em]
      \nonevent{wy1}{{-2}\geq0\mid\DW{y}{1}}{}
      \event{rx1}{\DR{x}{{-2}}}{left=of wy1}
      % \po{rx1}{wy1}
    \end{tikzinline}}
\end{gather*}

The substitution in the semantics of write (Candidate~\ref{def:rw:sc1}) allows
preconditions to be satisfied without imposing order:
\begin{gather*}
  x \GETS 0\SEMI r\GETS x\SEMI\IF{r\geq0}\THEN y\GETS1 \FI
  \\
  \hbox{\begin{tikzinline}[node distance=1em]
      \event{wy1}{0\geq0\mid\DW{y}{1}}{}
      \event{rx1}{\DR{x}{1}}{left=of wy1}
      \event{wx0}{\DW{x}{0}}{left=of rx1}
      \wk{wx0}{rx1}
    \end{tikzinline}}
\end{gather*}
In the JMM test cases \citep{PughWebsite}, such executions are justified via
compiler analysis, possibly in collusion with the scheduler:  If every racing
value read by a thread can be shown to satisfy a precondition, then the
precondition can be dropped.  For example, TC1 determines that the
following top-level execution should be allowed, as it is in our model:
\begin{gather*}
  x\GETS 0 \SEMI
  (r\GETS x\SEMI\IF{r\geq0}\THEN y\GETS1 \FI
  \PAR
  x\GETS y)
  \\
  \hbox{\begin{tikzinline}[node distance=1em]
  \event{wy1}{0\geq0\mid\DW{y}{1}}{}
  \event{rx1}{\DR{x}{1}}{left=of wy1}
  \event{wx0}{\DW{x}{0}}{below left=-.2em and 1em of rx1}
  \event{ry1}{\DR{y}{1}}{below=of rx1}
  \event{wx1}{\DW{x}{1}}{right=of ry1}
  \po{ry1}{wx1}
  \rf{wx1}{rx1}
  \rf{wy1}{ry1}
  \wk[out=10,in=155]{wx0}{wx1}
  \wk{wx0}{rx1}
    \end{tikzinline}}
\end{gather*}
Unlike \cite{DBLP:conf/lics/JeffreyR16}, this semantics is robust
with respect to the introduction of concurrent writes, as in TC9:
\begin{gather*}
  x\GETS 0 \SEMI
  (r\GETS x\SEMI\IF{r\geq0}\THEN y\GETS1 \FI
  \PAR
  x\GETS y
  \PAR
  x\GETS {-2})
  \\
  \hbox{\begin{tikzinline}[node distance=1em]
  \event{wy1}{0\geq0\mid\DW{y}{1}}{}
  \event{rx1}{\DR{x}{1}}{left=of wy1}
  \event{wx0}{\DW{x}{0}}{below left=-.2em and 1em of rx1}
  \event{ry1}{\DR{y}{1}}{below=of rx1}
  \event{wx1}{\DW{x}{1}}{right=of ry1}
  \event{wx2}{\DW{x}{{-2}}}{below right=-.2em and 1em of wy1}
  \po{ry1}{wx1}
  \rf{wx1}{rx1}
  \rf{wy1}{ry1}
  \wk[out=10,in=155]{wx0}{wx1}
  \wk{wx0}{rx1}
  \wk{wx1}{wx2}
    \end{tikzinline}}
\end{gather*}


\paragraph{Local reads and writes.}
In the semantics thus far, we have supposed that every read must be fulfilled
by a matching write action, and that the order between them must therefore be
part of the global pomset order.  This is overly restrictive for reads that
are fulfilled locally.  Consider Example 3.6 from
\citet{DBLP:journals/pacmpl/PodkopaevLV19}:
\begin{gather*}
  \tag{$\dagger$}\label{exlocal3.6}
  \aReg\GETS x\SEMI
  y\REL\GETS 1\SEMI
  \bReg\GETS y\SEMI
  z\GETS \bReg
  %z\GETS y
  \PAR
  x\GETS z
  \\
  \hbox{\begin{tikzinline}[node distance=1em]
  \event{a1}{\DR{x}{1}}{}
  \event{a2}{\DWRel{y}{1}}{right=of a1}
  \po{a1}{a2}
  \event{a3}{\DR{y}{1}}{right=of a2}
  \event{a4}{\DW{z}{1}}{right=of a3}
  \rf{a2}{a3}
  \po{a3}{a4}
  \event{b1}{\DR{z}{1}}{below=of a3}
  \event{b2}{\DW{x}{1}}{left=of b1}
  %\po[out=10,in=170]{a2}{a4}
  \po{b1}{b2}
  \rf{a4}{b1}
  \rf{b2}{a1}
    \end{tikzinline}}
\end{gather*}
This behavior is allowed by \armeight, but disallowed in the model presented
thus far, due to the evident cycle.

Thus, in the final definition, we drop the requirement for an explicit read
action when a relaxed read is matched by a local write.  Symmetrically, we
drop the requirement for an explicit write action when a relaxed write is
only used to match local reads.

There are two peculiarities in the definition of write which we explain at
the end of this subsection: First, we use the \emph{covering} operator
$(\relfilt{\aLoc} \aPSS)$ to limit the applicability of the internal write
rule.  Second, we use $\PAR$ rather than $\cup$ to combine pomsets created by
explicit write prefixing.

\begin{definition}
  \label{def:rw:local}
  Let $(\relfilt{\aLoc} \aPSS)$ be the subset of $\aPSS$ such that
  $\aPS\in\aPSS$
  when for every release $\aEv$ there is some $\bEv\le\aEv$
  such that $\bEv$ \externally  writes $\aLoc$.  % For $\amode\neq\modeRLX$, let
  % $(\relfilt {\aLoc} \aPSS)$ be the empty set.
  \begin{align*}
    \sem{\aReg\GETS\aLoc^\amode\SEMI \aCmd} & =
    \textstyle\bigcup_\aVal\; (\DRmode\aLoc\aVal) \prefix \sem{\aCmd}[\aLoc/\aReg]  
    \\[-.5ex] & \mkern2mu\cup
    %(\iDRmode{\aLoc}{}) \prefix
    \text{if $\amode\mathbin\neq\modeRLX$ then }\text{$\emptyset$ else }
    \sem{\aCmd}[\aLoc/\aReg]
    %\text{ else $\emptyset$}
    \\
    \sem{\aLoc^\amode\GETS\aExp\SEMI \aCmd} & =
    \textstyle\PAR_\aVal\; (\aExp=\aVal \mid \DWmode\aLoc\aVal) \prefix \sem{\aCmd}[\aExp/\aLoc]
    \\[-.5ex] & \mkern2mu\cup
    %(\iDWmode{\aLoc}{}) \prefix
    \text{if $\amode\mathbin\neq\modeRLX$ then }\text{$\emptyset$ else }
    (\relfilt{\aLoc} \sem{\aCmd}[\aExp/\aLoc])
    %\text{ else $\emptyset$}
  \end{align*}
\end{definition}

Revisiting \eqref{exlocal3.6}, note that $\sem{z\GETS \bReg}$ includes
$(\bReg=1\mid \DW{z}{1})$.  Applying the local read rule of
Definition~\ref{def:rw:local}, we have that
$\sem{\bReg\GETS y \SEMI z\GETS \bReg}$ includes $(y=1\mid \DW{z}{1})$.
Crucially, there is no explicit read event.  Further prepending
$\DWRel{y}{1}$, performs the substitution $(y=1\mid \DW{z}{1})[1/y]$
resulting in a tautology.  By removing the read event, the execution is
allowed:
\begin{gather*}
  % \aReg\GETS x\SEMI
  % y\REL\GETS 1\SEMI
  % z\GETS y
  % \PAR
  % x\GETS z
  % \\
  \hbox{\begin{tikzinline}[node distance=1em]
  \event{a1}{\DR{x}{1}}{}
  \event{a2}{\DWRel{y}{1}}{right=of a1}
  \po{a1}{a2}
  \internal{a3}{\DR{y}{1}}{right=of a2}
  \event{a4}{\DW{z}{1}}{right=of a3}
  %\rf{a2}{a3}
  \graypo{a3}{a4}
  \event{b1}{\DR{z}{1}}{below=of a3}
  \event{b2}{\DW{x}{1}}{left=of b1}
  %\po[out=10,in=170]{a2}{a4}
  \po{b1}{b2}
  \rf{a4}{b1}
  \rf{b2}{a1}
    \end{tikzinline}}
\end{gather*}
In drawings, we include a ``non-event''---dashed border---to mark each use of
a local rule.  In order to state \drfsc\ (\textsection\ref{sec:sc}), we make
such non-events explicit in \textsection\ref{sec:variants}.

Local reads invalidate \emph{thread inlining}, which is invalid in most relaxed models.
To see this, consider that the execution above is impossible for 
\begin{math}
  \aReg\GETS x\SEMI
  y\REL\GETS 1\PAR
  \bReg\GETS y\SEMI
  z\GETS \bReg
  \PAR
  x\GETS z.
\end{math}

We now address the two peculiarities in the definition.

First, the use of \textsf{cover} in Definition~\ref{def:rw:local} prevents
the local write rule from applying immediately preceding a release.  This
prevents bad executions such as:
\begin{gather*}
  x\GETS 1\SEMI
  x\GETS 2\SEMI
  y\REL\GETS 1
  \PAR
  \aReg\GETS y\ACQ \SEMI \bReg\GETS x
  \\[-1ex]
  \hbox{\begin{tikzinline}[node distance=1em]
  \event{a1}{\DW{x}{1}}{}
  \internal{a2}{\DW{x}{2}}{right=of a1}
  \graywk{a1}{a2}
  \event{a3}{\DWRel{y}{1}}{right=of a2}
  \graypo{a2}{a3}
  \event{b1}{\DRAcq{y}{1}}{right=of a3}
  \rf{a3}{b1}
  \event{b2}{\DR{x}{1}}{right=of b1}
  \po{b1}{b2}
  \rf[out=10,in=170]{a1}{b2}
    \end{tikzinline}}
\end{gather*}

Second, the use of $\PAR$ to combine pomsets ensures that the semantics
validates the disjunctive rule in Hoare logic:
\begin{displaymath}
  \frac{
    \hoare{\aForm^1}{\aCmd}{\bForm^1}
    \quad
    \hoare{\aForm^2}{\aCmd}{\bForm^2}
  }{
    \hoare{\aForm^1\lor\aForm^2}{\aCmd}{\bForm^1\lor\bForm^2}
  }
\end{displaymath}
We give a motivating example in \textsection\ref{sec:variants}, after
introducing address calculation.

% \paragraph{Par-closure and Case Analysis.}
% Par closure.
% \begin{definition}
%   \label{def-par-closed}
%   $\aPSS$ is \emph{par-closed} if $\aPS^i\!\in\aPSS$ implies $\aPS^1\!{\PAR}\aPS^2\!\in\aPSS$\!\!.
% \end{definition}

% \begin{definition}
%   Let $\aPS'\in\sem{\aReg\GETS\aLoc^\amode\SEMI \aCmd}$ if
%   \begin{math}
%     \aPS'= %(\iDRmode{\aLoc}{}) \prefix
%     \sem{\aCmd}[\aLoc/\aReg]
%   \end{math}
%   or
%   \begin{math}
%     \aPS'=\PAR_{k\in K}(\aForm_k \mid \DRmode\aLoc{\aVal_k}) \prefix \aPS_k
%   \end{math}
%   where $\aPS\in\sem{\aCmd}[\aLoc/\aReg]$ and $\aForm_k$ are
%   disjoint\footnote{For $i$, $j$ in $K$: if $\aForm_i \land \aForm_j$ is
%     satisfiable then $i=j$.}.
% \begin{align*}
%     \sem{\aReg\GETS\aLoc^\amode\SEMI \aCmd} & =
%     \textstyle\bigcup_K\; \PAR_{k\in K}(\aForm_k \mid \DRmode\aLoc{\aVal_k}) \prefix \aPS_k
%     \\[-.5ex] & \mkern2mu\cup
%     %(\iDRmode{\aLoc}{}) \prefix
%     \text{if $\amode\mathbin\neq\modeRLX$ then }\text{$\emptyset$ else }
%     \sem{\aCmd}[\aLoc/\aReg]
%     %\text{ else $\emptyset$}
%     \\
%     \sem{\aLoc^\amode\GETS\aExp\SEMI \aCmd} & \eqdef
%     \PAR_\aVal\; (\aExp=\aVal \mid \DWmode\aLoc\aVal) \prefix \sem{\aCmd}[\aExp/\aLoc]
%     \\[-.5ex] & \mkern2mu\cup
%     %(\iDWmode{\aLoc}{}) \prefix
%     \text{if $\amode\mathbin\neq\modeRLX$ then }\text{$\emptyset$ else }
%     (\relfilt{\aLoc} \sem{\aCmd}[\aExp/\aLoc])
%     %\text{ else $\emptyset$}
% \end{align*}
% \end{definition}

% Local Variables:
% mode: latex
% TeX-master: "paper"
% End:
