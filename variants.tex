\section{Extensions}
\label{sec:variants}

We extend the model to include additional features: fences, read-modify-write
actions (\RMWs), and address calculation. The proofs given later in the paper
extend to include these features.

\paragraph{Fences}

Syntactic fences
``$\FENCE^{\fmode} \SEMI \aCmd$'' have corresponding  actions: $(\DFS{\fmode})$.  The \emph{syntactic fence mode}
$(\fmode \!\!\BNFDEF\!\! \modeREL \!\BNFSEP\! \modeACQ \!\BNFSEP\! \modeSC)$
is either \emph{release}, \emph{acquire}, or \emph{sequentially-consistent}.
$(\DFS{\modeREL})$ is a release. $(\DFS{\modeACQ})$ is an acquire.
$(\DFS{\modeSC})$ is both a release and an acquire.
\begin{align*}
  \sem{\FENCE^{\fmode}\SEMI \aCmd} & \eqdef
  (\DFS{\fmode}) \prefix \sem{\aCmd}
\end{align*}
Syntactic fences require additional order to simulate $\modeRA$/$\modeSC$-accesses.
% Consider a variant of \ref{Pub1}:
% \begin{gather*}
%   \label{Pub2}\tag{Pub2}
%   x\GETS0\SEMI %y\GETS0\SEMI
%   x\GETS 1\SEMI \FENCE^\modeREL\SEMI y \GETS1
%   \PAR
%   r\GETS y \SEMI \FENCE^\modeACQ\SEMI s\GETS x
%   \\[-1ex]
%   \hbox{\begin{tikzinline}[node distance=1em]
%       \event{wx0}{\DW{x}{0}}{}
%       \event{wx1}{\DW{x}{1}}{right=of wx0}
%       \event{fr}{\DFS{\modeREL}}{right=of wx1}
%       \event{wy1}{\DW{y}{1}}{right=of fr}
%       \event{ry1}{\DR{y}{1}}{right=2.5em of wy1}
%       \event{fa}{\DFS{\modeACQ}}{right=of ry1}
%       \event{rx0}{\DR{x}{0}}{right=of fa}
%       \sync{wx1}{fr}
%       %\sync{fr}{wy1}
%       %\sync{ry1}{fa}
%       \sync{fa}{rx0}
%       \rf{wy1}{ry1}
%       \rf[out=10,in=170]{wx0}{rx0}
%       \wk{wx0}{wx1}
%     \end{tikzinline}}
% \end{gather*}
% The semantics of syntactic fences
% is subtly different from the fencing behavior of
% $\modeRA$/$\modeSC$-accesses in C11. 
%To ensure the order that invalidates this execution candidate,
We add the following rules 
to Definition \ref{def:prefix} of \emph{prefixing}:
\begin{enumerate}
\item[{\labeltextsc[P5e]{(P5e)}{5e}}] if $\bEv$ reads, and $\aEv$ is an acquiring fence, then
  $\bEv \lt' \aEv$, and
\item[{\labeltextsc[P5f]{(P5f)}{5f}}] if $\bEv$ is a releasing fence, and $\aEv$ writes, then
  $\bEv \lt' \aEv$.
\end{enumerate}
Consider the following variant of \ref{Pub1}:
\begin{gather*}
  \taglabel{Pub2}
  x\GETS0\SEMI %y\GETS0\SEMI
  x\GETS 1\SEMI \FENCE^\modeREL\SEMI y \GETS1
  \PAR
  r\GETS y \SEMI \FENCE^\modeACQ\SEMI s\GETS x
  \\[-1ex]
  \hbox{\begin{tikzinline}[node distance=1em]
      \event{wx0}{\DW{x}{0}}{}
      \event{wx1}{\DW{x}{1}}{right=of wx0}
      \event{fr}{\DFS{\modeREL}}{right=of wx1}
      \event{wy1}{\DW{y}{1}}{right=of fr}
      \event{ry1}{\DR{y}{1}}{right=2.5em of wy1}
      \event{fa}{\DFS{\modeACQ}}{right=of ry1}
      \event{rx0}{\DR{x}{0}}{right=of fa}
      \sync{wx1}{fr}
      \sync{fr}{wy1}
      \sync{ry1}{fa}
      \sync{fa}{rx0}
      \rf{wy1}{ry1}
      \rf[out=10,in=170]{wx0}{rx0}
      \wk{wx0}{wx1}
    \end{tikzinline}}
\end{gather*}
\ref{5f} requires that $(\DFS{\modeREL})\le(\DW{y}{1})$.  \ref{5e} requires
that $(\DR{y}{1})\le(\DFS{\modeACQ})$.  The attempted execution is
\emph{invalid}: the stale read $(\DR{x}{0})$ violates the last requirement of
fulfillment (Definition \ref{def:rf}).

% Fences can simulate release/acquire access:
% \begin{gather*}  
%   x\GETS1\SEMI
%   \FENCE^{\modeREL}\SEMI
%   y\GETS1
%   \PAR
%   r\GETS y\SEMI
%   \FENCE^{\modeACQ}\SEMI
%   s\GETS x
%   \\[-1ex]
%   \hbox{\begin{tikzinline}[node distance=1em]
%       \event{a1}{\DW{x}{1}}{}
%       \event{a2}{\DFS{\modeREL}}{right=of a1}
%       \po{a1}{a2}
%       \event{a3}{\DW{y}{1}}{right=of a2}
%       \po{a2}{a3}
%       \event{b1}{\DR{y}{1}}{right=3em of a3}
%       \event{b2}{\DFS{\modeACQ}}{right=of b1}
%       \po{b1}{b2}
%       \event{b3}{\DR{x}{0}}{right=of b2}
%       \po{b2}{b3}
%       \rf{a3}{b1}
%       \wkx[out=-170,in=-10]{b3}{a1}
%     \end{tikzinline}}
% \end{gather*}
As for \ref{RL}, redundant fence elimination \eqref{RF} follows from \ref{1}.
\begin{align*}
  \taglabel{RF}
  \sem{\FENCE^\amode \SEMI \FENCE^\amode\SEMI\aCmd} &\supseteq 
  \sem{\FENCE^\amode \SEMI \aCmd}
\end{align*}
\ref{RL} holds regardless of the access mode.

Our semantics does not suffer the weaknesses of C11 fences, noted by
\citet[Figs.~5 and 6]{DBLP:conf/pldi/LahavVKHD17}. We omit $0$\hyp{}initialization
in these examples:
\begin{gather*}
  \taglabel{SC3}
    x\GETS1
    \PAR
    r\GETS x\SEMI   
    \FENCE^{\modeSC}\SEMI
    r\GETS y  
    \PAR
    y\GETS 1 \SEMI
    \FENCE^{\modeSC}\SEMI
    r\GETS x  
    \\[-.1ex]
  \hbox{\begin{tikzinline}[node distance=1em]
  \event{a1}{\DW{x}{1}}{}
  \event{b1}{\DR{x}{1}}{right=3em of a1}
  \event{b2}{\DFS{\modeSC}}{right=of b1}
  \sync{b1}{b2}
  \event{b3}{\DR{y}{0}}{right=of b2}
  \sync{b2}{b3}
  \event{c1}{\DW{y}{1}}{right=3em of b3}
  \event{c2}{\DFS{\modeSC}}{right=of c1}
  \sync{c1}{c2}
  \event{c3}{\DR{x}{0}}{right=of c2}
  \sync{c2}{c3}
  \wk{b3}{c1}
  \rf{a1}{b1}
  \wk[out=-170,in=-10]{c3}{a1}
    \end{tikzinline}}
  \\
  \taglabel{SC4}
    x\GETS1\SEMI   
    z\REL\GETS1\SEMI   
    \PAR
    r\ACQ\GETS z\SEMI   
    \FENCE^{\modeSC}\SEMI
    r\GETS y  
    \PAR
    y\GETS 1 \SEMI
    \FENCE^{\modeSC}\SEMI
    r\GETS x  
    \\[-.1ex]
  \hbox{\begin{tikzinline}[node distance=.7em]
  \event{a1}{\DW{x}{1}}{}
  \event{a2}{\DWRel{z}{1}}{right=of a1}
  \sync{a1}{a2}
  \event{b1}{\DRAcq{z}{1}}{right=2em of a2}
  \event{b2}{\DFS{\modeSC}}{right=of b1}
  \sync{b1}{b2}
  \event{b3}{\DR{y}{0}}{right=of b2}
  \sync{b2}{b3}
  \event{c1}{\DW{y}{1}}{right=2em of b3}
  \event{c2}{\DFS{\modeSC}}{right=of c1}
  \sync{c1}{c2}
  \event{c3}{\DR{x}{0}}{right=of c2}
  \sync{c2}{c3}
  \wk{b3}{c1}
  \rf{a2}{b1}
  \wk[out=-170,in=-10]{c3}{a1}
    \end{tikzinline}}
\end{gather*}
The executions are disallowed, due to cycles.  While these results are
immediate in our model, it is worth noting that they are anything but
immediate in the various models of C11.  \citet{DBLP:conf/pldi/LahavVKHD17}
devote an entire paper to debugging the model of SC access in C11.
% It is a testament to the
% complexity of C11 that it allows these executions.

% \paragraph{Silent Actions and Fences.}

% For the actions of a data model, we require that there is a set
% $\Int\subseteq\Act$ such that
% $\Int\cap\fdom(\rreads)=\Int\cap\fdom(\rwrites)=\emptyset$.  We say $\aAct$
% is \emph{silent} if $\aAct\in\Int$.

% By definition, silent actions neither read nor write.

% To model the fencing behavior of local reads and writes, our example language
% includes silent actions for \emph{read fences} $(\DFR{\amode})$ and
% \emph{write fences} $(\DFW[\aLoc]{\amode})$.  For modes $\modeRA$ and
% $\modeSC$, read fences are acquires and write fences are releases.  As with
% all actions, fences are ordered with respect to preceding acquired and
% following releases.  For mode $\modeRLX$, the fences do nothing.  We
% emphasize that \emph{write fences} are silent, and thus do not interfere with
% \emph{write actions} when fulfilling a read.

% For use in \textsection\ref{sec:opt}, we annotate write fences with the
% location of the local write.  We extend the notion of \emph{conflict}
% (Definition \ref{def:prefix}) to include write fence actions, thus ordering
% $(\DFW[\aLoc]{\amode})$ with respect to reads and writes on
% $\aLoc$\nofootnote{Using this version of write fences, one could rephrase the
%   write rules to move cover filtering from silent local writes to explicit
%   ``visible'' writes in Definition \ref{def:cover}---and the matching items
%   in \textsection\ref{sec:variants} and Definition \ref{def:semi:seq}
%   (6b). In this case, we would choose $(\relfilt{\aLoc} \aPSS)$ be the set
%   $\aPSS'\subseteq\aPSS$ such that $\aPS'\in\aPSS'$ when for every
%   $\cEv',\,\aEv'\in\Event'$, if $\labelingAct(\cEv')=(\DFW[\aLoc]{\amode})$,
%   $\aEv'$ is a release, and $\cEv'\le\aEv'$, then there is some
%   $\bEv'\in\Event'$ such that $\bEv$ (visibly) writes $\aLoc$ and
%   $\cEv'\lt\bEv'\le\aEv'$.  This has the same effect, but results in a
%   slightly different semantics. The choice between the two is arbitrary.}.
% The additional order is harmless, since write fences do not interfere with
% fulfillment.
% % Finally, we also modify write fence actions so that they
% % name the variable that was written: $\DFW[\aLoc]{\amode}$.  These new actions
% % remain silent.  

% We also include syntactic fences of the form
% ``$\FENCE^{\fmode} \SEMI \aCmd$.''  The semantics of syntactic fences
% is subtly different from the fencing behavior of
% $\modeRA$/$\modeSC$-accesses in C11.  Thus we include separate actions for syntactic
% fences: $(\DFS{\fmode})$.  The \emph{syntactic fence mode}
% $(\fmode \!\!\BNFDEF\!\! \modeREL \!\BNFSEP\! \modeACQ \!\BNFSEP\! \modeSC)$
% is either \emph{release}, \emph{acquire}, or \emph{sequentially-consistent}.
% $(\DFS{\modeREL})$ is a release. $(\DFS{\modeACQ})$ is an acquire.
% $(\DFS{\modeSC})$ is both a release and an acquire.

% The additional order of syntactic fences is captured by adding the following
% to Definition \ref{def:prefix} of \emph{prefixing}:
% \begin{enumerate}
% \item[5e.] if $\bEv$ reads and $\labelingAct(\aEv)=(\DFS{\modeACQ})$, then
%   $\aEv \lt' \bEv$, and
% \item[5f.] if $\labelingAct(\bEv)=(\DFS{\modeREL})$ and $\aEv$ writes, then
%   $\bEv \lt' \aEv$.
% \end{enumerate}
% The semantic rules are:
% \begin{align*}
%   \sem{\FENCE^{\fmode}\SEMI \aCmd} & =
%   (\DFS{\fmode}) \prefix \sem{\aCmd}
%   \\
%   \sem{\aReg\GETS\aLoc^\amode\SEMI \aCmd} & =
%   \textstyle\bigcup_\aVal\; (\DRmode\aLoc\aVal) \prefix \sem{\aCmd}[\aLoc/\aReg]  
%   \\[-.5ex] &
%   \mkern2mu\cup
%   \,\mathhl{(\DFR{\amode}) \;\prefix}\,\;
%   \sem{\aCmd}[\aLoc/\aReg]
%   \\
%   \sem{\aLoc^\amode\GETS\aExp\SEMI \aCmd} & =
%   \;\;\textstyle\parallel_\aVal (\aExp=\aVal \mid \DWmode\aLoc\aVal) \prefix \sem{\aCmd}[\aExp/\aLoc]
%   \\[-.5ex] &
%   \mkern2mu\cup
%   \,\mathhl{(\DFW[\aLoc]{\amode}) \;\prefix}\,\;
%   (\relfilt{\aLoc} \sem{\aCmd}[\aExp/\aLoc])
% \end{align*}
% Our semantics does not suffer the weaknesses of C11 fences, noted by
% \citet[Figs.~5 and 6]{DBLP:conf/pldi/LahavVKHD17}. We omit $0$\hyp{}initialization
% in these examples:
% \begin{gather*}
%     x\GETS1
%     \PAR
%     r\GETS x\SEMI   
%     \FENCE^{\modeSC}\SEMI
%     r\GETS y  
%     \PAR
%     y\GETS 1 \SEMI
%     \FENCE^{\modeSC}\SEMI
%     r\GETS x  
%     \\[-.1ex]
%   \hbox{\begin{tikzinline}[node distance=1em]
%   \event{a1}{\DW{x}{1}}{}
%   \event{b1}{\DR{x}{1}}{right=3em of a1}
%   \event{b2}{\DFS{\modeSC}}{right=of b1}
%   \po{b1}{b2}
%   \event{b3}{\DR{y}{0}}{right=of b2}
%   \po{b2}{b3}
%   \event{c1}{\DW{y}{1}}{right=3em of b3}
%   \event{c2}{\DFS{\modeSC}}{right=of c1}
%   \po{c1}{c2}
%   \event{c3}{\DR{x}{0}}{right=of c2}
%   \po{c2}{c3}
%   \wk{b3}{c1}
%   \rf{a1}{b1}
%   \wk[out=-170,in=-10]{c3}{a1}
%     \end{tikzinline}}
%   \\
%     x\GETS1\SEMI   
%     z\REL\GETS1\SEMI   
%     \PAR
%     r\ACQ\GETS z\SEMI   
%     \FENCE^{\modeSC}\SEMI
%     r\GETS y  
%     \PAR
%     y\GETS 1 \SEMI
%     \FENCE^{\modeSC}\SEMI
%     r\GETS x  
%     \\[-.1ex]
%   \hbox{\begin{tikzinline}[node distance=.7em]
%   \event{a1}{\DW{x}{1}}{}
%   \event{a2}{\DWRel{z}{1}}{right=of a1}
%   \po{a1}{a2}
%   \event{b1}{\DRAcq{z}{1}}{right=2em of a2}
%   \event{b2}{\DFS{\modeSC}}{right=of b1}
%   \po{b1}{b2}
%   \event{b3}{\DR{y}{0}}{right=of b2}
%   \po{b2}{b3}
%   \event{c1}{\DW{y}{1}}{right=2em of b3}
%   \event{c2}{\DFS{\modeSC}}{right=of c1}
%   \po{c1}{c2}
%   \event{c3}{\DR{x}{0}}{right=of c2}
%   \po{c2}{c3}
%   \wk{b3}{c1}
%   \rf{a2}{b1}
%   \wk[out=-170,in=-10]{c3}{a1}
%     \end{tikzinline}}
% \end{gather*}
% The executions are disallowed, due to cycles.  It is a testament to the
% complexity of C11 that it allows these executions.

\paragraph{Read-Modify-Write} We discuss \RMW\ operations that work on a
single location in memory, such as \emph{fetch-and-add} ($\FADD$) and
\emph{compare-and-swap} ($\CAS$).  These operations can be modeled using read/write
actions or using an additional relation between events.  The second approach
is more general and less obvious, therefore we explain it here.

In Definition \ref{def:mmpomset}, we require that a \emph{(memory model) pomset}
be a tuple $(\Event, {\le}, \labeling, \rrmw)$, where ${\rrmw}\subseteq{\le}$
relates the two events of a successful \RMW.  Additionally, we require that:
\begin{itemize}
\item If $\cEv$, $\aEv$ write the same $x$, $\cEv\gtN \aEv$ and $\bEv \xrmw \aEv$ then  $\cEv\gtN \bEv$.
\item If $\cEv$, $\aEv$ write the same $x$, $\bEv\gtN \cEv$ and $\bEv \xrmw \aEv$ then  $\aEv\gtN \cEv$.
\end{itemize}
% In Definition~\ref{def:downset}, we require that downsets are \RMW\
% closed:
% $\Event'\subseteq\{ \bEv \in \Event \mid \exists\aEv\in\Event'.\; \aEv\xrmw\bEv\}$.

% ${\le'}={\le}\restrict{\Event'}$, 
% ${\xrmw'}={\xrmw}\restrict{\Event'}$, and
% ${\labeling'}={\labeling}\restrict{\Event'}$.  

Other than these two changes, nothing else changes.  In particular, \RMW{}s
require no special treatment in Definition~\ref{def:par}: the constituent
events of an \RMW{} may coalesce with other events as a result of parallel
composition.  We elide the obvious and tedious semantic rules that generate $\rrmw$.

% \begin{scope}
% \renewcommand{\aEv}{r}
% In Definition \ref{def:rf}, require that when $\bEv$ \emph{fulfills $\aEv$ on
%   $\aLoc$}:
% \begin{itemize}
% %\item there is no pair $(r,w)\in{\rrmw}$ such that $r\gtN\bEv\gtN w$, and
% \item if
%   $\aEv \gtN \bEv$,
%   $\bEv$ writes to $\aLoc$, and
%   $(\aEv,w)\in{\rrmw}$, 
%   then $w \gtN \bEv$.
% \end{itemize}
% \end{scope}
This definition ensures atomicity, disallowing executions such as
\cite[Ex.~3.2]{DBLP:journals/pacmpl/PodkopaevLV19}:
\begin{gather*}
  \taglabel{RMW1}
  \aLoc\GETS0\SEMI\bReg\GETS \FADD^{\modeRLX,\modeRLX}(\aLoc)
  \PAR
  x\GETS 2\SEMI s\GETS x
  \\
  \hbox{\begin{tikzinline}[node distance=1em]
  \event{a2}{\DR{x}{0}}{}
  \event{a1}{\DW{x}{0}}{left=of a2}
  \rf{a1}{a2}
  \event{a3}{\DW{x}{2}}{right=of a2}
  \wk{a2}{a3}
  \event{b2}{\DW{x}{1}}{right=of a3}
  \event{b3}{\DR{x}{1}}{right=of b2}
  \rmw[out=-15,in=-165]{a2}[below]{b2}
  \wk{a3}{b2}
  \rf{b2}{b3}
    \end{tikzinline}}
\end{gather*}

By using two actions rather than one, the definition allows examples such as the
following, which is allowed by \armeight{} 
\cite[Ex.~3.10]{DBLP:journals/pacmpl/PodkopaevLV19}:
\begin{gather*}
  \taglabel{RMW2}
  r \GETS y\SEMI
  z \GETS r
  \PAR
  r\GETS z\SEMI
  x\GETS 0\SEMI
  s\GETS \FADD^{\modeRLX,\modeRA}(x) \SEMI
  y\GETS s{+}1
  \\[-1ex]
  \hbox{\begin{tikzinline}[node distance=1em]
  \event{a1}{\DR{y}{1}}{}
  \event{a2}{\DW{z}{1}}{right=of a1}
  \po{a1}{a2}
  \event{b1}{\DR{z}{1}}{right=3em of a2}
  \rf{a2}{b1}
  \event{b2}{\DW{x}{0}}{right=of b1}
  \event{b3}{\DR{x}{0}}{right=of b2}
  \rf{b2}{b3}
  \event{b4}{\DWRel{x}{1}}{right=2em of b3}
  \rmw{b3}{b4}
  \event{b5}{\DW{y}{1}}{right=of b4}
  \sync[out=-15,in=-165]{b1}{b4}
  \po[out=-20,in=-160]{b3}{b5}
  \rf[out=170,in=10]{b5}{a1}
    \end{tikzinline}}
\end{gather*}

\paragraph{Address Calculation}

In the definition of a data model, we require
that locations have the form $\aLoc\!\!\BNFDEF\!\!\REF{\cVal}$, where $\cVal$
is a value.  Expressions may include neither memory locations nor the
operator $\REF{\cExp}^{\amode}$.
In our example language, we update the syntax of commands:
\begin{gather*}
  \aCmd
  \BNFDEF \cdots
  \BNFSEP \aReg\GETS\REF{\cExp}^{\amode}\SEMI \aCmd 
  \BNFSEP \REF{\cExp}^{\amode}\GETS\aExp\SEMI \aCmd
\end{gather*}
Address calculation can be encoded using the conditional.  We give the
semantics simply by expanding this encoding.  Applying this technique to Candidate
\ref{cand:ord}, we arrive at the following:
% \begin{align*}
%   \sem{\aReg\GETS\REF{\cExp}^\amode\SEMI \aCmd} & =
%   \textstyle\parallel_\cVal\bigcup_\aVal\; (\;\mathhl{\cExp=\cVal} \mid \DRmode{\REF{\cVal}}\aVal) \prefix \;\mathhl{\cExp=\cVal} \guard \sem{\aCmd}[\REF{\cVal}/\aReg]
%   % \\[-.5ex] & \mkern2mu\cup
%   % \textstyle\bigcup_{\cVal\phantom{,\aVal}}\; (\;\mathhl{\cExp=\cVal}\mid\DFR{\amode}) \prefix
%   % \sem{\aCmd}[\REF{\cVal}/\aReg]
%   \\
%   \sem{\REF{\cExp}^\amode\GETS\aExp\SEMI \aCmd} & =
%   \\[-1ex]
%   &\mkern-68mu\hbox to 0pt{$\begin{aligned}
%     &\textstyle\parallel_\cVal\bigcup_\aVal\;(\;\mathhl{\cExp=\cVal} \land \aExp=\aVal \mid \DWmode{\REF{\cVal}}\aVal) \prefix (\;\mathhl{\cExp=\cVal} \guard \sem{\aCmd}[\aExp/\REF{\cVal}])
%     \\[-.5ex]  \cup\!&
%     \textstyle\parallel_{\cVal}\phantom{\bigcup_\aVal}\; %(\;\mathhl{\cExp=\cVal}\mid\DFW[\REF{\cVal}]{\amode}) \prefix
%     (\relfilt{\REF{\cVal}} (\;\mathhl{\cExp=\cVal} \guard \sem{\aCmd}[\aExp/\REF{\cVal}]))
%   \end{aligned}$}
%   % \\[-9.5ex]
%   % \sem{\;\mathhl{\REF{\cExp}^\amode}\GETS\aExp\SEMI \aCmd} & =
%   % \\[6ex]
% \end{align*}
  \begin{align*}
  \sem{\aReg\GETS\REF{\cExp}^\amode\SEMI \aCmd} & \eqdef
    \;\textstyle\parallel_\cVal (\cExp=\cVal) \guard \bigl(
    \textstyle\bigcup_\aVal\;
    (\DRmode{\REF{\cVal}}\aVal) \prefix \sem{\aCmd} [\REF{\cVal}]/\aReg]
    \bigr)
    \\
  \sem{\REF{\cExp}^\amode\GETS\aExp\SEMI \aCmd} & \eqdef
    \;\textstyle\parallel_\cVal (\cExp=\cVal) \guard 
    \bigl(\textstyle\bigcup_\aVal\; (\aExp=\aVal \mid \DWmode{\REF{\cVal}}\aVal) \prefix \sem{\aCmd}[\aExp/\REF{\cVal}]
  \end{align*}
  The same technique to can be applied to Definition \ref{def:cover}---we
  elide the lengthy but obvious definition. For degenerate programs that include only
  constant references (every expression $\REF{\cExp}^\amode$ satisfies
  $\cExp=\cVal$, for some $\cVal$), the resulting definition produces exactly
  the same executions as before.
%   \begin{align*}
%     \sem{\aReg\GETS\aLoc^\modeRLXquiet\SEMI \aCmd} &\eqdef
%     \;\textstyle\parallel_\cVal (\cExp=\cVal) \guard \bigl(
%     \sem{\aCmd}[\REF{\cVal}]/\aReg]
%     \cup
%     \textstyle\bigcup_\aVal\;
%     (\DR[\modeRLXquiet]\REF{\cVal}]\aVal) \prefix \Rdis{\REF{\cVal}]}{\aVal}\sem{\aCmd} [\REF{\cVal}]/\aReg]
%     \bigr)
%     \\
%     \sem{\REF{\cVal}]^\modeRLXquiet\GETS\aExp\SEMI \aCmd} & \eqdef
%     \;\textstyle\parallel_\cVal (\cExp=\cVal) \guard 
%     \begin{aligned}[t]
%       \bigl(\textstyle\bigcup_\aVal\;& (\aExp=\aVal \mid \DW[\modeRLXquiet]\REF{\cVal}]\aVal) \prefix \Wdis{\REF{\cVal}]}{\aExp}\sem{\aCmd}[\aExp/\REF{\cVal}]]
%       \\[-.5ex]
%       \cup\;\;\;\; &\relfilt[]{\REF{\cVal}]} \sem{\aCmd}[\aExp/\REF{\cVal}]]
%       \bigr)
%     \end{aligned}
% \end{align*}

% \begin{aligned}
%   \sem{\aReg\GETS\REF{\cExp}^\amode\SEMI \aCmd} & =
%   \textstyle\bigcup_{\cVal,\aVal}\; (\cExp=\cVal \mid \DRmode{\REF{\cVal}}\aVal) \prefix \sem{\aCmd}[\REF{\cVal}/\aReg]  
%   \\[-.5ex] & \mkern2mu\cup
%   \textstyle\bigcup_{\cVal\phantom{,\aVal}}\; (\cExp=\cVal\mid\DFR{\amode}) \prefix
%   \sem{\aCmd}[\REF{\cVal}/\aReg]
%   \\
%   \sem{\REF{\cExp}^\amode\GETS\aExp\SEMI \aCmd} & =
%   \\[-1ex]
%   &\mkern-65mu\hbox to 0pt{$\begin{aligned}
%     &\textstyle\parallel_{\cVal,\aVal}\;(\cExp=\cVal \land \aExp=\aVal \mid \DWmode{\REF{\cVal}}\aVal) \prefix \sem{\aCmd}[\aExp/\REF{\cVal}]
%     \\[-.5ex]  \cup\!&
%     \textstyle\parallel_{\cVal\phantom{,\aVal}}\; (\cExp=\cVal\mid\DFW{\amode}) \prefix
%     (\relfilt{\REF{\cVal}} \sem{\aCmd}[\aExp/\REF{\cVal}])    
%   \end{aligned}$}
% \end{aligned}

The rewrites listed in \textsection\ref{sec:props}-\ref{sec:refine} remain valid, with the
following generalization: For address expressions $\REF{\aExp}$ and
$\REF{\bExp}$, replace $\aLoc=\bLoc$ by provable equality of $\aExp$ and
$\bExp$, and $\aLoc\neq\bLoc$ by provable inequality.  % Operations on sets can
% be defined similarly.

In Definition \ref{def:cover}, we were able to ensure disjunction closure
  by performing targeted, local weakening via $\Rdis{\aLoc}{\aVal}$ and
  $\Wdis{\aLoc}{\aExp}$.  This is much more difficult to do with address
  calculation, since a single bit of syntax can refer to multiple locations.
  Consider that 
  \begin{math}
    \sem{[r] \GETS 0\SEMI [0]\GETS \BANG r}
  \end{math}
includes both of the following pomsets (``$\BANG$'' is logical
negation: ``$\BANG \aExp$'' evaluates to $1$ if
$\aExp$ is $0$, and $0$ otherwise):
\begin{align*}
  \hbox{\begin{tikzinline}[node distance=2em]
      \eventl{d}{a}{r\EQ0\mathbin{\mid}\DW{[0]}{0}}{}
      \eventl{e}{b}{r\EQ0\mathbin{\mid}\DW{[0]}{1}}{right=of a}
      \wk[out=-15,in=-165]{a}{b}
    \end{tikzinline}}%\;\;\cdots
  &&
  %\\[-1.5ex]\intertext{and:}\\[-5ex]
  \hbox{\begin{tikzinline}[node distance=2em]
      \eventl{c}{a}{r\EQ1\mathbin{\mid}\DW{[1]}{0}}{}
      \eventl{d}{b}{r\EQ1\mathbin{\mid}\DW{[0]}{0}}{right=of a}
    \end{tikzinline}}%\;\;\cdots
\end{align*}
Thus, the disjunction closure also includes both of the following: % By using \!$\PAR$\!, it also includes:
\begin{align*}
  %\label{alanAddress}
  \hbox{\begin{tikzinline}[node distance=2em]
      \eventl{d}{a}{r\EQ0\lor r\EQ1\mathbin{\mid}\DW{[0]}{0}}{}
      \eventl{e}{b}{r\EQ0\mathbin{\mid}\DW{[0]}{1}}{right=of a}
      \wk[out=-15,in=-165]{a}{b}
      %\event{c}{r\EQ1\mathbin{\mid}\DW{[1]}{0}}{right=of b}
    \end{tikzinline}}%\;\;\cdots
  &&
  \hbox{\begin{tikzinline}[node distance=2em]
      \eventl{c}{a}{r\EQ1\mathbin{\mid}\DW{[1]}{0}}{}
      \eventl{d}{b}{r\EQ0\lor r\EQ1\mathbin{\mid}\DW{[0]}{0}}{right=of a}
      %\co{a}{b}
      %\event{c}{r\EQ1\mathbin{\mid}\DW{[1]}{0}}{right=of b}
    \end{tikzinline}}%\;\;\cdots
\end{align*}
In this example, the $d$ events that coalesce correspond to different statements
in the syntax.


\labeltext{Because}{page:strengthening} we do not enforce order between
reads, there is some danger that address calculations could introduce
anomalous behaviors that arise \emph{out of thin air} (\oota)
\cite{DBLP:conf/esop/BattyMNPS15}.  Consider the following program, where
initially $x=0$, $y=0$, $\REF{0}=0$, $\REF{1}=2$, and $\REF{2}=1$.  It should
only be possible to read $0$, disallowing the attempted execution below:
\begin{gather*}
  \begin{gathered}
  \taglabel{OOTA2}
  \aReg\GETS y\SEMI \bReg\GETS \REF{\aReg}\SEMI x\GETS \bReg
  \PAR
  \aReg\GETS x\SEMI \bReg\GETS \REF{\aReg}\SEMI y\GETS \bReg
  %x\GETS\REF{y} \PAR y\GETS\REF{x}
  \\
  \hbox{\begin{tikzinline}[node distance=1em]
  \event{a1}{\DR{y}{2}}{}
  \event{a2}{\DR{\REF{2}}{1}}{right=of a1}
  \event{a3}{\DW{x}{1}}{right=of a2}
  \po{a2}{a3}
  \po[out=10,in=170]{a1}{a3}
  \event{b1}{\DR{x}{1}}{right=3em of a3}
  \event{b2}{\DR{\REF{1}}{2}}{right=of b1}
  \event{b3}{\DW{y}{2}}{right=of b2}
  \po{b2}{b3}
  \po[out=10,in=170]{b1}{b3}
  \rf[out=-170,in=-10]{b3}{a1}
  \rf{a3}{b1}
    \end{tikzinline}}
\end{gathered}
\end{gather*}
Although no order is enforced between reads, the read-to-write order induced
by the semantics is sufficient to prohibit this \oota{} behavior.  Note the
intermediate state of $\sem{\bReg\GETS \REF{\aReg}\SEMI x\GETS \bReg}$:
\begin{gather*}
  \hbox{\begin{tikzinline}[node distance=1em]
  \event{a2}{\aReg{=}2\mid\DR{\REF{2}}{1}}{}
  \event{a3}{\aReg{=}2\mid\DW{x}{1}}{right=of a2}
  \po{a2}{a3}
    \end{tikzinline}}
\end{gather*}
The precondition on $(\DW{x}{1})$ is required by causal strengthening
(Definition~\ref{def:mmpomset}). % : if $\bEv\le\aEv$ then $\labelingForm(\aEv)$
% implies $\labelingForm(\bEv)$.



