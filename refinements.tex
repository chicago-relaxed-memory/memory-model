%\section{Refinements}
\section{Case Analysis, Access Elimination, and Read Introduction}
\label{sec:refine}

The previous section shows the simplicity and beauty of pomsets with
preconditions as a model weak memory.  In this section we look at some of the
complications and ugliness.

We limit attention to relaxed access---Candidate \ref{cand:ord} provides our
final definition of prefixing for $\modeRA$/$\modeSC$ access.  In particular,
we do not attempt to validate rewrites that eliminate $\modeRA$/$\modeSC$
accesses, beyond those already given.  

For relaxed access, the following definition enriches Candidate
\ref{cand:ord} with additional pomsets.  As we discuss below, this definition
validates read elimination \eqref{RE}, store forwarding \eqref{SF}, dead
store elimination \eqref{DS}, and case analysis \eqref{CA}.  We end this
section with a brief discussion of a read-enriched semantics that validates
irrelevant read introduction \eqref{RI}.

In this section, we elide
$\modeRLX$-mode annotations in definitions.
\begin{definition}
  \label{def:cover}
  Let $(\aForm \mid \aAct) \prefix \aPSS$ be the set $\PRE{\aPSS'}$ %$\aPSS'$
  where $\aPS'\in\aPSS'$ when there is some $\aPS\in\aPSS$ that satisfies
  \ref{1}-\ref{5} of Candidate \ref{def:prefix} such that:
  \begin{itemize}
  \item[{\labeltextsc[P6]{(P6)}{6}}]
    if $\bEv$ is a release, $\aEv_1$ is an acquire, $\aEv_1\le\aEv_2$, then $\labelingForm(\aEv_2)$
    is location independent.
  \end{itemize}  

  Let $(\Rdis{\aLoc}{\aVal}\aPSS)$ be the set $\aPSS'$ where $\aPS'\in\aPSS'$
  when there is $\aPS\in\aPSS$ such that: $\Event' = \Event$,
  ${\le'} = {\le}$, $\labelingAct' = \labelingAct$, and either
  $\labelingForm'(\aEv)$ implies $\labelingForm(\aEv)$ or $\aEv$ is
  $\le$-minimal\footnote{$\aEv$ is \emph{$\le$-minimal} if there is no $\bEv$
    such that $\bEv\le\aEv$} and
  $\labelingAct(\aEv)=\DR[\modeRLXquiet]{\aLoc}{\aVal}$.
  %
  % % $\labelingForm'(\aEv)$ implies $\TRUE$ if  and
  % % there is no  $\bEv$  such that $\bEv<\aEv$, and $\labelingForm'(\aEv)$ implies $\labelingForm(\aEv)$ otherwise.

  Let $(\Rdis{\aLoc}{}\aPSS) = \textstyle\bigcup_\aVal \Rdis{\aLoc}{\aVal}\aPSS$.
  % Let $(\Rdis{\aLoc}{}\aPSS)$ be the set $\aPSS'$ where $\aPS'\in\aPSS'$
  % when there is $\aPS\in\aPSS$ such that: $\Event' = \Event$, ${\le'} = {\le}$,
  % $\labelingAct' = \labelingAct$,
  % and either
  % $\labelingForm'(\aEv)$ implies $\labelingForm(\aEv)$ or $\aEv$ is
  % $\le$-minimal and
  % $\labelingAct(\aEv)=\DR[\modeRLXquiet]{\aLoc}{\aVal}$ (for some $\aVal$).
  % %
  % % $\labelingForm'(\aEv)$ implies $\TRUE$ if
  % % $\labelingAct(\aEv)=\DR[\modeRLXquiet]{\aLoc}{\aVal}$ (for some $\aVal$) and
  % % there is no  $\bEv$  such that $\bEv<\aEv$, and $\labelingForm'(\aEv)$ implies $\labelingForm(\aEv)$ otherwise.

  Let $(\Wdis{\aLoc}{\aExp}\aPSS)$ be the set $\aPSS'$ where $\aPS'\in\aPSS'$
  when there is $\aPS\in\aPSS$ such that: $\Event' = \Event$, ${\le'} = {\le}$,
  $\labelingAct' = \labelingAct$, 
  and either
  $\labelingForm'(\aEv)$ implies $\labelingForm(\aEv)$ or $\aEv$ is
  $\le$-minimal,
  $\labelingAct(\aEv)=\DW[\modeRLXquiet]{\aLoc}{\aVal}$, and
  $\labelingForm'(\aEv)$ implies $(\aExp=\aVal\lor\labelingForm(\aEv))$
  % $\labelingForm'(\aEv)$ implies $\aExp=\aVal\lor\labelingForm(\aEv)$ if $\labelingAct(\aEv)=\DW[\modeRLXquiet]{\aLoc}{\aVal}$ and
  % there is no $\bEv$  such that $\bEv<\aEv$, and $\labelingForm'(\aEv)$ implies $\labelingForm(\aEv)$ otherwise.
  
  Let $(\relfilt[]{\aLoc} \aPSS)$ be the set $\aPSS'\subseteq\aPSS$ such that
  $\aPS'\in\aPSS'$ when for every release $\aEv'\in\Event'$, there is some
  $\bEv'\in\Event'$ such that $\bEv' \le\aEv'$ and $\bEv'$ \externally writes
  $\aLoc$.
  % 
  % Let $(\relfilt[\modeRLXquiet]{\aLoc} \aPSS)$ be the set $\aPSS'\subseteq\aPSS$ such that
  % $\aPS'\in\aPSS'$
  % when %$\Event$ contains a write to $\aLoc$ and
  % %there is some $\cEv'\in\Event'$ that writes $\aLoc$ and
  % for every release $\aEv'\in\Event'$, %that does not write $\aLoc$,
  % there is some $\bEv'\in\Event'$ 
  % such that $\bEv' \le\aEv'$ and $\bEv'$ \externally  writes $\aLoc$.  % For $\amode\neq\modeRLXquiet$, let
  %
  % Let $(\relfilt[\modeRA]{\aLoc} \aPSS)=(\relfilt[\modeSC]{\aLoc} \aPSS)=\emptyset$.
  %
  %Let $(\relfilt[\amode]{\aLoc} \aPSS)=\emptyset$ when $\amode\neq\modeRLXquiet$.
  % $(\relfilt {\aLoc} \aPSS)$ be the empty set.
  \begin{align*}
    \sem{\aReg\GETS\aLoc^\modeRLXquiet\SEMI \aCmd} &\eqdef
    \;\mathhl{\Rdis{\aLoc}{}\sem{\aCmd}[\aLoc/\aReg]}\;
    \cup
    \textstyle\bigcup_\aVal\;
    (\DR[\modeRLXquiet]\aLoc\aVal) \prefix \;\mathhl{\Rdis{\aLoc}{\aVal}}\;\,\sem{\aCmd} [\aLoc/\aReg]
    % \sem{\aReg\GETS\aLoc^\modeRLXquiet\SEMI \aCmd} & =
    % \textstyle\bigcup_\aVal\; (\DR[\modeRLXquiet]\aLoc\aVal) \prefix \sem{\aCmd}[\aLoc/\aReg]  
    % \\[-.5ex] &
    % \;\mathhl{\cup\;
    %   \sem{\aCmd}[\aLoc/\aReg] %  \text{ if } \amode=\modeRLXquiet
    % }
    \\
    \sem{\aLoc^\modeRLXquiet\GETS\aExp\SEMI \aCmd} & \eqdef
    \;\mathhl{\Wdis{\aLoc}{\aExp}\bigl(\relfilt[]{\aLoc} \sem{\aCmd}[\aExp/\aLoc]\bigr)}\;
    \cup
    \textstyle\bigcup_\aVal\; (\aExp=\aVal \mid \DW[\modeRLXquiet]\aLoc\aVal) \prefix \sem{\aCmd}[\aExp/\aLoc]
  \end{align*}
  % \begin{align*}
  %   \sem{\aReg\GETS\aLoc^\modeRLXquiet\SEMI \aCmd} &\eqdef
  %   \;\mathhl{\sem{\aCmd}[\aLoc/\aReg]}\;
  %   \cup
  %   \textstyle\bigcup_\aVal\;
  %   % \sem{\aReg\GETS\aLoc^\modeRLXquiet\SEMI \aCmd} & =
  %   % \textstyle\bigcup_\aVal\; (\DR[\modeRLXquiet]\aLoc\aVal) \prefix \sem{\aCmd}[\aLoc/\aReg]  
  %   % \\[-.5ex] &
  %   % \;\mathhl{\cup\;
  %   %   \sem{\aCmd}[\aLoc/\aReg] %  \text{ if } \amode=\modeRLXquiet
  %   % }
  %   \\
  %   \sem{\aLoc^\modeRLXquiet\GETS\aExp\SEMI \aCmd} & \eqdef
  %   \;\mathhl{\relfilt[]{\aLoc} \sem{\aCmd}[\aExp/\aLoc]}\;
  %   \cup
  %   \textstyle\bigcup_\aVal\; (\aExp=\aVal \mid \DW[\modeRLXquiet]\aLoc\aVal) \prefix \;\mathhl{\Wdis{\aLoc}{\aExp}}\;\,\sem{\aCmd}[\aExp/\aLoc]
  % \end{align*}
\end{definition}
There are six changes in the definition: To validate read elimination, we
include $\sem{\aCmd}[\aLoc/\aReg]$.  To ensure that this does not allow stale
reads, we add requirement \ref{6} to the definition of prefixing.  To validate case
analysis on reads actions, we apply $\Rdis{\aLoc}{\aVal}$ to
$\sem{\aCmd} [\aLoc/\aReg]$ before prefixing.  To validate case analysis with
eliminated reads, we apply $\Rdis{\aLoc}{}$ to $\sem{\aCmd} [\aLoc/\aReg]$.
To validate write elimination, we include
$\relfilt[]{\aLoc} \sem{\aCmd}[\aExp/\aLoc]$. To validate case analysis with
eliminated writes, we apply $\Wdis{\aLoc}{\aVal}$.

\myparagraph{Read Elimination and Store Forwarding}

% Suppose you want to validate the following:
% % \begin{displaymath}
% %   \sem{r\GETS y\SEMI \IF{r}\THEN s\GETS x\FI \SEMI x\GETS 1 \SEMI z\GETS r}
% %   \supseteq
% %   \sem{x\GETS 1 \SEMI r\GETS y\SEMI  z\GETS r}
% % \end{displaymath}
% \begin{displaymath}
%   \sem{\IF{y}\THEN r\GETS x\SEMI x\GETS 1 \ELSE x\GETS 1 \FI}
%   \supseteq
%   \sem{x\GETS 1}
% \end{displaymath}
% Note that there is a spurious dependency from Read y to write x in the case
% that y reads non-zero value.  This is caused by causal strengthening on the
% coherence order.

In our work on microarchitecture \citep{2019-sp}, read actions could be
observed using cache effects.  Candidate \ref{cand:ord} maintains this
perspective---for example, it distinguishes $\sem{r\GETS x}$ and
$\supseteq\sem{\SKIP}$ % since
% $\sem{\SKIP}$ has a pomset with only the terminal action, and there is no
% such pomset in $\sem{r\GETS x}$.
% % $\not\supseteq\sem{r\GETS x\SEMI r\GETS{}x}$.
% This inequation holds 
even though there is no context in the language of this paper that can
distinguish these programs.  If one accepts that these programs should be equated
at an architectural level, then one would expect the semantics to validate
% Although Candidate \ref{cand:ord} validates redundant load elimination
% \eqref{RL}, it does \emph{not} validate either general 
read elimination \eqref{RE} and store forwarding \eqref{SF}.
\begin{align*}
  \taglabel{RE}
  \sem{\aReg  \GETS \aLoc\SEMI\aCmd} & \supseteq
  \sem{\aCmd}  
  &&\textif \aReg\not\in\free(\aCmd)&&\hbox{}
  \\
  \taglabel{SF}
  \sem{\aLoc^\amode \GETS \aExp \SEMI \aReg  \GETS \aLoc\SEMI\aCmd} &\supseteq 
  \sem{\aLoc^\amode \GETS \aExp \SEMI \aReg  \GETS \aExp\SEMI\aCmd}  
\end{align*}
These optimizations are validated by Definition \ref{def:cover}, since
$\sem{\aReg\GETS\aLoc^\modeRLXquiet\SEMI \aCmd}\supseteq
\sem{\aCmd}[\aLoc/\aReg]$.  The proof of \ref{SF} also appeals to the
definition of write and the definition of register assignment.

% \ref{RL} removes one read from a program that has two.  In contrast, \ref{RE}
% and \ref{SF} may remove the only read in an execution, as can be seen by
% taking $\aCmd=\SKIP$ above.

Let us revisit the internal read examples from \textsection\ref{sec:litmus}.
With read elimination, the read action $(\DR{y}{1})$ can be elided in
\ref{Internal2}; regardless, the substitution into the write of $z$ is the
same.  On a more troubling note, the read action $(\DR{x}{1})$ can be also
elided in \ref{Internal1}, potentially converting this non-execution into a
valid execution, violating \drfsc{}.  The addition of  \ref{6} to
the definition of prefixing prevents this outcome.  When computing
$\sem{x\GETS1 \SEMI a\REL\GETS1 \SEMI \IF{b\ACQ}\THEN y\GETS x \FI}$,
 \ref{6} prevents prefixing $(\DWRel{a}{1})$ in front of:
\begin{gather*}
  \hbox{\begin{tikzinline}[node distance=1.5em]
  \event{a6}{\DRAcq{b}{1}}{}
  \event{a7}{\DR{x}{1}}{right=of a6}
  \sync{a6}{a7}
  \event{a8}{x=1\mid\DW{y}{1}}{right=of a7}
  \graypo{a7}{a8}
  \sync[out=10,in=170]{a6}{a8}
    \end{tikzinline}}
\end{gather*}
In order to satisfy  \ref{6}, the precondition of $(\DW{y}{1})$
must be location independent.

% \eqref{SF} follows from item \ref{1} in the definition of
% prefixing,

% Note that this inclusion
% does not hold in the plain semantics: for example,
% $\sem{\aLoc \GETS \aExp \SEMI \aReg \GETS \aExp}$ has a pomset containing
% only a write and the terminal action, whereas
% $\sem{\aLoc \GETS \aExp \SEMI \aReg \GETS \aLoc}$ contains no such pomset.


\myparagraph{Dead Store Elimination}
Dead store elimination \eqref{DS} is symmetric to redundant load elimination.
\begin{align*}
  \taglabel{DS}
  \sem{\aLoc \GETS \aExp \SEMI \aLoc  \GETS \bExp\SEMI\aCmd} &\supseteq 
  \sem{\aLoc \GETS \bExp\SEMI\aCmd}    
\end{align*}
The rewrite is less general than \ref{RE} because general store elimination
is unsound.  For example, ``${x\GETS 0}$'' and ``${x\GETS 0\SEMI x\GETS 1}$''
can be distinguished by the context ``$\hole{}\PAR z\GETS x$''.

\ref{DS} is valid under Definition \ref{def:cover}.  A write may only be
removed if it is \emph{covered} by a following write.  This
prevents bad executions, such as:
\begin{gather*}
  %\taglabel{Cover}
  x\GETS 1\SEMI
  x\GETS 2\SEMI
  y\REL\GETS 1
  \PAR
  \aReg\GETS y\ACQ \SEMI \bReg\GETS x
  \\[-1ex]
  \hbox{\begin{tikzinline}[node distance=1.5em]
  \event{a1}{\DW{x}{1}}{}
  \internal{a2}{\DW{x}{2}}{right=of a1}
  \graywk{a1}{a2}
  \event{a3}{\DWRel{y}{1}}{right=of a2}
  \graypo{a2}{a3}
  \event{b1}{\DRAcq{y}{1}}{right=2em of a3}
  \rf{a3}{b1}
  \event{b2}{\DR{x}{1}}{right=of b1}
  \sync{b1}{b2}
  \rf[out=10,in=170]{a1}{b2}
  \sync[out=-10,in=-170]{a1}{a3}
    \end{tikzinline}}
\end{gather*}
In this diagram, we have included a ``non-event''---dashed border---to mark
the eliminated write.  In general, there may
need to be many following writes, one for each subsequent release.  

% The restriction Although store elimination is unsound generally, one might expect it to hold for dead
% stores \eqref{DS}.

% We discuss compiler optimization in \textsection\ref{sec:opt}.  % Irrelevant
% reads have no effect in our model, thus we can define correctness with
% respect to pomsets that have been saturated with arbitrary irrelevant reads.
% The same does not hold for writes.

% In our final definition of the semantics, we allow for the possibility of
% relaxed read and write elimination:

% In the semantics thus far, we have supposed that every read must be fulfilled
% by a matching write action, and that the order between them must therefore be
% part of the global pomset order.  This is overly restrictive for reads that
% are fulfilled locally.

% Consider Example 3.6 from
% \citet{DBLP:journals/pacmpl/PodkopaevLV19}:
% \begin{gather*}
%   %\tag{$\dagger$}
%   \label{Internal2}
%   \aReg\GETS x\SEMI
%   y\REL\GETS 1\SEMI
%   \bReg\GETS y\SEMI
%   z\GETS \bReg
%   %z\GETS y
%   \PAR
%   x\GETS z
%   \\
%   \nonumber
%   \hbox{\begin{tikzinline}[node distance=.5em and 1em]
%   \event{a1}{\DR{x}{1}}{}
%   \event{a2}{\DWRel{y}{1}}{right=of a1}
%   \po{a1}{a2}
%   \event{a3}{\DR{y}{1}}{right=of a2}
%   \event{a4}{\DW{z}{1}}{right=of a3}
%   \rf{a2}{a3}
%   %\po{a3}{a4}
%   \event{b1}{\DR{z}{1}}{right=2em of a4}
%   \event{b2}{\DW{x}{1}}{right=of b1}
%   %\po[out=10,in=170]{a2}{a4}
%   \po{b1}{b2}
%   \rf{a4}{b1}
%   \rf[out=170,in=10]{b2}{a1}
%     \end{tikzinline}}
% \end{gather*}
% This behavior is allowed by \armeight, but disallowed in the model presented
% thus far, due to the evident cycle.

% To allow this outcome, we remove the requirement for an explicit read action
% when a read is matched by a local write---and, symmetrically, for an explicit
% write action when a write is only used to match local reads.  We treat
% only relaxed local access in this section, extending to $\modeRA$/$\modeSC$
% local access after
% introducing fences in \textsection\ref{sec:variants}.
% % We first give the semantics of read.
% % \begin{definition} \ \vspace{-1ex}
% %   %\label{def:rw:local}
% %   \begin{align*}
% %     \sem{\aReg\GETS\aLoc^\amode\SEMI \aCmd} & =
% %     \textstyle\bigcup_\aVal\; (\DRmode\aLoc\aVal) \prefix \sem{\aCmd}[\aLoc/\aReg]  
% %     % \\[-.5ex] &
% %     % % \mkern2mu\cup
% %     % \;\mathhl{\cup\;
% %     %   % (\iDRmode{\aLoc}{}) \prefix
% %     %   \text{if $\amode\mathbin\neq\modeRLXquiet$ then }\text{$\emptyset$ else }
% %     %   \sem{\aCmd}[\aLoc/\aReg]
% %     %   % \text{ else $\emptyset$}
% %     % }
% %   \end{align*}
% % \end{definition}

% As in \eqref{ex555}, Item \ref{4b} (Definition \ref{def:pre-sc}) ensures that
% intervening writes are respected.  For example, 
% \begin{math}
%   {x\GETS 0\SEMI x\GETS 1\SEMI y\GETS x}
% \end{math}
% cannot write $0$ to $y$.
% %does not $\DW{y}{0}$.
% % NEED TO MAKE THIS POINT:
% % So consider the single threaded program

% % x=4; x=5; y=x;

% % Clearly, this should not be able to give us Wy1.
% % Here is why that is not possible:

% % [[ y=1 ]] contains (Wy1)

% % [[ if(r==4){y=1} ]] contains (r==4|Wy1)

% % [[ r=x; if(r==4){y=1} ]] contains (Rx5)  (x==4|Wy1) // with no order

% % [[ x=5; r=x; if(r==4){y=1} ]] contains (Wx5)-->(Rx5)  (5==4|Wy1) // 5 is forced.


% % Consider the program variant:

% % x=0; r=x; if(r<2){y=1} || x=5

% % [[ y=1 ]] contains (Wy1)

% % [[ if(r<2){y=1} ]] contains (r<2|Wy1)

% % [[ r=x; if(r<2){y=1} ]] contains (Rx5)  (x<2|Wy1)  //with no order

% % [[ x=0; r=x; if(r<2){y=1} ]] contains (Wx0)  (Rx5)  (x<2|Wy1) //does not work, since violates (6a) 𝜆Φ′(𝑒) implies 𝜆Φ(𝑒)[𝑣/𝑥].


% % 6. if 𝑎 externally reads 𝑣 from 𝑥 then both 
% % (6a) 𝜆Φ′(𝑒) implies 𝜆Φ(𝑒)[𝑣/𝑥], and 
% % (6b) if 𝑒 is a write then either 𝑐 <′ 𝑒 or 𝜆Φ′(𝑒) implies 𝜆Φ(𝑒)



\myparagraph{Case Analysis}
Definition \ref{def:cover} satisfies \emph{disjunction closure}.
\begin{definition}
  \label{def:dis}
  We say that $\aPS$ is a \emph{disjunct of $\aPS'$ and downset $\aPS''$} when
  $\Event=\Event' \supseteq\Event''$, % \supseteq \{ \bEv \in \Event' \mid \exists\aEv\in\Event''.\; \bEv\le\aEv\}$,
  ${\le}={\le'}\supseteq{\le''}$,
  $\labelingAct=\labelingAct' \supseteq\labelingAct''$, 
  $\labelingForm(\aEv)$ implies
  $\labelingForm'(\aEv)\lor \labelingForm''(\aEv)$ if $\aEv\in\Event''$, and
  $\labelingForm(\aEv)$ implies
  $\labelingForm'(\aEv)$ otherwise.
  % and:
  % \begin{displaymath}
  %   \labelingForm(\aEv) \text{ implies }
  %   \begin{cases}
  %     \labelingForm'(\aEv)\lor \labelingForm''(\aEv) & \text{if } \aEv\in\Event''
  %     \\
  %     \labelingForm'(\aEv) & \text{otherwise} %\text{if } \aEv\in(\Event'\setminus\Event'')
  %   \end{cases}
  % \end{displaymath}

  We say that $\aPSS$ is \emph{disjunction closed} if
  $\aPS\in\aPSS$ whenever there are $\{\aPS',\,\aPS''\}\subseteq \aPSS$
  such that $\aPS$ is a disjunct of $\aPS'$ and downset $\aPS''$.
\end{definition}
% \begin{align*}
%   \sem{\IF{r{=}1}\THEN x\GETS r\SEMI x\GETS 1 \ELSE x\GETS 1\FI}
%   \supseteq
%   \sem{x\GETS 1 \SEMI \IF{r{=}1}\THEN x\GETS r\FI}
% \end{align*}
% From the semantics of conditional it is obvious that Candidate \ref{cand:ord}
% satisfies the following:
% \begin{align*}
%   \sem{\IF{\aExp}\THEN\aCmd\ELSE\aCmd\FI} 
%   &\supseteq
%   \sem{\aCmd} 
% \end{align*}
% The reverse of code lifting \ref{CL} (\textsection\ref{sec:valid}) is case
% analysis \eqref{CA}:
Disjunction closure is sufficient to establish case analysis
\eqref{CA}:
\begin{align*}
  \taglabel{CA}
  \sem{\aCmd} &\supseteq
  \sem{\IF{\aExp}\THEN\aCmd\ELSE\aCmd\FI} 
\end{align*}
As we discuss below, Candidate \ref{cand:ord} is not disjunction closed;
however, it \emph{becomes} closed if \ref{5b} is strengthened to include
read-read coherence, thus sacrificing sacrifice \ref{CSE}.  This compromise
is considered reasonable for C11 atomics, which are meant to be used sparingly.
It is less attractive for relaxed access in safe languages, like Java.

We start the discussion of case analysis with a version of Candidate
\ref{cand:ord} that strengthens \ref{5b} to include \emph{read-read
  coherence}.  We then look at three relaxations in turn: write elimination,
reads elimination, and finally dropping read-read coherence.  Definition
\ref{def:cover} performs a bit of disjunctive closure in each case.

Write elimination causes disjunction closure to fail.  For example, consider
the following executions of $\cmdW$; the second uses write elimination. (To
make coalescing explicit, we show the element of the carrier set $\Event$ to
left of the label.)
\begin{gather*}
  \cmdW = x\GETS 1 \SEMI \IF{r}\THEN x\GETS 1 \FI
  \\
  \hbox{\begin{tikzinline}[node distance=2em]
      \eventl{d}{a}{\DW{x}{1}}{}
      \eventl{e}{b}{r\mid\DW{x}{1}}{right=of a}
      \wk[out=-15,in=-165]{a}{b}
    \end{tikzinline}}
  \qquad\qquad\qquad\qquad
  \hbox{\begin{tikzinline}[node distance=2em]
      \eventl{d}{b}{r\mid\DW{x}{1}}{}
    \end{tikzinline}}
\end{gather*}
Using disjunction, $\sem{\IF{s}\THEN \cmdW \ELSE \cmdW \FI}$ includes the
singleton $d{:}(\DW{x}{1})$, but $\sem{\cmdW}$ does not: In the definition of
composition, any actions with the same label and downset can coalesce.  Our
solution is to weaken the preconditions on writes to the same location when
eliminating a write.  We do this using the function $\Wdis{\aLoc}{\aExp}$,
which weakens the precondition of any $<$-minimal
$(\DW[\modeRLXquiet]{\aLoc}{\aVal})$ to $(\aExp=\aVal \lor \aForm)$, where
$\aForm$ is formula before prefixing.  When eliminating $(\DW{x}{1})$ in
front of $(r\mid\DW{x}{1})$ in the example above, we arrive at the tautology
$(1=1\lor r)$ for the ``old'' write.

With read-read coherence, reads and read elimination are completely symmetric.

% With write elimination, a symmetric argument can be applied to pomsets of $\cmdW$:
%  However, \ref{CA} fails for the semantics with read elimination, even with
% read-read coherence.  For example, consider the following executions of
% $\cmdR$; the second uses read elimination:
% \begin{align*}
%   \hbox{\begin{tikzinline}[node distance=2em]
%       \eventl{d}{a}{\DR{x}{0}}{}
%       \eventl{e}{b}{r\mid\DR{x}{0}}{right=of a}
%       \wk[out=-15,in=-165]{a}{b}
%     \end{tikzinline}}
%   &&
%   \hbox{\begin{tikzinline}[node distance=2em]
%       \eventl{d}{b}{r\mid\DR{x}{0}}{right=of a}
%     \end{tikzinline}}
% \end{align*}
% Taking the disjunction of their downsets, we expect the semantics of
% $\IF{s}\THEN \cmdR \ELSE \cmdR \FI$ to include $d{:}(\DR{x}{0})$, but it
% does not under Candidate \ref{cand:ord} with only read elimination.
% Using case analysis a second time, we expect $\sem{\cmdR}$ to contain:
% \begin{gather*}
%   \hbox{\begin{tikzinline}[node distance=2em]
%       \eventl{d}{a}{\DR{x}{0}}{}
%       \eventl{e}{b}{\DR{x}{0}}{right=of a}
%     \end{tikzinline}}
% \end{gather*}
% Our solution is to weaken preconditions on identical reads when prefixing.
% We do this using the function $\Rdis{\aLoc}{\aVal}$, which weakens the
% precondition of any $<$-minimal $\DR[\modeRLXquiet]{\aLoc}{\aVal}$ to
% $\TRUE$.
%

Relaxing read-read coherence causes disjunction closure to fail.  For
example, 
% \begin{align*}
%   \begin{gathered}
%     \cmdR = \bExp\GETS x \SEMI \IF{\aExp}\THEN \bExp\GETS x \FI\\
%     \hbox{\begin{tikzinline}[node distance=1.5em]
%         \event{a}{\DR{x}{0}}{}
%         \event{b}{\aExp\mid\DR{x}{0}}{right=of a}
%       \end{tikzinline}}
%   \end{gathered}
%   &&
%   \begin{gathered}
%     \cmdW = x\GETS 1 \SEMI \IF{\aExp}\THEN x\GETS 1 \FI\\
%     \hbox{\begin{tikzinline}[node distance=1.5em]
%         \event{a}{\DW{x}{1}}{}
%         \event{b}{\aExp\mid\DW{x}{1}}{right=of a}
%         \wk{a}{b}
%       \end{tikzinline}}
%   \end{gathered}
% \end{align*}
% $\cmdR$ reads twice, and $\cmdW$ writes twice.  Whereas the writes are
% ordered by \ref{5b}, the reads are not.  In the definition of composition,
% any actions with the same label and downset can coalesce, causing \ref{CA} to
% fail for $\cmdR$.  
consider the two sides of the composition defined by the conditional, where
$\cmdR = \aReg\GETS x \SEMI \IF{\aExp}\THEN \bReg\GETS x \FI$.
\begin{align*}
  \begin{gathered}
    \IF{\bExp}\THEN \cmdR \FI
    \\
    \hbox{\begin{tikzinline}[node distance=2em]
        \eventl{d}{a}{\bExp \mid\DR{x}{0}}{}
        \eventl{e}{b}{\bExp \land \aExp\mid\DR{x}{0}}{right=of a}
      \end{tikzinline}}
  \end{gathered}
  &&
  \begin{gathered}
    \IF{\lnot \bExp}\THEN \cmdR \FI
    \\
    \hbox{\begin{tikzinline}[node distance=2em]
        \eventl{e}{a}{\lnot \bExp \mid\DR{x}{0}}{}
        \eventl{d}{b}{\lnot \bExp \land \aExp\mid\DR{x}{0}}{right=of a}
      \end{tikzinline}}
  \end{gathered}
\end{align*}
Because the reads are unordered, they can be confused when coalescing, resulting in:
\begin{align*}
  \begin{gathered}
    \IF{\bExp}\THEN \cmdR \ELSE \cmdR \FI
    \\
    \hbox{\begin{tikzinline}[node distance=2em]
        \eventl{d}{a}{\bExp\lor (\lnot \bExp \land \aExp) \mid\DR{x}{0}}{}
        \eventl{e}{b}{(\bExp \land \aExp)\lor \lnot \bExp \mid\DR{x}{0}}{right=of a}
      \end{tikzinline}}    
  \end{gathered}
\end{align*}
which is:
\begin{align*}
  \begin{gathered}
    \hbox{\begin{tikzinline}[node distance=2em]
        \eventl{d}{a}{\bExp\lor \aExp\mid\DR{x}{0}}{}
        \eventl{e}{b}{\lnot \bExp\lor \aExp\mid\DR{x}{0}}{right=of a}
      \end{tikzinline}}
  \end{gathered}
\end{align*}
But this pomset does not occur in $\sem{\cmdR}$.  
Our solution is to weaken the preconditions on reads so that both
f$\sem{\cmdR}$ and $\sem{\IF{\bExp}\THEN \cmdR \ELSE \cmdR \FI}$ include: 
\begin{align*}
  \begin{gathered}
    \hbox{\begin{tikzinline}[node distance=2em]
        \eventl{d}{a}{\DR{x}{0}}{}
        \eventl{e}{b}{\DR{x}{0}}{right=of a}
      \end{tikzinline}}
  \end{gathered}
\end{align*}
Note that the precondition on the reads are weaker than one would expect.
This is not a problem for reads, since they must also be fulfilled---allowing
more reads \emph{increases} the obligations of fulfillment.  The same
solution would not work for writes---as we discussed at the end of
\textsection\ref{sec:props}, allowing more writes is simply unsound.
Fortunately, this problem does not occur when prefixing a write in front of
another write, due to the order required by \ref{5b}.

% The key property required to prove \ref{CA} is \emph{disjunction closure}.
% The example shows that disjunction closure fails for Candidate
% \ref{cand:ord}, due to the lack of read-read coherence.  Note that this is
% not a problem for pairs of $\modeRA$ or $\modeSC$ reads, since these are
% ordered by \ref{5c}.

% To validate \ref{CA} for this command, the semantics must also include the
% singleton $(\DW{x}{1})$.  The singleton is included in
% $\sem{\IF x\GETS 1\SEMI x\GETS r\ELSE x\GETS 1\SEMI x\GETS r \FI}$ by
% combining the right pomset above with the singleton downset $(\DW{x}{1})$
% from the left.



% \begin{gather*}
%   \labelingForm'(\aEv) \text{ implies }
%   \begin{cases}
%     \TRUE & \text{if } \labelingAct(\aEv)=\DR[\modeRLXquiet]{\aLoc}{\aVal} \textand
%     \text{there is no } \bEv \text{ such that } 
%     % \not\exists\bEv\in\Event.\;
%     \bEv<\aEv
%     \\
%     \labelingForm(\aEv) & \text{otherwise} 
%   \end{cases}
% \end{gather*}
% The definition of read prefixing for relaxed access is then:
% \begin{align*}
%   \sem{\aReg\GETS\aLoc^\modeRLXquiet\SEMI \aCmd} & = \textstyle\bigcup_\aVal\;
%   (\DR[\modeRLXquiet]\aLoc\aVal) \prefix \;\mathhl{\Rdis{\aLoc}{\aVal}}\;\,\sem{\aCmd} [\aLoc/\aReg]
% \end{align*}
% We give the final definition of prefixing after considering access elimination.
% Case analysis \eqref{CA} follows from disjunction closure
% (Definition \ref{def:dis}).


% The semantics is also closed with respect to \emph{disjunction}, which
% weakens pomsets by taking the disjunction of the formulae in their common
% downset.  For example, since $\sem{x \GETS 1{+}r{*}r{-}r}$ includes
% $(r\EQ0\mid\DW{x}{1})$ and $(r\EQ1\mid\DW{x}{1})$, it must also include
% $(r\EQ0\lor r\EQ1\mid\DW{x}{1})$ (see page \pageref{page:disjunction}).
% Likewise, $\sem{[r] \GETS 0\SEMI [0]\GETS \BANG r}$ must include
% \begin{math}
%   (r\EQ0\lor r\EQ1\mid\DW{[0]}{0}) 
% \end{math}
% despite the fact that
% \begin{math}
%   (r\EQ0\mid\DW{[0]}{0}) 
% \end{math}
% is generated by $[r] \GETS 0 $ and
% \begin{math}
%   (r\EQ1\mid\DW{[0]}{0}) 
% \end{math}
% is generated by $[0]\GETS \BANG r$
% % \begin{math}
% %   (r\EQ0\lor r\EQ1\mid\DW{[0]}{0}) (r\EQ0\mathbin\mid\DW{[0]}{1})
% % \end{math}
% % and
% % \begin{math}
% %   (r\EQ0\lor r\EQ1\mid\DW{[0]}{0}) (r\EQ1\mathbin\mid\DW{[1]}{0})
% % \end{math}
% (see page \pageref{page:disjunction2}).

% \labeltext{The}{page:disjunction} semantics is again driven by Hoare
% logic---for the preconditions of writes, the relevant rule is left
% disjunction:
% \begin{displaymath}
%   \frac{
%     \hoare{\aForm^1}{\aCmd}{\bForm}
%     \quad
%     \hoare{\aForm^2}{\aCmd}{\bForm}
%   }{
%     \hoare{\aForm^1\lor\aForm^2}{\aCmd}{\bForm}
%   }
% \end{displaymath}
% \todo{This is stupid:}
% Note that
% \begin{math}
%   \sem{x \GETS 1{+}r{*}r{-}r}
% \end{math}
% includes both % of the pomsets:
% % \begin{align*}
% %   \hbox{\begin{tikzinline}[node distance=.2em]
% %       \event{a}{r\EQ0\mid\DW{x}{1}}{}
% %     \end{tikzinline}}%\;\;\cdots
% %   \qquad
% %   %\\[-1.5ex]\intertext{and:}\\[-5ex]
% %   \hbox{\begin{tikzinline}[node distance=.2em]
% %       \event{a}{r\EQ1\mid\DW{x}{1}}{}
% %     \end{tikzinline}}%\;\;\cdots
% % \end{align*}
% \begin{math}
%   \hbox{\begin{tikzinline}[node distance=.2em]
%       \event{a}{r\EQ0\mid\DW{x}{1}}{}
%     \end{tikzinline}}%\;\;\cdots
% \end{math}
% and
% % \begin{math}
% %   \\[-1.5ex]\intertext{and:}\\[-5ex]
% \hbox{\begin{tikzinline}[node distance=.2em]
%     \event{a}{r\EQ1\mid\DW{x}{1}}{}
%   \end{tikzinline}}. %\;\;\cdots
% % \end{math}
% % \end{align*}
% By using $\parallel_\aVal$ in the definition of write, it also includes:
% % (where,
% % contrary to convention, we show disjoint events):
% \begin{math}
%   \hbox{\begin{tikzinline}[node distance=.5em]
%       \event{a}{r\EQ0\lor r\EQ1\mid\DW{x}{1}}{}
%     \end{tikzinline}}%\;\;\cdots
% \end{math}.
% An alternate definition using $\cup_\aVal$ would exclude this pomset.

% Given that the conditional is defined using $\parallel$, the use of $\parallel$
% in the definition of write is necessary to validate \emph{case analysis}:
% \begin{math}
%   \sem{\aCmd}
%   =
%   \sem{\IF{\aExp}\THEN \aCmd\ELSE \aCmd\FI}.
% \end{math}






  % We say that $\aPS^0$ is a \emph{disjunct} of $\aPS^1$ and $\aPS^2$ when
  % $\Event^0=\Event^1 =\Event^2$, ${\le^0}={\le^1}={\le^2}$,
  % $\labelingAct^0=\labelingAct^1 =\labelingAct^2$, and
  % $\labelingForm^0(\aEv)$ implies $\clabelingForm^1(\aEv)\lor \labelingForm^2(\aEv)$.

  % We say that $\aPS^0$ is a \emph{downset weakening of $\aPS^1$ by $\aPS^2$} when
  % $\Event^0=\Event^1 \supseteq\Event^2$, ${\le^0}={\le^1}\supseteq{\le^2}$,
  % $\labelingAct^0=\labelingAct^1 \supseteq\labelingAct^2$, and
  % $\labelingForm^0(\aEv)$ implies either
  % $\labelingForm^2(\aEv)$ where $\aEv\in\Event^2$, or $\labelingForm^1(\aEv)$
  % where $\aEv\not\in\Event^2$.

  % for all $\aEv\in \Event^1\setminus\Event^2$: $\labelingForm(\aEv)$ implies
  % $\labelingForm^1(\aEv)$.
  % \begin{displaymath}
  %   \labelingForm(\aEv) \text{ implies }
  %   \begin{cases}
  %     \labelingForm^2(\aEv) & \text{if } \aEv\in\Event^2
  %     \\
  %     \labelingForm^1(\aEv) & \text{otherwise} %\text{if } \aEv\in(\Event^1\setminus\Event^2)
  %   \end{cases}
  % \end{displaymath}

  % We say that $\aPS$ is a \emph{disjunct} of $\aPS^1$ and $\aPS^2$ if 
  % (1) $\Event = \Event^1$, ${\le}={\le^1}$, and
  % $\labelingAct=\labelingAct^1$, (2) 
  % $\Event^2\subseteq\Event^1$,
  % ${\le^2}={\le^1}\restrict{\Event^2}$, and
  % $\labelingAct^2=\labelingAct^1 \restrict{\Event^2}$, (3) for each
  % $\aEv\in(\Event^1\setminus\Event^2)$:
  % $\labelingForm(\aEv)$ implies $\labelingForm^1(\aEv)$, and (4)
  % for each $\aEv\in\Event^2$:
  % $\labelingForm(\aEv)$ implies $\labelingForm^1(\aEv)\lor \labelingForm^2(\aEv)$.

  % We say that $\aPS$ is a \emph{disjunct} of $\aPS^i$ ($i\in I$) if there is
  % some $k\in I$ and some downset $\aPS^i_{\nabla}$ of each $\aPS^i$ such that
  % (1) $\Event = \Event^k$, ${\le}={\le^k}$, and
  % $\labelingAct=\labelingAct^k$, (2) for every $i\in I$,
  % $\Event^k_{\nabla} = \Event^i_{\nabla}$,
  % ${\le^k}_{\nabla}={\le^i}_{\nabla}$, and
  % $\labelingAct^k_{\nabla}=\labelingAct^i_{\nabla}$, (3) for each
  % $\aEv\in(\Event^k\setminus\Event^k_{\nabla})$:
  % $\labelingForm(\aEv)$ implies $\labelingForm^k(\aEv)$, and (4)
  % for each $\aEv\in\Event^k_{\nabla}$:
  % $\labelingForm(\aEv)$ implies $\bigvee_i \labelingForm^i_{\nabla}(\aEv)$.

  % Let $\aPS''$ be a \emph{disjunct} of $\aPS^i$ for $i\in I$ if there is some
  % $\aPS'\in\PRE{\aPS''}$ such that $\Event' = \Event^i$, ${\le'}={\le^i}$,
  % $\labelingAct'=\labelingAct^i$, and $\labelingForm'(\aEv)$ implies
  % $\bigvee_i \labelingForm^i(\aEv)$.
  %
  % We say that $\aPS'$ is a \emph{disjunct} of $\aPS^i$  $(i\in I)$ if
  % $\Event' = \Event^i$, ${\le'}={\le^i}$, $\labelingAct'=\labelingAct^i$, and
  % $\labelingForm'(\aEv)$ implies $\bigvee_i \labelingForm^i(\aEv)$.
  % 
  % Let $\PRE{\aPSS}=\{\aPS'\mid\aPS'$ is a
  % downset of some $\aPS \in \aPSS\}$.
  %
  % Let $\aPS^{i}_{\nabla}$ denote a downset of ${\aPS^{i}}$, for $i\in\{1,2\}$.  We
  % say that $\aPS^{2}$ is a \emph{downset weakening} of $\aPS^1$ by
  % $\aPS^{2}_{\nabla}$ if there is some $\aPS^{1}_{\nabla}$ that implies $\aPS^{2}_{\nabla}$ and
  % $\aPS^1$ implies $\aPS^{2}$.
  % Let $\aPS^{2}$ be a \emph{downset weakening} of $\aPS^1$ by
  % $\aPS^{2}_{\nabla}\in\PRE{\aPS^{2}}$ if $\aPS^1$ implies $\aPS^{2}$ and there is
  % some $\aPS^{1}_{\nabla}\in\PRE{\aPS^{1}}$ such that $\aPS^{1}_{\nabla}$ implies
  % $\aPS^{2}_{\nabla}$.
  % Let
  % $\aPS'$ be an \emph{isomorphism} of $\aPS$ if there is a bijection
  % $f:\Event\fun\Event'$ such that $\labeling(\aEv)=\labeling'(f(\aEv))$, and
  % $\aEv\le\bEv$ iff $f(\aEv)\le'f(\bEv)$. %, and $\aEv\gtN\bEv$ iff $f(\aEv)\gtN'f(\bEv)$.


% Proof of disjunction closure:
% Four properties, 16 cases:
% d1 in E1'
% d2 in E1'
% d1 in E2'
% d2 in E2'


%\myparagraph{Irrelevant Read Elimination}










\myparagraph{Irrelevant Read Introduction}
Compilers sometimes introduce reads in order to lift code.  Consider the
following example \cite[\textsection1.4.5]{SevcikThesis}:
\begin{align*}
  \sem{\IF{\aReg} \THEN \bReg \GETS \aLoc \SEMI \bLoc \GETS \bReg \FI}
  &\not\supseteq
  \sem{\bReg \GETS \aLoc \SEMI \IF{\aReg} \THEN \bLoc \GETS \bReg \FI}
\end{align*}
The right-hand program is derived from the left by introducing an irrelevant
read in the else-branch, then moving the common code out of the
conditional.  Definition \ref{def:cover} does \emph{not} validate this rewrite.

Read introduction is only valid ``modulo irrelevant reads.'' We capture this
idea using \emph{read saturation}.  Read saturation allows us to add actions
of the form $(\DR{x}{v})$ to the left-hand side, validating the inclusion.
% pomsets and
% $(r\neq0\mid\DR{x}{v})$ to the right, thus equating them.

Let $\readc(\aPSS)$ be the set $\aPSS'$ where
$\aPS'\in\aPSS'$ when $\exists\aPS\in\aPSS$ and $\exists D$ such that $\Event'= \Event\uplus D$,
${\le'}\supseteq{\le}$,
${\labeling'} \supseteq{\labeling}$, and
$\forall \bEv\in D.\;\exists\aLoc.\;\exists\aVal$.
%for every $\bEv\in D$ there are $\aLoc$ and $\aVal$ such that
$\labelingAct'(\bEv)=(\DR[\modeRLXquiet]{\aLoc}{\aVal})$.
% \labeltext{Let}{page:readsat} $\readc(\aPSS)$ be the set $\aPSS'$ where
% $\aPS'\in\aPSS'$ when $\exists\aPS\in\aPSS$ and there exist
% $\Event''\subset\Event$ and $D$ such that $\Event'= \Event''\uplus D$,
% ${\le'}\supseteq{\le\restrict{\Event''}}$,
% ${\labeling'} \supseteq{\labeling\restrict{\Event''}}$, and for every
% $\bEv\in D$ there are $\aLoc$ and $\aVal$ such that
% $\labelingAct'(\bEv)=(\DR[\modeRLXquiet]{\aLoc}{\aVal})$.
%
% and
% %$\aPS[\aVal/\aLoc]$ is consistent.
% $\bigwedge_{\aEv\in\Event}\labelingForm(\aEv)[\aVal/\aLoc]$ is satisfiable.
% The last requirement ensures consistency with item \ref{4b} in the
% definition of prefixing.
%
Note that if $\aPSS'\supseteq\aPSS$, then
$\readc(\aPSS')\supseteq \readc(\aPSS)$.

Read introduction \eqref{RI} is valid under the saturated semantics.
\begin{align*}
  \taglabel{RI}
  \readc\sem{\aCmd} & \supseteq
  \readc\sem{\aReg  \GETS \aLoc\SEMI\aCmd}  
  &&\textif \aReg\not\in\free(\aCmd)&&\hbox{}  
\end{align*}

With \ref{RI}, the model satisfies all of the transformations of
\citet[\textsection 5.3-4]{SevcikThesis} except redundant write after read
elimination (see \textsection\ref{sec:valid}) and reordering with external
actions, which we do not model.
% \citet[\textsection 5.3]{SevcikThesis}
% Sevcik 5.3:
% !!!! Redundant Write after Read Elimination
% Redundant Read after Read Elimination 
% Roach Motel
% Irrelevant Read Introduction
% !!!! Reordering with external actions

% Sevcik 5.4:
% irrelevant read elimination,
% read after write elimination,
% write after write elimination,
% reordering of independent accesses


%
% \labeltext{Let}{page:readsat} $\readc(\aPSS)$ be the set $\aPSS'$ where
% $\aPS'\in\aPSS'$ when $\exists\aPS\in\aPSS$ and $\exists D$ such that
% $\Event'= \Event\uplus D$, ${\le'} \supseteq{\le}$,
% ${\labeling'} \supseteq{\labeling}$, and for every $\bEv\in D$ there are
% $\aLoc$ and $\aVal$ such that
% $\labelingAct'(\bEv)=(\DR[\modeRLXquiet]{\aLoc}{\aVal})$.
%