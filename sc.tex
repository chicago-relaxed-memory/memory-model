\section{SC-DRF}


In this section, we prove the SC-DRF theorem, which states that any program
that lacks data races under the SC semantics must only have executions that
are compatible with SC executions.

The program $x\GETS1\PAR x\GETS2$ is considered to have an SC data race, but
$x\GETS1\SEMI x\GETS2$ does not.  In our semantics the only difference
between these is that $x\GETS1\SEMI x\GETS2$ enforces weak order between the
writes.  Note also that the
$\sem{x\GETS1\SEMI a\GETS y}=\sem{a\GETS y\SEMI x\GETS1}$, yet these two must
be distinguished in SC, as per the load-buffering litmus test.

In order to define SC executions and SC data races, it is necessary to
augment our semantics to record program order.  We extend the definitions in
\textsection\ref{sec:semantics} with
${\rpox}\subseteq{\Event}\times{\Event}$, defined as follows
\begin{itemize}
\item
  ${\rpox'} = {\rpox}$
  when $\aPSS'=\aPSS\aSub$
  or $\aPSS'=\aForm\guard\aPSS$
\item
  ${\rpox'} = {\rpox}\restrict{\Event'}$
  when $\aPSS'=\nu\aLoc\st\aPSS$
\item
  ${\rpox'} = {\rpox}^1\cup{\rpox}^2$
  when $\aPSS'=\aPSS^1\parallel\aPSS^2$
\item
  ${\rpox'} = {\rpox}\cup\{(\cEv,\aEv)\mid\aEv\in\Event\}$
  when $\aPSS'=\aAct\prefix\aPSS$ and $\Event' = \Event \cup \{\cEv\}$
\end{itemize}
Let $\FA$ be the identity relation on acquire fences, and likewise $\FR$ on release fences.
We say $\aEv\rrf\bEv$ if $\aEv$ writes $\aLoc$, $\bEv$ reads $\aLoc$, and for any $\cEv$ that writes $\aLoc$ either $\cEv\gtN\aEv$ or $\bEv\gtN\cEv$.
Now define $\rsw$ and $\rhb$.
\begin{align*}
  {\rsw} &= \FR; {\rpox}^*; ({\rrf}\setminus{\rpox}); {\rpox}^*; \FA
  \\
  {\rhb} &= ({\rpox} \cup {\rsw})^+
\end{align*}
Two actions \emph{conflict} if they touch the same location and at least one
is a write action.  A pomset has a \emph{data race} if there are events
$\aEv$ and $\bEv$ such that 
\begin{itemize}
\item $\aEv$ and $\bEv$ are unordered by $\rhb$,
\item $\labelingAct(\aEv)$ and $\labelingAct(\bEv)$ conflict, and
\item $\labelingForm(\aEv)\land\labelingForm(\bEv)$ is satisfiable.
\end{itemize}
The semantics of programs includes SC executions.  Let $\semsc{\aCmd}$ be the
executions where
\begin{itemize}
\item ${\gtN}\cup{\rpox}$ is acyclic (and therefore ${\le}\cup{\rpox}$ is acyclic),
\item prefixing ($\prefix$) and composition ($\parallel$) take disjoint union, and
\item all reads are denoted by explicit actions.
\end{itemize}
Our semantics already ensures ${\le}\subseteq{\rsw}$.

We say that $\aCmd$ has an \emph{SC race} if there is some pomset in $\semsc{\aCmd}$
that contains a data race.

In this section we show that if $\semsc{\aCmd}\subseteq\sem{\aCmd}$ has only
race-free executions, then each pomset $\aPS\in\sem{\aCmd}$ is ``equivalent''
to some $\aPS'\in\semsc{\aCmd}$, where $\aPS'$ may have more events, but
preserves labeling and has less order.

We say that $\aPS\suborder\aPS'$ if there is an injection
$\inj:\Event'\rightarrow\Event$ such that:
\begin{itemize}
\item $\labelingAct'(\aEv) = \labelingAct(\inj(\aEv))$
\item $\labelingForm'(\aEv)$ implies $\labelingForm(\inj(\aEv))$
\item $\labelingForm(\bEv)$ implies $\bigvee_{\aEv\in\inj^{-1}(\bEv)}(\labelingForm'(\aEv))$
\item $\aEv\le'\bEv$ implies $\inj(\aEv)\le\inj(\bEv)$
\item $\aEv\gtN'\bEv$ implies $\inj(\aEv)\gtN\inj(\bEv)$
\end{itemize}

\begin{theorem}
  If $\semsc{\aCmd}$ contains only race-free executions and
  $\aPS\in\sem{\aCmd}$ then there exists some $\aPS'\in\semsc{\aCmd}$ such
  that $\aPS\suborder\aPS'$.
\end{theorem}

To define compatibility, we extend the definitions of
\textsection\ref{sec:semantics} to include an additional order: $\rird$.
\begin{itemize}
\item
  ${\rird'} = {\rird}$
  when $\aPSS'=\aPSS\aSub$
  or $\aPSS'=\aForm\guard\aPSS$
\item
  ${\rird'} = {\rird}\restrict{\Event'}$
  when $\aPSS'=\nu\aLoc\st\aPSS$
\item
  ${\rird'} = {\rird}^1\cup{\rird}^2$
  when $\aPSS'=\aPSS^1\parallel\aPSS^2$
\item
  ${\rird'} = {\rird}\cup\{(\cEv,\aEv)\mid\labelingForm(\aEv) \;\text{is dependent on}\; \aLoc\}$
  when $\aPSS'=\aAct\prefix\aPSS$, $\aAct$ writes $\aLoc$, and $\Event' = \Event \cup \{\cEv\}$
\end{itemize}

From $\rird$, we define ${\rrb}={\rird}^{-1};{\gtN}$.

We want that if there is an execution:
\[\begin{tikzpicture}[node distance=1em]
  \event{a}{\DW{\aLoc}{1}}{}
  \event{b}{\DW{\bLoc}{1}}{below right=1em and 6em of a}
  \event{c}{\DW{\aLoc}{2}}{above right=1em and 1em of b}
  \wk{a}{c}
  \ird{a}{b}
  \rb{b}{c}
\end{tikzpicture}\]
Then there is also
\[\begin{tikzpicture}[node distance=1em]
  \event{a}{\DW{\aLoc}{1}}{}
  \event{b}{\DW{\bLoc}{1}}{below right=1em and 6em of a}
  \event{c}{\DW{\aLoc}{2}}{above right=1em and 1em of b}
  \event{r}{\DR{\aLoc}{1}}{below right=.1em and 2em of a} 
  \wk{a}{c}
  \rf{a}{r}
  \po{r}{b}
\end{tikzpicture}\]


% To see that we need $[\aExp/\aLoc]$ in the rule for write, rather than $[\aVal/\aLoc]$
% consider example:
% \begin{verbatim}
% r=y; if (r) {x=r} else {x=r}; s=x; if (r==s) {z=1}
% \end{verbatim}
% or simplified:
% \begin{verbatim}
% r=y;x=r;s=x; if(s==r){z=1}
% \end{verbatim}
% If you read 37 for $y$, then the predicate on \texttt{Wz1} before the
% read is either $r=r$ or $v=r$, where $v=37$, for example.  In one case you
% get a dependency and in the other you do not.

