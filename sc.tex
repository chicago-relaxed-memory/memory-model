\section{Data Race Free Behaviors are Sequentially Consistent}
\label{sec:sc}

% For any $\aPSS$, then $\closed(\aPSS)$ is set enriched with useless reads
% (preserving augmentation closure) and where we remove any event whose
% precondition is not a tautology.
% \begin{definition}
%   Let $\addRead(\aPSS)$ be the set $\aPSS'$ where $\aPS'\in\aPSS'$ whenever
%   there is $\aPS\in\aPSS$ such that:
%   $\Event' = \Event\cup{\cEv}$,
%   ${\le'} \supseteq {\le}$, 
%   ${\gtN'} \supseteq {\gtN}$,
%   and
%   $\labelingAct'(\cEv) = (\DR{\aLoc}{\aVal})$ and $\labelingAct'(\aEv) = \labelingAct(\aEv)$,
% \end{definition}
% Then $\fclosed(\aPSS)$ 

In this section, we prove the SC-DRF theorem, which states that any program
that lacks data races under the SC semantics must only have executions that
are compatible with SC executions.  We present the result for programs of the
form $\vec{\aLoc}\GETS\vec{0}\SEMI\aCmd$, where $\aCmd$ is restriction-free
and all memory locations accessed. Thus internal actions only arise from
(intra-thread) implicit reads.

We say that two actions have a \emph{data-race conflict} if at least one
action is a write and the other is a write, read, or internal read to the
same location.  Define the relation $\reco$ so that $(\aEv,\bEv)\in{\reco}$
if $\aEv\gtN\bEv$ and $\aEv$ and $\bEv$ have a data-race conflict.

The program $x\GETS1\PAR x\GETS2$ is considered to have an SC data race, but
$x\GETS1\SEMI x\GETS2$ does not.  In our semantics the only difference
between these is that $x\GETS1\SEMI x\GETS2$ enforces weak order between the
writes.  Note also that the
$\sem{x\GETS1\SEMI a\GETS y}=\sem{a\GETS y\SEMI x\GETS1}$, yet these two must
be distinguished in SC, as per the load-buffering and store-buffering litmus tests.

In order to define SC executions and SC data races, it is necessary to
augment our semantics to record program order and internal reads.  We extend the definitions in
\textsection\ref{sec:semantics} with
${\rpox}\subseteq{\Event}\times{\Event}$, defined as follows:
\begin{itemize}
\item
  ${\rpox'} = {\rpox}$
  when $\aPSS'=\aPSS\aSub$
  or $\aPSS'=\aForm\guard\aPSS$
\item
  ${\rpox'} = {\rpox}\restrict{\Event'}$
  when $\aPSS'=\nu\aLoc\st\aPSS$
\item
  ${\rpox'} = {\rpox}^1\cup{\rpox}^2$
  when $\aPSS'=\aPSS^1\parallel\aPSS^2$
\item
  ${\rpox'} = {\rpox}\cup\{(\cEv,\aEv)\mid\aEv\in\Event\}$
  when $\aPSS'=\aAct\prefix\aPSS$ and $\Event' = \Event \cup \{\cEv\}$
\end{itemize}

Define the relation ${\rrfx}$ so that $(\aEv,\bEv)\in{\rrfx}$ if $\aEv$
writes $\aLoc$, $\bEv$ reads $\aLoc$, and for any $\cEv$ that writes $\aLoc$
either $\cEv\gtN\aEv$ or $\bEv\gtN\cEv$.  Let $\IDAcq$ be the identity
relation on acquire events, and likewise $\IDRel$ on release events.  Now
define $\rsw$ and $\rhb$\footnote{For simplicity, the definition of $\rsw$
  does not include release sequences or fences.  If we include these, then
  $\rhb$ relates more events and thus there are fewer races. Our results hold
  under either definition.}.
\begin{align*}
  {\rsw} &= \IDRel; ({\rrfx}\setminus{\rpox}); \IDAcq
  \\
  {\rhb} &= ({\rpox} \cup {\rsw})^+
\end{align*}
Note that our semantics guarantees that ${\rsw}\subseteq{\lt}$.

A pomset has a \emph{data race} if there are events $\aEv$ and $\bEv$ such
that
\begin{itemize}
\item $\aEv$ and $\bEv$ are unordered by $\rhb$,
\item $\labelingForm(\aEv)$ and $\labelingForm(\bEv)$ are tautologies, and
\item $\labelingAct(\aEv)$ and $\labelingAct(\bEv)$ have a data-race conflict.
\end{itemize}

%The semantics of programs includes SC executions.
\begin{definition}
  Let $\semsc{\aCmd}$ be the subset of $\sem{\aCmd}$ such that
  $\aPS\in\semsc{\aCmd}$ whenever $\aPS$ is a top-level pomset and
  ${\lt}\cup{\rpox} \cup {\reco}$ is acyclic.
\end{definition}
%\begin{itemize}
%\item ${\le}\cup{\rpox} \cup {\reco}$ is acyclic ,
% \item prefixing ($\prefix$) and composition ($\parallel$) take disjoint union, and
%\item all reads are denoted by explicit actions.
%\end{itemize}
We argue that this definition is sufficient to capture sequential
consistency: Any total order that linearizes the acyclic relation is
consistent with strong order ($\lt$) and the program order ($\rpox$).
Consistency with $\reco$ ensures that only the last write to a location is
read in such a total order.

%\begin{remark}\label{generator}
% In the rest of this section, we consider ``top-level'' programs of the form
% $\aCmd = \VAR\vec{\aLoc}\SEMI
%     \vec{\aLoc}\GETS\vec{0}\SEMI
%     \vec{\bLoc}\GETS\vec{0}\SEMI
%     \FENCE\SEMI
%     (\aCmd_1 \PAR \cdots \PAR \aCmd_n)
% $.
% We will consider closed and complete executions of $\aCmd$. We use $\semClosed{\aCmd}$ to stand for the subset of $\sem{\aCmd}$  with only  the pomsets that are $\vec{\aLoc}, \vec{\bLoc}$ closed.  
We only consider \emph{generators}, which are top-level pomsets that are
minimal with respect to augmentation and implication.  Since we are
considering finite programs without loops, the pomsets in the semantics of
threads are finite.  Thus, there are no infinite descending chains of
augmentations.

In generators, $\gtN$ is the transitive closure of $(\le \cup
\reco)$. Furthermore, any strong order that is outside of program order must
be induced by a reads-from.  In the two-thread case, we can state the latter
property as follows: suppose $\aEv$ and $\bEv$ are not related by program
order and $\aEv\lt\bEv$; then there exist $\bEv'$ that reads-from $\aEv'$
such that $\aEv\xpox\aEv'$, $\bEv'\xpox\bEv$ and
$\aEv \lt \aEv' \lt \bEv' \lt \bEv$.
%\end{remark}

We prove the following theorem in Appendix~\ref{drfproof}.
\begin{theorem}
  Let $\aPS$ be a generator for $\aCmd$.
  \begin{itemize}
  \item If $\aPS$ does not have a data race, $\aPS \in \semsc{\aCmd}$.
  \item If $\aPS$ has a data race, then there exists
    $\aPS'\in \semsc{\aCmd}$ that has a data race.
  \end{itemize}
\end{theorem}

\endinput

We say that $\aCmd$ has an \emph{SC race} if there is some pomset in $\semsc{\aCmd}$ that contains a data race.


In this section we show that if $\semsc{\aCmd}\subseteq\sem{\aCmd}$ has only
race-free executions, then each pomset $\aPS\in\sem{\aCmd}$ is ``equivalent''
to some $\aPS'\in\semsc{\aCmd}$, where $\aPS'$ may have more events, but
preserves labeling and has less order.

We say that $\aPS\suborder\aPS'$ if there is an injection
$\inj:\Event'\rightarrow\Event$ such that:
\begin{itemize}
\item $\labelingAct'(\aEv) = \labelingAct(\inj(\aEv))$
\item $\labelingForm'(\aEv)$ implies $\labelingForm(\inj(\aEv))$
\item $\labelingForm(\bEv)$ implies $\bigvee_{\aEv\in\inj^{-1}(\bEv)}(\labelingForm'(\aEv))$
\item $\aEv\le'\bEv$ implies $\inj(\aEv)\le\inj(\bEv)$
\item $\aEv\gtN'\bEv$ implies $\inj(\aEv)\gtN\inj(\bEv)$
\end{itemize}

\begin{theorem}
  If $\semsc{\aCmd}$ contains only race-free executions and
  $\aPS\in\sem{\aCmd}$ then there exists some $\aPS'\in\semsc{\aCmd}$ such
  that $\aPS\suborder\aPS'$.
\end{theorem}
% \begin{proof}
%   \begin{itemize}
%   \item
%     \begin{math}
%       \sem{\SKIP}
%       =
%       \{ \emptyset \} 
%     \end{math}
%   \item
%     \begin{math}
%       \sem{\FENCE_{\aF}\SEMI \aCmd}
%       =
%       (\DF{\aF}) \prefix \sem{\aCmd}
%     \end{math}
%   \item
%     \begin{math}
%       \sem{\aLoc\GETS\aExp\SEMI \aCmd}
%       =
%       \textstyle\bigcup_\aVal\; \bigl((\aExp=\aVal) \guard (\DW\aLoc\aVal) \prefix \sem{\aCmd}\bigr)[\aExp/\aLoc]
%     \end{math}
%   % \item
%   %   \begin{math}
%   %     \sem{\REL\aLoc\GETS\aExp\SEMI \aCmd}
%   %     =
%   %     \textstyle\bigcup_\aVal\; \bigl((\aExp=\aVal) \guard (\DWRel\aLoc\aVal) \prefix \sem{\aCmd}\bigr)[\aExp/\aLoc]
%   %   \end{math}
%   \item
%     \begin{math}
%       \sem{\aReg\GETS\aLoc\SEMI \aCmd}
%       =
%       \sem{\aCmd}[\aLoc/\aReg] \cup \textstyle\bigcup_\aVal\; (\DR\aLoc\aVal) \prefix \sem{\aCmd}[\aLoc/\aReg]
%     \end{math}
%   % \item
%   %   \begin{math}
%   %     \sem{\ACQ\aReg\GETS\aLoc\SEMI \aCmd}
%   %     =
%   %     \textstyle\bigcup_\aVal\; (\DRAcq\aLoc\aVal) \prefix \sem{\aCmd}[\aLoc/\aReg]
%   %   \end{math}
%   \item
%     \begin{math}
%       \sem{\IF{\aExp} \THEN \aCmd \ELSE \bCmd \FI}
%       =
%       \bigl((\aExp \neq 0) \guard \sem{\aCmd}\bigr) \parallel \bigl((\aExp=0) \guard \sem{\bCmd}\bigr)
%     \end{math}
%   \item
%     \begin{math}
%       \sem{\aCmd \PAR \bCmd}
%       =
%       \sem{\aCmd} \parallel \sem{\bCmd}
%     \end{math}
%   \item
%     \begin{math}
%       \sem{\VAR\aLoc\SEMI \aCmd}
%       =
%       \nu \aLoc \st \sem{\aCmd}
%     \end{math}
% \end{itemize}
  
% \end{proof}
% \end{theorem}


% To define compatibility, we extend the definitions of
% \textsection\ref{sec:semantics} to include an additional order: $\rird$.
% \begin{itemize}
% \item
%   ${\rird'} = {\rird}$
%   when $\aPSS'=\aPSS\aSub$
%   or $\aPSS'=\aForm\guard\aPSS$
% \item
%   ${\rird'} = {\rird}\restrict{\Event'}$
%   when $\aPSS'=\nu\aLoc\st\aPSS$
% \item
%   ${\rird'} = {\rird}^1\cup{\rird}^2$
%   when $\aPSS'=\aPSS^1\parallel\aPSS^2$
% \item
%   ${\rird'} = {\rird}\cup\{(\cEv,\aEv)\mid\labelingForm(\aEv) \;\text{is dependent on}\; \aLoc\}$
%   when $\aPSS'=\aAct\prefix\aPSS$, $\aAct$ writes $\aLoc$, and $\Event' = \Event \cup \{\cEv\}$
% \end{itemize}

% From $\rird$, we define ${\rrb}={\rird}^{-1};{\gtN}$.

% We want that if there is an execution:
% \begin{tikzdisplay}[node distance=1em]
%   \event{a}{\DW{\aLoc}{1}}{}
%   \event{b}{\DW{\bLoc}{1}}{below right=1em and 6em of a}
%   \event{c}{\DW{\aLoc}{2}}{above right=1em and 1em of b}
%   \wk{a}{c}
%   \ird{a}{b}
%   \rb{b}{c}
% \end{tikzdisplay}
% Then there is also
% \begin{tikzdisplay}[node distance=1em]
%   \event{a}{\DW{\aLoc}{1}}{}
%   \event{b}{\DW{\bLoc}{1}}{below right=1em and 6em of a}
%   \event{c}{\DW{\aLoc}{2}}{above right=1em and 1em of b}
%   \event{r}{\DR{\aLoc}{1}}{below right=.1em and 2em of a} 
%   \wk{a}{c}
%   \rf{a}{r}
%   \po{r}{b}
% \end{tikzdisplay}


% To see that we need $[\aExp/\aLoc]$ in the rule for write, rather than $[\aVal/\aLoc]$
% consider example:
% \begin{verbatim}
% r=y; if (r) {x=r} else {x=r}; s=x; if (r==s) {z=1}
% \end{verbatim}
% or simplified:
% \begin{verbatim}
% r=y;x=r;s=x; if(s==r){z=1}
% \end{verbatim}
% If you read 37 for $y$, then the predicate on \texttt{Wz1} before the
% read is either $r=r$ or $v=r$, where $v=37$, for example.  In one case you
% get a dependency and in the other you do not.

