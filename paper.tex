%% For double-blind review submission
\documentclass[acmsmall,10pt,review,anonymous]{acmart}\settopmatter{printfolios=true}
\AtEndPreamble{%
  \theoremstyle{acmdefinition}
  \newtheorem{remark}[theorem]{Remark}
}
%% For single-blind review submission

% \documentclass[acmsmall]{acmart}\settopmatter{printfolios=true}
% \hypersetup{bookmarksnumbered,bookmarksopen=true,bookmarksdepth=2}
%% For final camera-ready submission
%\documentclass[acmsmall,10pt]{acmart}\settopmatter{}

%% Note: Authors migrating a paper from PACMPL format to traditional
%% SIGPLAN proceedings format should change 'acmsmall' to
%% 'sigplan'.

%\documentclass[acmsmall,screen]{acmart}


%% Some recommended packages.
\usepackage{booktabs}   %% For formal tables:
                        %% http://ctan.org/pkg/booktabs
\usepackage{subcaption} %% For complex figures with subfigures/subcaptions
                        %% http://ctan.org/pkg/subcaption



% \makeatletter\if@ACM@journal\makeatother
% %% Journal information (used by PACMPL format)
% %% Supplied to authors by publisher for camera-ready submission
% \acmJournal{PACMPL}
% \acmVolume{1}
% \acmNumber{1}
% \acmArticle{1}
% \acmYear{2017}
% \acmMonth{1}
% \acmDOI{10.1145/nnnnnnn.nnnnnnn}
% \startPage{1}
% \else\makeatother
% %% Conference information (used by SIGPLAN proceedings format)
% %% Supplied to authors by publisher for camera-ready submission
% \acmConference[PL'17]{ACM SIGPLAN Conference on Programming Languages}{January 01--03, 2017}{New York, NY, USA}
% \acmYear{2017}
% \acmISBN{978-x-xxxx-xxxx-x/YY/MM}
% \acmDOI{10.1145/nnnnnnn.nnnnnnn}
% \startPage{1}
% \fi


%% Copyright information
%% Supplied to authors (based on authors' rights management selection;
%% see authors.acm.org) by publisher for camera-ready submission
% \setcopyright{none}             %% For review submission
%\setcopyright{acmcopyright}
%\setcopyright{acmlicensed}
%\setcopyright{rightsretained}
%\copyrightyear{2017}           %% If different from \acmYear


%% Bibliography style
\bibliographystyle{ACM-Reference-Format}
%% Citation style
%% Note: author/year citations are required for papers published as an
%% issue of PACMPL.
\citestyle{acmauthoryear}   %% For author/year citations



\usepackage{macros}






%%% The following is specific to POPL'18 and the paper
%%% 'Transactions in Relaxed Memory Architectures'
%%% by Brijesh Dongol, Radha Jagadeesan, and James Riely.
%%%
\setcopyright{acmcopyright}
\acmPrice{}
\acmDOI{10.1145/3158106}
\acmYear{2018}
\copyrightyear{2018}
\acmJournal{PACMPL}
\acmVolume{2}
\acmNumber{POPL}
\acmArticle{18}
\acmMonth{1}

% % at most 12 pages, excluding references, not anonymous
% % Titles and Short Abstracts Due	4 January 2019
% % Full Papers Due	11 January 2019
% % Author Feedback/Rebuttal Period	4–8 March 2019
% % Author Notification	29 March 2019
% % Conference	24–27 June 2019
% \documentclass[conference]{IEEEtran}
% \IEEEoverridecommandlockouts

\usepackage{macros}
\newcommand{\ignore}[1]{}
\newcommand{\todo}[1]{{\color{red}\textbf{\{#1\}}}}

\pagestyle{plain}

\begin{document}

%% Title information
\title{Let go of that which does not serve you: Semantic Independence in a Model of Relaxed Memory}
% \title{Semantic Dependency as a Model of Relaxed Memory}
% \title[Short Title]{Full Title}         %% [Short Title] is optional;
%                                         %% when present, will be used in
%                                         %% header instead of Full Title.
% \titlenote{with title note}             %% \titlenote is optional;
%                                         %% can be repeated if necessary;
%                                         %% contents suppressed with 'anonymous'
% \subtitle{Subtitle}                     %% \subtitle is optional
% \subtitlenote{with subtitle note}       %% \subtitlenote is optional;
%                                         %% can be repeated if necessary;
%                                         %% contents suppressed with 'anonymous'


%% Author information
%% Contents and number of authors suppressed with 'anonymous'.
%% Each author should be introduced by \author, followed by
%% \authornote (optional), \orcid (optional), \affiliation, and
%% \email.
%% An author may have multiple affiliations and/or emails; repeat the
%% appropriate command.
%% Many elements are not rendered, but should be provided for metadata
%% extraction tools.


\author{Radha Jagadeesan}
\affiliation{
  \department{School of Computing}
  \institution{DePaul University}
  \country{USA}
}
\email{rjagadeesan@cs.depaul.edu}

\author{Alan Jeffrey}
\orcid{0000-0001-6342-0318}
\affiliation{
  \institution{Mozilla Research}
  \country{USA}
}
\email{ajeffrey@mozilla.com}

\author{James Riely}
\orcid{0000-0002-8731-1463}
\affiliation{
  \department{School of Computing}
  \institution{DePaul University}
  \country{USA}
}
\email{jriely@cs.depaul.edu}


%% Paper note
%% The \thanks command may be used to createame a "paper note" ---
%% similar to a title note or an author note, but not explicitly
%% associated with a particular element.  It will appear immediately
%% above the permission/copyright statement.
% \thanks{Radha Jagadeesan and James Riely were supported in part by NSF CCR }                %% \thanks is optional
%                                         %% can be repeated if necesary
%                                         %% contents suppressed with 'anonymous'


%% Abstract
%% Note: \begin{abstract}...\end{abstract} environment must come
%% before \maketitle command
\begin{abstract}
Relaxed memory models must simultaneously achieve efficient implementability
and thread-compositional reasoning.  Is that why they have become so
complicated?  We argue that the answer is no: It is possible to achieve these
goals by combining an idea from the 60s (preconditions) with an idea from the
 80s (pomsets), at least for \textsc{x64} and \armeight.  We show that the
resulting model (1) supports compositional reasoning for temporal safety
properties, (2) supports all expected sequential compiler optimizations,
(3) satisfies the \drfsc\ criterion, and (4) compiles to \textsc{x64} and \armeight{}
microprocessors without requiring extra fences on relaxed accesses.

\endinput

Relaxed memory models must simultaneously achieve efficient implementability and thread-compositional reasoning.  Is that why they have become so complicated?  We argue that the answer is no: It is possible to achieve these goals by combining an idea from the 60s (preconditions) with an idea from the 80s (pomsets), at least for X64 and ARMv8.  We show that the resulting model (1) supports compositional reasoning for temporal safety properties, (2) supports all expected sequential compiler optimizations, (3) satisfies the DRF-SC criterion, and (4) compiles to X64 and ARMv8 microprocessors without requiring extra fences on relaxed accesses.



DePaul Abstract:

A memory model is a contract between a programmer and a system implementor
which indicates the allowed outcomes of any given program.  Some of the
things allowed on your computer might surprise you!

In your first systems class you learned a simple model of virtual memory: a
nice flat address space.  You also learned that this is a lie!  Memory
systems are remarkably complicated.  The situation is even more complex when
you consider the aggressive optimization performed by current compilers.
Although system designers are able to hide much of this complexity, they
can't hide it all without killing performance.  For fifteen years,
researchers have been looking for a model that is understandable to
programmers while still allowing efficient implementation.

In this talk, we present a (relatively) simple model that does almost
everything we want.  The model combines an idea from the 60s (preconditions)
with an idea from the 80s (pomsets).  We show that the resulting model (1)
supports compositional reasoning for temporal safety properties, (2) supports
all reasonable sequential compiler optimizations, (3) allows programmers to
use a simplistic model for race-free programs, and (4) compiles to X64 and
ARMv8 microprocessors without requiring extra fences on relaxed accesses.





Video material description:

This is a video presentation of our OOPSLA 2020 paper.  The paper presents a
model of relaxed memory using partial orders and preconditions.

The talk is a meditation on the notion of "Thin Air Reads" in relaxed memory
models.  We argue that the whole concept of "out of thin air" is too nebulous
to be useful.  Instead, we should concern ourselves with finding
compositional proof rules for expressive logics.

If Goldilocks were to visit the house of relaxed memory models, she would
find Promises too hot (allowing thin air reads) and Event structures too cold
(disallowing efficient implementation).  Pomsets with Preconditions are "just
right".





Info for session chairs:

This is our third major paper on relaxed memory.  In the talk I will argue
that: Promises/speculation (2010) are too hot, allowing thin air reads.
Event structures (2016) are too cold, disallowing efficient implementation.
Pomsets with Preconditions (2020) are "just right".




Twitter-length summary:

Goldilocks visits relaxed memory: Promises are too hot (allowing thin air).
Event structures are too cold (disallowing efficient implementation).
Pomsets with Preconditions are "just right".

Goldilocks visits relaxed memory: Promises too hot. Event structures too cold. Pomsets with Preconditions "just right".
https://2020.splashcon.org/details/splash-2020-oopsla/70/Pomsets-with-Preconditions-A-Simple-Model-of-Relaxed-Memory
@jriely @asajeffrey

\end{abstract}


%% 2012 ACM Computing Classification System (CSS) concepts
%% Generate at 'http://dl.acm.org/ccs/ccs.cfm'.
\begin{CCSXML}
<ccs2012>
<concept>
<concept_id>10003752.10003753.10003761.10003762</concept_id>
<concept_desc>Theory of computation~Parallel computing models</concept_desc>
<concept_significance>500</concept_significance>
</concept>
% <concept>
% <concept_id>10003752.10003753.10003761.10003763</concept_id>
% <concept_desc>Theory of computation~Distributed computing models</concept_desc>
% <concept_significance>500</concept_significance>
% </concept>
% <concept>
% <concept_id>10003752.10003753.10003761.10003764</concept_id>
% <concept_desc>Theory of computation~Process calculi</concept_desc>
% <concept_significance>500</concept_significance>
% </concept>
% <concept>
% <concept_id>10003752.10003790.10002990</concept_id>
% <concept_desc>Theory of computation~Logic and verification</concept_desc>
% <concept_significance>500</concept_significance>
% </concept>
% <concept>
% <concept_id>10003752.10010124.10010131.10010133</concept_id>
% <concept_desc>Theory of computation~Denotational semantics</concept_desc>
% <concept_significance>500</concept_significance>
% </concept>TT
% <concept>
% <concept_id>10003752.10010124.10010131.10010134</concept_id>
% <concept_desc>Theory of computation~Operational semantics</concept_desc>
% <concept_significance>500</concept_significance>
% </concept>
% <concept>
% <concept_id>10003752.10010124.10010131.10010135</concept_id>
% <concept_desc>Theory of computation~Axiomatic semantics</concept_desc>
% <concept_significance>500</concept_significance>
% </concept>
<concept>
<concept_id>10003752.10010124.10010138.10011119</concept_id>
<concept_desc>Theory of computation~Abstraction</concept_desc>
<concept_significance>500</concept_significance>
</concept>
</ccs2012>
\end{CCSXML}

\ccsdesc[500]{Theory of computation~Parallel computing models}
% \ccsdesc[500]{Theory of computation~Distributed computing models}
% \ccsdesc[500]{Theory of computation~Process calculi}
% \ccsdesc[500]{Theory of computation~Logic and verification}
% \ccsdesc[500]{Theory of computation~Denotational semantics}
% \ccsdesc[500]{Theory of computation~Operational semantics}
% \ccsdesc[500]{Theory of computation~Axiomatic semantics}
\ccsdesc[500]{Theory of computation~Abstraction}
 

%% End of generated code


%% Keywords
%% comma separated list
\keywords{Relaxed Memory Models, Hardware Transactional Memory}  %% \keywords is optional


%% \maketitle
%% Note: \maketitle command must come after title commands, author
%% commands, abstract environment, Computing Classification System
%% environment and commands, and keywords command.
\maketitle

% \title{What Happened}

% \author{
% \IEEEauthorblockN{Radha Jagadeesan}
% \IEEEauthorblockA{\textit{DePaul University}\\
%   \email{rjagadeesan@cs.depaul.edu}}
% \and
% \IEEEauthorblockN{Alan Jeffrey}
% \IEEEauthorblockA{\textit{Mozilla Research}\\
%   \email{ajeffrey@mozilla.com}}
% \and
% \IEEEauthorblockN{James Riely}
% \IEEEauthorblockA{\textit{DePaul University}\\
%   \email{jriely@cs.depaul.edu}}
% }

\section{Introduction}
\citet{Manson:2005:JMM:1047659.1040336} identify the central problem in the design of software relaxed memory models: ``The memory model must strike a balance between ease-of-use for programmers and implementation flexibility for system designers.''   \emph{None} of the extant memory models validate both ``implementation flexibility'' and ``ease-of-use''.   This paper provides a solution.

%In order to sharpen the criteria to evaluate memory models, we first outline desiderata  that concretize the above prescription.  

There are two dimensions to ``implementation flexibility''.  Firstly, the model should be realizable on modern hardware with minimal synchronization.  A canonical example is that the relaxed atomics of C11 (or the plain variables of Java) should not require any extra synchronization.

Secondly, the  model should facilitate compiler transformations.    Ideally, the model should support the known transformations used to optimize  synchronization free single threaded code.  A canonical example is the  commuting of independent statements.


There are also two dimensions to ``ease-of-use''.  Firstly,the \emph{data race free-sequentially consistent (\drfsc)} criterion~\cite{DBLP:journals/tpds/AdveH93, DBLP:conf/isca/AdveH90} permits the programmer to forget about relaxed memory for correctly synchronized programs.

Secondly, all programs, including those with data races,  should support compositional reasoning on temporal safety properties~\cite{PnueliSafety,Misra:1981:PNP:1313338.1313770,StarkSafety,Abadi:1993:CS:151646.151649}\nofootnote{TOO MUCH FOR HERE The simplest form of such a composition principle is:
\[
  \frac{
      \afo, \aPSS_1 \models\afo
      \qquad
      \afo, \aPSS_2 \models\afo
    }{\aPSS_1 \parallel \aPSS_2 \models \afo}
\]
where $\afo$ is a temporal safety property on read and write actions, and $\afo, \aPSS_1 \models\afo$ indicates that $\aPSS_1$ satisfies $\afo$ if the environment satisfies $\afo$.}.
%to permit us to reason separately about individual threads validating safety properties. 
% ``Out Of Thin Air'' (\oota) executions invalidate the 
% composability  of safety properties.
Consider the ``type unsafety'' example, from
\citep[Figure 8]{DBLP:journals/toplas/Lochbihler13}:
\begin{displaymath}
  %\label{types}
  %x\GETS 0 \SEMI  y\GETS 0  \SEMI  b\GETS 0 \SEMI  
  \begin{array}[t]{l}
    x\GETS y
    \PAR
    b \GETS 1   
    \PAR\\
    \aReg\GETS x \SEMI \IF{b} \THEN \bReg \GETS \NEW \texttt{C()} \ELSE \aReg \GETS \NEW \texttt{D()} \FI  \SEMI y \GETS \aReg \SEMI 
  \end{array}
\end{displaymath}
Informally, all threads satisfy the invariant that conjoins ``a creation of a \texttt{C} object is preceded by a read of \texttt{b} as $1$'' and ``a creation of a \texttt{D} object is preceded by a read of \texttt{b} as $0$''.  If composability of safety were to hold, the full program would satisfy the invariant.
\citeauthor{DBLP:journals/toplas/Lochbihler13} argues that composability is necessary to prove type safety of racy Java programs without partitioning memory by type---an unrealistic assumption for any memory allocator. %    and facilitates the proof of a temporal invariant that forbids``pointer forging''  within a single execution.  
 % Prima facie, the allocation operation can pick the same address for the two new objects, because only one of them occurs in any one execution.   This causes type safety to break for racy programs in the JMM, forcing~\citeauthor{DBLP:journals/toplas/Lochbihler13} to include the type information in the address itself.

Models that validate ``ease-of-use'' include Sequential Consistency (SC)~\citep{Lamport:1979:MMC:1311099.1311750}, \citep{Dolan:2018:BDR:3192366.3192421}, \citep{DBLP:conf/pldi/LahavVKHD17}, \citep{DBLP:conf/lics/JeffreyR16}, RC11 \citep{DBLP:conf/pldi/LahavVKHD17}, and~\citep{BoehmOOTA}.  However, these models all invalidate reordering of independent statements.  Several require extra fences after read actions in hardware implementations. % \cite{Dolan:2018:BDR:3192366.3192421,BoehmOOTA,DBLP:conf/lics/JeffreyR16} forbid breaking of the program order from reads to writes and thus require extra fences after read actions in hardware implementations.
% \citep{Boehm:2014:OGA:2618128.2618134} show that the RC11 model %\cite{DBLP:conf/pldi/LahavVKHD17}
% forces a dependency or a fence between a relaxed atomic read and a subsequent relaxed atomic write.  


% Whereas the JMM was designed with these two aims,
Java and  related models \citep{Manson:2005:JMM:1047659.1040336,DBLP:conf/esop/JagadeesanPR10,DBLP:conf/popl/KangHLVD17,Chakraborty:2019} invalidate compositional reasoning, as demonstrated by variants of \citeauthor{DBLP:journals/toplas/Lochbihler13}'s type unsafety example.  Thus, they fail to support either type safety or realistic memory allocation.


Our approach has two key ingredients.  

First, we focus on multi-copy atomicity (\mca), which holds that when a write becomes visible to one threads it must become visible to all threads \nofootnote{\mca\ is traditionally explored in hardware memory models.   \tso\ (see, e.g.~\cite{DBLP:journals/cacm/SewellSONM10}) and recent architectures, such as \armeight\ (see, e.g.~\cite{DBLP:journals/pacmpl/PulteFDFSS18}), are \mca, but not older architectures, such as \ppc\ (see, e.g.~\cite{DBLP:conf/pldi/SarkarSAMW11}) or \armseven\ (see, e.g.~\cite{DBLP:conf/popl/AlglaveFIMSSN09}).} \citep{DBLP:journals/pacmpl/PulteFDFSS18}.  As envisioned in chapter 3.3 of~\citet{AlglaveThesis},  this permits us to address cross-thread dependencies simply by a partial order in a pomset model~\citep{GISCHER1988199,Plotkin:1997:TSP:266557.266600}, with the acylicity of the pomset providing a global notion of time.  

Just as \mca{} in hardware dramatically simplifies the programmer model \citep{DBLP:journals/pacmpl/PulteFDFSS18},
the  acyclicity of the pomset provides a very simple global notion of time in our software
memory model.
% As far as we are aware, ours is the first language-level model to capture \mca.   The appeal of \mca{} in hardware is the dramatically simpler programmer model \citep{DBLP:journals/pacmpl/PulteFDFSS18}.  We believe it has the same appeal at the language-level.  

Second, we weaken the program-order within a thread to capture only essential dependencies.
Previous language models have used syntactic notions of dependency \cite{Batty:2011:MCC:1926385.1926394}.
Instead, we embed formulae in pomset events, using classical
Hoare logic~\citep{Hoare:1969:ABC:363235.363259, gordonHoare} to compute dependencies when calculating the semantics of read and write events. %, as presented by~\citet{}.  

Consider the following sequential program fragments:
\begin{align*}
& \aCmd_1: \aLoc \GETS 1 \SEMI \bLoc \GETS 1 \\[-1ex]
& \aCmd_2: \aReg \GETS \aLoc \SEMI \IF{\aReg} \THEN\ \bLoc \GETS 1 \ELSE \bLoc \GETS 1  \FI \\[-1ex]
& \aCmd_3: \aLoc \GETS 1 \SEMI \IF{\aReg} \THEN \bLoc \GETS 1 \FI \\[-1ex]
&\aCmd_4:  \bLoc \GETS 1 \SEMI \aLoc \GETS 1
\end{align*}
All these fragments satisfy  $\hoare{\TRUE}{\aCmd_i}{\bLoc =1}$; thus, in each case, the write of $y$ is independent of
any code that precedes it in program order.  This allows a compiler or processor to reorder the write with respect to the code that precedes it.\nofootnote{IN RELATED Thus, the model fully reaps the benefits of viewing a memory model in terms of (sequential) program transformations, eg. see~\citep{Saraswat:2007:TMM:1229428.1229469,DBLP:conf/fm/LahavV16,
DBLP:conf/popl/DemangeLZJPV13,DBLP:conf/esop/FerreiraFS10}, without explicitly being formalized as such.}.


We show the following.
\begin{itemize}
\item The model accounts for all the concurrency features of the C11 memory model, including relaxed, release-acquire and SC atomics, fences, and RMW.

\item The model satisfies the \drfsc\ criterion  (\textsection\ref{sec:sc}).

\item The model compiles to \armeight\ and \tso\ {\em without} extra synchronization for raw variables\nofootnote{Compilation to \armseven\ or \ppc\ requires extra synchronization.}  (\textsection\ref{sec:arm}).

\item The model validates compositional reasoning on safety properties (\textsection\ref{sec:logic}).  

\item The model validates single-threaded compiler optimizations, as discussed below (\textsection\ref{sec:opt}).
\end{itemize}

We discuss compiler optimizations both concretely and abstractly.
Concretely, we show the validity of specific optimizations, such as the roach motel laws for synchronization.
Abstractly, we establish sufficient conditions to replace any command $\aCmd$ by an equivalent $\bCmd$: if
$\aCmd$ and $\bCmd$ are synchronization free and sequentially equivalent, and furthermore $\bCmd$ is \emph{linear}---performs at most one read and at most one write on any location in any execution---then $\aCmd$ can be refined to $\bCmd$.  

The linearity restriction ensures that the context cannot interfere with the atomic execution of the command, and, dually, that the atomic execution of the command cannot interfere with the context.   To see the need for this, consider that the introduction of redundant reads is valid sequentially, but not valid concurrently. For example, $\aReg \GETS \aLoc \SEMI \IF{\aReg \NOTEQ \aReg}\THEN \bLoc \GETS 1 \FI$ cannot be refined to $\aReg \GETS \aLoc \SEMI \bReg \GETS \aLoc  \SEMI  \IF{\aReg \NOTEQ \bReg} \THEN \bLoc \GETS 1 \FI$.  In a concurrent context, the latter program may see different values for the two reads.
% On the last point, we demonstrate a further completeness result.  Whereas the JMM aims to validate {\em all} sequential optimizations, it is clear that is impossible.  For example, the introduction of redundant reads  but not valid concurrently. Thus, $\aReg \GETS \REF{\aLoc} \SEMI \IF{\aReg != \aReg} \cLoc \GETS 1 \FI$ cannot be replaced by $\aReg \GETS \aLoc \SEMI \bReg \GETS \aLoc  \SEMI 
% \IF{\aReg != \bReg} \THEN \cLoc \GETS 1 \FI$.  

% Our model does the best possible under these constraints.   Call a program fragment ``linear'' if it does at most one read and at most one write on any location in any execution.  Thus, the context is unable to interfere with the atomic execution of the command; dually, neither can the atomic execution of the command interfere with the context.  We show that if sequential and synchronization free $\aCmd$ and $\bCmd$ are sequentially equivalent, and furthermore $\bCmd$ is ``'linear'' in this sense, then $\aCmd$ can be validly replaced by $\bCmd$.  
% Similarly, for redundant writes; $\aLoc \GETS 1$ cannot be 
% replaced by $\aLoc \GETS 1 \SEMI \aLoc \GETS 1 $ in a model 
% with coherence.





%\paragraph*{Rest of the paper. }  We begin with an informal introduction to the modeling ideas in \textsection\ref{sec:model:intro}, developing the precise formalities in \textsection\ref{sec:model}.   \textsection\ref{sec:sc} proves the DRF theorem, whereas \textsection\ref{sec:arm} provides a compilation into \armeight\ and \tso.  Single threaded optimizations, and the associated completeness theorems are addressed in \textsection\ref{sec:opt}.  \textsection\ref{sec:logic} describes a temporal logic, and a compositional proof principle for proving safety properties.  \textsection\ref{sec:examples}.develops more illustrative examples.  We address related work in \textsection\ref{sec:ldrf} and conclude after.  An appendix contains details of proofs and further examples.

% We give an informal introduction to the model in \textsection\ref{sec:model:intro} before presenting the precise formalities in \textsection\ref{sec:model}.
% \textsection\ref{sec:sc} proves the DRF theorem, whereas \textsection\ref{sec:arm} provides a compilation into \armeight\ and \tso.  Single threaded optimizations, and the associated completeness theorems are addressed in \textsection\ref{sec:opt}.  \textsection\ref{sec:logic} describes a temporal logic, and a compositional proof principle for proving safety properties.
% We %present additional examples in \textsection\ref{sec:examples} and
% end with
% a discussion of related work in \textsection\ref{sec:ldrf}.
% An appendix
% contains details of proofs and further examples.

In the main paper, we present the model, examples, the results concerning
compositional reasoning and optimization, and a discussion of
related work.  The \drfsc\ and \armeight/\tso-compilation results may be found in the appendix.


\endinput

To the reader interested in models that forbid load buffering, we provide a way to adapt our model to forbid the relaxing of the program order from reads to writes, thus modeling~\cite{Dolan:2018:BDR:3192366.3192421,BoehmOOTA}.  Our new contributions for such a reader are an approach to validating data-sensitive compiler optimizations and compositional reasoning of temporal properties.   


We illustrate the last criterion with two examples.   
First, consider the well-known ``Out Of Thin Air'' (\oota) litmus test, with all variables initialized to $0$:
\begin{equation}
  \label{oota1}
  %x\GETS0\SEMI y\GETS0\SEMI
  (y\GETS x \PAR x \GETS y)
\end{equation}
Informally, both threads satisfy the invariant that conjoins ``A write of 1  to x  requires a prior read  of 1 from y'' and ``A write of 1  to y  requires a prior read  of 1 from x ''.  If composition holds, the full program satisfies the invariant.  Since the variable declaration closes the program from other writes to $x,y$, we  deduce the conjunction of  ``A write of 1  to x  requires a prior write  of 1 to x'' and ``A write of 1  to y  requires a prior write  of 1 from x'' . Thus, we deduce that ``A write of 1  to x  requires a prior write  of 1 to x'', and consequently ``there is no write of 1 to x''. 

provides an  {\em objectively  falsifiable} measurement of \oota\ in a memory model. 




Merge order.
Augment order.
Implication order.
Generator is minimal in all of these, so po has unique predecessors.

cross thread edges in generator are write-to-read and not from augmentation.

\begin{comment}
https://preshing.com/20131125/acquire-and-release-fences-dont-work-the-way-youd-expect/

Cannot encode R/A actions with actions+fences...

A release operation prevents preceding memory operations from being delayed
past it (a;Rel =/=> Rel;a)
 
A release fence prevents preceding memory operations from being delayed past
subsequent writes (a;FR;w =/=> w;a;FR)

An acquire operation prevents subsequent memory operations from being advanced
before it (Acq;a =/=> a;Acq)

An acquire fence prevents subsequent memory operations from being advanced
before prior reads (r;FA;a =/=> FA;a;r)

https://www.modernescpp.com/index.php/fences-as-memory-barriers

StoreLoad: Full fence allows a store before to be reordered with respect to a
load after (wx;F;ry) ===> (ry;F;wx)

StoreLoad+LoadLoad: Release fence also allows (rx;FR;ry) ===> (ry;FR;rx)

StoreLoad+StoreStore: Acquire fence also allows (wx;FR;wy) ===> (wy;FR;wx)

LoadStore: No fence allows a prior load to reorder w.r.t. a subsequent store
(rx;FR;wy) =/=> (wy;FR;rx)

https://preshing.com/20120710/memory-barriers-are-like-source-control-operations/
Good news is that a fullFence does it.

Bizarrely, it seems this is not supported in C++... You have to go to assembly.
\end{comment}

\section{Model}
\label{sec:model}


\subsection{Background: 3-valued pomsets}
\label{sec:pomsets}

Structures similar to 3-valued pomsets have come up in many guises, for example
rough sets~\cite{Pawlak1982} or ultrametrics over
$\{0,{}^1\!/_2,1\}$. They correspond to axioms A1--A3 of Lamport's
\emph{system executions}~\cite{DBLP:journals/dc/Lamport86}.
They are the notion of pomset given by interpreting
$\bEv\le\aEv$ in a 3-valued logic~\cite{Urquhart1986}. 


\begin{definition}
  \label{def:3valued}
  A \emph{3-valued pomset} with alphabet $\Alphabet$ is tuple $(\Event,
  {\le}, {\gtN}, \labeling)$, such that 
  \begin{itemize}
  \item $\Event$ is a set of \emph{states},
  %\item $\ESub\subseteq\Event$ is a set of \emph{accepting states}, 
  \item $\labeling: \Event \fun \Alphabet$ is a \emph{labeling},
  \item ${\le} \subseteq (\Event\times\Event)$ is a partial order, and
    % \begin{enumerate}
    % \item $\aEv \le \aEv$,
    % \item if $\bEv \le \aEv$ and $\aEv \le \bEv$ then $\bEv = \aEv$,
    %   \\(this follows from 5a and 5b)
    % \item if $\cEv \le \bEv \le \aEv$ then $\cEv \le \aEv$, and
    % \end{enumerate}
  \item ${\gtN} \subseteq (\Event\times\Event)$ such that:
    \begin{itemize}
    \item\label{5a} if $\bEv \le \aEv$ then $\bEv \gtN \aEv$,
    \item\label{5b} if $\bEv \le \aEv$ and $\aEv \gtN \bEv$ then $\bEv = \aEv$,
    \item\label{5c} if $\cEv \le \bEv \gtN \aEv$ or $\cEv \gtN \bEv \le \aEv$ then $\cEv \gtN \aEv$.
    \end{itemize}
\end{itemize}
\end{definition}
Note that $\gtN$ is reflexive.


We visualize a pomset as a graph where the nodes are drawn from $\Event$,
each node $\aEv$ is labeled with $\labeling(\aEv)$, and an edge
$\bEv \rightarrow \aEv$ corresponds to an ordering $\bEv\le\aEv$.  We
visualize $(\bEv \gtN \aEv)$ as a dashed arrow from $\bEv$ to $\aEv$.  For
example:
\begin{tikzdisplay}[node distance=1em]
  \event{rx1}{a}{}
  \event{wy0}{b}{below right=of rx1}
  \event{wy1}{c}{above right=of wy0}
  \po{rx1}{wy0}
  \po{rx1}{wy1}
  \wk{wy0}{wy1}
\end{tikzdisplay}
is a visualization of the pomset where:
\[\begin{array}{c}
    E = \{ 0,1,2 \}
    \quad
    {\labeling} = \{(0,a),\,(1,b),\,(2,c)\}
    \\
    {\le} = \{(0,1),\,(0,2)\}\cup\{(0,0),\,(1,1),\,(2,2)\}
    \quad
    {\gtN} = {\le}\cup\{(2,3)\}
\end{array}\]
We refer to edges introduced by $(\bEv \leq \aEv)$ as
\emph{strong edges} and by $(\bEv \gtN \aEv)$
as \emph{weak edges}.


\subsection{Data models}
\label{sec:preliminaries}

A \emph{data model} consists of:
\begin{itemize}
\item a set of \emph{values} $\Val$, ranged over by
  $\aVal$ and $\bVal$,
\item a set of \emph{registers} $\Reg$, ranged over by
  $\aReg$ and $\bReg$,
\item a set of \emph{expressions} $\Exp$, ranged over by
  $\aExp$, $\bExp$, $\cExp$ and $\dExp$,
\item a set of \emph{memory locations} $\Loc$, ranged over by $\aLoc$ and
  $\bLoc$, 
\item a set of \emph{logical formulae} $\Formulae$, ranged over by
  $\aForm$ and $\bForm$, and
\item a set of \emph{actions} $\Act$, ranged over by $\aAct$ and $\bAct$.
\end{itemize}

Let $\aSub$ range over substitutions of the form
$\aForm[\aLoc/\aReg]$ or $\aForm[\bExp/\aLoc]$.

We require that data models satisfy the following:
\begin{itemize}
\item the sets of values, registers and memory locations are disjoint,
\item values include at least the constants $0$ and $1$,
\item expressions include at least registers and values,
\item memory locations have the form $\REF{\aVal}$,
\item expressions do \emph{not} include memory locations or the operator $\REF{\aExp}$,
\item formulae include at least $\TRUE$, $\FALSE$, and equalities of the form
  $(\aExp=\aVal)$ and $(\REF{\aExp}=\aLoc)$,
\item formulae are closed under negation, conjunction, disjunction, and
  substitution\footnote{Since formulae are closed under substitutions of the
    form $\aForm[\aLoc/\aReg]$, they must include equalities of the form
    $(\aEExp=\aVal)$ and $(\REF{\aEExp}=\aLoc)$, where $\aEExp$ is an
    \emph{extended expression} that includes memory locations.  We elide the
    details.  By composition of the closure conditions, formulae must also be
    closed under that substitutions of the form
    $\aForm[\aExp/\aReg]=\aForm[\aLoc/\aReg][\aExp/\aLoc]$.}, and
\item there is a relation $\vDash$ between formulae.
\end{itemize}

We say that $\aForm$ is \emph{independent of $\aLoc$} whenever
$\aForm \vDash \aForm[\aVal/\aLoc] \vDash \aForm$ for every $\aVal$, and that
$\aForm$ is \emph{dependent on $\aLoc$} otherwise.  We say that $\aForm$ is
\emph{location independent} if it is independent of every location.

We say that $\aForm$ \emph{implies} $\bForm$ whenever $\aForm\vDash\bForm$,
that $\aForm$ is a \emph{tautology} whenever $\TRUE\vDash\aForm$, that
$\aForm$ is \emph{unsatisfiable} whenever $\aForm\vDash\FALSE$.

For the actions of a data model, we require that
\begin{itemize}
\item there are partial functions $\rreads$ and
  $\rwrites: \Act \fun (\Loc \times \Val)$,
\item there are sets $\Rel$ and $\Acq \subseteq\Act$, and
\item there is a function $\finternalize: (\Loc\times\Act) \fun \Act$ that
  satisfies the restrictions given below.
\end{itemize}

We say that $\aAct$ \emph{reads} $\aVal$ \emph{from} $\aLoc$ whenever
$\rreads(\aAct) = (\aLoc,\aVal)$, and that $\aAct$ \emph{writes} $\aVal$
\emph{to} $\aLoc$ whenever $\rwrites(\aAct) = (\aLoc,\aVal)$.  Two actions
\emph{conflict} if at least one action writes a location and the other either
reads or writes the same location.  Actions that read or write are
\emph{external}, other actions are \emph{internal}.
% Actions in
% $\Ext=\fdom(\rreads)\cup\fdom(\rwrites)$ are \emph{external}, whereas those
% in $\Int=\Act\setminus\Ext$ are \emph{internal}.

We say that $\aAct$ is an \emph{acquire} if $\aAct\in\Acq$, and that $\aAct$
is a \emph{release} if $\aAct\in\Rel$.  Actions that acquire or release are
\emph{synchronizations}, other actions are \emph{relaxed}.
% We say that $\aAct$ is a
% \emph{synchronization} if it is either a release or an acquire.

We require that $\finternalize$ satisfy the following:
\begin{itemize}
\item the codomain of $\finternalize$ includes only internal actions, %$\fcodom(\finternalize)\subseteq\Int$,
\item $\finternalize(\aAct)$ is an acquire exactly when $\aAct$ is an acquire, and 
\item $\finternalize(\aAct)$ is a release exactly when $\aAct$ is a release.
\end{itemize}


In examples, we use actions of the form $(\DR{\aLoc}{\aVal})$, which reads
$\aVal$ from $\aLoc$, and $(\DW{\aLoc}{\aVal})$, which writes $\aVal$ to
$\aLoc$, $(\DRAcq{\aLoc}{\aVal})$, which is an acquire that reads $\aVal$
from $\aLoc$, $(\DWRel{\aLoc}{\aVal})$, which is a release that writes
$\aVal$ to $\aLoc$, and $(\DF{})$, which is internal and both an acquire and
a release.  For each external action, we also define a corresponding internal
action
% which replaces the letter $\mathsf{R}$ or $\mathsf{W}$ with $\tau$.
denoted by prefixing $\tau$.
For example, $(\iDRAcq{\aLoc}{\aVal})$ is an acquiring internal action, which
neither reads nor writes. In pictures, we draw internal actions grayed out,
rather than using $\tau$.  For example, the ``read'' action is internal in:
\begin{tikzdisplay}[node distance=1em]
  \event{wx1}{\DW{x}{1}}{}
  \internal{rx1}{\DR{x}{1}}{below right=of rx1}
  \event{wy1}{\DW{y}{1}}{above right=of wy0}
  \po{wx1}{wy1}
\end{tikzdisplay}


% In examples, we use fence actions of the form $(\DF{\aF})$, where the annotation
% indicates that the fence is a release ($\FR$), an acquire ($\FA$) or both ($\FF$):
% \begin{displaymath}
%   \aF\BNFDEF\FR\BNFSEP\FA\BNFSEP\FF
% \end{displaymath}

\subsection{3-valued pomsets with preconditions}

Fix an alphabet $\Alphabet=(\Formulae\times\Act)$.
Define %$\labelingForm$ and $\labelingAct$ so that
$\labelingForm(\aEv)=\aForm$ and $\labelingAct(\aEv)=\aAct$ whenever
$\labelingForm(\aEv)=(\aForm\mid\aAct)$.

We write pairs in $(\Formulae\times\Act)$ as $(\aForm \mid \aAct)$.  We elide
$\aForm$ when it is a tautology.

We lift terminology from logical formulae and actions to events. For example,
we say that $\aEv$ is unsatisfiable when $\labelingForm(\aEv)$ is unsatisfiable,
and that $\aEv$ is an acquire when $\labelingAct(\aEv)$ is an acquire.


\begin{definition}
  A pomset is $\aLoc$-\emph{coherent} if, when restricted to events that read
  or write $\aLoc$, $\gtN$ forms a partial order.
\end{definition}
As we shall see below, for top-level pomsets we could equivalently require
that $\gtN$ forms a total order.

In this paper, we are not investigating microarchitecture.  So we make the
global assumption formulae can only get stronger in dependent actions:
\begin{definition}
  \label{def:3pre}
  A \emph{3-valued pomset with preconditions} is a 3-valued pomset such
  that
  \begin{itemize}
  \item it is $\aLoc$-coherent for every $\aLoc$, and
  \item $\labelingForm(\aEv)$ implies $\labelingForm(\bEv)$ whenever
    $\bEv\le\aEv$.
  \end{itemize}
\end{definition}

In the remainder of the paper, we refer to 3-valued pomsets with
preconditions simply as \emph{pomsets}.

We give the semantics of programs as sets of pomsets.  Each pomset
$\aPS\in\sem{\aCmd}$ will represent a single execution of $\aCmd$.  We do not
expect $\sem{\aCmd}$ to be prefixed closed; thus, one may view each
$\aPS\in\sem{\aCmd}$ as a \emph{completed} execution\footnote{NOTE: because
  implication closed, any event can go false, and we kill everything after
  it, so that means we do get a kind of prefix closure.}.  However, we do
expect the sets of pomsets given by the semantics to be closed with respect
to \emph{isomorphism}, \emph{augmentation} and \emph{implication}.
\begin{definition}
  $\aPS'$ is an \emph{isomorphism} of $\aPS$ if there is a bijection
  $f:\Event\fun\Event'$ such that $\labeling(\aEv)=\labeling'(f(\aEv))$,
  $\aEv\le\bEv$ iff $f(\aEv)\le'f(\bEv)$, and $\aEv\gtN\bEv$ iff
  $f(\aEv)\gtN'f(\bEv)$.

  $\aPS'$ is an \emph{augmentation} of $\aPS$ if $\Event'=\Event$,
  ${\labeling'}={\labeling}$, ${\le'}\supseteq{\le}$, and
  ${\gtN'}\supseteq{\gtN}$.

  $\aPS'$ \emph{implies} $\aPS$ if $\Event'=\Event$, ${\le'}={\le}$,
  ${\gtN'}={\gtN}$, $\labelingAct'=\labelingAct$, and $\labelingForm'(\aEv)$
  implies $\labelingForm(\aEv)$ for all $\aEv\in\Event$.
\end{definition}
% \begin{definition}
%   $\aPS'$ is an \emph{augmentation} of $\aPS$ if $\Event'=\Event$, ${\labeling'}={\labeling}$,
%   ${\le}\subseteq{\le'}$, %$\aEv\le\bEv$ implies $\aEv\le'\bEv$,
%   and ${\gtN}\subseteq{\gtN'}$. %$\aEv\gtN\bEv$ implies $\aEv\gtN'\bEv$,
%   % $\labelingAct'=\labelingAct$, and  $\labelingForm'(\aEv)$ implies $\labelingForm(\aEv)$.
%   % if $\labeling(\aEv) = (\bForm \mid \bAct)$ then $\labeling'(\aEv) = (\bForm' \mid \bAct)$ where $\bForm'$ implies $\bForm$.
% \end{definition}


% Restriction also filters a set of pomsets; we have
% $(\nu\aLoc\st\aPSS)\subseteq\aPSS$.
% The definition requires that we define
% when a read is possible.

% \begin{definition}\label{def:rf}
%   In a pomset, $\aEv$ \emph{can read $\aLoc$ from} $\bEv$ whenever: 
%   \begin{itemize}
%   \item $\bEv \lt \aEv$,  
%   \item $\aEv$ implies $\bEv$,
%   \item $\bEv$ writes $\aVal$ to $\aLoc$,
%     and $\aEv$ reads $\aVal$ from $\aLoc$, and
%   \item if $\cEv$ writes to $\aLoc$
%     then either $\cEv \gtN \bEv$ or $\aEv \gtN \cEv$.
%   \end{itemize}
% \end{definition}

At top-level, we expect the use of each variable to be \emph{closed}.

\begin{definition}
\label{def:x-closed}
  A pomset is \emph{$\aLoc$-closed} if, for every $\aEv\in\Event$:
  \begin{itemize}
  \item $\aEv$ is independent of $\aLoc$, and
  \item if $\aEv$ reads $\aVal$ from $\aLoc$, then there is some $\bEv$ such that
    % $\aEv$ can read $\aLoc$ from $\bEv$.
  \begin{itemize}
  \item $\bEv \lt \aEv$,  
  %\item $\aEv$ implies $\bEv$,
  \item $\bEv$ writes $\aVal$ to $\aLoc$, and
  \item if $\cEv$ writes to $\aLoc$
    then either $\cEv \gtN \bEv$ or $\aEv \gtN \cEv$.
  \end{itemize}    
  % \item if $\aEv$ writes $\aLoc$, then either $\aEv$ is a synchronization or there is
  %   some $\bEv$ such that $\bEv$ can read $\aLoc$ from $\aEv$.
  \end{itemize}

  A pomset is \emph{top-level} if it is $\aLoc$-closed
  for every $\aLoc$.
\end{definition}
% Our model of reads-from is strong, and could be weakened by replacing the
% requirement $\bEv\lt\aEv$ % in Definition~\ref{def:rf}
% by $\bEv\gtN\aEv$. It remains to be seen how this impacts the model.


For readability, we often highlight the reads-from edges as well.
% for example:
For example:
\begin{tikzdisplay}[node distance=1em]
  \event{wx0}{\DW{\aLoc}{0}}{}
  \event{wx1}{\DW{\aLoc}{1}}{right=of wx0}
  \event{rx1}{\DR{\aLoc}{1}}{right=2.5 em of wx1}
  \event{wx2}{\DW{\aLoc}{2}}{right=of rx1}
  \rf{wx1}{rx1}
  \wk{wx0}{wx1}
  \wk{rx1}{wx2}
\end{tikzdisplay}


Need to forbid cycles in $\gtN$ per location:
\[
  \IF{x\EQ1}\THEN r\GETS x \FI
  \PAR
  x\GETS 1
  \PAR
  x\GETS2
  \PAR
  \IF{x\EQ2}\THEN s\GETS x \FI
\]
which includes the execution:
\begin{tikzdisplay}[node distance=1em]
  \event{wx1}{\DW{x}{1}}{}
  \event{wx2}{\DW{x}{2}}{below=of wx1}
  \event{rx1a}{\DR{x}{1}}{left=of wx1}
  \event{rx2a}{\DR{x}{2}}{left=of wx2}
  \event{rx1b}{\DR{x}{1}}{right=of wx1}
  \event{rx2b}{\DR{x}{2}}{right=of wx2}
  \po{rx1a}{rx2a}
  \po{rx2b}{rx1b}
  \rf{wx1}{rx1a}
  \rf{wx1}{rx1b}
  \rf{wx2}{rx2a}
  \rf{wx2}{rx2b}
  \wk{rx1a}{wx2}
  \wk{rx2a}{wx1}
  \wk{rx1b}{wx2}
  \wk{rx2b}{wx1}
\end{tikzdisplay}
This satisfies the requirements for $x$-closure, but is not coherent.

With the restrictions in $x$-closure, forbidding cycles forces an order
between any writes that are read from:
\[
  x\GETS 3
  \PAR
  \IF{x\EQ1}\THEN r\GETS x \FI
  \PAR
  x\GETS 1
  \PAR
  x\GETS2
\]
\begin{tikzdisplay}[node distance=1em]
  \event{wx1}{\DW{x}{1}}{}
  \event{wx2}{\DW{x}{2}}{below=of wx1}
  \event{rx1a}{\DR{x}{1}}{left=of wx1}
  \event{rx2a}{\DR{x}{2}}{left=of wx2}
  \po{rx1a}{rx2a}
  \rf{wx1}{rx1a}
  \rf{wx2}{rx2a}
  \wk{rx1a}{wx2}
  \wk{wx1}{wx2}
  \event{wx3}{\DW{x}{3}}{below left=-.2em and 1em of rx1a}
  \wk{rx1a}{wx3}
  \wk{rx2a}{wx3}
\end{tikzdisplay}


Across variables, however, cycles in $\gtN$ arise naturally in non-multicopy
atomic examples, such as IRIW.
\[\begin{array}{rl}
  &x\GETS0\SEMI x\GETS 1
  \PAR
  \IF{x}\THEN r\GETS y \FI
 \\{}
  \PAR&
  y\GETS0\SEMI y\GETS 1
  \PAR
  \IF{y}\THEN s\GETS x \FI
\end{array}\]
which includes the execution:
\begin{tikzdisplay}[node distance=1em]
  \event{wx0}{\DW{x}{0}}{}
  \event{wx1}{\DW{x}{1}}{right=of wx0}
  \event{wy0}{\DW{y}{0}}{below=4ex of wx0}
  \event{wy1}{\DW{y}{1}}{right=of wy0}
  \event{ry1}{\DR{y}{1}}{right=2.5em of wy1}
  \event{rx0}{\DR{x}{0}}{right=of ry1}
  \event{rx1}{\DR{x}{1}}{right=2.5 em of wx1}
  \event{ry0}{\DR{y}{0}}{right=of rx1}
  \wk{wx0}{wx1}
  \wk{wy0}{wy1}
  \rf{wx1}{rx1}
  \rf[bend left]{wy0}{ry0}
  \rf{wy1}{ry1}
  \rf[bend right]{wx0}{rx0}
  \wk{rx0}{wx1}
  \wk{ry0}{wy1}
  \po{rx1}{ry0}
  \po{ry1}{rx0}
\end{tikzdisplay}

\subsection{Combinators}

We give the semantics using combinators over sets of pomsets, defined below.
Using $\aPSS$ to range over sets of pomsets, these are:
\begin{itemize}
\item \emph{substitution} $\aPSS\aSub$, which applies the substitution to
  every precondition,
\item \emph{restriction} $\nu\aLoc\st\aPSS$, which internalizes $\aLoc$ for
  pomsets that are $\aLoc$-closed and $\aLoc$-coherent,
% \item \emph{guarding} $\aForm\guard\aPSS$, which filters $\aPSS$,
%   keeping pomsets where all events imply $\aForm$,
% \item \emph{independency filtering} $\Loc\guard\aPSS$, which filters
%   $\aPSS$, keeping pomsets all events are independent of every location,
% \item \emph{write filtering} $\DW{\aLoc}{}\guard\aPSS$, which filters
%   $\aPSS$, keeping pomsets that have an initial write to $\aLoc$,
\item \emph{composition} $\aPSS^1\parallel\aPSS^2$, which unions pomsets, allowing events to be merged, and
\item \emph{prefixing} $\aAct\prefix\aPSS$, which adds an event with action
  $\aAct$ to pomsets in $\aPSS$, ordering $\aAct$ before any $\aEv$ whose predicate
  depends on the value read by $\aAct$.
\end{itemize}
These operations are similar to those from models of concurrency such
as~\cite{Brookes:1984:TCS:828.833}, but adapted here to the setting of
speculative evaluation.

We also define three filtering operations:
\begin{itemize}
\item \emph{guarding} $\aForm\guard\aPSS$, which
  keeps pomsets where all preconditions imply $\aForm$,
\item \emph{independency filtering} $\Loc\guard\aPSS$, which keeps pomsets
  where all preconditions are location independent,
\item \emph{write filtering} $\DW{\aLoc}{}\guard\aPSS$, which keeps pomsets
  that have an initial write to $\aLoc$,
\end{itemize}

%% A write generates a write event that may be visible
%% to other threads.  A read may see a
%% thread-local value, or it may generate a read event that must be justified by
%% another thread.  In the latter case, occurrences of $\aReg$ are replaced with
%% $\aLoc$ (rather than $\aVal$) to ensure that dependencies are tracked
%% properly.  The subsequent substitution of $\aVal$ for $\aLoc$ occurs in
%% Definition~\ref{def:prefix} of prefixing.

% We have completed the formal definition of our model of speculative
% evaluation, and now turn to examples.

%\subsubsection{Substitution and Guarding} 

% Substitution updates the preconditions in a pomset, thus we expect the number
% of pomsets to be unchanged; in addition, the number of events in each of the
% pomsets is unchanged.

% Guarding and restriction filter a set of pomsets; we have
% $(\aForm\guard\aPSS)\subseteq\aPSS$ and
% $(\nu\aLoc\st\aPSS)\subseteq\aPSS$.

The definitions of substitution, restriction and the filtering operations %guarding and write filtering
are straightforward\footnote{We have chosen the definition of restriction for
  its simplicity.  It is worth noting, however, that our definition does not
  support renaming of variables.  In particular
  $(\nu\aLoc\st\aPSS\parallel(\nu\aLoc\st\bPSS))$ is generally not the same
  as $(\nu\aLoc\st\aPSS\parallel(\nu\bLoc\st\bPSS[\bLoc/\aLoc]))$.  To
  support renaming, $(\nu\aLoc\st\aPSS)$ would need to either remove or
  relabel events that mention $\aLoc$.}:
\begin{definition}
  %For a substitution $\aSub$, of the form $[\aLoc/\aReg]$ or $[\bExp/\aLoc]$,
  Let $\aPSS\aSub$ be the set $\aPSS'$ where $\aPS'\in\aPSS'$ whenever
there is $\aPS\in\aPSS$ such that:
$\Event' = \Event$,
${\le'} = {\le}$, 
${\gtN'} = {\gtN}$,
and
$\labeling'(\aEv) = (\bForm\aSub \mid \aAct)$ when $\labeling(\aEv) = (\bForm \mid \aAct)$.
% \begin{itemize}
% \item if $\labeling(\aEv) = (\bForm \mid \aAct)$ then $\labeling'(\aEv) =
%   (\bForm\aSub \mid \aAct)$, and
% \item if $\labeling(\aEv) = (\bForm \mid \aSub)$ then $\labeling'(\aEv) = (\bForm\bSub \mid \aSub\bSub)$.
%\end{itemize}

  % Let $(\nu\aLoc\st\aPSS)$ be the subset of $\aPSS$ such that $\aPS\in\aPSS$ whenever
  % and $\aPS$ is $\aLoc$-coherent and $\aLoc$-closed.

Let $(\nu\aLoc\st\aPSS)$ be the set $\aPSS'$ where $\aPS'\in\aPSS'$ whenever
there is $\aLoc$-coherent and $\aLoc$-closed $\aPS\in\aPSS$ such that:
$\Event' = \Event$,
${\le'} = {\le}$, 
${\gtN'} = {\gtN}$,
and
$\labeling'(\aEv) = (\bForm \mid \finternalize(\aAct))$ when $\labeling(\aEv) = (\bForm \mid \aAct)$.

Let $(\aForm \guard \aPSS)$ be the subset of $\aPSS$ such that $\aPS\in\aPSS$ whenever
$\aForm$ implies $\labelingForm(\aEv)$, for every $\aEv\in\Event$. % if $\labelingAct(\aEv)$ writes.
% \begin{itemize}
% \item if $\labeling(\aEv) = (\bForm \mid \aActSub)$ then $\aForm$ implies $\bForm$.
% \end{itemize}

Let $(\Loc\guard \aPSS)$ be the subset of $\aPSS$ such that
$\aPS\in\aPSS$ whenever $\labelingForm(\aEv)$ is location independent, for every $\aEv\in\Event$.

Let $(\DW{\aLoc}{} \guard \aPSS)$ be the subset of $\aPSS$ such that
$\aPS\in\aPSS$ whenever there is some $\bEv$ that writes $\aLoc$ such that
$\bEv\leq\aEv$, for every release $\aEv\in\Event$.
\end{definition}
% Note that this liberalization allows reads more flexibility.  This is
% desirable in the language and architectural models, but not necessarily in
% microarchitectural models where reads are visible.


% \subsubsection{Restriction}
% \label{sec:restriction}


% We say that $\aPS' = \aPS\restrict{\Event'}$ when 
%  $\Event' \subseteq \Event$,
%  ${\labeling'} = {\labeling}\restrict{\Event'}$, 
%  ${\le'} = {\le}\restrict{\Event'}$, and
%  ${\gtN'} = {\gtN}\restrict{\Event'}$.

% \begin{definition}
%   Let $(\nu\aLoc\st\aPSS)$ be the subset of $\aPSS$ such that $\aPS\in\aPSS$ whenever
%   and $\aPS$ is $\aLoc$-coherent and $\aLoc$-closed.
%   % Let $(\nu\aLoc\st\aPSS)$ be the set $\aPSS'$ where $\aPS'\in\aPSS'$
%   % whenever there is $\aPS\in\aPSS$ such that $\aPS' = \aPS\restrict{\Event'}$
%   % and $\aPS'$ is $\aLoc$-coherent and $\aLoc$-closed.
%   % Let $(\nu\aLoc\st\aPSS)$ be the subset of $\aPSS$ such that $\aPS\in\aPSS$ whenever
%   % \begin{itemize}
%   % \item $\aEv$ is independent of $\aLoc$, and
%   % \item if $\aEv$ reads $\aLoc$, then there is some $\bEv$ such that $\aEv$ can read $\aLoc$ from $\bEv$.
%   % \end{itemize}
% \end{definition}
%This definition throws away useless writes.

%\subsubsection{Composition}
\begin{definition}
Let $\aPS' \in (\aPSS^1 \parallel \aPSS^2)$
whenever there are $\aPS^1 \in \aPSS^1$ and $\aPS^2 \in \aPSS^2$ such that:
\begin{itemize}
\item $\Event' = \Event^1 \cup \Event^2$,
\item ${\le'}\supseteq{\le^1}\cup{\le^2}$, %if $\aEv \le^1 \bEv$ or $\aEv \le^2 \bEv$ then $\aEv \le' \bEv$,
\item ${\gtN'}\supseteq{\gtN^1}\cup{\gtN^2}$, and %if $\aEv \gtN^1 \bEv$ or $\aEv \gtN^2 \bEv$ then $\aEv \gtN' \bEv$,
% \item if $\labeling'(\aEv) = (\aForm' \mid \aAct)$ then either:
%   \begin{itemize}
%   \item $\labeling^1(\aEv) = (\aForm^1 \mid \aAct)$ and $\labeling^2(\aEv) = (\aForm^2 \mid \aAct)$
%     and $\aForm'$ implies $\aForm^1 \lor \aForm^2$,
%   \item $\labeling^1(\aEv) = (\aForm^1 \mid \aAct)$ and $\aEv \not\in \Event^2$
%     and $\aForm'$ implies $\aForm^1$, or
%   \item $\labeling^2(\aEv) = (\aForm^2 \mid \aAct)$ and $\aEv \not\in \Event^1$
%     and $\aForm'$ implies $\aForm^2$.
%   \end{itemize}
\item either
  % \begin{gather*}
  %   \labelingAct'(\aEv) = \labelingAct^1(\aEv) = \labelingAct^2(\aEv) \textand \labelingForm'(\aEv) \textimplies \labelingForm^1(\aEv) \lor \labelingForm^2(\aEv),\\
  %   \aEv \not\in \Event^2,\; \labelingAct'(\aEv) = \labelingAct^1(\aEv) \textand \labelingForm'(\aEv) \textimplies \labelingForm^1(\aEv),\; \textor\\    
  %   \aEv \not\in \Event^1,\; \labelingAct'(\aEv) = \labelingAct^2(\aEv) \textand \labelingForm'(\aEv) \textimplies \labelingForm^2(\aEv).
  % \end{gather*}
  \begin{itemize}
  \item $\labelingAct'(\aEv) = \labelingAct^1(\aEv) = \labelingAct^2(\aEv)
    \textand \labelingForm'(\aEv) \textimplies \labelingForm^1(\aEv) \lor \labelingForm^2(\aEv)$,
  \item $\labelingAct'(\aEv) = \labelingAct^1(\aEv),\;\; \aEv \not\in \Event^2\,
    \textand \labelingForm'(\aEv) \textimplies \labelingForm^1(\aEv),\; \textor$
  \item $\labelingAct'(\aEv) = \labelingAct^2(\aEv),\;\; \aEv \not\in \Event^1\,
    \textand \labelingForm'(\aEv) \textimplies \labelingForm^2(\aEv)$.
  \end{itemize}
\end{itemize}
\end{definition}
Composition is used in giving the semantics for conditionals and concurrency.
$\aPSS^1 \parallel \aPSS^2$ contains the union of pomsets from $\aPSS^1$ and
$\aPSS^2$, allowing overlap as long as they agree on actions. For example, if
$\aPSS^1$ and $\aPSS^2$ contain:
\begin{tikzdisplay}[node distance=1em]
  \event{a}{\aForm \mid \aAct}{}
  \event{b}{\bForm^1 \mid \bAct}{right=of a}
  \po{a}{b}
  \event{b2}{\bForm^2 \mid \bAct}{right=6em of b}
  \event{c2}{\cForm \mid \cAct}{right=of b2}
  \wk{b2}{c2}
\end{tikzdisplay}
then $\aPSS^1 \parallel \aPSS^2$ contains:
\begin{tikzdisplay}[node distance=1em]
  \event{a}{\aForm \mid \aAct}{}
  \event{b}{\bForm^1 \lor \bForm^2 \mid \bAct}{right=of a}
  \event{c}{\cForm \mid \cAct}{right=of b}
  \po{a}{b}
  \wk{b}{c}
\end{tikzdisplay}

% We use $\aPSS^1 \parallel \aPSS^2$ in defining the semantics of conditionals
% and concurrency.
% It contains the union of pomsets from $\aPSS^1$ and $\aPSS^2$,
% allowing overlap as long as they agree on actions. For example, if
% $\aPSS^1$ and $\aPSS^2$ contain:
% \begin{tikzdisplay}[node distance=1em]
%   \event{a}{\aForm \mid \aAct}{}
%   \event{b}{\bForm^1 \mid \bAct}{right=of a}
%   \po{a}{b}
% \end{tikzpicture}\qquad\qquad\begin{tikzpicture}[node distance=1em]
%   \event{b}{\bForm^2 \mid \bAct}{}
%   \event{c}{\cForm \mid \cAct}{right=of b}
%   \wk{b}{c}
% \end{tikzdisplay}
% then $\aPSS^1 \parallel \aPSS^2$ contains:
% \begin{tikzdisplay}[node distance=1em]
%   \event{a}{\aForm \mid \aAct}{}
%   \event{b}{\bForm^1 \lor \bForm^2 \mid \bAct}{right=of a}
%   \event{c}{\cForm \mid \cAct}{right=of b}
%   \po{a}{b}
%   \wk{b}{c}
% \end{tikzdisplay}


%\subsubsection{Prefixing}
\begin{definition}
  \label{def:prefix}
Let $\aAct \prefix \aPSS$ be the set $\aPSS'$ where $\aPS'\in\aPSS'$ whenever
there is $\aPS\in\aPSS$ such that:
\begin{enumerate}
\item\label{pre-E} $\Event' = \Event \cup \{\cEv\}$,
\item\label{pre-le} ${\le'}\supseteq{\le}$, % if $\bEv \le \aEv$ then $\bEv \le' \aEv$,
\item\label{pre-gtN} ${\gtN'}\supseteq{\gtN}$, %if $\aEv \gtN \bEv$ then $\aEv \gtN' \bEv$,
% \item $\labelingAct'(\cEv) = \aAct$, 
% \item if $\labeling(\aEv) = (\bForm \mid \bAct)$ then $\labeling'(\aEv) =
%   (\bForm' \mid \bAct)$, where:
%   \begin{itemize}
%   \item if $\aAct$ is an acquire then $\bForm'$ is independent of every $\bLoc$,
%   \item if $\aAct$ does not read then $\bForm'$ implies $\bForm$,
%   \item if $\aAct$ reads then $\aVal$ from $\aLoc$ then
%     \begin{itemize}
%     \item $\bForm'$ implies $\bForm[\aVal/\aLoc]$, and
%     \item either $\bForm'$ implies $\bForm$ or $\cEv\lt'\aEv$, 
%     \end{itemize}
%   \end{itemize}
% \item if $\labelingAct(\aEv) = \bAct$ then:
%   \begin{itemize}
%   \item if $\aAct$ is an acquire or $\bAct$ is a release then $\cEv \lt' \aEv$, 
%   \item if $\aAct$ and $\bAct$ both touch the same location and one is a write,
%     then $\cEv \gtN' \aEv$, and
%   \end{itemize}
\item\label{pre-act} $\labelingAct'(\cEv) = \aAct$ and $\labelingAct'(\aEv) = \labelingAct(\aEv)$,
\item\label{pre-implies} either $\labelingForm'(\aEv)$ implies $\labelingForm(\aEv)$ or
  $\aAct$ is a read and if $\aEv$ is a write then $\cEv\lt'\aEv$,
% \item if $\aAct$ does not read then $\labelingForm'(\aEv)$ implies $\labelingForm(\aEv)$,
% \item if $\aAct$ reads then either $\labelingForm'(\aEv)$ implies $\labelingForm(\aEv)$ or $\cEv\lt'\aEv$,  %
\item\label{pre-read} if $\aAct$ reads $\aVal$ from $\aLoc$ then
   $\labelingForm'(\aEv)$ implies $\labelingForm(\aEv)[\aVal/\aLoc]$, %
\item\label{pre-coherence} if $\aAct$ conflicts with $\labelingAct(\aEv)$ 
    then $\cEv \gtN' \aEv$,
\item\label{pre-sync} if $\aAct$ is an acquire or $\labelingAct(\aEv)$ is a release then $\cEv \lt' \aEv$, and
\item\label{pre-acquire} if $\aAct$ is an acquire then $\labelingForm(\aEv)$ is location independent.
% \item if $\aAct$ is a read but not a synchronization then either
%   $\labelingForm'(\cEv)$ is unsatisfiable or there is some $\aEv$ such
%   that $\labelingForm'(\aEv)$ does not imply $\labelingForm(\aEv)$.
% \item if $\labeling(\aEv) = (\bForm \mid \bAct)$ then $\labeling'(\aEv) =
%   (\bForm' \mid \bAct)$, where:
%   \begin{itemize}
%   \item if $\aAct$ is an acquire or $\bAct$ is a release then $\cEv \lt' \aEv$, 
%   \item if $\aAct$ is an acquire then $\bForm$ is independent of every $\bLoc$,
%   \item if $\aAct$ and $\bAct$ both touch the same location and one is a write,
%     then $\cEv \gtN' \aEv$, and
%   \item $\bForm'$ implies \(\left\{\begin{array}{l@{~}ll}
%     % \bForm[\aVal/\aLoc]                     & \mbox{if $\aAct$ reads $\aVal$ from $\aLoc$ and $\cEv\lt'\aEv$} & \textsc{[dependent read]} \\
%     % \bForm[\aVal/\aLoc] \text{ and } \bForm & \mbox{if $\aAct$ reads $\aVal$ from $\aLoc$}                  & \textsc{[independent read]} \\
%     % \bForm                                  & \mbox{otherwise}                                              & \textsc{[non-read]} \\        
%     \bForm[\aVal/\aLoc]                     \\\quad \mbox{if $\aAct$ reads $\aVal$ from $\aLoc$ and $\cEv\lt'\aEv$} \\\qquad \textsc{[dependent read]} \\[\jot]
%     \bForm[\aVal/\aLoc] \text{ and } \bForm \\\quad \mbox{if $\aAct$ reads $\aVal$ from $\aLoc$}                  \\\qquad \textsc{[independent read]} \\[\jot]
%     \bForm                                  \\\quad \mbox{otherwise}                                              \\\qquad \textsc{[non-read]} \\
%   \end{array}\right.\)
%   \end{itemize}
\end{enumerate}
\end{definition}
% The last condition ensures that useless reads are not included.
% Otherwise, $\labelingForm'(\cEv)$ is unconstrained.

% In order to keep augmentation closure, we need to keep the unsatisfiable
% elements in the set of pomsets.

Item~\ref{pre-implies} allows one to choose a weaker precondition for $\aEv$
in $\aPS$ (right side of the or) if $\aAct$ is a read.  In this case,
item~\ref{pre-read} ensures that the precondition is not \emph{too} weak.  In
the case that one chooses the weaker precondition, item~\ref{pre-implies}
requires that there be an order from the new event to $\aEv$ when $\aEv$ is a
write.  No additional order is required when $\aEv$ is a pure read or an
internal action\footnote{The absence of order between reads is required for
  our compilation result, since the IMM does not enforce order between reads
  of a single thread.}.


$\aAct\prefix\aPSS$ adds a new event $\cEv$ with action $\aAct$ to each
pomset in $\aPSS$.  As in the definition of parallel composition, the
definition allows the new event to overlap with events in $\aPSS$ as long as
they agree on the action.  Overlapping of synchronization events is
disallowed by item~\ref{pre-sync}.

If $\cEv$ writes to a location that is also written by some $\aEv$ in $\aPSS$,
item~\ref{pre-coherence} introduces weak order between them: $\cEv \gtN \aEv$.  This
ensures that these writes cannot be given the reverse order in an augmentation.

If $\cEv$ reads from a location that occurs in the predicate of $\aEv$, then
prefixing introduces order from $\cEv$ to $\aEv$.
whose predicate depends on $\aLoc$. 
For example, if $\aPSS$ contains %a pomset with only
\begin{tikzinline}[node distance=1em]
  \event{b}{y=1 \mid \DW{x}{1}}{}
  \event{c}{x>0 \mid \DW{z}{1}}{right=of b}
\end{tikzinline}
then $(\DR{x}{1})\prefix\aPSS$ contains:
\begin{displaymathsmall}
\begin{tikzcenter}[node distance=1em]
  \event{a}{\DR{x}{1}}{}
  \event{b}{y=1 \mid \DW{x}{1}}{above right=0em and 2em of a}
  \event{c}{1>0 \mid \DW{z}{1}}{below right=0em and 2em of a}
  \po{a}{c}
  \wk{a}{b}
\end{tikzcenter}
\qquad\text{and}\qquad
\begin{tikzcenter}[node distance=1em]
  \event{a2}{\DR{x}{1}}{right=4em of a}
  \event{b2}{y=1 \mid \DW{x}{1}}{above right=0em and 2em of a2}
  \event{c2}{x>0 \mid \DW{z}{1}}{below right=0em and 2em of a2}
  \wk{a2}{b2}
\end{tikzcenter}
\end{displaymathsmall}
In order to weaken the predicate on $(\DW{z}{1})$, item~\ref{pre-implies}
requires that we include the order from $(\DR{x}{1})$ to $(\DW{z}{1})$.
If the precondition on $(\DW{z}{1})$ in $\aPSS$ was $x<0$, then, by
item~\ref{pre-read}, all preconditions for $(\DW{z}{1})$ in
$(\DR{x}{1})\prefix\aPSS$ must be equivalent to $\FALSE$, regardless of
the ordering of the events.

% For example, if $\aAct$ and $\bAct$ write to the same location, $\aAct$ reads
% $\aVal$ from $\aLoc$, $\bForm$ is independent of $\aLoc$, and $\aPSS$
% contains:
% $\footnotesize\begin{tikzpicture}[baselinecenter,node distance=1em]
%   \event{b}{\bForm \mid \bAct}{}
%   \event{c}{\cForm \mid \cAct}{right=of b}
% \end{tikzpicture}$
% then $\aAct\prefix\aPSS$ contains:
% \begin{tikzdisplay}[node distance=1em]
%   \event{a}{\aForm \mid \aAct}{}
%   \event{b}{\bForm \mid \bAct}{right=of a}
%   \event{c}{\cForm[\vec\aVal/\vec\aLoc] \mid \cAct}{right=of b}
%   \po[out=25,in=155]{a}{c}
%   \wk{a}{b}
%   \po{b}{c}
% \end{tikzdisplay}

% We say $\aEv$ \emph{depends on} $\cEv$ if
% $\labeling(\aEv) = (\bForm \mid \dontcare)$,
% $\labeling(\cEv) = (\dontcare \mid \aSub)$,
% and $\bForm$ depends on $\aSub$.

% We say $\aEv$ \emph{conflicts with}  $\bEv$ if
% $\labeling(\aEv) = (\dontcare \mid \aAct)$,
% $\labeling(\cEv) = (\dontcare \mid \bAct)$,
% $\aAct$ and $\bAct$ touch the same location, and either
% $\aAct$ or $\bAct$ is a write.

Item~\ref{pre-acquire} ensures that thread-local reads do
not cross acquire fences.  This prevents bad executions like the following:
\begin{verbatim}
   x=1; rel; acq; if (x) {y=1};  ||  acq; x=0; rel; 
\end{verbatim}
where the second thread is interleaved between the rel and acq of the first.
In item~\ref{pre-acquire}, we do not require that $\bForm'$ is independent of
every $\bLoc$; were we to require this, the definition would not be augment closed.

Item~\ref{pre-sync} ensures that events are ordered before a release and
after an acquire.  As an example, consider the program:
\[
  x\GETS0\SEMI f\GETS0\SEMI x\GETS 1\SEMI f \REL\GETS1 \PAR r\GETS f\ACQ; s\GETS x
\]
This has a top-level execution:
\begin{tikzdisplay}[node distance=1em]
  \event{wx0}{\DW{x}{0}}{}
  \event{wf0}{\DW{f}{0}}{below=of wx0}
  \event{wx1}{\DW{x}{1}}{right=of wx0}
  \event{wf1}{\DWRel{f}{1}}{right=of wf0}
  \event{rf1}{\DRAcq{f}{1}}{right=2.5em of wf1}
  \event{rx1}{\DR{x}{1}}{above=of rf1}
  \po{wx0}{wf1}
  \po{wf0}{wf1}
  \po{wx1}{wf1}
  \po{rf1}{rx1}
  \rf{wf1}{rf1}
  \rf{wx1}{rx1}
  \wk{wx0}{wx1}
\end{tikzdisplay}
but \emph{not}:
\begin{tikzdisplay}[node distance=1em]
  \event{wx0}{\DW{x}{0}}{}
  \event{wf0}{\DW{f}{0}}{below=of wx0}
  \event{wx1}{\DW{x}{1}}{right=of wx0}
  \event{wf1}{\DWRel{f}{1}}{right=of wf0}
  \event{rf1}{\DRAcq{f}{1}}{right=2.5em of wf1}
  \event{rx0}{\DR{x}{0}}{above=of rf1}
  \po{wx0}{wf1}
  \po{wf0}{wf1}
  \po{wx1}{wf1}
  \po{rf1}{rx1}
  \rf{wf1}{rf1}
  \rf[bend left]{wx0}{rx0}
  \wk{wx0}{wx1}
\end{tikzdisplay}
since $(\DW x0) \gtN (\DW x1) \lt (\DR x0)$, so this pomset does not satisfy the
requirements to be $x$-closed.
If we replace the release
with a plain write, then the outcome $(\DRAcq f1)$ and $(\DR x0)$ is possible:
\begin{tikzdisplay}[node distance=1em]
  \event{wx0}{\DW{x}{0}}{}
  \event{wf0}{\DW{f}{0}}{below=of wx0}
  \event{wx1}{\DW{x}{1}}{right=of wx0}
  \event{wf1}{\DW{f}{1}}{right=of wf0}
  \event{rf1}{\DRAcq{f}{1}}{right=2.5em of wf1}
  \event{rx0}{\DR{x}{0}}{above=of rf1}
  \wk{wf0}{wf1}
  \po{rf1}{rx0}
  \rf{wf1}{rf1}
  \rf[bend left]{wx0}{rx0}
  \wk{wx0}{wx1}
\end{tikzdisplay}
since no order is required between $(\DW x1)$ and $(\DW f1)$.  
Symmetrically, if we replace the acquire of the original program
with a plain read, then the outcome $(\DR f1)$ and $(\DR x0)$ is possible.

We end this section with the following lemma, which is immediate from the definitions.
\begin{lemma}
  \label{lem:monotone}
  All combinators are monotone with respect to subset order.  For example, 
  $\aAct\prefix\aPSS \subseteq \aAct\prefix\aPSS'$ whenever
  $\aPSS\subseteq\aPSS'$.
%   Suppose
%   $\aPSS'\supseteq\aPSS$, $\aPSS_1'\supseteq\aPSS_1$ and
%   $\aPSS_2'\supseteq\aPSS_2$.  Then we have the following.
% \begin{itemize}
% \item $\aPSS'\aSub \supseteq \aPSS\aSub$,
% \item $\nu\aLoc\st\aPSS' \supseteq \nu\aLoc\st\aPSS$,
% \item $\aForm\guard\aPSS' \supseteq \aForm\guard\aPSS$,
% \item $\Loc\guard\aPSS' \supseteq \Loc\guard\aPSS$,
% \item $\DW{\aLoc}{}\guard\aPSS' \supseteq \DW{\aLoc}{}\guard\aPSS$,
% \item $\aPSS'_1\parallel\aPSS'_2 \supseteq \aPSS_1\parallel\aPSS_2$, and
% \item $\aAct\prefix\aPSS' \supseteq \aAct\prefix\aPSS$.
% \end{itemize}
\end{lemma}

\subsection{Semantics of programs}
\label{sec:semantics}

We consider a simple shared-memory concurrent language, with statements
defined as follows.

\begin{comment}
\footnote{We only consider executions where register state is empty in
  forked threads.  Given item~\ref{pre-acquire} of
  Definition~\ref{def:prefix}, a sufficient condition is that parallel
  composition is always preceded by an acquire fence, as in programs of the
  form:
  \begin{displaymath}
    \VAR\vec{\aLoc}\SEMI
    \vec{\aLoc}\GETS\vec{0}\SEMI
    \vec{\bLoc}\GETS\vec{0}\SEMI
    \FENCE\SEMI
    (\aCmd^1 \PAR \cdots \PAR \aCmd^n)
  \end{displaymath}
  where $\aCmd^1$, \ldots, $\aCmd^n$ do not include $\PAR$.  To avoid clutter
  in drawings, we often drop the explicit fence.}.
\end{comment}

% \begin{align*}
% \aCmd,\,\bCmd
% \BNFDEF& \SKIP \tag{No Operation}
% \\[-1ex]\BNFSEP& \FENCE\SEMI \aCmd \tag{Full fence}
% \\[-1ex]\BNFSEP& \REF{\cExp}\GETS\aExp\SEMI \aCmd \tag{Relaxed write to memory}
% \\[-1ex]\BNFSEP& \REF{\cExp}\REL\GETS\aExp\SEMI \aCmd \tag{Releasing write to memory}
% \\[-1ex]\BNFSEP& \aReg\GETS\REF{\cExp}\SEMI \aCmd \tag{Relaxed read from memory}
% \\[-1ex]\BNFSEP& \aReg\GETS\REF{\cExp}\ACQ\SEMI \aCmd
% \\[-1ex]\BNFSEP& \IF{\aExp} \THEN \aCmd \ELSE \bCmd \FI
% \\[-1ex]\BNFSEP& \aCmd \PAR \bCmd
% \\[-1ex]\BNFSEP& \VAR\aLoc\SEMI \aCmd
% \end{align*}
\begin{align*}
\aCmd,\,\bCmd
\BNFDEF& \SKIP
%\BNFSEP \FENCE\SEMI \aCmd
\BNFSEP \aReg\GETS\aExp\SEMI \aCmd
\BNFSEP \aReg\GETS\REF{\cExp}\ACQ\SEMI \aCmd 
\BNFSEP \aReg\GETS\REF{\cExp}\SEMI \aCmd
\BNFSEP \REF{\cExp}\REL\GETS\aExp\SEMI \aCmd
\BNFSEP \REF{\cExp}\GETS\aExp\SEMI \aCmd
\\
\BNFSEP& \IF{\aExp} \THEN \aCmd \ELSE \bCmd \FI
\BNFSEP \aCmd \PAR \bCmd
\BNFSEP \VAR\aLoc\SEMI \aCmd
\end{align*}
We use common syntax sugar, such as \emph{extended expressions}, which include
memory locations.  For example, if the extended expression $\aEExp$ includes
a single occurrence of $\aLoc$, then $\bLoc\GETS\aEExp\SEMI \aCmd$ is
shorthand for $\aReg\GETS\aLoc\SEMI\bLoc\GETS\aEExp[\aReg/\aLoc]\SEMI \aCmd$.
Each occurrence of $\aLoc$ in an extended expression corresponds to an
independent read.  We also write
$\IF{\aExp} \THEN \aCmd^1 \ELSE \aCmd^2 \FI \SEMI \bCmd$ as shorthand for
$\IF{\aExp} \THEN \aCmd^1 \SEMI \bCmd\ELSE \aCmd^2 \SEMI \bCmd\FI$ and
$\VAR\aLoc\GETS\aExp\SEMI \aCmd$ as shorthand for
$\VAR\aLoc\SEMI\aLoc\GETS\aExp\SEMI \aCmd$.
% We write $\REL\aLoc\GETS\aExp\SEMI \aCmd$ as shorthand for $\FENCE_{\FR} \SEMI\aLoc\GETS\aExp\SEMI \aCmd$.
% We write $\ACQ\aReg\GETS\aLoc\SEMI \aCmd$ as shorthand for $\aReg\GETS\aLoc\SEMI \FENCE_{\FA}\SEMI \aCmd$.

% In Figure~\ref{fig:programs}, we give the semantics as sets of pomsets.  
% \begin{figure*}
The semantics of programs is as follows:
\allowdisplaybreaks
\begin{align*}
  \sem{\SKIP} & =
  \{ \emptyset \}
  \\
  % \sem{\FENCE\SEMI \aCmd} & =
  % (\DF{}) \prefix \sem{\aCmd}
  % \\
  \sem{\aReg\GETS\aExp\SEMI \aCmd} & =
  \sem{\aCmd}[\aExp/\aReg] 
  \\  
  \sem{\aReg\GETS\REFAcq{\cExp}\SEMI \aCmd} & =
  \textstyle\bigcup_\aLoc\; (\REF{\cExp}=\aLoc) \guard \textstyle\bigcup_\aVal\; (\DRAcq\aLoc\aVal) \prefix \sem{\aCmd}[\aLoc/\aReg] 
  \\
  \sem{\aReg\GETS\REF{\cExp}\SEMI \aCmd} & =
  \textstyle\bigcup_\aLoc\; (\REF{\cExp}=\aLoc) \guard \textstyle\bigcup_\aVal\; (\DR\aLoc\aVal) \prefix \sem{\aCmd}[\aLoc/\aReg] 
  \\ & \mkern2mu\cup \textstyle\bigcup_\aLoc\; (\REF{\cExp}=\aLoc) \guard \sem{\aCmd}[\aLoc/\aReg]
  \\
  \sem{\REFRel{\cExp}\GETS\aExp\SEMI \aCmd} & =
  \textstyle\bigcup_\aLoc\; (\REF{\cExp}=\aLoc) \guard \textstyle\bigcup_\aVal\;  (\aExp=\aVal) \guard\bigl((\DWRel\aLoc\aVal) \prefix \sem{\aCmd}\bigr)[\aExp/\aLoc] 
  \\
  \sem{\REF{\cExp}\GETS\aExp\SEMI \aCmd} & =
  \textstyle\bigcup_\aLoc\; (\REF{\cExp}=\aLoc) \guard \textstyle\bigcup_\aVal\;  (\aExp=\aVal) \guard\bigl((\DW\aLoc\aVal) \prefix \sem{\aCmd}\bigr)[\aExp/\aLoc]
  \\ & \mkern2mu\cup \textstyle\bigcup_\aLoc\; (\REF{\cExp}=\aLoc) \guard \DW{\aLoc}{} \guard \sem{\aCmd}[\aExp/\aLoc]
  \\
  \sem{\IF{\aExp} \THEN \aCmd \ELSE \bCmd \FI} & =
  \bigl((\aExp \neq 0) \guard \sem{\aCmd}\bigr) \parallel \bigl((\aExp=0) \guard \sem{\bCmd}\bigr) 
  \\
  \sem{\aCmd \PAR \bCmd} & =
  \sem{\aCmd} \parallel \Loc \guard \sem{\bCmd} 
  \\
  \sem{\VAR\aLoc\SEMI \aCmd} & =
  \nu \aLoc \st \sem{\aCmd}
  % \sem{\aLoc\GETS\aExp\SEMI \aCmd} & = \textstyle\bigcup_\aVal\; \bigl((\aExp=\aVal) \guard (\DW\aLoc\aVal) \prefix \sem{\aCmd}\bigr)[\aExp/\aLoc] \\
  % \sem{\REL\aLoc\GETS\aExp\SEMI \aCmd} & = \textstyle\bigcup_\aVal\; \bigl((\aExp=\aVal) \guard (\DWRel\aLoc\aVal) \prefix \sem{\aCmd}\bigr)[\aExp/\aLoc] \\
  % \sem{\aReg\GETS\REF{\cExp}\SEMI \aCmd} & = \textstyle\bigcup_\aLoc\; (\REF{\cExp}=\aLoc) \guard (\sem{\aCmd}[\aLoc/\aReg] \cup \textstyle\bigcup_\aVal\; (\DR\aLoc\aVal) \prefix \sem{\aCmd}[\aLoc/\aReg]) \\
  % \sem{\aReg\GETS\aLoc\SEMI \aCmd} & =  \sem{\aCmd}[\aLoc/\aReg] \cup \textstyle\bigcup_\aVal\; (\DR\aLoc\aVal) \prefix \sem{\aCmd}[\aLoc/\aReg] \\
  % \sem{\ACQ\aReg\GETS\aLoc\SEMI \aCmd} & =  \textstyle\bigcup_\aVal\; (\DRAcq\aLoc\aVal) \prefix \sem{\aCmd}[\aLoc/\aReg] \\
% \caption{Semantics of a concurrent shared-memory language}
% \label{fig:programs}
% \end{figure*}
\end{align*}

The semantics of relaxed writes is the union of two sets.  The first set adds
a write action to each pomset in $\sem{\aCmd}$; the second does not.  In
discussion we refer to these respectively as \emph{explicit} and
\emph{implicit} writes.  The semantics of relaxed reads is similar.  We
collectively refer to these as explicit and implicit \emph{actions}.

Note that synchronizing actions are always explicit.

Note that the rule for write uses the substitution $[\aExp/\aLoc]$ with
precondition $\aExp=\aVal$, rather than using $[\aVal/\aLoc]$ directly.
To see the need for this, consider
$\sem{\IF{\bReg\EQ\aReg}\THEN \cLoc\GETS 1\FI}$,
which includes
\begin{tikzinline}[node distance=1em]
  \event{c}{\bReg=\aReg \mid \DW\cLoc1}{}
\end{tikzinline}.
Therefore
$\sem{\bReg\GETS\aLoc\SEMI\IF{\bReg\EQ\aReg}\THEN \cLoc\GETS 1\FI}$
includes
\begin{tikzinline}[node distance=1em]
  \event{c}{\aLoc=\aReg \mid \DW\cLoc1}{}
\end{tikzinline}
and
$\sem{\aLoc\GETS\aReg\SEMI\bReg\GETS\aLoc\SEMI\IF{\bReg\EQ\aReg}\THEN \cLoc\GETS 1\FI}$
includes
\begin{tikzinline}[node distance=1em]
  \event{c}{\aReg=\aReg \mid \DW\cLoc1}{}
\end{tikzinline}
which is independent of $\aReg$.
%
If we took the semantics of write to use $[\aVal/\aLoc]$, then we would end
up with pomsets of the form
\begin{tikzinline}[node distance=1em]
  \event{c}{\aVal=\aReg \mid \DW\cLoc1}{}
\end{tikzinline}
which depend on $\aReg$.

It is worth emphasizing that prefixing does not necessarily induce a
dependency, even for read actions where the read is used.  To see that this
is desirable, consider  the semantics of
$\bLoc\GETS0\SEMI\aReg\GETS\bLoc\SEMI\IF{\aReg\leq1}\THEN \aLoc\GETS 2\FI$.
To begin, not that 
$\sem{\IF{\aReg\leq1}\THEN \aLoc\GETS 2\FI}$ includes
\begin{tikzinline}[node distance=1em]
  \event{c}{\aReg\leq1 \mid \DW\aLoc2}{}
\end{tikzinline}
which depends on $\aReg$.
Then $\sem{\aReg\GETS\bLoc\SEMI\IF{\aReg\leq1}\THEN \aLoc\GETS 2\FI}$ includes
\begin{tikzdisplay}[node distance=1em]
    \event{b}{\DR\bLoc1}{}
    \event{c}{\bLoc\leq1 \mid \DW\aLoc2}{right=of b}
\end{tikzdisplay}
which has no order between the read and write.
By prefixing a write to $\bLoc$, $\sem{\bLoc\GETS0\SEMI\aReg\GETS\bLoc\SEMI\IF{\aReg\leq1}\THEN \aLoc\GETS
  2\FI}$ discharges the precondition of the write to $\aLoc$, giving
\begin{tikzinline}[node distance=1em]
    \event{a}{\DW\bLoc0}{}
    \event{b}{\DR\bLoc1}{right=of a}
    \event{c}{0\leq1 \mid \DW\aLoc2}{right=of b}
\end{tikzinline}
which is simply:
\begin{tikzdisplay}[node distance=1em]
    \event{a}{\DW\bLoc0}{}
    \event{b}{\DR\bLoc1}{right=of a}
    \event{c}{\DW\aLoc2}{right=of b}
\end{tikzdisplay}
Here the thread-local value of $\bLoc$ discharges the predicate.
Using the left-hand side of the read rule, the semantics of this program also includes
\begin{tikzinline}[node distance=1em]
    \event{a}{\DW\bLoc0}{}
    \event{c}{\DW\aLoc2}{right=of a}
\end{tikzinline}.

A variant which indicates the branch taken:
$\sem{\bLoc\GETS0\SEMI\aReg\GETS\bLoc\SEMI\IF{\aReg\leq1}\THEN
  \aLoc\GETS2\SEMI\cLoc\GETS\aReg\FI}$
includes
\begin{tikzdisplay}[node distance=1em]
    \event{a}{\DW\bLoc0}{}
    \event{b}{\DR\bLoc1}{right=of a}
    \event{c}{\DW\aLoc2}{right=of b}
    \event{d}{\DW\cLoc1}{right=of c}
    \po[bend left]{b}{d}
\end{tikzdisplay}
A program to that witnesses the independence of $\DR\bLoc1$ and $\DW\aLoc2$ is
\begin{math}
  \IF{\bLoc\EQ0}\THEN
    \IF{\aLoc\EQ2}\THEN
      \bLoc\GETS1\SEMI
      \IF{\cLoc\EQ1}\THEN\PASS\FI
    \FI
  \FI
\end{math}.
Putting these in parallel gives you:
\begin{tikzdisplay}[node distance=1em]
    \event{a}{\DW\bLoc0}{}
    \event{b}{\DR\bLoc1}{right=of a}
    \event{c}{\DW\aLoc2}{right=of b}
    \event{d}{\DW\cLoc1}{right=of c}
    \po[bend left]{b}{d}
    \event{a2}{\DR\bLoc0}{below=of a}
    \event{b2}{\DR\aLoc2}{right=of a2}
    \event{c2}{\DW\bLoc1}{right=of b2}
    \event{d2}{\DR\cLoc1}{right=of c2}
    \po{a2}{b2}
    \po{b2}{c2}
    \po[bend right]{b2}{d2}
    \rf{a}{a2}
    \rf{c}{b2}
    \rf{c2}{b}
    \rf{d}{d2}
\end{tikzdisplay}

% Local Variables:
% mode: latex
% TeX-master: "paper"
% End:

\section{Data Race Free Behaviors are Sequentially Consistent}
\label{sec:sc}

% For any $\aPSS$, then $\closed(\aPSS)$ is set enriched with useless reads
% (preserving augmentation closure) and where we remove any event whose
% precondition is not a tautology.
% \begin{definition}
%   Let $\addRead(\aPSS)$ be the set $\aPSS'$ where $\aPS'\in\aPSS'$ whenever
%   there is $\aPS\in\aPSS$ such that:
%   $\Event' = \Event\cup{\cEv}$,
%   ${\le'} \supseteq {\le}$, 
%   ${\gtN'} \supseteq {\gtN}$,
%   and
%   $\labelingAct'(\cEv) = (\DR{\aLoc}{\aVal})$ and $\labelingAct'(\aEv) = \labelingAct(\aEv)$,
% \end{definition}
% Then $\fclosed(\aPSS)$ 



In this section, we prove the SC-DRF theorem, which states that any program
that lacks data races under the SC semantics must only have executions that
are compatible with SC executions.  We present the result for programs of the
form $\vec{\aLoc}\GETS\vec{0}\SEMI\FENCE\SEMI\aCmd$, where $\aCmd$ is
restriction-free.  Thus, all memory locations are initialized to $0$,
initialization happens-before the execution of any command, and internal actions only arise from (intra-thread) implicit reads.

% We say that two actions have a \emph{data-race conflict} if at least one
% action is a write and the other is a write, read, or internal read to the
% same location.
Define the relation $\reco$ so that $(\aEv,\bEv)\in{\reco}$
if $\aEv\gtN\bEv$ and $\aEv$ and $\bEv$ conflict.

The program $x\GETS1\PAR x\GETS2$ is considered to have an SC data race, but
$x\GETS1\SEMI x\GETS2$ does not.  In our semantics the only difference
between these is that $x\GETS1\SEMI x\GETS2$ enforces weak order between the
writes.  Note also that the
$\sem{x\GETS1\SEMI a\GETS y}=\sem{a\GETS y\SEMI x\GETS1}$, yet these two must
be distinguished in SC, as per the load-buffering and store-buffering litmus tests.

In order to define SC executions and SC data races, it is necessary to
augment our semantics to record program order.  We extend the definitions in
\textsection\ref{sec:semantics} with
${\rpox}\subseteq{\Event}\times{\Event}$, defined as follows:
\begin{itemize}
\item
  ${\rpox'} = {\rpox}$
  when $\aPSS'=\aPSS\aSub$
  or $\aPSS'=\aForm\guard\aPSS$
\item
  ${\rpox'} = {\rpox}\restrict{\Event'}$
  when $\aPSS'=\nu\aLoc\st\aPSS$
\item
  ${\rpox'} = {\rpox}^1\cup{\rpox}^2$
  when $\aPSS'=\aPSS^1\parallel\aPSS^2$
\item
  ${\rpox'} = {\rpox}\cup\{(\cEv,\aEv)\mid\aEv\in\Event\}$
  when $\aPSS'=\aAct\prefix\aPSS$ and $\Event' = \Event \cup \{\cEv\}$
\end{itemize}

Define the relation ${\rrfx}$ so that $(\aEv,\bEv)\in{\rrfx}$ if $\aEv$
writes $\aLoc$, $\bEv$ reads $\aLoc$, and for any $\cEv$ that writes $\aLoc$
either $\cEv\gtN\aEv$ or $\bEv\gtN\cEv$.  Let $\IDAcq$ be the identity
relation on acquire events, and likewise $\IDRel$ on release events.  Now
define $\rsw$ and $\rhb$\footnote{For simplicity, the definition of $\rsw$
  does not include release sequences or fences.  For example, we consider
  \begin{math}
    (\DWRel{x}{1})\prefix(\DW{x}{2}) \parallel (\DRAcq{x}{2})
  \end{math}
  to be racy, whereas this pomset is race-free using release sequences.  If
  we include release sequences and fences, then $\rhb$ relates more events
  and thus there are fewer races. Our results hold under either definition.}.
\begin{align*}
  {\rsw} &= \IDRel; ({\rrfx}\setminus{\rpox}); \IDAcq
  \\
  {\rhb} &= ({\rpox} \cup {\rsw})^+
\end{align*}
Note that our semantics guarantees that ${\rsw}\subseteq{\lt}$.

A pomset has a \emph{data race} if there are events $\aEv$ and $\bEv$ such
that
\begin{itemize}
\item $\aEv$ and $\bEv$ are unordered by $\rhb$,
\item $\labelingForm(\aEv)$ and $\labelingForm(\bEv)$ are tautologies, and
\item $\labelingAct(\aEv)$ and $\labelingAct(\bEv)$ conflict.
\end{itemize}

%The semantics of programs includes SC executions.
\begin{definition}
  Let $\semsc{\aCmd}$ be the subset of $\sem{\aCmd}$ such that
  $\aPS\in\semsc{\aCmd}$ whenever $\aPS$ is a top-level pomset and ${\lt}\cup{\rpox}$ is acyclic.
% ${\lt}\cup{\rpox} \cup {\reco}$ is acyclic.
\end{definition}
%\begin{itemize}
%\item ${\le}\cup{\rpox} \cup {\reco}$ is acyclic ,
% \item prefixing ($\prefix$) and composition ($\parallel$) take disjoint union, and
%\item all reads are denoted by explicit actions.
%\end{itemize}
We argue that this definition is sufficient to capture sequential
consistency: Any total order that linearizes the acyclic relation is
consistent with strong order ($\lt$) and the program order ($\rpox$).  Since $\le$ contains $\reco$, only the last write to a location is
read in such a total order.

%\begin{remark}\label{generator}
% In the rest of this section, we consider ``top-level'' programs of the form
% $\aCmd = \VAR\vec{\aLoc}\SEMI
%     \vec{\aLoc}\GETS\vec{0}\SEMI
%     \vec{\bLoc}\GETS\vec{0}\SEMI
%     \FENCE\SEMI
%     (\aCmd_1 \PAR \cdots \PAR \aCmd_n)
% $.
% We will consider closed and complete executions of $\aCmd$. We use $\semClosed{\aCmd}$ to stand for the subset of $\sem{\aCmd}$  with only  the pomsets that are $\vec{\aLoc}, \vec{\bLoc}$ closed.  
We only consider \emph{generators}, which are top-level pomsets that are
minimal with respect to augmentation and implication.  Since we are
considering finite programs without loops, the pomsets in the semantics of
threads are finite.  Thus, there are no infinite descending chains of
augmentations.


%\end{remark}

We prove the following theorem in \textsection\ref{drfproof}.
\begin{theorem}
  Let $\aPS$ be a generator for $\aCmd$.
  \begin{itemize}
  \item If $\aPS$ does not have a data race, $\aPS \in \semsc{\aCmd}$.
  \item If $\aPS$ has a data race, then there exists
    $\aPS'\in \semsc{\aCmd}$ that has a data race.
  \end{itemize}
\end{theorem}
A key step of this proof is an analysis of the closure properties of the semantics.  In order to perform this fine grained analysis of dependency, we describe a variant of the semantics using modal pomsets defined below.  
\begin{definition}
  \label{def:tvalpom}
  A \emph{\tvalpom} is a tuple
  $(\Event, {\sle}, {\gtN},
  \labeling)$, such that
  \begin{itemize}
   \item $(\Event, {\gtN},
  \labeling)$ is a pomset, and 
\item ${\sle} \subseteq {\gtN}$ is a partial order. 
  \end{itemize}
\end{definition}
These are the pomset equivalent of the \emph{modal transition systems} of~\citet{DBLP:conf/lics/LarsenT88},
\citet{DBLP:conf/esop/HuthJS01} call $\slt$ a ``must transition''
and $\geN$ a ``may transition''.  In anticipation, we have used  the terms ``strong order'' and ``weak order'' respectively, drawing $\slt$ as a solid arrow ``$\xpo$'' and $\geN$ dashed ``$\xwk$' in the pictures following Lamport's [\citeyear{DBLP:journals/dc/Lamport86}] notation.  The intuitive temporal meaning of $ \aEv \slt \bEv$ is that $\aEv$ {\em must} strictly precede $\bEv$, whereas $ \aEv \geN \bEv$ is intended to connote that $\aEv$ may precede $\bEv$.
 
The semantics of programs in the modal model proceeds as before, mutatis mutandis, with details described in the appendix.  This finer analysis of necessary and possible dependency allows us to establish the existence of pomsets in the semantics as we search for sequential witnesses to data races.    We provide some illustrative examples below.


For the program
\begin{math}
(y\GETS 0 \SEMI \aReg \GETS y  \SEMI x \GETS 1) \PAR
(x\GETS 0 \SEMI \bReg \GETS x \SEMI y \GETS 1):
\end{math}
\begin{displaymathsmall}
\mbox{From    }
\qquad\qquad
\begin{tikzcenter}[node distance=1em]
\event{wy0}{\DW{y}{0}}{}
\event{ry1}{\DR{y}{1}}{right=of wy0}
\event{wx1}{\DW{x}{1}}{right=of ry1}
\event{wx0}{\DW{x}{0}}{below=of wy0}
\event{rx1}{\DR{x}{1}}{right=of wx0}
\event{wy1}{\DW{y}{1}}{right=of rx1}
\rf{wx1}{rx1}
\rf{wy1}{ry1}
\wk{wx0}{rx1}
\wk{wy0}{ry1}
\end{tikzcenter}
\qquad
\mbox{  infer  }
\qquad
\begin{tikzcenter}[node distance=1em]
\event{wy0}{\DW{y}{0}}{}
\event{ry1}{\DR{y}{0}}{right=of wy0}
\event{wx1}{\DW{x}{1}}{right=of ry1}
\event{wx0}{\DW{x}{0}}{below=of wy0}
\event{rx1}{\DR{x}{0}}{right=of wx0}
\event{wy1}{\DW{y}{1}}{right=of rx1}
\rf{wx0}{rx1}
\rf{wy0}{ry1}
\wk{rx1}{wx1}
\wk{ry1}{wy1}
\end{tikzcenter}
\end{displaymathsmall}
For the program
\begin{math}
  (y\GETS 0 \SEMI   x \GETS 1  \SEMI \aReg \GETS y)
  \PAR
  (x\GETS 0 \SEMI  y \GETS 1  \SEMI  \bReg \GETS x):
\end{math}
\begin{displaymathsmall}
\mbox{From  }
\qquad \qquad
\begin{tikzcenter}[node distance=1em]
\event{wy0}{\DW{y}{0}}{}
\event{wx1}{\DW{x}{1}}{right=of wy0}
\event{ry0}{\DR{y}{0}}{right=of wx1}
\event{wx0}{\DW{x}{0}}{below=of wy0}
\event{wy1}{\DW{y}{1}}{right=of wx0}
\event{rx0}{\DR{x}{0}}{right=of wy1}
\rf[bend right]{wx0}{rx0}
\rf[bend left]{wy0}{ry0}
\wk{rx0}{wx1}
\wk{ry0}{wy1}
\wk{wx0}{wx1}
\wk{wy0}{wy1}
\end{tikzcenter}
\qquad
\ \mbox{ infer }
\qquad
\begin{tikzcenter}[node distance=1em]
\event{wy0}{\DW{y}{0}}{}
\event{wx1}{\DW{x}{1}}{right=of wy0}
\event{ry0}{\DR{y}{1}}{right=of wx1}
\event{wx0}{\DW{x}{0}}{below=of wy0}
\event{wy1}{\DW{y}{1}}{right=of wx0}
\event{rx0}{\DR{x}{1}}{right=of wy1}
\rf{wx1}{rx0}
\rf{wy1}{ry0}
\wk{wx0}{wx1}
\wk{wy0}{wy1}
\end{tikzcenter}
\end{displaymathsmall}
For the program
\begin{math}
(x\GETS 1) \PAR
(x\GETS 0):
\end{math}
\begin{displaymathsmall}
\mbox{From  }
\qquad\qquad
\begin{tikzcenter}[node distance=1em]
\event{wy0}{\DW{x}{1}}{}
\event{wx0}{\DW{x}{0}}{right=of wx0}
\wk{wy0}{wx0}
\end{tikzcenter}
\qquad
\qquad
\mbox{ infer   }
\qquad
\begin{tikzcenter}[node distance=1em]
\event{wy0}{\DW{x}{1}}{}
\event{wx0}{\DW{x}{0}}{right=of wx0}
\wk{wx0}{wy0}
\end{tikzcenter}
\end{displaymathsmall}


\endinput

We say that $\aCmd$ has an \emph{SC race} if there is some pomset in $\semsc{\aCmd}$ that contains a data race.


In this section we show that if $\semsc{\aCmd}\subseteq\sem{\aCmd}$ has only
race-free executions, then each pomset $\aPS\in\sem{\aCmd}$ is ``equivalent''
to some $\aPS'\in\semsc{\aCmd}$, where $\aPS'$ may have more events, but
preserves labeling and has less order.

We say that $\aPS\suborder\aPS'$ if there is an injection
$\inj:\Event'\rightarrow\Event$ such that:
\begin{itemize}
\item $\labelingAct'(\aEv) = \labelingAct(\inj(\aEv))$
\item $\labelingForm'(\aEv)$ implies $\labelingForm(\inj(\aEv))$
\item $\labelingForm(\bEv)$ implies $\bigvee_{\aEv\in\inj^{-1}(\bEv)}(\labelingForm'(\aEv))$
\item $\aEv\le'\bEv$ implies $\inj(\aEv)\le\inj(\bEv)$
\item $\aEv\gtN'\bEv$ implies $\inj(\aEv)\gtN\inj(\bEv)$
\end{itemize}

\begin{theorem}
  If $\semsc{\aCmd}$ contains only race-free executions and
  $\aPS\in\sem{\aCmd}$ then there exists some $\aPS'\in\semsc{\aCmd}$ such
  that $\aPS\suborder\aPS'$.
\end{theorem}
% \begin{proof}
%   \begin{itemize}
%   \item
%     \begin{math}
%       \sem{\SKIP}
%       =
%       \{ \emptyset \} 
%     \end{math}
%   \item
%     \begin{math}
%       \sem{\FENCE_{\aF}\SEMI \aCmd}
%       =
%       (\DF{\aF}) \prefix \sem{\aCmd}
%     \end{math}
%   \item
%     \begin{math}
%       \sem{\aLoc\GETS\aExp\SEMI \aCmd}
%       =
%       \textstyle\bigcup_\aVal\; \bigl((\aExp=\aVal) \guard (\DW\aLoc\aVal) \prefix \sem{\aCmd}\bigr)[\aExp/\aLoc]
%     \end{math}
%   % \item
%   %   \begin{math}
%   %     \sem{\REL\aLoc\GETS\aExp\SEMI \aCmd}
%   %     =
%   %     \textstyle\bigcup_\aVal\; \bigl((\aExp=\aVal) \guard (\DWRel\aLoc\aVal) \prefix \sem{\aCmd}\bigr)[\aExp/\aLoc]
%   %   \end{math}
%   \item
%     \begin{math}
%       \sem{\aReg\GETS\aLoc\SEMI \aCmd}
%       =
%       \sem{\aCmd}[\aLoc/\aReg] \cup \textstyle\bigcup_\aVal\; (\DR\aLoc\aVal) \prefix \sem{\aCmd}[\aLoc/\aReg]
%     \end{math}
%   % \item
%   %   \begin{math}
%   %     \sem{\ACQ\aReg\GETS\aLoc\SEMI \aCmd}
%   %     =
%   %     \textstyle\bigcup_\aVal\; (\DRAcq\aLoc\aVal) \prefix \sem{\aCmd}[\aLoc/\aReg]
%   %   \end{math}
%   \item
%     \begin{math}
%       \sem{\IF{\aExp} \THEN \aCmd \ELSE \bCmd \FI}
%       =
%       \bigl((\aExp \neq 0) \guard \sem{\aCmd}\bigr) \parallel \bigl((\aExp=0) \guard \sem{\bCmd}\bigr)
%     \end{math}
%   \item
%     \begin{math}
%       \sem{\aCmd \PAR \bCmd}
%       =
%       \sem{\aCmd} \parallel \sem{\bCmd}
%     \end{math}
%   \item
%     \begin{math}
%       \sem{\VAR\aLoc\SEMI \aCmd}
%       =
%       \nu \aLoc \st \sem{\aCmd}
%     \end{math}
% \end{itemize}
  
% \end{proof}
% \end{theorem}


% To define compatibility, we extend the definitions of
% \textsection\ref{sec:semantics} to include an additional order: $\rird$.
% \begin{itemize}
% \item
%   ${\rird'} = {\rird}$
%   when $\aPSS'=\aPSS\aSub$
%   or $\aPSS'=\aForm\guard\aPSS$
% \item
%   ${\rird'} = {\rird}\restrict{\Event'}$
%   when $\aPSS'=\nu\aLoc\st\aPSS$
% \item
%   ${\rird'} = {\rird}^1\cup{\rird}^2$
%   when $\aPSS'=\aPSS^1\parallel\aPSS^2$
% \item
%   ${\rird'} = {\rird}\cup\{(\cEv,\aEv)\mid\labelingForm(\aEv) \;\text{is dependent on}\; \aLoc\}$
%   when $\aPSS'=\aAct\prefix\aPSS$, $\aAct$ writes $\aLoc$, and $\Event' = \Event \cup \{\cEv\}$
% \end{itemize}

% From $\rird$, we define ${\rrb}={\rird}^{-1};{\gtN}$.

% We want that if there is an execution:
% \begin{tikzdisplay}[node distance=1em]
%   \event{a}{\DW{\aLoc}{1}}{}
%   \event{b}{\DW{\bLoc}{1}}{below right=1em and 6em of a}
%   \event{c}{\DW{\aLoc}{2}}{above right=1em and 1em of b}
%   \wk{a}{c}
%   \ird{a}{b}
%   \rb{b}{c}
% \end{tikzdisplay}
% Then there is also
% \begin{tikzdisplay}[node distance=1em]
%   \event{a}{\DW{\aLoc}{1}}{}
%   \event{b}{\DW{\bLoc}{1}}{below right=1em and 6em of a}
%   \event{c}{\DW{\aLoc}{2}}{above right=1em and 1em of b}
%   \event{r}{\DR{\aLoc}{1}}{below right=.1em and 2em of a} 
%   \wk{a}{c}
%   \rf{a}{r}
%   \po{r}{b}
% \end{tikzdisplay}


% To see that we need $[\aExp/\aLoc]$ in the rule for write, rather than $[\aVal/\aLoc]$
% consider example:
% \begin{verbatim}
% r=y; if (r) {x=r} else {x=r}; s=x; if (r==s) {z=1}
% \end{verbatim}
% or simplified:
% \begin{verbatim}
% r=y;x=r;s=x; if(s==r){z=1}
% \end{verbatim}
% If you read 37 for $y$, then the predicate on \texttt{Wz1} before the
% read is either $r=r$ or $v=r$, where $v=37$, for example.  In one case you
% get a dependency and in the other you do not.


\begin{verbatim}
ob does not contradict eco

ob does not contradict (co cap po):

Suppose that wx1 po wx2 then it cannot be that wx2 ob wx1.
We know that wx1 co wx2 by SC-PER-LOC

% Case 1. w1 is read externally, then we have
%   wx1 rfe r
% and
%   r fre w2
% so
%   wx1 obs+ wx2
% which contradicts EXTERNAL

% Case 2. wx1 is not read externally.
We show this by contradiction
Assume
  wx1 co wx2
and
  wx2 ob wx1

Note that
  po supseteq dob cup aob cup bob
So in order to get order into wx1, we must have
  wx2 (ob?; obs; ob?; obs; ob?) wx1

Note that we cannot have dob or bob into wx1 after obs, since then we would
also have it into wx2, creating a cycle in EXTERNAL.  This holds because both
dob and bob are closed on the right w.r.t. coi

So it must be that 
  wx2 (ob?; obs; ob?; wx0; coe) wx1, 
in which case we also have wx0 coe wx2, contradicting EXTERNAL
or 
  wx2 (ob?; obs; ob?; rx0; fre) wx1
in which case we also have rx0 fre wx2, contradicting EXTERNAL




Internal reads do not need to respect ob:
Arm allows the following:

  Ra1 -ctrl-> Wx1 -rfi-> Rx1 ---> Wb1    if(a){x=1}; b=x
   |                               |
  Wa1 <-------------------------- Rb1    a=b


Suppose that wx1 po rx2 and rx2 is read externally.
Then it cannot be that rx2 ob wx1.

Case 1: if wx1 co wx2, then we have wx1 coe wx2 rfe rx2, contradicting EXTERNAL
Case 2: if wx2 co wx1, then we have rx2 fr wx1, contradicting SC-PER-LOC



Suppose that rx1 po wx2 and rx1 is read externally.
Then it cannot be that wx2 ob rx1.

Case 1: if wx2 co wx1, then wx2 co wx1 rf rx1 po wx2, contradicting SC-PER-LOC 
Case 2: if wx1 co wx2, ....

Counterexample.
I believe ARM allows the following:

              Wx1                 x=1
               |
  Ra2 -ctrl-> Rx1 - - -> Wx2      if(a){r=x}; x=2
   |                      |
  Wa2 <----------------- Rx2      a=x

In ARM - - -> is an internal from-read



Other examples to type in:
Allowed:
Rx1 -> Wy0  Wy1
Ry1 -> Wz0  Wz1
Rz1 -> Wx0  Wx1

Forbidden:
Rx1 -> Wy0 Wy1
Ry1 -> Wx0 Wx1

\end{verbatim}


\section{Compiling to hardware}

\citet{DBLP:journals/pacmpl/PodkopaevLV19} define the \emph{Intermediate
  Memory Model (IMM)} and provide efficient implementations of the IMM into
several processor architectures, including TSO, ARMv8 and Power.

In this section, we show that any execution allowed by a sublanguage of the
IMM is also allowed by our semantics.  The sublanguage we consider bans
loops, read-modify-write (RMW) operations, and fences.  In addition, we take
the set of memory locations, $\Loc$, to be finite.  Syntactically, we drop
the superscript \textsf{rlx} on relaxed reads and writes; in addition, we use
structured conditionals rather than the more general \textsf{goto}.  We refer
to this sublanguage as $\muIMM$.

$\muIMM$ programs sit in the restriction-free fragment of our language, where
all memory locations are initialized to $0$ and parallel-composition occurs
only at top level.  In other words, $\muIMM$ programs have the form
\begin{displaymath}
  {\aLoc_1}\GETS{0}\SEMI
  \cdots\SEMI
  {\aLoc_m}\GETS{0}\SEMI
  (\aCmd^1 \PAR \cdots \PAR \aCmd^n)
\end{displaymath}
where $\aCmd^1$, \ldots, $\aCmd^n$ do not include either composition or
restriction.

Due to space limitations, we do not include a full description of the IMM.
The broad strokes of the argument given here should be clear, but interested
readers will need to refer to \citep{DBLP:journals/pacmpl/PodkopaevLV19} for
details.

Let $G$ be an execution graph for a $\muIMM$ program satisfying the
consistency requirements in \textsection3.4 of
\citep{DBLP:journals/pacmpl/PodkopaevLV19}. (Because the source language
lacks RMW operations, the ``is exclusive'' flag on every read will be
\textsf{not-ex} and the RMW mode on every write will be \textsf{normal}.)

Let $R^*$ denotes the reflexive and transitive closure of relation $R$.  Let
$R;S$ denote the composition of relations $R$ and $S$.

Given an execution graph $G$, we say that $\aEv$ is an \emph{internal read} if
$\aEv\in\fcodom(G.\mathsf{po}\cap G.\mathsf{rf})$.

From $G$ we construct a candidate pomset $\aPS$ as follows:
\begin{itemize}
\item $\Event= G.\textsf{E}$,
\item $\labelingAct(\aEv)=\tau G.\mathsf{lab}(e)$, if $\aEv$ is a relaxed
  internal read, 
\item $\labelingAct(\aEv)=G.\mathsf{lab}(e)$, if $\aEv$ is not a relaxed
  internal read,
\item $\labelingForm(\aEv)=\TRUE$,
\item ${\leq} = G.{\rar}^*$, and
\item ${\gtN} = {\leq} \cup {\leq}; {G.{\reco}} \cup {G.{\reco}} ; {\leq}$
\end{itemize}

We show that $\aPS$ is a top-level pomset, reasoning as follows.
First, we establish the criteria for a 3-valued pomset (Definition~\ref{def:3valued}).
\begin{itemize}
\item ${\le}$ is a partial order.  This holds since $G.{\rar}$ is acyclic.
\item If $\bEv \le \aEv$ then $\bEv \gtN \aEv$.  By construction.
\item If $\bEv \le \aEv$ and $\aEv \gtN \bEv$ then $\bEv = \aEv$.  ????
\item If $\cEv \le \bEv \gtN \aEv$ or $\cEv \gtN \bEv \le \aEv$ then
  $\cEv \gtN \aEv$. By construction.
\end{itemize}

Next, we establish the criteria for a 3-valued pomset with preconditions (Definition~\ref{def:3pre}).
\begin{itemize}
\item $\labelingForm(\aEv)$ implies $\labelingForm(\bEv)$ whenever
  $\bEv\le\aEv$.   Trivial, since every formula is $\TRUE$.
\item $\aPS$ is $\aLoc$-coherent; that is, when restricted to events that
  read or write $\aLoc$, $\gtN$ forms a partial order.
\end{itemize}

Finally, we establish the criteria for a top-level pomset
(Definition~\ref{def:x-closed}).
\begin{itemize}
\item $\aEv$ is location independent. Trivial, since every formula is $\TRUE$.
\item If $\aEv$ reads $\aVal$ from $\aLoc$, then there is some $\bEv$ such that
  \begin{itemize}
  \item $\bEv \lt \aEv$,  
  \item $\bEv$ writes $\aVal$ to $\aLoc$, and
  \item if $\cEv$ writes to $\aLoc$
    then either $\cEv \gtN \bEv$ or $\aEv \gtN \cEv$.
  \end{itemize}    
\end{itemize}

\section{Single-Threaded Optimizations}
\label{sec:opt}

A program is \emph{sequential} if it lacks $\!\!\PAR\!\!$, and
\emph{synchronization-free} if lacks fences and $\modeRA/\modeSC$ access.
We argue that our model is fully flexible with respect to 
optimization of such programs, as long as the
optimizations do not introduce new writes or ``relevant'' reads.  To do so, we
 isolate a \emph{linear} fragment of our language that ensure these
restrictions.  We then show the soundness of {\em all} transformations of
synchronization-free sequential programs into this fragment.  
%
We end this section with a discussion of specific optimizations, some of
which relax the linearity assumption and include synchronization.

\paragraph{Pomsets for Hoare Logic.}
We use Hoare logic to establish the soundness of transformations, developing
a pomset semantics for pairs of formulae $\semRW{\aForm}{\bForm}$ and showing
a relation between $\sem{\aCmd}$ and $\semRW{\aForm}{\bForm}$ for valid Hoare
triples $\hoare{\aForm}{\aCmd}{\bForm}$.

\begin{definition}
  \label{def:prepost}
  Let $\aPS\in\semRW{\aForm\land\cForm}{\bForm}$
  %We say that $\aPS$ \emph{satisfies} precondition $\aForm$ and postcondition $\bForm$ (notation $\aPS\in\semRW{\aForm}{\bForm}$)
  when it possible to satisfy the following:

  Let $\aLocs$ be the set of locations such that $\aLoc\in\aLocs$ exactly
  when $\aForm$ depends on $\aLoc$.  For each $\aLoc\in\aLocs$, choose
  $\bEv_\aLoc\in\Event$ that reads $\aLoc$.
  Let $\bEvs_\aLocs=\{\bEv_\aLoc\mid\aLoc\in\aLocs\}$.
  Let $\aSub$ be the substitution generated % from $\aLocs$
  as follows:
  % For $\aLoc\in\aLocs$, 
  $\aLoc\aSub = \aVal$ exactly when $\bEv_\aLoc$ reads $\aVal$ from $\aLoc$.

  Let $\bLocs$ be the set of locations such that $\bLoc\in\bLocs$ exactly
  when $\bForm$ depends on $\bLoc$.  For each $\bLoc\in\bLocs$, choose
  $\aEv_\bLoc\in\Event$ that writes $\bLoc$.
  Let $\aEvs_\bLocs=\{\aEv_\bLoc\mid\bLoc\in\bLocs\}$.
  Let $\bSub$ be the substitution generated % from $\bLocs$
  as follows:
  % For $\bLoc\in\bLocs$, 
  $\bLoc\bSub = \aVal$ exactly when $\aEv_\bLoc$ writes $\aVal$ to $\bLoc$.

  Require that $\aForm\aSub$ and $\bForm\bSub$ are satisfiable.

  Require that $\labelingForm(\aEv_\bLoc)$ implies $\cForm$.
  
  Require that if $\cEv\le\aEv_\bLoc$ and $\cEv$ is a read, then $\cEv\in\bEvs_\aLocs$.

  Require that if $\aEv_\bLoc\le\cEv$ and $\cEv$ is a write to $\bLocs$, then $\cEv\in\aEvs_\bLocs$.
\end{definition}
Pictorially, we have:
\begin{tikzdisplay}[node distance=.1ex and 2em]
  \event{r}{\bEvs_\aLocs}{}
  \event{w2}{\cForm\mid\aEvs_\bLocs}{below right=of r}
  \event{w1}{\cEvs_\bLocs}{above right=of w2}
  \po{r}{w2}
  \wk{w1}{w2}
\end{tikzdisplay}
Here, $\aEvs_\bLocs$ are the final writes to $\bLocs$, with precondition $\cForm$.
$\cEvs_\bLocs$ are other writes to $\bLocs$, which must be ordered before $\aEvs_\bLocs$.
$\bEvs_\aLocs$ are the reads that the writes depend upon.

Under this interpretation, precondition strengthening in Hoare logic
validates read introduction.  In our semantics, reads have no side effects.
Thus, it should be sound to introduce irrelevant reads.  Yet,
$\sem{x\GETS\aExp\SEMI r\GETS x\SEMI \aCmd}\neq \sem{x\GETS\aExp\SEMI\aCmd}$, even when $r$ does not appear in
$\aCmd$.  To make such equations hold, we define $\readc(\aPSS)$ to saturate
$\aPSS$ with reads.

In addition, Hoare postconditions are properties of completed executions.
For example, in $\hoare{\TRUE}{x\GETS1\SEMI x\GETS2}{x{=}2}$, the postcondition
does not hold for the prefix $x\GETS1$.  As a result, we also define
$\readc(\aPSS)$ to restrict attention to maximal executions.  A pomset
$\aPS\in\aPSS$ is \emph{maximal} if there is no $\aPS'\in\aPSS$ such that
$\aPS$ is a proper prefix of $\aPS'$.


Let $\readc(\aPSS)$ be the set $\aPSS'$ where $\aPS'\in\aPSS'$ when there is
a maximal pomset $\aPS\in\aPSS$ and some set $D$ such that
$\Event'= \Event'\uplus D$, ${\le'} \supseteq{\le}$,
$\labeling'(\aEv) = \labeling(\aEv)$, and for every $\bEv\in D$ there are
$\aLoc$ and $\aVal$ such that
$\labelingAct'(\bEv)=(\DR[\modeRLX]{\aLoc}{\aVal})$.

\begin{theorem}
  \label{thm:hoare}
  Let $\aCmd$ be synchronization-free and sequential.  Then
  $\hoare{\aForm}{\aCmd}{\bForm}$ if and only if
  $\notdisjoint{\semRW{\aForm}{\bForm}}{\readsem{\aCmd}}$.
\begin{proof}
  The proof proceeds by induction on the number of steps of derivation of the
  proof of the Hoare triple.

  We first consider the structural rules.  Precondition strengthening follows
  from augmentation closure.  The proof that the structural rule of
  disjunction:
  \begin{displaymath}
    \frac{\hoare{\aForm_1}{\aCmd}{\bForm_1},  \hoare{\aForm_2}{\aCmd}{\bForm_2}}{ \hoare{\aForm_1 \lor \aForm_2}{\aCmd}{\bForm_1\lor \bForm_2}} 
  \end{displaymath}
  holds  follows from closure of the semantics under disjuncts. The proof for the structural rule of conjunction:
  \begin{displaymath}
    \frac{\hoare{\aForm_1}{\aCmd}{\bForm_1},  \hoare{\aForm_2}{\aCmd}{\bForm_2}}{ \hoare{\aForm_1 \land \aForm_2}{\aCmd}{\bForm_1\land \bForm_2}} 
  \end{displaymath}
  follows from the fact that pomsets have only concurrency and no conflict.  

  The remaining cases that use the rules for deducing Hoare triples by
  structural induction on the command follow directly from the semantics.
  The only subtleties are in the write rule, which uses $\parallel$ to ensure
  disjunction closure.
\end{proof}
\end{theorem}
To illustrate multiple writes consider:
\begin{gather*}
  \hoare{\TRUE}{x\GETS 1 \SEMI x \GETS 2}{x=2}
  \\
  \hbox{\begin{tikzinline}[node distance=1em]
      \event{w1}{\DW{x}{1}}{}
      \event{w2}{\DW{x}{2}}{right=of w1}
      \wk{w1}{w2}
    \end{tikzinline}}
\end{gather*}
Preconditions can be placed in $\aForm$ or $\cForm$ in
Definition~\ref{def:prepost}, resulting in different pomsets:
\begin{gather*}
  \hoare{x=1}{y \GETS x}{y=1}
  \\
    \hbox{\begin{tikzinline}[node distance=1em]
        \event{r}{\DR{x}{1}}{}
        \event{w}{\DW{y}{1}}{right=of r}
        \po{r}{w}
      \end{tikzinline}}
    \qquad\qquad
    \hbox{\begin{tikzinline}[node distance=1em]
        \event{r}{\DR{x}{1}}{}
        \event{w}{x=1 \mid \DW{y}{1}}{right=1ex of r}
      \end{tikzinline}}
\end{gather*}
Control dependencies are calculated correctly:
\begin{gather*}
  \hoare{\TRUE}{\IF{x} \THEN y \GETS 1 \ELSE y \GETS 1 \FI}{y=1} 
  \\
    \hbox{\begin{tikzinline}[node distance=1em]
        \event{r}{\DR{x}{1}}{}
        \event{w}{\DW{y}{1}}{right=1ex of r}
      \end{tikzinline}}
\end{gather*}
% In contrast, in a semantics that forbids load buffering, where the best that one can prove is
% $\hoare{x=v} {r \GETS x \SEMI \IF{r=1} \THEN y \GETS 1 \ELSE y \GETS 1 \FI}{y= 1}
% $.

For any compatible set of preconditions, we can always find a single pomset
that includes all of the required writes.
\begin{gather*}
  \hoare{x_1=1}{y_1 \GETS x_1}{y_1=1}
  \qquad
  \hoare{x_2=1}{y_2 \GETS x_2}{y_2=1}
  \\
    \hbox{\begin{tikzinline}[node distance=1em]
        \event{rx}{\DR{x_1}{1}}{}
        \event{wy}{\DW{y_1}{1}}{right=of rx}
        \event{ru}{\DR{x_2}{1}}{right=1ex of wy}
        \event{wv}{\DW{y_2}{1}}{right=of ru}
        \rf{rx}{wy}
        \rf{ru}{wv}
      \end{tikzinline}}
\end{gather*}
\begin{corollary}
  If $\bigwedge_{i\in I}\aForm_i$ is satisfiable:
  \begin{displaymath}
    \textstyle\bigwedge_{i\in I}\hoare{\aForm_i}{\aCmd}{\bForm_i} \Longleftrightarrow
    \notdisjoint{\textstyle\bigcap_{i\in I}\semRW{\aForm_i}{\bForm_i}}{\readsem{\aCmd}}.
  \end{displaymath}
\end{corollary}

\paragraph{Linearity.}
A command $\aCmd$ is \emph{linear} for every $\aPS\in\sem{\aCmd}$, there is
at most one read and at most one write on any location.  Intuitively, this
means that the context around $\aCmd$ is unable to interfere with the atomic
execution of $\aCmd$; dually, neither can the atomic execution of $\aCmd$
interfere with the context.  From an arbitrary command, it is a
straightforward exercise to construct a linear command that is sequentially
equivalent.

We say that $\aCmd$ and $\aCmd'$ \emph{satisfy the same Hoare triples} when
$\hoare{\aForm}{\aCmd}{\bForm}$ if and only if
$\hoare{\aForm}{\aCmd'}{\bForm}$, for every $\aForm$ and $\bForm$.

\begin{corollary}\label{seqcompleteness}
  Let $\aCmd$ and $\aCmd'$ be synchronization-free and sequential.  Further,
  let $\aCmd'$ be linear.
  Then $\aCmd$ and $\aCmd'$ satisfy the same Hoare
  triples if and only if $\readsem{\aCmd} \supseteq \readsem{\aCmd'}$.
  % If $\aCmd$ and $\aCmd'$ satisfy the same Hoare
  % triples then $\readsem{\aCmd} \supseteq \readsem{\aCmd'}$.
%   \begin{proof}
%     The pomsets in the semantics of a linear sequential program fragment,
%     such as $\sem{\bCmd}$, are generated by the augmentation closure of
%     pomsets that have a special format that only include edges of the
%     form: \begin{tikzdisplay}[node distance=1em]
%       \event{r}{\smash{\vec{\bForm}}\mid\DR{\vec{\bLoc}}{\vec{\aVal}}}{}
%       \eventl{\aEv}{w}{\aForm\mid\DW{\aLoc}{\bVal}}{below right=of r}
%       \po{r}{w}
%     \end{tikzdisplay}
%     where $\vec{\bLoc}\GETS \vec{\aLoc}$ has no conflicting assignments to the same variable, and where each $\aLoc$ appears in at most one write event. Thus, using  theorem~\ref{hoareGen}, we deduce that $\sem{\bCmd}$ is completely determined by the Hoare triples satisfied by $\bCmd$.  
%    
%     By the hypothesis of this theorem, $\aCmd$ satisfies the same Hoare triples as $\bCmd$.  Using  theorem~\ref{hoareGen}, we deduce that $\sem{\bCmd} \subseteq \fsat(\sem{\aCmd})$. 
% \end{proof}
\end{corollary}

\paragraph{Valid Rewrites.}
\ When $\readc\sem{\aCmd} \supseteq \readc\sem{\aCmd'}$, we say that $\aCmd'$ is
a \emph{valid transformation} of $\aCmd$.

To enable reasoning about program fragments, transformation validity must be
preserved by \emph{contexts}.  In \textsection\ref{sec:model}, we defined the
semantics by prefixing one action at a time.  This helps to make the
semantics understandable, but it also creates impoverished contexts.

To allow for richer contexts, we appeal to the alternate presentation of the
language given in \textsection\ref{sec:semicolon}.  We refactor the syntax
of commands and define contexts:
\begin{align*}
  \aCmd,\,\bCmd
  \BNFDEF& \SKIP
  \mkern-2mu\BNFSEP\mkern-2mu \FENCE^{\fmode}
  \mkern-2mu\BNFSEP\mkern-2mu \aReg\GETS\aExp
  % \mkern-2mu\BNFSEP\mkern-2mu \aReg\GETS \aLoc^{\amode} 
  % \mkern-2mu\BNFSEP\mkern-2mu \aLoc^{\amode}\GETS\aExp
  \mkern-2mu\BNFSEP\mkern-2mu \aReg\GETS \REF{\cExp}^{\amode} 
  \mkern-2mu\BNFSEP\mkern-2mu \REF{\cExp}^{\amode}\GETS\aExp
  \\[-.5ex]
  \BNFSEP&\aCmd \PAR \bCmd
  \mkern-2mu\BNFSEP\mkern-2mu\aCmd \SEMI \bCmd
  \mkern-2mu\BNFSEP\mkern-2mu \VAR\aLoc\SEMI \aCmd
  \mkern-2mu\BNFSEP\mkern-2mu \IF{\aExp} \THEN \aCmd \ELSE \bCmd \FI
  \\
  \aCtxt,\,\bCtxt
  \BNFDEF& \hole{}
  \mkern-2mu\BNFSEP\mkern-2mu \aCtxt \PAR \bCmd
  \mkern-2mu\BNFSEP\mkern-2mu \aCmd \PAR \bCtxt
  \mkern-2mu\BNFSEP\mkern-2mu \aCtxt \SEMI \bCmd
  \mkern-2mu\BNFSEP\mkern-2mu \aCmd \SEMI \bCtxt
  \mkern-2mu\BNFSEP\mkern-2mu \VAR\aLoc\SEMI \aCtxt
  \\[-.5ex]
  \BNFSEP& \IF{\aExp} \THEN \aCtxt \ELSE \bCmd \FI
  \mkern-2mu\BNFSEP\mkern-2mu \IF{\aExp} \THEN \aCmd \ELSE \bCtxt \FI
\end{align*}


\begin{lemma}%\label{freadssatcomp}[Compositionality of $\freadsat$]
  Let $\bCtxt$ be a context
  % \footnote{%
  %   The results of this section hold for contexts of the example language
  %   given in \textsection\ref{sec:model} and extended in \textsection\ref{sec:variants}.
  %   % \begin{math}
  %   %   \begin{array}[t]{rcl}
  %   %     \aCtxt,\,\bCtxt
  %   %     &\BNFDEF& \hole{}
  %   %     \BNFSEP \aReg\GETS\aExp\SEMI \aCtxt
  %   %     \BNFSEP \aReg\GETS\aLoc^{\amode}\SEMI \aCtxt 
  %   %     \BNFSEP \aLoc^{\amode}\GETS\aExp\SEMI \aCtxt
  %   %     \\[-.5ex]
  %   %     &\BNFSEP&\aCtxt \PAR \bCmd
  %   %     \BNFSEP \aCmd \PAR \bCtxt
  %   %     \BNFSEP \VAR\aLoc\SEMI \aCtxt
  %   %     \\[-.5ex]
  %   %     &\BNFSEP& \IF{\aExp} \THEN \aCtxt \ELSE \bCmd \FI
  %   %     \BNFSEP \IF{\aExp} \THEN \aCmd \ELSE \bCtxt \FI
  %   %   \end{array}
  %   % \end{math}
  %   The results of this section hold for contexts of the example language
  %   given in \textsection\ref{sec:model} and extended in \textsection\ref{sec:variants}.
  %   The results also hold for the more general contexts of
  %   \textsection\ref{sec:semicolon}, which includes full sequential composition:
  %   \begin{displaymath}
  %     \begin{array}[t]{rcl}
  %       \aCtxt,\,\bCtxt
  %       &\BNFDEF& \hole{}
  %       \BNFSEP \aCtxt \SEMI \bCmd
  %       \BNFSEP \aCmd \SEMI \bCtxt
  %       \BNFSEP \aCtxt \PAR \bCmd
  %       \BNFSEP \aCmd \PAR \bCtxt
  %       \BNFSEP 
  %       \\[-.5ex]
  %       &\BNFSEP& \VAR\aLoc\SEMI \aCtxt
  %       \BNFSEP \IF{\aExp} \THEN \aCtxt \ELSE \bCmd \FI
  %       \BNFSEP \IF{\aExp} \THEN \aCmd \ELSE \bCtxt \FI
  %     \end{array}
  %   \end{displaymath}}
and $\readc\sem{\aCmd} \supseteq \readc\sem{\aCmd'}$:
\begin{displaymath}
  \readc\sem{\bCtxt\hole{\aCmd}} \supseteq \readc\sem{\bCtxt\hole{\aCmd'}}
\end{displaymath}
\end{lemma}

To discuss valid transformations without getting lost in notation, we present
them using simple locations, rather than calculated addresses.  The extension
is simple: For address expressions $\REF{\cExp}$ and $\REF{\dExp}$, replace
$\aLoc=\bLoc$ by provable equality of $\cExp$ and $\dExp$, and
$\aLoc\neq\bLoc$ by provable inequality.  Operations on
sets can be defined similarly.
%
Let $\free(\aCmd)$ be the set of locations and registers that occur in $\aCmd$.

Theorem \ref{thm:hoare} immediately validates peephole optimizations, such as
redundant load \eqref{RL}, store forwarding \eqref{SF}, dead store \eqref{DS},
and independent reorderings.  Using the semantics directly, we can prove
these properties without using $\readc$.
Note that if $\aPSS'\supseteq\aPSS$, then $\readc(\aPSS')\supseteq \readc(\aPSS)$.
\begin{align*}
  \taglabel{RL}
  \sem{\aReg \GETS \aLoc \SEMI \bReg  \GETS \aLoc} &\supseteq 
  \sem{\aReg \GETS \aLoc \SEMI \bReg  \GETS \aReg}
  \\
  \taglabel{SF}
  \sem{\aLoc \GETS \aExp \SEMI \bReg  \GETS \aLoc} &\supseteq 
  \sem{\aLoc \GETS \aExp \SEMI \bReg  \GETS \aExp}
  \\
  \taglabel{DS}
  \sem{\aLoc \GETS \aExp \SEMI \aLoc  \GETS \bExp} &\supseteq 
  \sem{\aLoc \GETS \bExp}    
  \\
  \taglabel{WW}
  \sem{\aLoc \GETS \aExp \SEMI \bLoc  \GETS \bExp} &=
  \sem{\bLoc  \GETS \bExp\SEMI \aLoc \GETS \aExp} &%\text{if } \aLoc{\neq}\bLoc
  \\
  \taglabel{RW}
  \sem{\aReg \GETS \aLoc \SEMI \bLoc  \GETS \bExp} &=
  \sem{\bLoc  \GETS \bExp\SEMI \aReg \GETS \aLoc} &%\text{if } \aLoc{\neq}\bLoc
  \\
  \taglabel{RR}
  \sem{\aReg \GETS \aLoc \SEMI \bReg  \GETS \bLoc} &=
  \sem{\bReg  \GETS \bLoc\SEMI \aReg \GETS \aLoc} &%\text{if } \aLoc{\neq}\bLoc
\end{align*}
\eqref{RR} requires either $\aReg\neq\bReg$ or $\aLoc=\bLoc$.  \eqref{WW} and
\eqref{RW} require that two sides of the semicolon have disjoint ids; for example,
\eqref{RW} requires $\disjoint{\free(\aReg \GETS \aLoc)}{\free(\bLoc \GETS \bExp)}$.
\eqref{RL} and \eqref{SF} follow from prefix closure.
\eqref{SF} follows from the write elimination allowed in Definition \ref{def:cover}.

Since reads are unordered in our model, read optimizations are not limited by
the power of aliasing analysis, as they are with stronger models of coherence
\cite[\textsection2.3]{DBLP:conf/java/Pugh99}.  Composing \eqref{RR} and
\eqref{RL}, for $\aReg_2\neq\bReg$ we have:
\begin{displaymath}
  \sem{r_1\GETS \aLoc \SEMI
  s\GETS \bLoc \SEMI  
  r_2\GETS \aLoc}
  \supseteq
  \sem{r_1\GETS \aLoc \SEMI
  s\GETS \bLoc \SEMI  
  r_2\GETS r_1}
\end{displaymath}
This holds regardless of whether $\aLoc=\bLoc$.
% \begin{displaymathsmall}
%   \sem{r_1\GETS \REF{\aExp} \SEMI
%   s\GETS \REF{\bExp} \SEMI  
%   r_2\GETS \REF{\aExp}}
%   \supseteq
%   \sem{r_1\GETS \REF{\aExp} \SEMI
%   s\GETS \REF{\bExp} \SEMI  
%   r_2\GETS r_1}
% \end{displaymathsmall}
% This holds regardless of whether $\aExp=\bExp$.


By induction on the length of the pomsets in $\aCmd$, we can use the
reorderings to establish, more generally, that when $\aCmd$ and $\bCmd$ are
assignment sequences and $\disjoint{\free(\aCmd)}{\free(\bCmd)}$:
\begin{gather*}
  \sem{\aCmd \SEMI \bCmd} = \sem{\bCmd \SEMI \aCmd} 
\end{gather*}
% Appealing directly to the semantics, we can establish general properties for
% redundant load \eqref{RLp}, store forwarding \eqref{SFp}, and roach-motel
% \eqref{A}, \eqref{R}:
The semantics also validates roach-motel reorderings.  Let $\aCmd$ be
synchronization-free, with disjoint ids as before:
\begin{align*}
  % \taglabelp{RL}
  % \sem{\aReg \GETS \aLoc  \SEMI \bReg \GETS \aLoc  \SEMI \aCmd} &\supseteq
  % \sem{\aReg \GETS \aLoc \SEMI \aCmd[\aReg/\bReg]}
  % \\
  % \taglabelp{SF} 
  % \sem{\aLoc \GETS \aExp \SEMI \bReg \GETS \aLoc \SEMI \aCmd} &\supseteq 
  % \sem{\aLoc \GETS \aExp \SEMI \aCmd[\aExp/\bReg]}  
  % \\
  \taglabel{A}
  \sem{\aCmd \SEMI \bReg \GETS \aLoc \ACQ} &\supseteq
  \sem{\bReg \GETS \aLoc \ACQ\SEMI  \aCmd}
  \\
  \taglabel{R}
  \sem{\aLoc \REL \GETS \aExp \SEMI \aCmd } &\supseteq
  \sem{\aCmd \SEMI \aLoc \REL \GETS \aExp }
\end{align*}
% Again, we suppose that $\aCmd$ is an assignment sequence and disjoint
% identifiers.
% \eqref{A} and \eqref{R} require
% disjoint names, as in \eqref{WW} and
% \eqref{RW}. In addition, \eqref{A} and \eqref{R} require that $\aCmd$ is
% synchronization-free.

As expected, sequential and parallel composition commute with conditionals
and location binding, and conditionals and location binding commute with each
other.  We show sequential scope extrusion \eqref{SSE}, which concerns
sequential composition and location binding:
\begin{align*}
  % \taglabel{PSE}
  % \sem{\aCmd\PAR \VAR\aLoc\SEMI\bCmd}\allowbreak &=
  % \sem{\VAR\aLoc\SEMI(\aCmd\PAR\bCmd)}
  % \\
  \taglabel{SSE}
  \sem{\aCmd\SEMI \VAR\aLoc\SEMI\bCmd}\allowbreak &=
  \sem{\VAR\aLoc\SEMI(\aCmd\SEMI\bCmd)}
  % \\
  % \taglabel{CSE}
  % \sem{\IF{\aExp}\THEN\aCmd\ELSE \VAR\aLoc\SEMI\bCmd\FI}\allowbreak &=
  % \sem{\VAR\aLoc\SEMI \IF{\aExp}\THEN\aCmd\ELSE\bCmd\FI}
\end{align*}
\eqref{SSE} requires that $\aLoc$ does not appear in $\aCmd$.

Many laws hold for the conditional.  We show case analysis \eqref{CA} and
dead code elimination \eqref{DC}.  The correctness of \eqref{CA} follows from
disjunction closure in the semantics.
\begin{align*}
  \taglabel{CA}
  \sem{\IF{\aExp}\THEN\aCmd\ELSE\aCmd\FI} &=
  \sem{\aCmd}
  \\
  \taglabel{DC}
  \sem{\IF{\aExp}\THEN\aCmd\ELSE\bCmd\FI} &=
  \sem{\aCmd}
\end{align*}
\eqref{DC} requires that $\aExp$ be a tautology.

% Suppose $\cExp\neq\dExp$ is a tautology.  Then the following 
%reorderings hold.
%  \begin{align*}
% \begin{align*}
%    \tag{R-R}
%    \sem{\aReg \GETS \REF{\cExp} \SEMI \bReg \GETS \REF{\dExp} 
%\SEMI \aCmd} &=
%    \sem{\bReg \GETS \REF{\dExp} \SEMI \aReg \GETS \REF{\cExp} 
%\SEMI \aCmd}
%    &&\textif \aReg\notin\free(\dExp) \textand \bReg\notin\free(\cExp)
 %   \\
 %   \tag{R-W}
 %   \sem{\aReg \GETS \REF{\cExp} \SEMI \REF{\dExp} \GETS \bExp 
%\SEMI \aCmd} &=
%    \sem{\REF{\dExp} \GETS \bExp \SEMI \aReg \GETS \REF{\cExp} 
%\SEMI \aCmd}
%    &&\textif \aReg\notin\free(\dExp) \cup\free(\bExp)
%    \\
 %   \tag{W-W}
 %   \sem{\REF{\cExp} \GETS \aExp \SEMI \REF{\dExp} \GETS \bExp \SEMI 
%\aCmd} &=
 %   \sem{\REF{\dExp} \GETS \bExp \SEMI \REF{\cExp} \GETS \aExp \SEMI 
%\aCmd}
%  \end{align*}
% \begin{proof}
% The semantics of sequential composition $\aCmd \SEMI \bCmd$, where 
% the only enforced $\lt$ relationships come from conflict on locations or 
%release or acquire actions.    The roach-motel reorderings that increase 
%the scope of synchronization are valid because the pomsets in the 
%semantics of the right hand sides are augmentations of a pomset on the 
%left hand side. 
%\end{proof}
%\end{lemma}


% \begin{lemma}%[Reorderings]
% Suppose $\cExp\neq\dExp$ is a tautology.  Then the following 
%reorderings hold.
 % \begin{align*}
%    \tag{R-Acq}
 %   \sem{\aReg \GETS \REF{\cExp} \SEMI \bReg \GETS \REF{\dExp}\ACQ 
%\SEMI \aCmd} &\supseteq
 %   \sem{\bReg \GETS \REF{\dExp}\ACQ\SEMI \aReg \GETS \REF{\cExp} 
%\SEMI \aCmd}
% &&\textif \aReg\notin\free(\dExp) \textand \bReg\notin\free(\cExp)
 %   \\
 %   \tag{W-Acq} 
 %   \sem{\REF{\dExp} \GETS \bExp \SEMI\aReg \GETS \REF{\cExp}\ACQ 
%\SEMI\aCmd} &\supseteq
%    \sem{\aReg \GETS \REF{\cExp}\ACQ  \SEMI \REF{\dExp}\GETS \bExp 
%\SEMI \aCmd} 
%    &&\textif \aReg\notin\free(\dExp) \cup\free(\bExp)
 %   \\
% \tag{Rel-R} 
%\sem{\REF{\dExp}\REL \GETS \bExp \SEMI\aReg \GETS \REF{\cExp} 
%\SEMI \aCmd} &\supseteq
 %   \sem{ \aReg \GETS \REF{\cExp} \SEMI \REF{\dExp}\REL \GETS \bExp 
%\SEMI \aCmd}
%   &&\textif \aReg\notin\free(\dExp) \cup\free(\bExp)
 %   \\
 %   \tag{Rel-W}
 %   \sem{\REF{\cExp}\REL \GETS \aExp \SEMI \REF{\dExp} \GETS \bExp %\SEMI \aCmd} &\supseteq
 %   \sem{\REF{\dExp} \GETS \bExp \SEMI \REF{\cExp}\REL \GETS \aExp 
%\SEMI \aCmd}
%  \end{align*}
%\begin{proof}
%The proof follows from noticing that the pomsets in the semantics of the 
%right hand sides are augmentations of a pomset on the left hand side.  
%\end{proof}
%\end{lemma}

% \paragraph*{Compiler optimizations.} Reordering and peephole optimizations
% can be combined to describe common compiler optimizations.  We illustrate
% using common subexpression elimination,
% following~\citet{Dolan:2018:BDR:3192366.3192421}:
% Consider the command 
% \begin{math}
%   (\aReg \GETS \aLoc *2  \SEMI \aCmd \SEMI \bReg \GETS \aLoc * 2)
% \end{math}
% where $\aCmd$ is independent of $\aReg$.  Reordering yields
% \begin{math}
%   (\aCmd \SEMI \aReg \GETS \aLoc *2  \SEMI  \bReg \GETS \aLoc * 2),
% \end{math}
% followed by redundant load to yield
% \begin{math}
%   (\aCmd \SEMI \aReg \GETS \aLoc * 2 \SEMI  \bReg \GETS \aReg).
% \end{math}

% Similarly, the treatment of loop-invariant code motion, dead-store
% elimination and constant propagation
% from~\citet{Dolan:2018:BDR:3192366.3192421} follow.

% Since our model is more generous about permitted reorderings, we 
% permit optimizations that they forbid.  Consider:
%\begin{math}
 % (\aReg \GETS \aLoc \SEMI \bLoc \GETS \cLoc  \SEMI \aLoc \GETS 
%\aReg).
%\end{math}
%Reordering, permitted by us, but forbidden by them, yields
%\begin{math}
%  (\aReg \GETS \aLoc \SEMI \aLoc \GETS \aReg \SEMI \bLoc \GETS 
%\cLoc),
%\end{math}
%followed by the valid elimination of redundant load
%\begin{math}
%  (\aReg \GETS \aLoc \SEMI \aLoc \GETS \aReg \SEMI \bLoc \GETS %
%\cLoc).
%\end{math}



\paragraph{Invalid Rewrites.}
Relevant read introduction is invalid:
\begin{align*}
  %\tag{Read-Intro-Invalid}
  \sem{\aReg \GETS \aLoc \SEMI \IF{\aReg {\neq} \aReg} \THEN \cLoc \GETS 1 \FI}
  &\not\supseteq
  \sem{\aReg \GETS \aLoc \SEMI \bReg \GETS \aLoc  \SEMI \IF{\aReg {\neq}\bReg} \THEN \cLoc \GETS 1 \FI}
\end{align*}
These are distinguished by the context
\begin{math}
  \hole{} \PAR x\GETS1\PAR x\GETS2.
\end{math}

Write introduction is invalid, even for equal values:
\begin{align*}
  \sem{\aLoc \GETS 1 \SEMI \aCmd} 
  &\not\supseteq
  \sem{\aLoc \GETS 1 \SEMI \aLoc \GETS 1 \SEMI \aCmd}
\end{align*}
These are distinguished by the context:
\begin{displaymath}
  \hole{} \PAR
  r\GETS x \SEMI
  x\GETS2 \SEMI
  s\GETS x\SEMI
  \IF{\aReg {=} \aReg} \THEN \cLoc \GETS 1 \FI
\end{displaymath}
With weaker notions of coherence
\cite{Manson:2005:JMM:1047659.1040336, Dolan:2018:BDR:3192366.3192421}, these
commands are indistinguishable.

Thread inlining is invalid (see \textsection\ref{sec:model}):
\begin{math}
  \sem{\aCmd \PAR \bCmd}
  \not\supseteq
  \sem{\aCmd \SEMI \bCmd}.
\end{math}


% // p and q might be aliased
% int i = p
% // concurrent write to p.x by another thread
% int j = q 
% int k = p

% Since p and q only might be aliased, but are not definitely aliased, then the
% use of q cannot be optimized away (if it were known that p and q pointed to
% the same object, then it would be legal to replace the assignments to j and k
% with assignments of the value of i). Consider the case where p and q are in
% fact aliased, and another thread writes to the memory location for p/q
% between the first use of p and the use of q; the use of q will see the
% new value. It will be illegal for the second use of p (stored into k) to
% get the same value as was stored into i. However, a fairly standard compiler
% optimization would involve eliminating the getfield for k and replacing it
% with a reuse of the value stored into i. Un- fortunately, that optimization
% is illegal in any language that requires Coherence.

% One way to think of it is that since a read of a memory location may cause
% the thread to become aware of a write by another thread, it must be treated
% in the compiler as a possible write.



\citet{BoehmOOTA} considers the following programs:
\begin{gather*}
  \sem{r\GETS y\SEMI x\GETS r}
  \not\supseteq
  \sem{r\GETS y\SEMI \IF{r \NOTEQ 1} \THEN z\GETS 1\SEMI r\GETS 1\FI \SEMI x\GETS r}
\end{gather*}
The left command is half of the \oota{} example from
\textsection\ref{sec:logic} \eqref{oota1}.  The right command is dubbed \rfub{}, for
\emph{Register assignment From an Unexecuted Branch}.
\citeauthor{BoehmOOTA} observes that in the context $x\GETS y \PAR \hole{}$,
these programs have different behaviors.  Yet the \oota{} example on the left
never writes $1$.  Why should the unexecuted branch change that?  As it turns
out, both branches of the conditional in \rfub{} can execute, since the write
to $x$ is independent of the read from $y$.  Considering just the two threads
above, we have $\hoare{\TRUE}{\text{\rfub}}{x=1}$, but not
$\hoare{\TRUE}{\text{\oota}}{x=1}$.  As a result, it is expected that \rfub{}
may have additional behaviors.  The change in the thread from \oota{} to
\rfub{} is not a valid refinement under Hoare logic and thus it is not valid
in our semantics.
% Let $\aCmd$ be the right
% thread in \eqref{rfub}.
% \begin{align*}
%   \aCmd = r\GETS y\SEMI \IF{r \NOTEQ 1} \THEN z\GETS 1\SEMI r\GETS 1\FI  \SEMI x\GETS r 
% \end{align*}
% Then, $\hoare{\TRUE}{\aCmd}{x=1} $ and $\hoare{y \neq 1}{\bCmd}{z=1}$.  Thus, the traditional sequential semantics {\em does not} attribute any cause for the write of $1$ to $x$, as seen by the execution:
% \begin{tikzdisplay}[node distance=1em]
%  \event{ry}{\DR{y}{1}}{}
%  \event{wa}{\DW{z}{1}}{right=of ry}
%  \event{wx}{\DW{x}{1}}{right=of wa}
%  \rf{ry}{wa}
% \end{tikzdisplay}
% % Consider a fragment from example~\eqref{rfub}.
% % \begin{align*}
% % \bCmd:  r \GETS y \SEMI \IF{r \NOTEQ 42} \THEN a \GETS 1 \SEMI s \GETS 42 \FI \SEMI x \GETS 42
% % \end{align*}
% % Then, $\hoare{\TRUE}{\bCmd}{x=42} $ and $\hoare{y \neq 42}{\bCmd}{a= 1} $.  Thus, the traditional sequential semantics {\em does not} attribute any cause for the write of $42$ to $x$, as seen by the execution:
% % \begin{tikzdisplay}[node distance=1em]
% %  \event{ry}{\DR{y}{41}}{}
% %  \event{wx}{\DW{x}{42}}{right=of ry}
% %  \event{wa}{\DW{a}{1}}{below=of rx}
% %  \rf{ry}{wa}
% % \end{tikzdisplay}
% Consider the
% following, where all locations are initialized to $0$.  In \textsection\ref{sec:logic} we provide machinery to prove that \oota{} is
% incapable of writing $1$.  The question is whether this should be possible
% for \rfub, which changes \oota{} only to include a \emph{Register assignment
%   From an Unexecuted Branch} \cite{BoehmOOTA}:?  
% \begin{align*}
%   \label{oota}  \tag{\textsc{oota}}
%   y\GETS x
%   \PAR&
%   r\GETS y\SEMI
%   x\GETS r
%   \\
%   \tag{\rfub1}
%   \label{rfub}
%   y\GETS x
%   \PAR&
%   r\GETS y\SEMI
%   \IF{r \NOTEQ 1} \THEN z\GETS 1\SEMI r\GETS 1\FI  \SEMI x\GETS r 
% \end{align*}
% %and the following \emph{Out Of Thin Air} litmus test:



% {\bf MOVE TO APPENDIX??.  

% Given $\aCmd$, define 
% $\linV{\aCmd}$ as follows.
% \begin{definition}
% With each shared variable $\aLoc$, we associate a thread local variable $\aLocLoc$ and two boolean variables $\aRead,\aChanged$ that will be local to the program fragment.  

% \begin{align*}
% \linV{\aCmd}= & \VAR\vec{\aLocLoc} \SEMI \VAR \vec{\aRead} \SEMI \VAR \vec{\aChanged} \SEMI  \\
% & \vec{\aRead} =\vec{0} \SEMI \vec{\aChanged} = \vec{0} \SEMI \\
% & \aCmd' \SEMI   \\
% & \overline{\IF \aChanged\  \aLoc= \aLocLoc}
% \end{align*}
% where:
% $\aCmd'$ is derived from $\aCmd$ by replacing:
% \begin{itemize}
% \item  each read $\aReg = \aLoc$ by $\aRead = 1 \SEMI \IF {(\aChanged \lor\ \aRead)} \THEN\  \aReg= \aLocLoc \ELSE\  \aReg= \aLoc \SEMI \FI $,
% \item each write $\aLoc = \aExp$ by $\aChanged =1 \SEMI \aLocLoc = \aExp$
% \end{itemize}
% \end{definition}
% From the soundness and completeness of Hoare logic for sequential operational semantics (eg. see formalization in~\citet{gordonHoare}), we deduce for any $\aForm, \bForm$: 
% \[ \hoare{\aForm}{\aCmd}{\bForm}  \Longleftrightarrow\  \hoare{\aForm}{\linV{\aCmd}}{\bForm} \]
% }

\endinput 




\subsection{Full abstraction for synchronization free threads}
Our semantics is complete for reasoning about full-thread optimizations of synchronization free programs.

In the rest of this section, we only consider commands $\aCmd,\bCmd$ that are
restriction-free, composition-free (ie. single-threaded), and
synchronization-free (ie. no acquire, release, fence).

In order to develop the proof, 

This closure permits us to  describe a normal form for the top-level pomsets that arise in single-threaded and synchronization free code.  
\begin{definition}
$\aPS$ is in normal form if:
\begin{itemize}
% \item All preconditions on events are tautologies.
\item If $\bEv \lt \aEv$ and  $\bEv$ is a write, then $\aEv$ is a write or read on the same variable.
\item If $\bEv \lt \aEv$ and  $\aEv$ is a read, then $\bEv$ is a write on the same variable. 
    
% \item $\aEv \gtN \bEv$ only if $ \aEv\ (\lt \cup \reco)^{\star} \  \bEv$.
\item If $\aEv, \aEv'$ have the same read action label, then there exists a write $\bEv$ on the same location such that $\aEv \gtN \bEv \gtN \aEv'$.
%\item  If $\aEv, \aEv'$ have the same write action label, then 
%there exists a event $\bEv$ on the same location as $\aEv$ such 
%that  $\bEv$ has a different action label and $\aEv \gtN \bEv \gtN 
%\aEv'$.
\end{itemize}
\end{definition}
In a normal form pomset, the successors of a write event (resp. the predecessors of a read) are related in the coherence order to the event.  Any two events with the same read label are separated by a write in the $\reco$ order. 

It suffices to consider normal forms when distinguishing single-threaded, synchronization free code at the top level.  The normal forms determine the full semantics by the closure properties of the semantics.

\begin{lemma}\label{unrhd}
Every top-level pomset of $\freadsat\sem{\aCmd}$ is a top-level pomset of $\freadsat\sem{\bCmd}$ if and only if 
every top-level normal form pomset of $\freadsat\sem{\aCmd}$ is a top-level normal form pomset of $\freadsat\sem{\bCmd}$.
\begin{proof}
The proof proceeds by induction on structure.  The key case is prefixing.  The only edges $\lt$-edges out of writes  and into read actions enforced by prefixing are $\reco$ edges.  Read prefixing permits the reuse of events to ensure that distinct read events not separated by $\reco$ have distinct read labels.  
\end{proof}
\end{lemma}

We develop testers for top-level pomsets in normal form.  We  follow~\citet{Plotkin:1997:TSP:266557.266600}, albeit in a concrete form appropriate to our setting.   

Let $\aPS$ be a top-level pomset in normal form with events $\aEv_1, \ldots,
\aEv_n$.  For all $i$, we assume a new location $b_i$.    Let $\freshaval$ be a fresh value that does not occur in $\aPS$.  

Let $\vec{\bEv}$ be all  the predecessors of $\aEv_i$ in $\lt$. Let $\vec{c}$ be the corresponding sub vector of $b_i$'s. 
\begin{itemize}
\item 
If $\aEv_i$ has label $\DR{\aLoc}{\aVal}$, we define a program $\testP{\aPS}{i}$ as follows:  
\[
  \testP{\aPS}{i} = (
  % \vec{\aReg} \GETS \vec{c}  \SEMI
  \IF{\vec{c}=\vec{1}} \THEN x \GETS \aVal \SEMI  \FENCE \SEMI b_i \GETS 1 \FI)
\]
\item 
If $\aEv_i$ has label $\DW{\aLoc}{\aVal}$, we define a program $\testP{\aPS}{i}$ as follows:
\[
  \testP{\aPS}{i} = (
  % \vec{\aReg} \GETS \vec{c}  \SEMI
  \IF{\vec{c}=\vec{1}} \THEN %\aReg = \aLoc \SEMI
  \IF{\aLoc = \aVal} \THEN \aLoc = \freshaval \SEMI \FENCE \SEMI b_i \GETS 1  \FI \FI)
\]
\end{itemize}
$\testP{\aPS}{i}$ is as expected, matching the reads (resp. writes) in $\aPS$ by writes (resp. reads).  The fresh value is used to flush out prior writes and reset to see fresh writes.    

\begin{definition}[Tester for $\aPS$]\label{testAPS} 
The tester context for $\aPS$, $\testP{\aPS}{}\hole{}$ is 
\[
b \GETS b_{1} \land  \cdots \land b_{n}
\PAR  \testP{\aPS}{1} 
\PAR   \cdots 
\PAR  \testP{\aPS}{n} 
\PAR \hole{}
\]
\end{definition}
The constraints of the normal form ensure that if the labels of any two events are the same, then one is a predecessor of the other. 

\begin{lemma}\label{tester}
$\sem{\testP{\aPS}{}\hole{\aCmd}}$  has a pomset that sets $b$ to $1$ iff $\aPS \in \freadsat\sem{\aCmd}$.
\begin{proof}
It suffices to prove that there is a pomset that sets $b$ to $1$ in
 $\freadsat\sem{b \GETS b_{1} \land  \cdots \land b_{n}
\PAR  \testP{\aPS}{1} 
\PAR   \cdots 
\PAR  \testP{\aPS}{n}} \parallel \bPS$ iff $\aPS$ is an augmentation of $\bPS$. 

We prove by induction on $\lt$ that $\testP{\aPS}{i}\hole{\aCmd}$  sets $b_i$ to $1$ iff the event $\aEv_i$ is enabled. Result follows.
\end{proof}
\end{lemma}

\begin{theorem}
  Let $\aCmd_1$ and $\aCmd_2$ be restriction-free, composition-free and
  synchronization free.  Then all top-level pomsets of
  $\freadsat\sem{\aCmd_1}$ are also pomsets of $\freadsat\sem{\aCmd_2}$ if
  and only if for all parallel contexts $\bCmd$, all top-level pomsets of
  $\freadsat\sem{\aCmd_1 \PAR \bCmd}$ are also pomsets of
  $\freadsat\sem{\aCmd_2 \PAR \bCmd}$
\begin{proof}
  The forward implication follows from the compositionality of the semantics.

  For the converse, pick a top level pomset in normal form in
  $\aPS \in \freadsat\sem{\aCmd_1} \setminus \freadsat\sem{\aCmd_2} $.  The
  required context is given by the tester context $\testP{\aPS}{}\hole{}$.
 \end{proof}
\end{theorem}

The full abstraction theorem applies only to full threads.  The reason for this limited statement is that our impoverished language does not permit  parallel composition to be used freely as the continuation in arbitrary sequential contexts.  A full abstraction theorem that applies also to commands can be achieved with these richer distinguishing contexts.  

\section{Understanding ``Out Of Thin Air'' using Temporal Logic}
\label{sec:logic}

A significant challenge for a software memory model is to relax order enough
to allow efficient implementation without admitting anomalous
behaviors---called \emph{out of thin air} (\oota) in the literature
\cite{vacuous,DBLP:conf/esop/BattyMNPS15,BoehmOOTA}.  The most famous example
is \ref{OOTA3} from \textsection\ref{sec:intro}.  Here we inline
initialization in order to fit the format of our proof rules:
\begin{align*}
  \label{OOTA3}\tag{\textsc{oota2}}
  y\GETS0\SEMI 
  y\GETS x
  \!\PAR\!
  x\GETS0\SEMI
  r\GETS y\SEMI x\GETS r  
  &&
  %\nonumber
  \smash{\hbox{\begin{tikzinline}[node distance=1.2em]
  \event{rx}{\DR{x}{1}}{}
  \event{wy}{\DW{y}{1}}{right=of rx}
  \po{rx}{wy}
  \event{y0}{\DW{y}{0}}{left=of rx}
  \event{x0}{\DW{x}{0}}{right=2em of wy}
  \event{ry}{\DR{y}{1}}{right=2em of x0}
  \event{wx}{\DW{x}{1}}{right=of ry}
  \po{ry}{wx}
  \rf[out=10,in=170]{wy}{ry}
  \rf[out=170,in=10]{wx}{rx}
  \wk[out=-15,in=-165]{y0}{wy}
  \wk[out=-15,in=-165]{x0}{wx}
    \end{tikzinline}}}
\end{align*}
Although Java does not allow \oota{} behaviors of \ref{OOTA3},
\citet{DBLP:journals/toplas/Lochbihler13} showed that it does allow \oota\
behaviors of \ref{OOTA1}, from \textsection\ref{sec:intro}.  In
\cite{DBLP:conf/lics/JeffreyR16}, we described a logic that rules out
\ref{OOTA3} but not \ref{OOTA1}.  In this section, we provide a more accurate
test of \oota{} behaviors by enhancing our previous logic with temporal features.

On first read, we suggest that readers skip to the examples and the
discussion that follows, coming back to the details of the logic as
necessary.  Example~\ref{ex:thin} discusses the canonical \oota{} example
\ref{OOTA3}; the analysis is trivial and well-known
\cite{DBLP:conf/lics/JeffreyR16, DBLP:conf/popl/KangHLVD17}.
Example~\ref{ex:lochb} is more interesting.  It discusses a variant of
\citeauthor{DBLP:journals/toplas/Lochbihler13}'s example \ref{OOTA1},
from the introduction.
% In this case, the violation is a subtler
% temporal property.  We develop a logic sufficient to prove that our semantics
% disallows \oota\ on \eqref{lochbihler}.  

The logic given here is not meant to be definitive; on page
\pageref{page:logic:limits}, we discuss \oota{} examples that require
non-trivial extensions
\cite{DBLP:conf/esop/SvendsenPDLV18,DBLP:journals/pacmpl/ChakrabortyV19}.

\noparagraph{Definitions}
We adapt past linear temporal logic (\pLTL)
\cite{Lichtenstein:1985:GP:648065.747612} to pomsets by dropping the previous
instant operator and adopting strict versions of the temporal operators.
The atoms of our logic are write and read events.
% \begin{displaymath}
%   \afo \QUAD::=\QUAD
%   \DR{\aLoc}{\aVal}
%   \mid
%   \DW\aLoc\aVal
%   \afo \wedge\bfo
%   \mid \lnot \afo
%   \once\afo
%   \mid \always\afo
% \end{displaymath}
%\begin{definition} %[Satisfaction]
Given a pomset $\aPS$ and event $\aEv$, define:\nofootnote{Let $\FALSE$, $\lor$,
  $\Rightarrow$ and $\once$ as usual;
  for example,
  $\once\afo = \lnot(\always\lnot\afo)$.}
\begin{displaymath}
  \renewcommand{\arraycolsep}{.2ex}
    \begin{array}{lrll}
      \aPS,\aEv &\models& \DW{\aLoc}{\aVal} &\text{ if } \labelingAct(\aEv) = \DW{\aLoc}{\aVal} \text{ and } \TRUE \text{ implies } \labelingForm(\aEv) \\
      \aPS,\aEv &\models& \DR{\aLoc}{\aVal} &\text{ if } \labelingAct(\aEv) = \DR{\aLoc}{\aVal} \text{ and } \TRUE \text{ implies } \labelingForm(\aEv) \\
      \aPS,\aEv &\models& \afo\land\bfo &\text{ if } \aPS,\aEv \models  \afo \text{ and } \aPS,\aEv \models  \bfo \\
      \aPS,\aEv &\models& \TRUE\\
      \aPS,\aEv &\models& \lnot\afo &\text{ if } \aPS,\aEv \not\models \afo \\
      \aPS,\aEv &\models& \always\afo &\text{ if } \forall \bEv \lt \aEv.\; \aPS,\bEv \models \afo\\
      \aPS,\aEv &\models& \once\afo &\text{ if } \exists \bEv \lt \aEv.\;  \aPS,\bEv \models \afo 
    \end{array} 
  \end{displaymath}

  Define $\FALSE$, $\lor$, and $\Rightarrow$ as usual.

  % \begin{definition}
  Let $\aPS \models \afo$ if
  $\aPS,\aEv \models\afo$, for all $\aEv \in \Event$.

  Let $\aPSS\models \afo$
  if $\aPS \models\afo$, for all $\aPS \in \aPSS$.
  
Let
  \begin{math}
    \afo, \aPSS \models \bfo  \text{ if } \{ \aPS \mid \aPS \models \afo \} \parallel \aPSS \models \bfo.
  \end{math}
%\end{definition}

  Let $\afo$ be \emph{downclosed} when
  $\{ \aPS \mid \aPS \models \afo \}$ is.

% Thus, $\aPS\models \afo \land \always\afo$ whenever $\aPS \models
  % \afo$. This fact relies on the use of universal quantification in the definition.

% We define other connectives as standard:
% $\once\afo = \lnot(\always\lnot\afo)$,
% %$\FALSE = \lnot(\TRUE)$
% $\afo\lor\bfo = \lnot(\lnot(\afo)\land\lnot(\bfo))$, and
% $\afo\Rightarrow\bfo = \lnot(\afo) \lor\ \bfo$.
% \begin{displaymath}
% \begin{array}{lrl}
% \once\afo &=& \lnot(\always\lnot\afo) \\
% \FALSE &=& \lnot(\TRUE) \\
% \afo\lor\bfo &=& \lnot(\lnot(\afo)\land\lnot(\bfo)) \\
% \afo\Rightarrow\bfo &=& \lnot(\afo) \lor\ \bfo
% \end{array}
% \end{displaymath}
%Let $$ be defined as $$. 
%In addition, let $\FALSE$, $\lor$ and $$ be defined in the
%standard way.
% $\afo\lor\bfo$ for $\lnot(\lnot \afo \land \lnot \bfo)$,
% and $\afo \Rightarrow \bfo$ for $\lnot \afo \lor \bfo$.
  The past operators do not include the current instant, and so
  do \emph{not} satisfy
  $(\always\afo\Rightarrow\once\afo)$. The order-minimal elements always validate
    $\always\afo$ and invalidate
    $\once\afo$.
  However, we can prove the following:
% \begin{align*}  
%   \frac{\aPS \models \afo \Rightarrow\once{\afo}}{\aPS \models \lnot \afo}\text{(Coinduction)}
%   &&
%   \frac{\aPS \models \always\afo \Rightarrow\afo}{\aPS \models \afo}\text{(Induction)}
% \end{align*}
% \begin{lemma}
% Given an pomset $\aPS$.  
\begin{align*}
  \tag{Induction}
  \aPS \models& (\always\afo \Rightarrow\afo) \Rightarrow\afo
  \\[-1ex]
  \tag{Coinduction}
  \aPS \models& (\afo \Rightarrow\once{\afo}) \Rightarrow\lnot \afo
  \\[-1ex]
  \tag{Weakening}
  \aPS \models& (\afo \Rightarrow\once{\bfo}) \Rightarrow (\once\afo \Rightarrow\once{\bfo})
\end{align*}
% \end{lemma}
% \begin{proof}
% We prove that any node in a pomset satisfies these formulas.  
%The proof for both rules proceeds by induction on the length of the maximal path from a root to a node. 
%\end{proof}

% \begin{description}
% \item[Coinduction.]
%   \begin{math}
%     (\afo \Rightarrow\once{\afo}) \Rightarrow\lnot \afo
%   \end{math}
% \item[Induction.] 
%   \begin{math}
%     (\always\afo \Rightarrow\afo) \Rightarrow\afo
%   \end{math}
% \end{description}


%We now present two proof rules for programs. 

%\paragraph*{Proof rules for programs}
We present two additional proof rules. 
The first provides a logical view of \emph{$\aLoc$-closure} (Def.~\ref{def:rf}):
%The soundness proof is straightforward.
% \begin{math}
%   \closed(\aLoc) = (\DR{\aLoc}{\aVal} \Rightarrow \once \DW{\aLoc}{\aVal}).
% \end{math}
% Although this definition does not mention intervening writes, it is
% sufficient for our example.  
\begin{displaymath}
  %\tag{Closing $\aLoc$}
  \frac{
    \afo \text{ is independent of } \aLoc
    \qquad
    %\aPS \models \closed(\aLoc) \Rightarrow \afo
    \aPS \models (\DR{\aLoc}{\aVal} \Rightarrow \once \DW{\aLoc}{\aVal}) \Rightarrow \afo
  }{
    \nu \aLoc \DOT \aPS \models \afo
  }
\end{displaymath}
%It is straightforward to establish that this rule is sound.
% Although it does
% not mention intervening writes, the rule is sufficient for our examples.

The second rule describes concurrent composition, in the style of~\citet{Abadi:1993:CS:151646.151649}.  To simplify the presentation, we
consider the special case with a single invariant.
% We view the
% composition result as capturing key aspects of no-ThinAirRead, as will become
% clearer in the examples below.
% In order to state the theorem, we generalize the satisfaction relation to
% include environment assumptions.

\begin{proposition}%[Composition]
  Let $\afo$ be downclosed.  Let $\aPSS_1, \aPSS_2$ be
  augmentation\hyp{}closed. %\footnote{$\aPS'$ is an augmentation of $\aPS$ if
 %   $\Event'=\Event$, $\aEv\le\bEv$ implies $\aEv\le'\bEv$, $\aEv\gtN\bEv$
 %   implies $\aEv\gtN'\bEv$, and
 %   % $\labeling'(\aEv)=\labeling(\aEv)$
 %   if $\labeling(\aEv) = (\bForm \mid \bAct)$ then
 %   $\labeling'(\aEv) = (\bForm' \mid \bAct)$ where $\bForm'$ implies
 %   $\bForm$.}
  Then:
  \begin{displaymath}
    %\tag{Composition}
    \frac{
      \afo, \aPSS_1 \models\afo
      \qquad
      \afo, \aPSS_2 \models\afo
    }{\aPSS_1 \parallel \aPSS_2 \models \afo}
  \end{displaymath}
\end{proposition}
\begin{proof}[Proof sketch]
  We will show that all downsets in the downset closures of
  $\aPSS_1 \parallel \aPSS_2$ satisfy the required property.  Proof proceeds
  by induction on downsets of $\aPS \in \aPSS_1 \parallel \aPSS_2$.
  %
  The case for empty downset  follows from assumption that  $\afo$ is downset closed.  
  %
  For the inductive case, consider %$\aPS$ in the downset closure of $\aPSS_1 \parallel \aPSS_2$, i.e.
  $\aPS \in \aPS_1 \parallel \aPS_2$ where
  $\aPS_i \in \aPSS_i$.  Since $\aPSS_1$ and $\aPSS_2$ are augmentation
  closed, we can assume that the restriction of $\aPS$ to the events of
  $\aPS_i$ coincides with $\aPS_i$, for $i=1,2$.
  %
  Consider a downset $\aPS'$ derived by removing a maximal element $\aEv$ from
  $\aPS$.  Suppose $\aEv$ comes from $\aPS_1$ (the other case is
  symmetric). Since $\aPS_2$ is a downset of $\aPS'$ and $\aPS' \models \afo$
  by induction hypothesis, we deduce that $\aPS_2 \models \afo$.
  % Thus, $\aPS_2 \in \mods{(\afo)}$.
  Since $\aPS_1 \in \aPSS_1$, by assumption $\afo, \aPSS_1 \models\afo$ we
  deduce that $\aPS \models \afo$.
\end{proof}

% The logic is defined with respect to downclosed sets, but $\sem{\aCmd}$
% includes only completed pomsets.  For reasoning in the logic, we downclose
% the semantics, considering pomsets that may not be completed.  Let
% $\semdown{\aCmd}=\{\aPS'\mid\aPS'$ is a downset of some
% $\aPS \in \sem{\aCmd}\}$.

\begin{example}
\label{ex:thin}
\noparagraph{Basic Examples}
With all variables initialized to $0$, we show that \ref{OOTA3}
satisfies
\begin{math}
  \lnot\DW{x}{1}.
\end{math}

We start with the invariant:
\begin{displaymath}
  [\DW{x}{1}\Rightarrow\once\DR{y}{1}]
  \land
  [\DW{y}{1}\Rightarrow\once\DR{x}{1}]
\end{displaymath}
This invariant holds for each thread; thus, it holds for the
aggregate program by composition.  Closing $y$ yields
\begin{math}
  \DR{y}{1} \Rightarrow \once\DW{y}{1}.
\end{math}
Weakening the right conjunct: % yields
\begin{math}
  \once\DW{y}{1}\Rightarrow\once\DR{x}{1}.
\end{math}
Chaining these together: %yields
\begin{math}
  \DR{y}{1} \Rightarrow \once\DR{x}{1}.
\end{math}
Weakening:  %yields
\begin{math}
  \once\DR{y}{1} \allowbreak\Rightarrow \once\DR{x}{1}. 
\end{math}
Chaining into the left conjunct:  %yields
\begin{math}
  \DW{x}{1} \Rightarrow \once\DR{x}{1}. 
\end{math}
Closing $x$, 
% \begin{math}
%   \DR{x}{1} \Rightarrow \once\DW{x}{1}.
% \end{math}
weakening, 
% \begin{math}
%   \once\DR{x}{1} \Rightarrow \once\DW{x}{1}.
% \end{math}
then chaining: %, yields
\begin{math}
  \DW{x}{1} \Rightarrow \once\DW{x}{1}. 
\end{math}
By coinduction, 
\begin{math}
  \lnot\DW{x}{1}.
\end{math}
%as required.
\end{example}

The same reasoning can be applied to the control flow variant of \ref{OOTA3}
\cite[CYC]{DBLP:conf/popl/VafeiadisBCMN15}:
\begin{math}
  %\tag{\textsc{cyc}}\label{CYC}
  \IF{x}\THEN y\GETS 1 \FI \!\PAR\! \IF{y}\THEN x\GETS 1 \FI.
  % &&
  % %\nonumber
  % \hbox{\begin{tikzinline}[node distance=1.5em]
  % \event{rx}{\DR{x}{1}}{}
  % \event{wy}{\DW{y}{1}}{right=of rx}
  % \po{rx}{wy}
  % \event{ry}{\DR{y}{1}}{right=2em of wy}
  % \event{wx}{\DW{x}{1}}{right=of ry}
  % \po{ry}{wx}
  % \rf{wy}{ry}
  % \rf[out=170,in=10]{wx}{rx}
  %   \end{tikzinline}}
\end{math}
The program is data-race-free. Thus, allowing an execution that writes $1$
would violate \drfsc{}.

\begin{example}
  \label{ex:lochb}
  \noparagraph{Lochbihler's Example} The essential temporal property of
  \ref{OOTA1} is ``allocation at type \texttt{C} is preceded by a read of $1$
  from $z$.''  Because our language lacks object creation, we cannot consider
  \ref{OOTA1} directly.  Instead we study \ref{OOTA4}, which has the same
  temporal structure: ``a write of $1$ to location $a$ is preceded by a read
  of $1$ from $z$.''
% A more general principle, in the spirit of~\citet{Abadi:1993:CS:151646.151649} can be proved.  We chose the simple case of temporal invariants to illustrate the idea in a simple form.  Even this simple version has interesting consequences. 
\begin{gather*}
  \tag{\textsc{oota4}}\label{OOTA4}
  %   Z=1;
  % ||
  %   a=X; // 1
  %   Y=a;
  % ||
  %   b=Z; // 0
  %   if(b){
  %     X=1
  %   } else {
  %     c=Y; // 1
  %     X=c;
  %     W=c;
  %   }
  %     \VAR  x\GETS0\SEMI \VAR  y\GETS0\SEMI \VAR  z\GETS0\SEMI
  \begin{gathered}
  z\GETS0\SEMI
    z\GETS1
  \PAR
  y\GETS0\SEMI
    y\GETS x
  \PAR
  x\GETS0\SEMI
  a\GETS0\SEMI
  \IF{z}\THEN x\GETS1 \ELSE r\GETS y \SEMI x\GETS r \SEMI a\GETS r \FI
  \\
\hbox{\begin{tikzinline}[node distance=1.2em]
  \event{wz0}{\DW{z}{0}}{}
  \event{wz1}{\DW{z}{1}}{right=of wz0}
  \wk{wz0}{wz1}
  \event{wy0}{\DW{y}{0}}{right=2em of wz1}
  \event{rx}{\DR{x}{1}}{right=of wy0}
  \event{wy}{\DW{y}{1}}{right=of rx}
  \po{rx}{wy}
  \event{wx0}{\DW{x}{0}}{right=2em of wy}
  \event{wa0}{\DW{a}{0}}{right=of wx0}
  \event{rz}{\DR{z}{0}}{right=of wa0}
  \event{ry}{\DR{y}{1}}{right=of rz}
  \event{wx}{\DW{x}{1}}{right=of ry}
  \event{wa}{\DW{a}{1}}{right=of wx}
  \po{ry}{wx}
  \rf[out=15,in=165]{wy}{ry}
  \rf[out=-170,in=-10]{wx}{rx}
  \po[out=-18,in=-162]{rz}{wa}
  \po[out=25,in=155]{ry}{wa}
  \wk[out=25,in=155]{wy0}{wy}
  \wk[out=15,in=165]{wx0}{wx}
  \wk[out=15,in=165]{wa0}{wa}
  \rf[out=-10,in=-170]{wz0}{rz}
\end{tikzinline}}
% \hbox{\begin{tikzinline}[node distance=1.5em]
%   \event{rx}{\DR{x}{1}}{}
%   \event{wz1}{\DW{z}{1}}{left=3em of rx}
%   \event{wy}{\DW{y}{1}}{right=of rx}
%   \po{rx}{wy}
%   \event{rz}{\DR{z}{0}}{right=3em of wy}
%   \event{ry}{\DR{y}{1}}{right=of rz}
%   \event{wx}{\DW{x}{1}}{right=of ry}
%   \event{wa}{\DW{a}{1}}{right=of wx}
%   \po{ry}{wx}
%   \rf[out=15,in=165]{wy}{ry}
%   \rf[out=-170,in=-10]{wx}{rx}
%   \po[out=-10,in=-170]{rz}{wa}
%   \po[out=15,in=165]{ry}{wa}
% \end{tikzinline}}
  \end{gathered}  
  \end{gather*}
  Attempting to write $1$ to $a$ results in the cycle shown above.  Thus, the
  outcome is disallowed by our semantics.  It was also disallowed by our
  event structures model, although the logic given there is insufficient to
  establish this fact \citep[\textsection9]{DBLP:journals/lmcs/JeffreyR19}.
  The outcome is \emph{allowed} by
  \citep{DBLP:conf/popl/KangHLVD17,DBLP:conf/esop/JagadeesanPR10,DBLP:journals/pacmpl/ChakrabortyV19}.
% \begin{tikzdisplay}[node distance=1.5em]
%   \event{wy0}{\DW{y}{0}}{}
%   \event{rx}{\DR{x}{1}}{right=4.5em of wy0}
%   \event{wy}{\DW{y}{1}}{right=of rx}
%   \po{rx}{wy}
%   \wk[bend left]{wy0}{wy}
%   \event{wx0}{\DW{x}{0}}{below=of wy0}
%   \event{rz}{\DR{z}{0}}{right=of wx0}
%   \event{ry}{\DR{y}{1}}{right=of rz}
%   \event{wx}{\DW{x}{1}}{right=of ry}
%   \event{ry1}{\DR{y}{1}}{right=of wx}
%   \event{wa}{\DW{a}{1}}{right=of ry1}
%   \rf{wy}{ry1}
%   \po{ry}{wx}
%   \wk[bend right]{wx0}{wx}
%   \rf{wy}{ry}
%   \rf{wx}{rx}
%   \event{wz0}{\DW{z}{0}}{below=of wx0}
%   \event{wz1}{\DW{z}{1}}{right=of wz0}
%   \rf{wz0}{rz}
%   \wk{wz0}{wz1}
%   \po{ry1}{wa}
%   \po[bend right]{rz}{wa}
% \end{tikzdisplay}

To establish that this outcome is disallowed here, we prove 
\begin{math}
  \lnot\DW{a}{1},
\end{math}
starting with invariant:
% which holds for each of the three threads, and thus, by composition, for the
% aggregate program:
\begin{scope}
\small
\begin{align*}
  [\once\DW{y}{1} \Rightarrow \once\DR{x}{1}]
  \land
  [\notonce\DW{a}{1} \Rightarrow (\once\DR{y}{1} \land \always(\DW{x}{1} \Rightarrow \once\DR{y}{1}))]
\end{align*}
\end{scope}
Closing $y$ and chaining into the left conjunct:
% \begin{math}
%   \once\DR{y}{1} \Rightarrow \once\DW{y}{1}. % \Rightarrow \once\DR{x}{1}
% \end{math}
% Chaining this implication on the left:
\begin{math}
  \once\DR{y}{1} \Rightarrow \once\DR{x}{1}.
\end{math}
% We can weaken this to:
% \begin{math}
%   \once\DR{y}{1} \Rightarrow \once\DR{x}{1}. % \Rightarrow \once\DR{x}{1}
% \end{math}
Chaining into the right conjunct:
\begin{displaymath}
  \notonce\DW{a}{1} \Rightarrow (\once\DR{x}{1} \land \always(\DW{x}{1} \Rightarrow \once\DR{x}{1}))
\end{displaymath}
Closing $x$:
% \begin{math}
%   \once\DR{x}{1} \Rightarrow \once\DW{x}{1}.
% \end{math}
%  Weakening and chaining again:
%we can replace $\once\DR{x}{1}$ with $\once\DW{x}{1}$:
\begin{math}
  \notonce\DW{a}{1} \Rightarrow (\once\DW{x}{1} \land \always(\DW{x}{1} \Rightarrow \once\DW{x}{1}).
\end{math}
Applying coinduction to the right conjunct:
\begin{displaymath}
  \notonce\DW{a}{1} \Rightarrow (\once\DW{x}{1} \land \always(\lnot \DW{x}{1}))
\end{displaymath}
Simplifying:
\begin{math}
  \notonce\DW{a}{1} \Rightarrow \FALSE,
\end{math}
as required.
\end{example}


\begin{comment}
  \color{red} Need to sort this out.
  Alan proposes:
\begin{verbatim}
     (W y 2) => <>(R x 1)
     (W y 1) => <>(R x 0)
     (W x 1) => <>(R y 1)
   <>(W x 1) => not(<>(W x 2))  --- which should be???  <>(W x 0) => not(<>(W x 1))
\end{verbatim}

2020/09/30: This seems to go bad because of initialization...
The formula
\begin{verbatim}
<>Wx0 => not(<>Wx1)
\end{verbatim}
does not hold for
\begin{verbatim}
x=0; x=y
\end{verbatim}

2020/09/10:  I am worried about the compositionality of this predicate:
\begin{verbatim}
I think
   <>(W x 0 => not(<>(W x 1)))
holds for 
   x=0; r=y 
and
   x=1
but not
   x=0; r=y || x=1
as shown by the execution
   Wx1 < Wx0 < Ry0
\end{verbatim}
  
It is impossible to fulfill $(\DR{y}{1})$ in the following
\cite[RNG]{DBLP:conf/esop/SvendsenPDLV18}:
\begin{align*}
  \taglabel{OOTA5}
    ( y\GETS x+1
    \PAR
    x\GETS y ) && \hbox{\begin{tikzinline}[node distance=1.5em]
        \event{rx}{\DR{x}{1}}{}
        \event{wy}{\DW{y}{2}}{right=of rx}
        \po{rx}{wy}
        \event{ry}{\DR{y}{1}}{right=3em of wy}
        \event{wx}{\DW{x}{1}}{right=of ry}
        \po{ry}{wx}
        \rf[out=170,in=10]{wx}{rx}
      \end{tikzinline}}
\end{align*}
The proof proceeds as before, starting with the following invariant:
\begin{gather*}
  [\DW{y}{2} \Rightarrow \once\DR{x}{1}] \land
  [\once\DW{x}{1} \Rightarrow \once\DR{y}{1}] \land
  [\once\DW{y}{1} \Rightarrow \once\DR{x}{0}] \land
  [\once\DW{x}{0} \Rightarrow \lnot(\once\DW{x}{1})]
\end{gather*}
\begin{verbatim}
  Wy2 => <>Rx1  /\  <>Wx1 => <>Ry1  /\  <>Wy1 => <>Rx0  
close x and y                                          
  Wy2 => <>Wx1  /\  <>Wx1 => <>Wy1  /\  <>Wy1 => <>Wx0  
chain
  Wy1 => <>Wx0  
chain with <>Wx0 => not(<>Wx1)
\end{verbatim}
\end{comment}

% Many examples are superficially similar, but in fact have fewer dependencies.
% A referee for a previous version of this paper expected that the following example is
% ``the same'':
% \begin{gather*}
%   \tag{OOTA?}\label{OOTA?}
%     y\GETS x
%   \PAR
%     \IF{y}\THEN r\GETS y\SEMI x\GETS r\SEMI a\GETS r \ELSE x\GETS1 \FI
%   \\
%   \hbox{\begin{tikzinline}[node distance=1.5em]
%   \event{rx}{\DR{x}{1}}{}
%   \event{wy}{\DW{y}{1}}{right=of rx}
%   \po{rx}{wy}
%   \event{ry}{\DR{y}{1}}{right=2em of wy}
%   \event{wx}{\DW{x}{1}}{right=of ry}
%   \event{wa}{\DW{a}{1}}{right=of wx}
%   \rf[out=-15,in=-165]{wy}{ry}
%   \rf[out=170,in=10]{wx}{rx}
%   \po[out=-15,in=-165]{ry}{wa}
%     \end{tikzinline}}
% \end{gather*}
% In this execution, $\DW{x}{1}$ is independent of $\DR{y}{1}$, thus there is no
% \oota{} behavior.

Many examples are superficially similar, but in fact have fewer dependencies,
such as \eqref{OOTA?} from \textsection\ref{sec:intro}.

\noparagraph{RFUB: Register assignment From an Unexecuted Branch}
\citeauthor{BoehmOOTA}'s [\citeyear{BoehmOOTA}] \ref{RFUB} example presents
another potential form of \oota{} behavior.
% , in the context of compiler
% optimization.
Our analysis shows that there is no \oota{} behavior in
\ref{RFUB}, only a false dependency:
%\citet{BoehmOOTA} \labeltext{considers}{page:rfub} the following programs:
\begin{gather*}
  \tag{\textsc{rfub}}\label{RFUB}
  \sem{r\GETS y\SEMI x\GETS r}
  \not\supseteq
  \sem{r\GETS y\SEMI \IF{r \NOTEQ 1} \THEN z\GETS 1\SEMI r\GETS 1\FI \SEMI x\GETS r}
\end{gather*}
The left command is half of \ref{OOTA3}. %, from \textsection\ref{sec:logic}.
The right command is dubbed \rfub{}, for \emph{Register assignment From an
  Unexecuted Branch}.  \citeauthor{BoehmOOTA} observes that in the context
$x\GETS y \PAR \hole{}$, these programs have different behaviors.  Yet the
\oota{} example on the left never writes $1$.  Why should the unexecuted
branch change that?  Because of the conditional, the write to $x$ in
\ref{RFUB} is independent of the read from $y$.  It useful to considering the
Hoare logic formulas satisfied by the two threads above: we have
$\hoare{\TRUE}{\text{\rfub}}{x=1}$, but not
$\hoare{\TRUE}{\text{\oota}}{x=1}$.  The change in the thread from
\ref{OOTA3} to \ref{RFUB} is not a valid refinement under Hoare logic; as a
result, it is expected that \ref{RFUB} may have additional behaviors.

Understanding \oota{} behavior is notoriously difficult, even for the
greatest minds in the field!  % We believe that \emph{logic} is the only tool
% that can cut the horrible knot that semanticists have tied themselves in.
% Preconditions provide a \emph{natural} solution to working out these
% dependencies.
This example shows the wisdom of using existing tools, such as preconditions
and Hoare logic, to model new problems, such as relaxed memory.
% We don't
% need to abandon established ideas; we only need to adapt them!
% On page \pageref{page:rfub}, we discuss \citeauthor{BoehmOOTA}'s
% [\citeyear{BoehmOOTA}] \ref{RFUB} example, which presents another potential
% form of \oota{} behavior, in the context of compiler optimization.  Our
% analysis shows that there is no \oota{} behavior in \ref{RFUB}, instead
% \citeauthor{BoehmOOTA}'s analysis has a false dependency.

% Understanding \oota{} behavior is notoriously difficult, even for the
% greatest minds in the field!  We believe that \emph{logic} is the only tool
% that can cut the horrible knot that semanticists have tied themselves in.
% Preconditions provide a \emph{natural} solution to working out these
% dependencies.

% \endinput

% \paragraph{Load buffering and thin air.}
% The program
% \begin{math}
%   %x\GETS0\SEMI y\GETS0\SEMI
%   (y\GETS x \PAR \bReg\GETS y\SEMI x\GETS1)
% \end{math}
% has top level executions that result in the final outcome $x = y = 1$, such as:
% \begin{tikzdisplay}[node distance=1.5em]
%   % \event{wx0}{\DW{x}{0}}{}
%   % \event{wy0}{\DW{y}{0}}{below=wx0}
%   \event{rx}{\DR{x}{1}}{}
%   \event{wy}{\DW{y}{1}}{right=of rx}
%   \po{rx}{wy}
%   \event{ry}{\DR{y}{1}}{right=3em of wy}
%   \event{wx}{\DW{x}{1}}{right=of ry}
%   \rf{wy}{ry}
%   \rf[out=170,in=10]{wx}{rx}
%   %\po{rx}{wy}
% \end{tikzdisplay}
% In \textsection\ref{sec:logic} we provide machinery to prove that this
% outcome is impossible if there is order from read to write in both
% threads.  This order can be achieved by replacing the second thread
% \begin{math}
%   (\bReg\GETS y\SEMI x\GETS1)
% \end{math}
% with 
% \begin{math}
%   (\bReg\GETS y\ACQ\SEMI x\GETS1)
% \end{math}
% or
% \begin{math}
%   (\IF{y}\THEN x\GETS 1\FI)
% \end{math}
% or
% \begin{math}
%   (x\GETS y).
% \end{math}

% A more interesting example is the following variant of \eqref{types}:
% \begin{displaymath}
%   %\label{OOTA4}
%   % x\GETS0\SEMI
%   %y\GETS0\SEMI   
%   (
%     y\GETS x
%   \PAR
%     \IF{z}\THEN x\GETS1 \ELSE x\GETS y\SEMI a\GETS y \FI
%   \PAR
%     z\GETS0\SEMI z\GETS1
%   )
% \end{displaymath}
% This program is allowed to write $1$ to $a$ under many speculative
% memory models
% \cite{Manson:2005:JMM:1047659.1040336,DBLP:conf/esop/JagadeesanPR10,DBLP:conf/popl/KangHLVD17},
% even though the read of $1$ from $y$ in the else branch of the second
% thread arises out of thin air.   \citet{DBLP:journals/toplas/Lochbihler13}
% argues that such executions compromise type safety unless object allocation
% partitions memory by type.
% In our model, the attempted execution is:
% \begin{tikzdisplay}[node distance=1.5em]
%   \event{rx}{\DR{x}{1}}{}
%   \event{wy}{\DW{y}{1}}{below=of rx}
%   \po{rx}{wy}
%   \event{ry}{\DR{y}{1}}{right=of rx}
%   \event{wx}{\DW{x}{1}}{below=of ry}
%   \po{ry}{wx}
%   \rf{wy}{ry}
%   \rf{wx}{rx}
%   \event{rz}{\DR{z}{0}}{right=of ry}
%   \event{wz0}{\DW{z}{0}}{right=of rz}
%   \rf{wz0}{rz}
%   \event{wz1}{\DW{z}{1}}{right=of wz0}
%   \wk{wz0}{wz1}
%   \event{ry1}{\DR{y}{1}}{below=of rz}
%   \rf[bend right]{wy}{ry1}
%   \event{wa}{\DW{a}{1}}{right=of ry1}
%   \po{ry1}{wa}
%   \po{rz}{wa}
% \end{tikzdisplay}
% This is forbidden by the evident cycle.


% \begin{verbatim}



% y=x+1; a=y || x=y
% prove a!=2

% Wyv_1 /\ Wyv_2 => v_1 == v_2 (and maybe v_1==0 \/ v_2==0)
% Wx1 => <>-1 Ry1
% Wy1 => <>-1 Rx1
% \end{verbatim}

% Local Variables:
% mode: latex
% TeX-master: "paper"
% End:

\section{Examples}
\label{sec:examples}

\subsection{Blockers}
\label{app:blockers}

Unfortunately by itself, this is not enough. The problem is the final
clause saying that there does not exist an $\aLoc$-\emph{blocking}
event $\cEv$ between $\bEv$ and $\aEv$. Unfortunately, concurrency can
turn events that were not $\aLoc$-blockers into an $\aLoc$-blocker,
\emph{even if the new thread does not mention $\aLoc$.}
We give an example to show this in \refapp{blockers}.
This is a problem in that it means the preliminary model violates
\emph{scope extrusion}~\cite{Milner:1999:CMS:329902},
in that we can find programs $\aCmd$ and $\bCmd$ such that
$\sem{\VAR\aLoc\SEMI(\aCmd\PAR\bCmd)}$ is not the same as
$\sem{(\VAR\aLoc\SEMI\aCmd)\PAR\bCmd}$, even if $\bCmd$ does not mention~$\aLoc$.

Recall the preliminary definition of reads-from in \S\ref{sec:pomsets}, which
defined an $\aLoc$-blocker to be and event $\cEv$ that writes to $\aLoc$ such that
$\bEv \lt \cEv \lt \aEv$.  Were we to adopt this definition, then concurrent
threads could turn events that were not $\aLoc$-blockers into an
$\aLoc$-blocker, even if the new thread does not mention $\aLoc$.

To see this, consider the program
\begin{math}
  (
  \aLoc\GETS1\SEMI
  \bLoc\GETS\aLoc
  \PAR
  \aLoc\GETS\cLoc+1\SEMI
  \bLoc\GETS\aLoc
  \PAR
  \IF{z=2}\THEN\aReg\GETS\aLoc\FI
  )
\end{math}
with execution:
\begin{tikzdisplay}[node distance=1em]
  \event{wx1}{\DW{\aLoc}{1}}{}
  \event{rz1}{\DR{\cLoc}{1}}{right=of wx1}
  \event{wx2}{\DW{\aLoc}{2}}{right=of rz1}
  \event{rz2}{\DR{\cLoc}{2}}{right=of wx2}
  \event{rx1}{\DR{\aLoc}{1}}{right=of rz2}
  \event{rx1a}{\DR{\aLoc}{1}}{below=of wx1}
  \event{wy1}{\DW{\bLoc}{1}}{below=of rx1a}
  \event{rx2a}{\DR{\aLoc}{2}}{below=of wx2}
  \event{wy2}{\DW{\bLoc}{2}}{below=of rx2a}
  \rf{wx1}{rx1a}
  \po{rx1a}{wy1}
  \rf{wx2}{rx2a}
  \po{rx2a}{wy2}
  \po{rz1}{wx2}
  \po{rz2}{rx1}
  \rf[out=20,in=160]{wx1}{rx1}
\end{tikzdisplay}
and the program
\begin{math}
  (
  \cLoc\GETS\bLoc\SEMI
  \cLoc\GETS\bLoc
  )
\end{math}
with execution:
\begin{tikzdisplay}[node distance=1em]
  \event{ry1}{\DR{\bLoc}{1}}{}
  \event{wz1}{\DW{\cLoc}{1}}{right=of ry1}
  \event{ry2}{\DR{\bLoc}{2}}{right=of wz1}
  \event{wz2}{\DW{\cLoc}{2}}{right=of ry2}
  \po{ry1}{wz1}
  \po{ry2}{wz2}
\end{tikzdisplay}
If these are placed in parallel, then a possible execution is:
\begin{tikzdisplay}[node distance=1em]
  \event{wx1}{\DW{\aLoc}{1}}{}
  \event{rz1}{\DR{\cLoc}{1}}{right=of wx1}
  \event{wx2}{\DW{\aLoc}{2}}{right=of rz1}
  \event{rz2}{\DR{\cLoc}{2}}{right=of wx2}
  \event{rx1}{\DR{\aLoc}{1}}{right=of rz2}
  \event{rx1a}{\DR{\aLoc}{1}}{below=of wx1}
  \event{wy1}{\DW{\bLoc}{1}}{below=of rx1a}
  \event{rx2a}{\DR{\aLoc}{2}}{below=of wx2}
  \event{wy2}{\DW{\bLoc}{2}}{below=of rx2a}
  \rf{wx1}{rx1a}
  \po{rx1a}{wy1}
  \rf{wx2}{rx2a}
  \po{rx2a}{wy2}
  \po{rz1}{wx2}
  \po{rz2}{rx1}
  \event{ry1}{\DR{\bLoc}{1}}{below=of wy1}
  \event{wz1}{\DW{\cLoc}{1}}{right=of ry1}
  \event{ry2}{\DR{\bLoc}{2}}{below=of wy2}
  \event{wz2}{\DW{\cLoc}{2}}{right=of ry2}
  \po{ry1}{wz1}
  \po{ry2}{wz2}
  \rf{wy1}{ry1}
  \rf{wz1}{rz1}
  \rf{wy2}{ry2}
  \rf{wz2}{rz2}
\end{tikzdisplay}
and now the $(\DW{\aLoc}{2})$ event is an $\aLoc$-blocker,
so $(\DR{\aLoc}{1})$ cannot
read from $(\DW{\aLoc}{1})$.

In the final definition of reads-from in \S\ref{sec:pomsets} we
ruled out $\aLoc$-blockers by requiring that any
event $\cEv$ that writes to $\aLoc$ has
either $\cEv \gtN \bEv$ or $\aEv \gtN \cEv$.
With this definition, in order for $(\DR{\aLoc}{1})$ to read from
$(\DW{\aLoc}{1})$, we either need $(\DW{\aLoc}{2}) \gtN (\DW{\aLoc}{1})$
or $(\DR{\aLoc}{1}) \gtN (\DW{\aLoc}{2})$, for example:
\begin{tikzdisplay}[node distance=1em]
  \event{wx1}{\DW{\aLoc}{1}}{}
  \event{rz1}{\DR{\cLoc}{1}}{right=of wx1}
  \event{wx2}{\DW{\aLoc}{2}}{right=of rz1}
  \event{rz2}{\DR{\cLoc}{2}}{right=of wx2}
  \event{rx1}{\DR{\aLoc}{1}}{right=of rz2}
  \event{rx1a}{\DR{\aLoc}{1}}{below=of wx1}
  \event{wy1}{\DW{\bLoc}{1}}{below=of rx1a}
  \event{rx2a}{\DR{\aLoc}{2}}{below=of wx2}
  \event{wy2}{\DW{\bLoc}{2}}{below=of rx2a}
  \rf{wx1}{rx1a}
  \po{rx1a}{wy1}
  \rf{wx2}{rx2a}
  \po{rx2a}{wy2}
  \po{rz1}{wx2}
  \po{rz2}{rx1}
  \rf[out=20,in=160]{wx1}{rx1}
  \wk[out=-150,in=-30]{rx1}{wx2}
  \wk{wy1}{wy2}
\end{tikzdisplay}
then putting this in parallel as before results in:
\begin{tikzdisplay}[node distance=1em]
  \event{wx1}{\DW{\aLoc}{1}}{}
  \event{rz1}{\DR{\cLoc}{1}}{right=of wx1}
  \event{wx2}{\DW{\aLoc}{2}}{right=of rz1}
  \event{rz2}{\DR{\cLoc}{2}}{right=of wx2}
  \event{rx1}{\DR{\aLoc}{1}}{right=of rz2}
  \event{rx1a}{\DR{\aLoc}{1}}{below=of wx1}
  \event{wy1}{\DW{\bLoc}{1}}{below=of rx1a}
  \event{rx2a}{\DR{\aLoc}{2}}{below=of wx2}
  \event{wy2}{\DW{\bLoc}{2}}{below=of rx2a}
  \rf{wx1}{rx1a}
  \po{rx1a}{wy1}
  \rf{wx2}{rx2a}
  \po{rx2a}{wy2}
  \po{rz1}{wx2}
  \po{rz2}{rx1}
  \rf[out=20,in=160]{wx1}{rx1}
  \wk[out=-150,in=-30]{rx1}{wx2}
  \wk{wy1}{wy2}
  \event{ry1}{\DR{\bLoc}{1}}{below=of wy1}
  \event{wz1}{\DW{\cLoc}{1}}{right=of ry1}
  \event{ry2}{\DR{\bLoc}{2}}{below=of wy2}
  \event{wz2}{\DW{\cLoc}{2}}{right=of ry2}
  \po{ry1}{wz1}
  \po{ry2}{wz2}
  \rf{wy1}{ry1}
  \rf{wz1}{rz1}
  \rf{wy2}{ry2}
  \rf{wz2}{rz2}
  \wk[out=30,in=150]{wz1}{wz2}
\end{tikzdisplay}
but this is \emph{not} a valid 3-valued pomset,
since $(\DW{\aLoc}{2}) \lt (\DR{\aLoc}{1})$ but also $(\DR{\aLoc}{1}) \gtN (\DW{\aLoc}{2})$,
which is a contradiction.


%\section{Understanding ``Out Of Thin Air'' using Temporal Logic}
\label{sec:logic}

A significant challenge for a software memory model is to relax order enough
to allow efficient implementation without admitting anomalous
behaviors---called \emph{out of thin air} (\oota) in the literature
\cite{vacuous,DBLP:conf/esop/BattyMNPS15,BoehmOOTA}.  The most famous example
is \ref{OOTA3} from \textsection\ref{sec:intro}.  Here we inline
initialization in order to fit the format of our proof rules:
\begin{align*}
  \label{OOTA3}\tag{\textsc{oota2}}
  y\GETS0\SEMI 
  y\GETS x
  \!\PAR\!
  x\GETS0\SEMI
  r\GETS y\SEMI x\GETS r  
  &&
  %\nonumber
  \smash{\hbox{\begin{tikzinline}[node distance=1.2em]
  \event{rx}{\DR{x}{1}}{}
  \event{wy}{\DW{y}{1}}{right=of rx}
  \po{rx}{wy}
  \event{y0}{\DW{y}{0}}{left=of rx}
  \event{x0}{\DW{x}{0}}{right=2em of wy}
  \event{ry}{\DR{y}{1}}{right=2em of x0}
  \event{wx}{\DW{x}{1}}{right=of ry}
  \po{ry}{wx}
  \rf[out=10,in=170]{wy}{ry}
  \rf[out=170,in=10]{wx}{rx}
  \wk[out=-15,in=-165]{y0}{wy}
  \wk[out=-15,in=-165]{x0}{wx}
    \end{tikzinline}}}
\end{align*}
Although Java does not allow \oota{} behaviors of \ref{OOTA3},
\citet{DBLP:journals/toplas/Lochbihler13} showed that it does allow \oota\
behaviors of \ref{OOTA1}, from \textsection\ref{sec:intro}.  In
\cite{DBLP:conf/lics/JeffreyR16}, we described a logic that rules out
\ref{OOTA3} but not \ref{OOTA1}.  In this section, we provide a more accurate
test of \oota{} behaviors by enhancing our previous logic with temporal features.

On first read, we suggest that readers skip to the examples and the
discussion that follows, coming back to the details of the logic as
necessary.  Example~\ref{ex:thin} discusses the canonical \oota{} example
\ref{OOTA3}; the analysis is trivial and well-known
\cite{DBLP:conf/lics/JeffreyR16, DBLP:conf/popl/KangHLVD17}.
Example~\ref{ex:lochb} is more interesting.  It discusses a variant of
\citeauthor{DBLP:journals/toplas/Lochbihler13}'s example \ref{OOTA1},
from the introduction.
% In this case, the violation is a subtler
% temporal property.  We develop a logic sufficient to prove that our semantics
% disallows \oota\ on \eqref{lochbihler}.  

The logic given here is not meant to be definitive; on page
\pageref{page:logic:limits}, we discuss \oota{} examples that require
non-trivial extensions
\cite{DBLP:conf/esop/SvendsenPDLV18,DBLP:journals/pacmpl/ChakrabortyV19}.

\noparagraph{Definitions}
We adapt past linear temporal logic (\pLTL)
\cite{Lichtenstein:1985:GP:648065.747612} to pomsets by dropping the previous
instant operator and adopting strict versions of the temporal operators.
The atoms of our logic are write and read events.
% \begin{displaymath}
%   \afo \QUAD::=\QUAD
%   \DR{\aLoc}{\aVal}
%   \mid
%   \DW\aLoc\aVal
%   \afo \wedge\bfo
%   \mid \lnot \afo
%   \once\afo
%   \mid \always\afo
% \end{displaymath}
%\begin{definition} %[Satisfaction]
Given a pomset $\aPS$ and event $\aEv$, define:\nofootnote{Let $\FALSE$, $\lor$,
  $\Rightarrow$ and $\once$ as usual;
  for example,
  $\once\afo = \lnot(\always\lnot\afo)$.}
\begin{displaymath}
  \renewcommand{\arraycolsep}{.2ex}
    \begin{array}{lrll}
      \aPS,\aEv &\models& \DW{\aLoc}{\aVal} &\text{ if } \labelingAct(\aEv) = \DW{\aLoc}{\aVal} \text{ and } \TRUE \text{ implies } \labelingForm(\aEv) \\
      \aPS,\aEv &\models& \DR{\aLoc}{\aVal} &\text{ if } \labelingAct(\aEv) = \DR{\aLoc}{\aVal} \text{ and } \TRUE \text{ implies } \labelingForm(\aEv) \\
      \aPS,\aEv &\models& \afo\land\bfo &\text{ if } \aPS,\aEv \models  \afo \text{ and } \aPS,\aEv \models  \bfo \\
      \aPS,\aEv &\models& \TRUE\\
      \aPS,\aEv &\models& \lnot\afo &\text{ if } \aPS,\aEv \not\models \afo \\
      \aPS,\aEv &\models& \always\afo &\text{ if } \forall \bEv \lt \aEv.\; \aPS,\bEv \models \afo\\
      \aPS,\aEv &\models& \once\afo &\text{ if } \exists \bEv \lt \aEv.\;  \aPS,\bEv \models \afo 
    \end{array} 
  \end{displaymath}

  Define $\FALSE$, $\lor$, and $\Rightarrow$ as usual.

  % \begin{definition}
  Let $\aPS \models \afo$ if
  $\aPS,\aEv \models\afo$, for all $\aEv \in \Event$.

  Let $\aPSS\models \afo$
  if $\aPS \models\afo$, for all $\aPS \in \aPSS$.
  
Let
  \begin{math}
    \afo, \aPSS \models \bfo  \text{ if } \{ \aPS \mid \aPS \models \afo \} \parallel \aPSS \models \bfo.
  \end{math}
%\end{definition}

  Let $\afo$ be \emph{downclosed} when
  $\{ \aPS \mid \aPS \models \afo \}$ is.

% Thus, $\aPS\models \afo \land \always\afo$ whenever $\aPS \models
  % \afo$. This fact relies on the use of universal quantification in the definition.

% We define other connectives as standard:
% $\once\afo = \lnot(\always\lnot\afo)$,
% %$\FALSE = \lnot(\TRUE)$
% $\afo\lor\bfo = \lnot(\lnot(\afo)\land\lnot(\bfo))$, and
% $\afo\Rightarrow\bfo = \lnot(\afo) \lor\ \bfo$.
% \begin{displaymath}
% \begin{array}{lrl}
% \once\afo &=& \lnot(\always\lnot\afo) \\
% \FALSE &=& \lnot(\TRUE) \\
% \afo\lor\bfo &=& \lnot(\lnot(\afo)\land\lnot(\bfo)) \\
% \afo\Rightarrow\bfo &=& \lnot(\afo) \lor\ \bfo
% \end{array}
% \end{displaymath}
%Let $$ be defined as $$. 
%In addition, let $\FALSE$, $\lor$ and $$ be defined in the
%standard way.
% $\afo\lor\bfo$ for $\lnot(\lnot \afo \land \lnot \bfo)$,
% and $\afo \Rightarrow \bfo$ for $\lnot \afo \lor \bfo$.
  The past operators do not include the current instant, and so
  do \emph{not} satisfy
  $(\always\afo\Rightarrow\once\afo)$. The order-minimal elements always validate
    $\always\afo$ and invalidate
    $\once\afo$.
  However, we can prove the following:
% \begin{align*}  
%   \frac{\aPS \models \afo \Rightarrow\once{\afo}}{\aPS \models \lnot \afo}\text{(Coinduction)}
%   &&
%   \frac{\aPS \models \always\afo \Rightarrow\afo}{\aPS \models \afo}\text{(Induction)}
% \end{align*}
% \begin{lemma}
% Given an pomset $\aPS$.  
\begin{align*}
  \tag{Induction}
  \aPS \models& (\always\afo \Rightarrow\afo) \Rightarrow\afo
  \\[-1ex]
  \tag{Coinduction}
  \aPS \models& (\afo \Rightarrow\once{\afo}) \Rightarrow\lnot \afo
  \\[-1ex]
  \tag{Weakening}
  \aPS \models& (\afo \Rightarrow\once{\bfo}) \Rightarrow (\once\afo \Rightarrow\once{\bfo})
\end{align*}
% \end{lemma}
% \begin{proof}
% We prove that any node in a pomset satisfies these formulas.  
%The proof for both rules proceeds by induction on the length of the maximal path from a root to a node. 
%\end{proof}

% \begin{description}
% \item[Coinduction.]
%   \begin{math}
%     (\afo \Rightarrow\once{\afo}) \Rightarrow\lnot \afo
%   \end{math}
% \item[Induction.] 
%   \begin{math}
%     (\always\afo \Rightarrow\afo) \Rightarrow\afo
%   \end{math}
% \end{description}


%We now present two proof rules for programs. 

%\paragraph*{Proof rules for programs}
We present two additional proof rules. 
The first provides a logical view of \emph{$\aLoc$-closure} (Def.~\ref{def:rf}):
%The soundness proof is straightforward.
% \begin{math}
%   \closed(\aLoc) = (\DR{\aLoc}{\aVal} \Rightarrow \once \DW{\aLoc}{\aVal}).
% \end{math}
% Although this definition does not mention intervening writes, it is
% sufficient for our example.  
\begin{displaymath}
  %\tag{Closing $\aLoc$}
  \frac{
    \afo \text{ is independent of } \aLoc
    \qquad
    %\aPS \models \closed(\aLoc) \Rightarrow \afo
    \aPS \models (\DR{\aLoc}{\aVal} \Rightarrow \once \DW{\aLoc}{\aVal}) \Rightarrow \afo
  }{
    \nu \aLoc \DOT \aPS \models \afo
  }
\end{displaymath}
%It is straightforward to establish that this rule is sound.
% Although it does
% not mention intervening writes, the rule is sufficient for our examples.

The second rule describes concurrent composition, in the style of~\citet{Abadi:1993:CS:151646.151649}.  To simplify the presentation, we
consider the special case with a single invariant.
% We view the
% composition result as capturing key aspects of no-ThinAirRead, as will become
% clearer in the examples below.
% In order to state the theorem, we generalize the satisfaction relation to
% include environment assumptions.

\begin{proposition}%[Composition]
  Let $\afo$ be downclosed.  Let $\aPSS_1, \aPSS_2$ be
  augmentation\hyp{}closed. %\footnote{$\aPS'$ is an augmentation of $\aPS$ if
 %   $\Event'=\Event$, $\aEv\le\bEv$ implies $\aEv\le'\bEv$, $\aEv\gtN\bEv$
 %   implies $\aEv\gtN'\bEv$, and
 %   % $\labeling'(\aEv)=\labeling(\aEv)$
 %   if $\labeling(\aEv) = (\bForm \mid \bAct)$ then
 %   $\labeling'(\aEv) = (\bForm' \mid \bAct)$ where $\bForm'$ implies
 %   $\bForm$.}
  Then:
  \begin{displaymath}
    %\tag{Composition}
    \frac{
      \afo, \aPSS_1 \models\afo
      \qquad
      \afo, \aPSS_2 \models\afo
    }{\aPSS_1 \parallel \aPSS_2 \models \afo}
  \end{displaymath}
\end{proposition}
\begin{proof}[Proof sketch]
  We will show that all downsets in the downset closures of
  $\aPSS_1 \parallel \aPSS_2$ satisfy the required property.  Proof proceeds
  by induction on downsets of $\aPS \in \aPSS_1 \parallel \aPSS_2$.
  %
  The case for empty downset  follows from assumption that  $\afo$ is downset closed.  
  %
  For the inductive case, consider %$\aPS$ in the downset closure of $\aPSS_1 \parallel \aPSS_2$, i.e.
  $\aPS \in \aPS_1 \parallel \aPS_2$ where
  $\aPS_i \in \aPSS_i$.  Since $\aPSS_1$ and $\aPSS_2$ are augmentation
  closed, we can assume that the restriction of $\aPS$ to the events of
  $\aPS_i$ coincides with $\aPS_i$, for $i=1,2$.
  %
  Consider a downset $\aPS'$ derived by removing a maximal element $\aEv$ from
  $\aPS$.  Suppose $\aEv$ comes from $\aPS_1$ (the other case is
  symmetric). Since $\aPS_2$ is a downset of $\aPS'$ and $\aPS' \models \afo$
  by induction hypothesis, we deduce that $\aPS_2 \models \afo$.
  % Thus, $\aPS_2 \in \mods{(\afo)}$.
  Since $\aPS_1 \in \aPSS_1$, by assumption $\afo, \aPSS_1 \models\afo$ we
  deduce that $\aPS \models \afo$.
\end{proof}

% The logic is defined with respect to downclosed sets, but $\sem{\aCmd}$
% includes only completed pomsets.  For reasoning in the logic, we downclose
% the semantics, considering pomsets that may not be completed.  Let
% $\semdown{\aCmd}=\{\aPS'\mid\aPS'$ is a downset of some
% $\aPS \in \sem{\aCmd}\}$.

\begin{example}
\label{ex:thin}
\noparagraph{Basic Examples}
With all variables initialized to $0$, we show that \ref{OOTA3}
satisfies
\begin{math}
  \lnot\DW{x}{1}.
\end{math}

We start with the invariant:
\begin{displaymath}
  [\DW{x}{1}\Rightarrow\once\DR{y}{1}]
  \land
  [\DW{y}{1}\Rightarrow\once\DR{x}{1}]
\end{displaymath}
This invariant holds for each thread; thus, it holds for the
aggregate program by composition.  Closing $y$ yields
\begin{math}
  \DR{y}{1} \Rightarrow \once\DW{y}{1}.
\end{math}
Weakening the right conjunct: % yields
\begin{math}
  \once\DW{y}{1}\Rightarrow\once\DR{x}{1}.
\end{math}
Chaining these together: %yields
\begin{math}
  \DR{y}{1} \Rightarrow \once\DR{x}{1}.
\end{math}
Weakening:  %yields
\begin{math}
  \once\DR{y}{1} \allowbreak\Rightarrow \once\DR{x}{1}. 
\end{math}
Chaining into the left conjunct:  %yields
\begin{math}
  \DW{x}{1} \Rightarrow \once\DR{x}{1}. 
\end{math}
Closing $x$, 
% \begin{math}
%   \DR{x}{1} \Rightarrow \once\DW{x}{1}.
% \end{math}
weakening, 
% \begin{math}
%   \once\DR{x}{1} \Rightarrow \once\DW{x}{1}.
% \end{math}
then chaining: %, yields
\begin{math}
  \DW{x}{1} \Rightarrow \once\DW{x}{1}. 
\end{math}
By coinduction, 
\begin{math}
  \lnot\DW{x}{1}.
\end{math}
%as required.
\end{example}

The same reasoning can be applied to the control flow variant of \ref{OOTA3}
\cite[CYC]{DBLP:conf/popl/VafeiadisBCMN15}:
\begin{math}
  %\tag{\textsc{cyc}}\label{CYC}
  \IF{x}\THEN y\GETS 1 \FI \!\PAR\! \IF{y}\THEN x\GETS 1 \FI.
  % &&
  % %\nonumber
  % \hbox{\begin{tikzinline}[node distance=1.5em]
  % \event{rx}{\DR{x}{1}}{}
  % \event{wy}{\DW{y}{1}}{right=of rx}
  % \po{rx}{wy}
  % \event{ry}{\DR{y}{1}}{right=2em of wy}
  % \event{wx}{\DW{x}{1}}{right=of ry}
  % \po{ry}{wx}
  % \rf{wy}{ry}
  % \rf[out=170,in=10]{wx}{rx}
  %   \end{tikzinline}}
\end{math}
The program is data-race-free. Thus, allowing an execution that writes $1$
would violate \drfsc{}.

\begin{example}
  \label{ex:lochb}
  \noparagraph{Lochbihler's Example} The essential temporal property of
  \ref{OOTA1} is ``allocation at type \texttt{C} is preceded by a read of $1$
  from $z$.''  Because our language lacks object creation, we cannot consider
  \ref{OOTA1} directly.  Instead we study \ref{OOTA4}, which has the same
  temporal structure: ``a write of $1$ to location $a$ is preceded by a read
  of $1$ from $z$.''
% A more general principle, in the spirit of~\citet{Abadi:1993:CS:151646.151649} can be proved.  We chose the simple case of temporal invariants to illustrate the idea in a simple form.  Even this simple version has interesting consequences. 
\begin{gather*}
  \tag{\textsc{oota4}}\label{OOTA4}
  %   Z=1;
  % ||
  %   a=X; // 1
  %   Y=a;
  % ||
  %   b=Z; // 0
  %   if(b){
  %     X=1
  %   } else {
  %     c=Y; // 1
  %     X=c;
  %     W=c;
  %   }
  %     \VAR  x\GETS0\SEMI \VAR  y\GETS0\SEMI \VAR  z\GETS0\SEMI
  \begin{gathered}
  z\GETS0\SEMI
    z\GETS1
  \PAR
  y\GETS0\SEMI
    y\GETS x
  \PAR
  x\GETS0\SEMI
  a\GETS0\SEMI
  \IF{z}\THEN x\GETS1 \ELSE r\GETS y \SEMI x\GETS r \SEMI a\GETS r \FI
  \\
\hbox{\begin{tikzinline}[node distance=1.2em]
  \event{wz0}{\DW{z}{0}}{}
  \event{wz1}{\DW{z}{1}}{right=of wz0}
  \wk{wz0}{wz1}
  \event{wy0}{\DW{y}{0}}{right=2em of wz1}
  \event{rx}{\DR{x}{1}}{right=of wy0}
  \event{wy}{\DW{y}{1}}{right=of rx}
  \po{rx}{wy}
  \event{wx0}{\DW{x}{0}}{right=2em of wy}
  \event{wa0}{\DW{a}{0}}{right=of wx0}
  \event{rz}{\DR{z}{0}}{right=of wa0}
  \event{ry}{\DR{y}{1}}{right=of rz}
  \event{wx}{\DW{x}{1}}{right=of ry}
  \event{wa}{\DW{a}{1}}{right=of wx}
  \po{ry}{wx}
  \rf[out=15,in=165]{wy}{ry}
  \rf[out=-170,in=-10]{wx}{rx}
  \po[out=-18,in=-162]{rz}{wa}
  \po[out=25,in=155]{ry}{wa}
  \wk[out=25,in=155]{wy0}{wy}
  \wk[out=15,in=165]{wx0}{wx}
  \wk[out=15,in=165]{wa0}{wa}
  \rf[out=-10,in=-170]{wz0}{rz}
\end{tikzinline}}
% \hbox{\begin{tikzinline}[node distance=1.5em]
%   \event{rx}{\DR{x}{1}}{}
%   \event{wz1}{\DW{z}{1}}{left=3em of rx}
%   \event{wy}{\DW{y}{1}}{right=of rx}
%   \po{rx}{wy}
%   \event{rz}{\DR{z}{0}}{right=3em of wy}
%   \event{ry}{\DR{y}{1}}{right=of rz}
%   \event{wx}{\DW{x}{1}}{right=of ry}
%   \event{wa}{\DW{a}{1}}{right=of wx}
%   \po{ry}{wx}
%   \rf[out=15,in=165]{wy}{ry}
%   \rf[out=-170,in=-10]{wx}{rx}
%   \po[out=-10,in=-170]{rz}{wa}
%   \po[out=15,in=165]{ry}{wa}
% \end{tikzinline}}
  \end{gathered}  
  \end{gather*}
  Attempting to write $1$ to $a$ results in the cycle shown above.  Thus, the
  outcome is disallowed by our semantics.  It was also disallowed by our
  event structures model, although the logic given there is insufficient to
  establish this fact \citep[\textsection9]{DBLP:journals/lmcs/JeffreyR19}.
  The outcome is \emph{allowed} by
  \citep{DBLP:conf/popl/KangHLVD17,DBLP:conf/esop/JagadeesanPR10,DBLP:journals/pacmpl/ChakrabortyV19}.
% \begin{tikzdisplay}[node distance=1.5em]
%   \event{wy0}{\DW{y}{0}}{}
%   \event{rx}{\DR{x}{1}}{right=4.5em of wy0}
%   \event{wy}{\DW{y}{1}}{right=of rx}
%   \po{rx}{wy}
%   \wk[bend left]{wy0}{wy}
%   \event{wx0}{\DW{x}{0}}{below=of wy0}
%   \event{rz}{\DR{z}{0}}{right=of wx0}
%   \event{ry}{\DR{y}{1}}{right=of rz}
%   \event{wx}{\DW{x}{1}}{right=of ry}
%   \event{ry1}{\DR{y}{1}}{right=of wx}
%   \event{wa}{\DW{a}{1}}{right=of ry1}
%   \rf{wy}{ry1}
%   \po{ry}{wx}
%   \wk[bend right]{wx0}{wx}
%   \rf{wy}{ry}
%   \rf{wx}{rx}
%   \event{wz0}{\DW{z}{0}}{below=of wx0}
%   \event{wz1}{\DW{z}{1}}{right=of wz0}
%   \rf{wz0}{rz}
%   \wk{wz0}{wz1}
%   \po{ry1}{wa}
%   \po[bend right]{rz}{wa}
% \end{tikzdisplay}

To establish that this outcome is disallowed here, we prove 
\begin{math}
  \lnot\DW{a}{1},
\end{math}
starting with invariant:
% which holds for each of the three threads, and thus, by composition, for the
% aggregate program:
\begin{scope}
\small
\begin{align*}
  [\once\DW{y}{1} \Rightarrow \once\DR{x}{1}]
  \land
  [\notonce\DW{a}{1} \Rightarrow (\once\DR{y}{1} \land \always(\DW{x}{1} \Rightarrow \once\DR{y}{1}))]
\end{align*}
\end{scope}
Closing $y$ and chaining into the left conjunct:
% \begin{math}
%   \once\DR{y}{1} \Rightarrow \once\DW{y}{1}. % \Rightarrow \once\DR{x}{1}
% \end{math}
% Chaining this implication on the left:
\begin{math}
  \once\DR{y}{1} \Rightarrow \once\DR{x}{1}.
\end{math}
% We can weaken this to:
% \begin{math}
%   \once\DR{y}{1} \Rightarrow \once\DR{x}{1}. % \Rightarrow \once\DR{x}{1}
% \end{math}
Chaining into the right conjunct:
\begin{displaymath}
  \notonce\DW{a}{1} \Rightarrow (\once\DR{x}{1} \land \always(\DW{x}{1} \Rightarrow \once\DR{x}{1}))
\end{displaymath}
Closing $x$:
% \begin{math}
%   \once\DR{x}{1} \Rightarrow \once\DW{x}{1}.
% \end{math}
%  Weakening and chaining again:
%we can replace $\once\DR{x}{1}$ with $\once\DW{x}{1}$:
\begin{math}
  \notonce\DW{a}{1} \Rightarrow (\once\DW{x}{1} \land \always(\DW{x}{1} \Rightarrow \once\DW{x}{1}).
\end{math}
Applying coinduction to the right conjunct:
\begin{displaymath}
  \notonce\DW{a}{1} \Rightarrow (\once\DW{x}{1} \land \always(\lnot \DW{x}{1}))
\end{displaymath}
Simplifying:
\begin{math}
  \notonce\DW{a}{1} \Rightarrow \FALSE,
\end{math}
as required.
\end{example}


\begin{comment}
  \color{red} Need to sort this out.
  Alan proposes:
\begin{verbatim}
     (W y 2) => <>(R x 1)
     (W y 1) => <>(R x 0)
     (W x 1) => <>(R y 1)
   <>(W x 1) => not(<>(W x 2))  --- which should be???  <>(W x 0) => not(<>(W x 1))
\end{verbatim}

2020/09/30: This seems to go bad because of initialization...
The formula
\begin{verbatim}
<>Wx0 => not(<>Wx1)
\end{verbatim}
does not hold for
\begin{verbatim}
x=0; x=y
\end{verbatim}

2020/09/10:  I am worried about the compositionality of this predicate:
\begin{verbatim}
I think
   <>(W x 0 => not(<>(W x 1)))
holds for 
   x=0; r=y 
and
   x=1
but not
   x=0; r=y || x=1
as shown by the execution
   Wx1 < Wx0 < Ry0
\end{verbatim}
  
It is impossible to fulfill $(\DR{y}{1})$ in the following
\cite[RNG]{DBLP:conf/esop/SvendsenPDLV18}:
\begin{align*}
  \taglabel{OOTA5}
    ( y\GETS x+1
    \PAR
    x\GETS y ) && \hbox{\begin{tikzinline}[node distance=1.5em]
        \event{rx}{\DR{x}{1}}{}
        \event{wy}{\DW{y}{2}}{right=of rx}
        \po{rx}{wy}
        \event{ry}{\DR{y}{1}}{right=3em of wy}
        \event{wx}{\DW{x}{1}}{right=of ry}
        \po{ry}{wx}
        \rf[out=170,in=10]{wx}{rx}
      \end{tikzinline}}
\end{align*}
The proof proceeds as before, starting with the following invariant:
\begin{gather*}
  [\DW{y}{2} \Rightarrow \once\DR{x}{1}] \land
  [\once\DW{x}{1} \Rightarrow \once\DR{y}{1}] \land
  [\once\DW{y}{1} \Rightarrow \once\DR{x}{0}] \land
  [\once\DW{x}{0} \Rightarrow \lnot(\once\DW{x}{1})]
\end{gather*}
\begin{verbatim}
  Wy2 => <>Rx1  /\  <>Wx1 => <>Ry1  /\  <>Wy1 => <>Rx0  
close x and y                                          
  Wy2 => <>Wx1  /\  <>Wx1 => <>Wy1  /\  <>Wy1 => <>Wx0  
chain
  Wy1 => <>Wx0  
chain with <>Wx0 => not(<>Wx1)
\end{verbatim}
\end{comment}

% Many examples are superficially similar, but in fact have fewer dependencies.
% A referee for a previous version of this paper expected that the following example is
% ``the same'':
% \begin{gather*}
%   \tag{OOTA?}\label{OOTA?}
%     y\GETS x
%   \PAR
%     \IF{y}\THEN r\GETS y\SEMI x\GETS r\SEMI a\GETS r \ELSE x\GETS1 \FI
%   \\
%   \hbox{\begin{tikzinline}[node distance=1.5em]
%   \event{rx}{\DR{x}{1}}{}
%   \event{wy}{\DW{y}{1}}{right=of rx}
%   \po{rx}{wy}
%   \event{ry}{\DR{y}{1}}{right=2em of wy}
%   \event{wx}{\DW{x}{1}}{right=of ry}
%   \event{wa}{\DW{a}{1}}{right=of wx}
%   \rf[out=-15,in=-165]{wy}{ry}
%   \rf[out=170,in=10]{wx}{rx}
%   \po[out=-15,in=-165]{ry}{wa}
%     \end{tikzinline}}
% \end{gather*}
% In this execution, $\DW{x}{1}$ is independent of $\DR{y}{1}$, thus there is no
% \oota{} behavior.

Many examples are superficially similar, but in fact have fewer dependencies,
such as \eqref{OOTA?} from \textsection\ref{sec:intro}.

\noparagraph{RFUB: Register assignment From an Unexecuted Branch}
\citeauthor{BoehmOOTA}'s [\citeyear{BoehmOOTA}] \ref{RFUB} example presents
another potential form of \oota{} behavior.
% , in the context of compiler
% optimization.
Our analysis shows that there is no \oota{} behavior in
\ref{RFUB}, only a false dependency:
%\citet{BoehmOOTA} \labeltext{considers}{page:rfub} the following programs:
\begin{gather*}
  \tag{\textsc{rfub}}\label{RFUB}
  \sem{r\GETS y\SEMI x\GETS r}
  \not\supseteq
  \sem{r\GETS y\SEMI \IF{r \NOTEQ 1} \THEN z\GETS 1\SEMI r\GETS 1\FI \SEMI x\GETS r}
\end{gather*}
The left command is half of \ref{OOTA3}. %, from \textsection\ref{sec:logic}.
The right command is dubbed \rfub{}, for \emph{Register assignment From an
  Unexecuted Branch}.  \citeauthor{BoehmOOTA} observes that in the context
$x\GETS y \PAR \hole{}$, these programs have different behaviors.  Yet the
\oota{} example on the left never writes $1$.  Why should the unexecuted
branch change that?  Because of the conditional, the write to $x$ in
\ref{RFUB} is independent of the read from $y$.  It useful to considering the
Hoare logic formulas satisfied by the two threads above: we have
$\hoare{\TRUE}{\text{\rfub}}{x=1}$, but not
$\hoare{\TRUE}{\text{\oota}}{x=1}$.  The change in the thread from
\ref{OOTA3} to \ref{RFUB} is not a valid refinement under Hoare logic; as a
result, it is expected that \ref{RFUB} may have additional behaviors.

Understanding \oota{} behavior is notoriously difficult, even for the
greatest minds in the field!  % We believe that \emph{logic} is the only tool
% that can cut the horrible knot that semanticists have tied themselves in.
% Preconditions provide a \emph{natural} solution to working out these
% dependencies.
This example shows the wisdom of using existing tools, such as preconditions
and Hoare logic, to model new problems, such as relaxed memory.
% We don't
% need to abandon established ideas; we only need to adapt them!
% On page \pageref{page:rfub}, we discuss \citeauthor{BoehmOOTA}'s
% [\citeyear{BoehmOOTA}] \ref{RFUB} example, which presents another potential
% form of \oota{} behavior, in the context of compiler optimization.  Our
% analysis shows that there is no \oota{} behavior in \ref{RFUB}, instead
% \citeauthor{BoehmOOTA}'s analysis has a false dependency.

% Understanding \oota{} behavior is notoriously difficult, even for the
% greatest minds in the field!  We believe that \emph{logic} is the only tool
% that can cut the horrible knot that semanticists have tied themselves in.
% Preconditions provide a \emph{natural} solution to working out these
% dependencies.

% \endinput

% \paragraph{Load buffering and thin air.}
% The program
% \begin{math}
%   %x\GETS0\SEMI y\GETS0\SEMI
%   (y\GETS x \PAR \bReg\GETS y\SEMI x\GETS1)
% \end{math}
% has top level executions that result in the final outcome $x = y = 1$, such as:
% \begin{tikzdisplay}[node distance=1.5em]
%   % \event{wx0}{\DW{x}{0}}{}
%   % \event{wy0}{\DW{y}{0}}{below=wx0}
%   \event{rx}{\DR{x}{1}}{}
%   \event{wy}{\DW{y}{1}}{right=of rx}
%   \po{rx}{wy}
%   \event{ry}{\DR{y}{1}}{right=3em of wy}
%   \event{wx}{\DW{x}{1}}{right=of ry}
%   \rf{wy}{ry}
%   \rf[out=170,in=10]{wx}{rx}
%   %\po{rx}{wy}
% \end{tikzdisplay}
% In \textsection\ref{sec:logic} we provide machinery to prove that this
% outcome is impossible if there is order from read to write in both
% threads.  This order can be achieved by replacing the second thread
% \begin{math}
%   (\bReg\GETS y\SEMI x\GETS1)
% \end{math}
% with 
% \begin{math}
%   (\bReg\GETS y\ACQ\SEMI x\GETS1)
% \end{math}
% or
% \begin{math}
%   (\IF{y}\THEN x\GETS 1\FI)
% \end{math}
% or
% \begin{math}
%   (x\GETS y).
% \end{math}

% A more interesting example is the following variant of \eqref{types}:
% \begin{displaymath}
%   %\label{OOTA4}
%   % x\GETS0\SEMI
%   %y\GETS0\SEMI   
%   (
%     y\GETS x
%   \PAR
%     \IF{z}\THEN x\GETS1 \ELSE x\GETS y\SEMI a\GETS y \FI
%   \PAR
%     z\GETS0\SEMI z\GETS1
%   )
% \end{displaymath}
% This program is allowed to write $1$ to $a$ under many speculative
% memory models
% \cite{Manson:2005:JMM:1047659.1040336,DBLP:conf/esop/JagadeesanPR10,DBLP:conf/popl/KangHLVD17},
% even though the read of $1$ from $y$ in the else branch of the second
% thread arises out of thin air.   \citet{DBLP:journals/toplas/Lochbihler13}
% argues that such executions compromise type safety unless object allocation
% partitions memory by type.
% In our model, the attempted execution is:
% \begin{tikzdisplay}[node distance=1.5em]
%   \event{rx}{\DR{x}{1}}{}
%   \event{wy}{\DW{y}{1}}{below=of rx}
%   \po{rx}{wy}
%   \event{ry}{\DR{y}{1}}{right=of rx}
%   \event{wx}{\DW{x}{1}}{below=of ry}
%   \po{ry}{wx}
%   \rf{wy}{ry}
%   \rf{wx}{rx}
%   \event{rz}{\DR{z}{0}}{right=of ry}
%   \event{wz0}{\DW{z}{0}}{right=of rz}
%   \rf{wz0}{rz}
%   \event{wz1}{\DW{z}{1}}{right=of wz0}
%   \wk{wz0}{wz1}
%   \event{ry1}{\DR{y}{1}}{below=of rz}
%   \rf[bend right]{wy}{ry1}
%   \event{wa}{\DW{a}{1}}{right=of ry1}
%   \po{ry1}{wa}
%   \po{rz}{wa}
% \end{tikzdisplay}
% This is forbidden by the evident cycle.


% \begin{verbatim}



% y=x+1; a=y || x=y
% prove a!=2

% Wyv_1 /\ Wyv_2 => v_1 == v_2 (and maybe v_1==0 \/ v_2==0)
% Wx1 => <>-1 Ry1
% Wy1 => <>-1 Rx1
% \end{verbatim}

% Local Variables:
% mode: latex
% TeX-master: "paper"
% End:



% \subsection{Release/acquire synchronization}
% \label{app:ra}

% % In relaxed memory models, synchronization actions act as memory fences: that
% % is, they are a barrier to reordering memory accesses.  In this section, we
% % present a simple model of release/acquire fencing. In
% % \S\ref{sec:transactions}, we show that this can be scaled up to a model of
% % transactional memory.

% % We assume there are sets $\Rel$ and $\Acq \subseteq\Act$.  We say that
% % $\aAct$ is a \emph{release action} if $\aAct\in\Rel$ and $\aAct$ is an
% % \emph{acquire action} if $\aAct\in\Acq$.
% % In a pomset, a release event is one labeled with a release action,
% % and an acquire event is one labeled by an acquire action.
% % To give the semantics of fences, we add extra constraints
% % to Definition~\ref{def:prefix} of prefixing %$\aAct\prefix\aPSS$
% % (recalling that $\cEv$ is the %$\aAct$-labeled
% % event being introduced):
% % \begin{itemize}
% % \item $\cEv \le \aEv$ whenever $\cEv$ is an acquire event or $\aEv$ is a release event, and
% % \item if $\cEv$ is an acquire event then $\aEv$ is independent of $\aLoc$,
% %   for every $\aLoc$.
% % \end{itemize}
% % The first constraint ensures that events are ordered before a release and
% % after an acquire.  The second constraint ensures that thread-local reads do
% % not cross acquire fences.

% % We can develop a simple model of release/acquire synchronization using the
% % following actions: % we will use
% % % releasing writes and acquiring reads:
% % \begin{itemize}
% % \item $(\DWRel{\aLoc}{\aVal})$, a release action that writes $\aVal$ to $\aLoc$, and
% % \item $(\DRAcq{\aLoc}{\aVal})$, an acquire action that reads $\aVal$ from $\aLoc$.
% % \end{itemize}
% % The semantics of programs with releasing write and acquiring read are similar
% % to regular write and read, with $\DWRel\aLoc\aVal$ replacing
% % $\DW\aLoc\aVal$ and $\DRAcq\aLoc\aVal$ replacing $\DR\aLoc\aVal$:
% % \begin{eqnarray*}
% %   \sem{\REL\aLoc\GETS\aExp\SEMI \aCmd} & = & \textstyle\bigcup_\aVal\; \bigl((\aExp=\aVal) \guard (\DWRel\aLoc\aVal) \prefix \sem{\aCmd}[\aExp/\aLoc]\bigr) \\
% %   \sem{\ACQ\aReg\GETS\aLoc\SEMI \aCmd} & = & \textstyle\bigcup_\aVal\; (\DRAcq\aLoc\aVal) \prefix \sem{\aCmd}[\aLoc/\aReg]
% % \end{eqnarray*}

% To see the need for the first constraint on prefixing, consider the program:
% \[
%   \VAR x\GETS0\SEMI \VAR f\GETS0\SEMI
%   (x\GETS 1\SEMI f \REL\GETS1 \PAR \ACQ r\GETS f; s\GETS x)
% \]
% This has an execution:
% \begin{tikzdisplay}[node distance=1em]
%   \event{wx0}{\DW{x}{0}}{}
%   \event{wf0}{\DW{f}{0}}{right=of wx0}
%   \event{wx1}{\DW{x}{1}}{below=of wx0}
%   \event{wf1}{\DWRel{f}{1}}{right=of wx1}
%   \event{rf1}{\DRAcq{f}{1}}{right=2.5em of wf1}
%   \event{rx1}{\DR{x}{1}}{right=of rf1}
%   \po{wx0}{wf1}
%   \po{wf0}{wf1}
%   \po{wx1}{wf1}
%   \po{rf1}{rx1}
%   \rf{wf1}{rf1}
%   \rf[out=20,in=160]{wx1}{rx1}
%   \wk{wx0}{wx1}
% \end{tikzdisplay}
% but \emph{not}:
% \begin{tikzdisplay}[node distance=1em]
%   \event{wx0}{\DW{x}{0}}{}
%   \event{wf0}{\DW{f}{0}}{right=of wx0}
%   \event{wx1}{\DW{x}{1}}{below=of wx0}
%   \event{wf1}{\DWRel{f}{1}}{right=of wx1}
%   \event{rf1}{\DRAcq{f}{1}}{right=2.5em of wf1}
%   \event{rx0}{\DR{x}{0}}{right=of rf1}
%   \po{wx0}{wf1}
%   \po{wf0}{wf1}
%   \po{wx1}{wf1}
%   \po{rf1}{rx1}
%   \rf{wf1}{rf1}
%   \rf[out=-20,in=160]{wx0}{rx0}
%   \wk{wx0}{wx1}
% \end{tikzdisplay}
% since $(\DW x0) \gtN (\DW x1) \lt (\DR x0)$, so this pomset does not satisfy the
% requirements to be $x$-closed.
% If we replace the release
% with a plain write, then the outcome $(\DRAcq f1)$ and $(\DR x0)$ is possible:
% \begin{tikzdisplay}[node distance=1em]
%   \event{wx0}{\DW{x}{0}}{}
%   \event{wf0}{\DW{f}{0}}{right=of wx0}
%   \event{wx1}{\DW{x}{1}}{below=of wx0}
%   \event{wf1}{\DW{f}{1}}{right=of wx1}
%   \event{rf1}{\DRAcq{f}{1}}{right=2.5em of wf1}
%   \event{rx0}{\DR{x}{0}}{right=of rf1}
%   \wk{wf0}{wf1}
%   \po{rf1}{rx0}
%   \rf{wf1}{rf1}
%   \rf[out=-20,in=160]{wx0}{rx0}
%   \wk{wx0}{wx1}
% \end{tikzdisplay}
% since no order is required between $(\DW x1)$ and $(\DW f1)$.  
% Symmetrically, if we replace the acquire of the original program
% with a plain read, then the outcome $(\DR f1)$ and $(\DR x0)$ is possible.
% % \begin{verbatim}
% %   x := 0; rel f := 0; ||
% %   acq r := f; if (r == 0) { x := x+1; rel f := 1; } ||
% %   acq s := f; if (r == 1) { x := x+1; rel f := 2; }
% % \end{verbatim}
% % This has an execution:
% % \begin{tikzdisplay}[node distance=1em]
% %   \event{wx0}{\DW{x}{0}}{}
% %   \event{wf0}{\DWRel{f}{0}}{below=of wx0}
% %   \event{rf0}{\DRAcq{f}{0}}{right=2.5 em of wx0}
% %   \event{rx0}{\DR{x}{0}}{below=of rf0}
% %   \event{wx1}{\DW{x}{1}}{below=of rx0}
% %   \event{wf1}{\DWRel{f}{1}}{below=of wx1}
% %   \event{rf1}{\DRAcq{f}{1}}{right=2.5 em of rf0}
% %   \event{rx1}{\DR{x}{1}}{below=of rf1}
% %   \event{wx2}{\DW{x}{2}}{below=of rx1}
% %   \event{wf2}{\DWRel{f}{2}}{below=of wx2}
% %   \po{wx0}{wf0}
% %   \po{rf0}{rx0}
% %   \po{rx0}{wx1}
% %   \po{wx1}{wf1}
% %   \po{rf1}{rx1}
% %   \po{rx1}{wx2}
% %   \po{wx2}{wf2}
% %   \rf{wf0}{rf0}
% %   \rf{wx0}{rx0}
% %   \rf{wf1}{rf1}
% %   \rf{wx1}{rx1}
% % \end{tikzdisplay}
% % but \emph{not}:
% % \begin{tikzdisplay}[node distance=1em]
% %   \event{wx0}{\DW{x}{0}}{}
% %   \event{wf0}{\DWRel{f}{0}}{below=of wx0}
% %   \event{rf0}{\DRAcq{f}{0}}{right=2.5 em of wx0}
% %   \event{rx0}{\DR{x}{0}}{below=of rf0}
% %   \event{wx1}{\DW{x}{1}}{below=of rx0}
% %   \event{wf1}{\DWRel{f}{1}}{below=of wx1}
% %   \event{rf1}{\DRAcq{f}{1}}{right=2.5 em of rf0}
% %   \event{rx0b}{\DR{x}{0}}{below=of rf1}
% %   \event{wx1b}{\DW{x}{1}}{below=of rx0b}
% %   \event{wf2}{\DWRel{f}{2}}{below=of wx1b}
% %   \po{wx0}{wf0}
% %   \po{rf0}{rx0}
% %   \po{rx0}{wx1}
% %   \po{wx1}{wf1}
% %   \po{rf1}{rx0b}
% %   \po{rx0b}{wx1b}
% %   \po{wx1b}{wf2}
% %   \rf{wf0}{rf0}
% %   \rf{wx0}{rx0}
% %   \rf{wf1}{rf1}
% %   \rf{wx0}{rx0b}
% % \end{tikzdisplay}
% % since $(\DW x0) \lt (\DW x1) \lt (\DR x0)$, so this pomset does not satisfy the
% % requirements to be an rf-pomset.

% % The notion rf-pomset is sufficient to capture hardware models and
% % release/acquire access in C++, where reads-from implies happens-before
% % \cite{alglave}.  To model C++ relaxed access, it
% % would be necessary to use a more general notion of rf-pomset, where
% % $(\bEv,\aLoc,\aEv) \in \RF$ does not necessarily imply $\bEv \lt \aEv$, instead
% % requiring that $(\mathord\lt \cup \mathord\RF)$ be acyclic.

% %% To see the need for the second constraint on prefixing, consider the program:
% %% \begin{displaymath}
% %%   (
% %%   x\GETS1\SEMI
% %%   \REL f\GETS 1\SEMI
% %%   \ACQ r\GETS f\SEMI
% %%   y\GETS x
% %%   )
% %%   \PAR
% %%   (
% %%   \ACQ s\GETS f\SEMI
% %%   x\GETS2\SEMI
% %%   \REL f\GETS 2\SEMI
% %%   )
% %% \end{displaymath}
% %% whose semantics includes execution:
% %% \begin{displaymath}
% %% \begin{tikzpicture}[node distance=1em]
% %%   \event{wx1}{\DW{x}{1}}{}
% %%   \event{wf1}{\DWRel{f}{1}}{right=of wx1}
% %%   \event{rf1}{\DRAcq{f}{2}}{below=of wf1}
% %%   \event{wx2}{\DW{x}{2}}{right=of rf1}
% %%   \event{wf2}{\DWRel{f}{1}}{right=of wx2}
% %%   \event{rf2}{\DRAcq{f}{2}}{above=of wf2}
% %%   \event{wy1}{\DW{y}{1}}{right=of rf2}
% %%   \po{wx1}{wf1}
% %%   \rf{wf1}{rf1}
% %%   \po{rf1}{wx2}
% %%   \po{wx2}{wf2}
% %%   \rf{wf2}{rf2}
% %%   \po{rf2}{wy1}
% %% \end{tikzpicture}
% %% \end{displaymath}
% %% This execution exists because
% %% \begin{math}
% %%   \sem{y\GETS x}
% %% \end{math}
% %% includes
% %% \begin{math}
% %%   (x=1\mid \DW{y}{1})
% %% \end{math}
% %% and the precondition $x=1$ is fulfilled by the preceding write $x\GETS1$.  In
% %% implementation term, this execution is reading $1$ from $x$ in a ``stale
% %% cache.''  The alternative execution that attempts to read $1$ from the $x$ in
% %% ``main memory,'' has an explicit $(\DR{x}{1})$ between $(\DRAcq{f}{2})$ and
% %% $(\DW{y}{1})$, and thus will fail to be $x$-closed.

% %% To prevent thread-local writes from crossing release/acquire pairs, we
% %% require that pomsets in the semantics of acquire have no free locations.
% %% This corresponds to the idea that acquires flush the read cache, and
% %% therefore reads must reload values from main memory after an acquire.

% % In addition, we must change the semantics of write from
% % \S\ref{sec:sets-of-pomsets} to ensure that an action is generated for every
% % write that might be published by a subsequent release action.
% % Formally, $\sem{\aLoc\GETS\aExp\SEMI \aCmd}$ only includes pomsets
% % from $\sem{\aCmd}[\aExp/\aLoc]$ that contain a write to
% % $\aLoc$ that is not preceded by a release.


\subsection{Relaxed memory}
\label{sec:relaxed-memory}

% In \S\ref{sec:info-flow-attack} we presented an information flow attack
% on relaxed memory, similar to Spectre in that it relies on speculative
% evaluation. Unlike Spectre it does not depend on timing attacks,
% but instead is based on the sensitivity of relaxed memory to data
% dependencies. % For this reason, we present a simple model of relaxed
% % memory, which is strong enough to capture this attack.

Our model includes concurrent memory accesses, which can introduce concurrent
reads-from. 
Since we are allowing events to be partially ordered, this gives a simple
model of relaxed memory.

This model does not introduce thin-air reads (TAR).
For example the TAR pit
\((
  x\GETS y \PAR y \GETS x
)\)
fails to produce a value for $x$ from thin air
since this produces a cycle in $\le$, as shown on the left below:
\begin{align*}
\begin{tikzpicture}[node distance=1em]
  \event{ry42}{\DR{y}{42}}{}
  \event{wx42}{\DW{x}{42}}{below=of ry42}
  \event{rx42}{\DR{x}{42}}{right=2.5em of ry42}
  \event{wy42}{\DW{y}{42}}{below=of rx42}
  \po{ry42}{wx42}
  \po{rx42}{wy42}
  \rf{wx42}{rx42}
  \rf{wy42}{ry42}
\end{tikzpicture}
&&
\begin{tikzpicture}[node distance=1em]
  \event{ry1}{\DR{y}{1}}{}
  \event{wx1}{\DW{x}{1}}{below=of ry1}
  \event{rx1}{\DR{x}{1}}{right=2.5em of ry1}
  \event{wy1}{\DW{y}{1}}{below=of rx1}
  \po{ry1}{wx1}
  \rf{wx1}{rx1}
  \rf{wy1}{ry1}
\end{tikzpicture}
\end{align*}
This cycle can be broken by removing a dependency. For example
\((
  x\GETS y \PAR r\GETS x\SEMI y \GETS r+1-r
)\)
has the execution on the right above.
% \begin{tikzdisplay}[node distance=1em]
%   \event{ry1}{\DR{y}{1}}{}
%   \event{wx1}{\DW{x}{1}}{below=of ry1}
%   \event{rx1}{\DR{x}{1}}{right=2.5em of ry1}
%   \event{wy1}{\DW{y}{1}}{below=of rx1}
%   \po{ry1}{wx1}
%   \rf{wx1}{rx1}
%   \rf{wy1}{ry1}
% \end{tikzdisplay}
Note that $(\DR x1) \not\le (\DW y1)$, so this does not introduce a cycle.

Although it is not the primary focus of this paper, our model may be an
attractive model of relaxed memory.  Many prior models either permit
thin-air executions that our model forbids or forbid desirable executions
that our model permits.
%% In \S\ref{sec:logic}, we develop a logic which allows us to prove that our
%% semantics forbids thin air examples that are permitted by prior speculative
%% models
%% \cite{Manson:2005:JMM:1047659.1040336,Jagadeesan:2010:GOS:2175486.2175503,DBLP:conf/popl/KangHLVD17}.
% Our model passes all of the causality test cases
% \cite{PughWebsite}.
%% Significantly, this
%% includes test case 9, which is forbidden by \cite{DBLP:conf/lics/JeffreyR16},
%% one of the few models that disallows the thin air example from
%% \S\ref{sec:logic}.  We present this test case in the appendix, where we also
%% discuss the thread inlining examples from
%% \cite{Manson:2005:JMM:1047659.1040336}.

% In \refapp{logic}, we present a variant of the TAR-pit
% example %from \S\ref{sec:relaxed-memory}
% that is allowed under prior speculative semantics
% \cite{Manson:2005:JMM:1047659.1040336,Jagadeesan:2010:GOS:2175486.2175503,DBLP:conf/popl/KangHLVD17}.
% We develop a logic that allows us to prove that the problematic execution is
% forbidden in our model.  \citet{DBLP:conf/esop/BattyMNPS15} showed that the
% thin-air problem has no per-candidate-execution solution for C++.  This
% result does not apply to our model, which has a different notion of
% dependency.

% as the semantics of a conditional can depend on the semantics
% of both branches.

\citet{PughWebsite} developed a set of twenty {causality test cases} in the
process of revising the Java Memory Model (JMM)
\cite{Manson:2005:JMM:1047659.1040336}.  Using hand calculation, we have
confirmed that our model gives the desired result for all twenty cases,
unrolling loops as necessary.  Our model also gives the desired results for
all of the examples in \citet[\textsection 4]{DBLP:conf/esop/BattyMNPS15} and
all but one in \citet[\textsection 5.3]{SevcikThesis}:
redundant-write-after-read-elimination fails for any
sensible non-coherent semantics.  Our model agrees with the JMM on the
``surprising and controversial behaviors'' of \citet[\textsection
8]{Manson:2005:JMM:1047659.1040336}, and thus fails to validate thread
inlining.
% In \refapp{tc}, we discuss three of the causality test cases and the thread
% inlining example from \cite{Manson:2005:JMM:1047659.1040336}.%  In presenting the
% examples, we unroll loops, correct typos and simplify the code.  

% \subsection{Causality test cases}
% \label{app:tc}

% \citet{PughWebsite} developed a set of twenty {causality test cases} in the
% process of revising the Java Memory Model (JMM)
% \cite{Manson:2005:JMM:1047659.1040336}.  Using hand calculation, we have
% confirmed that our model gives the desired result for all twenty cases,
% unrolling loops as necessary.  Our model also gives the desired results for
% all of the examples in \citet[\textsection 4]{DBLP:conf/esop/BattyMNPS15} and
% all but one in \citet[\textsection 5.3]{SevcikThesis}:
% redundant-write-after-read-elimination fails for any
% sensible non-coherent semantics.  Our model agrees with the JMM on the
% ``surprising and controversial behaviors'' of \citet[\textsection
% 8]{Manson:2005:JMM:1047659.1040336}, and thus fails to validate thread
% inlining.

\subsection{Causality test cases}
\label{app:tc}

\citet{PughWebsite} developed a set of twenty {causality test cases} in the
process of revising the Java Memory Model (JMM)
\cite{Manson:2005:JMM:1047659.1040336}.  Using hand calculation, we have
confirmed that our model gives the desired result for all twenty cases,
unrolling loops as necessary.  Our model also gives the desired results for
all of the examples in \citet[\textsection 4]{DBLP:conf/esop/BattyMNPS15} and
all but one in \citet[\textsection 5.3]{SevcikThesis}:
redundant-write-after-read-elimination fails for any
sensible non-coherent semantics.  Our model agrees with the JMM on the
``surprising and controversial behaviors'' of \citet[\textsection
8]{Manson:2005:JMM:1047659.1040336}, and thus fails to validate thread
inlining.

In this section, we discuss three of the causality test cases and the thread
inlining from \cite{Manson:2005:JMM:1047659.1040336}.  In presenting the
examples, we unroll loops, correct typos and simplify the code.  

\subsubsection{Causality test case 8}

Test case 8 asks whether:
\begin{displaymath}
  \VAR x\GETS 0\SEMI
  \VAR y\GETS 0\SEMI
  (\IF{x<2}\THEN y\GETS 1\FI 
  \PAR
  x\GETS y)
\end{displaymath}
may read $1$ for both $x$ and $y$.  This behavior is allowed, since
``interthread analysis could determine that $x$ and $y$ are always either $0$
or $1$.''  This breaks the dependency between the read of $x$ and the write
to $y$ in the first thread, allowing the write to be moved earlier.

The semantics of TC8 includes
\begin{tikzdisplay}[node distance=1em]
  \event{ix}{\DW{x}{0}}{}
  \event{iy}{\DW{y}{0}}{right=of ix}
  \event{rx1}{\DR{x}{1}}{right=2.1em of iy}
  \event{wy1}{\DW{y}{1}}{right=of rx1}
  \event{ry1}{\DR{y}{1}}{right=2.1em of wy1}
  \event{wx1}{\DW{x}{1}}{right=of ry1}
  \po{ry1}{wx1}
  \po[out=30,in=150]{ix}{rx1}
  \rf[in=-25,out=-160]{wx1}{rx1}
  \rf[out=20,in=160]{wy1}{ry1}
  \wk[out=-25,in=-150]{ix}{wx1}
  \wk[out=25,in=155]{iy}{wy1}
\end{tikzdisplay}
Where we require $(\DW{x}{0})\lt(\DR{x}{1})$ but not $(\DR{x}{1})\lt(\DW{y}{1})$.
To see why this execution exists, consider the left thread with syntax sugar
removed:
\begin{displaymath}
  r\GETS x\SEMI \IF{r<2}\THEN y\GETS 1\FI
\end{displaymath}
\begin{math}
  \sem{\IF{r<2}\THEN y\GETS 1\FI}
\end{math}
includes
\begin{math}
  (r<2\mid\DW{y}{1}).
\end{math}
% \begin{tikzdisplay}[node distance=1em]
%   \event{wy1}{r<2\mid\DW{y}{1}}{}
% \end{tikzdisplay}
Thus, by Figure~\ref{fig:programs}, 
\begin{math}
  \sem{r\GETS x\SEMI \IF{r<2}\THEN y\GETS 1\FI}
\end{math}
includes
\begin{math}
  (\DR{x}{1}) \prefix (r<2\mid\DW{y}{1})[x/r]
\end{math}
which simplifies to
\begin{math}
  (\DR{x}{1}) \prefix (x<2\mid\DW{y}{1}),
\end{math}
which, by Definition~\ref{def:prefix}, includes:
\begin{tikzdisplay}[node distance=1em,baselinecenter]
    \event{rx1}{\DR{x}{1}}{}
    \event{wy1}{x<2\mid\DW{y}{1}}{right=of rx1}
  \end{tikzdisplay}
Here we have used the \textsc{[non-ordering read]} clause of Definition~\ref{def:prefix}:
``$\bForm'$ implies $\bForm[\aVal/\aLoc] \land \bForm$, if $\aAct$ reads $\aVal$ from $\aLoc$,''
where $a=(\DR{x}{1})$,  $\bForm=\bForm'=(x<2)$.  We can use this case since
$x<2$ implies $1<2\land x<2$.

Prefixing with $(\DW{x}{0})$ allows us to discharge the assumption $x<2$,
arriving at:
\begin{tikzdisplay}[node distance=1em,baselinecenter]
    \event{ix}{\DW{x}{0}}{}
    \event{rx1}{\DR{x}{1}}{right=2.5 em of ix}
    \event{wy1}{\DW{y}{1}}{right=of rx1}
    \po{ix}{rx1}
  \end{tikzdisplay}
Here we have used the \textsc{[ordering read]}
clause of \ref{def:prefix}:
``$\bForm'$ implies $\bForm[\aVal/\aLoc]$, if $\aAct$ reads $\aVal$ from $\aLoc$ and $\cEv\lt'\aEv$,''
where $a=(\DW{x}{0})$,  $\bForm=(x<2)$ and $\bForm'=\TRUE$.  As long as
require
\begin{math}
  (\DW{x}{0})\lt
  (\DR{x}{1}),
\end{math}
we can use this case since $\TRUE$ implies $0<2$.

\subsubsection{Causality test case 9}

Test case 9 asks whether:
\begin{displaymath}
  \VAR x\GETS 0\SEMI
  \VAR y\GETS 0\SEMI
  (\IF{x<2}\THEN y\GETS 1\FI 
  \PAR
  x\GETS y
  \PAR
  y\GETS 2\SEMI)
\end{displaymath}
may read $1$ for both $x$ and $y$.  This behavior is also allowed.  This is
``similar to test case $8$, except that $x$ is not always $0$ or
$1$. However, a compiler might determine that the read of $x$ by thread $1$
will never see the write by thread $3$ (perhaps because thread $3$ will be
scheduled after thread $1$)''

Reasoning as for test case 8, the semantics of test case 9 includes:
\begin{tikzdisplay}[node distance=1em]
  \event{ix}{\DW{x}{0}}{}
  \event{iy}{\DW{y}{0}}{right=of ix}
  \event{rx1}{\DR{x}{1}}{right=2.2 em of iy}
  \event{wy1}{\DW{y}{1}}{right=of rx1}
  \event{ry1}{\DR{y}{1}}{right=2.2em of wy1}
  \event{wx1}{\DW{x}{1}}{right=of ry1}
  \event{wx2}{\DW{x}{2}}{below=3ex of $(ix)!0.5!(iy)$}%{right=2.5em of wx1}
  \po{ry1}{wx1}
  \po[out=30,in=150]{ix}{rx1}
  \rf[in=-25,out=-160]{wx1}{rx1}
  \rf[out=20,in=160]{wy1}{ry1}
  \wk[out=-25,in=-150]{ix}{wx1}
  \wk[out=25,in=155]{iy}{wy1}
  \wk{ix}{wx2}
  % \wk[out=-25,in=-150]{ix}{wx2}
\end{tikzdisplay}

Thus, with respect to the introduction of new threads, our model appears to
be more robust than the event structures semantics of
\cite{DBLP:conf/lics/JeffreyR16}, which fails on this test case.

\subsubsection{Causality test case 14}

Test case 14 asks whether:
\begin{multline*}
  \VAR a\GETS 0\SEMI
  \VAR b\GETS 0\SEMI
  \VAR y\GETS 0\SEMI\\[-.5ex]
  (\IF{a}\THEN b\GETS 1\ELSE y\GETS 1\FI 
  \PAR\\[-.5ex]
  \WHILE(y+b==0) \THEN\SKIP\FI\; a\GETS1)
\end{multline*}
may read $1$ for $a$ and $b$, yet $0$ for $y$.  Here $a$ and $b$ are regular
variables and $y$ is volatile, which is equivalent to release/acquire in this
example.  This behavior is also disallowed, since ``in all sequentially
consistent executions, [the read of $a$ gets $0$] and the program is
correctly synchronized. Since the program is correctly synchronized in all SC
executions, no non-SC behaviors are allowed.''

Unrolling the loop once, we have:
\begin{multline*}
  \VAR a\GETS 0\SEMI
  \VAR b\GETS 0\SEMI
  \VAR y\GETS 0\SEMI\\[-.5ex]
  (\IF{a}\THEN b\GETS 1\ELSE y\GETS 1\FI 
  \PAR\\[-.5ex]
  \IF{y\lor b}\THEN a\GETS 1\FI)
\end{multline*}
We argue that any execution with $(\DR{a}{1})$, $(\DR{b}{1})$, and
$(\DR{y}{0})$ must be cyclic.  The closure requirements require that
\begin{math}
  (\DW{a}{1})\lt(\DR{a}{1})
  \;\text{and}\;
  (\DR{b}{1})\lt(\DR{b}{1}).
\end{math}
Ignoring initialization, least ordered execution that includes all of these
actions is:
\begin{tikzdisplay}[node distance=1em]
  \event{ra1}{\DR{a}{1}}{}
  \event{wb1}{\DW{b}{1}}{below=of ra1}
  \nonevent{wy1}{\DW{y}{1}}{left=of wb1}
  \event{rb1}{\DR{b}{1}}{right=4.5em of ra1}
  \event{ry0}{\DR{y}{0}}{right=of rb1}
  \event{wa1}{\DW{a}{1}}{below=of rb1}
  \po{ra1}{wb1}
  \po{rb1}{wa1}
  \rf{wa1}{ra1}
  \rf{wb1}{rb1}
\end{tikzdisplay}
where the read of $a$ is ordering for $(\DW{b}{1})$ but
not $(\DW{y}{1})$, and the read of $b$ is ordering for $(\DW{a}{1})$ but the
read of $y$ is not.  $(\DW{y}{1})$ is crossed out, since its
precondition must imply $(\lnot a)[1/a]$, which is equivalent to $\FALSE$.
To avoid order from $(\DR{y}{0})$ to $(\DW{a}{1})$, we
have strengthened the predicate on $(\DW{a}{1})$ from $(y\lor b)$ to
$(y=0\land b=1)$.  Note that we cannot use this trick symmetrically to remove
the order from $(\DR{b}{1})$ to $(\DW{a}{1})$, since $b=1$ does not follow
from the initialization of $b$.


\subsubsection{Thread inlining}

One property one could ask of a model of shared memory is thread
inlining: any execution of $\sem{P\SEMI Q}$ is an execution of $\sem{P
  \PAR Q}$. This is \emph{not} a goal of our model, and indeed is not
satisfied, due to the different semantics of concurrent and sequential
memory accesses. We demonstrate this by considering an example from
the Java Memory Model~\cite{Manson:2005:JMM:1047659.1040336}, which shows that Java does not
satisfy thread inlining either.

The lack of thread inlining is related to the different dependency
relations introduced by sequential and concurrent access.
Recall from \S\ref{sec:sequential-memory} that the program
\verb`(x := 0; y := x+1;)` has execution:
\begin{tikzdisplay}[node distance=1em]
  \event{wx0}{\DW{x}{0}}{}
  \event{wy1}{\DW{y}{1}}{right=of wx0}
\end{tikzdisplay}
but that \verb`(x := 1; || y := x+1;)` has:
\begin{tikzdisplay}[node distance=1em]
  \event{wx1}{\DW{x}{1}}{}
  \event{rx1}{\DR{x}{1}}{right=2.5em of wx1}
  \event{wy2}{\DW{y}{2}}{right=of rx1}
  \rf{wx1}{rx1}
  \po{rx1}{wy2}
\end{tikzdisplay}
That is, in the sequential case there is no dependency from the
write of $x$ to the write of $y$, but in the concurrent case there
is such a dependency.

This can be used to construct a counter-example to thread inlining, based on~\cite[Ex~11]{Manson:2005:JMM:1047659.1040336}:
\begin{verbatim}
  x := 0; if (x == 1) { z := 1; } else { x := 1; } || y := x; || x := y;
\end{verbatim}
This has no execution containing $(\DW z1)$. Any attempt to build such an execution
results in a cycle:
\begin{tikzdisplay}[node distance=1em]
  \event{rx1a}{\DR{x}{1}}{}
  \event{wz1}{\DW{z}{1}}{right=of rx1a}
  \nonevent{wx1a}{\DW{x}{1}}{right=of wz1}
  \event{rx1b}{\DR{x}{1}}{below=of wx1a}%{right=2.5em of wx1a}
  \event{wy1}{\DW{y}{1}}{right=of rx1b}
  \event{ry1}{\DR{y}{1}}{right=2.5em of wy1}
  \event{wx1b}{\DW{x}{1}}{right=of ry1}
  \po{rx1a}{wz1}
  \po[out=25, in=150]{rx1a}{wx1a}
  \po{rx1b}{wy1}
  \po{ry1}{wx1b}
  \rf{wy1}{ry1}
  \rf[out=160, in=30]{wx1b}{rx1a}
  \rf[out=160, in=30]{wx1b}{rx1b}
\end{tikzdisplay}
Inlining the thread \verb|(y := x)| gives~\cite[Ex~12]{Manson:2005:JMM:1047659.1040336}:
\begin{verbatim}
  x := 0; if (x == 1) { z := 1; } else { x := 1; } y := x; || x := y;
\end{verbatim}
with execution:
\begin{tikzdisplay}[node distance=1em]
  \event{rx1a}{\DR{x}{1}}{}
  \event{wz1}{\DW{z}{1}}{right=of rx1a}
  \nonevent{wx1a}{\DW{x}{1}}{right=of wz1}
  \event{wy1}{\DW{y}{1}}{right=of wx1a}
  \event{ry1}{\DR{y}{1}}{right=2.5em of wy1}
  \event{wx1b}{\DW{x}{1}}{right=of ry1}
  \po{rx1a}{wz1}
  \po[out=25, in=150]{rx1a}{wx1a}
  \po{ry1}{wx1b}
  \rf{wy1}{ry1}
  \rf[out=160, in=30]{wx1b}{rx1a}
\end{tikzdisplay}
To see why this execution exists, consider the program fragment:
\begin{verbatim}
  if (x == 1) { z := 1; } else { x := 1; } y := x;
\end{verbatim}
Removing the syntax sugar, this is:
\begin{verbatim}
  r1 := x; if (r1 == 1) {
    z := 1; r2 := x; y := r2; skip
  } else {
    x := 1; r3 := x; y := r3; skip
  }
\end{verbatim}
Now, $\sem{z := 1\SEMI r_2 := x\SEMI y := r_2\SEMI \SKIP}$
includes pomset:
\begin{tikzdisplay}[node distance=1em]
  \event{wz1}{r_1=1 \mid \DW{z}{1}}{}
  \event{wy1}{r_1=x=1 \mid \DW{y}{1}}{right=of wz1}
\end{tikzdisplay}
and $\sem{x := 1\SEMI r_3 := x\SEMI y := r_3\SEMI \SKIP}$
includes pomset:
\begin{tikzdisplay}[node distance=1em]
  \event{wx1a}{r_1\neq 1 \mid \DW{x}{1}}{}
  \event{wy1}{r_1\neq 1 \mid \DW{y}{1}}{right=of wx1a}
\end{tikzdisplay}
so  $\sem{\IF{r_1 = 1} \THEN z := 1\SEMI r_2 := x\SEMI y := r_2\SEMI \SKIP \ELSE x := 1\SEMI r_3 := x\SEMI y := r_3\SEMI \SKIP \FI}$ includes:
\begin{tikzdisplay}[node distance=1em]
  \event{wz1}{r_1=1 \mid \DW{z}{1}}{}
  \event{wx1a}{r_1\neq1 \mid \DW{x}{1}}{right=of wz1}
  \event{wy1}{(r_1=x=1) \lor (r_1\neq1) \mid \DW{y}{1}}{below=3ex of $(wz1)!0.5!(wx1a)$}
\end{tikzdisplay}
which means $\sem{\IF{r_1 = 1} \THEN z := 1\SEMI r_2 := x\SEMI y := r_2\SEMI \SKIP \ELSE x := 1\SEMI r_3 := x\SEMI y := r_3\SEMI \SKIP \FI}[x/r_1]$ includes:
\begin{tikzdisplay}[node distance=1em]
  \event{wz1}{x=1 \mid \DW{z}{1}}{}
  \event{wx1a}{x\neq1 \mid \DW{x}{1}}{right=of wz1}
  \event{wy1}{(x=x=1) \lor (x\neq1)) \mid \DW{y}{1}}{below=3ex of $(wz1)!0.5!(wx1a)$}%{right=of wx1a}
\end{tikzdisplay}
Now $(x=x=1) \lor (x\neq1)$ is a tautology, so this is just:
\begin{tikzdisplay}[node distance=1em]
  \event{wz1}{x=1 \mid \DW{z}{1}}{}
  \event{wx1a}{x\neq1 \mid \DW{x}{1}}{right=of wz1}
  \event{wy1}{\DW{y}{1}}{right=of wx1a}
\end{tikzdisplay}
and so $\sem{r_1 \GETS x\SEMI \IF{r_1 = 1} \THEN z := 1\SEMI r_2 := x\SEMI y := r_2\SEMI \SKIP \ELSE x := 1\SEMI r_3 := x\SEMI y := r_3\SEMI \SKIP \FI}$ includes:
\begin{tikzdisplay}[node distance=1em]
  \event{rx1a}{\DR{x}{1}}{}
  \event{wz1}{1=1 \mid \DW{z}{1}}{right=of rx1a}
  \event{wx1a}{1\neq1 \mid \DW{x}{1}}{right=of wz1}
  \event{wy1}{\DW{y}{1}}{right=of wx1a}
  \po{rx1a}{wz1}
  \po[out=25, in=150]{rx1a}{wx1a}
\end{tikzdisplay}
which simplifies to:
\begin{tikzdisplay}[node distance=1em]
  \event{rx1a}{\DR{x}{1}}{}
  \event{wz1}{\DW{z}{1}}{right=of rx1a}
  \nonevent{wx1a}{\DW{x}{1}}{right=of wz1}
  \event{wy1}{\DW{y}{1}}{right=of wx1a}
  \po{rx1a}{wz1}
  \po[out=25, in=150]{rx1a}{wx1a}
\end{tikzdisplay}
as required. The rest of the example is straightforward, and shows that our semantics
agrees with the JMM in not supporting thread inlining.



% \subsection{Word tearing}

% \todo{Remove this section, since it's not needed for transactions?}

% In \S\ref{sec:transactions}, we shall be considering transactional memory,
% and in \S\ref{sec:transactions} show that we can model a simplified version
% of an information flow attack on transactions. In order to model transactions,
% we need to consider actions that can write many memory locations at once,
% since this is part of the semantics of commitment. To lead up to this, we first
% consider a simpler scenario of many-location writes and reads, which is word
% tearing.

% In word tearing, a program contains a write instruction with data larger
% than the hardware word size, for example copying a byte array, or assigning
% a 64-bit float on a 32-bit architecture. For example, consider the program:
% \begin{verbatim}
%   (x := [0, 0];) || (x := [1, 1];) || (r := x;)
% \end{verbatim}
% This has executions in which the read of $x$ only reads from one of the writes,
% for example:
% \begin{tikzdisplay}[node distance=1em]
%   \event{wx00}{\DW{x}{[0,0]}}{}
%   \event{wx11}{\DW{x}{[1,1]}}{right=2.5em of wx00}
%   \event{rx00}{\DR{x}{[0,0]}}{right=2.5em of wx11}
%   \rf[out=20, in=160]{wx00}{rx00}
% \end{tikzdisplay}
% but also has executions in which the read of $x$ reads from both writes,
% for example:
% \begin{tikzdisplay}[node distance=1em]
%   \event{wx00}{\DW{x}{[0,0]}}{}
%   \event{wx11}{\DW{x}{[1,1]}}{right=2.5em of wx00}
%   \event{rx01}{\DR{x}{[0,1]}}{right=2.5em of wx11}
%   \rfx[out=20, in=160]{wx00}{x[0]}{rx01}
%   \rfx[out=-20, in=-160]{wx11}{x[1]}{rx01}
% \end{tikzdisplay}
% Word tearing can occur, for example, in Java extended floating point~\cite{jmm},
% LLVM 64-bit instructions on 32-bit hardware~\cite{llvm}, or in
% JavaScript SharedArrayBuffers~\cite{js-sab}.

% \newcommand{\rfControl}[4][]{\draw[rf,#1](#2) .. controls (#3) .. (#4);}
% \begin{tikzdisplay}[node distance=1em]
%   \event{wx0}{\DW{x}{0}}{}
%   \event{wx1}{\DW{x}{1}}{right=of wx0}
%   \event{wy0}{\DW{y}{0}}{right=2.5em of wx1}
%   \event{wy1}{\DW{y}{1}}{right=of wy0}
%   \event{rx1}{\DR{x}{1}}{right=2.5 em of wy1}
%   \event{ry0}{\DR{y}{0}}{right=of rx1}
%   \event{ry1}{\DR{y}{1}}{right=2.5 em of ry0}
%   \event{rx0}{\DR{x}{0}}{right=of ry1}
%   \rf[out=20,in=160]{wx1}{rx1}
%   \rf[out=20,in=160]{wy0}{ry0}
%   \rf[out=340,in=200]{wy1}{ry1}
%   \coordinate (a) [below=of wy1];
%   \rfControl[out=340,in=200]{wx0}{a}{rx0}
%   \wk{wx0}{wx1}
%   \wk{wy0}{wy1}
%   \po{rx1}{ry0}
%   \po{ry1}{rx0}
% \end{tikzdisplay}


% Batty section 4:
% \cite[\S4]{DBLP:conf/esop/BattyMNPS15},
% Example LB+ctrldata+ctrl-double (language must allow)
% r1=loadrlx(x) //reads 42
% if (r1 == 42)
%   storerlx(y,r1)

% r2=loadrlx(y) //reads 42
% if (r2 == 42)
%   storerlx (x,42)
% else
% storerlx (x,42)

% a:RRLX x=42 sb,dd,cd
% c:RRLX y=42 sb,cd
%   This is forbidden on hardware if compiled naively, as the architectures respect read-to-write control dependencies, but in practice compilers will collapse con- ditionals like that of the second thread, removing the control dependencies from the read of y to the writes of x and making the code identical to the previous example. As that example is allowed and observable on hardware (and we pre- sume that it would be impractical to outlaw such optimisation for C or C++), the language must also allow this execution. But this execution has a cycle in the union of reads-from and dependency, so we cannot simply exclude all those.
% Then one might hope for some other adaptation of the C/C++11 model, but the following example shows at least that there is no per-candidate-execution solution.
% Example LB+ctrldata+ctrl-single (language can and should forbid)
% r1=loadrlx(x) //reads 42 if (r1 == 42)
% storerlx (y,r1) r2=loadrlx (y) //reads 42 if (r2 == 42)
% a:RRLX x=42 sb,dd,cd
% rf
% b:WRLX y=42
% c:RRLX y=42 sb,cd
% rf
% d:WRLX x=42
% rf rf
% b:WRLX y=42 d:WRLX x=42
%   storerlx (x,42)

% Local Variables:
% mode: latex
% TeX-master: "paper"
% End:
\section{Comparison to related work}\label{sec:ldrf}
A memory consistency model for a shared-memory multiprocessor defines the a read may return.  There has been extensive research on hardware and software memory models.~\citet{DBLP:journals/pacmpl/PodkopaevLV19} is a recent attempt to provide formal foundations to bridge the gap between the two.  \citet{AlglaveThesis} provides a historical and conceptual perspective on hardware memory models.  This paper focusses on software memory models.   For a survey and history of the Java Memory Model, we refer the reader to\citet{DBLP:journals/toplas/Lochbihler13}; for a survey and history of C11, we refer the reader to~\citet{DBLP:phd/ethos/Batty15}.  

This paper follows a line of work on relaxed memory models in the framework of true concurrency~\cite{DBLP:conf/lics/JeffreyR16,Pichon-Pharabod:2016:CSR:2837614.2837616,DBLP:conf/esop/CenciarelliKS07}.  

We discuss in more detail the relationship with the most closely related work.  

\subsection{Memory models via program transformations}
We discuss prior approaches to definining memory models using program transformations in chronological order.  The general flavor in these approaches is to consider SC executions, albeit with the threads subject to sequential program transformation  \citet{Saraswat:2007:TMM:1229428.1229469} aims to describe a JMM model this way.  DRF holds, but it was later discoeverd that there is \oota\ behavior.

~\cite{DBLP:conf/esop/FerreiraFS10} describes a relaxed memory model as being parametrised by available program transformations.  They show DRF, but do not study \oota, relate to concrete models, or demonstrate compilation results.

~\citet{DBLP:conf/popl/DemangeLZJPV13} studies BMM, a model whose only permitted reordering is of a relaxed write with a following relaxed read.  The paper develops an axiomatic and studies effects on compilation.  By design, it is designed as a restriction of the JMM that eschews several compiler optimizations. 

~\citet{DBLP:conf/fm/LahavV16} characterizes \tso\ as being derived by considering Write-Read (WR) reordering and Read-After-Write (RaW) elimination.  They also show that the release acquires of C11 are less expressive than considering WR, RaW and thread-inlining.  Our paper is inspired by their implicit challenge, ``Some memory models can be defined via transformations.
 But there is more to weak memory than transformations''.  

\subsection{Modeling the invalidation of load buffering} 

In this subsection, we argue that our paper is relevant even to the reader who desires a semantics that invalidates load buffering.  

There are two related lines of research that suggest to prohibit the reordering of loads with following stores.  
\begin{itemize}
\item 
~\citet{Dolan:2018:BDR:3192366.3192421} imposes this restriction in order to ``bound data races in space and time''.  Thus, in their model, one can reason sequentially about the the assignment to $w$ and deduce that $x=1 \Rightarrow\ z= 1$ despite the concurrent races on $u$, and  the past and future races on $y$\footnote{Concurrent, past and future stated wrt assignment to $w$.}.   In this program $r,s$ are registers, and $x,y,z,u,v,w$ are shared variables.
\begin{align*}
\mbox{Thread 1: } & u \GETS 0 \SEMI y \GETS 1 \SEMI v^{rel}  \GETS 1 \SEMI z \GETS 2 \SEMI r \GETS u \SEMI y \GETS 3 \SEMI \\[-.5ex]
\mbox{Thread 2: } & y \GETS 2 \SEMI  \IF{v^{acq}==1} \THEN\ w \GETS 1 + y -y \SEMI u \GETS 1 \SEMI x \GETS 1  \FI \SEMI s \GETS u \SEMI  y \GETS 2 \SEMI
\end{align*}
\item ~\citet{BoehmOOTA} describes a series of ``\oota'' like examples, and uses this restriction to rule out example~\ref{rfub}.  
\end{itemize}

This restriction can be captured naturally in our model.  We make two changes.
\begin{itemize}
\item Modify the semantic rule for read to remove the possibility of an internal read:
  \begin{align*}
    \sem{\aReg\GETS\REF{\cExp}\SEMI \aCmd} & =
    \textstyle\bigcup_{\aLoc=\REF{\bVal}}\; ({\cExp}=\bVal) \guard \textstyle\bigcup_\aVal\; (\DR\aLoc\aVal) \prefix \sem{\aCmd}[\aLoc/\aReg] 
  \end{align*}
\item Modify item \ref{pre-read}b in the definition of prefixing to require
  order from a read to subsequent write:
  \begin{enumerate}
  \item[(\ref{pre-read}b$'$)] if $\aEv$ is a write then  $\cEv\lt'\aEv$ 
  \end{enumerate}
\end{itemize}
In \ref{pre-read}b$'$ we removed the ``or or $\labelingForm'(\aEv)$ implies
$\labelingForm(\aEv)$.''

While~\citet{Dolan:2018:BDR:3192366.3192421} already describes DRF and  the compilation to \armeight, our approach yields the following new insights.
\begin{itemize}
\item Compositional treatment of temporal invariants from\textsection\ref{sec:logic} holds mutatis mutandis, since all executions of the restricted model are already accounted for in the general model.

\item With regards to single-threaded optimizations in \textsection\ref{sec:opt}, our approach provides different methods to prove optimizations, that complement the extant methods of~\citet{Dolan:2018:BDR:3192366.3192421} with data sensitivity; for example, our methods provides a simple proof of  the transformation of $
\IF{\aExp}\THEN\ x \GETS 1 \SEMI \aCmd \ELSE x\GETS 1 \SEMI \bCmd\FI$ to 
$x \GETS 1 \SEMI \IF{\aExp}\THEN\  \aCmd \ELSE  \bCmd\FI$.

\end{itemize}

\subsection{Revisiting \oota}
~\citet{DBLP:conf/esop/BattyMNPS15} describe the problem of thin-air executions.~\citet{BoehmOOTA} revisits community discussions on ``\oota'' like examples.  We classify the models on the basis of their behavior on three examples, namely examples~\eqref{oota1}, \eqref{types}, \eqref{alan} and \eqref{rfub}.  
\begin{itemize}
\item RC11\cite{DBLP:conf/pldi/LahavVKHD17}, JMM~\cite{Manson:2005:JMM:1047659.1040336} and the related models of~\citet{DBLP:conf/esop/JagadeesanPR10} and \citet{DBLP:conf/popl/KangHLVD17} forbid only \eqref{oota1}.
\item The model of this paper forbids \eqref{oota1}, \eqref{types} and \eqref{alan}.
\item Forbidding load buffering as per \citep{Dolan:2018:BDR:3192366.3192421,BoehmOOTA} forbids all four examples.  RC11\cite{DBLP:conf/pldi/LahavVKHD17} also forbids all four examples. 
\end{itemize}

One of our novel contributions to this line of research is the desideratum of compositional reasoning of safety properties. The The presence of such compositional reasoning suffices to remove the undesirable effects of \oota, eg. type safety.  And while \oota\ in a model is an elusive creature, and hence hard to prove or disprove,  the support of a model for a compositional proof principle is provable or falsifiable.  

\subsection{Relationship to models of speculation}
 ~\citet{2019-sp} is a close formal cousin to our approach.   In the spectrum from micro architectures to architectures to software,whereas~\citet{2019-sp} lies between the micro and architectural levels, we aim to be as abstract as possible to bridge the gap to programming languages.  This is reflected in the several differences in the formal model.  

We consider a single global acyclic order in contrast to the two studied in~\citet{2019-sp}.  We do not track false branches of conditionals, as seen in our monotonicity axiom, whereby any event with a precondition $\FALSE$ can only have $\FALSE$ successors in $\lt$, whereas in that paper, specualtion on false branches can lead to observable effects.   We are also less accurate  about tracking reads, as reflected in the removal of program order between reads and reads.   

~\citet{2019-sp} does not study the properties as a memory model.  We conjecture that our proofs of DRF and compositional reasoning can be carried over to that setting.~\citet{2019-sp} also preserves some $\rpox$ between reads, whereas we do not; so, we certainly support more compiler optimizations of single threaded code. 




\section{Conclusions}
\label{sec:outro}

% Local Variables:
% mode: latex
% TeX-master: "paper"
% End:
\begin{small}
\bibliography{bib}
\end{small}
\newpage
\appendix

\section{Three valued pomsets}
In order to perform a sharper analysis of dependency, we present an alternate semantics using three valued pomsets defined below.
\begin{definition}
  A \emph{\tvalpom} is a tuple
  $(\Event, {\sle}, {\gtN},
  \labeling)$, such that
  \begin{itemize}
   \item $(\Event, {\gtN},
  \labeling)$ is a pomset, and 
\item ${\sle} \subseteq {\gtN}$ is a partial order. 
  \end{itemize}
\end{definition}
We write $\bEv\slt\aEv$ when $\bEv\sle\aEv$ and $\bEv\neq\aEv$, and similarly for $\gtN$.   Thus, $(\sle \cup \reco)^{\star} \subseteq \geN$.  

\citet{DBLP:conf/esop/HuthJS01} call $\slt$ a ``must transition''
and $\geN$ a ``may transition''--- we have used  the terms ``strong order'' and ``weak order'' respectively.  In pictures, we have adopted
\citeauthor{DBLP:journals/dc/Lamport86}'s notation, drawing $\slt$ as a solid arrow ``$\xpo$'' and $\geN$ dashed ``$\xwk$''.  The intuitive temporal meaning of $ \aEv \lt \bEv$ is that $\aEv$ {\em must} strictly precede $\bEv$, whereas $ \aEv \gtN \bEv$ is intended to connote that $\aEv$ may precede $\bEv$. Thus, if  $ \neg (\aEv \gtN \bEv)$, $\aEv$ cannot precede $\bEv$.  

These intuitions motivate the definition of reads-from in this refined context.
\begin{definition}
  We say that $\bEv$ \emph{fulfills $\aEv$ on $\aLoc$} if $\bEv$ writes
  $\aVal$ to $\aLoc$, $\aEv$ reads $\aVal$ from $\aLoc$,
  \begin{itemize}
  \item $\bEv \slt \aEv$, and
  \item if an event $\cEv$ writes to $\aLoc$ then either $\cEv \gtN \bEv$ or $\aEv \gtN \cEv$.
  \end{itemize}
\end{definition}

We list out a few observations to illustrate the relationship between \tvalpom s and memory pomsets.  We are given a  \tvalpom, 
  $(\Event, {\sle}, {\gtN}, \labeling)$.  Then:
\begin{itemize}
\item $(\Event, {\gtN},\labeling)$ is a memory pomset with the same reads-from relation.  

\item Let $\reco$ be the restriction of $\gtN$ to conflicting actions on the same location.  Then, $(\Event, {\sle}, (\sle \cup \gtN)^{\star}, \labeling)$ is a \tvalpom, and $(\sle \cup \reco)^{\star} \subseteq \gtN$.
\end{itemize}

\paragraph*{Changes to definitions}
The definition of the semantics of programs using \tvalpom\ largely follows the one using memory pomsets.  We sketch the changes to definitions below.

\begin{itemize}
\item 
Augmentation has to include ${\slt}$. i.e 
$\aPS'$ is an \emph{augmentation} of $\aPS$ if $\Event'=\Event$,
  ${\labeling'}={\labeling}$, ${\sle'}\supseteq{\sle}$, and
  ${\gtN'}\supseteq{\gtN}$.

\item The definitions of substitution, restriction and the filtering operations  stay the same, with $\sle$ carried over unchanged.  For example:
\begin{definition}
Let $\aPSS\aSub$ be the set $\aPSS'$ where $\aPS'\in\aPSS'$ whenever
there is $\aPS\in\aPSS$ such that:
$\Event' = \Event$,
${\sle'} = {\sle}$, 
${\gtN'} = {\gtN}$,
and
$\labeling'(\aEv) = (\bForm\aSub \mid \aAct)$ when $\labeling(\aEv) = (\bForm \mid \aAct)$.
\end{definition}

\item In composition,we require ${\sle'}\supseteq{\sle^1}\cup{\sle^2}$

\item The changes to the definition \ref{def:prefix} of prefixing are as follows.  The key changes are that synchronization and dependency enforce $\slt$ whereas coherence only enforces $\gtN$. 
\begin{itemize}
\item ${\sle'}\supseteq{\sle}$.

\item \ref{pre-read} changes to: if $\aAct$ reads $\aVal$ from $\aLoc$ then both
  \begin{enumerate}
  \item[(\ref{pre-read}a)] $\labelingForm'(\aEv)$ implies $\labelingForm(\aEv)[\aVal/\aLoc]$, and
  \item[(\ref{pre-read}b)] if $\aEv$ is a write then either $\cEv\slt'\aEv$
    or $\labelingForm'(\aEv)$ implies $\labelingForm(\aEv)$,
  \end{enumerate}

\item ~\ref{pre-coherence} changes to:
 if $\aAct$ is a write that conflicts with $\labelingAct(\aEv)$ 
    then $\cEv \gtN' \aEv$,

\item \ref{pre-sync} if $\aAct$ is an acquire or $\labelingAct(\aEv)$ is a release then $\cEv \slt' \aEv$,$\cEv \gtN \aEv$
\end{itemize}
\end{itemize}

We use $\tsem{\aCmd}$ to stand for the \tvalpom\ semantics of $\aCmd$.  

\subsection{Generators. } \tvalpom s provide a characterization of generators from section~\ref{sec:sc}.  

Recall that \emph{generators} in the memory pomset semantics are pomsets that are minimal with respect to augmentation and implication.  These generators are induced by pomsets  that are minimal with respect to augmentation and implication in the \tvalpom\ semantics  in the following sense.  

$(\Event, {\gtN},\labeling)$ is a generator for $\sem{aCmd}$ 
if there exists  $(\Event, \slt, {\gtN},\labeling) \in \tsem{\aCmd}$ minimal wrt augmentation and implication, and  $\gtN = (\sle \cup \reco)^{\star}$.

Furthermore, any strong order that is outside of program order must be induced by a reads-from.  In the two-thread case, we can state the latter
property as follows: suppose $\aEv$ and $\bEv$ are not related by program
order and $\aEv\slt\bEv$; then there exist $\bEv'$ that reads-from $\aEv'$
such that $\aEv\xpox\aEv'$, $\bEv'\xpox\bEv$ and
$\aEv \slt \aEv' \slt \bEv' \slt \bEv$.

\subsection{Closure properties}
The fine grain analysis of dependency in the three-valued semantics allows us to establish some closure properties of the semantics of programs.  

We say that $\aPS' = \aPS\restrict{\Event'}$ when 
 $\Event' \subseteq \Event$,
 ${\labeling'} = {\labeling}\restrict{\Event'}$,   and
 ${\le'} = {\le}\restrict{\Event'}$.
% ${\gtN'} = {\gtN}\restrict{\Event'}$.

\begin{definition}
Let $(\aPS \after \aEv) = {\{ \bEv\in\Event \mid \aEv \le \bEv
  \}}$ be the set of events that follow $\aEv$ in $\aPS$.
\end{definition}

The semantics of read is ``input''-enabled, since it permits the read of any visible value.   Thus, any racy read in a program can be replaced by a read of a earlier value (wrt $\reco$), even while  the races with existing independent writes are maintained.   A canonical example to keep in mind for this lemma is the program:
\begin{align*}
\mbox{Thread 1: } &\VAR y\GETS 0 \SEMI \aReg \GETS y  \SEMI x \GETS 1  \SEMI \\[-.5ex]
\mbox{Thread 2: } &\VAR x\GETS 0 \SEMI \bReg \GETS x \SEMI y \GETS 1  \SEMI 
\end{align*}
with both registers getting value $1$ via the execution:
\begin{tikzdisplay}[node distance=1em]
\event{wy0}{\DW{y}{0}}{}
\event{ry1}{\DR{y}{1}}{right=of wy0}
\event{wx1}{\DW{x}{1}}{right=of ry1}
\event{wx0}{\DW{x}{0}}{below=of wy0}
\event{rx1}{\DR{x}{1}}{right=of wx0}
\event{wy1}{\DW{y}{1}}{right=of rx1}
\rf{wx1}{rx1}
\rf{wy1}{ry1}
\wk{wx0}{rx1}
\wk{wy0}{ry1}
\end{tikzdisplay}
The lemma constructs the execution:
\begin{tikzdisplay}[node distance=1em]
\event{wy0}{\DW{y}{0}}{}
\event{ry1}{\DR{y}{0}}{right=of wy0}
\event{wx1}{\DW{x}{1}}{right=of ry1}
\event{wx0}{\DW{x}{0}}{below=of wy0}
\event{rx1}{\DR{x}{0}}{right=of wx0}
\event{wy1}{\DW{y}{1}}{right=of rx1}
\rf{wx0}{rx1}
\rf{wy0}{ry1}
\wk{rx1}{wx1}
\wk{ry1}{wy1}
\end{tikzdisplay}


\begin{lemma}\label{inputen}
%Let $\aCmd = \vec{\aLoc}\GETS\vec{0}\SEMI \FENCE\SEMI (\aCmd^1 \PAR \cdots \PAR \aCmd^n)$.
Let $\aCmd$ be of form $\aLoc \GETS 0 \SEMI; \ldots$.
Let $\aPS \in \tsem{\aCmd}$ be a top level pomset.  
Let $\aEv \in \aPS$ read from write event $\bEv$  on $\aLoc$, where $\bEv$ is not the initializing write for $\aLoc$.  Let $\neg(\bEv \xhb \aEv)$.
Then, there exists $\bPS \in \tsem{\aCmd}$ such that:
\begin{itemize}
%\item $(\exists \aEv' \in \Event_{\bPS})$ such that $
%\Event_{\bPS}$ is the disjoint union of  $\Event_{\aPS} \setminus  
%(\aPS \after \aEv))$ and $(\bPS \after \aEv')$.
\item $\aEv'$ reads from $\aLoc$, with matching write event $\bEv'$, such that $\bEv' \xeco \bEv$ in $\bPS$
\item The restriction of $\sle$  in $\aPS$ to $\Event_{\aPS} \setminus  (\aPS \after \aEv)$ agrees with the the restriction of $\sle$ in $\bPS$ to $\Event_{\bPS} \setminus  (\aPS \after \aEv)$  in  $\bPS$.  
\item The restriction of $\le$  in $\aPS$ to $\Event_{\aPS} \setminus  (\aPS \after \aEv)$ agrees with the the restriction of $\le$ in $\bPS$ to $\Event_{\bPS} \setminus  (\aPS \after \aEv)$  in  $\bPS$.  
\end{itemize}
\end{lemma}
\begin{proof}
The form of $\aCmd$ ensures that there is always a write to $\aLoc$ that is related by $\xhb$ to any read.  Thus, there is at least one other write than can satisfy the read recorded as  $\aEv$.  

The key observation behind the proof is that change in a  prefixing read action can only affect the events that are dependent, ie. in the $\slt$ order to the read action.  
\end{proof}


In the following lemma,  invert the $\reco$ relationship between a read and a write.   A canonical example to keep in mind for this lemma is the program:
\begin{align*}
\mbox{Thread 1: } &\VAR y\GETS 0 \SEMI   x \GETS 1  \SEMI \aReg \GETS y  \SEMI \\[-.5ex]
\mbox{Thread 2: } &\VAR x\GETS 0 \SEMI  y \GETS 1  \SEMI  \bReg \GETS x \SEMI
\end{align*}
with both registers getting value $0$ via the execution:
\begin{tikzdisplay}[node distance=1em]
\event{wy0}{\DW{y}{0}}{}
\event{wx1}{\DW{x}{1}}{right=of wy0}
\event{ry0}{\DR{y}{0}}{right=of wx1}
\event{wx0}{\DW{x}{0}}{below=of wy0}
\event{wy1}{\DW{y}{1}}{right=of wx0}
\event{rx0}{\DR{x}{0}}{right=of wy1}
\rf[bend right]{wx0}{rx0}
\rf[bend left]{wy0}{ry0}
\wk{rx0}{wx1}
\wk{ry0}{wy1}
\wk{wx0}{wx1}
\wk{wy0}{wy1}
\end{tikzdisplay}
The lemma constructs the execution:
\begin{tikzdisplay}[node distance=1em]
\event{wy0}{\DW{y}{0}}{}
\event{wx1}{\DW{x}{1}}{right=of wy0}
\event{ry0}{\DR{y}{1}}{right=of wx1}
\event{wx0}{\DW{x}{0}}{below=of wy0}
\event{wy1}{\DW{y}{1}}{right=of wx0}
\event{rx0}{\DR{x}{1}}{right=of wy1}
\rf{wx1}{rx0}
\rf{wy1}{ry0}
\wk{wx0}{wx1}
\wk{wy0}{wy1}
\end{tikzdisplay}

\begin{lemma}\label{removerw}
Let $\aPS \in \tsem{\aCmd}$.   
Let $\bEv \in \aPS$ be a write on $\aLoc$. 
Let $\aEv \in \aPS$ read from $\aLoc$ such that $\aEv \xeco \bEv$ and $\neg(\aEv \slt \bEv)$.  Then, there exists $\bPS \in \tsem{\aCmd}$ such that:
\begin{itemize}
\item $\aEv' \in \bPS \setminus \aPS$ reads from $\aLoc$, with matching write $\bEv$.
\item The restriction of $\sle$ in $\aPS$ to $\Event_{\aPS} \setminus (\aPS\ \after\ \aEv)$ agrees with the the restriction of $\sle$ in $\bPS$ to $\Event_{\bPS} \setminus  (\aPS\ \after\ \aEv)$.  
\end{itemize}
\end{lemma}
\begin{proof}
The proof proceeds similar to the above proof; in this case, replace the value read in $\aEv$ to come from $\bEv$.  
\end{proof}
Any new  event $\bEv'$ in $\bPS \after \aEv'$ reading from $\aLoc$ cannot have a matching write event $\bEv'' \xeco \bEv$ since that  implies $\bEv' \xeco \bEv$ and a $\reco$ cycle $\bEv \slt \aEv \slt \aEv' \xeco \bEv$.  Thus, the above lemma can be iterated if the new pomset is has any further reads that precede $\bEv$ in $\reco$, so we can finally derive a pomset with no reads and writes satisfying the hypothesis of the lemma.   



The $\reco$ order between writes that are not related by $\lt$ can be reversed. 
A canonical example to keep in mind for this lemma is the program:
\begin{align*}
\mbox{Thread 1: } &\VAR x\GETS 1 \\[-.5ex]
\mbox{Thread 2: } &\VAR x\GETS 0 
\end{align*}
\begin{tikzdisplay}[node distance=1em]
\event{wy0}{\DW{x}{1}}{}
\event{wx0}{\DW{x}{0}}{below=of wx0}
\wk{wy0}{wx0}
\end{tikzdisplay}
The lemma constructs the execution:
\begin{tikzdisplay}[node distance=1em]
\event{wy0}{\DW{x}{1}}{}
\event{wx0}{\DW{x}{0}}{below=of wx0}
\wk{wx0}{wy0}
\end{tikzdisplay}
\begin{lemma}\label{cohww}
Let $\aPS \in \tsem{\aCmd}$.  Let $\bEv, \aEv$ be a writes to $\aLoc$ such that:
\begin{itemize}
\item $\bEv\gtN \aEv$  
\item forall writes $\cEv$ to $\aLoc$ such that  $ \bEv \gtN \cEv \gtN  \aEv$,  it is the case that  $ \neg(\cEv \slt \aEv)$ and $\neg(\cEv \xpox \aEv)$
\end{itemize}

Then, there exists $\bPS \in \tsem{\aCmd}$ such that $\Event_{\aPS} = \Event_{\bPS}$, $\sle_{\aPS} = \sle_{\bPS}$, and 
$\aEv \gtN \bEv$ in $\bPS$. 
\end{lemma}
\begin{proof}
We show how to interchange $\aEv, \bEv$ adjacent in $\gtN$, ie. we assume that  $\neg(\exists \cEv) \  \bEv \gtN \cEv \gtN \aEv$.  The full proof follows by induction.

Since  $\sem{\aCmd}$ is augmentation closed, it suffices to show that we can build $\bPS$ while satisfying the constraints between $\slt,\gtN$.  We list the changes below.
\begin{itemize}
\item $\aEv \gtN \bEv$ in $\bPS$
\item Forall reads $\cEv$ matched to $\aEv$, change from $\bEv \gtN \cEv$ in $\aPS$ to $\cEv \gtN \bEv$ in $\bPS$
\item Forall reads $\cEv$ matched to $\bEv$, change from $\cEv \gtN \aEv$ in $\aPS$ to $\aEv \gtN \cEv$ in $\bPS$
\end{itemize}

\end{proof}

\section{Proof of DRF}\label{drfproof}

In this section of the appendix, we develop a proof of DRF for \tvalpom s.  By the results in the earlier section, it yields DRF for the memory model pomset semantics, since the races are identical in both models.

In the rest of this section, we assume that $\aPS$ is a generator for
$\tsem{\aCmd}$.

We prove:
\begin{description}
\item[DRF1: ] If $\aPS$ does not have a race, $\aPS \in \tsemsc{\aCmd}$. 
\item[DRF2: ] If $\aPS$ has a race, then there exists $\bPS\in \tsemClosed{\aCmd}$ such that $\bPS \in \tsemsc{\aCmd}$ and has a race.
\end{description}

\paragraph*{Proof of DRF1}
We first show that if $\aPS \in \tsem{\aCmd} \setminus \tsemsc{\aCmd}$, then $\aPS$ has a race.  By assumption, there is a cycle in  $\rpox \cup \slt \cup \xeco$.  Let this cycle be $\aEv_0, \aEv'_0, \aEv_1, \aEv'_1, \ldots, \aEv_n, \aEv'_n, \aEv_0$ where for all $i$, $\aEv_i \xpox \aEv'_i$ and $\aEv'_i  \not\xpox \aEv'_{i+1}$.
If for all $i$, $\aEv'_i  \xhb \aEv'_{i+1}$, then the above is a cycle in $\rhb$, which is a contradiction.
So, there is at least one $i$ such that $\aEv'_i  \not\xhb \aEv'_{i+1}$.  There are two cases to consider.
\begin{itemize}
\item $\aEv'_i  \xeco \aEv'_{i+1}$.   In this case, there is a race.
\item  $\aEv'_i  \slt \aEv'_{i+1}$.  In this case, $\aEv'_i$ is a write and $\aEv'_{i+1}$ is a conflicting read, so there is a race. 
\end{itemize}


\paragraph*{Proof of DRF2}

We define a size $|\aPS|$ as follows: $\size(\aPS)$ is the number of events in $\aPS$.    Since we are considering loop free programs, there is an $\aPS \in \tsemsc{\aCmd}$ with maximum size, which we identify as $\size(\aCmd)$.  

We prove by induction on $\size(\aCmd) - \size(\bPS)$ that given $(\aPS, \bPS)$ such that:
\begin{itemize}
\item $\bPS$ is a prefix of some $\aPS' \in \tsemsc{\aCmd}$
\item $\bPS$ is a prefix of $\aPS$ under all of $\xpox,\gtN,\lt$ 
\item $\aPS$ has a race
\end{itemize}
there exists $\bPS\in \tsem{\aCmd}$ that demonstrates the race.

The required theorem follows by setting $\bPS$ to be the empty pomset.

For the base case, $\bPS = |\aPS|$.  In this case, $\aPS$ is the required witness.

Otherwise, consider a maximal sequential prefix, extending $\bPS$, wrt to all of  $\rpox,\reco,\slt$.  If it strictly contains $\bPS$, result follows from induction hypothesis.  

If not, $\bPS$ is already maximal.  Consider the set of all events in $\aPS \setminus \bPS$ that are minimal wrt $\rhb$.  In particular, these events will also be minimal wrt $\rpox$.  

If one of these events, say $\aEv$  is a write, we proceed as follows.   Using $\rhb$-minimality of $\aEv$, we deduce $\rpox$ minimality of $\aEv$.  Using the generator properties, we deduce that $\aEv$ is $\slt$-minimal .  Using lemma~\ref{removerw}, we build $\aPS_1$ from $\aPS$ without changing $\bPS$ to ensure that there are is no read $\bEv \in \aPS_1 \setminus \bPS$ such that $\bEv \xeco \aEv$.  Using lemma~\ref{cohww}, we build $\aPS_2$ from $\aPS_1$ without changing $\bPS$ to ensure that there are is no write $\bEv \in \aPS_2 \setminus \bPS$ such that $\bEv \xeco \aEv$.  Thus, $\aEv$ is $\reco$-minimal in $\aPS_2 \setminus \bPS$.  Result follows from induction hypothesis by considering $(\aPS_2,\bPS_1)$ where $\bPS_1$ is got from $\bPS$ by adding $\aEv$.  


So, we can assume that  all events in $\aPS \setminus \bPS$, say $\aEv_0, \ldots, \aEv_n$  that are minimal wrt $\rhb$ are reads, and we have  events 
$\aEv'_0, \aEv'_1, \ldots, \aEv'_n, \aEv_0$ such that:
\[
\begin{array}{lrl}
\aEv_i \xpox\ \aEv'_i \\
\aEv'_i \  (\reco\ \cup \slt)  \ \aEv_{(i+1)\mod n}
\end{array}
\]
Let $\bEv$ be the matching write for $\aEv_{(i+1)\mod n}$. If $\bEv_i \in \bPS$bEv , then by $\reco$ prefix closure of $\bPS$, $\bEv \xeco\ \aEv'_i$ and $\aEv_{(i+1)\mod n} \reco\ \aEv'_i$, which is a contradiction to $\reco$ being a partial order per location.  So, we can assume that $\aEv'_i \  \slt  \ \aEv_{(i+1)\mod n}$. 

We proceed as follows.  We use lemma~\ref{inputen} on the  pomset $\aPS$ and read $\aEv_{(i+1)\mod n}$ and write $\aEv'_i$ to construct $\cPS$ that changes the value read in $\aEv_{j}$ to a value from $\bPS$.  $\dPS$  is derived adding the modified read yielded by lemma~\ref{inputen} to $\bPS$.  Result follows by induction hypothesis since $\dPS$ is a prefix of $\cPS$ under all of $\xpox,\lt, \reco$,  $\cPS$ has a race, and $\size(\dPS) = \size(\bPS) + 1$. 

\section{Proof of compilation for ARMv8}
\label{sec:arm:proof}

In this section, we develop the proof of correctness of compilation to \armeight.  In order to ease readability, we reproduce the definitions from the main text. 



Given a relation $R$, $R^?$ denotes reflexive closure, $R^+$ denotes
transitive closure and $R^*$ denotes reflexive and transitive closure.  Given relations $R$ and $S$, $R;S$ denotes composition.


The ARMv8 model is described using the following relations.
\begin{itemize}
\item $\IDR$, $\IDW$, $\IDAcq$, $\IDRel$: identity on reads, writes, acquires
  and releases.
% \item $\IDR$ identity on reads
% \item $\IDW$: identity on writes
% \item $\IDAcq$: identity on acquires
% \item $\IDRel$: identity on releases
\item $\IDLoc$: relates any two events that touch the same location.
\item $\rpox$: program order.
\item $\rdata$, $\rctrl$, $\raddr$: data, control and address dependencies.
\item $\rrfx$: reads-from. $\rrfx^{-1}$ relates each read to a matching write
  on the same location.
\item $\rco$: coherence, which is a total order on the writes to a single
  location.
\item ${\rfr}\eqdef{\rco};\rrfx^{-1}$: from-read, which relates reads to
  subsequent writes.
\end{itemize}
For any relation, the cross-thread subrelation is denoted by appending $e$;
the intra-thread subrelation is denoted by appending $i$.  For example,
${\rrfe}\eqdef{\rrfx}\setminus{\rpox}$ and ${\rrfi}\eqdef{\rrfx}\cap{\rpox}$.
The subrelation restriction attention to actions on the same location is
given by appending $\mathsf{loc}$.  For example, ${\rpoloc}\eqdef{\rpox}\cap{\IDLoc}$.

The ARMv8 model defines the following relations.
In our presentation, we have elided rules concerning fences and RMW operations.
\begin{align*}
  \tag{Extended coherence}
  {\reco} &\eqdef {\rrf} \cup {\rfr} \cup {\rco}
  \\
  \tag{Observed externally}
  {\robs} &\eqdef \smash{
    {\rrfe} \cup {\rfre} \cup {\rcoe}
  }
  \\
  \tag{Dependency order}
  {\rdob} &\eqdef\smash{
    ({\raddr}\cup{\rdata}); {\rrfi}^?
    \cup ({\rctrl}\cup{\rdata}); {\IDW}; {\rcoi}^?
    \cup {\raddr}; {\rpox}; {\IDW}
  }
  \\
  \tag{Barrier order}
  {\rbob} &\eqdef\smash{
    {\IDAcq}; {\rpox}
    \cup {\rpox};{\IDRel}; {\rcoi}^?
  }
  \\
  \tag{Acyclic order}
  {\rob} &\eqdef\smash{
    ({\robs} \cup {\rdob} \cup {\rbob})^+
  }
\end{align*}
\begin{definition}
  An RMW-free and fence-free execution is \emph{ARM-consistent} if
  \begin{align*}&
    \tag{\textsc{$\rrfx$-completeness}}\label{rf-comp}
    \fcodom(\rrfx)=\fdom(\rreads)
    \\[-1ex]&
    \tag{\textsc{$\rco$-totality}}\label{co-tot}
    \text{For every location $\aLoc$, $\rco$ totally orders the writes of $\aLoc$}  
    \\[-1ex]&
    \tag{\textsc{sc-per-loc}}\label{sc-per-loc}
    {\rpoloc} \cup {\rrfx} \cup {\rfr} \cup {\rco}\;\text{is acyclic}
    \\[-1ex]&
    \tag{\textsc{external}}\label{external}
    {\rob}\;\text{is acyclic}
  \end{align*}
\end{definition}

% Use these to refer to the rules in text:
%\ref{rf-comp} 
%\ref{co-tot}
%\ref{sc-per-loc}
%\ref{external}


Given an execution graph $G$, we say that $\aEv$ is an \emph{internal read} if
$\aEv\in\fcodom(\mathsf{po}\cap \mathsf{rf})$.    We are going to translate internal reads of execution graphs into internal reads of the semantics.  

From $G$ we construct a candidate pomset $\aPS$ as follows:
\begin{itemize}
\item $\Event= \textsf{E}$,
\item $\labelingAct(\aEv)=\tau \mathsf{lab}(e)$, if $\aEv$ is a relaxed
  internal read, 
\item $\labelingAct(\aEv)=\mathsf{lab}(e)$, if $\aEv$ is not a relaxed
  internal read,
\item $\labelingForm(\aEv)=\TRUE$,
\item ${\le} = {\rob}$, and
\item ${\gtN} = ({\rob} \cup {\reco})^*$
\end{itemize}
To reempphasize, in this candidate pomset, $\rob$ is calculated by considering the definition of $\rob$ without $\rrfi$, ie.:
\begin{align*}
  \tag{Dependency order}
  {\rdob} &\eqdef\smash{
    ({\raddr}\cup{\rdata});
    \cup ({\rctrl}\cup{\rdata}); {\IDW}; {\rcoi}^?
    \cup {\raddr}; {\rpox}; {\IDW}
  }
  \\
  \tag{Barrier order}
  {\rbob} &\eqdef\smash{
    {\IDAcq}; {\rpox}
    \cup {\rpox};{\IDRel}; {\rcoi}^?
  }
  \\
  \tag{Acyclic order}
  {\rob} &\eqdef\smash{
    ({\robs} \cup {\rdob} \cup {\rbob})^+
  }
\end{align*}


We show that $\aPS$ is a top-level pomset, reasoning as follows.

 First, we establish the criteria for a 3-valued pomset (Definition~\ref{def:3valued}).
\begin{itemize}
\item ${\le}$ is a partial order.  This holds since $G.{\rar}$ is acyclic.
\item If $\bEv \le \aEv$ then $\bEv \gtN \aEv$.  By construction.
\item If $\bEv \le \aEv$ and $\aEv \gtN \bEv$ then $\bEv = \aEv$.  Proved below.
\item If $\cEv \le \bEv \gtN \aEv$ or $\cEv \gtN \bEv \le \aEv$ then
  $\cEv \gtN \aEv$. By construction.
\end{itemize}

Next, we establish the criteria for a 3-valued pomset with preconditions (Definition~\ref{def:3pre}).
\begin{itemize}
\item $\labelingForm(\aEv)$ implies $\labelingForm(\bEv)$ whenever
  $\bEv\le\aEv$.   Trivial, since every formula is $\TRUE$.
\item $\aPS$ is $\aLoc$-coherent; that is, when restricted to events that
  read or write $\aLoc$, $\gtN$ forms a partial order.
\end{itemize}

Finally, we establish the criteria for a top-level pomset
(Definition~\ref{def:x-closed}).
\begin{itemize}
\item $\aEv$ is location independent. Trivial, since every formula is $\TRUE$.
\item If $\aEv$ reads $\aVal$ from $\aLoc$, then there is some $\bEv$ such that
  \begin{itemize}
  \item $\bEv \lt \aEv$,  
  \item $\bEv$ writes $\aVal$ to $\aLoc$, and
  \item if $\cEv$ writes to $\aLoc$
    then either $\cEv \gtN \bEv$ or $\aEv \gtN \cEv$.
  \end{itemize}    
\end{itemize}

\subsection{Proof that  $({\rob} \cup {\reco})^*$ is irreflexive. }

\paragraph*{Proof of lemma~\ref{extendob}. } 


Let $\aEv, \bEv$ be distinct events and $\bEv'\ (\xob\cap \xpox) \ \bEv\ ((\xeco\cap \xpox) \setminus \xrfi) \  \aEv\ (\xob \cap \xpox)  \ \aEv'$.  Then $\bEv' \xob \aEv'$.

\begin{proof}
If $\bEv'$ is an acquire,  or $\aEv$ is an release, or $\aEv'$ is a release, result is immediate.

We next consider the case where $\aEv$ is a read.  In this case,  $\bEv$ is a write.  Since $\bEv\ ((\xeco\cap \xpox) \setminus \xrfi) \  \aEv$, there is a write $\bEv_1$ such that $ \bEv \xcoe\ \bEv_1 \ \xrfe\ \aEv' $.  So, $\bEv \xob \aEv$ and result follows in this case. 


So, it suffices to prove the following assuming that $\bEv'$ is not an acquire and $\aEv'$ is not a release and $\aEv$ is not a release or a read and $\aEv, \bEv$ are distinct.
\begin{itemize}
\item If $\bEv'\ (\xob\cap \xpox)  \ \bEv(\xeco\cap\xpox)\aEv$ then $\bEv'\xob\aEv$.
\item If $\bEv\ (\xeco\cap\xpox) \ \aEv(\xob\cap\xpox)\aEv'$ then $\bEv\xob\aEv'$.
\end{itemize}


We first prove that if $\bEv'\ (\xob \cap \xpox) \ \bEv\ (\xeco \cap \xpox) \ \aEv$ then $\bEv'\xob\aEv$.   Proof proceeds by cases on the witness for $\bEv'\ (\xob\cap \xpox) \ \bEv$. 
\begin{itemize}
\item  If $\bEv' \xbob  \bEv$, then: 
\[ \bEv'\ (\smash{
    {\IDAcq}; {\rpox}
    \cup {\rpox};{\IDRel}; {\rcoi}^?) \ 
  }
\bEv
\]
Since $\bEv'$ is not an acquire, $\bEv' ({\rpox};{\IDRel}; {\rcoi}^?) \bEv$, so $\bEv$ is a write.  Since $\aEv$ is not a read,  $\bEv \xcoi\ \aEv$. Thus, result follows.

\item If $\bEv' \xdob  \bEv$, then: 
\[ \bEv'\ 
\smash{
    ( ({\rctrl}\cup{\rdata}); {\IDW}; {\rcoi}^?
    \cup {\raddr}; {\rpox}; {\IDW}
  } \
\bEv
\]
So, $\bEv$ is a write.  Since $\aEv$ is also a write, we deduce that 
\[ \bEv'\ 
\smash{
    ( ({\rctrl}\cup{\rdata}); {\IDW}; {\rcoi}^?
    \cup {\raddr}; {\rpox}; {\IDW}
  } \
\aEv
\]
\end{itemize}


We next prove  that if $\bEv\ (\xeco \cap \xpox) \ \aEv\ (\xob\cap \xpox) \ \aEv'$ then $\bEv\xob\aEv'$, under the assumptions that  $\aEv'$ is not a release and $\aEv$ is not a release or a read and $\aEv, \bEv$ are distinct.


 Proof proceeds by cases on the witness for $\aEv (\xob\cap \xpox) \aEv'$.  

\begin{itemize}
\item  If $\aEv \xbob  \aEv'$, then: 
\[ \aEv\ (\smash{
    {\IDAcq}; {\rpox}
    \cup {\rpox};{\IDRel}; {\rcoi}^?) \ 
  }
\aEv'
\]
Since $\aEv$ is not a read, $\aEv ({\rpox};{\IDRel}; {\rcoi}^?) \aEv'$.  Result follows since  $\bEv \xpox\ \aEv$.


\item If $\aEv \xdob  \aEv'$, then $\aEv$ is a read.  

\end{itemize}
\end{proof}


\begin{lemma}\label{obeco1}
If $\bEv\xob\aEv$ then $\lnot(\aEv\xeco\bEv)$.

\begin{proof}
Proof by contradiction.  Let 
\[ \aEv \xob \aEv' \xeco \bEv' \xob \bEv \xob \cEv \xob \cEv' \xob \aEv \]
where $\aEv' \xpox \bEv'$.

By lemma~\ref{extendob}, if $\aEv \not=\aEv'$, we deduce $\aEv \xob \bEv'$, and thus $\aEv \xob \bEv$.  If $\bEv \not=\bEv'$, we deduce $\aEv' \xob \bEv$ and thus $\aEv \xob \bEv$.

Thus, if $\aEv \not=\aEv'$ or $\bEv \not=\bEv'$, then there is a cycle $\aEv \xob \bEv \xob \cEv \xob \cEv' \xob \aEv$.  

So we can assume that  $\aEv' = \aEv$, $\bEv' = \bEv$ and 
\[ \aEv  \xeco \bEv \xob \cEv \xob \cEv' \xeco \aEv \]
where all of $\aEv, \bEv, \cEv, \cEv'$ access the same location and at least one of $\aEv,\bEv$ is a write, at least one of $\aEv,\cEv'$ is a write, and at least one of $\bEv,\cEv$ is a write.

We reason by cases.
\begin{itemize}
\item If $\cEv'$ is a write or both $(\aEv, \bEv)$ are writes.

We deduce that $\bEv \xeco \cEv' \xeco \aEv$ and thus $\bEv \xeco \aEv$.

\item $\cEv'$ is a read.  $\aEv$ is a write.  $\bEv$ is a read.

In this case $\cEv$ is a write.  From $\cEv \xob \aEv$, we deduce $\cEv \xeco \aEv$. Combining with $\bEv \xeco \cEv$, we deduce that $\bEv \xeco \aEv$.  


\end{itemize}
In either case, there is a contradiction $\aEv \xeco \bEv \xeco \aEv$.
\end{proof}
\end{lemma}


\begin{lemma}\label{obeco2}
$({\rob} \cup {\reco})^*$ is irrreflexive.
\end{lemma}
\begin{proof}
The simple case that $\rob; \reco$ is irreflexive is proved above.  The full proof by contradiction.  

Let $n \geq 1$ be the minimum such that:
\begin{align*} 
&\aEv^0_0 \xob \aEv^0_1 \xeco \bEv^0_0 \xob \bEv^0_1  \\
(\xeco \cap \xob) &  \   \aEv^1_0 \xob \aEv^1_1 \xeco \bEv^1_0 \xob \bEv^1_1 \\
(\xeco \cap \xob) & \ \ldots \\
& \ldots \bEv^n_1 \\
 (\xeco \cap \xob) & \  \aEv^0_0
\end{align*}
where  for all $i$, we have:
\[ \aEv^i_0 \xpox \aEv^i_1 (\xeco \cap \xpox) \bEv^i_0 \xpox \bEv^i_1\] and 
\[ \neg (\bEv^i_1 \xpox (\aEv^{(i+1) \mod n}_0 \]

For any $i$, if $\aEv^i_0 \not= \aEv^i_1$ or $\bEv^i_0 \xpox \bEv^i_1$, via lemma~\ref{extendob}, we deduce that $\aEv^i_0  \xob \bEv^i_1$, contradicting minimality of $n$.  

So, we can assume that $n \geq 1$ is such that:
\begin{align*} 
&\ \aEv^0 \xeco  \bEv^0 \\
(\xeco \cap \xob) &  \   \aEv^1  \xeco \bEv^1 \\
(\xeco \cap \xob) & \ \ldots \\
& \ldots \bEv^n \\
 (\xeco \cap \xob) & \ \aEv^0
\end{align*}
which is a contradiction since it is a cycle in $\xeco$ and since at least one of $\aEv^i ,\bEv^i$ is a write for all $i$. 
\end{proof}

\endinput 








\begin{comment}
Operation        Implementation
Relaxed read     ldr                     
Relaxed write    str             
Acquiring read   ldar            
Releasing write  stlr
Fence            dmb.sy
\end{comment}

\begin{comment}
ob does not contradict eco

ob does not contradict (co cap po):

Suppose that wx1 po wx2 then it cannot be that wx2 ob wx1.
We know that wx1 co wx2 by SC-PER-LOC

% Case 1. w1 is read externally, then we have
%   wx1 rfe r
% and
%   r fre w2
% so
%   wx1 obs+ wx2
% which contradicts EXTERNAL

% Case 2. wx1 is not read externally.
We show this by contradiction
Assume
  wx1 co wx2
and
  wx2 ob wx1

Note that
  po supseteq dob cup aob cup bob
So in order to get order into wx1, we must have
  wx2 (ob?; obs; ob?; obs; ob?) wx1

Note that we cannot have dob or bob into wx1 after obs, since then we would
also have it into wx2, creating a cycle in EXTERNAL.  This holds because both
dob and bob are closed on the right w.r.t. coi

So it must be that 
  wx2 (ob?; obs; ob?; wx0; coe) wx1, 
in which case we also have wx0 coe wx2, contradicting EXTERNAL
or 
  wx2 (ob?; obs; ob?; rx0; fre) wx1
in which case we also have rx0 fre wx2, contradicting EXTERNAL




Internal reads do not need to respect ob:
Arm allows the following:

  Ra1 -ctrl-> Wx1 -rfi-> Rx1 ---> Wb1    if(a){x=1}; b=x
   |                               |
  Wa1 <-------------------------- Rb1    a=b


Suppose that wx1 po rx2 and rx2 is read externally.
Then it cannot be that rx2 ob wx1.

Case 1: if wx1 co wx2, then we have wx1 coe wx2 rfe rx2, contradicting EXTERNAL
Case 2: if wx2 co wx1, then we have rx2 fr wx1, contradicting SC-PER-LOC



Suppose that rx1 po wx2 and rx1 is read externally.
Then it cannot be that wx2 ob rx1.

Case 1: if wx2 co wx1, then wx2 co wx1 rf rx1 po wx2, contradicting SC-PER-LOC 
Case 2: if wx1 co wx2, for a contradiction, suppose wx2 ob rx1.
then we need another thread involved to get order from wx2 to rx1.
To get order into the read, there are several options:
- use cross thread read, then dob; but dob does not include reads in it's domain.
  An attempt to do this is something like:

              Wx1                 x=1
               |
  Ra2 -ctrl-> Rx1 - - -> Wx2      if(a){r=x}; x=2
   |                      |
  Wa2 <----------------- Rx2      a=x

  But the ctrl dependency is not included in ob between reads.
- use cross thread read then barrier, but then you contradict EXTERNAL
- create and ob edge from Rx2 to Wx1.
  An attempt to do this is, 

  Wx1 <-------------- Ra1       
   |                   |        But cannot get Wx2 --> Wa1 without a barrier
  Rx1 - - -> Wx2 ---> Wa1       

  Wx1 <----- Rx2                         
   |          |                 contradicts SC-PER-LOC 
  Rx1 - - -> Wx2                         


Other examples to type in:
Allowed:
Rx1 -> Wy0  Wy1
Ry1 -> Wz0  Wz1
Rz1 -> Wx0  Wx1

Forbidden:
Rx1 -> Wy0 Wy1
Ry1 -> Wx0 Wx1

\end{comment}



\begin{comment}
\citet{DBLP:journals/pacmpl/PodkopaevLV19} define the \emph{Intermediate
  Memory Model (IMM)} and provide efficient implementations of the IMM into
several processor architectures, including TSO, ARMv8 and Power.

In this section, we show that any execution allowed by a sublanguage of the
IMM is also allowed by our semantics.  The sublanguage we consider bans
loops, read-modify-write (RMW) operations, and fences.  In addition, we take
the set of memory locations, $\Loc$, to be finite.  Syntactically, we drop
the superscript \textsf{rlx} on relaxed reads and writes; in addition, we use
structured conditionals rather than the more general \textsf{goto}.  We refer
to this sublanguage as $\muIMM$.

$\muIMM$ programs sit in the restriction-free fragment of our language, where
all memory locations are initialized to $0$ and parallel-composition occurs
only at top level.  In other words, $\muIMM$ programs have the form
\begin{displaymath}
  {\aLoc_1}\GETS{0}\SEMI
  \cdots\SEMI
  {\aLoc_m}\GETS{0}\SEMI
  (\aCmd^1 \PAR \cdots \PAR \aCmd^n)
\end{displaymath}
where $\aCmd^1$, \ldots, $\aCmd^n$ do not include either composition or
restriction.

Due to space limitations, we do not include a full description of the IMM.
The broad strokes of the argument given here should be clear, but interested
readers will need to refer to \citep{DBLP:journals/pacmpl/PodkopaevLV19} for
details.
\end{comment}



\endinput

\section{Proof of DRF}

For any $\aPS$, then $\closed(\aPS)$ is set enriched with useless reads
(preserving augmentation closure) and where we remove any event whose
precondition is not a tautology.

For top level programs:
\begin{displaymath}
  \semClosed{\VAR\vec{\aLoc}\SEMI
    \vec{\aLoc}\GETS\vec{0}\SEMI
    \vec{\bLoc}\GETS\vec{0}\SEMI
    \FENCE\SEMI
    (\aCmd^1 \PAR \cdots \PAR \aCmd^n)}
  =
  \VAR\vec{\aLoc}\SEMI
    \vec{\aLoc}\GETS\vec{0}\SEMI
    \vec{\bLoc}\GETS\vec{0}\SEMI
    \FENCE\SEMI
    (\semClosed{\aCmd^1} \PAR \cdots \PAR \semClosed{\aCmd^n})
\end{displaymath}

\begin{definition}
A thread: top level component of a parallel composition
\end{definition}

\begin{definition}
$\aPS$ is a generator of  $\semClosed{\aCmd}$ if for all $\bPS \in \semClosed{\aCmd}$ such that $\aPS$ augments $\bPS$, $\aPS = \bPS$.
\end{definition}


Since the program we consider are loop free, for any command $\aCmd$, the size of the pomsets in $\aCmd$ are bounded by a constant, that we denote by $\size(\aCmd)$.  
 

\section{Generators for semantics of programs with parallel composition}
All generators $\aPS$  satisfy the following factorization of cross-thread $\lt$.  

\begin{lemma}\label{pargen}
Consider the subset of pomsets of $\semClosed{\aCmd \PAR \bCmd}$ that are  $\aLoc$-closed for all $\aLoc$.  

Let $\aPS$  be any generator.  
%\begin{itemize}
% \item
 Let $\aEv\lt\bEv$ and $\aEv \in \semClosed{\aCmd}$ and  $\bEv \in \semClosed{\bCmd} $.
  
Then there is a write  $\aEv' \in \semClosed{\aCmd}$, and  a read $\bEv' \in \semClosed{\bCmd}$ such that  $\bEv'$ reads-from $\aEv'$ and $\aEv \lt \aEv' \lt \bEv' \lt \bEv$. 

%\item $\aEv \gtN \bEv$ only if $ \aEv  [\lt \cup (\le; \reco;\le)^{\star}]  
%\bEv$.

% \item If $\aEv\lt\bEv$ and $\aEv, \bEv \in \semClosed{\aCmd}$, 
%then there exists 

%There exists a release action $\aEv'$ in $\sem{\aCmd}$, a 
%matching acquire action $\bEv'$ in $\sem{\bCmd}$ such that $
%\aEv \lt \aEv'$, $\bEv' \lt \bEv$ and $\aEv' \lt \bEv'$.

\end{lemma}








The proof of lemma~\ref{cohsat} yields the following two corollaries.
\begin{corollary}\label{cohrw}
Let $\aPS \in \sem{\aCmd}$ be a generator. Let 
\begin{itemize}
\item $\bEv'$ be a read from $\aLoc$ with matching write $\bEv$.  \item $\aEv$ be a write to $\aLoc$ such that  $\bEv' \gtN \aEv$.   \item Forall writes $\cEv$ to $\aLoc$ such that  $ \bEv \gtN \cEv \gtN  \aEv$,  it is the case that  $ \neg(\bEv' \lt \cEv)$ and $\neg(\bEv \xpox \cEv) ]$
\end{itemize}

Then, there exists $\bPS \in \sem{\aCmd}$, also a generator, such that $\Event_{\aPS} = \Event_{\bPS}$, $\le_{\aPS} = \le_{\bPS}$, and $\aEv \gtN \bEv'$ in $\bPS$.
\end{corollary}
\begin{corollary}\label{cohwr}
Let $\aPS \in \sem{\aCmd}$ be a generator. Let 
\begin{itemize}
\item $\aEv'$  read from $\aLoc$ with matching write $\aEv$. 
\item $\bEv$ be a  write to $\aLoc$ such that  $\bEv \gtN \aEv'$.  \item Forall writes $\cEv$ to $\aLoc$ such that  $ \bEv \gtN \cEv \gtN  \aEv$ and $\cEv \not= \aEv$,  it is the case that  $ \neg(\cEv \lt \aEv')$ and $\neg(\cEv \xpox \aEv) ]$. 
\end{itemize}

Then, there exists $\bPS \in \sem{\aCmd}$, also a generator, such that:
$\Event_{\aPS} = \Event_{\bPS}$, $\le_{\aPS} = \le_{\bPS}$, and 
$\aEv' \gtN \bEv$ in $\bPS$.  

\end{corollary}
        

===============good lemma. Not used. ==================




\begin{definition}
$ \aEv \xeco  \bEv$ if both $\aEv$ and $\bEv$ touch the same location, at least one is a write, and $\aEv \xird \bEv$  or $\aEv \xrb \bEv$ or $\aEv\xird \bEv$ or $\bEv \gtN \aEv$.
\end{definition}




% \section{Introduction}
\label{sec:intro}

\citet{Hoare:1969:ABC:363235.363259} described a logic for understanding
sequential execution.  For example, the rule for assignment states that:
\begin{displaymath}
  \hoare{\aForm[\aExp/\aLoc]}{\aLoc\GETS\aExp}{\aForm}
\end{displaymath}
This views a program as a \emph{predicate transformer}, made explicit by
\citet{Dijkstra:1975:GCN:360933.360975}.
\begin{displaymath}
  \fwp(\aLoc\GETS\aExp,\;\aForm) = \aForm[\aExp/\aLoc]
\end{displaymath}
This interpretation gives a natural notion of program equality.  It defines a
semantics for all programs.  It is compositional.  It supports type-safe
execution.  It is preserved by compilers and single-threaded hardware.
(Indeed, equivalence under this model can be taken as the correctness
condition for compiler and hardware optimizations.)


Concurrent programming is more difficult to understand, but in certain
special cases substantial progress has been made, such as race-free
shared-memory
\citep{OHearn:2007:RCL:1235896.1236121,OHearn:2019:SL:3310134.3211968} and
message-passing \citep{Hennessy:1980:ONC:646234.758793,Cleaveland2018}.

For \emph{shared-memory programs with data races (SMP-DR)}, no canonical
model has arisen.  A sensible model should induce a natural notion of program
equality.  It should define a semantics for all programs.  It should be
compositional.  It should support type-safe execution.  It should be
preserved by current compilers and multi-threaded hardware.  It should also be
\emph{coherent}, in that writes to a single location appear to be totally
ordered.  The last two points are descriptive, in that they capture existing
systems, yet they bound the definition in different directions: To support
optimization, the model cannot be to strong.  To support coherence, the model
cannot be too weak\footnote{From this point of view, coherence is what
  distinguishes shared memory from a truly distributed system.}.

Concurrent models require that we identify events for read and write actions,
which might later allow for interaction with another thread.  Yet, the
correctness of transformations on single-threaded code are defined
semantically.  Getting these two world-views to play nice has proven
incredibly challenging.

Rather than a single canonical model, we have arrived at a spectrum of models
for SMP-DR.  These can be characterized as either \emph{strong} or
\emph{weak}.  The strong models invalidate some common compiler and hardware
optimizations and reorderings, such as common subexpression elimination or
load-buffering.

Sequential consistency (SC) \citep{Lamport:1979:MMC:1311099.1311750} is
universally accepted as the strongest sensible model of SMP-DR.  In addition
to invalidating optimizations, SC is also non-compositional.  Nevertheless,
SC and other strong models my work well in practice
\cite{Singh:2012:ESC:2337159.2337220,Dolan:2018:BDR:3192366.3192421,Ou:2018:TUC:3288538.3276506,Liu:2019:ASC:3314221.3314611}.

Attempts to define a weakest model for SMP-DR have largely failed.
\begin{enumerate}
\item \citet{Manson:2005:JMM:1047659.1040336} and similar models
  \cite{DBLP:conf/esop/JagadeesanPR10,DBLP:conf/popl/KangHLVD17} all require
  shenanigans for type safety
  \cite{DBLP:journals/toplas/Lochbihler13,DBLP:conf/tldi/GotoJPR12}.  This is
  a variant of the thin-air problem.
\item \citet{DBLP:conf/lics/JeffreyR16} is too strong.
\item C11 does not define a semantics for racy programs.  Even restricted to
  the defined subset, C11 still allows thin-air type unsafety on relaxed
  atomics. The attempts to repair C11 have lead to strong models.
\end{enumerate}

In this paper, we propose a candidate definition for
\begin{center}
  \emph{the weakest sensible model of SMP-DR}.
\end{center}

Our model uses Hoare logic for sequential code and partially-ordered sets of
read-write events for concurrent code.  As usual, the order denotes temporal
dependency between two events.

Our first insight is that these two approaches can be combined by associating
logical predicates with events in the partial order.  Sequencing affects the
predicates, as in Hoare logic.  Parallel composition, instead, combines the
events of two partial orders.

Our second insight is that when combining partial orders from the two
branches of a conditional, events may coalesce, resulting in weaker
predicates via disjunction.  This allows events that occur within both
branches of a conditional to float in the partial order, independent of the
condition itself.

There are several details to sort out, which we dutifully perform:
\begin{itemize}
\item give semantics of release and acquire fences
\item establish DRF-SC
\item establish logical reasoning
\item establish congruence properties for language constructs
\item establish that compiler optimizations are permitted
\item establish that hardware optimizations are permitted
\end{itemize}

Things we do not do:
\begin{itemize}
\item give semantics of read-modify-write operations
\item give semantics for non-terminating programs
\item prove type safety for a nontrivial language
\end{itemize}
About the last point, we do prove that our semantics disallows examples such
as Figure 27 of \cite{DBLP:journals/toplas/Lochbihler13}, which is closely
related to type safety.


% Local Variables:
% mode: latex
% TeX-master: "paper"
% End:
% \section{Sequential Composition}
\label{sec:semicolon}
We provide an alternative semantics that supports full sequential
composition, building $\sem{\aCmd\SEMI\bCmd}$ from $\sem{\aCmd}$ and
$\sem{\bCmd}$.  To simplify the definitions, we assume that each register is
assigned at most once syntactically
\cite{Rosen:1988:GVN:73560.73562}\nofootnote{One can remove this restriction
  by defining conflict (Definition \ref{def:prefix}) to include actions that
  read into the same register; this includes order between reads in item 5b.
  In addition, let $\READS(\aPS)=\{\aEv\mid\aEv$ is a read into $\aReg$ and
  there is no $\aEv$ that reads into $\aReg$ such that $\dEv\le\aEv\}$.  For
  Theorem \ref{thm:seq} to hold, register conflict must then be included in
  $\semold{}$.}.  Since we exclude loops, this trivially ensures that each
register is assigned at most once semantically.

We refactor the syntax:
\begin{align*}
  \aCmd,\,\bCmd
  \BNFDEF& \SKIP
  \mkern-2mu\BNFSEP\mkern-2mu \FENCE^{\fmode}
  \mkern-2mu\BNFSEP\mkern-2mu \aReg\GETS\aExp
  % \mkern-2mu\BNFSEP\mkern-2mu \aReg\GETS \aLoc^{\amode} 
  % \mkern-2mu\BNFSEP\mkern-2mu \aLoc^{\amode}\GETS\aExp
  \mkern-2mu\BNFSEP\mkern-2mu \aReg\GETS \REF{\cExp}^{\amode} 
  \mkern-2mu\BNFSEP\mkern-2mu \REF{\cExp}^{\amode}\GETS\aExp
  \\[-.5ex]
  \BNFSEP&\aCmd \PAR \bCmd
  \mkern-2mu\BNFSEP\mkern-2mu\aCmd \SEMI \bCmd
  \mkern-2mu\BNFSEP\mkern-2mu \VAR\aLoc\SEMI \aCmd
  \mkern-2mu\BNFSEP\mkern-2mu \IF{\aExp} \THEN \aCmd \ELSE \bCmd \FI
\end{align*}
% Without loss of generality,
% To keep the presentation as simple as possible, we include neither address
% calculation nor \RMW{}s.  These extensions are straightforward, but
% notationally cumbersome.

\paragraph{Explicit Substitutions.}
Let $\aEExp$ range over \emph{extended expressions}, which may include memory
locations.  We introduce explicit substitutions over extended expressions,
following the conventions of \citet{DBLP:conf/icalp/RitterP97}:
\begin{gather*}
  \aLocReg\BNFDEF \aLoc \BNFSEP \aReg
  \qquad\quad
  \aSub,\,\bSub,\, \SUBDRS{\dEvs} %\beforeSub,\,\afterSub
  \BNFDEF \SUBEMP \BNFSEP \SUBPAR{\aSub}{\aEExp/\aLocReg}
  \BNFSEP \aSub\SUBSEQ\aSub'
\end{gather*}
$\SUBEMP$ is the identity substitution.  We write
$\SUBPAR{\SUBEMP}{\aEExp/\aLocReg}$ as $\SUB{\aEExp/\aLocReg}$.

Application is written $\aSub\SUBAPP\aForm$.  We only apply substitutions to
formulae---which do not bind locations or registers.  The definition is
homomorphic over the syntax of formulae. For the basis, 
\begin{math}
  \SUBPAR{\aSub}{\aEExp/\bLocReg}\SUBAPP\aLocReg
\end{math}
is $\aEExp$ if $\bLocReg=\aLocReg$ and is $\aSub \SUBAPP\aLocReg$ otherwise.

%and $\SUBPAR{\SUBPAR{\aSub}{\bEExp/\aLocReg}}{\aEExp/\aLocReg}$ as $\SUBPAR{\aSub}{\aEExp/\aLocReg}$.

Sequencing is defined so that
% \begin{math}
%   \aSub\SUBSEQ\SUBPAR{\bSub}{\aEExp/\aLocReg}
%   = 
%   \SUBPAR{\aSub\SUBSEQ\bSub}{\aSub\SUBAPP\aEExp/\aLocReg}
% \end{math}
\begin{math}
  \aSub\SUBSEQ\SUB{\aEExp/\aLocReg}
  \allowbreak= 
  \SUBPAR{\aSub}{\aSub\SUBAPP\aEExp/\aLocReg}
\end{math}
and
\begin{math}
  (\beforeSub\SUBSEQ\afterSub)\SUBAPP\aForm = \beforeSub\SUBAPP(\afterSub\SUBAPP\aForm).
\end{math}

We say that $\aSub$ \emph{subsumes} $\bSub$ if for every $\aLocReg$, either
$\bSub\SUBAPP\aLocReg=\aSub\SUBAPP\aLocReg$ or $\bSub\SUBAPP\aLocReg=\aLocReg$.
For example, every substitution subsumes $\SUBEMP$.
% Goal:
% \begin{math}
%   (\aForm\aSub)\bSub =
%   \aForm(\aSub;\bSub)
% \end{math}
% Pure substitution: $\fdom(\aSub)$ disjoint $\fcodom(\aSub)$.
% Pure substitutions are idempotent.
% $\aSub$ and $\bSub$ are composable if $\fdom(\aSub)$ disjoint $\fcodom(\bSub)$
% \begin{displaymath}
%   (\aSub;\bSub)(x) =
%   \begin{cases}
%     \aSub(\bSub(x)) & \text{if } x \in \fdom(\bSub)\\
%     \aSub(x) & \text{otherwise}
%   \end{cases}
% \end{displaymath}

\paragraph{Substitutions in the Data Model.}
Let $\Sub$ be the set of all (explicit) substitutions.  We include
substitutions as actions: $\Sub\subseteq\Act$.  All substitution actions are
\emph{termination} actions.  For comparison with the semantics of
\textsection\ref{sec:model}, we identify $\SUBEMP$ and $\DSTOP$.

In the context of a
pomset $\aPS$, we use $\Sub$ also to represent the set of substitution
events:
\begin{math}
  \{ \aEv\in\Event \mid \labelingAct(\aEv) \in \Sub \}.
\end{math}
% Alternatively, we
% could extend the Definition \ref{def:mmpomset} to include an optional
% substitution:
Recall that our pomsets contain at most one termination event, which is ordered
after every other event.  
In examples, we typically drop order into
substitution actions, instead drawing them as accepting states.

We modify read actions both to name the register that was written
and also to include a substitution:
$\DRreg[\amode]{\aReg}{\bSub}{\aLoc}{\aVal}$.
We say $(\DRreg[\amode]{\aReg}{\bSub}{\aLoc}{\aVal})$ reads \emph{into}
$\aReg$.

We say that $\aAct$ \emph{substitutes} $\aSub$ if either
$\aAct=\aSub$, $\aAct=(\DRreg[\amode]{\aReg}{\aSub}{\aLoc}{\aVal})$, or
$\aAct$ is neither a read nor a termination and $\aSub=\SUBEMP$.

We modify Definition \ref{def:rf} to require that when $\aPS$ is
\emph{$\aLoc$-closed}, every substitution in $\aPS$ is the identity on
$\aLoc$: If $\aEv$ substitutes $\aSub$ then $\aSub\SUBAPP\aLoc=\aLoc$.

We lift the notion of subsumption from explicit substitutions to actions and
pomsets in the obvious way: $\aAct$ subsumes $\bAct$ if 
%either (1) $\aAct=\bAct$, (2) $\aAct=\aSub$, $\bAct=\bSub$ and $\aSub$
%subsumes $\bSub$, or (3)
$\aAct$ substitutes $\aSub$,
$\bAct$ substitutes $\bSub$, and $\aSub$ subsumes
$\bSub$.
$\aPS'$ \emph{subsumes} $\aPS$ if $\Event'=\Event$, ${\le'}={\le}$,
$\labelingForm'=\labelingForm$, and $\labelingAct'(\aEv)$
subsumes $\labelingAct(\aEv)$. 

The semantics of programs is closed w.r.t.~\emph{reverse subsumption}: if
$\aPS\in\sem{\aCmd}$ and $\aPS$ is subsumed by $\aPS'$, then
$\aPS'\in\sem{\aCmd}$.

Subsumption is dual to implication (Definition \ref{def:closure:properties}):
Strong\-er preconditions impose a greater burden on the preceding code;
stronger substitutions can better mitigate this burden in the code that
follows.  In examples, we only show executions that are implication and
augmentation minimal; similarly, we only show executions that are
subsumption-maximal.

Suppose $\aPS$ is a completed, subsumption-maximal pomset.  Then
$\labelingAct(\Event\cap\Sub)$ is defined, and all reads satisfy the invariant:
if $\labelingAct(\aEv)=(\DRreg[\amode]{\aReg}{\bSub}{\aLoc}{\aVal})$ then there are
$\beforeSub$ and $\afterSub$ such that
$\bSub=\beforeSub\SUBSEQ\SUB{\aLoc/\aReg}\SUBSEQ\afterSub$ and
$(\beforeSub\SUBSEQ\afterSub) = \labelingAct(\Event\cap\Sub)$.



We define notation to lift sequential composition of substitutions into
read actions:
\begin{itemize}
\item Let $\aAct$ \emph{before} $\aPS$ be $\aAct$ when
  $\disjoint{\Event}{\Sub}$ or $\aAct$ is not a read, and
  $\DRreg[\amode]{\aReg}{(\bSub\SUBSEQ\aSub)}{\aLoc}{\aVal}$ when
  $\aSub=\labelingAct(\Event\cap\Sub)$ and
  $\aAct=\DRreg[\amode]{\aReg}{\bSub}{\aLoc}{\aVal}$.
\item 
  Let $\aAct$ \emph{after} $\aPS$ be $\aAct$ when $\disjoint{\Event}{\Sub}$
  or $\aAct$ is not a read, and
  $\DRreg[\amode]{\aReg}{(\aSub\SUBSEQ\bSub)}{\aLoc}{\aVal}$ when
  $\aSub=\labelingAct(\Event\cap\Sub)$ and
  $\aAct=\DRreg[\amode]{\aReg}{\bSub}{\aLoc}{\aVal}$.
\end{itemize}

\paragraph{Semantics of the Example Language.}

We elide the definitions of parallel composition, conditional and location
binding, which can be imported directly from
\textsection\ref{sec:model}---using the modified definition of
$\aLoc$-closure discussed above.

Note that when coalescing reads or termination events using \!$\PAR$\!,
the substitutions must be identical.  Using reverse subsumption closure, this
is always possible; further it guarantees that the resulting substitution is subsumed by
the substitutions on either side.  Thus, register state is only preserved
after composition if it is shared on both sides.
For example,
\begin{math}
  (r\GETS 1\PAR r\GETS 1)\SEMI x\GETS r
\end{math}
can write $1$ to $x$, but
\begin{math}
  (r\GETS 1\PAR \SKIP)\SEMI x\GETS r
\end{math}
cannot.

To simplify the base cases, we use a literal notation for pomsets and define
$\CLOSE{\aPS}$ to be the smallest set that includes $\aPS$ and is closed
w.r.t.~prefixing, implication, augmentation, and reverse subsumption: Let
$\aPS''\in\CLOSE{\aPS}$ if there is $\aPS'\in\PRE(\aPS)$ such (1) $\aPS''$
implies $\aPS'$, (2) $\aPS''$ is an augmentation of $\aPS'$ and (3) $\aPS''$
is subsumed by $\aPS'$.
\begingroup
\allowdisplaybreaks
\begin{gather*}
  \begin{aligned}
  \sem{\SKIP} & \eqdef
  \CLOSE{\TIKZ{\final{f}{\SUBEMP}{}}}
  \\  
  \sem{\aReg\GETS\aExp} & \eqdef
  \CLOSE{\TIKZ{\final{f}{\SUB{\aExp/\aReg}}{}}}
  \\
  \sem{\FENCE^{\fmode}} & =
  \CLOSE{\TIKZ{
      \event{a}{\DFS{\fmode}}{}
      \final{f}{\SUBEMP}{right=of a}
      \po{a}{f}
    }} 
  \\
  \sem{\aReg\GETS\REF{\cExp}^\amode} & =
  \textstyle\bigcup_{\cVal,\aVal}\;
  \CLOSE{\TIKZ{
      \event{a}{\cExp=\cVal\mid\DRreg[\amode]{\aReg}{\SUB{\REF{\cVal}/\aReg}}{\REF{\cVal}}{\aVal}}{}
      \final{f}{\SUBEMP}{right=of a}
      \po{a}{f}
    }}
  % \\
  % \sem{\aReg\GETS\aLoc^\amode} & =
  % \textstyle\bigcup_\aVal\;
  % \CLOSE{\TIKZ{
  %     \event{a}{\DRreg[\amode]{\aReg}{\SUB{\aLoc/\aReg}}{\aLoc}{\aVal}}{}
  %     \final{f}{\SUBEMP}{right=of a}
  %     \po{a}{f}
  %   }}
  % \\[-.5ex] &
  % \mkern2mu\cup
  % \CLOSE{\TIKZ{
  %     \event{a}{\DFR{\amode}}{}
  %     \final{f}{\SUB{\aLoc/\aReg}}{right=of a}
  %     \po{a}{f}
  %   }}
  \\
  \sem{\REF{\cExp}^\amode\GETS\aExp} & =
  \;\;\textstyle\parallel_{\cVal,\aVal}\;
  \CLOSE{\TIKZ{
      \event{a}{\cExp=\cVal\land\aExp=\aVal \mid \DW[\amode]{\REF{\cVal}}{\aVal}}{}
      \final{f}{\SUB{\aExp/\REF{\cVal}}}{right=of a}
      \po{a}{f}
    }}
  % \\
  % \sem{\aLoc^\amode\GETS\aExp} & =
  % \;\;\textstyle\parallel_\aVal
  % \CLOSE{\TIKZ{
  %     \event{a}{\aExp=\aVal \mid \DW[\amode]{\aLoc}{\aVal}}{}
  %     \final{f}{\aExp=\aVal \mid \SUB{\aExp/\aLoc}}{right=of a}
  %     \po{a}{f}
  %   }}
  % \\[-.5ex] &
  % \mkern2mu\cup
  % \CLOSE{\TIKZ{
  %     \event{a}{\DFW[\aLoc]{\amode}}{}
  %     \final{f}{\SUB{\aExp/\aLoc}}{right=of a}
  %     \po{a}{f}
  %   }}
  \\
  \sem{\aCmd \SEMI \bCmd} &= \sem{\aCmd} \sequence \sem{\bCmd}
  % \\
  % \sem{\aCmd \PAR \bCmd} &= \sem{\aCmd} \parallel\sem{\bCmd}
\end{aligned}
  % \\
  % \begin{aligned}
  %   \sem{\aCmd \PAR \bCmd} &= \sem{\aCmd} \parallel \killS\sem{\bCmd}
  %   &
  %   \killS(\aPS)&=\{ \aEv\in\Event \mid \labelingAct(\aEv)\notin\Sub \}
  %   \\
  %   \sem{\aCmd \SEMI \bCmd} &= \sem{\aCmd} \sequence \sem{\bCmd}
  %   &
  %   \killS(\aPSS)&=\{\aPS\restrict{\killS(\aPS)} \mid \aPS\in\aPSS \}
  % \end{aligned}
\end{gather*}
\endgroup

The most significant challenge is to define semantic sequencing.  The
definition is complex, not only because of the bookkeeping required by
explicit substitutions.  Unfortunately, \emph{disjunction} and \emph{prefix
  weakening} (Definition \ref{def:closure:properties}) do not come easily.

Recall \eqref{alanAddress} from \textsection\ref{sec:variants}:
\begin{math}
  \sem{a[r] \GETS 0\SEMI a[0]\GETS \BANG r}.
\end{math}
In the semantics, an event from the first statement can coalesce with an
event from the second.  Thus when computing
\begin{math}
  \sem{a[r] \GETS 0} \sequence \sem{a[0]\GETS \BANG r},
\end{math}
we must coalesce events with incompatible preconditions ($r{=}0$, $r{=}1$)
that occur on different sides of the sequencing operator.  This makes a
direct definition difficult.

Instead of a direct definition, we first construct the sequential composition
\emph{without} coalescing events, then close the resulting set of pomsets to
ensure the required properties.

There is also a challenge dealing with redundant write elimination:
\begin{math}
  \sem{x\GETS 1\SEMI x^\modeREL\GETS 2} 
\end{math}
should contain a pomset that includes only
\begin{math}
  (\DW[\modeRA]{x}{2}).
\end{math}
We achieve this using the same strategy: closing after the construction.

Let $\cEv$ be an \emph{unused write} in $\aPS$ when it is a relaxed write to
some $\aLoc$ such that (1) $\cEv$ fulfills no reads, (2) there is some
$\bEv\gt\cEv$ that writes $\aLoc$, and (3) for every release $\aEv\gt\cEv$
there is some $\aEv\gt\bEv\ge\cEv$ that writes $\aLoc$.

Let $\DISJUNCT(\aPSS)$ be the least set that
includes $\aPSS$ and that is closed w.r.t.~disjunction, prefix weakening, and
unused write removal.

% Let $\DISJUNCT(\aPSS)$ be the set $\aPSS''$
% where $\aPS''\in\aPSS''$ if there are $\aPS'\in\PRE(\aPS'')$ and
% $\aPS^i\in\aPSS$ such $\Event' = \Event^i$, ${\le'}={\le^i}$,
% $\labelingAct'=\labelingAct^i$, and $\labelingForm''(\aEv)$ implies
% $\labelingForm'(\aEv)\cup\bigvee_i \labelingForm^i(\aEv)$.

In the following definition, $\aPSS'$ is defined using the subset of
$\aPSS^1$ that have an accepting state: $\labelingAct(\Event^1\cap\Sub)$ is only defined for
$\{\aPS^1\in\aPSS^1\mid \notdisjoint{\Event^1}{\Sub}\}$.  To ensure prefix
closure, we include $\aPSS^1$ before applying $\DISJUNCT$.
% A simple inductive proof shows that for any pomset in $\aPS\in\sem{\aCmd}$, there is
% extension $\aPS'\in\sem{\aCmd}$ with an accepting state, such that $\aPS\in\PRE(\aPS')$.

% We give the definition as two cases: (1) We include all the
% pomsets in $\aPSS^1$\!\!.  (2) We include composed pomsets for every $\aPS^1$
% with an accepting state.

% \begin{gather*}
%   \begin{aligned}
%     \killS(\aPS)&=\{ \aEv\in\Event \mid \labelingAct(\aEv)\notin\Sub \}
%     &
%     \killS(\aPSS)&=\{\aPS\restrict{\killS(\aPS)} \mid \aPS\in\aPSS \}
%   \end{aligned}
% \end{gather*}

% Relative to the previous definitions, the base cases are handled as follows:
% \begin{itemize}
% \item $\sem{\SKIP}$ introduces the identity substitution,
% \item $\sem{\aLoc\GETS\aExp}$ introduces $\SUB{\aExp/\aLoc}$,
% \item $\sem{\aReg\GETS\aExp}$ introduces $\SUB{\aExp/\aReg}$,
% \item $\sem{\aReg\GETS\aLoc}$, local rule, introduces $\SUB{\aLoc/\aReg}$,
% \item $\sem{\aReg\GETS\aLoc}$, nonlocal rule, introduces
%   $(\DR{\aLoc}{\aVal}) \lt \SUB{\aVal/\aReg}$.
% \end{itemize}
% Base cases:
% \begin{align*}
%   \sem{\SKIP}
%   =&
%   \TIKZ{\final{f}{}{}} 
%   \\
%   \sem{\aLoc\GETS\aExp}
%   =&
%   \textstyle\bigcup_\aVal\; \TIKZ{\event{a}{(\aExp=\aVal\mid\DW\aLoc\aVal)}{}\final{f}{\aExp/\aLoc}{right=of a}}
%   \\
%   \sem{\aReg\GETS\aLoc}
%   =&
%   \TIKZ{\final{f}{\aLoc/\aReg}{}}
%   \cup
%   \textstyle\bigcup_\aVal\; \TIKZ{\event{a}{(\DR\aLoc\aVal)}{}\final{f}{\aLoc/\aReg}{right=of a}\po{a}{f}}
% \end{align*}


% Here's the def for prefixing:
% \begin{enumerate}
% \item[1.] $\Event' = \Event \uplus \{\bEv\}$,
% \item[2.] ${\le'}\supseteq{\le}$,
% \item[3a.] $\labelingAct'(\bEv) = \aAct$,
% \item[3b.] $\labelingForm'(\bEv)$ implies $\aForm$,
% \item[4a.] $\labelingAct'(\aEv) = \labelingAct(\aEv)$,
% \item[4b.] if $\bEv$ \externally reads $\aVal$ from $\aLoc$ then
%   $\labelingForm'(\aEv)$ implies $\labelingForm(\aEv)[\aVal/\aLoc]$,
% \item[4c.] if $\bEv$ does not \externally read then $\labelingForm'(\aEv)$
%   implies $\labelingForm(\aEv)$, 
% \item[5a.] if $\labelingForm'(\aEv)$ does not imply $\labelingForm(\aEv)$ and
%   $\aEv$ writes, then $\bEv\lt'\aEv$,
% \item[5b.] if $\bEv$ and $\aEv$ are \external actions in conflict, then
%   $\bEv\lt'\aEv$,
% \item[5c.] if $\bEv$ is an acquire or $\aEv$ is a release, then
%   $\bEv \lt' \aEv$,
% \item[5d.] if $\bEv$ is an SC write and $\aEv$ is an SC read, then
%   $\bEv \lt' \aEv$, 
% \item[5e.] if $\bEv$ reads and $\labelingAct(\aEv)=\DFS{\modeACQ}$, then
%   $\aEv \lt' \bEv$,
% \item[5f.] if $\labelingAct(\bEv)=\DFS{\modeREL}$ and $\aEv$ writes, then
%   $\bEv \lt' \aEv$, and
% \item[6.] if $\bEv$ is a release, $\aEv_1$ is an acquire, $\aEv_1\le\aEv_2$, then $\labelingForm(\aEv_2)$
%   is location independent.
% \end{enumerate}


% \begin{definition}
%   \label{def:semi:seq}
%   % Let $\READS(\aPS)$ be the set $\dEvs\subseteq\Event$ where
%   % $\aEv\in\dEvs$ when $\aEv$ is a read of some $\aLoc$, and
%   % there is no $\dEv>\aEv$ that reads from $\aLoc$.

%   % Let $\dEvs$ be a \emph{disjoint read set} of $\aPS$  when 
%   % $\dEvs\subseteq\Event$,
%   % all $\aEv\in\dEvs$ are reads, and
%   % if $\aEv\in\dEvs$ reads from $\aLoc$ then
%   % (1) if $\dEv\in\dEvs$ reads from $\aLoc$ then $\dEv=\aEv$, and
%   % (2) if $\dEv\in\Event$ reads from $\aLoc$ then $\aEv\not\le\dEv$.

%   Let $\dEvs$ be a \emph{maximal disjoint read set} of $\aPS$  when 
%   (1) $\dEvs\subseteq\Event$,
%   (2) all $\aEv\in\dEvs$ are reads,
%   (3) if $\aEv\in\Event$ reads from $\aLoc$ then some $\dEv\in\dEvs$ reads from $\aLoc$, and
%   (4) if $\aEv\in\dEvs$ reads from $\aLoc$ then
%   (4a) if $\dEv\in\dEvs$ reads from $\aLoc$ then $\dEv=\aEv$, and
%   (4b) if $\dEv\in\Event$ reads from $\aLoc$ then $\aEv\not\le\dEv$.

%   Let $\drs(\aPS)$ be the set of maximal disjoint read sets of $\aPS$.

%   Let $\SUBDRS{\dEvs}$ be the substitution such
%   that $\SUBDRS{\dEvs}\SUBAPP\aLoc$ is $\aVal$ if some
%   $\dEv\in\SUBDRS{\dEvs}$ reads $\aVal$ from $\aLoc$, and is $\aLoc$
%   otherwise---where $\dEvs\in\drs(\aPS)$.
  
%   Let $\DISJUNCT(\aPSS)$ be the set $\aPSS''$ where $\aPS''\in\aPSS''$ if
%   there are $\aPS'\in\PRE(\aPS'')$ and $\aPS^i\in\aPSS$ such
%   $\Event' = \Event^i$, ${\le'}={\le^i}$, $\labelingAct'=\labelingAct^i$, and
%   $\labelingForm'(\aEv)$ implies $\bigvee_i \labelingForm^i(\aEv)$.
  
%   % Let $\aPS''$ be a \emph{disjunct} of $\aPS^i$ if there is some
%   % $\aPS'\in\PRE(\aPS'')$ such that $\Event' = \Event^i$, ${\le'}={\le^i}$,
%   % $\labelingAct'=\labelingAct^i$, and $\labelingForm'(\aEv)$ implies
%   % $\bigvee_i \labelingForm^i(\aEv)$.

%   Let $(\aPSS^1 \sequence \aPSS^2)$ be the set
%   $\{\aPS^1\in\aPSS^1\mid \disjoint{\Event^1}{\Sub}\}\cup \DISJUNCT(\aPSS')$
%   where $\aPS'\in\aPSS'$ when there are $\aPS^1 \in \aPSS^1$,
%   $\cEv\in\Event^1\cap\Sub$,
%   $\aSub=\labelingAct^1(\cEv)$, 
%   and $\aPS^2\in\aPSS^2$
%   such that the following hold.
%   Let $\bEv$ range over $\Event^1\setminus\Sub$, and $\aEv$ range over $\Event^2$.  
%   % For each $\aEv$, let $B_\aEv\subseteq\Event^1$ be a set of
%   % reads such that no two events read the same variable, and let $\aSub_\aEv$ be
%   % the associated substitution.
% \begin{enumerate}
% \item[1.] $\Event' = \Event^1\setminus\Sub \uplus \Event^2$,
% \item[2.] ${\le'}\supseteq{\le^1}\cup{\le^2}$, 
% %\item[2b.] if $\bEv\le\cEv$ then $\bEv\le'\aEv$, for $\aEv\in\Event^2\cap\Sub$,
% \item[3a.] $\labelingAct'(\bEv) = \labelingAct^1(\bEv)$,
% \item[3b.] $\labelingForm'(\bEv)$ implies $\labelingForm^1(\bEv)$,
% \item[4a1.] $\labelingAct'(\aEv) = \labelingAct^2(\aEv)$,  for $\aEv\in\Event^2\setminus\Sub$,
% \item[4a2.] $\labelingAct'(\aEv) =  \aSub\SUBSEQ\labelingAct^2(\aEv)$,  for $\aEv\in\Event^2\cap\Sub$, 
% % \item[4b.] if $\dEv\in\READS(\aPS^1)$ reads $\aVal$ from $\aLoc$, there is
% %   some $\bSub$ such that
% %   $\labelingForm'(\aEv)$ implies
% %   $\SUBPAR{\bSub}{\aVal/\aLoc}\SUBAPP(\aSub\SUBAPP\labelingAct^2(\aEv))$,
% \item[4bc.] if
%   $\dEvs\in\drs(\aPS^1)$ then $\labelingForm'(\aEv)$ implies
%   $\SUBDRS{\dEvs} \SUBAPP (\aSub\SUBAPP\labelingForm^2(\aEv))$, 
% % \item[4c.] 
% %   %if $\READS(\aPS^1)=\emptyset$
% %   if $\emptyset\in\drs(\aPS^1)$
% %   then $\labelingForm'(\aEv)$ implies
% %   $\labelingForm(\aEv)$,   
% \item[5a.] if $\aEv$ writes, there is some $\dEvs\in\READS(\aPS^1)$
%   such that when $\dEv\in\dEvs$, either $\dEv\le'\aEv$ or
%   $\dEv$ reads $\aLoc$ and $\labelingForm'(\aEv)$ implies  
%   $\SUBPAR{\SUBDRS{\dEvs}}{\aLoc/\aLoc} \SUBAPP (\aSub\SUBAPP\labelingForm^2(\aEv))$,
% % \item[----]  
% %  for every $\dEvs\in\drs(\aPS^1)$, either
% %   \begin{itemize}
% %   \item[4c]
% %     $\labelingForm'(\aEv)$ implies $\labelingForm(\aEv)$
% %     if $\labelingForm'(\aEv)$ implies
% %     $\aSub\SUBAPP\labelingForm^2(\aEv)$ and $\aEv$ writes,
% %     then for some $\dEvs\in\drs(\aPS^1)$ and every $\dEv\in \dEvs$,
% %     $\dEv\ORDER\aEv$, or
% %   \item[5a] if $\labelingForm'(\aEv)$ implies
% %     $\aSub\SUBAPP\labelingForm^2(\aEv)$ and $\aEv$ writes,
% %     then for some $\dEvs\in\drs(\aPS^1)$ and every $\dEv\in \dEvs$,
% %     $\dEv\ORDER\aEv$,    
% %   \end{itemize}
% \item[5b-f.] as before,
% % \item[5b.] if $\bEv$ and $\aEv$ are \external actions in conflict, then
% %   $\bEv\ORDER\aEv$,
% % \item[5c.] if $\bEv$ is an acquire or $\aEv$ is a release, then
% %   $\bEv \ORDER \aEv$,
% % \item[5d.] if $\bEv$ is an SC write and $\aEv$ is an SC read, then
% %   $\bEv \ORDER \aEv$, 
% % \item[5e.] if $\bEv$ reads and $\labelingAct(\aEv)=\DFS{\modeACQ}$, then
% %   $\aEv \ORDER \bEv$,
% % \item[5f.] if $\labelingAct(\bEv)=\DFS{\modeREL}$ and $\aEv$ writes, then
% %   $\bEv \ORDER \aEv$, 
% \item[6a.] if $\bEv$ is a release, $\dEv\in\Event'$ is an acquire,
%   $\bEv\le'\dEv\le'\aEv$, then $\labelingForm(\aEv)$ is location independent, and
% \item[6b.] if $\labelingAct(\bEv)=\DFW[\aLoc]{\amode}$, $\bEv\le\aEv$, and
%   $\aEv$ is a release that does not write $\aLoc$, then  some
%   $\dEv$ writes $\aLoc$ and $\bEv\le'\dEv\le'\aEv$. %such that $\dEv$ %(explicitly)
% \end{enumerate}
% \end{definition}


\begin{definition}
  \label{def:semi:seq}
  
%Let $\READS(\aPS)=\{\aEv\mid\aEv \text{ is a read}\}$.
%For $\dEvs\subseteq \READS(\aPS)$, let $\SUBDRS{\dEvs}$ be the derived

  % if
%   $\labelingAct^1(\bEv)
%   =\DRreg[\amode]{\aReg}{\beforeSub}{\aLoc}{\aVal}$, where if
%   $\aEv\in\Event^2\cap\Sub$ then $\aSub=\labelingAct^2(\aEv)$ otherwise $\aSub=\SUBEMP$,

% Let $\SUBDRS{\aPS}$ be a substitution of values for registers that is derived
% from the reads of $\aPS$ as follows: $\SUBDRS{\aPS}\SUBAPP\aReg$ is $\aVal$
% if some $\aEv$ reads $\aVal$ into $\aReg$, and is $\aReg$ otherwise.

  % Let $\dEvs$ be a \emph{disjoint read set} of $\aPS$  when 
  % $\dEvs\subseteq\Event$,
  % all $\aEv\in\dEvs$ are reads, and
  % if $\aEv\in\dEvs$ reads from $\aLoc$ then
  % (1) if $\dEv\in\dEvs$ reads from $\aLoc$ then $\dEv=\aEv$, and
  % (2) if $\dEv\in\Event$ reads from $\aLoc$ then $\aEv\not\le\dEv$.

  % Let $\dEvs$ be a \emph{maximal disjoint read set} of $\aPS$  when 
  % (1) $\dEvs\subseteq\Event$,
  % (2) all $\aEv\in\dEvs$ are reads,
  % (3) if $\aEv\in\Event$ reads from $\aLoc$ then some $\dEv\in\dEvs$ reads from $\aLoc$, and
  % (4) if $\aEv\in\dEvs$ reads from $\aLoc$ then
  % (4a) if $\dEv\in\dEvs$ reads from $\aLoc$ then $\dEv=\aEv$, and
  % (4b) if $\dEv\in\Event$ reads from $\aLoc$ then $\aEv\not\le\dEv$.

  % Let $\drs(\aPS)$ be the set of maximal disjoint read sets of $\aPS$.

  % Let $\SUBDRS{\dEvs}$ be the substitution such
  % that $\SUBDRS{\dEvs}\SUBAPP\aLoc$ is $\aVal$ if some
  % $\dEv\in\SUBDRS{\dEvs}$ reads $\aVal$ from $\aLoc$, and is $\aLoc$
  % otherwise---where $\dEvs\in\drs(\aPS)$.
  
  % Let $\aPS''$ be a \emph{disjunct} of $\aPS^i$ if there is some
  % $\aPS'\in\PRE(\aPS'')$ such that $\Event' = \Event^i$, ${\le'}={\le^i}$,
  % $\labelingAct'=\labelingAct^i$, and $\labelingForm'(\aEv)$ implies
  % $\bigvee_i \labelingForm^i(\aEv)$.

  Let $(\aPSS^1 \sequence \aPSS^2)$ be the set
  \begin{math}
    %\{\aPS^1{\in}\aPSS^1\mid \disjointsmall{\Event^1}{\Sub}\}
    \DISJUNCT(\aPSS^1
    \cup
    \aPSS')
  \end{math}
  where $\aPS'\in\aPSS'$ when there are $\aPS^1 \in \aPSS^1$,
  % $\cEv\in\Event^1\cap\Sub$,
  % $\aSubC=\labelingAct^1(\cEv)$, 
  $\aSubC=\labelingAct(\Event^1\cap\Sub)$, 
  %$\dEvs=\READS(\aPS^1)$,
  and $\aPS^2\in\aPSS^2$
  such that the following hold.

Let $\SUBDRS{\aPS}$ be a substitution of values for registers that is derived
from the reads of $\aPS^1$ as follows: $\SUBDRS{\aPS}\SUBAPP\aReg$ is $\aVal$
if some $\bEv\in\Event^1$ reads $\aVal$ into $\aReg$, and is $\aReg$ otherwise.

Let $\bEv$ range over $\Event^1\setminus\Sub$, and $\aEv$ range over $\Event^2$.  

  % For each $\aEv$, let $B_\aEv\subseteq\Event^1$ be a set of
  % reads such that no two events read the same variable, and let $\aSub_\aEv$ be
  % the associated substitution.

\begin{enumerate}
\item[1.] $\Event' = \Event^1\setminus\Sub \uplus \Event^2$,
\item[2.] ${\le'}\supseteq{\le^1}\cup{\le^2}$, 
\item[3a.] $\labelingAct'(\bEv) = \labelingAct^1(\bEv)$ before $\aPS^2$,
% \item[3a1.] $\labelingAct'(\bEv) = \labelingAct^1(\bEv)$ %, for $\bEv\in\Event^1\setminus\READS(\aPS^1)$,
% \item[3a2.] $\labelingAct'(\bEv) = \labelingAct^1(\bEv)
%   =\DRreg[\amode]{\aReg}{\beforeSub}{\aLoc}{\aVal}$,
%   if
%   $\labelingAct^1(\bEv)
%   =\DRreg[\amode]{\aReg}{\beforeSub}{\aLoc}{\aVal}$, where if
%   $\aEv\in\Event^2\cap\Sub$ then $\aSub=\labelingAct^2(\aEv)$ otherwise $\aSub=\SUBEMP$,
\item[3b.] $\labelingForm'(\bEv)$ implies $\labelingForm^1(\bEv)$,
\item[4a1.] $\labelingAct'(\aEv) = \labelingAct^2(\aEv)$  after $\aPS^1$,  for $\aEv\in\Event^2\setminus\Sub$,
\item[4a2.] $\labelingAct'(\aEv) = \aSubC\SUBSEQ\labelingAct^2(\aEv)$,  for $\aEv\in\Event^2\cap\Sub$, 
\item[4bc.] $\labelingForm'(\aEv)$ implies
  $(\SUBDRS{\aPS^1} \SUBSEQ \aSubC)\SUBAPP\labelingForm^2(\aEv)$, 
\item[5a.] if $\aEv$ writes and
  $\bEv=\DRreg[\amode]{\aReg}{\bSub}{\aLoc}{\aVal}$
  then either $\bEv\le'\aEv$ or
  $\labelingForm'(\aEv)$ implies  
  $(\SUBDRS{\aPS^1}\SUBSEQ \bSub)\SUBAPP\labelingForm^2(\aEv)$,
% \item[old5a.] if $\aEv$ writes 
%   and $\dEv\in\dEvs$, then either $\dEv\le'\aEv$ or
%   $\dEv$ reads from $\aLoc$ into $\aReg$ and $\labelingForm'(\aEv)$ implies  
%   $\SUBPAR{\SUBDRS{\dEvs}}{\aLoc/\aReg} \SUBAPP (\aSubC\SUBAPP\labelingForm^2(\aEv))$,
\item[5b-f.] as before (see \textsection\ref{sec:model} and
  \textsection\ref{sec:variants}). %generalizing 5e: if $\aEv\in\Sub$ then $\dEv\le'\aEv$,
% \item[6.] if $\bEv$ is a release, $\dEv\in\Event'$ is an acquire,
%   $\bEv\le'\dEv\le'\aEv$, then $\labelingForm(\aEv)$ is location independent, and
% \item[6b.] if $\labelingAct(\bEv)=\DFW[\aLoc]{\amode}$, $\bEv\le\aEv$, and
%   $\aEv$ is a release that does not write $\aLoc$, then  some
%   $\dEv$ writes $\aLoc$ and $\bEv\le'\dEv\le'\aEv$. %such that $\dEv$ %(explicitly)
% \item[1.] $\Event' = \Event^1\setminus\Sub \uplus \Event^2$,
% \item[2.] ${\le'}\supseteq{\le^1}\cup{\le^2}$, 
% \item[3a.] $\labelingAct'(\bEv) = \labelingAct^1(\bEv)$,
% \item[3b.] $\labelingForm'(\bEv)$ implies $\labelingForm^1(\bEv)$,
% \item[4a1.] $\labelingAct'(\aEv) = \labelingAct^2(\aEv)$,  for $\aEv\in\Event^2\setminus\Sub$,
% \item[4a2.] $\labelingAct'(\aEv) =  \aSubC\SUBSEQ\labelingAct^2(\aEv)$,  for $\aEv\in\Event^2\cap\Sub$, 
% \item[4bc.] $\labelingForm'(\aEv)$ implies
%   $(\SUBDRS{\dEvs} \SUBSEQ \aSubC)\SUBAPP\labelingForm^2(\aEv)$, 
% \item[5a.] if $\aEv$ writes 
%   and $\dEv\in\dEvs$, then either $\dEv\le'\aEv$ or
%   $\dEv$ reads from $\aLoc$ into $\aReg$ and $\labelingForm'(\aEv)$ implies  
%   $\SUBPAR{\SUBDRS{\dEvs}}{\aLoc/\aReg} \SUBAPP (\aSubC\SUBAPP\labelingForm^2(\aEv))$,
% \item[5b-f.] as before,
% \item[6a.] if $\bEv$ is a release, $\dEv\in\Event'$ is an acquire,
%   $\bEv\le'\dEv\le'\aEv$, then $\labelingForm(\aEv)$ is location independent, and
% \item[6b.] if $\labelingAct(\bEv)=\DFW[\aLoc]{\amode}$, $\bEv\le\aEv$, and
%   $\aEv$ is a release that does not write $\aLoc$, then  some
%   $\dEv$ writes $\aLoc$ and $\bEv\le'\dEv\le'\aEv$. %such that $\dEv$ %(explicitly)
\end{enumerate}
\end{definition}


% \begin{enumerate}
% \item[4b.] if $\bEv$ \externally reads $\aVal$ from $\aLoc$ then
%   $\labelingForm'(\aEv)$ implies $\labelingForm(\aEv)[\aVal/\aLoc]$,
% \item[4c.] if $\bEv$ does not \externally read then
%   $\labelingForm'(\aEv)$ implies $\labelingForm(\aEv)$, and
% \item[5a.] if %$\bEv$ \externally reads and
%   $\labelingForm'(\aEv)$ does not imply $\labelingForm(\aEv)$ and $\aEv$ writes, then
%   $\bEv\lt'\aEv$,
% \end{enumerate}

The item numbers are chosen to match those of the corresponding clauses in
\textsection\ref{sec:model}.  Item 5 is morally unchanged.  We only add
addition 5g, which requires that the accepting state be final (and therefore
excluded from any prefix).
Item 4 is more complicated here:
% Item 2b ensures that order into the substitution $\cEv\in\Event^1$ is
% preserved into any substitution in $\Event'$.
In item 4a, the label of the accepting state is calculated differently from
other states. Instead, items 4b and 4c collapse into a single item here.
% Item 6b is necessary to ensure the effect of $\relfilt{}$ in the local write
% rule\footnote{Because of this, we believe it is necessary to include silent
% actions for local writes when giving a semantics for sequential
% composition.}.
The requirement for a final write is imposed at top level---see below.

In item 4bc, note that the domain of $\aSubC$ is disjoint from the domain
of $\SUBDRS{\aPS^1}$, although registers in the domain of $\aSubC$ may
appear in the expressions in the codomain of $\SUBDRS{\aPS^1}$.

In item 5a, recall that $\bSub=\beforeSub\SUBSEQ\SUB{\aLoc/\aReg}\SUBSEQ\afterSub$ and
$(\beforeSub\SUBSEQ\afterSub) = \aSub$.
Note that
$\SUBDRS{\aPS^1}\SUBSEQ \bSub$ is insensitive to the value assigned to $\aReg$ by
$\SUBDRS{\aPS^1}$.

As a simple example, consider the following:
\begin{gather*}
  \begin{gathered}
    r\GETS y
    \\[-1ex]
    \hbox{\begin{tikzinline}[node distance=.2em]
      \event{a}{\DRreg{r}{\SUB{y/r}}{y}{1}}{}
      \final{f}{\SUBEMP}{below=of a}
      \end{tikzinline}}
  \end{gathered}
  \qquad
  \begin{gathered}
    x\GETS r
    \\[-1ex]
    \hbox{\begin{tikzinline}[node distance=.2em]
      \event{b}{r\EQ1 \mid \DW{x}{1}}{}
      \final{f}{r\EQ1 \mid \SUB{r/x}}{below=of b}
      \end{tikzinline}}
  \end{gathered}
  \qquad
  \begin{gathered}
    s\GETS x
    \\[-1ex]
    \hbox{\begin{tikzinline}[node distance=.2em]
      \event{c}{\DFR{}}{}
      \final{f}{\SUB{x/s}}{below=of c}
      \end{tikzinline}}
  \end{gathered}
  \qquad
  \begin{gathered}
    z\GETS s
    \\[-1ex]
    \hbox{\begin{tikzinline}[node distance=.2em]
      \event{d}{\DW{z}{1}}{}
      \final{f}{\SUB{s/z}}{below=of d}
      \end{tikzinline}}
  \end{gathered}
  \\
  % \begin{gathered}
  %   r\GETS y\SEMI x\GETS r
  %   \\[-1ex]
  %   \hbox{\begin{tikzinline}[node distance=1em]
  %       \event{a}{\DRreg{r}{}{y}{1}}{}
  %       \event{b}{\DW{x}{1}}{right=of a}
  %       \final{f}{\SUB{y/r, 1/x}}{right=of b}
  %       \po{a}{b}
  %     \end{tikzinline}}
  % \end{gathered}
  % \qquad
  % \begin{gathered}
  %   s\GETS x
  %   \\[-1ex]
  %   \hbox{\begin{tikzinline}[node distance=1em]
  %     \event{c}{\DFR{}}{}
  %     \final{f}{\SUB{x/s}}{right=of c}
  %     \po{c}{f}
  %     \end{tikzinline}}
  % \end{gathered}
  % \\
  \begin{gathered}
    r\GETS y\SEMI x\GETS r \SEMI s\GETS x \SEMI z\GETS s
    \\[-1ex]
    \hbox{\begin{tikzinline}[node distance=1em]
        \event{a}{\DRreg{r}{\SUB{y/r, r/x, x/s, s/z}}{y}{1}}{}
        \event{b}{\DW{x}{1}}{right=of a}
        \event{c}{\DFR{}}{right=of b}
        \event{d}{\DW{z}{1}}{right=of c}
        \final{f}{\SUB{r/x, x/s, s/z}}{right=of d}
        \po{a}{b}
        \po[out=-10,in=-160]{a}{d}
      \end{tikzinline}}
  \end{gathered}
\end{gather*}

To see the need for parallel substitution of all register values via $\SUBDRS{}$ in 4bc, consider that the
precondition of $\DW{y}{1}$ must be $\FALSE$ after composing:
\begin{gather*}
  \begin{gathered}
    r\GETS x\SEMI s\GETS z
    \\[-1ex]
    \hbox{\begin{tikzinline}[node distance=1em]
      \event{a}{\DRreg{r}{\SUB{x/r}}{x}{1}}{}
      \event{b}{\DRreg{s}{\SUB{z/s}}{z}{2}}{right=of a}
      \final{f}{\SUBEMP}{right=of b}
      \end{tikzinline}}
  \end{gathered}
  \qquad
  \begin{gathered}
     \IF{r{=}s}\THEN y\GETS1\FI
    \\[-1ex]
    \hbox{\begin{tikzinline}[node distance=1em]
      \event{c}{r{=}s\mid\DW{y}{1}}{}
      \final{f}{r{=}s\mid\SUB{1/y}}{right=of c}
      \end{tikzinline}}
  \end{gathered}
  \\
  \begin{gathered}
    r\GETS x\SEMI s\GETS z \SEMI \IF{r{=}s}\THEN y\GETS1\FI
    \\[-1ex]
    \hbox{\begin{tikzinline}[node distance=1em]
      \event{a}{\DRreg{r}{\SUB{x/r}}{x}{1}}{}
      \event{b}{\DRreg{s}{\SUB{z/s}}{z}{2}}{right=of a}
      \nonevent{c}{1{=}2\mid\DW{y}{1}}{right=of b}
      \nonfinal{f}{1{=}2\mid\SUB{1/y}}{right=of c}
      \end{tikzinline}}
  \end{gathered}
\end{gather*}
This pomset candidate does not satisfy the \emph{compatibility} requirement
of Definition \ref{def:mmpomset}.  Note, however, that
$\IF{r{=}s}\THEN y\GETS1\FI$ is shorthand for
$\IF{r{=}s}\THEN y\GETS1\ELSE\SKIP\FI$, and thus we do have a valid pomset
for this composition, even with this choice of read values:
\begin{gather*}
  \begin{gathered}
    r\GETS x\SEMI s\GETS z
    \\[-1ex]
    \hbox{\begin{tikzinline}[node distance=1em]
      \event{a}{\DRreg{r}{\SUB{x/r}}{x}{1}}{}
      \event{b}{\DRreg{s}{\SUB{z/s}}{z}{2}}{right=of a}
      \final{f}{\SUBEMP}{right=of b}
      \end{tikzinline}}
  \end{gathered}
  \qquad
  \begin{gathered}
     \IF{r{=}s}\THEN y\GETS1\FI
    \\[-1ex]
    \hbox{\begin{tikzinline}[node distance=1em]
      \final{f}{r{\neq}s\mid\SUBEMP}{}
      \end{tikzinline}}
  \end{gathered}
  \\
  \begin{gathered}
    r\GETS x\SEMI s\GETS z \SEMI \IF{r{=}s}\THEN y\GETS1\FI
    \\[-1ex]
    \hbox{\begin{tikzinline}[node distance=1em]
      \event{a}{\DRreg{r}{\SUB{x/r}}{x}{1}}{}
      \event{b}{\DRreg{s}{\SUB{z/s}}{z}{2}}{right=of a}
      \final{f}{1{\neq}2\mid\SUBEMP}{right=of b}
      \end{tikzinline}}
  \end{gathered}
\end{gather*}


The see the need for substitutions on read actions, used in 5a, consider the
following, where
$\bSub=\SUB{0/x,\allowbreak x/r,\allowbreak {-}2/x}$:
\begin{gather*}
  \begin{gathered}
    x\GETS 0\SEMI r\GETS x\SEMI x\GETS{-2} 
    \\[-1ex]
    \hbox{\begin{tikzinline}[node distance=1em]
      \event{a}{\DW{x}{0}}{}
      \event{b}{\DRreg{r}{\bSub}{x}{1}}{right=of a}
      \event{c}{\DW{x}{{-}2}}{right=of b}
      \wk{a}{b}
      \wk{b}{c}
      \final{f}{\SUB{{-2}/x}}{right=of c}
      \end{tikzinline}}
  \end{gathered}
  \quad
  \begin{gathered}
    \IF{r{\geq}0}\THEN y\GETS1\FI
    \\[-1ex]
    \hbox{\begin{tikzinline}[node distance=1em]
      \event{d}{r{\geq}0\mid\DW{y}{1}}{}
      \final{f}{r{\geq}0\mid\SUB{1/y}}{right=of d}
      \end{tikzinline}}
  \end{gathered}
  \\
  \begin{gathered}
    x\GETS 0\SEMI r\GETS x\SEMI x\GETS{-2} \SEMI \IF{r{\geq}0}\THEN y\GETS1\FI
    \\[-1ex]
    \hbox{\begin{tikzinline}[node distance=1em]
      \event{a}{\DW{x}{0}}{}
      \event{b}{\DRreg{r}{\SUBPAR{\bSub}{1/y}}{x}{1}}{right=of a}
      \event{c}{\DW{x}{{-}2}}{right=of b}
      \wk{a}{b}
      \wk{b}{c}
      \event{d}{0{\geq}0\mid\DW{y}{1}}{right=of c}
      \final{f}{0{\geq}0\mid\SUB{{-2}/x,1/y}}{right=of d}
      \end{tikzinline}}
  \end{gathered}
\end{gather*}


The semantics validates expected equations, such as
\begin{math}
  \sem{\aCmd_1\SEMI\aCmd_2}\sequence\sem{\aCmd_3} =
  \sem{\aCmd_1}\sequence\sem{\aCmd_2\SEMI\aCmd_3},
\end{math}
\begin{math}
  \sem{\IF{\aExp} \THEN \aCmd \SEMI\bCmd_1\ELSE \aCmd\SEMI\bCmd_2\FI} =
  \sem{\aCmd}\sequence\sem{\IF{\aExp} \THEN \bCmd_1 \ELSE \bCmd_2\FI},
\end{math}
and
\begin{math}
  \sem{\IF{\aExp} \THEN \aCmd_1 \SEMI\bCmd\ELSE\allowbreak \aCmd_2\SEMI\bCmd\FI} =
  \sem{\IF{\aExp} \THEN \aCmd_1 \ELSE \aCmd_2\FI}\sequence\sem{\bCmd}.
\end{math}


The semantics is equivalent to that of the main text.
Let $\killS\aPSS$ be the set $\aPSS'\subseteq\aPSS$ where $\aPS'\in\aPSS'$
every $\aEv'\in\Event'$ substitutes $\SUBEMP$.

\begin{proposition}
  \label{thm:seq}
  Let $\semold{\aCmd}$ be the semantics of \textsection\ref{sec:model},
  adopting the read actions of this section---the read substitution
  being $\SUBEMP$.
\begin{displaymath}
  \semold{\aCmd} = \killS\sem{\aCmd}
\end{displaymath}
% \begin{align*}
%   \semold{\aCmd} &= \{ \aPS\in\sem{\aCmd} \mid \disjoint{\Event}{\Sub}\}
%   \\
%   \sem{\aCmd_1\SEMI\aCmd_2}\sequence\sem{\aCmd_3} &= \sem{\aCmd_1}\sequence\sem{\aCmd_2\SEMI\aCmd_3}
%   \\
%   \hbox{\small$\sem{\IF{\aExp} \THEN \aCmd_1 \ELSE \aCmd_2\FI}\sequence\sem{\bCmd}$} &= 
%   \hbox{\small$\sem{\IF{\aExp} \THEN \aCmd_1 \SEMI\bCmd\ELSE \aCmd_2\SEMI\bCmd\FI}$}
% \end{align*}
\end{proposition}





% \begin{comment}
% Plan: 
%   a.  Define pom1; pom2  for pomsets
%   b.  [| C1 ; C2 |] = cup { pom1; pom2 | pom1 in C1, pom2 in C2}

% Def:
%    pom1; pom2
%              cup_L  L prefix pom2 
%              where L is a  linearization of pom1
% \end{comment}

% \section{Variations}

% \citet{2019-sp} define \emph{3-valued pomsets with preconditions} to model
% security flaws that arise from speculative evaluation in computer
% microarchitecture (such as Spectre \cite{DBLP:journals/corr/abs-1801-01203}).
% \begin{definition}
%   %\label{def:3valued}
%   A \emph{3-valued pomset with preconditions} is a tuple
%   $(\Event, {\le}, {\gtN}, \labeling)$, such that
%   \begin{itemize}
%   \item $\Event$ is a set of \emph{states},
%   \item $\labeling: \Event \fun (\Formulae\times\Act)$ is a \emph{labeling},
%   \item ${\le} \subseteq (\Event\times\Event)$ is a partial order, and
%   \item ${\gtN} \subseteq (\Event\times\Event)$ such that:
%     \begin{itemize}
%     \item\label{5a} if $\bEv \le \aEv$ then $\bEv \gtN \aEv$, \hfill
%       (Inclusion)
%     \item\label{5b} if $\bEv \le \aEv$ and $\aEv \gtN \bEv$ then
%       $\bEv = \aEv$, and \hfill (Consistency)
%     \item\label{5c} if $\cEv \le \bEv \gtN \aEv$ or $\cEv \gtN \bEv \le \aEv$
%       then $\cEv \gtN \aEv$.  \hfill (Semi-transitivity)
%     \end{itemize}
%   \end{itemize}

%   A \emph{(memory model) pomset} is a 3-valued pomset with preconditions,
%   such that
%   \begin{itemize}
%   \item if $\bEv\le\aEv$ then $\labelingForm(\aEv)$ implies
%     $\labelingForm(\bEv)$, and \hfill (Causal-strengthening)
%   \end{itemize}
% \end{definition}
% The axioms for $\gtN$ are adapted
% from \citet[A1--A3]{DBLP:journals/dc/Lamport86}.  

% \begin{definition}
%   A pomset \emph{coherent} if, when restricted to events that read or write
%   any single location $\aLoc$, $\gtN$ forms a partial order.
% \end{definition}

% \begin{figure}
% \begin{eqnarray*}
%   \sem{\SKIP}
%   & = & \TIKZ{\final{f}{}{}} 
%   \\
%   \sem{\aLoc\GETS\aExp}
%   & = & \textstyle\bigcup_\aVal\; \TIKZ{\event{a}{(\aExp=\aVal\mid\DW\aLoc\aVal)}{}\final{f}{\aExp/\aLoc}{right=of a}}
%   \\
%   \sem{\aReg\GETS\aLoc}
%   & = &
%   \TIKZ{\final{f}{\aLoc/\aReg}{}}
%   \cup
%   \textstyle\bigcup_\aVal\; \TIKZ{\event{a}{(\DR\aLoc\aVal)}{}\final{f}{\aLoc/\aReg}{right=of a}\po{a}{f}}
%   \\
%   \sem{\IF{\aExp} \THEN \aCmd \ELSE \bCmd \FI}
%   & = & \bigl((\aExp \neq 0) \guard \sem{\aCmd}\bigr) \parallel \bigl((\aExp=0) \guard \sem{\bCmd}\bigr)
%   \\
%   \sem{\aCmd \SEQ \bCmd}
%   & = & \sem{\aCmd} \sequence \sem{\bCmd}
%   \\
%   \sem{\aCmd \PAR \bCmd}
%   & = & \sem{\aCmd}\fork \parallel \sem{\bCmd}
%   \\
%   \sem{\VAR\aLoc\SEMI \aCmd}
%   & = & \nu \aLoc \DOT \sem{\aCmd}
% \end{eqnarray*}
% \caption{Semantics of a concurrent shared-memory language}
% \label{fig:semi:programs}
% \end{figure}

% \section{Sequential Composition}
% \label{sec:semi:model}
% \subsection{Data models}
% \label{sec:semi:preliminaries}

% A \emph{data model} consists of:
% \begin{itemize}
% \item a set of \emph{substitutions} $\Sub$, ranged over by
%   $\aSub$ and $\bSub$,
% \end{itemize}
% % Let $\LR=\Loc\cup\Reg$, be the set of \emph{locations}, ranged over by
% % $\aLR$.  Let $\Sub=\LR\partialfun\Exp$ be the set of \emph{substitutions},
% % ranged over by $\aSub$ and $\bSub$.
% Let $\aActSub$ and $\bActSub$ range over $(\Act\cup\Sub)$.
% % The empty substitution is written as $[\,]$.

% We require that data models satisfy the following:
% \begin{itemize}
% \item substitutions include at least $[\aLoc/\aReg]$ and $[\aExp/\aLoc]$,
% \item substitutions are closed under composition,
% \end{itemize}
% % By composition, it follows that substitutions must include $[\aExp/\aReg]$
% % which is equal to $[\aLoc/\aReg][\aExp/\aLoc]$.

% \subsection{Modal pomsets}
% \label{sec:semi:pomsets}

% We fix the alphabet $\Alphabet=(\Formulae\times(\Act\cup\Sub))$.  With this
% alphabet, the labeling of a pomset determines two disjoint sets of events.
% We refer to those that map to actions $\Sub$ as \emph{final}, and those that
% map to substitutions $\Act$ as \emph{nonfinal}.
% When we need to distinguish these, we let the set of events
% $\Event=\EAct\uplus\ESub$, where $\EAct$ denotes the nonfinal events 
% and $\ESub$ denotes the final events.

% We write pairs in $(\Formulae\times(\Act\cup\Sub))$ as $(\aForm \mid \aActSub)$.
% \subsection{Semantics of programs}
% \label{sec:semi:semantics}

% Semantics in Figure \ref{fig:semi:programs}

% We give the semantics using combinators over sets of pomsets, defined below.
% Using $\aPSS$ to range over sets of pomsets, these are:
% \begin{itemize}
% \item \emph{forking} $\aPSS\fork$, which removes final states from
%   $\aPSS$, 
% \item \emph{sequencing} $\aPSS_1\sequence\aPSS_2$, which prepends
%   $\aPSS_1$ to $\aPSS_2$, calculating dependencies between the two.
% \end{itemize}

% Forking is %and substitution each perform
% a simple transformation on each pomset
% in a set of pomsets.

% \begin{definition}
% Let $\aPSS\fork$ be the set $\aPSS'$ where $\aPS'\in\aPSS'$ whenever
% there is $\aPS\in\aPSS$ such that:
% $\Event' = \EAct$,
% $\labeling'\subseteq\labeling$,
% $\aEv\le'\bEv$ whenever $\aEv\le\bEv$, and
% $\aEv\gtN'\bEv$ whenever $\aEv\gtN\bEv$.
% \end{definition}

% \subsection{Sequencing}
% \begin{definition}
%   \label{def:semi:seq}
%   Let $\aPS' \in (\aPSS_1 \sequence \aPSS_2)$ whenever there are
%   $\aPS_1 \in \aPSS_1$ and $\aPS_{\cEv}\in\aPSS_2$, for every
%   $\cEv\in\ESub_1$, such that:
% \begin{itemize}
% \item $\Event' = \EAct_1 \cup \bigcup_{\cEv}\Event_{\cEv}$,
% \item if $\bEv \le_1 \aEv$ or $\bEv\le_{\cEv} \aEv$ then $\bEv \le' \aEv$,
% \item if $\bEv \gtN_1 \aEv$ or $\bEv\gtN_{\cEv} \aEv$ then $\bEv \gtN' \aEv$,
% \item if $\bEv\in\EAct_1$ then $\labeling'(\bEv) = \labeling_1(\bEv)$, and
% \item if $\labeling_1(\bEv) = (\dontcare \mid \aAct)$ and
%   $\labeling_{\cEv}(\aEv) = (\bForm \mid \bAct)$ then:
%   \begin{itemize}
%   \item if $\aAct$ is an acquire or $\bAct$ is a release then $\bEv \lt' \aEv$,
%   \item if $\aAct$ is an acquire then $\bForm$ is independent of every $\bLoc$,
%   \item either $\aAct$ and $\bAct$ are provably separable, or both are reads, or
%     $\bEv \gtN' \aEv$, and
%   \item if $\aAct$ and $\bAct$ conflict
%     then $\bEv \gtN' \aEv$, and
%   \end{itemize}
% \item 
%   if
%   $\labeling_1(\cEv) = (\dontcare \mid \aSub)$, and
%   $\labeling_{\cEv}(\aEv) = (\bForm \mid \bActSub)$ then
%   $\labeling'(\aEv) = (\bForm' \mid \bActSub\aSub)$ where:
%   \begin{itemize}
%   \item $\bForm'$ implies
%     $\lor\{\aForm\mid\bEv\in\ESub_1 \land \aEv\in\aPS\!_{\bEv}
%     \land \labeling_1(\bEv)=(\aForm{\mid}\dontcare)\}$,
%   \item $\bForm'$ implies $\bForm\aSub$, and
%   \item either $\bForm'$ implies $\bForm$ or $\bEv\lt_1\cEv$ implies $\bEv\lt'\aEv$.
%     %$\bEv\in\EAct_1$ such that.
%   \end{itemize}
% \end{itemize}
% \end{definition}


% \begin{comment}
% The first constraint ensures that events are ordered before a release and
% after an acquire.  The second constraint ensures that thread-local reads do
% not cross acquire fences.
 
% The second constraint prevents bad executions like the following:
%    x=1; rel; acq; if (x) {y=1};  ||  acq; x=0; rel; 
% where the second thread is interleaved between the rel and acq of the first.

% Note that you cannot require that $\bForm'$ is independent of every $\bLoc$
% because then it's not augment closed.
% \end{comment}


% % To see that we need $[\aExp/\aLoc]$ in the rule for write, rather than $[\aVal/\aLoc]$
% % consider example:
% % \begin{verbatim}
% % r=y; if (r) {x=r} else {x=r}; s=x; if (r==s) {z=1}
% % \end{verbatim}
% % or simplified:
% % \begin{verbatim}
% % r=y;x=r;s=x; if(s==r){z=1}
% % \end{verbatim}
% % If you read 37 for $y$, then the predicate on \texttt{Wz1} before the
% % read is either $r=r$ or $v=r$, where $v=37$, for example.  In one case you
% % get a dependency and in the other you do not.



% Local Variables:
% mode: latex
% TeX-master: "paper"
% End:

\end{document}

% Local Variables:
% mode: latex
% TeX-master: t
% End: