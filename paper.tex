% at most 12 pages, % excluding references. 
\documentclass[sigplan,10pt,
review,anonymous,
]{acmart}
\makeatletter\if@ACM@anonymous\acmSubmissionID{{}}\fi\makeatother %Including this in anonymous so page breaks are right for version with institutions
\settopmatter{printfolios=true}
\AtEndPreamble{%
  % \theoremstyle{acmdefinition}
  % \theoremstyle{plain}
  % \newtheorem{theorem}{Theorem}[section]
  % \newtheorem{proposition}[theorem]{Proposition}
  % \theoremstyle{acmdefinition}
  % \newtheorem{definition}[theorem]{Definition}
  \theoremstyle{acmdefinition}
  \newtheorem{remark}[theorem]{Remark}
  \newtheorem{candidate}[theorem]{Candidate}
  \renewcommand{\theequation}{\fnsymbol{equation}}
}
%% For single-blind review submission

% \documentclass[acmsmall]{acmart}\settopmatter{printfolios=true}
% \hypersetup{bookmarksnumbered,bookmarksopen=true,bookmarksdepth=2}
%% For final camera-ready submission
%\documentclass[acmsmall,10pt]{acmart}\settopmatter{}

%% Note: Authors migrating a paper from PACMPL format to traditional
%% SIGPLAN proceedings format should change 'acmsmall' to
%% 'sigplan'.

%\documentclass[acmsmall,screen]{acmart}


%% Some recommended packages.
\usepackage{booktabs}   %% For formal tables:
                        %% http://ctan.org/pkg/booktabs
\usepackage{subcaption} %% For complex figures with subfigures/subcaptions
                        %% http://ctan.org/pkg/subcaption



% \makeatletter\if@ACM@journal\makeatother
% %% Journal information (used by PACMPL format)
% %% Supplied to authors by publisher for camera-ready submission
% \acmJournal{PACMPL}
% \acmVolume{1}
% \acmNumber{1}
% \acmArticle{1}
% \acmYear{2017}
% \acmMonth{1}
% \acmDOI{10.1145/nnnnnnn.nnnnnnn}
% \startPage{1}
% \else\makeatother
% %% Conference information (used by SIGPLAN proceedings format)
% %% Supplied to authors by publisher for camera-ready submission
% \acmConference[PL'17]{ACM SIGPLAN Conference on Programming Languages}{January 01--03, 2017}{New York, NY, USA}
% \acmYear{2017}
% \acmISBN{978-x-xxxx-xxxx-x/YY/MM}
% \acmDOI{10.1145/nnnnnnn.nnnnnnn}
% \startPage{1}
% \fi


%% Copyright information
%% Supplied to authors (based on authors' rights management selection;
%% see authors.acm.org) by publisher for camera-ready submission
% \setcopyright{none}             %% For review submission
%\setcopyright{acmcopyright}
%\setcopyright{acmlicensed}
%\setcopyright{rightsretained}
%\copyrightyear{2017}           %% If different from \acmYear


%% Bibliography style
\bibliographystyle{ACM-Reference-Format}
%% Citation style
%% Note: author/year citations are required for papers published as an
%% issue of PACMPL.
\citestyle{acmauthoryear}   %% For author/year citations

% AFAICT this is the only way to get the copyright notice to be CC-BY-4.0
% which is the recommended license for POPL 2020.
\setcopyright{rightsretained}
\begin{makeatletter}
  \gdef\@copyrightpermission{%
    This paper is published under the Creative Commons Attribution~4.0
    International (CC-BY~4.0) license.
  }
\end{makeatletter}

\usepackage{macros}

% % at most 12 pages, excluding references, not anonymous
% % Titles and Short Abstracts Due	4 January 2019
% % Full Papers Due	11 January 2019
% % Author Feedback/Rebuttal Period	4–8 March 2019
% % Author Notification	29 March 2019
% % Conference	24–27 June 2019
% \documentclass[conference]{IEEEtran}
% \IEEEoverridecommandlockouts

\usepackage{macros}
\newcommand{\ignore}[1]{}
\newcommand{\todo}[1]{{\color{red}\textbf{\{#1\}}}}
\newcommand{\slt}{\vartriangleleft}
\newcommand{\sle}{\trianglelefteq}
\newcommand{\tvalpom}{modal pomset}
\newcommand{\tsem}[1]{\llbracket#1\rrbracket_{{\bf M}}}
\newcommand{\tsemsc}[1]{\llbracket#1\rrbracket_{{\bf MSC}}}
\newcommand{\tsemClosed}[1]{\closed\tsem{#1}}
\newcommand{\readc}{{\mathsf{read}}}

\begin{document}

%% Title information
\title{Let go of that which does not serve you}
\subtitle{Multi-Copy Atomicity and Semantic Independence in a Model of Relaxed Memory}
%\title[Let go of that which does not serve you]{Let go of that which does not serve you: Semantic~Independence in a Model of Relaxed Memory}
% \title{Semantic Dependency as a Model of Relaxed Memory}
% \title[Short Title]{Full Title}         %% [Short Title] is optional;
%                                         %% when present, will be used in
%                                         %% header instead of Full Title.
% \titlenote{with title note}             %% \titlenote is optional;
%                                         %% can be repeated if necessary;
%                                         %% contents suppressed with 'anonymous'

% \subtitle{Subtitle}                     %% \subtitle is optional
% \subtitlenote{with subtitle note}       %% \subtitlenote is optional;
%                                         %% can be repeated if necessary;
%                                         %% contents suppressed with 'anonymous'


%% Author information
%% Contents and number of authors suppressed with 'anonymous'.
%% Each author should be introduced by \author, followed by
%% \authornote (optional), \orcid (optional), \affiliation, and
%% \email.
%% An author may have multiple affiliations and/or emails; repeat the
%% appropriate command.
%% Many elements are not rendered, but should be provided for metadata
%% extraction tools.


\author{Radha Jagadeesan}
\affiliation{
  %\department{School of Computing}
  \institution{DePaul University}
  %\country{USA}
}
%\email{rjagadeesan@cs.depaul.edu}

\author{Alan Jeffrey}
\orcid{0000-0001-6342-0318}
\affiliation{
  \institution{Mozilla Research}
  %\country{USA}
}
%\email{ajeffrey@mozilla.com}

\author{James Riely}
\orcid{0000-0002-8731-1463}
\affiliation{
  %\department{School of Computing}
  \institution{DePaul University}
  %\country{USA}
}
%\email{jriely@cs.depaul.edu}


%% Paper note
\thanks{Radha Jagadeesan and James Riely were supported in part by NSF CCR}
%% The \thanks command may be used to createame a "paper note" ---
%% similar to a title note or an author note, but not explicitly
%% associated with a particular element.  It will appear immediately
%% above the permission/copyright statement.
%                                         %% can be repeated if necesary
%                                         %% contents suppressed with 'anonymous'


%% Abstract
%% Note: \begin{abstract}...\end{abstract} environment must come
%% before \maketitle command
\begin{abstract}
A memory model, designed to be as weak as possible.
\begin{itemize}
\item Compilation result (compile to C++, or LLVM, or hardware models) without fences
\item Soundness of optimizations (list from LDRF paper + those from promising
  semantics)  Follow dolan.
\item DRF programs have SC semantics
\item satisfies JMM test cases and other examples
\item supports a program logic and no-TAR
\item is compositional
\end{itemize}  

Other things:
\begin{itemize}
\item Full abstraction:  If two programs are different, then there is a
  distinguishing context of the form P; [] || Q; R
\item precisely models dependency
\end{itemize}


\end{abstract}


%% 2012 ACM Computing Classification System (CSS) concepts
%% Generate at 'http://dl.acm.org/ccs/ccs.cfm'.
\begin{CCSXML}
<ccs2012>
<concept>
<concept_id>10003752.10003753.10003761.10003762</concept_id>
<concept_desc>Theory of computation~Parallel computing models</concept_desc>
<concept_significance>500</concept_significance>
</concept>
% <concept>
% <concept_id>10003752.10003753.10003761.10003763</concept_id>
% <concept_desc>Theory of computation~Distributed computing models</concept_desc>
% <concept_significance>500</concept_significance>
% </concept>
% <concept>
% <concept_id>10003752.10003753.10003761.10003764</concept_id>
% <concept_desc>Theory of computation~Process calculi</concept_desc>
% <concept_significance>500</concept_significance>
% </concept>
% <concept>
% <concept_id>10003752.10003790.10002990</concept_id>
% <concept_desc>Theory of computation~Logic and verification</concept_desc>
% <concept_significance>500</concept_significance>
% </concept>
% <concept>
% <concept_id>10003752.10010124.10010131.10010133</concept_id>
% <concept_desc>Theory of computation~Denotational semantics</concept_desc>
% <concept_significance>500</concept_significance>
% </concept>TT
% <concept>
% <concept_id>10003752.10010124.10010131.10010134</concept_id>
% <concept_desc>Theory of computation~Operational semantics</concept_desc>
% <concept_significance>500</concept_significance>
% </concept>
% <concept>
% <concept_id>10003752.10010124.10010131.10010135</concept_id>
% <concept_desc>Theory of computation~Axiomatic semantics</concept_desc>
% <concept_significance>500</concept_significance>
% </concept>
<concept>
<concept_id>10003752.10010124.10010138.10011119</concept_id>
<concept_desc>Theory of computation~Abstraction</concept_desc>
<concept_significance>500</concept_significance>
</concept>
</ccs2012>
\end{CCSXML}

\ccsdesc[500]{Theory of computation~Parallel computing models}
% \ccsdesc[500]{Theory of computation~Distributed computing models}
% \ccsdesc[500]{Theory of computation~Process calculi}
% \ccsdesc[500]{Theory of computation~Logic and verification}
% \ccsdesc[500]{Theory of computation~Denotational semantics}
% \ccsdesc[500]{Theory of computation~Operational semantics}
% \ccsdesc[500]{Theory of computation~Axiomatic semantics}
\ccsdesc[500]{Theory of computation~Abstraction}
 

%% End of generated code


%% Keywords
%% comma separated list
\keywords{Relaxed Memory Models, Hardware Transactional Memory}  %% \keywords is optional


%% \maketitle
%% Note: \maketitle command must come after title commands, author
%% commands, abstract environment, Computing Classification System
%% environment and commands, and keywords command.
\maketitle

% \title{What Happened}

% \author{
% \IEEEauthorblockN{Radha Jagadeesan}
% \IEEEauthorblockA{\textit{DePaul University}\\
%   \email{rjagadeesan@cs.depaul.edu}}
% \and
% \IEEEauthorblockN{Alan Jeffrey}
% \IEEEauthorblockA{\textit{Mozilla Research}\\
%   \email{ajeffrey@mozilla.com}}
% \and
% \IEEEauthorblockN{James Riely}
% \IEEEauthorblockA{\textit{DePaul University}\\
%   \email{jriely@cs.depaul.edu}}
% }

\section{Introduction}
\label{sec:intro}

Fixing the model in a way that forbids all “out-of-thin- air” behaviors and still allows the most efficient compilation is beyond the scope of the current paper. 

We present a model with the following properties:
\begin{itemize}
\item Compilation result (compile to C++, or LLVM, or hardware models) without fences
\item Soundness of optimizations (list from LDRF paper + those from promising
  semantics)  Follow dolan.
\item DRF programs have SC semantics
  
\item Full abstraction:  If two programs are different, then there is a
  distinguishing context of the form P; [] || Q; R
\item precisely models dependency
\item is compositional
\item satisfies JMM test cases and other examples
\item supports a program logic and no-TAR
\end{itemize}

% Local Variables:
% mode: latex
% TeX-master: "paper"
% End:
\section{Model}
\label{sec:model}
\subsection{Data models}
\label{sec:preliminaries}

A \emph{data model} consists of:
\begin{itemize}
\item a set of \emph{values} $\Val$, ranged over by
  $\aVal$ and $\bVal$,
\item a set of \emph{registers} $\Reg$, ranged over by
  $\aReg$ and $\bReg$,
\item a set of \emph{expressions} $\Exp$, ranged over by
  $\aExp$ and $\bExp$,
\item a set of \emph{memory locations} $\Loc$, ranged over by
  $\aLoc$ and $\bLoc$,
\item a set of \emph{logical formulae} $\Formulae$, ranged over by
  $\aForm$ and $\bForm$, and
\item a set of \emph{actions} $\Act$, ranged over by $\aAct$ and $\bAct$,
\end{itemize}
Let $\aSub$ range over substitutions of the form $\aForm[\aLoc/\aReg]$ or $\aForm[\bExp/\aLoc]$.

We require that data models satisfy the following:
\begin{itemize}
\item values include at least the constants $0$ and $1$,
\item expressions include at least registers and values,
\item expressions do not include memory locations,
\item formulae include at least $\TRUE$, $\FALSE$, and equalities of the form $(\aExp=\bExp)$,
\item formulae are closed under negation, conjunction, disjunction, and
  substitution, 
\item there is a relation $\vDash$ between formulae, 
\item there are partial functions $\rreads$ and $\rwrites: \Act \fun (\Loc
  \times \Val)$, and
\item there are sets $\Rel$ and $\Acq \subseteq\Act$.
\end{itemize}
Since formulae are closed under substitutions of the form
$\aForm[\aLoc/\aReg]$, they must include equalities of the form
$(\aEExp=\bEExp)$, where $\aEExp$ and $\bEExp$ are \emph{extended
  expressions} that include memory locations.  We elide the details.
(By composition of the closure conditions, formulae must also be closed
under that substitutions of the form $\aForm[\aExp/\aReg]=\aForm[\aLoc/\aReg][\aExp/\aLoc]$.)

We say that $\aAct$ \emph{reads} $\aVal$ \emph{from} $\aLoc$ whenever
$\rreads(\aAct) = (\aLoc,\aVal)$, and that $\aAct$ \emph{writes} $\aVal$
\emph{to} $\aLoc$ whenever $\rwrites(\aAct) = (\aLoc,\aVal)$.  In examples,
the actions are of the form $(\DR{\aLoc}{\aVal})$, which reads $\aVal$ from
$\aLoc$, and $(\DW{\aLoc}{\aVal})$, which writes $\aVal$ to $\aLoc$.

We say that $\aAct$ is an \emph{acquire} if $\aAct\in\Acq$, and that $\aAct$
is a \emph{release} if $\aAct\in\Rel$.  We say that $\aAct$ is a \emph{fence}
if it is either a release or an acquire.  In examples, the actions are of the
form $(\DRAcq{\aLoc}{\aVal})$, which is an acquire that reads $\aVal$ from
$\aLoc$, $(\DWRel{\aLoc}{\aVal})$, which is a release that writes $\aVal$ to
$\aLoc$, and $(\DF)$, which is both an acquire and a release and neither
reads nor writes.

We say that $\aForm$ \emph{implies} $\bForm$ whenever $\aForm\vDash\bForm$,
that $\aForm$ is a \emph{tautology} whenever $\TRUE\vDash\aForm$, that
$\aForm$ is \emph{unsatisfiable} whenever $\aForm\vDash\FALSE$, and that
$\aForm$ is \emph{independent of $\aLoc$} whenever
$\aForm \vDash \aForm[\aVal/\aLoc] \vDash \aForm$ for every $\aVal$.


\subsection{3-valued pomsets}
\label{sec:pomsets}

\begin{definition}
  A \emph{(3-valued) pomset} with alphabet $\Alphabet$ is tuple $(\Event,
  {\le}, {\gtN}, \labelling)$, such that 
  \begin{itemize}
  \item $\Event$ is a set of \emph{states},
  %\item $\ESub\subseteq\Event$ is a set of \emph{accepting states}, 
  \item $\labelling: \Event \fun \Alphabet$ is a \emph{labelling},
  \item ${\le} \subseteq (\Event\times\Event)$ is a partial order, and
    % \begin{enumerate}
    % \item $\aEv \le \aEv$,
    % \item if $\bEv \le \aEv$ and $\aEv \le \bEv$ then $\bEv = \aEv$,
    %   \\(this follows from 5a and 5b)
    % \item if $\cEv \le \bEv \le \aEv$ then $\cEv \le \aEv$, and
    % \end{enumerate}
  \item ${\gtN} \subseteq (\Event\times\Event)$ such that:
    \begin{itemize}
    \item\label{5a} if $\bEv \le \aEv$ then $\bEv \gtN \aEv$,
    \item\label{5b} if $\bEv \le \aEv$ and $\aEv \gtN \bEv$ then $\bEv = \aEv$,
    \item if $\cEv \le \bEv \gtN \aEv$ or $\cEv \gtN \bEv \le \aEv$ then $\cEv \gtN \aEv$.
    \end{itemize}
\end{itemize}
\end{definition}
In the remainder of the paper, we drop the prefix ``3-valued'', referring to
3-valued pomsets simply as \emph{pomsets}.

We fix the alphabet $\Alphabet=(\Formulae\times\Act)$.  We write pairs in
$(\Formulae\times\Act)$ as $(\aForm \mid \aAct)$.  We elide $\aForm$ when
$\aForm$ is a tautology.  Define $\labellingForm$ and $\labellingAct$ so that
$\labellingForm(\aEv)=\aForm$ and $\labellingAct(\aEv)=\aAct$ whenever
$\labellingForm(\aEv)=(\aForm\mid\aAct)$.

We lift terminology from logical formulae and actions to events, for example
if $\labelling(\aEv)=(\aForm\mid\aAct)$ then we say $\aEv$ is unsatisfiable
whenever $\aForm$ is unsatisfiable, $\aEv$ writes $\aVal$ to $\aLoc$ whenever
$\aAct$ writes $\aVal$ to $\aLoc$, and so forth.

We visualize a pomset as a graph where the nodes are drawn from
$\Event$, each node $\aEv$ is labelled with $\labelling(\aEv)$,
and an edge $\bEv \rightarrow \aEv$ corresponds to an ordering
$\bEv\le\aEv$. For example:
\[\begin{tikzpicture}[node distance=1em]
  \event{rx1}{\DR{\aLoc}{1}}{}
  \nonevent{wy0}{\DW{\bLoc}{0}}{right=of rx1}
  \event{wy1}{\DW{\bLoc}{1}}{right=of wy0}
  \po{rx1}{wy0}
  \po[out=30,in=150]{rx1}{wy1}
\end{tikzpicture}\]
is a visualization of the pomset where:
\[\begin{array}{c}
  E = \{ 0,1,2 \} \quad
  0 \le 1 \quad
  0 \le 2 \quad
  \labelling(0) = (\TRUE, \DR{\aLoc}{1}) \\
  \labelling(1) = (\FALSE, \DW{\bLoc}{0}) \quad
  \labelling(2) = (\TRUE, \DW{\bLoc}{1}) \quad
\end{array}\]
We visualize $(\bEv \gtN \aEv)$ as a dashed
arrow from $\bEv$ to $\aEv$.
We refer to edges introduced by $(\bEv < \aEv)$ as
\emph{strong edges} and by $(\bEv \gtN \aEv)$
as \emph{weak edges}.
For readability, we often highlight the reads-from edges as well.
% for example:
For example:
\[\begin{tikzpicture}[node distance=1em]
  \event{wx0}{\DW{\aLoc}{0}}{}
  \event{wx1}{\DW{\aLoc}{1}}{right=of wx0}
  \event{rx1}{\DR{\aLoc}{1}}{right=2.5 em of wx1}
  \event{wx2}{\DW{\aLoc}{2}}{right=of rx1}
  \rf{wx1}{rx1}
  \wk{wx0}{wx1}
  \wk{rx1}{wx2}
\end{tikzpicture}\]

\subsection{Semantics of programs}
\label{sec:semantics}

\begin{figure*}
\begin{eqnarray*}
  \sem{\SKIP} & = & \{ \emptyset \} \\
  \sem{\FENCE\SEMI \aCmd} & = & (\DF) \prefix \sem{\aCmd} \\
  \sem{\aLoc\GETS\aExp\SEMI \aCmd} & = & \textstyle\bigcup_\aVal\; \bigl((\aExp=\aVal) \guard (\DW\aLoc\aVal) \prefix \sem{\aCmd}[\aExp/\aLoc]\bigr) \\
  \sem{\REL\aLoc\GETS\aExp\SEMI \aCmd} & = & \textstyle\bigcup_\aVal\; \bigl((\aExp=\aVal) \guard (\DWRel\aLoc\aVal) \prefix \sem{\aCmd}[\aExp/\aLoc]\bigr) \\
  \sem{\aReg\GETS\aLoc\SEMI \aCmd} & = & \sem{\aCmd}[\aLoc/\aReg] \cup \textstyle\bigcup_\aVal\; (\DR\aLoc\aVal) \prefix \sem{\aCmd}[\aLoc/\aReg] \\
  \sem{\ACQ\aReg\GETS\aLoc\SEMI \aCmd} & = & \textstyle\bigcup_\aVal\; (\DRAcq\aLoc\aVal) \prefix \sem{\aCmd}[\aLoc/\aReg] \\
  \sem{\IF (\aExp) \THEN \aCmd \ELSE \bCmd \FI} & = & \bigl((\aExp \neq 0) \guard \sem{\aCmd}\bigr) \parallel \bigl((\aExp=0) \guard \sem{\bCmd}\bigr) \\
  \sem{\aCmd \PAR \bCmd} & = & \sem{\aCmd} \parallel \sem{\bCmd} \\
  \sem{\VAR\aLoc\SEMI \aCmd} & = & \nu \aLoc \st \sem{\aCmd}
\end{eqnarray*}
\caption{Semantics of a concurrent shared-memory language}
\label{fig:programs}
\end{figure*}

In Figure~\ref{fig:programs}, we give the semantics of a simple shared-memory
concurrent language as sets of pomsets.  
Each pomset
$\aPS\in\sem{\aCmd}$ represents a single execution of $\aCmd$.  We do not
expect $\sem{\aCmd}$ to be prefixed closed; thus, one may view each
$\aPS\in\sem{\aCmd}$ as a \emph{completed} execution.  However, the sets of
pomsets given by our semantics \emph{are} closed with respect to
isomorphism and augmentation.

NOTE: because augment closed, any event can go false, and we kill everything
after it, so that means we do get a kind of prefix closure.

In this paper, we are not investigating speculative execution.  So we make
the global assumption formulae can only get stronger in dependent actions:
if $\aEv<\bEv$ then
$\labellingForm(\bEv)$ implies $\labellingForm(\aEv)$.
 
\begin{definition}
  $\aPS'$ is an \emph{isomorphism} of $\aPS$ if there is a bijection
  $f:\Event\fun\Event'$ such that
  % \begin{itemize}
  % \item
  $\labelling(\aEv)=\labelling'(f(\aEv))$,
  % \item
  $\aEv\le\bEv$ iff $f(\aEv)\le'f(\bEv)$, and
  % \item
  $\aEv\gtN\bEv$ iff $f(\aEv)\gtN'f(\bEv)$.
  % \end{itemize}
\end{definition}
Augmentation may create additional order and strengthening
preconditions.
\begin{definition}
  $\aPS'$ is an augmentation of $\aPS$ if $\Event'=\Event$,
  ${\le}\subseteq{\le'}$, %$\aEv\le\bEv$ implies $\aEv\le'\bEv$,
  ${\gtN}\subseteq{\gtN'}$, %$\aEv\gtN\bEv$ implies $\aEv\gtN'\bEv$, 
  $\labellingAct'=\labellingAct$, and %$\labellingAct'(\aEv)=\labellingAct(\aEv)$ and
  $\labellingForm'(\aEv)$ implies $\labellingForm(\aEv)$.
  % if $\labelling(\aEv) = (\bForm \mid \bAct)$ then
  % $\labelling'(\aEv) = (\bForm' \mid \bAct)$ where $\bForm'$ implies
  % $\bForm$.
\end{definition}

We give the semantics using combinators over sets of pomsets, defined below.
Using $\aPSS$ to range over sets of pomsets, these are:
\begin{itemize}
\item \emph{substitution} $\aPSS\aSub$, which applies substitution $\aSub$ to
  every precondition of $\aPSS$,
\item \emph{guarding} $\aForm\guard\aPSS$, which filters $\aPSS$,
  keeping pomsets whose events have preconditions that imply $\aForm$,
\item \emph{restriction} $\nu\aLoc\st\aPSS$, which filters $\aPSS$ to include
  only pomsets where every event $\aEv$ that reads from $\aLoc$ can read from
  some $\bEv$ and where no precondition can depend on $\aLoc$,
\item \emph{composition} $\aPSS_1\parallel\aPSS_2$, which unions pomsets from
  $\aPSS_1$ and $\aPSS_2$, allowing events to be merged, and
\item \emph{prefixing} $\aAct\prefix\aPSS$, which adds an event with action
  $\aAct$ to pomsets in $\aPSS$, ordering $\aAct$ before any $\aEv$ whose predicate
  depends on the value read by $\aAct$.
\end{itemize}
These operations are similar to those from models of concurrency such
as~\cite{Brookes:1984:TCS:828.833}, but adapted here to the setting of
speculative evaluation.

%% A write generates a write event that may be visible
%% to other threads.  A read may see a
%% thread-local value, or it may generate a read event that must be justified by
%% another thread.  In the latter case, occurrences of $\aReg$ are replaced with
%% $\aLoc$ (rather than $\aVal$) to ensure that dependencies are tracked
%% properly.  The subsequent substitution of $\aVal$ for $\aLoc$ occurs in
%% Definition~\ref{def:prefix} of prefixing.

% We have completed the formal definition of our model of speculative
% evaluation, and now turn to examples.

\subsection{Substitution and Guarding} 

Substitution updates the preconditions in a pomset, thus we expect the number
of pomsets to be unchanged; in addition, the number of events in each of the
pomsets is unchanged.
\begin{definition}
  %For a substitution $\aSub$, of the form $[\aLoc/\aReg]$ or $[\bExp/\aLoc]$,
  Let $\aPSS\aSub$ be the set $\aPSS'$ where $\aPS'\in\aPSS'$ whenever
there is $\aPS\in\aPSS$ such that:
$\Event' = \Event$,
${\le'} = {\le}$, 
${\gtN'} = {\gtN}$,
and
$\labelling'(\aEv) = (\bForm\aSub \mid \aAct)$, where $\labelling(\aEv) = (\bForm \mid \aAct)$.
% \begin{itemize}
% \item if $\labelling(\aEv) = (\bForm \mid \aAct)$ then $\labelling'(\aEv) =
%   (\bForm\aSub \mid \aAct)$, and
% \item if $\labelling(\aEv) = (\bForm \mid \aSub)$ then $\labelling'(\aEv) = (\bForm\bSub \mid \aSub\bSub)$.
%\end{itemize}
\end{definition}

Guarding filters a set of pomsets; we have
$(\aForm\guard\aPSS)\subseteq\aPSS$.
The definition is straightforward:
\begin{definition}
Let $(\aForm \guard \aPSS)$ be the subset of $\aPSS$ such that $\aPS\in\aPSS$ whenever
$\aForm$ implies $\labellingForm(\aEv)$.
% \begin{itemize}
% \item if $\labelling(\aEv) = (\bForm \mid \aActSub)$ then $\aForm$ implies $\bForm$.
% \end{itemize}
\end{definition}

\subsection{Restriction}
\label{sec:restriction}

% Restriction also filters a set of pomsets; we have
% $(\nu\aLoc\st\aPSS)\subseteq\aPSS$.
% The definition requires that we define
% when a read is possible.

\begin{definition}\label{def:rf}
  In a pomset, $\aEv$ \emph{can read $\aLoc$ from} $\bEv$ whenever: 
  \begin{itemize}
  \item $\bEv < \aEv$,  
  \item $\aEv$ implies $\bEv$,
  \item $\bEv$ writes $\aVal$ to $\aLoc$,
    and $\aEv$ reads $\aVal$ from $\aLoc$, and
  \item if $\cEv$ writes to $\aLoc$
    then either $\cEv \gtN \bEv$ or $\aEv \gtN \cEv$.
  \end{itemize}
\end{definition}


\begin{definition}
\label{def:x-closed}
A pomset is $\aLoc$-closed if, for every $\aEv\in\Event$:
  \begin{itemize}
  \item $\aEv$ is independent of $\aLoc$, and
  \item if $\aEv$ reads $\aLoc$, then there is some $\bEv$ such that $\aEv$
    can read $\aLoc$ from $\bEv$,
  \item if $\aEv$ writes $\aLoc$, then either $\aEv$ is a fence or there is
    some $\bEv$ such that $\bEv$ can read $\aLoc$ from $\aEv$.
  \end{itemize}
\end{definition}
% Our model of reads-from is strong, and could be weakened by replacing the
% requirement $\bEv<\aEv$ % in Definition~\ref{def:rf}
% by $\bEv\gtN\aEv$. It remains to be seen how this impacts the model.

% \begin{definition}
%   A 3-valued pomset is $\aLoc$-\emph{coherent}
%   if, when restricted to events which touch $\aLoc$,
%   $\gtN$ forms a total order.
% \end{definition}

We say that $\aPS' = \aPS\restrict{\Event'}$ when 
 $\Event' \subseteq \Event$,
 ${\labelling'} = {\labelling}\restrict{\Event'}$, 
 ${\le'} = {\le}\restrict{\Event'}$, and
 ${\gtN'} = {\gtN}\restrict{\Event'}$.

\begin{definition}
  Let $(\nu\aLoc\st\aPSS)$ be the set $\aPSS'$ where $\aPS'\in\aPSS'$
  whenever there is $\aPS\in\aPSS$ such that $\aPS' = \aPS\restrict{\Event'}$
  and $\aPS'$ is $\aLoc$-closed.
  % Let $(\nu\aLoc\st\aPSS)$ be the subset of $\aPSS$ such that $\aPS\in\aPSS$ whenever
  % \begin{itemize}
  % \item $\aEv$ is independent of $\aLoc$, and
  % \item if $\aEv$ reads $\aLoc$, then there is some $\bEv$ such that $\aEv$ can read $\aLoc$ from $\bEv$.
  % \end{itemize}
\end{definition}
This definition throws away useless writes.
I don't think we need coherence any more.

\subsection{Composition}
Composition is used in giving the semantics for conditionals and concurrency.
$\aPSS_1 \parallel \aPSS_2$ contains the union of pomsets from $\aPSS_1$ and
$\aPSS_2$, allowing overlap as long as they agree on actions. For example, if
$\aPSS_1$ and $\aPSS_2$ contain:
\[\begin{tikzpicture}[node distance=1em]
  \event{a}{\aForm \mid \aAct}{}
  \event{b}{\bForm_1 \mid \bAct}{right=of a}
  \po{a}{b}
\end{tikzpicture}\qquad\qquad\begin{tikzpicture}[node distance=1em]
  \event{b}{\bForm_2 \mid \bAct}{}
  \event{c}{\cForm \mid \cAct}{right=of b}
  \wk{b}{c}
\end{tikzpicture}\]
then $\aPSS_1 \parallel \aPSS_2$ contains:
\[\begin{tikzpicture}[node distance=1em]
  \event{a}{\aForm \mid \aAct}{}
  \event{b}{\bForm_1 \lor \bForm_2 \mid \bAct}{right=of a}
  \event{c}{\cForm \mid \cAct}{right=of b}
  \po{a}{b}
  \wk{b}{c}
\end{tikzpicture}\]

\begin{definition}
Let $\aPS' \in (\aPSS^1 \parallel \aPSS^2)$
whenever there are $\aPS^1 \in \aPSS^1$ and $\aPS^2 \in \aPSS^2$ such that:
\begin{itemize}
\item $\Event' = \Event^1 \cup \Event^2$,
\item ${\le'}\supseteq{\le^1}\cup{\le^2}$, %if $\aEv \le^1 \bEv$ or $\aEv \le^2 \bEv$ then $\aEv \le' \bEv$,
\item ${\gtN'}\supseteq{\gtN^1}\cup{\gtN^2}$, %if $\aEv \gtN^1 \bEv$ or $\aEv \gtN^2 \bEv$ then $\aEv \gtN' \bEv$,
% \item if $\labelling'(\aEv) = (\aForm' \mid \aAct)$ then either:
%   \begin{itemize}
%   \item $\labelling^1(\aEv) = (\aForm^1 \mid \aAct)$ and $\labelling^2(\aEv) = (\aForm^2 \mid \aAct)$
%     and $\aForm'$ implies $\aForm^1 \lor \aForm^2$,
%   \item $\labelling^1(\aEv) = (\aForm^1 \mid \aAct)$ and $\aEv \not\in \Event^2$
%     and $\aForm'$ implies $\aForm^1$, or
%   \item $\labelling^2(\aEv) = (\aForm^2 \mid \aAct)$ and $\aEv \not\in \Event^1$
%     and $\aForm'$ implies $\aForm^2$.
%   \end{itemize}
\item Either
  % \begin{gather*}
  %   \labellingAct'(\aEv) = \labellingAct^1(\aEv) = \labellingAct^2(\aEv) \textand \labellingForm'(\aEv) \textimplies \labellingForm^1(\aEv) \lor \labellingForm^2(\aEv),\\
  %   \aEv \not\in \Event^2,\; \labellingAct'(\aEv) = \labellingAct^1(\aEv) \textand \labellingForm'(\aEv) \textimplies \labellingForm^1(\aEv),\; \textor\\    
  %   \aEv \not\in \Event^1,\; \labellingAct'(\aEv) = \labellingAct^2(\aEv) \textand \labellingForm'(\aEv) \textimplies \labellingForm^2(\aEv).
  % \end{gather*}
  \begin{itemize}
  \item $\labellingAct'(\aEv) = \labellingAct^1(\aEv) = \labellingAct^2(\aEv)
    \textand \labellingForm'(\aEv) \textimplies \labellingForm^1(\aEv) \lor \labellingForm^2(\aEv)$,
  \item $\labellingAct'(\aEv) = \labellingAct^1(\aEv),\;\; \aEv \not\in \Event^2\,
    \textand \labellingForm'(\aEv) \textimplies \labellingForm^1(\aEv),\; \textor$
  \item $\labellingAct'(\aEv) = \labellingAct^2(\aEv),\;\; \aEv \not\in \Event^1\,
    \textand \labellingForm'(\aEv) \textimplies \labellingForm^2(\aEv)$.
  \end{itemize}
\end{itemize}
\end{definition}
% We use $\aPSS_1 \parallel \aPSS_2$ in defining the semantics of conditionals
% and concurrency.
% It contains the union of pomsets from $\aPSS_1$ and $\aPSS_2$,
% allowing overlap as long as they agree on actions. For example, if
% $\aPSS_1$ and $\aPSS_2$ contain:
% \[\begin{tikzpicture}[node distance=1em]
%   \event{a}{\aForm \mid \aAct}{}
%   \event{b}{\bForm_1 \mid \bAct}{right=of a}
%   \po{a}{b}
% \end{tikzpicture}\qquad\qquad\begin{tikzpicture}[node distance=1em]
%   \event{b}{\bForm_2 \mid \bAct}{}
%   \event{c}{\cForm \mid \cAct}{right=of b}
%   \wk{b}{c}
% \end{tikzpicture}\]
% then $\aPSS_1 \parallel \aPSS_2$ contains:
% \[\begin{tikzpicture}[node distance=1em]
%   \event{a}{\aForm \mid \aAct}{}
%   \event{b}{\bForm_1 \lor \bForm_2 \mid \bAct}{right=of a}
%   \event{c}{\cForm \mid \cAct}{right=of b}
%   \po{a}{b}
%   \wk{b}{c}
% \end{tikzpicture}\]


\subsection{Prefixing}
Prefixing is used in giving the semantics of reads and writes.
$\aAct\prefix\aPSS$ adds a new event $\cEv$ with action $\aAct$ to each
pomset in $\aPSS$.  As in the definition of parallel composition, the
definition allows the new event to overlap with events in $\aPSS$ as long as
they agree on the action.

If $\cEv$ writes to a location that is also written by $\aEv$ in $\aPSS$,
then prefixing introduces weak order between them: $\aEv \gtN \cEv$.  This
ensures that these writes cannot be given the reverse order in an augmentation.

If $\cEv$ reads from a location that occurs in the predicate of $\aEv$, then
prefixing introduces order from $\cEv$ to $\aEv$.
whose predicate depends on $\aLoc$. 
For example, if $\aAct$ and $\bAct$ write to the same location, $\aAct$ reads
$\aVal$ from $\aLoc$, $\bForm$ is independent of $\aLoc$, and $\aPSS$
contains:
\[\begin{tikzpicture}[node distance=1em]
  \event{b}{\bForm \mid \bAct}{}
  \event{c}{\cForm \mid \cAct}{right=of b}
  \po{b}{c}
\end{tikzpicture}\]
then $\aAct\prefix\aPSS$ contains:
\[\begin{tikzpicture}[node distance=1em]
  \event{a}{\aForm \mid \aAct}{}
  \event{b}{\bForm \mid \bAct}{right=of a}
  \event{c}{\cForm[\vec\aVal/\vec\aLoc] \mid \cAct}{right=of b}
  \po[out=25,in=155]{a}{c}
  \wk{a}{b}
  \po{b}{c}
\end{tikzpicture}\]

% We say $\aEv$ \emph{depends on} $\cEv$ if
% $\labelling(\aEv) = (\bForm \mid \dontcare)$,
% $\labelling(\cEv) = (\dontcare \mid \aSub)$,
% and $\bForm$ depends on $\aSub$.

% We say $\aEv$ \emph{conflicts with}  $\bEv$ if
% $\labelling(\aEv) = (\dontcare \mid \aAct)$,
% $\labelling(\cEv) = (\dontcare \mid \bAct)$,
% $\aAct$ and $\bAct$ touch the same location, and either
% $\aAct$ or $\bAct$ is a write.

\begin{definition}
  \label{def:prefix}
Let $\aAct \prefix \aPSS$ be the set $\aPSS'$ where $\aPS'\in\aPSS'$ whenever
there is $\aPS\in\aPSS$ such that:
\begin{itemize}
\item $\Event' = \Event \cup \{\cEv\}$,
\item ${\le'}\supseteq{\le}$, % if $\bEv \le \aEv$ then $\bEv \le' \aEv$,
\item ${\gtN'}\supseteq{\gtN}$, %if $\aEv \gtN \bEv$ then $\aEv \gtN' \bEv$,
% \item $\labellingAct'(\cEv) = \aAct$, 
% \item if $\labelling(\aEv) = (\bForm \mid \bAct)$ then $\labelling'(\aEv) =
%   (\bForm' \mid \bAct)$, where:
%   \begin{itemize}
%   \item if $\aAct$ is an acquire then $\bForm'$ is independent of every $\bLoc$,
%   \item if $\aAct$ does not read then $\bForm'$ implies $\bForm$,
%   \item if $\aAct$ reads then $\aVal$ from $\aLoc$ then
%     \begin{itemize}
%     \item $\bForm'$ implies $\bForm[\aVal/\aLoc]$, and
%     \item either $\bForm'$ implies $\bForm$ or $\cEv<'\aEv$, 
%     \end{itemize}
%   \end{itemize}
% \item if $\labellingAct(\aEv) = \bAct$ then:
%   \begin{itemize}
%   \item if $\aAct$ is an acquire or $\bAct$ is a release then $\cEv <' \aEv$, 
%   \item if $\aAct$ and $\bAct$ both touch the same location and one is a write,
%     then $\cEv \gtN' \aEv$, and
%   \end{itemize}
\item $\labellingAct'(\aEv) = \labellingAct'(\aEv)$ unless $\aEv=\cEv$.   $\labellingAct'(\cEv) = \aAct$, 
\item if $\aAct$ does not read then $\labellingForm'(\aEv)$ implies $\labellingForm(\aEv)$,
\item if $\aAct$ reads then $\aVal$ from $\aLoc$ then
  \begin{itemize}
  \item $\labellingForm'(\aEv)$ implies $\labellingForm(\aEv)[\aVal/\aLoc]$, and
  \item either $\labellingForm'(\aEv)$ implies $\labellingForm(\aEv)$ or $\cEv<'\aEv$, 
  \end{itemize}
\item if $\aAct$ is an acquire then $\labellingForm'(\aEv)$ is independent of every $\bLoc$,
\item if $\aAct$ is an acquire or $\labellingAct(\aEv)$ is a release then $\cEv <' \aEv$, 
\item if $\aAct$ and $\labellingAct(\aEv)$ both touch the same location and one is a write,
    then $\cEv \gtN' \aEv$, and
\item if $\aAct$ is a pure read (not acquire or release), then either
  $\labellingForm'(\cEv)$ is unsatisfiable or there is some $\aEv$ such
  that $\labellingForm'(\aEv)$ does not imply $\labellingForm(\aEv)$.
% \item if $\labelling(\aEv) = (\bForm \mid \bAct)$ then $\labelling'(\aEv) =
%   (\bForm' \mid \bAct)$, where:
%   \begin{itemize}
%   \item if $\aAct$ is an acquire or $\bAct$ is a release then $\cEv <' \aEv$, 
%   \item if $\aAct$ is an acquire then $\bForm$ is independent of every $\bLoc$,
%   \item if $\aAct$ and $\bAct$ both touch the same location and one is a write,
%     then $\cEv \gtN' \aEv$, and
%   \item $\bForm'$ implies \(\left\{\begin{array}{l@{~}ll}
%     % \bForm[\aVal/\aLoc]                     & \mbox{if $\aAct$ reads $\aVal$ from $\aLoc$ and $\cEv<'\aEv$} & \textsc{[dependent read]} \\
%     % \bForm[\aVal/\aLoc] \text{ and } \bForm & \mbox{if $\aAct$ reads $\aVal$ from $\aLoc$}                  & \textsc{[independent read]} \\
%     % \bForm                                  & \mbox{otherwise}                                              & \textsc{[non-read]} \\        
%     \bForm[\aVal/\aLoc]                     \\\quad \mbox{if $\aAct$ reads $\aVal$ from $\aLoc$ and $\cEv<'\aEv$} \\\qquad \textsc{[dependent read]} \\[\jot]
%     \bForm[\aVal/\aLoc] \text{ and } \bForm \\\quad \mbox{if $\aAct$ reads $\aVal$ from $\aLoc$}                  \\\qquad \textsc{[independent read]} \\[\jot]
%     \bForm                                  \\\quad \mbox{otherwise}                                              \\\qquad \textsc{[non-read]} \\
%   \end{array}\right.\)
%   \end{itemize}
\end{itemize}
\end{definition}
The last condition ensures that useless reads are not included.
Otherwise, $\labellingForm'(\cEv)$ is unconstrained.

In order to keep augmentation closure, we need to keep the unsatisfiable
elements in the set of pomsets.


\begin{comment}
The first constraint ensures that events are ordered before a release and
after an acquire.  The second constraint ensures that thread-local reads do
not cross acquire fences.
 
The second constraint prevents bad executions like the following:
   x=1; rel; acq; if (x) {y=1};  ||  acq; x=0; rel; 
where the second thread is interleaved between the rel and acq of the first.

Note that you cannot require that $\bForm'$ is independent of every $\bLoc$
because then it's not augment closed.
\end{comment}

\subsection{Sequential stuff}

Definitions of $\fpo$ can be done by just modifying the definition of
prefixing.
\begin{itemize}
\item $\fpo(\aPSS\aSub)=\fpo(\aPSS)$
\item $\fpo(\aForm\guard\aPSS)=\fpo(\aPSS)$
\item $\fpo(\nu\aLoc\st\aPSS)=\fpo(\aPSS)$
\item $\fpo(\aPSS_1\parallel\aPSS_2)=\fpo(\aPSS_1)\cup\fpo(\aPSS_2)$
\item $\fpo(\aAct\prefix\aPSS)$ just adds the new event to $\fpo(\aPSS)$
\end{itemize}
$\fird$ (internal read dependency) is more complicated.
We need to change the rules
\begin{eqnarray*}
  \sem{\aLoc\GETS\aExp\SEMI \aCmd} & = & \textstyle\bigcup_\aVal\; \bigl((\aExp=\aVal) \guard (\DW\aLoc\aVal) \prefix \sem{\aCmd}[\aExp/\aLoc]\bigr) \\
  \sem{\REL\aLoc\GETS\aExp\SEMI \aCmd} & = & \textstyle\bigcup_\aVal\; \bigl((\aExp=\aVal) \guard (\DWRel\aLoc\aVal) \prefix \sem{\aCmd}[\aExp/\aLoc]\bigr) 
\end{eqnarray*}
to
\begin{eqnarray*}
  \sem{\aLoc\GETS\aExp\SEMI \aCmd} & = & \textstyle\bigcup_\aVal\; \bigl((\aExp=\aVal) \guard (\DW\aLoc\aVal) \prefix \sem{\aCmd}\bigr)[\aExp/\aLoc] \\
  \sem{\REL\aLoc\GETS\aExp\SEMI \aCmd} & = & \textstyle\bigcup_\aVal\; \bigl((\aExp=\aVal) \guard (\DWRel\aLoc\aVal) \prefix \sem{\aCmd}\bigr)[\aExp/\aLoc] 
\end{eqnarray*}
so that prefixing can talk about occurrences of $\aLoc$ in $\sem{\aCmd}$.
Then you get an $\fird$ from the new event $\cEv$ to an old event $\aEv$ if
$\labellingForm(\aEv)$ is dependent on $\aLoc$ (ie, it is \emph{not} independent).

% To see that we need $[\aExp/\aLoc]$ in the rule for write, rather than $[\aVal/\aLoc]$
% consider example:
% \begin{verbatim}
% r=y; if (r) {x=r} else {x=r}; s=x; if (r==s) {z=1}
% \end{verbatim}
% or simplified:
% \begin{verbatim}
% r=y;x=r;s=x; if(s==r){z=1}
% \end{verbatim}
% If you read 37 for $y$, then the predicate on \texttt{Wz1} before the
% read is either $r=r$ or $v=r$, where $v=37$, for example.  In one case you
% get a dependency and in the other you do not.

\subsection{Example}

An example to show that we need $[\aExp/\aLoc]$ in the rule for write, rather
than $[\aVal/\aLoc]$:

$\sem{\IF(\bReg\EQ\aReg)\THEN \cLoc\GETS 1\FI}$
includes
\[\begin{tikzpicture}[node distance=1em]
  \event{c}{\bReg=\aReg \mid \DW\cLoc1}{}
\end{tikzpicture}\]
therefore
$\sem{\bReg\GETS\aLoc\SEMI\IF(\bReg\EQ\aReg)\THEN \cLoc\GETS 1\FI}$
includes
\[\begin{tikzpicture}[node distance=1em]
  \event{c}{\aLoc=\aReg \mid \DW\cLoc1}{}
\end{tikzpicture}\]
and
$\sem{\aLoc\GETS\aReg\SEMI\bReg\GETS\aLoc\SEMI\IF(\bReg\EQ\aReg)\THEN \cLoc\GETS 1\FI}$
includes
\[\begin{tikzpicture}[node distance=1em]
  \event{c}{\aReg=\aReg \mid \DW\cLoc1}{}
\end{tikzpicture}\]
which is independent of $\aReg$.

If we took the semantics of write to use $[\aVal/\aLoc]$, then we would end
up with pomsets of the form
\[\begin{tikzpicture}[node distance=1em]
  \event{c}{\aVal=\aReg \mid \DW\cLoc1}{}
\end{tikzpicture}\]
which depend on $\aReg$.

To see why we don't mention dependency in sequential composition, consider
$\sem{\IF(\aReg\leq1)\THEN \aLoc\GETS 2\FI}$ which includes
\[\begin{tikzpicture}[node distance=1em]
  \event{c}{\aReg\leq1 \mid \DW\aLoc2}{}
\end{tikzpicture}\]
which depends on $\aReg$.

$\sem{\aReg\GETS\bLoc\SEMI\IF(\aReg\leq1)\THEN \aLoc\GETS 2\FI}$ includes
\[\begin{tikzpicture}[node distance=1em]
    \event{b}{\DR\bLoc1}{}
    \event{c}{\bLoc\leq1 \mid \DW\aLoc2}{right=of b}
\end{tikzpicture}\]
which has no dependency between the read and write.

$\sem{\bLoc\GETS0\SEMI\aReg\GETS\bLoc\SEMI\IF(\aReg\leq1)\THEN \aLoc\GETS
  2\FI}$ discharges the precondition of the write, giving
\[\begin{tikzpicture}[node distance=1em]
    \event{a}{\DW\bLoc0}{}
    \event{b}{\DR\bLoc1}{right=of a}
    \event{c}{0\leq1 \mid \DW\aLoc2}{right=of b}
\end{tikzpicture}\]
which is simply:
\[\begin{tikzpicture}[node distance=1em]
    \event{a}{\DW\bLoc0}{}
    \event{b}{\DR\bLoc1}{right=of a}
    \event{c}{\DW\aLoc2}{right=of b}
\end{tikzpicture}\]
The semantics of this program also includes
\[\begin{tikzpicture}[node distance=1em]
    \event{a}{\DW\bLoc0}{}
    \event{c}{\DW\aLoc2}{right=of a}
\end{tikzpicture}\]

A variant of this which indicates the branch taken:
$\sem{\bLoc\GETS0\SEMI\aReg\GETS\bLoc\SEMI\IF(\aReg\leq1)\THEN
  \aLoc\GETS2\SEMI\cLoc\GETS\aReg\FI}$
includes
\[\begin{tikzpicture}[node distance=1em]
    \event{a}{\DW\bLoc0}{}
    \event{b}{\DR\bLoc1}{right=of a}
    \event{c}{\DW\aLoc2}{right=of b}
    \event{d}{\DW\cLoc1}{right=of c}
    \po[bend left]{b}{d}
\end{tikzpicture}\]

A program to witness this is
\begin{displaymath}
  \IF(\bLoc\EQ0)\THEN
    \IF(\aLoc\EQ2)\THEN
      \bLoc\GETS1\SEMI
      \IF(\cLoc\EQ1)\THEN\PASS\FI
    \FI
  \FI
\end{displaymath}

Putting these in parallel gives you
\[\begin{tikzpicture}[node distance=1em]
    \event{a}{\DW\bLoc0}{}
    \event{b}{\DR\bLoc1}{right=of a}
    \event{c}{\DW\aLoc2}{right=of b}
    \event{d}{\DW\cLoc1}{right=of c}
    \po[bend left]{b}{d}
    \event{a2}{\DR\bLoc0}{below=of a}
    \event{b2}{\DR\aLoc2}{right=of a2}
    \event{c2}{\DW\bLoc1}{right=of b2}
    \event{d2}{\DR\cLoc1}{right=of c2}
    \po{a2}{b2}
    \po{b2}{c2}
    \po[bend right]{b2}{d2}
    \rf{a}{a2}
    \rf{c}{b2}
    \rf{c2}{b}
    \rf{d}{d2}
\end{tikzpicture}\]


% Local Variables:
% mode: latex
% TeX-master: "paper"
% End:

\section{Test Cases and Additional Features}
\label{sec:variants}

After discussing a few test cases, we extend the model to include additional
features: fences, address calculation, and read-modify-write
(\RMW)\footnote{The proofs given later in the paper extend to all of these
  features, with one exception: we believe the \armeight/\tso-compilation
  strategy is correct for \RMW---it is borrowed from
  \citet{DBLP:journals/pacmpl/PodkopaevLV19}---but have not proved it.}.  We
then provide an alternative semantics that supports full sequential
composition, building $\sem{\aCmd\SEMI\bCmd}$ from $\sem{\aCmd}$ and
$\sem{\bCmd}$.

\paragraph{Test Cases.}
Our model gives the desired results for the test cases of \citet{PughWebsite},
\citet[\textsection 5.3]{SevcikThesis}, and \citet[\textsection
4]{DBLP:conf/esop/BattyMNPS15}.  It also agrees with the ``surprising and
controversial behaviors'' of \citet[\textsection
8]{Manson:2005:JMM:1047659.1040336}.

We present two examples that are hallmarks of \mca{} architectures.
The analysis follows from a few simple principles.  
% \begin{gather*}
%   \renewcommand{\arraycolsep}{1pt}
%   \hbox{\small
%     $\begin{array}{ccccc}
%     &x\GETS0\SEMI x\GETS 1
%     &\PAR&
%     r\GETS x \SEMI s\GETS y
%     %\IF{x}\THEN r\GETS y \FI
%     \\
%     \PAR
%     &y\GETS0\SEMI y\GETS 1
%     &\PAR&
%     r\GETS y \SEMI s\GETS x
%     %\IF{y}\THEN s\GETS x \FI
%   \end{array}$}
%   \quad
%   \hbox{\begin{tikzinlinesmall}[node distance=.5em and 1em]
%   \event{wx0}{\DW{x}{0}}{}
%   \event{wx1}{\DW{x}{1}}{right=of wx0}
%   \event{wy0}{\DW{y}{0}}{below=2ex of wx0}
%   \event{wy1}{\DW{y}{1}}{right=of wy0}
%   \event{ry1}{\DR{y}{1}}{right=2.5em of wy1}
%   \event{rx0}{\DR{x}{0}}{right=of ry1}
%   \event{rx1}{\DR{x}{1}}{right=2.5 em of wx1}
%   \event{ry0}{\DR{y}{0}}{right=of rx1}
%   \wk{wx0}{wx1}
%   \wk{wy0}{wy1}
%   \rf{wx1}{rx1}
%   \rf[bend left]{wy0}{ry0}
%   \rf{wy1}{ry1}
%   \rf[bend right]{wx0}{rx0}
%   \wk{rx0}{wx1}
%   \wk{ry0}{wy1}
%     \end{tikzinlinesmall}}
% \end{gather*}
\begin{gather*}
  \\[-2.5ex]
  \renewcommand{\arraycolsep}{1pt}
  \hbox{\small
    $\begin{array}{ccccc}
    &x\GETS0\SEMI x\GETS 1
    &\PAR&
    r\GETS x\ACQ \SEMI s\GETS y
    %\IF{x}\THEN r\GETS y \FI
    \\
    \PAR
    &y\GETS0\SEMI y\GETS 1
    &\PAR&
    r\GETS y\ACQ \SEMI s\GETS x
    %\IF{y}\THEN s\GETS x \FI
  \end{array}$}
  \quad
  \smash{\hbox{\begin{tikzinlinesmall}[baseline=-10pt,node distance=.5em and 1em]
  \event{wx0}{\DW{x}{0}}{}
  \event{wx1}{\DW{x}{1}}{right=of wx0}
  \event{wy0}{\DW{y}{0}}{below=2ex of wx0}
  \event{wy1}{\DW{y}{1}}{right=of wy0}
  \event{ry1}{\DRAcq{y}{1}}{right=2.5em of wy1}
  \event{rx0}{\DR{x}{0}}{right=of ry1}
  \event{rx1}{\DRAcq{x}{1}}{right=2.5 em of wx1}
  \event{ry0}{\DR{y}{0}}{right=of rx1}
  \wk{wx0}{wx1}
  \wk{wy0}{wy1}
  \rf{wx1}{rx1}
  \rf[bend left]{wy0}{ry0}
  \rf{wy1}{ry1}
  \rf[bend right]{wx0}{rx0}
  \wk{rx0}{wx1}
  \wk{ry0}{wy1}
  \po{rx1}{ry0}
  \po{ry1}{rx0}
    \end{tikzinlinesmall}}}
  \\[-1ex]
\end{gather*}
In this variant of \iriw\ (Independent Reads of Independent Writes), order 
is imposed by \emph{coherence} (between the writes), \emph{fulfillment}
(between read and write), and \emph{fencing} (from acquiring read to relaxed read).
Given the evident cycle, the candidate execution is invalid.
% Note that if you enforce the dependency between the reads, then the execution
% is disallowed, because the is a cycle on weak edges restricted to $x$:
% \begin{math}
%   (\DW{x}{1})\le (\DR{y}{0}) \gtN (\DW{y}{1})
% \end{math}
% therefore
% \begin{math}
%   (\DW{x}{1}) \gtN  (\DW{y}{1});
% \end{math}
% then
% \begin{math}
%   (\DW{y}{1})\le (\DR{y}{0}) \gtN (\DW{y}{1})
% \end{math}
% therefore
% \begin{math}
%   (\DW{x}{1}) \gtN  (\DW{y}{1});
% \end{math}

It is also impossible for all threads to read $1$ in the following, due to
\emph{control dependencies}, \emph{coherence}, and \emph{fulfillment}.
\begin{gather*}
  \hbox{\small$\IF{x}\THEN y\GETS0 \FI \SEMI y\GETS1
  {\PAR}
  \IF{y}\THEN z\GETS0 \FI \SEMI z\GETS1
  {\PAR}
  \IF{z}\THEN x\GETS0 \FI \SEMI x\GETS1
  $}
  \\[-.5ex]
  \hbox{\begin{tikzinlinesmall}[node distance=1em]
  \event{a1}{\DR{x}{1}}{}
  \event{a2}{\DW{y}{0}}{right=of a1}
  \po{a1}{a2}
  \event{a3}{\DW{y}{1}}{right=of a2}
  \wk{a2}{a3}
  \event{b1}{\DR{x}{1}}{right=of a3}
  \event{b2}{\DW{y}{0}}{right=of b1}
  \po{b1}{b2}
  \event{b3}{\DW{y}{1}}{right=of b2}
  \wk{b2}{b3}
  \event{c1}{\DR{x}{1}}{right=of b3}
  \event{c2}{\DW{y}{0}}{right=of c1}
  \po{c1}{c2}
  \event{c3}{\DW{y}{1}}{right=of c2}
  \wk{c2}{c3}
  \rf{a3}{b1}
  \rf{b3}{c1}
  \rf[out=173,in=7]{c3}{a1}  
    \end{tikzinlinesmall}}
\end{gather*}

In either example, the execution is allowed if the cycle is broken---for
example, by changing $x\ACQ$ to $x\RLX$ in \iriw.


% \citet{PughWebsite} developed a set of twenty {causality test cases} in the
% process of revising the Java Memory Model (JMM)
% \cite{Manson:2005:JMM:1047659.1040336}.
% %Using hand calculation, we have confirmed that
% Our model gives the desired result for all twenty cases,
% unrolling loops as necessary.
% % confirmed that our model gives the desired result these test cases, with the
% % following caveats: to model TC14 and TC15 we must unroll loops; the behaviors
% % of TC1 and TC8 are only allowed in our model if the inferred range of
% % variables is included as a global assumption in the logic. 
% Our model also gives the desired results for the examples of
% \citet{PughWebsite}, \citet[\textsection 5.3]{SevcikThesis},
% \citet[\textsection 4]{DBLP:conf/esop/BattyMNPS15}.  Our model agrees with
% ``surprising and controversial behaviors'' of
% \citet[\textsection 8]{Manson:2005:JMM:1047659.1040336}.  We elide the
% details.
% \textsection\ref{sec:examples} develops some of these examples.

\paragraph{Silent Actions and Fences.}

For the actions of a data model, we require that there is a set
$\Int\subseteq\Act$ such that
$\Int\cap\fdom(\rreads)=\Int\cap\fdom(\rwrites)=\emptyset$.  We say $\aAct$
is \emph{silent} if $\aAct\in\Int$.  By definition, silent actions neither
read nor write.

To model the fencing behavior of local reads and writes, our example language
includes internal actions of the form $\DFR{\amode}$ and $\DFW{\amode}$.  For
$\amode\in\{\modeRA,\,\modeSC\}$, $\DFR{\amode}$ is an acquire, and
$\DFW{\amode}$ is a release.  The fences with access mode $\modeRLX$ do
nothing; we include them only to remove case analysis from the definitions.

Because the semantics of syntactic fences is subtly different from the
fencing behavior of $\modeRA$/$\modeSC$-accesses, our example language
includes separate internal actions for syntactic fences: $\DFS{\fmode}$.  The
\emph{syntactic fence mode}
$(\fmode \!\!\BNFDEF\!\! \modeREL \!\BNFSEP\! \modeACQ \!\BNFSEP\! \modeSC)$ is either
\emph{release}, \emph{acquire}, or \emph{sequentially-consistent}.
$\DFS{\modeREL}$ is a release. $\DFS{\modeACQ}$ is an acquire.
$\DFS{\modeSC}$ is both a release and an acquire.

To Definition \ref{def:prefix} of prefixing, add:
\begin{enumerate}
\item[5e.] if $\bEv$ reads and $\labelingAct(\aEv)=\DFS{\modeACQ}$, then
  $\aEv \lt' \bEv$, and
\item[5f.] if $\labelingAct(\bEv)=\DFS{\modeREL}$ and $\aEv$ writes, then
  $\bEv \lt' \aEv$.
\end{enumerate}

As concrete syntax for commands, we write ``$\FENCE^{\fmode} \SEMI \aCmd$.''
\begin{align*}
  \sem{\FENCE^{\fmode}\SEMI \aCmd} & =
  (\DFS{\fmode}) \prefix \sem{\aCmd}
  \\
  \sem{\aReg\GETS\aLoc^\amode\SEMI \aCmd} & =
  \textstyle\bigcup_\aVal\; (\DRmode\aLoc\aVal) \prefix \sem{\aCmd}[\aLoc/\aReg]  
  \\[-.5ex] &
  \mkern2mu\cup
  \,\mathhl{(\DFR{\amode}) \;\prefix}\,\;
  \sem{\aCmd}[\aLoc/\aReg]
  \\
  \sem{\aLoc^\amode\GETS\aExp\SEMI \aCmd} & =
  \textstyle\PAR_\aVal\; (\aExp=\aVal \mid \DWmode\aLoc\aVal) \prefix \sem{\aCmd}[\aExp/\aLoc]
  \\[-.5ex] &
  \mkern2mu\cup
  \,\mathhl{(\DFW{\amode}) \;\prefix}\,\;
  (\relfilt{\aLoc} \sem{\aCmd}[\aExp/\aLoc])
\end{align*}

Our semantics does not suffer the weaknesses of C11 fences, noted by
\citet[Figs.~5 and 6]{DBLP:conf/pldi/LahavVKHD17}. We omit $0$-initialization
in these examples:
\begin{gather*}
    x\GETS1
    \PAR
    r\GETS x\SEMI   
    \FENCE^{\modeSC}\SEMI
    r\GETS y  
    \PAR
    y\GETS 1 \SEMI
    \FENCE^{\modeSC}\SEMI
    r\GETS x  
    \\[-.1ex]
  \hbox{\begin{tikzinline}[node distance=1em]
  \event{a1}{\DW{x}{1}}{}
  \event{b1}{\DR{x}{1}}{right=3em of a1}
  \event{b2}{\DFS{\modeSC}}{right=of b1}
  \po{b1}{b2}
  \event{b3}{\DR{y}{0}}{right=of b2}
  \po{b2}{b3}
  \event{c1}{\DW{y}{1}}{right=3em of b3}
  \event{c2}{\DFS{\modeSC}}{right=of c1}
  \po{c1}{c2}
  \event{c3}{\DR{x}{0}}{right=of c2}
  \po{c2}{c3}
  \wk{b3}{c1}
  \rf{a1}{b1}
  \wk[out=-170,in=-10]{c3}{a1}
    \end{tikzinline}}
  \\
    x\GETS1\SEMI   
    z\REL\GETS1\SEMI   
    \PAR
    r\ACQ\GETS z\SEMI   
    \FENCE^{\modeSC}\SEMI
    r\GETS y  
    \PAR
    y\GETS 1 \SEMI
    \FENCE^{\modeSC}\SEMI
    r\GETS x  
    \\[-.1ex]
  \hbox{\begin{tikzinline}[node distance=.7em]
  \event{a1}{\DW{x}{1}}{}
  \event{a2}{\DWRel{z}{1}}{right=of a1}
  \po{a1}{a2}
  \event{b1}{\DRAcq{z}{1}}{right=2em of a2}
  \event{b2}{\DFS{\modeSC}}{right=of b1}
  \po{b1}{b2}
  \event{b3}{\DR{y}{0}}{right=of b2}
  \po{b2}{b3}
  \event{c1}{\DW{y}{1}}{right=2em of b3}
  \event{c2}{\DFS{\modeSC}}{right=of c1}
  \po{c1}{c2}
  \event{c3}{\DR{x}{0}}{right=of c2}
  \po{c2}{c3}
  \wk{b3}{c1}
  \rf{a2}{b1}
  \wk[out=-170,in=-10]{c3}{a1}
    \end{tikzinline}}
\end{gather*}
The executions are disallowed, due to cycles.  It is a testament to the
complexity of C11 that it allows these executions.

\paragraph{Address Calculation.}
In the definition of a data model, remove the location set $\Loc$ and require
that locations have the form $\aLoc\!\!\BNFDEF\!\!\REF{\cVal}$, where $\cVal$
is a value.  Expressions may include neither memory locations nor the
operator $\REF{\cExp}^{\amode}$.
In our example language, we update the syntax of commands:
\begin{gather*}
  \aCmd
  \BNFDEF \cdots
  \BNFSEP \aReg\GETS\REF{\cExp}^{\amode}\SEMI \aCmd 
  \BNFSEP \REF{\cExp}^{\amode}\GETS\aExp\SEMI \aCmd
  \\
\begin{aligned}
  \sem{\aReg\GETS\REF{\cExp}^\amode\SEMI \aCmd} & =
  \textstyle\bigcup_{\cVal,\aVal}\; (\;\mathhl{\cExp=\cVal} \mid \DRmode{\REF{\cVal}}\aVal) \prefix \sem{\aCmd}[\REF{\cVal}/\aReg]  
  \\[-.5ex] & \mkern2mu\cup
  \textstyle\bigcup_{\cVal\phantom{,\aVal}}\; (\;\mathhl{\cExp=\cVal}\mid\DFR{\amode}) \prefix
  \sem{\aCmd}[\REF{\cVal}/\aReg]
  \\
  \sem{\REF{\cExp}^\amode\GETS\aExp\SEMI \aCmd} & =
  \\[-1ex]
  &\mkern-65mu\hbox to 0pt{$\begin{aligned}
    &\textstyle\PAR_{\cVal,\aVal}\;(\;\mathhl{\cExp=\cVal} \land \aExp=\aVal \mid \DWmode{\REF{\cVal}}\aVal) \prefix \sem{\aCmd}[\aExp/\REF{\cVal}]
    \\[-.5ex]  \cup\!&
    \textstyle\PAR_{\cVal\phantom{,\aVal}}\; (\;\mathhl{\cExp=\cVal}\mid\DFW{\amode}) \prefix
    (\relfilt{\REF{\cVal}} \sem{\aCmd}[\aExp/\REF{\cVal}])    
  \end{aligned}$}
  % \\[-9.5ex]
  % \sem{\;\mathhl{\REF{\cExp}^\amode}\GETS\aExp\SEMI \aCmd} & =
  % \\[6ex]
\end{aligned}
% \begin{aligned}
%   \sem{\aReg\GETS\REF{\cExp}^\amode\SEMI \aCmd} & =
%   \textstyle\bigcup_{\cVal,\aVal}\; (\cExp=\cVal \mid \DRmode{\REF{\cVal}}\aVal) \prefix \sem{\aCmd}[\REF{\cVal}/\aReg]  
%   \\[-.5ex] & \mkern2mu\cup
%   \textstyle\bigcup_{\cVal\phantom{,\aVal}}\; (\cExp=\cVal\mid\DFR{\amode}) \prefix
%   \sem{\aCmd}[\REF{\cVal}/\aReg]
%   \\
%   \sem{\REF{\cExp}^\amode\GETS\aExp\SEMI \aCmd} & =
%   \\[-1ex]
%   &\mkern-65mu\hbox to 0pt{$\begin{aligned}
%     &\textstyle\PAR_{\cVal,\aVal}\;(\cExp=\cVal \land \aExp=\aVal \mid \DWmode{\REF{\cVal}}\aVal) \prefix \sem{\aCmd}[\aExp/\REF{\cVal}]
%     \\[-.5ex]  \cup\!&
%     \textstyle\PAR_{\cVal\phantom{,\aVal}}\; (\cExp=\cVal\mid\DFW{\amode}) \prefix
%     (\relfilt{\REF{\cVal}} \sem{\aCmd}[\aExp/\REF{\cVal}])    
%   \end{aligned}$}
% \end{aligned}
\end{gather*}

Let us revisit the discussion of the use of \!$\PAR$\!
in Candidate~\ref{cand2.9}.
Note that 
\begin{math}
  \sem{a[r] \GETS 0\SEMI a[0]\GETS \BANG r}
\end{math}
includes both of the following pomsets (``$\BANG \aExp$'' evaluates to $1$ if
$\aExp$ is $0$, and $0$ otherwise):
\begin{align*}
  \hbox{\begin{tikzinline}[node distance=.2em]
      \event{a}{r\EQ0\mathbin{\mid}\DW{a[0]}{0}}{}
      \event{b}{r\EQ0\mathbin{\mid}\DW{a[0]}{1}}{right=of a}
    \end{tikzinline}}%\;\;\cdots
  \qquad
  %\\[-1.5ex]\intertext{and:}\\[-5ex]
  \hbox{\begin{tikzinline}[node distance=.2em]
      \event{a}{r\EQ1\mathbin{\mid}\DW{a[1]}{0}}{}
      \event{b}{r\EQ1\mathbin{\mid}\DW{a[0]}{0}}{right=of a}
    \end{tikzinline}}%\;\;\cdots
\end{align*}
By using \!$\PAR$\!, it also includes:
\begin{gather*}
  \hbox{\begin{tikzinline}[node distance=.5em]
      \event{a}{r\EQ0\lor r\EQ1\mathbin{\mid}\DW{a[0]}{0}}{}
      \event{b}{r\EQ0\mathbin{\mid}\DW{a[0]}{1}}{right=of a}
      %\event{c}{r\EQ1\mathbin{\mid}\DW{a[1]}{0}}{right=of b}
    \end{tikzinline}}%\;\;\cdots
\end{gather*}
In this example, the events that coalesce correspond to different statements
in the syntax.


Because we do not enforce order between reads, there is some danger that
address calculations could allow thin air behavior.  Consider the following
attempted execution, where all memory addresses are initialized to $0$,
except that $\REF{2}$ is $1$ and $\REF{1}$ is $2$:
\begin{gather*}
  \aReg\GETS y\SEMI \bReg\GETS \REF{\aReg}\SEMI x\GETS \bReg
  \PAR
  \aReg\GETS x\SEMI \bReg\GETS \REF{\aReg}\SEMI y\GETS \bReg
  %x\GETS\REF{y} \PAR y\GETS\REF{x}
  \\
  \hbox{\begin{tikzinline}[node distance=1em]
  \event{a1}{\DR{y}{2}}{}
  \event{a2}{\DR{\REF{2}}{1}}{right=of a1}
  \event{a3}{\DW{x}{1}}{right=of a2}
  \po{a2}{a3}
  \po[out=10,in=170]{a1}{a3}
  \event{b1}{\DR{x}{1}}{right=3em of a3}
  \event{b2}{\DR{\REF{1}}{2}}{right=of b1}
  \event{b3}{\DW{y}{2}}{right=of b2}
  \po{b2}{b3}
  \po[out=10,in=170]{b1}{b3}
  \rf[out=-170,in=-10]{b3}{a1}
  \rf{a3}{b1}
    \end{tikzinline}}
\end{gather*}
Although there is no order enforced between the reads, the read-to-write
order induced by the semantics is sufficient to prohibit this thin-air
behavior.  Note the intermediate state:
\begin{gather*}
  \bReg\GETS \REF{\aReg}\SEMI x\GETS \bReg
  \\
  \hbox{\begin{tikzinline}[node distance=1em]
  \event{a2}{\aReg{=}2\mid\DR{\REF{2}}{1}}{}
  \event{a3}{\aReg{=}2\mid\DW{x}{1}}{right=of a2}
  \po{a2}{a3}
    \end{tikzinline}}
\end{gather*}
The precondition on the write is required by the last clause of
Definition~\ref{def:mmpomset}: if $\bEv\le\aEv$ then $\labelingForm(\aEv)$
implies $\labelingForm(\bEv)$.

\paragraph{Read-Modify-Write.} We discuss \RMW\ operations that work on a
single location in memory, such as Fetch-And-Add ($\FADD$) and
compare-and-swap ($\CAS$).  These operations can be modeled using read/write
actions or using an additional relation between events.  The second approach
is more general and less obvious, therefore we explain it here.

In Definition \ref{def:mmpomset}, require that a \emph{(memory model) pomset}
be a tuple $(\Event, {\le}, \labeling, \xrmw)$, where ${\xrmw}\subseteq{\le}$
relates the two events of a successful \RMW.  Additionally, require that:
\begin{itemize}
\item If $\cEv$ and $\aEv$ both write $x$, $\cEv\gtN \aEv$ and $\bEv \xrmw \aEv$ then  $\cEv\gtN \bEv$.
\item If $\cEv$ and $\aEv$ both write $x$, $\bEv\gtN \cEv$ and $\bEv \xrmw \aEv$ then  $\aEv\gtN \cEv$.
\end{itemize}
We elide the obvious and tedious semantic rules.

% \begin{scope}
% \renewcommand{\aEv}{r}
% In Definition \ref{def:rf}, require that when $\bEv$ \emph{fulfills $\aEv$ on
%   $\aLoc$}:
% \begin{itemize}
% %\item there is no pair $(r,w)\in{\rrmw}$ such that $r\gtN\bEv\gtN w$, and
% \item if
%   $\aEv \gtN \bEv$,
%   $\bEv$ writes to $\aLoc$, and
%   $(\aEv,w)\in{\rrmw}$, 
%   then $w \gtN \bEv$.
% \end{itemize}
% \end{scope}
This definition ensures atomicity, disallowing executions such as
\cite[Ex.~3.2]{DBLP:journals/pacmpl/PodkopaevLV19}:
\begin{gather*}
  \aLoc\GETS0\SEMI\bReg\GETS \FADD(\aLoc)
  \PAR
  x\GETS 2\SEMI s\GETS x
  \\
  \hbox{\begin{tikzinline}[node distance=1em]
  \event{a2}{\DR{x}{0}}{}
  \event{a1}{\DW{x}{0}}{left=of a2}
  \rf{a1}{a2}
  \event{a3}{\DW{x}{2}}{right=of a2}
  \wk{a2}{a3}
  \event{b2}{\DW{x}{1}}{right=of a3}
  \event{b3}{\DR{x}{1}}{right=of b2}
  \rmw[out=-15,in=-165]{a2}[below]{b2}
  \wk{a3}{b2}
  \rf{b2}{b3}
    \end{tikzinline}}
\end{gather*}

By using two actions rather than one, we allow examples such as the
following, which is allowed by \armeight{} 
\cite[Ex.~3.10]{DBLP:journals/pacmpl/PodkopaevLV19}:
\begin{gather*}
  r \GETS y\SEMI
  z \GETS r
  \PAR
  r\GETS z\SEMI
  x\GETS 0\SEMI
  s\GETS \FADD^{\modeRLX,\modeRA}(x) \SEMI
  y\GETS s{+}1
  \\[-1ex]
  \hbox{\begin{tikzinline}[node distance=1em]
  \event{a1}{\DR{y}{1}}{}
  \event{a2}{\DW{z}{1}}{right=of a1}
  \po{a1}{a2}
  \event{b1}{\DR{z}{1}}{right=3em of a2}
  \rf{a2}{b1}
  \event{b2}{\DW{x}{0}}{right=of b1}
  \event{b3}{\DR{x}{0}}{right=of b2}
  \rf{b2}{b3}
  \event{b4}{\DWRel{x}{1}}{right=2em of b3}
  \rmw{b3}{b4}
  \event{b5}{\DW{y}{1}}{right=of b4}
  \po[out=-15,in=-165]{b1}{b4}
  \po[out=-20,in=-160]{b3}{b5}
  \rf[out=170,in=10]{b5}{a1}
    \end{tikzinline}}
\end{gather*}

\paragraph{Sequential Composition.}
We refactor the syntax:
\begin{align*}
  \aCmd,\,\bCmd
  \BNFDEF& \SKIP
  \BNFSEP \FENCE^{\fmode}
  \BNFSEP \aReg\GETS\aExp
  \BNFSEP \aReg\GETS\aLoc^{\amode} 
  \BNFSEP \aLoc^{\amode}\GETS\aExp
  \\[-.5ex]
  \BNFSEP&\aCmd \PAR \bCmd
  \BNFSEP\aCmd \SEMI \bCmd
  \BNFSEP \VAR\aLoc
  \BNFSEP \IF{\aExp} \THEN \aCmd \ELSE \bCmd \FI
\end{align*}
For clarity, we do not include address calculation or \RMW{}s in this
subsection.


We introduce explicit substitutions,
following the conventions of \citet{DBLP:conf/icalp/RitterP97}: 
\begin{align*}
  \aLocReg\!\!\BNFDEF\!\! \aLoc \!\BNFSEP\! \aReg
  &&
  \aSub,\,\bSub\!\!\BNFDEF\!\! \SUBEMP \!\BNFSEP\! \SUBPAR{\aSub}{\aExp/\aLocReg}
  \!\BNFSEP\! \aSub\SUBSEQ\bSub
\end{align*}
We write application as $\aForm\SUBAPP\aSub$ and write
$\SUBPAR{\SUBEMP}{\aExp/\aLocReg}$ as $\SUB{\aExp/\aLocReg}$.  The definition
of application is not complicated our lack of support for renaming locations:
We only apply substitutions to formulae, which have no binders.

% Goal:
% \begin{math}
%   (\aForm\aSub)\bSub =
%   \aForm(\aSub;\bSub)
% \end{math}
% Pure substitution: $\fdom(\aSub)$ disjoint $\fcodom(\aSub)$.
% Pure substitutions are idempotent.
% $\aSub$ and $\bSub$ are composable if $\fdom(\aSub)$ disjoint $\fcodom(\bSub)$
% \begin{displaymath}
%   (\aSub;\bSub)(x) =
%   \begin{cases}
%     \aSub(\bSub(x)) & \text{if } x \in \fdom(\bSub)\\
%     \aSub(x) & \text{otherwise}
%   \end{cases}
% \end{displaymath}

Let $\Sub$ be the set of all (explicit) substitutions.  In a data model,
substitutions are silent actions: $\Sub\subseteq\Int$.  Pomsets may contain
at most one event $\cEv$ with a label in $\Sub$, and that may be no
$\aEv$ such that $\cEv\le\aEv$.

Substitutions track dependencies for the follower.

Write fence actions need to indicate which variable was written:
$\DFW[\aLoc]{\amode}$.  We extend the notion of \emph{conflict} to include
write fence actions (thus ordering them w.r.t.~reads and writes on the same
location).

To simplify the definition of the base cases, we use a literal notation for
pomsets, drawing substitutions as accepting states.  Let
$\aPS''\in\CLOSE{\aPS}$ if there is $\aPS'\in\PRE(\aPS)$ such that $\aPS''$
implies $\aPS'$ and $\aPS''$ is an augmentation of $\aPS'$.
The definitions for conditional and location binding are unchanged.
\begin{gather*}
  \begin{aligned}
  \sem{\SKIP} & \eqdef
  \CLOSE{\TIKZ{\final{f}{\SUBEMP}{}}}
  \\  
  \sem{\aReg\GETS\aExp} & \eqdef
  \CLOSE{\TIKZ{\final{f}{\SUB{\aExp/\aReg}}{}}}
  \\
  \sem{\FENCE^{\fmode}} & =
  \CLOSE{\TIKZ{
      \event{a}{\DFS{\fmode}}{}
      \final{f}{\SUBEMP}{right=of a}
    }} 
  \\
  \sem{\aReg\GETS\aLoc^\amode} & =
  \textstyle\bigcup_\aVal\;
  \CLOSE{\TIKZ{
      \event{a}{\DRmode\aLoc\aVal}{}
      \final{f}{\SUB{\aVal/\aReg}}{right=of a}
      \po{a}{f}
    }}
  \\[-.5ex] &
  \mkern2mu\cup
  \CLOSE{\TIKZ{
      \event{a}{\DFR{\amode}}{}
      \final{f}{\SUB{\aLoc/\aReg}}{right=of a}
      \po{a}{f}
    }}
  \\
  \sem{\aLoc^\amode\GETS\aExp} & =
  \textstyle\PAR_\aVal\;
  \CLOSE{\TIKZ{
      \event{a}{\aExp=\aVal \mid \DWmode\aLoc\aVal}{}
      \final{f}{\aExp=\aVal \mid \SUB{\aExp/\aLoc}}{right=of a}
    }}
  \\[-.5ex] &
  \mkern2mu\cup
  \CLOSE{\TIKZ{
      \event{a}{\DFW[\aLoc]{\amode}}{}
      \final{f}{\SUB{\aExp/\aLoc}}{right=of a}
    }}
  \end{aligned}
  \\
  \begin{aligned}
    \sem{\aCmd \PAR \bCmd} &= \sem{\aCmd} \parallel \killS\sem{\bCmd}
    &
    \killS(\aPS)&=\{ \aEv\in\Event \mid \labelingAct(\aEv)\notin\Sub \}
    \\
    \sem{\aCmd \SEMI \bCmd} &= \sem{\aCmd} \sequence \sem{\bCmd}
    &
    \killS(\aPSS)&=\{\aPS\restrict{\killS(\aPS)} \mid \aPS\in\aPSS \}
  \end{aligned}
\end{gather*}
To ensure a unique accepting state, we break the symmetry of \!$\PAR$\!,
choosing to keep the substitution on the left.
% \begin{gather*}
%   \begin{aligned}
%     \killS(\aPS)&=\{ \aEv\in\Event \mid \labelingAct(\aEv)\notin\Sub \}
%     &
%     \killS(\aPSS)&=\{\aPS\restrict{\killS(\aPS)} \mid \aPS\in\aPSS \}
%   \end{aligned}
% \end{gather*}

% Relative to the previous definitions, the base cases are handled as follows:
% \begin{itemize}
% \item $\sem{\SKIP}$ introduces the identity substitution,
% \item $\sem{\aLoc\GETS\aExp}$ introduces $\SUB{\aExp/\aLoc}$,
% \item $\sem{\aReg\GETS\aExp}$ introduces $\SUB{\aExp/\aReg}$,
% \item $\sem{\aReg\GETS\aLoc}$, local rule, introduces $\SUB{\aLoc/\aReg}$,
% \item $\sem{\aReg\GETS\aLoc}$, nonlocal rule, introduces
%   $(\DR{\aLoc}{\aVal}) \lt \SUB{\aVal/\aReg}$.
% \end{itemize}
% Base cases:
% \begin{align*}
%   \sem{\SKIP}
%   =&
%   \TIKZ{\final{f}{}{}} 
%   \\
%   \sem{\aLoc\GETS\aExp}
%   =&
%   \textstyle\bigcup_\aVal\; \TIKZ{\event{a}{(\aExp=\aVal\mid\DW\aLoc\aVal)}{}\final{f}{\aExp/\aLoc}{right=of a}}
%   \\
%   \sem{\aReg\GETS\aLoc}
%   =&
%   \TIKZ{\final{f}{\aLoc/\aReg}{}}
%   \cup
%   \textstyle\bigcup_\aVal\; \TIKZ{\event{a}{(\DR\aLoc\aVal)}{}\final{f}{\aLoc/\aReg}{right=of a}\po{a}{f}}
% \end{align*}


% Here's the def for prefixing:
% \begin{enumerate}
% \item[1.] $\Event' = \Event \uplus \{\bEv\}$,
% \item[2.] ${\le'}\supseteq{\le}$,
% \item[3a.] $\labelingAct'(\bEv) = \aAct$,
% \item[3b.] $\labelingForm'(\bEv)$ implies $\aForm$,
% \item[4a.] $\labelingAct'(\aEv) = \labelingAct(\aEv)$,
% \item[4b.] if $\bEv$ \externally reads $\aVal$ from $\aLoc$ then
%   $\labelingForm'(\aEv)$ implies $\labelingForm(\aEv)[\aVal/\aLoc]$,
% \item[4c.] if $\bEv$ does not \externally read then $\labelingForm'(\aEv)$
%   implies $\labelingForm(\aEv)$, 
% \item[5a.] if $\labelingForm'(\aEv)$ does not imply $\labelingForm(\aEv)$ and
%   $\aEv$ writes, then $\bEv\lt'\aEv$,
% \item[5b.] if $\bEv$ and $\aEv$ are \external actions in conflict, then
%   $\bEv\lt'\aEv$,
% \item[5c.] if $\bEv$ is an acquire or $\aEv$ is a release, then
%   $\bEv \lt' \aEv$,
% \item[5d.] if $\bEv$ is an SC write and $\aEv$ is an SC read, then
%   $\bEv \lt' \aEv$, 
% \item[5e.] if $\bEv$ reads and $\labelingAct(\aEv)=\DFS{\modeACQ}$, then
%   $\aEv \lt' \bEv$,
% \item[5f.] if $\labelingAct(\bEv)=\DFS{\modeREL}$ and $\aEv$ writes, then
%   $\bEv \lt' \aEv$, and
% \item[6.] if $\bEv$ is a release, $\aEv_1$ is an acquire, $\aEv_1\le\aEv_2$, then $\labelingForm(\aEv_2)$
%   is location independent.
% \end{enumerate}



\begin{definition}
  \label{def:semi:seq}
  Let $\DISJUNCT(\aPSS)$ be the set $\aPSS'$ where $\aPS'\in\aPSS'$ if there
  are $\aPS^i\in\aPSS$ such $\Event' = \Event^i$, ${\le'}={\le^i}$,
  $\labelingAct'=\labelingAct^i$, and $\labelingForm'(\aEv)$ implies
  $\bigvee_i \labelingForm^i(\aEv)$.
  
  Let $(\aPSS^1 \sequence \aPSS^2)$ be the set
  $\{\aPS^1\in\aPSS^1\mid \disjoint{\Event^1}{\Sub}\}\cup \DISJUNCT(\aPSS')$
  where $\aPS'\in\aPSS'$ when there are $\aPS^1 \in \aPSS^1$,
  $\cEv\in\Event^1\cap\Sub$ and $\aPS^2\in\aPSS^2$ such that the following
  hold.  Let $\bEv$ range over $\Event^1\setminus\Sub$, and $\aEv$ range over
  $\Event^2$.
\begin{enumerate}
\item[1.] $\Event' = \Event^1\setminus\Sub \uplus \Event^2$,
\item[2.] ${\le'}\supseteq{\le^1}\cup{\le^2}$, 
\item[3a.] $\labelingAct'(\bEv) = \labelingAct^1(\bEv)$,
\item[3b.] $\labelingForm'(\bEv)$ implies $\labelingForm^1(\bEv)$,
\item[4a1.] $\labelingAct'(\aEv) = \labelingAct^2(\aEv)$,  for $\aEv\in\Event^2\setminus\Sub$,
\item[4a2.] $\labelingAct'(\aEv) =  \labelingAct^2(\aEv)\SUBSEQ\labelingAct^1(\cEv)$,  for $\aEv\in\Event^2\cap\Sub$, 
\item[4bc.] $\labelingForm'(\aEv)$ implies $\labelingForm^2(\aEv)\SUBAPP\labelingAct^1(\cEv)$,
% \item[4b.] if $\bEv$ \externally reads $\aVal$ from $\aLoc$ then
%   $\labelingForm'(\aEv)$ implies $\labelingForm^2(\aEv)[\aVal/\aLoc]$,
% \item[4c.] if $\bEv$ does not \externally read then $\labelingForm'(\aEv)$
%   implies $\labelingForm^2(\aEv)$,
\item[5a.] if $\labelingForm'(\aEv)$ does not imply $\labelingForm^2(\aEv)$ and
  $\aEv$ writes, then $\cEv\ORDER\aEv$,
\item[5b-f.] as before,
% \item[5b.] if $\bEv$ and $\aEv$ are \external actions in conflict, then
%   $\bEv\ORDER\aEv$,
% \item[5c.] if $\bEv$ is an acquire or $\aEv$ is a release, then
%   $\bEv \ORDER \aEv$,
% \item[5d.] if $\bEv$ is an SC write and $\aEv$ is an SC read, then
%   $\bEv \ORDER \aEv$, 
% \item[5e.] if $\bEv$ reads and $\labelingAct(\aEv)=\DFS{\modeACQ}$, then
%   $\aEv \ORDER \bEv$,
% \item[5f.] if $\labelingAct(\bEv)=\DFS{\modeREL}$ and $\aEv$ writes, then
%   $\bEv \ORDER \aEv$, 
\item[6a.] if $\bEv$ is a release, $\dEv\in\Event'$ is an acquire,
  $\bEv\le'\dEv\le'\aEv$, then $\labelingForm(\aEv)$ is location independent, and
\item[6b.] if $\labelingAct(\bEv)=\DFW[\aLoc]{\amode}$, $\bEv\le\aEv$, and
  $\aEv$ is a release that does not write $\aLoc$, then  some
  $\dEv$ writes $\aLoc$ and $\bEv\le'\dEv\le'\aEv$. %such that $\dEv$ %(explicitly)
\end{enumerate}
6b ensures the effect of $\relfilt{}$ in the local write rule.
\end{definition}


\begin{comment}
Plan: 
  a.  Define pom1; pom2  for pomsets
  b.  [| C1 ; C2 |] = cup { pom1; pom2 | pom1 in C1, pom2 in C2}

Def:
   pom1; pom2
             cup_L  L prefix pom2 
             where L is a  linearization of pom1
\end{comment}

% \section{Variations}

% \citet{2019-sp} define \emph{3-valued pomsets with preconditions} to model
% security flaws that arise from speculative evaluation in computer
% microarchitecture (such as Spectre \cite{DBLP:journals/corr/abs-1801-01203}).
% \begin{definition}
%   %\label{def:3valued}
%   A \emph{3-valued pomset with preconditions} is a tuple
%   $(\Event, {\le}, {\gtN}, \labeling)$, such that
%   \begin{itemize}
%   \item $\Event$ is a set of \emph{states},
%   \item $\labeling: \Event \fun (\Formulae\times\Act)$ is a \emph{labeling},
%   \item ${\le} \subseteq (\Event\times\Event)$ is a partial order, and
%   \item ${\gtN} \subseteq (\Event\times\Event)$ such that:
%     \begin{itemize}
%     \item\label{5a} if $\bEv \le \aEv$ then $\bEv \gtN \aEv$, \hfill
%       (Inclusion)
%     \item\label{5b} if $\bEv \le \aEv$ and $\aEv \gtN \bEv$ then
%       $\bEv = \aEv$, and \hfill (Consistency)
%     \item\label{5c} if $\cEv \le \bEv \gtN \aEv$ or $\cEv \gtN \bEv \le \aEv$
%       then $\cEv \gtN \aEv$.  \hfill (Semi-transitivity)
%     \end{itemize}
%   \end{itemize}

%   A \emph{(memory model) pomset} is a 3-valued pomset with preconditions,
%   such that
%   \begin{itemize}
%   \item if $\bEv\le\aEv$ then $\labelingForm(\aEv)$ implies
%     $\labelingForm(\bEv)$, and \hfill (Causal-strengthening)
%   \end{itemize}
% \end{definition}
% The axioms for $\gtN$ are adapted
% from~\citet[A1--A3]{DBLP:journals/dc/Lamport86}.  

% \begin{definition}
%   A pomset \emph{coherent} if, when restricted to events that read or write
%   any single location $\aLoc$, $\gtN$ forms a partial order.
% \end{definition}

\section{Sequential Composition}
\label{sec:semicolon}
We provide an alternative semantics that supports full sequential
composition, building $\sem{\aCmd\SEMI\bCmd}$ from $\sem{\aCmd}$ and
$\sem{\bCmd}$.  To simplify the definitions\footnote{This restriction can be
  removed as follows. Separate registers into two categories: those used for
  direct assignment ($\aReg\GETS\aExp$) and those used for reads
  ($\aReg\GETS \REF{\cExp}^{\amode}$).  When constructing a pomset, define
  \emph{program order} $(\xpox)$ in the obvious way.  Let $\bEv$ be
  \emph{visible} if it reads into some $\aReg$ and there is no $\aEv$ that
  reads into $\aReg$ such that $\bEv\xpox\aEv$.  Let $\SUBDRS{\aPS}$ be
  derived from visible reads (rather than all reads).  In item 5a of
  Definition \ref{def:semi:seq}, only consider events, $\bEv$, that are
  visible reads.}, we assume, without loss of generality
\cite{Rosen:1988:GVN:73560.73562}, that each register is assigned at most
once syntactically.  Since we exclude loops and functions, this trivially
ensures that each register is assigned at most once per pomset.

We refactor the syntax:
\begin{align*}
  \aCmd,\,\bCmd
  \BNFDEF& \SKIP
  \mkern-2mu\BNFSEP\mkern-2mu \FENCE^{\fmode}
  \mkern-2mu\BNFSEP\mkern-2mu \aReg\GETS\aExp
  \mkern-2mu\BNFSEP\mkern-2mu \aReg\GETS \REF{\cExp}^{\amode} 
  \mkern-2mu\BNFSEP\mkern-2mu \REF{\cExp}^{\amode}\GETS\aExp
  \\[-.5ex]
  \BNFSEP&\aCmd \PAR \bCmd
  \mkern-2mu\BNFSEP\mkern-2mu\aCmd \SEMI \bCmd
  \mkern-2mu\BNFSEP\mkern-2mu \VAR\aLoc\SEMI \aCmd
  \mkern-2mu\BNFSEP\mkern-2mu \IF{\aExp} \THEN \aCmd \ELSE \bCmd \FI
\end{align*}
\paragraph{Explicit Substitutions.}
Let $\aEExp$ range over \emph{extended expressions}, which may include memory
locations.  We introduce explicit substitutions over extended expressions,
following the conventions of \citet{DBLP:conf/icalp/RitterP97}:
\begin{gather*}
  \aLocReg\BNFDEF \aLoc \BNFSEP \aReg
  \qquad\quad
  \aSub,\,\bSub %,\, \SUBDRS{\dEvs} 
  \BNFDEF \SUBEMP \BNFSEP \SUBPAR{\aSub}{\aEExp/\aLocReg}
  \BNFSEP \aSub\SUBSEQ\aSub'
\end{gather*}
$\SUBEMP$ is the identity substitution.  We write
$\SUBPAR{\SUBEMP}{\aEExp/\aLocReg}$ as $\SUB{\aEExp/\aLocReg}$.

Application is written $\aSub\SUBAPP\aForm$.  We only apply substitutions to
formulae---which do not bind locations or registers.  The definition is
homomorphic over the syntax of formulae. For the basis, 
\begin{math}
  \SUBPAR{\aSub}{\aEExp/\bLocReg}\SUBAPP\aLocReg
\end{math}
is $\aEExp$ if $\bLocReg=\aLocReg$ and is $\aSub \SUBAPP\aLocReg$ otherwise.

Sequencing is defined so that
\begin{math}
  \aSub\SUBSEQ\SUB{\aEExp/\aLocReg}
  \allowbreak= 
  \SUBPAR{\aSub}{\aSub\SUBAPP\aEExp/\aLocReg}
\end{math}
and
\begin{math}
  (\beforeSub\SUBSEQ\afterSub)\SUBAPP\aForm = \beforeSub\SUBAPP(\afterSub\SUBAPP\aForm).
\end{math}

We say that $\aSub$ \emph{subsumes} $\bSub$ if for every $\aLocReg$, either
$\bSub\SUBAPP\aLocReg=\aSub\SUBAPP\aLocReg$ or $\bSub\SUBAPP\aLocReg=\aLocReg$.
For example, every substitution subsumes $\SUBEMP$.

We say that $\aSub$ is \emph{independent of $\aLoc$} if $\aSub\SUBAPP\aLoc=\aLoc$.

Let $\Sub$ be the set of all (explicit) substitutions.

\paragraph{Substitutions in the Model.}\

Change Definition~\ref{def:mmpomset}, of \emph{memory model pomset}, so that
$\labeling: \Event \fun (\Formulae\times\Act\times\Sub)$, from which we
derive the additional function $\labelingSub:\Event\fun\Sub$.

We write triples in $(\Formulae\times\Act\times\Sub)$ as
$(\aForm \mid \aAct\mid \aSub)$, eliding $\aForm$ when it is a tautology,
eliding $\aAct$ when it is the termination action, and eliding $\aSub$ when
it is the identity substitution.

As a notational convenience, let $\labelingSub(\aPS)$ return the substitution
of the termination event in $\aPS$, if one exists.  

We ignore substitutions, except on read events and termination events.  In
the definition of sequential composition, read substitutions are used to
calculate dependencies, as in item 5a of Definition \ref{def:prefix}.  The
terminal substitutions in $\aPSS^1$ are composed with the substitutions in
$\aPSS^2$ when calculating $(\aPSS^1\sequence\aPSS^2)$, and symmetrically.

Extend Definition \ref{def:closure:properties}, of relations between pomsets,
to include subsumption: $\aPS'$ \emph{subsumes} $\aPS$ if
$\Event'=\Event$, ${\le'}={\le}$, $\labelingForm'=\labelingForm$, and
$\labelingAct'=\labelingAct$ and $\labelingSub'(\aEv)$ subsumes
$\labelingSub(\aEv)$.

The semantics of programs is closed w.r.t.~\emph{reverse subsumption}: if
$\aPS\in\sem{\aCmd}$ and $\aPS$ is subsumed by $\aPS'$, then
$\aPS'\in\sem{\aCmd}$.

Subsumption is dual to implication: Strong\-er preconditions impose a greater
burden on the preceding code; stronger substitutions can better mitigate this
burden in following code.  In examples, we only show executions that are
implication and augmentation minimal; similarly, we only show executions that
are subsumption-maximal.

Change the partial function $\rreads$, from the data model, so that
$\rreads:\Act\fun(\Reg \times \Loc \times \Val)$.  When
$\rreads(\aAct) = (\aReg,\aLoc,\aVal)$, we say that $\aAct$ \emph{reads}
$\aVal$ \emph{from} $\aLoc$ \emph{into} $\aReg$.

The semantics satisfies the following invariant: If $\aPS$ is
subsumption\hyp{}maximal and $\labelingSub(\aPS)$ is defined, then for every
$\aEv$ that reads $\aLoc$ into $\aReg$, there are $\beforeSub$ and
$\afterSub$ such that
$\labelingSub(\aPS) = (\beforeSub\SUBSEQ\afterSub)$ and
$\labelingSub(\aEv)= (\beforeSub\SUBSEQ\SUB{\aLoc/\aReg}\SUBSEQ\afterSub)$.

Let $\SUBDRS{\aPS}$ be the substitution of values for registers that is
derived from the reads of $\aPS$ as follows:
$\SUBDRS{\aPS}\SUBAPP\aReg=\aVal$ if some event in $\aPS$ reads $\aVal$ into
$\aReg$, and $\SUBDRS{\aPS}\SUBAPP\aReg=\aReg$ otherwise.

\paragraph{Semantics of the Example Language.}\

As concrete syntax for read actions, we write
$(\DRreg[\amode]{\aReg}{\aLoc}{\aVal})$.

Change Definition \ref{def:rf}, of \emph{$\aLoc$-closed}, to require that
every substitution is independent of $\aLoc$.

Change Definition \ref{def:par}, of $\aPS' \in (\aPSS^1 \parallel \aPSS^2)$,
to additionally require that for all $\aEv\in\Event'$, either:
\begin{align*}
  &\labelingSub'(\aEv) \text{ is subsumed by both } \labelingSub^1(\aEv) \textand \labelingSub^2(\aEv),\\[-1ex]
  &\labelingSub'(\aEv) \text{ is subsumed by } \labelingSub^1(\aEv) \textand \aEv \not\in \Event^2,\; \textor\\[-1ex]
  &\labelingSub'(\aEv) \text{ is subsumed by } \labelingSub^2(\aEv) \textand \aEv \not\in \Event^1.
\end{align*}

Parallel composition, conditional and location binding are otherwise
unchanged from \textsection\ref{sec:model}.

Note that only shared register state is preserved by parallel composition.
For example,
\begin{math}
  (r\GETS 1\PAR r\GETS 1)\SEMI x\GETS r
\end{math}
can write $1$ to $x$, but
\begin{math}
  (r\GETS 1\PAR \SKIP)\SEMI x\GETS r
\end{math}
cannot.

To simplify the base cases, we use a literal notation for pomsets and define
$\CLOSE{\aPS}$ to be the smallest set that includes $\aPS$ and is closed
w.r.t.~prefixing, implication, augmentation, and reverse subsumption.
% : Let
% $\aPS''\in\CLOSE{\aPS}$ if there is $\aPS'\in\PRE(\aPS)$ such (1) $\aPS''$
% implies $\aPS'$, (2) $\aPS''$ is an augmentation of $\aPS'$ and (3) $\aPS''$
% is subsumed by $\aPS'$.
\begingroup
\allowdisplaybreaks
\begin{gather*}
  \begin{aligned}
  \sem{\SKIP} & \eqdef
  \CLOSE{\TIKZ{\final{f}{}{}}}
  \\  
  \sem{\aReg\GETS\aExp} & \eqdef
  \CLOSE{\TIKZ{\final{f}{\SUB{\aExp/\aReg}}{}}}
  \\
  \sem{\FENCE^{\fmode}} & =
  \CLOSE{\TIKZ{
      \event{a}{\DFS{\fmode}}{}
      \final{f}{}{right=of a}
      \sync{a}{f}
    }} 
  \\
  \sem{\aReg\GETS\REF{\cExp}^\amode} & =
  \textstyle\bigcup_{\cVal,\aVal}\;
  \CLOSE{\TIKZ{
      \event{a}{\cExp=\cVal\mid\DRreg[\amode]{\aReg}{\REF{\cVal}}{\aVal}\mid \SUB{\REF{\cVal}/\aReg}}{}
      \final{f}{}{right=of a}
      \sync{a}{f}
    }}
  \\& \cup
  \textstyle\bigcup_{\cVal\phantom{,\aVal}}\;
  \CLOSE{\TIKZ{
      \event{a}{\cExp=\cVal\mid\DRreg[\amode]{\aReg}{\REF{\cVal}}{\aVal}}{}
      \final{f}{\SUB{\REF{\cVal}/\aReg}}{right=of a}
      \sync{a}{f}
    }}
  \\
  \sem{\REF{\cExp}^\amode\GETS\aExp} & =
  \;\;\textstyle\parallel_{\cVal,\aVal}\;
  \CLOSE{\TIKZ{
      \event{a}{\cExp=\cVal\land\aExp=\aVal \mid \DW[\amode]{\REF{\cVal}}{\aVal}}{}
      \final{f}{\SUB{\aExp/\REF{\cVal}}}{right=of a}
      \sync{a}{f}
    }}
                                \\
  \sem{\aCmd \SEMI \bCmd} &= \sem{\aCmd} \sequence \sem{\bCmd}
    \end{aligned}
  \end{gather*}
\endgroup

Whereas a write introduces the substitution $\SUB{\aExp/\REF{\cVal}}$ on
the terminal event, a read introduces the substitution
$\SUB{\REF{\cVal}/\aReg}$ on the read event itself.

The most significant challenge is defining sequential composition.
Unfortunately, \emph{disjunction} and \emph{prefix weakening} (Definition
\ref{def:closure:properties}) do not come easily.

Recall \eqref{alanAddress} from \textsection\ref{sec:variants}:
\begin{math}
  \sem{a[r] \GETS 0\SEMI a[0]\GETS \BANG r}.
\end{math}
In the semantics, an event from the first statement can coalesce with an
event from the second.  Thus when computing
\begin{math}
  \sem{a[r] \GETS 0} \sequence \sem{a[0]\GETS \BANG r},
\end{math}
we must coalesce events with incompatible preconditions ($r{=}0$, $r{=}1$)
that occur on different sides of the sequencing operator.  This makes a
direct definition difficult.

Instead of a direct definition, we first construct the sequential composition
\emph{without} coalescing events, then close the resulting set of pomsets to
ensure the required properties.

There is also a challenge dealing with redundant write elimination:
\begin{math}
  \sem{x\GETS 1\SEMI x\GETS 2} 
\end{math}
should contain a pomset that includes only
\begin{math}
  (\DW{x}{2}).
\end{math}
We achieve this using the same strategy: closing after the construction.

Let $\cEv$ be an \emph{unused write} in $\aPS$ when it is a relaxed write to
some $\aLoc$ such that (1) $\cEv$ fulfills no reads, (2) there is some
$\bEv\gt\cEv$ that writes $\aLoc$, and (3) for every release $\aEv\gt\cEv$
there is some $\aEv\gt\bEv\ge\cEv$ that writes $\aLoc$.

Finally, there is a challenge in calculating the preconditions for the events
of $\aPSS^2$ in $(\aPSS^1\sequence\aPSS^2)$ when terminal event of $\aPSS^1$ is
missing, and symmetrically.  Again, we use the same strategy: We compute
sequential composition using completed executions, then prefix close.

\begin{definition}
  \label{def:semi:seq}
  Let $\DISJUNCT(\aPSS)$ be the least set that
  includes $\aPSS$ and that is closed w.r.t.~disjunction, prefixing, prefix weakening, and
  unused write removal.                  

  Let $(\aPSS^1 \sequence \aPSS^2)$ be the set
  \begin{math}
    \DISJUNCT(\aPSS')
  \end{math}
  where $\aPS'\in\aPSS'$ when there are $\aPS^1 \in \aPSS^1$,
  and $\aPS^2\in\aPSS^2$
  such that the following hold.

Let $\bEv$ range over $\Event^1$.  Let $\aEv$ range over $\Event^2$.  
      
\begin{enumerate}
\item[1.] $\Event' = \Event^2 \uplus \{\bEv\in\Event^1\mid\bEv\text{ is not a termination}\}$,
\item[2.] ${\le'}\supseteq{\le^1}\cup{\le^2}$, 
\item[3a.] $\labelingAct'(\bEv) = \labelingAct^1(\bEv)$,
\item[3b.] $\labelingForm'(\bEv)$ implies $\labelingForm^1(\bEv)$,
\item[3c.] $\labelingSub'(\bEv)$ is subsumed by $\labelingSub^1(\bEv)\SUBSEQ \labelingSub(\aPS^2)$,
\item[4a1.] $\labelingAct'(\aEv) = \labelingAct^2(\aEv)$,
\item[4a2.] $\labelingSub'(\aEv)$ is subsumed by $\labelingSub(\aPS^1)\SUBSEQ\labelingSub^2(\aEv)$,
\item[4bc.] $\labelingForm'(\aEv)$ implies
  $\SUBDRS{\aPS^1} \SUBAPP \bigl(\labelingSub(\aPS^1)\SUBAPP\labelingForm^2(\aEv)\bigr)$, 
\item[5a.] if $\bEv$ is a read and $\aEv$ is a write,
  then either $\bEv\le'\aEv$ or $\labelingForm'(\aEv)$ implies $\SUBDRS{\aPS^1}\SUBAPP \bigl(\labelingSub^1(\bEv)\SUBAPP\labelingForm^2(\aEv)\bigr)$,
\item[5b-f.] as before (see \textsection\ref{sec:model}-\ref{sec:variants}). %generalizing 5e: if $\aEv\in\Sub$ then $\dEv\le'\aEv$,
\end{enumerate}
\end{definition}

The item numbers are chosen to match those of the corresponding clauses in
\textsection\ref{sec:model}.  

Items 4b and 4c collapse into a single item here.
In 4bc, note that the domain of $\labelingSub(\aPS^1)$ is disjoint from the domain
of $\SUBDRS{\aPS^1}$, although registers in the domain of $\labelingSub(\aPS^1)$ may
appear in the expressions in the codomain of $\SUBDRS{\aPS^1}$.

Item 5 is morally unchanged.  
In 5a, recall that
$\labelingSub(\aPS) = (\beforeSub\SUBSEQ\afterSub)$ and
$\labelingSub(\aEv)= (\beforeSub\SUBSEQ\SUB{\aLoc/\aReg}\SUBSEQ\afterSub)$.
Note that
$\SUBDRS{\aPS^1}\SUBSEQ \labelingSub^1(\bEv)$ is insensitive to the value assigned to $\aReg$ by
$\SUBDRS{\aPS^1}$.

As a simple example, consider the following:
\begingroup
\allowdisplaybreaks
\begin{gather*}
  \begin{gathered}
    r\GETS y
    \\[-1ex]
    \hbox{\begin{tikzinline}[node distance=.2em]
      \event{a}{\DRreg{r}{y}{1} \mid \SUB{y/r}}{}
      \final{f}{}{below=of a}
      \end{tikzinline}}
  \end{gathered}
  \qquad
  \begin{gathered}
    x\GETS r
    \\[-1ex]
    \hbox{\begin{tikzinline}[node distance=.2em]
      \event{b}{r\EQ1 \mid \DW{x}{1}}{}
      \final{f}{r\EQ1 \mid \SUB{r/x}}{below=of b}
      \end{tikzinline}}
  \end{gathered}
  \qquad
  \begin{gathered}
    s\GETS x
    \\[-1ex]
    \hbox{\begin{tikzinline}[node distance=.2em]
      % \event{c}{\DRreg{r}{x}{1} \mid \SUB{x/s}}{}
      % \final{f}{}{below=of c}
      \event{c}{\DRreg{s}{x}{2}}{}
      \final{f}{\SUB{x/s}}{below=of c}
      \end{tikzinline}}
  \end{gathered}
  \qquad
  \begin{gathered}
    z\GETS s
    \\[-1ex]
    \hbox{\begin{tikzinline}[node distance=.2em]
      \event{d}{\DW{z}{1}}{}
      \final{f}{\SUB{s/z}}{below=of d}
      \end{tikzinline}}
  \end{gathered}
  \\
  \begin{gathered}
    r\GETS y\SEMI x\GETS r \SEMI s\GETS x \SEMI z\GETS s
    \\[-1ex]
    \hbox{\begin{tikzinline}[node distance=1em]
        \event{a}{\DRreg{r}{y}{1}\mid \SUB{y/r, r/x, x/s, s/z}}{}
        \event{b}{\DW{x}{1}}{right=of a}
        \event{c}{\DRreg{s}{x}{2}}{right=of b}
        \event{d}{\DW{z}{1}}{right=of c}
        \final{f}{\SUB{r/x, x/s, s/z}}{right=of d}
        \po{a}{b}
        \po[out=-10,in=-160]{a}{d}
      \end{tikzinline}}
  \end{gathered}
\end{gather*}

To see the need for parallel substitution of all register values via $\SUBDRS{\aPS^1}$ in 4bc, consider that the
precondition of $(\DW{y}{1})$ must be $\FALSE$ after composing the following:
\begin{gather*}
  \begin{gathered}
    r\GETS x\SEMI s\GETS z
    \\[-1ex]
    \hbox{\begin{tikzinline}[node distance=.5em]
      \event{a}{\DRreg{r}{x}{1} \mid \SUB{x/r}}{}
      \event{b}{\DRreg{s}{z}{2} \mid \SUB{z/s}}{right=of a}
      \final{f}{}{right=of b}
      \end{tikzinline}}
  \end{gathered}
  \qquad
  \begin{gathered}
     \IF{r{=}s}\THEN y\GETS1\FI
    \\[-1ex]
    \hbox{\begin{tikzinline}[node distance=.5em]
      \event{c}{r{=}s\mid\DW{y}{1}}{}
      \final{f}{r{=}s\mid\SUB{1/y}}{right=of c}
      \end{tikzinline}}
  \end{gathered}
  \\
  \begin{gathered}
    r\GETS x\SEMI s\GETS z \SEMI \IF{r{=}s}\THEN y\GETS1\FI
    \\[-1ex]
    \hbox{\begin{tikzinline}[node distance=.5em]
      \event{a}{\DRreg{r}{x}{1} \mid \SUB{x/r}}{}
      \event{b}{\DRreg{s}{z}{2} \mid \SUB{z/s}}{right=of a}
      \nonevent{c}{1{=}2\mid\DW{y}{1}}{right=of b}
      \nonfinal{f}{1{=}2\mid\SUB{1/y}}{right=of c}
      \end{tikzinline}}
  \end{gathered}
\end{gather*}
This pomset candidate does not satisfy the \emph{compatibility} requirement
of Definition \ref{def:mmpomset}.  However, note that
$(\IF{r{=}s}\THEN y\GETS1\FI)$ is shorthand for
$(\IF{r{=}s}\THEN y\GETS1\ELSE\SKIP\FI)$, and thus we do have a valid pomset
for this composition, even with this choice of read values:
\begin{gather*}
  \begin{gathered}
    r\GETS x\SEMI s\GETS z
    \\[-1ex]
    \hbox{\begin{tikzinline}[node distance=.5em]
      \event{a}{\DRreg{r}{x}{1} \mid \SUB{x/r}}{}
      \event{b}{\DRreg{s}{z}{2} \mid \SUB{z/s}}{right=of a}
      \final{f}{}{right=of b}
      \end{tikzinline}}
  \end{gathered}
  \qquad
  \begin{gathered}
     \IF{r{=}s}\THEN y\GETS1\FI
    \\[-1ex]
    \hbox{\begin{tikzinline}[node distance=.5em]
      \final{f}{r{\neq}s}{}
      \end{tikzinline}}
  \end{gathered}
  \\
  \begin{gathered}
    r\GETS x\SEMI s\GETS z \SEMI \IF{r{=}s}\THEN y\GETS1\FI
    \\[-1ex]
    \hbox{\begin{tikzinline}[node distance=.5em]
      \event{a}{\DRreg{r}{x}{1} \mid \SUB{x/r}}{}
      \event{b}{\DRreg{s}{z}{2} \mid \SUB{z/s}}{right=of a}
      \final{f}{1{\neq}2}{right=of b}
      \end{tikzinline}}
  \end{gathered}
\end{gather*}

The see the need for substitutions on read actions, used in 5a, consider the
following, where
$\bSub=\SUB{0/x,\allowbreak x/r,\allowbreak {-}2/x}$:
\begin{gather*}
  \begin{gathered}
    x\GETS 0\SEMI r\GETS x\SEMI x\GETS{-2} 
    \\[-1ex]
    \hbox{\begin{tikzinline}[node distance=1em]
      \event{a}{\DW{x}{0}}{}
      \event{b}{\DRreg{r}{x}{1} \mid \bSub}{right=of a}
      \event{c}{\DW{x}{{-}2}}{right=of b}
      \wk{a}{b}
      \wk{b}{c}
      \final{f}{\SUB{{-2}/x}}{right=.5em of c}
      \end{tikzinline}}
  \end{gathered}
  \quad
  \begin{gathered}
    \IF{r{\geq}0}\THEN y\GETS1\FI
    \\[-1ex]
    \hbox{\begin{tikzinline}[node distance=1em]
      \event{d}{r{\geq}0\mid\DW{y}{1}}{}
      \final{f}{r{\geq}0\mid\SUB{1/y}}{right=.5em of d}
      \end{tikzinline}}
  \end{gathered}
  \\
  \begin{gathered}
    x\GETS 0\SEMI r\GETS x\SEMI x\GETS{-2} \SEMI \IF{r{\geq}0}\THEN y\GETS1\FI
    \\[-1ex]
    \hbox{\begin{tikzinline}[node distance=1em]
      \event{a}{\DW{x}{0}}{}
      \event{b}{\DRreg{r}{x}{1} \mid \SUBPAR{\bSub}{1/y}}{right=of a}
      \event{c}{\DW{x}{{-}2}}{right=of b}
      \wk{a}{b}
      \wk{b}{c}
      \event{d}{0{\geq}0\mid\DW{y}{1}}{right=of c}
      \final{f}{0{\geq}0\mid\SUB{{-2}/x,1/y}}{right=of d}
      \end{tikzinline}}
  \end{gathered}
\end{gather*}
\endgroup

The semantics validates expected equations, such as
\begin{math}
  \sem{\aCmd_1\SEMI\aCmd_2}\sequence\sem{\aCmd_3} =
  \sem{\aCmd_1}\sequence\sem{\aCmd_2\SEMI\aCmd_3},
\end{math}
\begin{math}
  \sem{\IF{\aExp} \THEN \aCmd \SEMI\bCmd_1\ELSE \aCmd\SEMI\bCmd_2\FI} =
  \sem{\aCmd}\sequence\sem{\IF{\aExp} \THEN \bCmd_1 \ELSE \bCmd_2\FI},
\end{math}
and
\begin{math}
  \sem{\IF{\aExp} \THEN \aCmd_1 \SEMI\bCmd\ELSE\allowbreak \aCmd_2\SEMI\bCmd\FI} =
  \sem{\IF{\aExp} \THEN \aCmd_1 \ELSE \aCmd_2\FI}\sequence\sem{\bCmd}.
\end{math}

The semantics is equivalent to that of the main text.

Let $\killS\aPSS$ be the set $\aPSS'\subseteq\aPSS$ where $\aPS'\in\aPSS'$
if $\labelingSub(\aEv')=\SUBEMP$, for every $\aEv'\in\Event'$.

\begin{proposition}
  \label{thm:seq}
  Let $\semold{\aCmd}$ be the semantics of \textsection\ref{sec:model},
  adopting the read actions of this section---the read substitution
  being $\SUBEMP$.
\begin{displaymath}
  \semold{\aCmd} = \killS\sem{\aCmd}
\end{displaymath}
\end{proposition}
 

% \newcommand{\rel}{{\tt rel}}


\section{Single Threaded Optimizations}

As we have seen already, our model {\em invalidates} thread inlining.  We argue that our model is fully flexible with respect to single threaded optimizations in two ways; firstly, by considering concrete examples of single threaded optimizations, and secondly, by proving a full abstraction theorem for single threads.  

In this section, we only consider commands $\aCmd$ that do not have any occurrence of $\PAR$. 

\begin{definition}
$\aCmd \unrhd \bCmd$ if $\sem{\aCmd} \supseteq \sem{\bCmd}$
\end{definition}
Thus, in this case $\aCmd$ can be transformed into $\bCmd$ in any program context.  

We follow the terminology and presentation of section~7.1 of
~\citet{Dolan:2018:BDR:3192366.3192421}, to maintain a clear comparison with models that enforce extra ordering.

\paragraph*{Peephole optimizations. } 
Certain transformations involving adjacent operations on the same location are permissible. 
 
\begin{lemma}
Forall $\PAR$-free $\aCmd$,
\begin{eqnarray*}
\mbox{Redundant load} &&\aReg \GETS \aLoc \SEMI \bReg \GETS \aLoc  \SEMI \aCmd \unrhd\  \aReg \GETS \aLoc \SEMI \bReg \GETS \aReg \SEMI \aCmd \\
\mbox{Store forwarding} &&
\aLoc \GETS \aReg \SEMI \bReg \GETS \aLoc \SEMI \aCmd\unrhd\ \aLoc \GETS \aReg \SEMI \bReg \GETS \aReg \SEMI \aCmd \\
\mbox{Dead Store} && \aLoc \GETS \aReg \SEMI \aLoc \GETS \bReg \SEMI \aCmd \unrhd\ \aLoc \GETS \bReg \SEMI \aCmd
\end{eqnarray*}
\end{lemma}
The proof of correctness of peephole optimisations goes via arguing directly with the semantic definitions.  

Dead store follows from the definition of write, by noting that  $\guard \DW{\aLoc}{} \guard \sem{\aLoc \GETS \bReg \SEMI \aCmd}[\aExp/\aLoc] =  \sem{\aLoc \GETS \bReg \SEMI \aCmd}[\aExp/\aLoc]$. 
  
\paragraph*{Reorderings of independent statements}

$\aLoc$ and $\aCmd$ are independent if there are no writes to $\aLoc$ or reads from $\aLoc$ in $\aCmd$.  
\begin{lemma}[Reorderings]
Let $\aLoc$ and $\aCmd$ be independent.  Let there be no release actions in $\aCmd$.  Then:
\begin{align*}
& \sem{\aLoc = \aReg \SEMI \aCmd} = \sem{\aLoc = \aReg \PAR \aCmd} \\
& \sem{\aReg = \aLoc \SEMI \aCmd} = \sem{\aReg= \aLoc \PAR \aCmd} 
\end{align*}
\end{lemma}
The proof follows immediately from the semantics of prefixing where the only enforced $\lt,\gtN$ relationships come from conflict on $\aLoc$ or release actions in $\aCmd$.

\paragraph*{Compiler optimisations.} Reordering and peephole optimisations can be combined to  to describe common  compiler optimizations.  We illustrate following~\citet{Dolan:2018:BDR:3192366.3192421}.

Common subexpression elimination. Consider the command 
\[ \aReg \GETS \aLoc *2  \SEMI \aCmd \SEMI \bReg \GETS \aLoc * 2 \]
where $\aCmd$ is independent of $\bReg$.  Then, reordering yields
\[\aCmd \SEMI \aReg \GETS \aLoc *2  \SEMI  \bReg \GETS \aLoc * 2 \]
followed by redundant load to yield:
\[\aCmd \SEMI \aReg \GETS \aLoc * 2 \SEMI  \bReg \GETS \aReg\]

Similarly, the treatment of loop-invariant code motion, dead-store elimination and constant propagation from~\citet{Dolan:2018:BDR:3192366.3192421} follows mutatis-mutandis, since our model is more generous about permitted reorderings.  Thus, we permit  redundant store elimination that they forbid.
Consider:
\[ \aReg \GETS \aLoc \SEMI \bLoc \GETS \cLoc  \SEMI \aLoc \GETS \aReg \]
Reordering, permitted by us, but forbidden by them, yields
\[ \aReg \GETS \aLoc \SEMI \aLoc \GETS \aReg \SEMI \bLoc \GETS \cLoc  \]
followed by the valid elimination of redundant load:
\[ \aReg \GETS \aLoc \SEMI \aLoc \GETS \aReg \SEMI \bLoc \GETS \cLoc  \]

\endinput










\subsection{Single thread full abstraction}
We first develop some infrastructure to identify the minimal elements, wrt augmentation, $\aCmd$.  
\begin{definition}
$\aPS$ is a generator of  $\sem{\aCmd}$ if for all $\bPS \in \sem{\aCmd}$ such that $\aPS$ augments $\bPS$, $\aPS = \bPS$.
\end{definition}

$\reco$ is the specialization of the $\gtN$ to a per-location basis.  
\begin{definition}
 $ \aEv' \xeco  \bEv'$ if both $\aEv'$ and $\bEv'$ touch the same location, $\bEv' \gtN \aEv'$, and at least one of them is a write.  
\end{definition}

\begin{lemma}\label{onethread}
Let $\aCmd$ be a thread.  Then, all generators $\aPS$ of  $\semClosed{\aCmd}$  are such that:
\begin{itemize}
\item  If $\aEv\lt\bEv$ and $\aEv$ is a release or a write, then $\bEv$ is a release action.
\item  If $\aEv\lt\bEv$ and $\bEv$ is an acquire or a read, then $\aEv$ is an acquire action.
%\item  If both $\aEv$ and $\bEv$ touch the same location and, 
%then $\bEv \gtN \aEv$ or $\aEv \gtN \bEv$.
 \item  $\aEv \gtN \bEv$ if and only if $ \aEv [\lt \cup (\le; \reco; 
\le)] \bEv$. 
\end{itemize}
\end{lemma}
The proof proceeds considering the minimal requirements on the order relations imposed by the semantics.  The key case is prefixing.  The only required $\lt$-edges out of writes are into release actions, and the only required $\lt$-edges into reads arise from read actions. 

Any optimisation which does not shrink the subrelations of $\rpox$ listed in lemma~\ref{onethread} is permissible.  







In particular, if there are no release or acquire actions in $\aCmd$:
\begin{itemize}
\item Writes are maximal
\item Reads are minimal
\end{itemize}
thus providing a particularly simple view of the semantics of threads without synchronization primitives.   
% \section{Invariant Reasoning in Temporal Logic}
\label{sec:logic}

A significant challenge for a software memory model is to relax order enough
to allow efficient implementation without admitting anomalous
behaviors---called \emph{out of thin air} (\oota) in the literature
\cite{vacuous,DBLP:conf/esop/BattyMNPS15,BoehmOOTA}.  The most famous example is:
\begin{align}
  \label{oota1}
  (y\GETS x \PAR r\GETS y\SEMI x\GETS r)
  &&
  %\nonumber
  \hbox{\begin{tikzinline}[node distance=1em]
  \event{rx}{\DR{x}{1}}{}
  \event{wy}{\DW{y}{1}}{right=of rx}
  \po{rx}{wy}
  \event{ry}{\DR{y}{1}}{right=2em of wy}
  \event{wx}{\DW{x}{1}}{right=of ry}
  \po{ry}{wx}
  \rf{wy}{ry}
  \rf[out=170,in=10]{wx}{rx}
    \end{tikzinline}}
\end{align}
Although Java does not allow \oota{} behaviors of \eqref{oota1},
\citet{DBLP:journals/toplas/Lochbihler13} showed that it does allow \oota\
behaviors of \eqref{lochbihler}, from \textsection\ref{sec:intro}.
\citet{DBLP:conf/lics/JeffreyR16} described a logic that rules out \eqref{oota1} but not \eqref{lochbihler}.  In this section, we  provide a more accurate test of \oota{} behaviors  by enhancing their logic with temporal features.
% In this case, the violation is a subtler
% temporal property.  We develop a logic sufficient to prove that our semantics
% disallows \oota\ on \eqref{lochbihler}.  
The logic is not meant to be
definitive; in \textsection\ref{sec:outro}, we discuss \oota{} examples that require non-trivial extensions.

We adapt past linear temporal logic (\pLTL)
\cite{Lichtenstein:1985:GP:648065.747612} to pomsets by dropping the previous
instant operator and adopting strict versions of the temporal operators.
The atoms of our logic are write and read events.
% \begin{displaymath}
%   \afo \QUAD::=\QUAD
%   \DR{\aLoc}{\aVal}
%   \mid
%   \DW\aLoc\aVal
%   \afo \wedge\bfo
%   \mid \lnot \afo
%   \once\afo
%   \mid \always\afo
% \end{displaymath}
%\begin{definition} %[Satisfaction]
Given a pomset $\aPS$ and event $\aEv$, define\footnote{Let $\FALSE$, $\lor$,
  $\Rightarrow$ and $\once$ as usual;
  for example,
  $\once\afo = \lnot(\always\lnot\afo)$.}:
  \begin{displaymath}
    \begin{array}{lrl}
      \aPS,\aEv &\models& \DW{\aLoc}{\aVal}, \text{ if } \labelingAct(\aEv) = \DW{\aLoc}{\aVal} \text{ and } \TRUE \text{ implies } \labelingForm(\aEv) \\
      \aPS,\aEv &\models& \DR{\aLoc}{\aVal}, \text{ if } \labelingAct(\aEv) = \DR{\aLoc}{\aVal} \text{ and } \TRUE \text{ implies } \labelingForm(\aEv) \\
      \aPS,\aEv &\models& \afo\land\bfo, \text{ if } \aPS,\aEv \models  \afo \text{ and } \aPS,\aEv \models  \bfo \\
      \aPS,\aEv &\models& \TRUE\\
      \aPS,\aEv &\models& \lnot\afo, \text{ if } \aPS,\aEv \not\models \afo \\
      \aPS,\aEv &\models& \once\afo,\mkern1.5mu \text{if } (\exists \bEv \lt \aEv)\;  \aPS,\bEv \models \afo \\
      \aPS,\aEv &\models& \always\afo, \text{ if } (\forall \bEv \lt \aEv)\; \aPS,\bEv \models \afo
    \end{array} 
  \end{displaymath}

  % \begin{definition}
  Let $\aPS \models \afo$ if
  $\aPS,\aEv \models\afo$, for all $\aEv \in \Event$.

  Let $\aPSS\models \afo$
  if $\aPS \models\afo$, for all $\aPS \in \aPSS$.
  
Let
  \begin{math}
    \afo, \aPSS \models \bfo  \text{ if } \{ \bPSS \mid \bPSS \models \afo \} \parallel \aPSS \models \bfo.
  \end{math}
%\end{definition}

  Let $\afo$ be \emph{prefix closed} when
  $\{ \bPSS \mid \bPSS \models \afo \}$ is.

% Thus, $\aPS\models \afo \land \always\afo$ whenever $\aPS \models
  % \afo$. This fact relies on the use of universal quantification in the definition.

% We define other connectives as standard:
% $\once\afo = \lnot(\always\lnot\afo)$,
% %$\FALSE = \lnot(\TRUE)$
% $\afo\lor\bfo = \lnot(\lnot(\afo)\land\lnot(\bfo))$, and
% $\afo\Rightarrow\bfo = \lnot(\afo) \lor\ \bfo$.
% \begin{displaymath}
% \begin{array}{lrl}
% \once\afo &=& \lnot(\always\lnot\afo) \\
% \FALSE &=& \lnot(\TRUE) \\
% \afo\lor\bfo &=& \lnot(\lnot(\afo)\land\lnot(\bfo)) \\
% \afo\Rightarrow\bfo &=& \lnot(\afo) \lor\ \bfo
% \end{array}
% \end{displaymath}
%Let $$ be defined as $$. 
%In addition, let $\FALSE$, $\lor$ and $$ be defined in the
%standard way.
% $\afo\lor\bfo$ for $\lnot(\lnot \afo \land \lnot \bfo)$,
% and $\afo \Rightarrow \bfo$ for $\lnot \afo \lor \bfo$.
  The past operators do not include the current instant, and so
  do \emph{not} satisfy
  $(\always\afo\Rightarrow\once\afo)$\footnote{The order-minimal elements always validate
    $\always\afo$ and invalidate
    $\once\afo$.}.
  However, the following hold:
% \begin{align*}  
%   \frac{\aPS \models \afo \Rightarrow\once{\afo}}{\aPS \models \lnot \afo}\text{(Coinduction)}
%   &&
%   \frac{\aPS \models \always\afo \Rightarrow\afo}{\aPS \models \afo}\text{(Induction)}
% \end{align*}
% \begin{lemma}
% Given an pomset $\aPS$.  
\begin{align*}
  \tag{Induction}
  \aPS \models& (\always\afo \Rightarrow\afo) \Rightarrow\afo
  \\[-1ex]
  \tag{Coinduction}
  \aPS \models& (\afo \Rightarrow\once{\afo}) \Rightarrow\lnot \afo
  \\[-1ex]
  \tag{Weakening}
  \aPS \models& (\afo \Rightarrow\once{\bfo}) \Rightarrow (\once\afo \Rightarrow\once{\bfo})
\end{align*}
% \end{lemma}
% \begin{proof}
% We prove that any node in a pomset satisfies these formulas.  
%The proof for both rules proceeds by induction on the length of the maximal path from a root to a node. 
%\end{proof}

% \begin{description}
% \item[Coinduction.]
%   \begin{math}
%     (\afo \Rightarrow\once{\afo}) \Rightarrow\lnot \afo
%   \end{math}
% \item[Induction.] 
%   \begin{math}
%     (\always\afo \Rightarrow\afo) \Rightarrow\afo
%   \end{math}
% \end{description}


%We now present two proof rules for programs. 

%\paragraph*{Proof rules for programs}
We present two proof rules for programs. 
The first provides a logical view of \emph{$\aLoc$-closure} (Definition~\ref{def:rf}):
%The soundness proof is straightforward.
% \begin{math}
%   \closed(\aLoc) = (\DR{\aLoc}{\aVal} \Rightarrow \once \DW{\aLoc}{\aVal}).
% \end{math}
% Although this definition does not mention intervening writes, it is
% sufficient for our example.  
\begin{displaymath}
  %\tag{Closing $\aLoc$}
  \frac{
    \afo \text{ is independent of } \aLoc
    \qquad
    %\aPS \models \closed(\aLoc) \Rightarrow \afo
    \aPS \models (\DR{\aLoc}{\aVal} \Rightarrow \once \DW{\aLoc}{\aVal}) \Rightarrow \afo
  }{
    \nu \aLoc \DOT \aPS \models \afo
  }
\end{displaymath}
%It is straightforward to establish that this rule is sound.
% Although it does
% not mention intervening writes, the rule is sufficient for our examples.

The second rule describes concurrent composition, in the style of~\citet{Abadi:1993:CS:151646.151649}.  To simplify the presentation, we
consider the special case with a single invariant.
% We view the
% composition result as capturing key aspects of no-ThinAirRead, as will become
% clearer in the examples below.
% In order to state the theorem, we generalize the satisfaction relation to
% include environment assumptions.

\begin{proposition}%[Composition]
  Let $\afo$ be prefix-closed.  Let $\aPSS_1, \aPSS_2$ be
  augmentation\hyp{}closed.%\footnote{$\aPS'$ is an augmentation of $\aPS$ if
 %   $\Event'=\Event$, $\aEv\le\bEv$ implies $\aEv\le'\bEv$, $\aEv\gtN\bEv$
 %   implies $\aEv\gtN'\bEv$, and
 %   % $\labeling'(\aEv)=\labeling(\aEv)$
 %   if $\labeling(\aEv) = (\bForm \mid \bAct)$ then
 %   $\labeling'(\aEv) = (\bForm' \mid \bAct)$ where $\bForm'$ implies
 %   $\bForm$.}
  Then:
  \begin{displaymath}
    %\tag{Composition}
    \frac{
      \afo, \aPSS_1 \models\afo
      \qquad
      \afo, \aPSS_2 \models\afo
    }{\aPSS_1 \parallel \aPSS_2 \models \afo}
  \end{displaymath}
\end{proposition}
\begin{proof}[Proof sketch]
  We will show that all prefixes in the prefix closures of
  $\aPSS_1 \parallel \aPSS_2$ satisfy the required property.  Proof proceeds
  by induction on prefixes of $\aPS \in \aPSS_1 \parallel \aPSS_2$.
  %
  The case for empty prefix  follows from assumption that  $\afo$ is prefix closed.  
  %
  For the inductive case, consider %$\aPS$ in the prefix closure of $\aPSS_1 \parallel \aPSS_2$, i.e.
  $\aPS \in \aPS_1 \parallel \aPS_2$ where
  $\aPS_i \in \aPSS_i$.  Since $\aPSS_1$ and $\aPSS_2$ are augmentation
  closed, we can assume that the restriction of $\aPS$ to the events of
  $\aPS_i$ coincides with $\aPS_i$, for $i=1,2$.
  %
  Consider a prefix $\aPS'$ derived by removing a maximal element $\aEv$ from
  $\aPS$.  Suppose $\aEv$ comes from $\aPS_1$ (the other case is
  symmetric). Since $\aPS_2$ is a prefix of $\aPS'$ and $\aPS' \models \afo$
  by induction hypothesis, we deduce that $\aPS_2 \models \afo$.
  % Thus, $\aPS_2 \in \mods{(\afo)}$.
  Since $\aPS_1 \in \aPSS_1$, by assumption $\afo, \aPSS_1 \models\afo$ we
  deduce that $\aPS \models \afo$.
\end{proof}
When all variables are initially bound to $0$, we show that \eqref{oota1}
satisfies
\begin{math}
  \lnot\DW{x}{1}.
\end{math}
 We start with the invariant:
\begin{displaymath}
  [\DW{x}{1}\Rightarrow\once\DR{y}{1}]
  \land
  [\DW{y}{1}\Rightarrow\once\DR{x}{1}]
\end{displaymath}
This invariant holds for each thread; thus, it holds for the
aggregate program by composition.  Closing $y$ yields
\begin{math}
  \DR{y}{1} \Rightarrow \once\DW{y}{1}.
\end{math}
Weakening the right conjunct: % yields
\begin{math}
  \once\DW{y}{1}\Rightarrow\once\DR{x}{1}.
\end{math}
Chaining these together: %yields
\begin{math}
  \DR{y}{1} \Rightarrow \once\DR{x}{1}.
\end{math}
Weakening:  %yields
\begin{math}
  \once\DR{y}{1} \allowbreak\Rightarrow \once\DR{x}{1}. 
\end{math}
Chaining into the left conjunct:  %yields
\begin{math}
  \DW{x}{1} \Rightarrow \once\DR{x}{1}. 
\end{math}
Closing $x$, 
% \begin{math}
%   \DR{x}{1} \Rightarrow \once\DW{x}{1}.
% \end{math}
weakening, 
% \begin{math}
%   \once\DR{x}{1} \Rightarrow \once\DW{x}{1}.
% \end{math}
then chaining: %, yields
\begin{math}
  \DW{x}{1} \Rightarrow \once\DW{x}{1}. 
\end{math}
By coinduction, 
\begin{math}
  \lnot\DW{x}{1},
\end{math}
as required.

We consider a variant of
\citeauthor{DBLP:journals/toplas/Lochbihler13}'s example \eqref{lochbihler}:
% A more general principle, in the spirit of~\citet{Abadi:1993:CS:151646.151649} can be proved.  We chose the simple case of temporal invariants to illustrate the idea in a simple form.  Even this simple version has interesting consequences. 
\begin{align*}
  \label{alan}
  %   Z=1;
  % ||
  %   a=X; // 1
  %   Y=a;
  % ||
  %   b=Z; // 0
  %   if(b){
  %     X=1
  %   } else {
  %     c=Y; // 1
  %     X=c;
  %     W=c;
  %   }
  %\VAR  x\GETS0\SEMI \VAR  y\GETS0\SEMI \VAR  z\GETS0\SEMI
  z\GETS1
  \PAR
    y\GETS x
  \PAR
    \IF{z}\THEN x\GETS1 \ELSE r\GETS y \SEMI x\GETS r \SEMI a\GETS r \FI
\end{align*}
This variant retains the essential temporal aspects of \eqref{lochbihler}.
In this case, all threads satisfy the invariant ``a write to $a$ is preceded
by a read of $z$ as $1$.''

Attempting to write $1$ to $a$ results in a cycle:
\begin{tikzdisplay}[node distance=1em]
  \event{rx}{\DR{x}{1}}{}
  \event{wz1}{\DW{z}{1}}{left=2em of rx}
  \event{wy}{\DW{y}{1}}{right=of rx}
  \po{rx}{wy}
  \event{rz}{\DR{z}{0}}{right=2em of wy}
  \event{ry}{\DR{y}{1}}{right=of rz}
  \event{wx}{\DW{x}{1}}{right=of ry}
  \event{wa}{\DW{a}{1}}{right=of wx}
  \po{ry}{wx}
  \rf[out=-15,in=-165]{wy}{ry}
  \rf[out=170,in=10]{wx}{rx}
  \po[out=10,in=170]{rz}{wa}
  \po[out=-15,in=-165]{ry}{wa}
\end{tikzdisplay}
% \begin{tikzdisplay}[node distance=1em]
%   \event{wy0}{\DW{y}{0}}{}
%   \event{rx}{\DR{x}{1}}{right=4.5em of wy0}
%   \event{wy}{\DW{y}{1}}{right=of rx}
%   \po{rx}{wy}
%   \wk[bend left]{wy0}{wy}
%   \event{wx0}{\DW{x}{0}}{below=of wy0}
%   \event{rz}{\DR{z}{0}}{right=of wx0}
%   \event{ry}{\DR{y}{1}}{right=of rz}
%   \event{wx}{\DW{x}{1}}{right=of ry}
%   \event{ry1}{\DR{y}{1}}{right=of wx}
%   \event{wa}{\DW{a}{1}}{right=of ry1}
%   \rf{wy}{ry1}
%   \po{ry}{wx}
%   \wk[bend right]{wx0}{wx}
%   \rf{wy}{ry}
%   \rf{wx}{rx}
%   \event{wz0}{\DW{z}{0}}{below=of wx0}
%   \event{wz1}{\DW{z}{1}}{right=of wz0}
%   \rf{wz0}{rz}
%   \wk{wz0}{wz1}
%   \po{ry1}{wa}
%   \po[bend right]{rz}{wa}
% \end{tikzdisplay}

We prove the formula
\begin{math}
  \lnot\DW{a}{1},
\end{math}
starting with invariant:
% which holds for each of the three threads, and thus, by composition, for the
% aggregate program:
\begin{scope}
\small
\begin{align*}
  [\once\DW{y}{1} \Rightarrow \once\DR{x}{1}]
  \land
  [\notonce\DW{a}{1} \Rightarrow (\once\DR{y}{1} \land \always(\DW{x}{1} \Rightarrow \once\DR{y}{1}))]
\end{align*}
\end{scope}
Closing $y$ and chaining into the left conjunct:
% \begin{math}
%   \once\DR{y}{1} \Rightarrow \once\DW{y}{1}. % \Rightarrow \once\DR{x}{1}
% \end{math}
% Chaining this implication on the left:
\begin{math}
  \once\DR{y}{1} \Rightarrow \once\DR{x}{1}.
\end{math}
% We can weaken this to:
% \begin{math}
%   \once\DR{y}{1} \Rightarrow \once\DR{x}{1}. % \Rightarrow \once\DR{x}{1}
% \end{math}
Chaining into the right conjunct:
\begin{displaymath}
  \notonce\DW{a}{1} \Rightarrow (\once\DR{x}{1} \land \always(\DW{x}{1} \Rightarrow \once\DR{x}{1}))
\end{displaymath}
Closing $x$:
% \begin{math}
%   \once\DR{x}{1} \Rightarrow \once\DW{x}{1}.
% \end{math}
%  Weakening and chaining again:
%we can replace $\once\DR{x}{1}$ with $\once\DW{x}{1}$:
\begin{math}
  \notonce\DW{a}{1} \Rightarrow (\once\DW{x}{1} \land \always(\DW{x}{1} \Rightarrow \once\DW{x}{1}).
\end{math}
Applying coinduction to the right conjunct:
\begin{displaymath}
  \notonce\DW{a}{1} \Rightarrow (\once\DW{x}{1} \land \always(\lnot \DW{x}{1}))
\end{displaymath}
Simplifying:
\begin{math}
  \notonce\DW{a}{1} \Rightarrow \FALSE,
\end{math}
as required.

Many examples are superficially similar, but can write $a$ without cycles.
In the following, $\DW{x}{1}$ is independent of $\DR{y}{1}$.
\begin{gather*}
    y\GETS x
  \PAR
    \IF{y}\THEN r\GETS y\SEMI x\GETS r\SEMI a\GETS r \ELSE x\GETS1 \FI
  \\
  \hbox{\begin{tikzinline}[node distance=1em]
  \event{rx}{\DR{x}{1}}{}
  \event{wy}{\DW{y}{1}}{right=of rx}
  \po{rx}{wy}
  \event{ry}{\DR{y}{1}}{right=2em of wy}
  \event{wx}{\DW{x}{1}}{right=of ry}
  \event{wa}{\DW{a}{1}}{right=of wx}
  \rf[out=-15,in=-165]{wy}{ry}
  \rf[out=170,in=10]{wx}{rx}
  \po[out=-15,in=-165]{ry}{wa}
    \end{tikzinline}}
\end{gather*}


% \endinput

% \paragraph{Load buffering and thin air.}
% The program
% \begin{math}
%   %x\GETS0\SEMI y\GETS0\SEMI
%   (y\GETS x \PAR \bReg\GETS y\SEMI x\GETS1)
% \end{math}
% has top level executions that result in the final outcome $x = y = 1$, such as:
% \begin{tikzdisplay}[node distance=1em]
%   % \event{wx0}{\DW{x}{0}}{}
%   % \event{wy0}{\DW{y}{0}}{below=wx0}
%   \event{rx}{\DR{x}{1}}{}
%   \event{wy}{\DW{y}{1}}{right=of rx}
%   \po{rx}{wy}
%   \event{ry}{\DR{y}{1}}{right=3em of wy}
%   \event{wx}{\DW{x}{1}}{right=of ry}
%   \rf{wy}{ry}
%   \rf[out=170,in=10]{wx}{rx}
%   %\po{rx}{wy}
% \end{tikzdisplay}
% In \textsection\ref{sec:logic} we provide machinery to prove that this
% outcome is impossible if there is order from read to write in both
% threads.  This order can be achieved by replacing the second thread
% \begin{math}
%   (\bReg\GETS y\SEMI x\GETS1)
% \end{math}
% with 
% \begin{math}
%   (\bReg\GETS y\ACQ\SEMI x\GETS1)
% \end{math}
% or
% \begin{math}
%   (\IF{y}\THEN x\GETS 1\FI)
% \end{math}
% or
% \begin{math}
%   (x\GETS y).
% \end{math}

% A more interesting example is the following variant of \eqref{types}:
% \begin{displaymath}
%   %\label{alan}
%   % x\GETS0\SEMI
%   %y\GETS0\SEMI   
%   (
%     y\GETS x
%   \PAR
%     \IF{z}\THEN x\GETS1 \ELSE x\GETS y\SEMI a\GETS y \FI
%   \PAR
%     z\GETS0\SEMI z\GETS1
%   )
% \end{displaymath}
% This program is allowed to write $1$ to $a$ under many speculative
% memory models
% \cite{Manson:2005:JMM:1047659.1040336,DBLP:conf/esop/JagadeesanPR10,DBLP:conf/popl/KangHLVD17},
% even though the read of $1$ from $y$ in the else branch of the second
% thread arises out of thin air.   \citet{DBLP:journals/toplas/Lochbihler13}
% argues that such executions compromise type safety unless object allocation
% partitions memory by type.
% In our model, the attempted execution is:
% \begin{tikzdisplay}[node distance=1em]
%   \event{rx}{\DR{x}{1}}{}
%   \event{wy}{\DW{y}{1}}{below=of rx}
%   \po{rx}{wy}
%   \event{ry}{\DR{y}{1}}{right=of rx}
%   \event{wx}{\DW{x}{1}}{below=of ry}
%   \po{ry}{wx}
%   \rf{wy}{ry}
%   \rf{wx}{rx}
%   \event{rz}{\DR{z}{0}}{right=of ry}
%   \event{wz0}{\DW{z}{0}}{right=of rz}
%   \rf{wz0}{rz}
%   \event{wz1}{\DW{z}{1}}{right=of wz0}
%   \wk{wz0}{wz1}
%   \event{ry1}{\DR{y}{1}}{below=of rz}
%   \rf[bend right]{wy}{ry1}
%   \event{wa}{\DW{a}{1}}{right=of ry1}
%   \po{ry1}{wa}
%   \po{rz}{wa}
% \end{tikzdisplay}
% This is forbidden by the evident cycle.


% \begin{verbatim}



% y=x+1; a=y || x=y
% prove a!=2

% Wyv_1 /\ Wyv_2 => v_1 == v_2 (and maybe v_1==0 \/ v_2==0)
% Wx1 => <>-1 Ry1
% Wy1 => <>-1 Rx1
% \end{verbatim}

% Local Variables:
% mode: latex
% TeX-master: "paper"
% End:

% \section{Data Race Free Behaviors are Sequentially Consistent}
\label{sec:sc}

% For any $\aPSS$, then $\closed(\aPSS)$ is set enriched with useless reads
% (preserving augmentation closure) and where we remove any event whose
% precondition is not a tautology.
% \begin{definition}
%   Let $\addRead(\aPSS)$ be the set $\aPSS'$ where $\aPS'\in\aPSS'$ whenever
%   there is $\aPS\in\aPSS$ such that:
%   $\Event' = \Event\cup{\cEv}$,
%   ${\le'} \supseteq {\le}$, 
%   ${\gtN'} \supseteq {\gtN}$,
%   and
%   $\labelingAct'(\cEv) = (\DR{\aLoc}{\aVal})$ and $\labelingAct'(\aEv) = \labelingAct(\aEv)$,
% \end{definition}
% Then $\fclosed(\aPSS)$ 



In this section, we prove the SC-DRF theorem, which states that any program
that lacks data races under the SC semantics must only have executions that
are compatible with SC executions.  We present the result for programs of the
form $\vec{\aLoc}\GETS\vec{0}\SEMI\aCmd$, where $\aCmd$ is restriction-free.  Thus, all memory locations are initialized to $0$, and internal actions only arise from (intra-thread) implicit reads.

We say that two actions have a \emph{data-race conflict} if at least one
action is a write and the other is a write, read, or internal read to the
same location.  Define the relation $\reco$ so that $(\aEv,\bEv)\in{\reco}$
if $\aEv\gtN\bEv$ and $\aEv$ and $\bEv$ have a data-race conflict.

The program $x\GETS1\PAR x\GETS2$ is considered to have an SC data race, but
$x\GETS1\SEMI x\GETS2$ does not.  In our semantics the only difference
between these is that $x\GETS1\SEMI x\GETS2$ enforces weak order between the
writes.  Note also that the
$\sem{x\GETS1\SEMI a\GETS y}=\sem{a\GETS y\SEMI x\GETS1}$, yet these two must
be distinguished in SC, as per the load-buffering and store-buffering litmus tests.

In order to define SC executions and SC data races, it is necessary to
augment our semantics to record program order and internal reads.  We extend the definitions in
\textsection\ref{sec:semantics} with
${\rpox}\subseteq{\Event}\times{\Event}$, defined as follows:
\begin{itemize}
\item
  ${\rpox'} = {\rpox}$
  when $\aPSS'=\aPSS\aSub$
  or $\aPSS'=\aForm\guard\aPSS$
\item
  ${\rpox'} = {\rpox}\restrict{\Event'}$
  when $\aPSS'=\nu\aLoc\st\aPSS$
\item
  ${\rpox'} = {\rpox}^1\cup{\rpox}^2$
  when $\aPSS'=\aPSS^1\parallel\aPSS^2$
\item
  ${\rpox'} = {\rpox}\cup\{(\cEv,\aEv)\mid\aEv\in\Event\}$
  when $\aPSS'=\aAct\prefix\aPSS$ and $\Event' = \Event \cup \{\cEv\}$
\end{itemize}

Define the relation ${\rrfx}$ so that $(\aEv,\bEv)\in{\rrfx}$ if $\aEv$
writes $\aLoc$, $\bEv$ reads $\aLoc$, and for any $\cEv$ that writes $\aLoc$
either $\cEv\gtN\aEv$ or $\bEv\gtN\cEv$.  Let $\IDAcq$ be the identity
relation on acquire events, and likewise $\IDRel$ on release events.  Now
define $\rsw$ and $\rhb$\footnote{For simplicity, the definition of $\rsw$
  does not include release sequences or fences.  For example, we consider
  \begin{math}
    (\DWRel{x}{1})\prefix(\DW{x}{2}) \parallel (\DRAcq{x}{2})
  \end{math}
  to be racy, whereas this pomset is race-free using release sequences.  If
  we include release sequences and fences, then $\rhb$ relates more events
  and thus there are fewer races. Our results hold under either definition.}.
\begin{align*}
  {\rsw} &= \IDRel; ({\rrfx}\setminus{\rpox}); \IDAcq
  \\
  {\rhb} &= ({\rpox} \cup {\rsw})^+
\end{align*}
Note that our semantics guarantees that ${\rsw}\subseteq{\lt}$.

A pomset has a \emph{data race} if there are events $\aEv$ and $\bEv$ such
that
\begin{itemize}
\item $\aEv$ and $\bEv$ are unordered by $\rhb$,
\item $\labelingForm(\aEv)$ and $\labelingForm(\bEv)$ are tautologies, and
\item $\labelingAct(\aEv)$ and $\labelingAct(\bEv)$ have a data-race conflict.
\end{itemize}

%The semantics of programs includes SC executions.
\begin{definition}
  Let $\semsc{\aCmd}$ be the subset of $\sem{\aCmd}$ such that
  $\aPS\in\semsc{\aCmd}$ whenever $\aPS$ is a top-level pomset and ${\lt}\cup{\rpox}$ is acyclic.
% ${\lt}\cup{\rpox} \cup {\reco}$ is acyclic.
\end{definition}
%\begin{itemize}
%\item ${\le}\cup{\rpox} \cup {\reco}$ is acyclic ,
% \item prefixing ($\prefix$) and composition ($\parallel$) take disjoint union, and
%\item all reads are denoted by explicit actions.
%\end{itemize}
We argue that this definition is sufficient to capture sequential
consistency: Any total order that linearizes the acyclic relation is
consistent with strong order ($\lt$) and the program order ($\rpox$).  Since $\le$ contains $\reco$, the consistency with $\reco$ ensures that only the last write to a location is
read in such a total order.

%\begin{remark}\label{generator}
% In the rest of this section, we consider ``top-level'' programs of the form
% $\aCmd = \VAR\vec{\aLoc}\SEMI
%     \vec{\aLoc}\GETS\vec{0}\SEMI
%     \vec{\bLoc}\GETS\vec{0}\SEMI
%     \FENCE\SEMI
%     (\aCmd_1 \PAR \cdots \PAR \aCmd_n)
% $.
% We will consider closed and complete executions of $\aCmd$. We use $\semClosed{\aCmd}$ to stand for the subset of $\sem{\aCmd}$  with only  the pomsets that are $\vec{\aLoc}, \vec{\bLoc}$ closed.  
We only consider \emph{generators}, which are top-level pomsets that are
minimal with respect to augmentation and implication.  Since we are
considering finite programs without loops, the pomsets in the semantics of
threads are finite.  Thus, there are no infinite descending chains of
augmentations.


%\end{remark}

We prove the following theorem in \textsection\ref{drfproof}.
\begin{theorem}
  Let $\aPS$ be a generator for $\aCmd$.
  \begin{itemize}
  \item If $\aPS$ does not have a data race, $\aPS \in \semsc{\aCmd}$.
  \item If $\aPS$ has a data race, then there exists
    $\aPS'\in \semsc{\aCmd}$ that has a data race.
  \end{itemize}
\end{theorem}
A key step of this proof is an analysis of the closure properties of the semantics.  In order to perform this fine grained analysis of dependency, we describe a variant of the semantics using three valued pomsets defined below.  
\begin{definition}
  \label{def:tvalpom}
  A \emph{\tvalpom} is a tuple
  $(\Event, {\sle}, {\gtN},
  \labeling)$, such that
  \begin{itemize}
   \item $(\Event, {\gtN},
  \labeling)$ is a pomset, and 
\item ${\sle} \subseteq {\gtN}$ is a partial order. 
  \end{itemize}
\end{definition}
\citet{DBLP:conf/esop/HuthJS01} call $\slt$ a ``must transition''
and $\geN$ a ``may transition''--- we have used  the terms ``strong order'' and ``weak order'' respectively.  In the pictures so far, we have anticipated this development and adopted
\citeauthor{DBLP:journals/dc/Lamport86}'s notation, drawing $\slt$ as a solid arrow ``$\xpo$'' and $\geN$ dashed ``$\xwk$''.  The intuitive temporal meaning of $ \aEv \lt \bEv$ is that $\aEv$ {\em must} strictly precede $\bEv$, whereas $ \aEv \gtN \bEv$ is intended to connote that $\aEv$ may precede $\bEv$. Thus, if  $ \neg (\aEv \gtN \bEv)$, $\aEv$ cannot precede $\bEv$.  
 
The semantics of programs in the three-valued model proceed mutatis mutandis, with details described in the appendix.  This finer analysis of necessary and possible dependency allows us to establish the existence of pomsets in the semantics as we search for sequential witnesses to data races.    We provide some illustrative examples below.

\noindent
For the program:
\begin{align*}
\mbox{Thread 1: } &\VAR y\GETS 0 \SEMI \aReg \GETS y  \SEMI x \GETS 1  \SEMI \\[-.5ex]
\mbox{Thread 2: } &\VAR x\GETS 0 \SEMI \bReg \GETS x \SEMI y \GETS 1  \SEMI 
\end{align*}
\begin{displaymathsmall}
\mbox{From    }
\begin{tikzcenter}[node distance=1em]
\event{wy0}{\DW{y}{0}}{}
\event{ry1}{\DR{y}{1}}{right=of wy0}
\event{wx1}{\DW{x}{1}}{right=of ry1}
\event{wx0}{\DW{x}{0}}{below=of wy0}
\event{rx1}{\DR{x}{1}}{right=of wx0}
\event{wy1}{\DW{y}{1}}{right=of rx1}
\rf{wx1}{rx1}
\rf{wy1}{ry1}
\wk{wx0}{rx1}
\wk{wy0}{ry1}
\end{tikzcenter}
\mbox{  infer  }
\begin{tikzcenter}node distance=1em]
\event{wy0}{\DW{y}{0}}{}
\event{ry1}{\DR{y}{0}}{right=of wy0}
\event{wx1}{\DW{x}{1}}{right=of ry1}
\event{wx0}{\DW{x}{0}}{below=of wy0}
\event{rx1}{\DR{x}{0}}{right=of wx0}
\event{wy1}{\DW{y}{1}}{right=of rx1}
\rf{wx0}{rx1}
\rf{wy0}{ry1}
\wk{rx1}{wx1}
\wk{ry1}{wy1}
\end{tikzcenter}
\end{displaymathsmall}
For the program:
\begin{align*}
\mbox{Thread 1: } &\VAR y\GETS 0 \SEMI   x \GETS 1  \SEMI \aReg \GETS y  \SEMI \\[-.5ex]
\mbox{Thread 2: } &\VAR x\GETS 0 \SEMI  y \GETS 1  \SEMI  \bReg \GETS x \SEMI
\end{align*}
\begin{displaymathsmall}
\mbox{From  }
\begin{tikzcenter}[node distance=1em]
\event{wy0}{\DW{y}{0}}{}
\event{wx1}{\DW{x}{1}}{right=of wy0}
\event{ry0}{\DR{y}{0}}{right=of wx1}
\event{wx0}{\DW{x}{0}}{below=of wy0}
\event{wy1}{\DW{y}{1}}{right=of wx0}
\event{rx0}{\DR{x}{0}}{right=of wy1}
\rf[bend right]{wx0}{rx0}
\rf[bend left]{wy0}{ry0}
\wk{rx0}{wx1}
\wk{ry0}{wy1}
\wk{wx0}{wx1}
\wk{wy0}{wy1}
\end{tikzcenter}
\ \mbox{ infer }
\begin{tikzcenter}[node distance=1em]
\event{wy0}{\DW{y}{0}}{}
\event{wx1}{\DW{x}{1}}{right=of wy0}
\event{ry0}{\DR{y}{1}}{right=of wx1}
\event{wx0}{\DW{x}{0}}{below=of wy0}
\event{wy1}{\DW{y}{1}}{right=of wx0}
\event{rx0}{\DR{x}{1}}{right=of wy1}
\rf{wx1}{rx0}
\rf{wy1}{ry0}
\wk{wx0}{wx1}
\wk{wy0}{wy1}
\end{tikzcenter}
\end{displaymathsmall}
For the program:
\begin{align*}
\mbox{Thread 1: } &\VAR x\GETS 1 \\[-.5ex]
\mbox{Thread 2: } &\VAR x\GETS 0 
\end{align*}
\begin{displaymathsmall}
\mbox{From  }
\begin{tikzcenter}[node distance=1em]
\event{wy0}{\DW{x}{1}}{}
\event{wx0}{\DW{x}{0}}{right=of wx0}
\wk{wy0}{wx0}
\end{tikzcenter}
\mbox{ infer   }
\begin{tikzcenter}[node distance=1em]
\event{wy0}{\DW{x}{1}}{}
\event{wx0}{\DW{x}{0}}{right=of wx0}
\wk{wx0}{wy0}
\end{tikzcenter}
\end{displaymathsmall}


\endinput

We say that $\aCmd$ has an \emph{SC race} if there is some pomset in $\semsc{\aCmd}$ that contains a data race.


In this section we show that if $\semsc{\aCmd}\subseteq\sem{\aCmd}$ has only
race-free executions, then each pomset $\aPS\in\sem{\aCmd}$ is ``equivalent''
to some $\aPS'\in\semsc{\aCmd}$, where $\aPS'$ may have more events, but
preserves labeling and has less order.

We say that $\aPS\suborder\aPS'$ if there is an injection
$\inj:\Event'\rightarrow\Event$ such that:
\begin{itemize}
\item $\labelingAct'(\aEv) = \labelingAct(\inj(\aEv))$
\item $\labelingForm'(\aEv)$ implies $\labelingForm(\inj(\aEv))$
\item $\labelingForm(\bEv)$ implies $\bigvee_{\aEv\in\inj^{-1}(\bEv)}(\labelingForm'(\aEv))$
\item $\aEv\le'\bEv$ implies $\inj(\aEv)\le\inj(\bEv)$
\item $\aEv\gtN'\bEv$ implies $\inj(\aEv)\gtN\inj(\bEv)$
\end{itemize}

\begin{theorem}
  If $\semsc{\aCmd}$ contains only race-free executions and
  $\aPS\in\sem{\aCmd}$ then there exists some $\aPS'\in\semsc{\aCmd}$ such
  that $\aPS\suborder\aPS'$.
\end{theorem}
% \begin{proof}
%   \begin{itemize}
%   \item
%     \begin{math}
%       \sem{\SKIP}
%       =
%       \{ \emptyset \} 
%     \end{math}
%   \item
%     \begin{math}
%       \sem{\FENCE_{\aF}\SEMI \aCmd}
%       =
%       (\DF{\aF}) \prefix \sem{\aCmd}
%     \end{math}
%   \item
%     \begin{math}
%       \sem{\aLoc\GETS\aExp\SEMI \aCmd}
%       =
%       \textstyle\bigcup_\aVal\; \bigl((\aExp=\aVal) \guard (\DW\aLoc\aVal) \prefix \sem{\aCmd}\bigr)[\aExp/\aLoc]
%     \end{math}
%   % \item
%   %   \begin{math}
%   %     \sem{\REL\aLoc\GETS\aExp\SEMI \aCmd}
%   %     =
%   %     \textstyle\bigcup_\aVal\; \bigl((\aExp=\aVal) \guard (\DWRel\aLoc\aVal) \prefix \sem{\aCmd}\bigr)[\aExp/\aLoc]
%   %   \end{math}
%   \item
%     \begin{math}
%       \sem{\aReg\GETS\aLoc\SEMI \aCmd}
%       =
%       \sem{\aCmd}[\aLoc/\aReg] \cup \textstyle\bigcup_\aVal\; (\DR\aLoc\aVal) \prefix \sem{\aCmd}[\aLoc/\aReg]
%     \end{math}
%   % \item
%   %   \begin{math}
%   %     \sem{\ACQ\aReg\GETS\aLoc\SEMI \aCmd}
%   %     =
%   %     \textstyle\bigcup_\aVal\; (\DRAcq\aLoc\aVal) \prefix \sem{\aCmd}[\aLoc/\aReg]
%   %   \end{math}
%   \item
%     \begin{math}
%       \sem{\IF{\aExp} \THEN \aCmd \ELSE \bCmd \FI}
%       =
%       \bigl((\aExp \neq 0) \guard \sem{\aCmd}\bigr) \parallel \bigl((\aExp=0) \guard \sem{\bCmd}\bigr)
%     \end{math}
%   \item
%     \begin{math}
%       \sem{\aCmd \PAR \bCmd}
%       =
%       \sem{\aCmd} \parallel \sem{\bCmd}
%     \end{math}
%   \item
%     \begin{math}
%       \sem{\VAR\aLoc\SEMI \aCmd}
%       =
%       \nu \aLoc \st \sem{\aCmd}
%     \end{math}
% \end{itemize}
  
% \end{proof}
% \end{theorem}


% To define compatibility, we extend the definitions of
% \textsection\ref{sec:semantics} to include an additional order: $\rird$.
% \begin{itemize}
% \item
%   ${\rird'} = {\rird}$
%   when $\aPSS'=\aPSS\aSub$
%   or $\aPSS'=\aForm\guard\aPSS$
% \item
%   ${\rird'} = {\rird}\restrict{\Event'}$
%   when $\aPSS'=\nu\aLoc\st\aPSS$
% \item
%   ${\rird'} = {\rird}^1\cup{\rird}^2$
%   when $\aPSS'=\aPSS^1\parallel\aPSS^2$
% \item
%   ${\rird'} = {\rird}\cup\{(\cEv,\aEv)\mid\labelingForm(\aEv) \;\text{is dependent on}\; \aLoc\}$
%   when $\aPSS'=\aAct\prefix\aPSS$, $\aAct$ writes $\aLoc$, and $\Event' = \Event \cup \{\cEv\}$
% \end{itemize}

% From $\rird$, we define ${\rrb}={\rird}^{-1};{\gtN}$.

% We want that if there is an execution:
% \begin{tikzdisplay}[node distance=1em]
%   \event{a}{\DW{\aLoc}{1}}{}
%   \event{b}{\DW{\bLoc}{1}}{below right=1em and 6em of a}
%   \event{c}{\DW{\aLoc}{2}}{above right=1em and 1em of b}
%   \wk{a}{c}
%   \ird{a}{b}
%   \rb{b}{c}
% \end{tikzdisplay}
% Then there is also
% \begin{tikzdisplay}[node distance=1em]
%   \event{a}{\DW{\aLoc}{1}}{}
%   \event{b}{\DW{\bLoc}{1}}{below right=1em and 6em of a}
%   \event{c}{\DW{\aLoc}{2}}{above right=1em and 1em of b}
%   \event{r}{\DR{\aLoc}{1}}{below right=.1em and 2em of a} 
%   \wk{a}{c}
%   \rf{a}{r}
%   \po{r}{b}
% \end{tikzdisplay}


% To see that we need $[\aExp/\aLoc]$ in the rule for write, rather than $[\aVal/\aLoc]$
% consider example:
% \begin{verbatim}
% r=y; if (r) {x=r} else {x=r}; s=x; if (r==s) {z=1}
% \end{verbatim}
% or simplified:
% \begin{verbatim}
% r=y;x=r;s=x; if(s==r){z=1}
% \end{verbatim}
% If you read 37 for $y$, then the predicate on \texttt{Wz1} before the
% read is either $r=r$ or $v=r$, where $v=37$, for example.  In one case you
% get a dependency and in the other you do not.


% \section{Efficient Implementation on ARMv8}
\label{sec:arm}

, with one exception: we believe the \armeight/\tso-compilation
  strategy is correct for \RMW---it is borrowed from
  \citet{DBLP:journals/pacmpl/PodkopaevLV19}---but have not proved it.

  In this section, we consider the fragment of our language without
restriction.  For simplicity, we allow release and acquire synchronization
but ban fences.  We assume that all memory locations are initialized to $0$
and parallel-composition occurs only at top level.  We take the set of memory
locations to be finite.  In other words, we assume that programs have the
form
\begin{displaymath}
  {\aLoc_1}\GETS{0}\SEMI
  \cdots\SEMI
  {\aLoc_m}\GETS{0}\SEMI
  (\aCmd^1 \PAR \cdots \PAR \aCmd^n)
\end{displaymath}
where $\aCmd^1$, \ldots, $\aCmd^n$ do not include composition, restriction or
fence operations.

Our language can be translated to ARM using \texttt{ldr} for relaxed read,
\texttt{ldar} for acquiring read, \texttt{str} for relaxed write, and
\texttt{stlr} for releasing write.  Relative to the ARM specification, we
have removed loops and read-modify-write (RMW) operations, in addition to
fences.\footnote{We leave out fences for simplicity.  Following
  \citet{DBLP:journals/pacmpl/PodkopaevLV19}, our $\FENCE$ instruction can be
  translated to \texttt{dmb.sy}, since it has release-acquire semantics.
  Acquire fences map to \texttt{dmb.ld}, and release fences to
  \texttt{dmb.sy} --- \texttt{dmb.st} does not provide order to prior
  reads.}

We show that any ARM-consistent execution graph for this sublanguage can be
considered an execution of our semantics.  
%
% Syntactically, we drop the superscript \textsf{rlx} on relaxed reads and
% writes; in addition, we use structured conditionals rather than the more
% general \textsf{goto}.  We refer to this sublanguage as $\muIMM$.
% (Because the source language lacks RMW operations, the ``is
% exclusive'' flag on every read will be \textsf{not-ex} and the RMW mode on
% every write will be \textsf{normal}.)
%
Due to space limitations, we do not include a full description ARM
consistency in the main text .
Here we provide a birds eye view of the details, drawing on the intuitions gleaned from~\citep{DBLP:journals/pacmpl/PulteFDFSS18}.  
Interested readers should see \textsection\ref{sec:arm:proof}
for further details.

An ARM execution graph $G$ defines many relations, including program order
($\rpox$), reads-from ($\rrfx$), coherence ($\rco$) and several dependency
orders.  From these are derived:
\begin{itemize}
\item ${\rpoloc}$, which is the subrelation of $\rpox$ that only relates
  actions on the same location,
\item ${\rob}$, which is required to be acyclic (\ref{external}), and
\item $\reco$, with the requirement that ${\rpoloc}\cup{\reco}$ be acyclic (\ref{sc-per-loc}).
\end{itemize}
% Let $G$ be an execution graph satisfying the ARM consistency 
%requirements.
Given an execution graph $G$, we say that $\aEv$ is an \emph{internal read} if $\aEv\in\fcodom({\rpox}\cap {\rrfx})$.

The ${\rob}$ order is an acyclic global order on events, agreed upon by all threads, reflecting the progress of time in an \armeight\ execution.  The cross thread component of the ordering is induced by the ordering on conflicting actions on the same location from different threads.    The intra thread component of the ordering is induced by barrier ordering and data ordering.  Notably, these dependencies  are determined syntactically.  In particular, $\rob$
may not necessarily include the intra thread component of $\rpoloc$ ordering.  

This motivates the translation of an \armeight\ execution into our setting.  In our setting, the progress of time is given by $\lt$.   We accommodate intra-thread reordering by internal read actions, thus excusing us from the obligation of placing them on the global $\lt$-timeline.  

Formally, from $G$ we construct a candidate pomset $\aPS$ as follows:
\begin{itemize}
\item $\Event= \textsf{E}$,
\item $\labelingAct(\aEv)=\tau \mathsf{lab}(e)$, if $\aEv$ is a relaxed
  internal read, 
\item $\labelingAct(\aEv)=\mathsf{lab}(e)$, if $\aEv$ is not a relaxed
  internal read,
\item $\labelingForm(\aEv)=\TRUE$,
% \item ${\le} = {\rob}^?$, where $?$ denotes reflexive closure, and
\item ${\gtN} = ({\rob} \cup {\reco})^*$, where $*$ denotes reflexive and transitive closure.
\end{itemize}

\begin{theorem}
  If $G$ is ARM consistent, the constructed candidate satisfies the
  requirements for a top-level memory model pomset.
\end{theorem}
Any $\lt$-ordering imposed in our model
is enforced by \armeight, since our notion of semantic dependency is more
permissive than \armeight 's syntactic dependency.  So, the heart of the proof is showing the acyclicity of $({\rob} \cup {\reco})^*$ for the events under consideration.  Since the cross thread portion of $\reco$ ($\rcoe,\rfre,\rrfe$) is included in $\rob$, this result is really about the influence of $\reco \cap \rpox$.  Our translation of \armeight 's \rrfi\  as silent internal actions removes them from order considerations.   Consequently, we only have to consider the suborder of ${\rob}$ derived without ever using  $\rrfi$ for the following key property demonstrated in \textsection\ref{sec:arm:proof}.
\begin{lemma}\label{extendob}
Let $\aEv, \bEv$ be distinct events and $\bEv'\ (\xob\cap \xpox) \ \bEv\ ((\xeco\cap \xpox) \setminus \xrfi) \  \aEv\ (\xob \cap \xpox)  \ \aEv'$.  Then $\bEv' \xob \aEv'$.
\end{lemma}



\begin{remark}[Proof for TSO]
  The proof for compilation into \tso\ is very similar.  In particular the facts listed above hold for \tso, where $\rob$ is replaced by (the
  transitive closure of) the propagation relation defined for \tso\ 
  \citep{alglave}.
\end{remark}


% \section{Conclusions}
\label{sec:outro}


% Local Variables:
% mode: latex
% TeX-master: "paper"
% End:
\begin{small}
\bibliography{bib}
\end{small}
\newpage
\appendix
%\section{Anton's Example}
A fix might be to allow elimination of RA/SC reads when fulfilled by the same
thread.  That is, extend the first clause of definition 4.1 to apply to RA/SC
as well as RLX.

We would need to change P6 to make it a filter operation in order to deal
with the acquires that were removed.

This does invalidate the claim in the last paragraph of section 8 about
elimination not creating new behaviors...  Would it break anything else?  ARM
does not seem to enforce the order... so why should we?

\section{Completed Pomsets}

This must not be allowed:
\begin{gather*}
  \begin{gathered}
    \PR{1}{r}\SEMI
    \PW{x}{r}\SEMI
    \PW{y}{2}\SEMI
    \PW{y}{r}
    \\
    \smash[b]{\hbox{\begin{tikzinline}[node distance=1.5em]
          \event{a}{\DW{x}{1}}{}
          \event{b}{\DW{y}{2}}{right=of a}
          \event{d}{\DSTOP}{right=of b}
          \sync{b}{d}
          \sync[out=-20,in=-160]{a}{d}
        \end{tikzinline}}}
  \end{gathered}
\end{gather*}
While computing this, we have the following.
\begin{gather*}
  \begin{gathered}
    \PW{y}{r}
    \\
    \smash[b]{\hbox{\begin{tikzinline}[node distance=1.5em]
          \event{b}{r{=}2\mid\DW{y}{2}}{}
          \event{d}{r{=}2\mid\DSTOP}{right=of b}
          \sync{b}{d}
        \end{tikzinline}}}
  \end{gathered}
\end{gather*}
The precondition on $\DSTOP$ is required by causal strengthening.

We can pun when prefixing, giving us
\begin{gather*}
  \begin{gathered}
    \PW{y}{2}\SEMI
    \PW{y}{r}
    \\
    \smash[b]{\hbox{\begin{tikzinline}[node distance=1.5em]
          \event{b}{\DW{y}{2}}{}
          \event{d}{r{=}2\mid\DSTOP}{right=of b}
          \sync{b}{d}
        \end{tikzinline}}}
  \end{gathered}
\end{gather*}
Prefixing the write to $x$:
\begin{gather*}
  \begin{gathered}
    \PW{x}{r}\SEMI
    \PW{y}{2}\SEMI
    \PW{y}{r}
    \\
    \smash[b]{\hbox{\begin{tikzinline}[node distance=1.5em]
          \event{a}{r{=}1\mid\DW{x}{1}}{}
          \event{b}{\DW{y}{2}}{right=of a}
          \event{d}{r{=}1 \land r{=}2\mid\DSTOP}{right=of b}
          \sync{b}{d}
          \sync[out=-20,in=-160]{a}{d}
        \end{tikzinline}}}
  \end{gathered}
\end{gather*}

There is a bug in the definition of parallel composition.  For the
conditional, it is correct as written.  For concurrency, termination events
should be combined with $\land$, not $\lor$.
\begin{gather*}
  \begin{gathered}
    \PW{x}{1}\SEMI
    \PW{x}{r}
    \\
    \smash[b]{\hbox{\begin{tikzinline}[node distance=1.5em]
          \event{b}{\DW{x}{1}}{}
          \event{d}{r{=}1\mid\DSTOP}{right=of b}
          \sync{b}{d}
        \end{tikzinline}}}
  \end{gathered}
  \qquad \qquad \qquad \qquad \qquad \qquad
  \begin{gathered}
    \PW{y}{2}\SEMI
    \PW{y}{r}
    \\
    \smash[b]{\hbox{\begin{tikzinline}[node distance=1.5em]
          \event{b}{\DW{y}{2}}{}
          \event{d}{r{=}2\mid\DSTOP}{right=of b}
          \sync{b}{d}
        \end{tikzinline}}}
  \end{gathered}
  \\
  \begin{gathered}
    \PW{x}{1}\SEMI
    \PW{x}{r}
    \PAR
    \PW{y}{2}\SEMI
    \PW{y}{r}
    \\
    \smash[b]{\hbox{\begin{tikzinline}[node distance=1.5em]
          \event{b}{\DW{x}{1}}{}
          \event{c}{\DW{y}{2}}{right=of b}
          \event{d}{r{=}1\lor r{=}2\mid\DSTOP}{right=of c}
          \sync{c}{d}
          \sync[out=-20,in=-160]{b}{d}
        \end{tikzinline}}}
  \end{gathered}
\end{gather*}

\section{Non-MCA Ideas}
Status of IRIW is unclear in our model, since we allow everything allowed by
power...
\begin{gather*}
  \begin{gathered}
    % \PW{x}{0}\SEMI
    \PW{x}{1}
    \PAR
    % \PW{y}{0}\SEMI
    \PW{y}{1}
    \PAR
    \PR[\mRA]{x}{r}\SEMI \PR{y}{s}
    \PAR
    \PR[\mRA]{y}{s} \SEMI \PR{x}{r}
    \\
    %\smash[b]{
      \hbox{\begin{tikzinline}[node distance=1.5em]
          % \event{wx0}{\DW{x}{0}}{}
          % \event{wx1}{\DW{x}{1}}{right=of wx0}
          % \event{wy0}{\DW{y}{0}}{below=4ex of wx0}
          % \event{wy1}{\DW{y}{1}}{right=of wy0}
          \event{wx1}{\DW{x}{1}}{}
          \event{wy1}{\DW{y}{1}}{below=4ex of wx1}
          \event{ry1}{\DR[\mRA]{y}{1}}{right=2.5em of wy1}
          \event{rx0}{\DR{x}{0}}{right=of ry1}
          \event{rx1}{\DR[\mRA]{x}{1}}{right=2.5 em of wx1}
          \event{ry0}{\DR{y}{0}}{right=of rx1}
          % \wk{wx0}{wx1}
          % \wk{wy0}{wy1}
          % \rf[bend left]{wy0}{ry0}
          % \rf[bend right]{wx0}{rx0}
          \sync{rx1}{ry0}
          \sync{ry1}{rx0}
          \rf{wx1}{rx1}
          \rf{wy1}{ry1}
          \wk{rx0}{wx1}
          \wk{ry0}{wy1}
        \end{tikzinline}}
    %}
  \end{gathered}
\end{gather*}




\section{Additional RMW Examples}

It is not possible for two \RMW{}s to see the same write.
\begin{gather*}
  \begin{gathered}
    \PW{x}{0} \SEMI \bigl(\FADD^{\mRLX,\mRLX}(x,1) \PAR \FADD^{\mRLX,\mRLX}(x,1)\bigr)
    \\
    \hbox{\begin{tikzinline}[node distance=2em]
        \event{a0}{\DW{x}{0}}{}
        \event{a1}{\DR{x}{0}}{right=3em of a0}
        \event{a2}{\DW{x}{1}}{right=of a1}
        \event{b1}{\DR{x}{0}}{right=3em of a2}
        \event{b2}{\DW{x}{1}}{right=of b1}
        \rmw{a1}{a2}
        \rf{a0}{a1}
        \rf[out=-15,in=-165]{a0}{b1}
        \wk[out=-15,in=-165]{a1}{b2}
        \wk{b1}{a2}
        \graywk[bend left]{a2}{b1}
        \rmw{b1}{b2}
      \end{tikzinline}}
  \end{gathered}
  \taglabel{rmw0}
\end{gather*}
The gray arrow is required the \RMW{} atomicity axioms.

\citet{DBLP:conf/pldi/LeeCPCHLV20} introduce \PS{2.0} to refine the treatment of
\RMW{}s in the promising semantics (\PS{}).  Their examples have the expected
results here, with far less work.  First they recall that \PS{} requires
quantification over multiple futures in order to disallow executions such as
\ref{CDRF}:
\begin{gather*}
  \taglabel{CDRF}
    \begin{gathered}
      r\GETS \FADD^{\mRA,\mRA}(x,1)\SEMI \IF{r{=}0}\THEN \PW{y}{1} \FI
      \PAR
      r\GETS \FADD^{\mRA,\mRA}(x,1)\SEMI \IF{r{=}0}\THEN \IF{y}\THEN \PW{x}{0} \FI \FI
      \\
      \hbox{\begin{tikzinline}[node distance=2em]
          \event{a1}{\DR[\mRA]{x}{0}}{}
          \event{a1b}{\DW[\mRA]{x}{1}}{below=1em of a1}
          \event{a2}{\DW{y}{1}}{right=of a1}
          \event{b0}{\DR[\mRA]{x}{0}}{right=3em of a2}
          \event{b0b}{\DW[\mRA]{x}{1}}{below=1em of b0}
          \event{b1}{\DR{y}{1}}{right=of b0}
          \event{b2}{\DW{x}{0}}{right=of b1}
          \rmw{a1}{a1b}
          \rmw{b0}{b0b}
          \rf[out=-13,in=-163]{a2}{b1}
          \po{a1}{a2}
          \sync{b0}{b1}
          \po{b1}{b2}
          \rf[out=-165,in=-12]{b2}{a1}
        \end{tikzinline}}
    \end{gathered}
  \end{gather*}
This execution is clearly impossible, due to the cycle above.  In this
diagram, we have not drawn order adjacent to the writes of the \RMW{}s, since
this is not necessary to produce the cycle.
If \ref{CDRF} is allowed then \drfra{} fails.


  
\PS{} does not support global value range analysis, as modeled by \ref{GA+E} below.  Our
semantics permits \ref{GA+E}:
\begin{gather*}
  \taglabel{GA+E}
    \begin{gathered}
      \PW{x}{0} \SEMI
      \bigl(
        r\GETS \CAS^{\mRLX,\mRLX}(x,0,1)\SEMI \IF{r{<}10}\THEN \PW{y}{1} \FI
        \PAR
        \PW{x}{42}\SEMI \PW{x}{y}
      \bigr)
      \\
      \hbox{\begin{tikzinline}[node distance=2em]
          \event{a0}{\DW{x}{0}}{}
          \event{a1}{\DR{x}{1}}{right=3em of a0}
          \event{a2}{0{<}10\mid\DW{y}{1}}{right=of a1}
          \event{b0}{\DW{x}{42}}{right=3em of a2}
          \event{b1}{\DR{y}{1}}{right=of b0}
          \event{b2}{\DW{x}{1}}{right=of b1}
          %\rmw{a1}{a2}
          \rf[out=-15,in=-160]{a2}{b1}
          \po{b1}{b2}
          \rf[out=-165,in=-15]{b2}{a1}
          \wk[out=10,in=170]{a0}{b0}
          \wk[out=15,in=165]{b0}{b2}
        \end{tikzinline}}
    \end{gathered}
\end{gather*}
\PS{} also does not support register promotion, as modeled by \ref{RP} below.    Our
semantics permits \ref{RP}:
\begin{gather*}
  \taglabel{RP}
    \begin{gathered}
      \PR{x}{r}\SEMI
      s\GETS \FADD^{\mRLX,\mRLX}(z,r)\SEMI \PW{y}{s{+}1}
      \PAR
      \PW{x}{y}
      \\
      \hbox{\begin{tikzinline}[node distance=2em]
          \event{a0}{\DR{x}{1}}{}
          \event{a1}{\DR{z}{0}}{right=of a0}
          \event{a1b}{\DW{z}{1}}{right=of a1}
          \event{a2}{\DW{y}{1}}{right=of a1b}
          \event{b0}{\DR{y}{1}}{right=3em of a2}
          \event{b1}{\DW{x}{1}}{right=of b0}
          \rmw{a1}{a1b}
          \po[out=20,in=160]{a0}{a1b}
          \po[out=20,in=160]{a1}{a2}
          \po{b0}{b1}
          \rf{a2}{b0}
          \rf[out=-165,in=-15]{b1}{a0}
        \end{tikzinline}}
    \end{gathered}
\end{gather*}


\section{Examples from LDRF for PS}
\ref{CDRF} shows that our semantics is not too permissive for $\mRA$-\RMW{}s.
But what about $\mRLX$-\RMW{}s.  The following execution is allowed by \armeight,
and \PS{2.0}, but disallowed by \PS{2.1}.
\begin{gather*}
  \taglabel{RMW-W}
  \begin{gathered}
    r\GETS \FADD^{\mRLX,\mRLX}(x,1)\SEMI \PW{y}{1}
    \PAR
    \PR{y}{r}\SEMI s\GETS \FADD^{\mRLX,\mRLX}(x,r)
    \\
    \hbox{\begin{tikzinline}[node distance=2em]
        \event{a1}{\DR{x}{1}}{}
        \event{a1b}{\DW{x}{2}}{below=1em of a1}
        \event{a2}{\DW{y}{1}}{right=of a1}
        \event{b1}{\DR{y}{1}}{right=3em of a2}
        \event{b2}{\DR{x}{0}}{right=of b1}
        \event{b2b}{\DW{x}{1}}{below=1em of b2}
        \rmw{a1}{a1b}
        \rmw{b2}{b2b}
        \rf{a2}{b1}
        \po{b1}{b2b}
        \rf[out=-175,in=-20]{b2b}{a1}
      \end{tikzinline}}
  \end{gathered}
\end{gather*}

If this $\ldrfra{z}$?
\begin{gather*}
  \taglabel{Naive-LDRF-RA-Fail}
  \begin{gathered}
    \IF{y}\THEN \PW{x}{z} \ELSE \PW{x}{1} \FI
    \PAR
    \PR{x}{r}\SEMI \PW{z}{1}\SEMI \PW{y}{r}
    \\
    \hbox{\begin{tikzinline}[node distance=2em]
        \event{a1}{\DR{y}{1}}{}
        \event{a2}{\DR{z}{1}}{right=of a1}
        \event{a3}{\DW{x}{1}}{right=of a2}
        \event{b1}{\DR{x}{1}}{right=3em of a3}
        \event{b2}{\DW{z}{1}}{right=of b1}
        \event{b3}{\DW{y}{1}}{right=of b2}
        \po{a2}{a3}
        \po[in=165,out=15]{b1}{b3}
        \rf[out=-165,in=-15]{b2}{a2}
        \rf[out=-165,in=-15]{b3}{a1}
        \rf{a3}{b1}
      \end{tikzinline}}
  \end{gathered}
\intertext{Interpreting $\{z\}$ as $\mRA$:}
    \\
  \begin{gathered}
    \hbox{\begin{tikzinline}[node distance=2em]
        \event{a1}{\DR{y}{1}}{}
        \event{a2}{\DR{z}{1}}{right=of a1}
        \event{a3}{\DW{x}{1}}{right=of a2}
        \event{b1}{\DR{x}{1}}{right=3em of a3}
        \event{b2}{\DW{z}{1}}{right=of b1}
        \event{b3}{\DW{y}{1}}{right=of b2}
        \po{a2}{a3}
        \po[in=165,out=15]{b1}{b3}
        \rf[out=-165,in=-15]{b2}{a2}
        \rf[out=-165,in=-15]{b3}{a1}
        \rf{a3}{b1}
        \sync{a1}{a2}
        \sync{b2}{b3}
      \end{tikzinline}}
  \end{gathered}
\end{gather*}

Our semantics already disallows \ref{LDRF-Fail-PS}, which is similar to \ref{OOTA4}.
\begin{gather*}  
  \taglabel{LDRF-Fail-PS}
  \begin{gathered}
  \IF{x}\THEN
    \FADD(w,1)\SEMI
    \PW{y}{1}\SEMI
    \PW{z}{1}
  \FI
  \PAR
  \IF{\BANG z}\THEN
    \PW{x}{1}
  \ELSE
    \IF{\BANG \FADD(w,1)}\THEN
      \PW{x}{\PR{y}{}}
    \FI
  \FI
    \\
    \hbox{\begin{tikzinline}[node distance=2em]
        \event{a1}{\DR{x}{1}}{}
        \event{a2}{\DR{w}{1}}{right=of a1}
        \event{a3}{\DW{w}{2}}{right=of a2}
        \event{a4}{\DW{y}{1}}{right=of a3}
        \event{a5}{\DW{z}{1}}{right=of a4}
        \event{b1}{\DR{z}{1}}{right=5em of a5}
        \event{b2}{\DR{w}{0}}{right=of b1}
        \event{b3}{\DW{w}{1}}{right=of b2}
        \event{b4}{\DR{y}{1}}{right=of b3}
        \event{b5}{\DW{x}{1}}{right=of b4}
        \rmw{a2}{a3}
        \po[out=15,in=165]{a1}{a3}
        \po[out=15,in=165]{a1}{a4}
        \po[out=15,in=165]{a1}{a5}        
        \rmw{b2}{b3}
        \po{b4}{b5}
        \po[out=15,in=165]{b2}{b5}        
        \po[out=15,in=165]{b1}{b3}
        \rf{a5}{b1}
        \rf[out=15,in=165]{a4}{b4}
        \rf[out=-165,in=-15]{b3}{a2}
        \rf[out=-165,in=-15]{b5}{a1}
      \end{tikzinline}}
  \end{gathered}
\end{gather*}
\begin{comment}
  \centering  
\begin{verbatim}
a := X                  b := Z                 
if a = 1 then           if b = 0 then          
  _ := FADD(W , 1)        X := 1               
  Y := 1                else                   
  Z := 1                  c := FADD(W, 1) /0   
                          if c = 0 then        
                            d := Y             
                            X := d             
\end{verbatim}
\includegraphics[width=\textwidth]{LDRF-Fail-PS}
\caption{LDRF-Fail-PS}
\end{comment}


If \RMW{}s simply use the same semantics as read and write, then we allow
\ref{LDRF-PF-Fail}, which is used to show failure of $\ldrfsc{}$.
\begin{gather*}  
  \taglabel{LDRF-PF-Fail}
  \begin{gathered}
    \PW{y}{0}\SEMI
    \IF{y}\THEN
      \IF{\BANG\CAS(x,0,1)}\THEN
        \IF{z}\THEN
          \PW{x}{2}
    \FI\FI\FI
    \PAR
    \PW{y}{1}\SEMI
    \IF{1{\neq}\CAS(x,0,3)}\THEN
      \PW{z}{1}
    \FI
    \\
    \hbox{\begin{tikzinline}[node distance=2em]
        \event{a1}{\DW{y}{0}}{}
        \event{a2}{\DR{y}{1}}{right=of a1}
        \event{a3}{\DR{x}{0}}{right=of a2}
        \event{a4}{\DW{x}{1}}{right=of a3}
        \event{a5}{\DR{z}{1}}{right=of a4}
        \event{a6}{\DW{x}{2}}{right=of a5}
        \event{b1}{\DW{y}{1}}{right=5em of a6}
        \event{b2}{\DR{x}{2}}{right=of b1}
        \event{b3}{\DW{z}{1}}{right=of b2}
        \wk{a1}{a2}
        \rmw{a3}{a4}
        \po[out=15,in=165]{a2}{a6}
        %\po[out=15,in=165]{a3}{a6}
        \po{a5}{a6}
        \wk[out=-20,in=-160]{a4}{a6}
        %\po{b2}{b3}
        \rf[out=15,in=165]{a6}{b2}
        \rf[out=-165,in=-15]{b3}{a5}
        \rf[out=-165,in=-15]{b1}{a2}
      \end{tikzinline}}
  \end{gathered}
\end{gather*}
To disallow this, we need to retain the dependency
\begin{math}
  \DRP{x}{2}\xpo \DWP{z}{1}.
\end{math}
For this, we need to avoid the substitution for $x$.  This is clearer in the
LICS semantics.  You just use L6 rather than L5 for the independent case on
\RMW{}s.

\begin{comment}
  \centering  
\begin{verbatim}
Y := 0                   Y := 1                 
a := Y                   d := CAS(X,0,1) /37?   
if a != 0 then           if d != 42 then        
  b := CAS(X,0,42)         L := 1               
  if b = 0 then
    c := L
    if c = 1 then
      Xsrlx := 37
\end{verbatim}
\includegraphics[width=.8\textwidth]{LDRF-PF-Fail.png}
\caption{LDRF-PF-Fail}
\end{comment}


\section{Proof of Efficient Implementation on ARM8}
\label{sec:arm:proof}

In this section, we develop the proof of correctness of compilation to \armeight.  

Our language can be translated to \armeight{} following
\citet{DBLP:journals/pacmpl/PodkopaevLV19}, thus using \texttt{ldr} for
relaxed read, \texttt{ldar} for $\mRA$/$\mSC$ reads are acquires,
\texttt{str} for relaxed write, and \texttt{stlr} for $\mRA$/$\mSC$
writes.  $\FENCE$ instruction can be translated to \texttt{dmb.sy}, since it
has release-acquire semantics.  Acquire fences map to \texttt{dmb.ld}, and
release fences to \texttt{dmb.sy} --- \texttt{dmb.st} does not provide order
to prior reads.

Relative to the \armeight{} specification, we have removed loops, and
read-modify-write (RMW) operations.  The implementation of RMW operations
follows \citet{DBLP:journals/pacmpl/PodkopaevLV19}.  We only omit it because
a systematic treatment includes a loop to account for the possible failure of
the RMW operation.

Given a relation $R$, $R^?$ denotes reflexive closure, $R^+$ denotes
transitive closure and $R^*$ denotes reflexive and transitive closure.  Given
relations $R$ and $S$, $R;S$ denotes composition.


The \armeight{} model is described using the following relations.
\begin{itemize}
\item $\IDR$, $\IDW$, $\IDAcq$, $\IDRel$: identity on reads, writes, acquires
  and releases.
% \item $\IDR$ identity on reads
% \item $\IDW$: identity on writes
% \item $\IDAcq$: identity on acquires
% \item $\IDRel$: identity on releases
\item $\IDLoc$: relates any two events that touch the same location.
\item $\rpox$: program order.
\item $\rdata$, $\rctrl$, $\raddr$: data, control and address dependencies.
\item $\rrfx$: reads-from. $\rrfx^{-1}$ relates each read to a matching write
  on the same location.
\item $\rco$: coherence, which is a total order on the writes to a single
  location.
\item ${\rfr}\eqdef{\rco};\rrfx^{-1}$: from-read, which relates reads to
  subsequent writes.
\end{itemize}
For any relation, the cross-thread subrelation is denoted by appending $e$;
the intra-thread subrelation is denoted by appending $i$.  For example,
${\rrfe}\eqdef{\rrfx}\setminus{\rpox}$ and ${\rrfi}\eqdef{\rrfx}\cap{\rpox}$.
The subrelation restriction attention to actions on the same location is
given by appending $\mathsf{loc}$.  For example, ${\rpoloc}\eqdef{\rpox}\cap{\IDLoc}$.

The \armeight{} model defines the following relations.
In our presentation, we have elided rules concerning fences and RMW operations.
\begin{align*}
  \tag{Extended coherence}
  {\reco} &\eqdef {\rrf} \cup {\rfr} \cup {\rco}
  \\
  \tag{Observed externally}
  {\robs} &\eqdef \smash{
    {\rrfe} \cup {\rfre} \cup {\rcoe}
  }
  \\
  \tag{Dependency order}
  {\rdob} &\eqdef \smash{({\raddr}\cup{\rdata}); {\rrfi}^?}
  \\[-1ex]
  &\cup \smash{({\rctrl}\cup{\rdata}); {\IDW}; {\rcoi}^?}
  \\[-1ex]
  &\cup \smash{{\raddr}; {\rpox}; {\IDW}}
  \\
  \tag{Barrier order}
  {\rbob} &\eqdef\smash{
    {\IDAcq}; {\rpox}
    \cup {\rpox};{\IDRel}; {\rcoi}^?
  }
  \\
  \tag{Acyclic order}
  {\rob} &\eqdef\smash{
    ({\robs} \cup {\rdob} \cup {\rbob})^+
  }
\end{align*}
\begin{definition}
  An RMW-free and fence-free execution is \emph{\armeight{}-consistent} if
  \begin{align*}&
    \tag{\textsc{$\rrfx$-completeness}}\label{rf-comp}
    \fcodom(\rrfx)=\fdom(\rreads)
    \\[-1ex]&
    \tag{\textsc{$\rco$-totality}}\label{co-tot}
    \text{For every location $\aLoc$, $\rco$ totally orders the writes of $\aLoc$}  
    \\[-1ex]&
    \tag{\textsc{sc-per-loc}}\label{sc-per-loc}
    {\rpoloc} \cup {\rrfx} \cup {\rfr} \cup {\rco}\;\text{is acyclic}
    \\[-1ex]&
    \tag{\textsc{external}}\label{external}
    {\rob}\;\text{is acyclic}
  \end{align*}
\end{definition}

% Use these to refer to the rules in text:
%\ref{rf-comp} 
%\ref{co-tot}
%\ref{sc-per-loc}
%\ref{external}


Given an execution graph $G$, we say that $\aEv$ is an \emph{internal read} if
$\aEv\in\fcodom(\mathsf{po}\cap \mathsf{rf})$.    

% We are going to translate internal reads of execution graphs into 
% internal reads of the semantics.  

From $G$ we construct a candidate pomset $\aPS$ as follows:
\begin{itemize}
\item $\Event= \textsf{E}$,
% \item $\labelingAct(\aEv)=\tau \mathsf{lab}(e)$, if $\aEv$ is a % relaxed internal read, 
\item $\labelingAct(\aEv)=\mathsf{lab}(e)$, if $\aEv$ is not a relaxed
  internal read,
\item $\labelingForm(\aEv)=\TRUE$,
\item ${\gtN}=({\reco}\cup{\robi})^*$.
\end{itemize}
The relation ${\robi}$ is defined from ${\rob}$ by restricting the order into and out of an read that is in
the codomain of the $\rrfi$ relation.  More formally, let $\bEv\xdobi\aEv$ when $\bEv\xdob\aEv$ and
$\bEv\notin\fcodom(\rrfi), \aEv \notin\fcodom(\rrfi)$.  

Let $\robi$ be defined as for $\rob$, simply replacing $\rdob$ with $\rdobi$.

% considering the definition of $\rob$ without $\rrfi$, ie.:
% \begin{align*}
 %  \tag{Dependency order}
%   {\rdob} &\eqdef\smash{
 %    ({\raddr}\cup{\rdata});
 %    \cup ({\rctrl}\cup{\rdata}); {\IDW}; {\rcoi}^?
 %    \cup {\raddr}; {\rpox}; {\IDW}
 %  }
 %  \\
 %  \tag{Barrier order}
 %  {\rbob} &\eqdef\smash{
 %    {\IDAcq}; {\rpox}
 %    \cup {\rpox};{\IDRel}; {\rcoi}^?
 %  }
 %  \\
 %  \tag{Acyclic order}
 % %  {\rob} &\eqdef\smash{
 %   ({\robs} \cup {\rdob} \cup {\rbob})^+
  % }
% \end{align*}


We show that $\aPS$ is a top-level pomset, reasoning as follows.
% We establish the criteria for a top-level memory-model pomset:
\begin{itemize}
%\item ${\le}$ is a partial order.  This holds since $G.{\rar}$ is 
%acyclic.
% \item If $\bEv \le \aEv$ then $\bEv \gtN \aEv$.  By construction.
% \item If $\bEv \le \aEv$ and $\aEv \gtN \bEv$ then $\bEv = \aEv$.  Proved below.
% \item If $\cEv \le \bEv \gtN \aEv$ or $\cEv \gtN \bEv \le \aEv$ then
%   $\cEv \gtN \aEv$. By construction.
% \end{itemize}

% Next, we establish the criteria for a 3-valued pomset with preconditions (Definition~\ref{def:3pre}).
% \begin{itemize}
\item $\labelingForm(\aEv)$ implies $\labelingForm(\bEv)$ whenever
  $\bEv\le\aEv$.   Trivial, since every formula is $\TRUE$.
% \item $\aPS$ is $\aLoc$-coherent; that is, when restricted to events that
%   read or write $\aLoc$, $\gtN$ forms a partial order.
% \end{itemize}

% Finally, we establish the criteria for a top-level pomset
% (Definition~\ref{def:x-closed}).
% \begin{itemize}
%\item $\aEv$ is location independent. Trivial, since every formula 
% is $\TRUE$.
\item If $\aEv$ reads $\aVal$ from $\aLoc$, then there is some $\bEv$ such that
  \begin{itemize}
  \item $\bEv \lt \aEv$,  
  \item $\bEv$ writes $\aVal$ to $\aLoc$, and
  \item if $\cEv$ writes to $\aLoc$
    then either $\cEv \gtN \bEv$ or $\aEv \gtN \cEv$.
  \end{itemize}    
\end{itemize}

\subsection{Proof that  $({\robi} \cup {\reco})^*$ is irreflexive. }


\begin{lemma}\label{extendob}
Let $\aEv, \bEv$ be distinct events and $\bEv'\ ({\xob}\cap {\xpox}) \ \bEv\allowbreak\ (({\xeco}\cap {\xpox}) \setminus {\xrfi}) \  \aEv\ ({\xob} \cap {\xpox})  \ \aEv'$.  Then $\bEv' \xob \aEv'$.

\begin{proof}
If $\bEv'$ is an acquire,  or $\aEv$ is an release, or $\aEv'$ is a release, result is immediate.

We next consider the case where $\aEv$ is a read.  In this case,  $\bEv$ is a write.  Since $\bEv\ ((\xeco\cap \xpox) \setminus \xrfi) \  \aEv$, there is a write $\bEv_1$ such that $ \bEv \xcoe\ \bEv_1 \ \xrfe\ \aEv' $.  So, $\bEv \xob \aEv$ and result follows in this case. 


So, it suffices to prove the following assuming that $\bEv'$ is not an acquire and $\aEv'$ is not a release and $\aEv$ is not a release or a read and $\aEv, \bEv$ are distinct.
\begin{itemize}
\item If $\bEv'\ (\xob\cap \xpox)  \ \bEv(\xeco\cap\xpox)\aEv$ then $\bEv'\xob\aEv$.
\item If $\bEv\ (\xeco\cap\xpox) \ \aEv(\xob\cap\xpox)\aEv'$ then $\bEv\xob\aEv'$.
\end{itemize}


We first prove that if $\bEv'\ (\xob \cap \xpox) \ \bEv\ (\xeco \cap \xpox) \ \aEv$ then $\bEv'\xob\aEv$.   Proof proceeds by cases on the witness for $\bEv'\ (\xob\cap \xpox) \ \bEv$. 
\begin{itemize}
\item  If $\bEv' \xbob  \bEv$, then: 
\[ \bEv'\ (\smash{
    {\IDAcq}; {\rpox}
    \cup {\rpox};{\IDRel}; {\rcoi}^?) \ 
  }
\bEv
\]
Since $\bEv'$ is not an acquire, $\bEv' ({\rpox};{\IDRel}; {\rcoi}^?) \bEv$, so $\bEv$ is a write.  Since $\aEv$ is not a read,  $\bEv \xcoi\ \aEv$. Thus, result follows.

\item If $\bEv' \xdob  \bEv$, then: 
\[ \bEv'\ 
\smash{
    ( ({\rctrl}\cup{\rdata}); {\IDW}; {\rcoi}^?
    \cup {\raddr}; {\rpox}; {\IDW}
  } \
\bEv
\]
So, $\bEv$ is a write.  Since $\aEv$ is also a write, we deduce that 
\[ \bEv'\ 
\smash{
    ( ({\rctrl}\cup{\rdata}); {\IDW}; {\rcoi}^?
    \cup {\raddr}; {\rpox}; {\IDW}
  } \
\aEv
\]
\end{itemize}


We next prove  that if $\bEv\ (\xeco \cap \xpox) \ \aEv\ (\xob\cap \xpox) \ \aEv'$ then $\bEv\xob\aEv'$, under the assumptions that  $\aEv'$ is not a release and $\aEv$ is not a release or a read and $\aEv, \bEv$ are distinct.


 Proof proceeds by cases on the witness for $\aEv (\xob\cap \xpox) \aEv'$.  

\begin{itemize}
\item  If $\aEv \xbob  \aEv'$, then: 
\[ \aEv\ (\smash{
    {\IDAcq}; {\rpox}
    \cup {\rpox};{\IDRel}; {\rcoi}^?) \ 
  }
\aEv'
\]
Since $\aEv$ is not a read, $\aEv ({\rpox};{\IDRel}; {\rcoi}^?) \aEv'$.  Result follows since  $\bEv \xpox\ \aEv$.


\item If $\aEv \xdob  \aEv'$, then $\aEv$ is a read.  \popQED
\end{itemize}
\end{proof}
\end{lemma}


We now turn to proving that $({\robi} \cup {\reco})^*$ is acyclic.  

It suffices to consider possible cycles in $({\robi} \cup {\reco})^*$ that do not involve the read events fulfilled by $\rrfi$.    This is because the read events that are fulfilled by $\rrfi$ have the following properties:
\begin{itemize}
\item Any in-edge into the event factors through an in-edge from an acquire or the fulfilling write to the read
\item Any out-edge from the event factors through an out-edge to a release
\end{itemize}
Thus, if there is a cycle involving a read event fulfilled by $\rrfi$, there is also a cycle without such a read event. 


{\em In the rest of this section, we only consider events that are not read events fulfilled by $\rrfi$.}

\begin{lemma}\label{obeco1}
If $\bEv\xobi\aEv$ then $\lnot(\aEv\xeco\bEv)$.

\begin{proof}

Proof by contradiction.  Let 
\[  \aEv \xeco \bEv \xobi \aEv  \]

At least one of $\aEv,\bEv$ is a write.  So, if $\aEv \not\xpox \bEv$, then $\aEv \xob \bEv$, and we have a cycle in $\xob$.  

So, we conclude that $\aEv \xpox \bEv$

Since $\bEv \xobi \aEv$, there exists $\bEv \not\xpox \cEv$, $\bEv (\xeco \cap \xobi)  \cEv$, $\cEv \xobi \aEv$.

We reason by cases.
\begin{itemize}
\item If $\cEv$ is a write or $\aEv$ is a write,  $\cEv \xeco \aEv$ and we have an $\xeco$ cycle.

\item Otherwise, $\aEv$ is a read, $\bEv$ is a write. 

Let $\bEv \xobi \cEv_0 \xobi \cEv_1 \ldots \cEv_n \xobi \aEv$ be the witness.   If $\cEv_n \xpox \aEv$, then $\cEv_n \xobi \bEv$ and we have an $\xob$ cycle.

So, $\cEv_n \not\xpox \aEv$.  Thus, $\cEv_n$ is the write fulfilling
$\aEv$. So, we deduce: $\cEv_n \xeco \aEv$ and $\bEv \xeco \cEv_n$ yielding an $\xeco$ cycle. \popQED
\end{itemize}
\end{proof}
\end{lemma}

\begin{lemma}\label{obeco2}
$({\robi} \cup {\reco})^*$ is irreflexive.
\end{lemma}
\begin{proof}
The simple case that $\robi; \reco$ is irreflexive is proved above.  The full proof is by contradiction.  

Let $n \geq 1$ be  such that:
\begin{align*} 
&\aEv^0_0 \xobi \aEv^0_1 \xeco \bEv^0_0 \xobi \bEv^0_1  \\
(\xeco \cap \xobi) &  \   \aEv^1_0 \xobi \aEv^1_1 \xeco \bEv^1_0 \xobi \bEv^1_1 \\
(\xeco \cap \xobi) & \ \ldots \\
& \ldots \bEv^n_1 \\
 (\xeco \cap \xobi) & \  \aEv^0_0
\end{align*}
where  for all $i$, we have:
\[ \aEv^i_0 \xpox \aEv^i_1 (\xeco \cap \xpox) \bEv^i_0 \xpox \bEv^i_1\] and 
\[ \neg (\bEv^i_1 \xpox \aEv^{(i+1) \mod n}_0 ) \]
with the proviso that we have chosen the cycle with the minimum number of events.  

For any $i$, if $\aEv^i_0 \not= \aEv^i_1$ or $\bEv^i_0 \xpox \bEv^i_1$, via lemma~\ref{extendob}, we deduce that $\aEv^i_0  \xob \bEv^i_1$, contradicting minimality of number of events in cycle.  

So, we can assume that $n \geq 1$ is such that:
\begin{align*} 
&\ \aEv^0 \xeco  \bEv^0 \\
(\xeco \cap \xob) &  \   \aEv^1  \xeco \bEv^1 \\
(\xeco \cap \xob) & \ \ldots \\
& \ldots \bEv^n \\
 (\xeco \cap \xob) & \ \aEv^0
\end{align*}
which is a contradiction since it is a cycle in $\xeco$. 
% and since at least one of $\aEv^i ,\bEv^i$ is a write for all $i$. 
\end{proof}




\end{document}

% Local Variables:
% mode: latex
% TeX-master: t
% End:
