% {\bf MOVE TO APPENDIX??.  

% Given $\aCmd$, define 
% $\linV{\aCmd}$ as follows.
% \begin{definition}
% With each shared variable $\aLoc$, we associate a thread local variable $\aLocLoc$ and two boolean variables $\aRead,\aChanged$ that will be local to the program fragment.  

% \begin{align*}
% \linV{\aCmd}= & \VAR\vec{\aLocLoc} \SEMI \VAR \vec{\aRead} \SEMI \VAR \vec{\aChanged} \SEMI  \\
% & \vec{\aRead} =\vec{0} \SEMI \vec{\aChanged} = \vec{0} \SEMI \\
% & \aCmd' \SEMI   \\
% & \overline{\IF \aChanged\  \aLoc= \aLocLoc}
% \end{align*}
% where:
% $\aCmd'$ is derived from $\aCmd$ by replacing:
% \begin{itemize}
% \item  each read $\aReg = \aLoc$ by $\aRead = 1 \SEMI \IF {(\aChanged \lor\ \aRead)} \THEN\  \aReg= \aLocLoc \ELSE\  \aReg= \aLoc \SEMI \FI $,
% \item each write $\aLoc = \aExp$ by $\aChanged =1 \SEMI \aLocLoc = \aExp$
% \end{itemize}
% \end{definition}
% From the soundness and completeness of Hoare logic for sequential operational semantics (eg. see formalization in~\citet{gordonHoare}), we deduce for any $\aForm, \bForm$: 
% \[ \hoare{\aForm}{\aCmd}{\bForm}  \Longleftrightarrow\  \hoare{\aForm}{\linV{\aCmd}}{\bForm} \]
% }

\newcommand{\aMem}{M}
\newcommand{\bMem}{N}
\newcommand{\upd}[3]{#1[#2 \mapsto #3]}
\newcommand{\aStore}{S}
\newcommand{\bStore}{T}
\newcommand{\config}[3]{((#1,#2), #3)}
\newcommand{\aConfig}{\config{\aMem}{\aStore}{\aCmd}}
\newcommand{\bConfig}{\config{\bMem}{\bStore}{\bCmd}}
\newcommand{\trans}{\xrightarrow}
\newcommand{\configCmd}[1]{\aCmd_{#1}}
\newcommand{\aconfigCmd}[1]{{\rm Cmd}_{\aConfig}}
\newcommand{\aConfigV}{{\mathcal C}}
\newcommand{\bConfigV}{{\mathcal D}}
\newcommand{\seqsem}[1]{{\rm SEQ}\sem{#1}}

\section{Relation to Hoare Logic}
In this section, we consider commands without synchronization, concurrency and local registers, i.e.:
\begin{align*}
  % \amode \BNFDEF& \modeRLX \BNFSEP \modeRA \BNFSEP \modeSC
  % \\
\aCmd,\,\bCmd
\BNFDEF& \SKIP
\BNFSEP \aReg\GETS\aExp\SEMI \aCmd
\BNFSEP \aReg\GETS\aLoc^{\amode}\SEMI \aCmd 
\BNFSEP \aLoc^{\amode}\GETS\aExp\SEMI \aCmd
\BNFSEP \IF{\aExp} \THEN \aCmd \ELSE \bCmd \FI
\end{align*}
where $\amode$ is only $\modeRLX$.  So, in the rest of the section, we elide all access modifiers.  
For the purposes of this section, we assume that registers are single assignment.   

For such commands, we consider the evident sequential operational semantics relating configurations with a memory and a store; we use $\aConfigV, \bConfigV$ to range over configurations.  The transition relation is of the form
\[ \aConfig \trans{} \bConfig \] and is described below. 
We write $\upd{\aMem}{\aLoc}{\aVal}$ for the memory that results by updating the value of $\aLoc$ by $\aVal$ (and leaving the other locations untouched).
\begin{align*}
\config{\aMem}{\aStore}{\aReg\GETS\aVal \SEMI \aCmd} & \trans\  
\config{\aMem}{\aStore}{\aCmd[\aVal/\aReg]} \\
\config{\upd{\aMem}{\aLoc}{\aVal}}{\aStore}{\aReg\GETS\aLoc \SEMI \aCmd} & \trans\ 
\config{\upd{\aMem}{\aLoc}{\aVal}}{\aStore}{\aCmd[\aVal/\aReg]} \\
\config{\aMem}{\aStore}{\aLoc \GETS \aVal \SEMI \aCmd}
& \trans\  
\config{\upd{\aMem}{\aLoc}{\aVal}}{\aStore}{ \aCmd} \\
\config{\aMem}{\aStore}{\IF{0==0} \THEN \aCmd \ELSE \bCmd}
& \trans\  
\config{\aMem}{\aStore}{ \bCmd} \\
\config{\aMem}{\aStore}{\IF{n==0} \THEN \aCmd \ELSE \bCmd}
& \trans\ 
\config{\aMem}{\aStore}{ \aCmd}, n \neq 0 
\end{align*}

With a configuration $\aConfigV = \aConfig$, we associate a command $\aMem \SEMI \aStore \aCmd$, that converts the initial memory and store into assingnments, and define:
\[ \sem{\aConfig} = \eqdef\sem{\aMem \SEMI \aStore \aCmd}  \]

\begin{definition}[Sequential Pomsets]
A pomset $\aPS$ is {\em sequential} if for any $\aLoc$:
\begin{itemize}
\item $\aPS$ has atmost one write to $\aLoc$, and
\item All reads of $\aLoc$ read the same value, and 
\item there are no edges out of any read to $\aLoc$
\item All preconditions are true
\end{itemize}
$\seqsem{\aCmd}  = \{ \aPS \mid \aPS \in \sem{\aCmd}, \aPS\ \mbox{ is sequential} \}$
\end{definition}

\begin{lemma}\label{soundp}
If $\aConfigV \trans \bConfigV$, then $\freadsat(\seqsem{\aConfigV}) = \freadsat(\seqsem{\bConfigV})$.
\end{lemma}
\begin{proof}
The proof follows from the peephole optimizations from section~\ref{sec:opt}, where the resriction to sequential pomsets strengthens the inequalities to equalities.    

\eqref{RL} becomes a equality for sequential pomsets because all reads of $\aLoc$ read the same value.  

\eqref{SF} becomes a equality for sequential pomsets because there are no edges out of a read. Since there are no non-trvial preconditions, this forces the value of the read to agree with the prior write.   The proof is completed by noting that read-saturation acccounts for the read missing from the right hand side.

\eqref{DS}  becomes a equality for sequential pomsets because they only track the final write. 

{\bf COmment to James.  Add law for:  r=M; C and C[M/r]}
\end{proof}

\begin{verbatim}
ReadSat(SEQ([| C }) = ReadSat(SEQ([ final memory of C --> }) 


So: ReadSat(SEQ([| C }) = ReadSat(SEQ([| D })

  ==> ReadSat(SEQ([ final memory of C --> })  = ReadSat(SEQ([ final memory of D --> }) 

 

From which we conclude by Hoare completeness
\end{verbatim}

\begin{corollary}
Let $\aForm$ be a conjunction of atoms of the form $\aLoc = \aVal$.  Then, for any configurations $\aConfigV, \bConfigV$:
\[ (\forall \aForm) \hoare{\TRUE}{\aConfigV}{\aForm} \Leftrightarrow\   \hoare{\TRUE}{\bConfigV}{\aForm}  \]
IFF
\[ \freadsat(\seqsem{\aConfigV}) = \freadsat(\seqsem{\bConfigV}) \]
\end{corollary}
\begin{proof}
By the soundness and completeness of standard Hoare logic, as applied to our language where sequential composition is restricted to prefixing, we deduce that:
\[ (\forall \aForm) \hoare{\TRUE}{\aConfigV}{\aForm} \Leftrightarrow\   \hoare{\TRUE}{\bConfigV}{\aForm} 
\]
IFF
\[ \aConfigV \trans{} \config{\aMem}{\aStore}{\SKIP} \Leftrightarrow\  \bConfigV \trans{} \config{\aMem}{\bStore}{\SKIP} \]

Result follows from lemma~\ref{soundp} by observing that $\seqsem{\config{\aMem}{\aStore}{\SKIP}}$ is the set of writes determined by $\aMem$.
\end{proof}


\endinput






For such commands, we establish the relationship between the sequential semantics and the pomset semantics. 



In this section, we explore the 

Comments from Radha:

Linear is incorrect as stated. eg. skip is Hoare equiv to r=x;x =r (which is
linear) but cannot be substituted by it.  It is necessary to add the proviso
linear AND (writes only if original program writes)

Thm 7.2 to:

    
a. Defn of linear needs to forbid reads after writes. 

The theorem does not hold for:
\begin{verbatim}
   x=1; r=x
\end{verbatim}
vs
\begin{verbatim}
   x=1
\end{verbatim}
They are both linear. Satisfy the same hoare triples.  But have different
read(C)




Note that if $\aPSS'\supseteq\aPSS$, then
$\readc(\aPSS')\supseteq \readc(\aPSS)$.

% We give an abstract characterization of valid optimizations, establishing
% sufficient conditions to replace any command $\aCmd$ by an equivalent
% $\bCmd$: If $\aCmd$ and $\bCmd$ are synchronization\hyp{}free and
% sequentially equivalent, and furthermore $\bCmd$ is \emph{linear}---performs
% at most one read and at most one write on each location in each
% execution---then $\aCmd$ can be refined to $\bCmd$.

A program is \emph{sequential} if it lacks $\!\!\PAR\!\!$, and
\emph{synchronization-free} if lacks fences and $\modeRA/\modeSC$ access.  We
argue that our model is fully flexible with respect to optimization of such
programs, as long as the optimizations do not introduce new writes or
``relevant'' reads.  To do so, we first formalize a
relation between Hoare triples and pomsets with preconditions.
We then isolate a fragment of our
language and show
the soundness of {\em all} transformations of synchronization-free sequential
programs into this fragment.

%
\paragraph{Pomsets for Hoare Logic}
%\label{sec:hoare}
In \textsection\ref{sec:model}, we used Hoare logic
\cite{Hoare:1969:ABC:363235.363259,gordonHoare} to reason about
\emph{dependency analysis} for sequential code.  Here, we use an alternative
interpretation of Hoare logic to establish the soundness of \emph{program
  transformations}.  We develop a pomset semantics for pairs of formulae
$\semRW{\aForm}{\bForm}$, and show a relation between $\sem{\aCmd}$ and
$\semRW{\aForm}{\bForm}$ for valid Hoare triples
$\hoare{\aForm}{\aCmd}{\bForm}$.  In \textsection\ref{sec:model},
preconditions were discharged by read events.  Here, instead, preconditions
are derived from the read events themselves.

\begin{definition}
  \label{def:prepost}
  Let $\aPS\in\semRW{\aForm\land\cForm}{\bForm}$
  %We say that $\aPS$ \emph{satisfies} precondition $\aForm$ and postcondition $\bForm$ (notation $\aPS\in\semRW{\aForm}{\bForm}$)
  when it is possible to satisfy the following:

  Let $\aLocs$ be the set of locations such that $\aLoc\in\aLocs$ exactly
  when $\aForm$ depends on $\aLoc$.  For each $\aLoc\in\aLocs$, choose
  $\bEv_\aLoc\in\Event$ that reads $\aLoc$.
  Let $\bEvs_\aLocs=\{\bEv_\aLoc\mid\aLoc\in\aLocs\}$.
  Let $\aSub$ be the substitution generated % from $\aLocs$
  as follows:
  % For $\aLoc\in\aLocs$, 
  $\aLoc\aSub = \aVal$ exactly when $\bEv_\aLoc$ reads $\aVal$ from $\aLoc$.

  Let $\bLocs$ be the set of locations such that $\bLoc\in\bLocs$ exactly
  when $\bForm$ depends on $\bLoc$.  For each $\bLoc\in\bLocs$, choose
  $\aEv_\bLoc\in\Event$ that writes $\bLoc$.
  Let $\aEvs_\bLocs=\{\aEv_\bLoc\mid\bLoc\in\bLocs\}$.
  Let $\bSub$ be the substitution generated % from $\bLocs$
  as follows:
  % For $\bLoc\in\bLocs$, 
  $\bLoc\bSub = \aVal$ exactly when $\aEv_\bLoc$ writes $\aVal$ to $\bLoc$.

  Require that $\aForm\aSub$ and $\bForm\bSub$ are satisfiable.

  Require that $\labelingForm(\aEv_\bLoc)$ implies $\cForm$ (for each $\aEv_\bLoc$).
  
  Require that if $\cEv\le\aEv_\bLoc$ and $\cEv$ is a read, then $\cEv\in\bEvs_\aLocs$.

  Require that if $\aEv_\bLoc\le\cEv$ and $\cEv$ is a write to $\bLocs$, then $\cEv\in\aEvs_\bLocs$.
\end{definition}
Pictorially, we have:
\begin{tikzdisplay}[node distance=.1ex and 2em]
  \event{r}{\bEvs_\aLocs}{}
  \event{w2}{\cForm\mid\aEvs_\bLocs}{below right=of r}
  \event{w1}{\cEvs_\bLocs}{above right=of w2}
  \po{r}{w2}
  \wk{w1}{w2}
\end{tikzdisplay}
Here, $\aEvs_\bLocs$ are the final writes to $\bLocs$, with precondition $\cForm$.
$\cEvs_\bLocs$ are other writes to $\bLocs$, which must be ordered before $\aEvs_\bLocs$.
$\bEvs_\aLocs$ are the reads that the writes depend upon.

Precondition strengthening in Hoare logic validates read introduction, which
is invalid in our semantics.  Thus, in this section we consider pomsets that
are saturated with irrelevant reads.

% Recall that postconditions are properties of \emph{completed} executions.
% For example, in $\hoare{\TRUE}{x\GETS1\SEMI x\GETS2}{x{=}2}$, the
% postcondition does not hold for the prefix $x\GETS1$.  % As a result, we
% restrict attention to completed executions.  % A pomset
% $\aPS\in\aPSS$ is \emph{maximal} if there is no $\aPS'\in\aPSS$ such that
% $\aPS$ is a proper prefix of $\aPS'$.


\labeltext{Let}{page:readsat} $\readc(\aPSS)$ be the set $\aPSS'$ where $\aPS'\in\aPSS'$ when
$\exists\aPS\in\aPSS$ and  $\exists D$ such that $\Event'= \Event'\uplus D$,
${\le'} \supseteq{\le}$, $\labeling'(\aEv) = \labeling(\aEv)$, and for every
$\bEv\in D$ there are $\aLoc$ and $\aVal$ such that
$\labelingAct'(\bEv)=(\DR[\modeRLX]{\aLoc}{\aVal})$.

\todo{Let xs=ws, ys=vs be such that xi=wi not in ys vs. Then:
   {ys=vs} C {xs=ws}  <==>  [ys=vs | xs=ws] cap read(C) not= emptyset}
\begin{theorem}
  \label{thm:hoare}
  If $\aCmd$ is synchronization-free and sequential,
  $\hoare{\aForm}{\aCmd}{\bForm}\Longleftrightarrow\notdisjoint{\semRW{\aForm}{\bForm}}{\readsem{\aCmd}}$.
\begin{proof}
  The proof proceeds by induction on the derivation of the Hoare triple.

  We first consider the structural rules.  Precondition strengthening follows
  from augmentation closure.   The structural rule for
  disjunction
  {\small\begin{math}
    \frac{\hoare{\aForm_1}{\aCmd}{\bForm_1}\quad  \hoare{\aForm_2}{\aCmd}{\bForm_2}}{ \hoare{\aForm_1 \lor \aForm_2}{\aCmd}{\bForm_1\lor \bForm_2}} 
  \end{math}}
follows from disjunction closure (Definition~\ref{def:dis}). The structural rule for conjunction
  {\small\begin{math}
    \frac{\hoare{\aForm_1}{\aCmd}{\bForm_1}\quad \hoare{\aForm_2}{\aCmd}{\bForm_2}}{ \hoare{\aForm_1 \land \aForm_2}{\aCmd}{\bForm_1\land \bForm_2}} 
  \end{math}}
  follows from the fact that pomsets have only concurrency and no conflict.  

  The remaining cases follow directly from the semantics.  The only subtlety
  is the write rule, which uses $\parallel$ to ensure disjunction closure.
\end{proof}
\end{theorem}
% To illustrate multiple writes consider:
% \begin{gather*}
%   \hoare{\TRUE}{x\GETS 1 \SEMI x \GETS 2}{x=2}
%   \\
%   \hbox{\begin{tikzinline}[node distance=1em]
%       \event{w1}{\DW{x}{1}}{}
%       \event{w2}{\DW{x}{2}}{right=of w1}
%       \wk{w1}{w2}
%     \end{tikzinline}}
% \end{gather*}

Preconditions can be placed in $\aForm$ or $\cForm$ in
Definition~\ref{def:prepost}, resulting in different pomsets:
\begin{gather*}
  \hoare{x=1}{y \GETS x}{y=1}
  \\
    \hbox{\begin{tikzinline}[node distance=1em]
        \event{r}{\DR{x}{1}}{}
        \event{w}{\DW{y}{1}}{right=of r}
        \po{r}{w}
      \end{tikzinline}}
    \qquad\qquad
    \hbox{\begin{tikzinline}[node distance=1em]
        \event{r}{\DR{x}{1}}{}
        \event{w}{x=1 \mid \DW{y}{1}}{right=1ex of r}
      \end{tikzinline}}
\end{gather*}
Control dependencies are calculated correctly:
\begin{gather*}
  \hoare{\TRUE}{\IF{x} \THEN y \GETS 1 \ELSE y \GETS 1 \FI}{y=1} 
  \\
    \hbox{\begin{tikzinline}[node distance=1em]
        \event{r}{\DR{x}{1}}{}
        \event{w}{\DW{y}{1}}{right=1ex of r}
      \end{tikzinline}}
\end{gather*}
% In contrast, in a semantics that forbids load buffering, where the best that one can prove is
% $\hoare{x=v} {r \GETS x \SEMI \IF{r=1} \THEN y \GETS 1 \ELSE y \GETS 1 \FI}{y= 1}
% $.

For any consistent set of preconditions, we can always find a single pomset:
\begin{gather*}
  \hoare{x_1=1}{y_1 \GETS x_1}{y_1=1}
  \qquad
  \hoare{x_2=1}{y_2 \GETS x_2}{y_2=1}
  \\
    \hbox{\begin{tikzinline}[node distance=1em]
        \event{rx}{\DR{x_1}{1}}{}
        \event{wy}{\DW{y_1}{1}}{right=of rx}
        \event{ru}{\DR{x_2}{1}}{right=1ex of wy}
        \event{wv}{\DW{y_2}{1}}{right=of ru}
        \rf{rx}{wy}
        \rf{ru}{wv}
      \end{tikzinline}}
\end{gather*}
% \begin{corollary}
%   If $\bigwedge_{i\in I}\aForm_i$ is satisfiable,
%   \begin{math}
%     \textstyle\bigwedge_{i\in I}\hoare{\aForm_i}{\aCmd}{\bForm_i} \Longleftrightarrow
%     \notdisjoint{\textstyle\bigcap_{i\in I}\semRW{\aForm_i}{\bForm_i}}{\readsem{\aCmd}}.
%   \end{math}
% \end{corollary}

\paragraph{Linearity}
%\label{sec:linear}
A command $\aCmd$ is \emph{linear} if for every $\aPS\in\sem{\aCmd}$, (1) there is
at most one read and at most one write on any location, and (2) if there is
both a read and a write to the same location, then the reads is ordered
before the write.

Intuitively, this means that the context around $\aCmd$ is unable to
interfere with the atomic execution of $\aCmd$; dually, neither can the
atomic execution of $\aCmd$ interfere with the context.  From an arbitrary
command, it is a straightforward exercise to construct a linear command that
is sequentially equivalent.

We say that $\aCmd$ and $\aCmd'$ \emph{satisfy the same Hoare triples} when
$\hoare{\aForm}{\aCmd}{\bForm}$ if and only if
$\hoare{\aForm}{\aCmd'}{\bForm}$, for every $\aForm$ and $\bForm$.

\begin{corollary}\label{seqcompleteness}
  Let $\aCmd$ and $\aCmd'$ be synchronization-free and sequential.  Further,
  let $\aCmd'$ be linear.
  Then $\aCmd$ and $\aCmd'$ satisfy the same Hoare
  triples if and only if $\readsem{\aCmd} \supseteq \readsem{\aCmd'}$.
  % If $\aCmd$ and $\aCmd'$ satisfy the same Hoare
  % triples then $\readsem{\aCmd} \supseteq \readsem{\aCmd'}$.
%   \begin{proof}
%     The pomsets in the semantics of a linear sequential program fragment,
%     such as $\sem{\bCmd}$, are generated by the augmentation closure of
%     pomsets that have a special format that only include edges of the
%     form: \begin{tikzdisplay}[node distance=1em]
%       \event{r}{\smash{\vec{\bForm}}\mid\DR{\vec{\bLoc}}{\vec{\aVal}}}{}
%       \eventl{\aEv}{w}{\aForm\mid\DW{\aLoc}{\bVal}}{below right=of r}
%       \po{r}{w}
%     \end{tikzdisplay}
%     where $\vec{\bLoc}\GETS \vec{\aLoc}$ has no conflicting assignments to the same variable, and where each $\aLoc$ appears in at most one write event. Thus, using  theorem~\ref{hoareGen}, we deduce that $\sem{\bCmd}$ is completely determined by the Hoare triples satisfied by $\bCmd$.  
%    
%     By the hypothesis of this theorem, $\aCmd$ satisfies the same Hoare triples as $\bCmd$.  Using  theorem~\ref{hoareGen}, we deduce that $\sem{\bCmd} \subseteq \fsat(\sem{\aCmd})$. 
% \end{proof}
\end{corollary}

The linearity restriction ensures that the context cannot interfere with the
atomic execution of the command, and, dually, that the atomic execution of
the command cannot interfere with the context.  To see the need for this,
consider that the introduction of redundant relevant reads is valid sequentially, but
not valid concurrently. For example,
$\aReg \GETS \aLoc \SEMI \IF{\aReg \NOTEQ \aReg}\THEN \bLoc \GETS 1 \FI$
cannot be refined to
$\aReg \GETS \aLoc \SEMI \bReg \GETS \aLoc \SEMI \IF{\aReg \NOTEQ \bReg}
\THEN \bLoc \GETS 1 \FI$.  In a concurrent context, the latter program may
see different values for the two reads.

Order is necessary to exclude optimizations such as
$x\GETS 1 \SEMI r \GETS x$ to $x\GETS 1$. They are both linear. Satisfy the
same Hoare triples.  \todo{But have different read(C).}

\endinput 




\subsection{Full abstraction for synchronization free threads}
Our semantics is complete for reasoning about full-thread optimizations of synchronization free programs.

In the rest of this section, we only consider commands $\aCmd,\bCmd$ that are
restriction-free, composition-free (ie. single-threaded), and
synchronization-free (ie. no acquire, release, fence).

In order to develop the proof, 

This closure permits us to  describe a normal form for the top-level pomsets that arise in single-threaded and synchronization free code.  
\begin{definition}
$\aPS$ is in normal form if:
\begin{itemize}
% \item All preconditions on events are tautologies.
\item If $\bEv \lt \aEv$ and  $\bEv$ is a write, then $\aEv$ is a write or read on the same variable.
\item If $\bEv \lt \aEv$ and  $\aEv$ is a read, then $\bEv$ is a write on the same variable. 
    
% \item $\aEv \gtN \bEv$ only if $ \aEv\ (\lt \cup \reco)^{\star} \  \bEv$.
\item If $\aEv, \aEv'$ have the same read action label, then there exists a write $\bEv$ on the same location such that $\aEv \gtN \bEv \gtN \aEv'$.
%\item  If $\aEv, \aEv'$ have the same write action label, then 
%there exists a event $\bEv$ on the same location as $\aEv$ such 
%that  $\bEv$ has a different action label and $\aEv \gtN \bEv \gtN 
%\aEv'$.
\end{itemize}
\end{definition}
In a normal form pomset, the successors of a write event (resp. the predecessors of a read) are related in the coherence order to the event.  Any two events with the same read label are separated by a write in the $\reco$ order. 

It suffices to consider normal forms when distinguishing single-threaded, synchronization free code at the top level.  The normal forms determine the full semantics by the closure properties of the semantics.

\begin{lemma}\label{unrhd}
Every top-level pomset of $\freadsat\sem{\aCmd}$ is a top-level pomset of $\freadsat\sem{\bCmd}$ if and only if 
every top-level normal form pomset of $\freadsat\sem{\aCmd}$ is a top-level normal form pomset of $\freadsat\sem{\bCmd}$.
\begin{proof}
The proof proceeds by induction on structure.  The key case is prefixing.  The only edges $\lt$-edges out of writes  and into read actions enforced by prefixing are $\reco$ edges.  Read prefixing permits the reuse of events to ensure that distinct read events not separated by $\reco$ have distinct read labels.  
\end{proof}
\end{lemma}

We develop testers for top-level pomsets in normal form.  We  follow~\citet{Plotkin:1997:TSP:266557.266600}, albeit in a concrete form appropriate to our setting.   

Let $\aPS$ be a top-level pomset in normal form with events $\aEv_1, \ldots,
\aEv_n$.  For all $i$, we assume a new location $b_i$.    Let $\freshaval$ be a fresh value that does not occur in $\aPS$.  

Let $\vec{\bEv}$ be all  the predecessors of $\aEv_i$ in $\lt$. Let $\vec{c}$ be the corresponding sub vector of $b_i$'s. 
\begin{itemize}
\item 
If $\aEv_i$ has label $\DR{\aLoc}{\aVal}$, we define a program $\testP{\aPS}{i}$ as follows:  
\[
  \testP{\aPS}{i} = (
  % \vec{\aReg} \GETS \vec{c}  \SEMI
  \IF{\vec{c}=\vec{1}} \THEN x \GETS \aVal \SEMI  \FENCE \SEMI b_i \GETS 1 \FI)
\]
\item 
If $\aEv_i$ has label $\DW{\aLoc}{\aVal}$, we define a program $\testP{\aPS}{i}$ as follows:
\[
  \testP{\aPS}{i} = (
  % \vec{\aReg} \GETS \vec{c}  \SEMI
  \IF{\vec{c}=\vec{1}} \THEN %\aReg = \aLoc \SEMI
  \IF{\aLoc = \aVal} \THEN \aLoc = \freshaval \SEMI \FENCE \SEMI b_i \GETS 1  \FI \FI)
\]
\end{itemize}
$\testP{\aPS}{i}$ is as expected, matching the reads (resp. writes) in $\aPS$ by writes (resp. reads).  The fresh value is used to flush out prior writes and reset to see fresh writes.    

\begin{definition}[Tester for $\aPS$]\label{testAPS} 
The tester context for $\aPS$, $\testP{\aPS}{}\hole{}$ is 
\[
b \GETS b_{1} \land  \cdots \land b_{n}
\PAR  \testP{\aPS}{1} 
\PAR   \cdots 
\PAR  \testP{\aPS}{n} 
\PAR \hole{}
\]
\end{definition}
The constraints of the normal form ensure that if the labels of any two events are the same, then one is a predecessor of the other. 

\begin{lemma}\label{tester}
$\sem{\testP{\aPS}{}\hole{\aCmd}}$  has a pomset that sets $b$ to $1$ iff $\aPS \in \freadsat\sem{\aCmd}$.
\begin{proof}
It suffices to prove that there is a pomset that sets $b$ to $1$ in
 $\freadsat\sem{b \GETS b_{1} \land  \cdots \land b_{n}
\PAR  \testP{\aPS}{1} 
\PAR   \cdots 
\PAR  \testP{\aPS}{n}} \parallel \bPS$ iff $\aPS$ is an augmentation of $\bPS$. 

We prove by induction on $\lt$ that $\testP{\aPS}{i}\hole{\aCmd}$  sets $b_i$ to $1$ iff the event $\aEv_i$ is enabled. Result follows.
\end{proof}
\end{lemma}

\begin{theorem}
  Let $\aCmd_1$ and $\aCmd_2$ be restriction-free, composition-free and
  synchronization free.  Then all top-level pomsets of
  $\freadsat\sem{\aCmd_1}$ are also pomsets of $\freadsat\sem{\aCmd_2}$ if
  and only if for all parallel contexts $\bCmd$, all top-level pomsets of
  $\freadsat\sem{\aCmd_1 \PAR \bCmd}$ are also pomsets of
  $\freadsat\sem{\aCmd_2 \PAR \bCmd}$
\begin{proof}
  The forward implication follows from the compositionality of the semantics.

  For the converse, pick a top level pomset in normal form in
  $\aPS \in \freadsat\sem{\aCmd_1} \setminus \freadsat\sem{\aCmd_2} $.  The
  required context is given by the tester context $\testP{\aPS}{}\hole{}$.
 \end{proof}
\end{theorem}

The full abstraction theorem applies only to full threads.  The reason for this limited statement is that our impoverished language does not permit  parallel composition to be used freely as the continuation in arbitrary sequential contexts.  A full abstraction theorem that applies also to commands can be achieved with these richer distinguishing contexts.  
