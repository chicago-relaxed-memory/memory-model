\section{Comments for new version}
Cleanup of \eqref{alan}
\begin{equation}
  \label{alan}
  (
    y\GETS0\SEMI y\GETS x
  \PAR
    x\GETS0\SEMI \IF{z}\THEN x\GETS1 \ELSE x\GETS y\SEMI a\GETS y \FI
  \PAR
    z\GETS0\SEMI z\GETS1
  )
\end{equation}
% \begin{tikzdisplay}[node distance=1em]
%   \event{rx}{\DR{x}{1}}{}
%   \event{wy}{\DW{y}{1}}{below=of rx}
%   \po{rx}{wy}
%   \event{wy0}{\DW{y}{0}}{left=of rx}
%   \wk{wy0}{wy}
%   \event{ry}{\DR{y}{1}}{right=4em of rx}
%   \event{wx}{\DW{x}{1}}{below=of ry}
%   \po{ry}{wx}
%   \event{wx0}{\DW{x}{0}}{right=of ry}
%   \wk{wx0}{wx}
%   \rf{wy}{ry}
%   \rf{wx}{rx}
%   \event{rz}{\DR{z}{0}}{right=of wx0}
%   \event{wz0}{\DW{z}{0}}{right=4em of rz}
%   \rf{wz0}{rz}
%   \event{wz1}{\DW{z}{1}}{below=of wz0}
%   \wk{wz0}{wz1}
%   \event{ry1}{\DR{y}{1}}{right=of wx}
%   \rf[bend right]{wy}{ry1}
%   \event{wa}{\DW{a}{1}}{right=of ry1}
%   \po{ry1}{wa}
%   \po{rz}{wa}
% \end{tikzdisplay}
\begin{tikzdisplay}[node distance=1em]
  \event{wy0}{\DW{y}{0}}{}
  \event{rx}{\DR{x}{1}}{right=4.5em of wy0}
  \event{wy}{\DW{y}{1}}{right=of rx}
  \po{rx}{wy}
  \wk[bend left]{wy0}{wy}
  \event{wx0}{\DW{x}{0}}{below=of wy0}
  \event{rz}{\DR{z}{0}}{right=of wx0}
  \event{ry}{\DR{y}{1}}{right=of rz}
  \event{wx}{\DW{x}{1}}{right=of ry}
  \event{ry1}{\DR{y}{1}}{right=of wx}
  \event{wa}{\DW{a}{1}}{right=of ry1}
  \rf{wy}{ry1}
  \po{ry}{wx}
  \wk[bend right]{wx0}{wx}
  \rf{wy}{ry}
  \rf{wx}{rx}
  \event{wz0}{\DW{z}{0}}{below=of wx0}
  \event{wz1}{\DW{z}{1}}{right=of wz0}
  \rf{wz0}{rz}
  \wk{wz0}{wz1}
  \po{ry1}{wa}
  \po[bend right]{rz}{wa}
\end{tikzdisplay}

Referee's program:
\begin{equation}
  \label{referee}
  (
    y\GETS0\SEMI y\GETS x
  \PAR
    x\GETS0\SEMI \IF{y}\THEN x\GETS y\SEMI a\GETS1 \ELSE x\GETS1 \FI
  )
\end{equation}
\begin{tikzdisplay}[node distance=1em]
  \event{wy0}{\DW{y}{0}}{}
  \event{rx}{\DR{x}{1}}{right=4.5em of wy0}
  \event{wy}{\DW{y}{1}}{right=of rx}
  \po{rx}{wy}
  \wk[bend left]{wy0}{wy}
  \event{wx0}{\DW{x}{0}}{below=of wy0}
  \event{ry1}{\DR{y}{1}}{right=of wx0}
  \event{ry}{\DR{y}{1}}{right=of ry1}
  \event{wx}{\DW{x}{1}}{right=of ry}
  \event{wa}{\DW{a}{1}}{right=4.5em of wx}
  \wk[bend right]{wx0}{wx}
  \po{ry}{wx}
  \rf{wy}{ry1}
  \rf{wy}{ry}
  \rf{wx}{rx}
  \po[bend right]{ry1}{wa}
\end{tikzdisplay}
Note that there is no order between $\DR{y}{1}$ and $\DW{x}{1}$.


From \cite{vacuous}:

\begin{quotation}
  It would be nice to have a succinct description of the set of test cases in
  which perturbation functions introduced inconsistencies. Ali Sezgin pointed
  out this set is described by $rf \cup sdep$, where rf is the reads-from
  relationship and sdep is ``semantic dependence'', roughly defined as those
  dependency relationships in which at least some changes in the value at the
  head of the dependency relationship propagate through, resulting in a
  change at the tail of that relationship.

  Prohibiting executions that have cycles in $rf \cup sdep$ can therefore be
  expected to prohibit OOTA behaviors.

  One beneficial consequence of this relationship to semantic dependency is
  that $rf \cup nsdep$ cycles are allowed, where $nsdep \cap sdep$ is the empty set and
  where $nsdep \cup sdep = dep$. This means that the compiler is free to replace
  expressions that are known to always result in a single value with the
  corresponding constant, without danger of introducing OOTA behavior. We
  hypothesize that non-speculative code-reordering optimizations are
  similarly unable to introduce OOTA behavior.

  Defining ``semantic dependency'' sufficiently for formal modeling remains an
  open issue. In the general case, this the question of whether or not a
  given dependency is a semantic dependency is of course
  undecidable. However, this question can be decided straightforwardly in
  many common cases. One approach would be to flag dependencies that the tool
  was unable to classify. Another approach would be to consider cases that a
  given compiler might optimize, and to classify other cases as semantic
  dependencies.
\end{quotation}

And also this:

\begin{quotation}
  This perturbation analysis appears to be equivalent to requiring that $rf \cup
  sdep$ be acyclic. This is an extremely important result: It means that any
  compiler optimization that substitutes a constant value for a read known to
  return that value cannot induce OOTA behavior. This constraint should also
  ensure that non-speculative code-movemenet optimizations should be
  similarly unable to induce OOTA behavior.

  Effective and efficient modeling of semantic dependencies (sdep) remains an
  important open problem, although there is important work in progress
  \cite{Pichon-Pharabod:2016:CSR:2837614.2837616}.
\end{quotation}

% Local Variables:
% mode: latex
% TeX-master: "paper"
% End: