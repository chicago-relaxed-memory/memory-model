
\section{Introduction}

\subsection{scribbles}
Merge order.
Augment order.
Implication order.
Generator is minimal in all of these, so po has unique predecessors.

cross thread edges in generator are write-to-read and not from augmentation.

\subsection{intro}

\citet{Hoare:1969:ABC:363235.363259} described a logic for understanding
sequential execution.  For example, the rule for assignment states that:
\begin{displaymath}
  \hoare{\aForm[\aExp/\aLoc]}{\aLoc\GETS\aExp}{\aForm}
\end{displaymath}
This views a program as a \emph{predicate transformer}, made explicit by
\citet{Dijkstra:1975:GCN:360933.360975}.
\begin{displaymath}
  \fwp(\aLoc\GETS\aExp,\;\aForm) = \aForm[\aExp/\aLoc]
\end{displaymath}
This interpretation gives a natural notion of program equality.  It defines a
semantics for all programs.  It is compositional.  It supports type-safe
execution.  It is preserved by compilers and single-threaded hardware.
(Indeed, equivalence under this model can be taken as the correctness
condition for compiler and hardware optimizations.)


Concurrent programming is more difficult to understand, but in certain
special cases substantial progress has been made, such as race-free
shared-memory
\citep{OHearn:2007:RCL:1235896.1236121,OHearn:2019:SL:3310134.3211968} and
message-passing \citep{Hennessy:1980:ONC:646234.758793,Cleaveland2018}.

For \emph{shared-memory programs with data races (SMP-DR)}, no canonical
model has arisen.  A sensible model should induce a natural notion of program
equality.  It should define a semantics for all programs.  It should be
compositional.  It should support type-safe execution.  It should be
preserved by current compilers and multi-threaded hardware.  It should also be
\emph{coherent}, in that writes to a single location appear to be totally
ordered.  The last two points are descriptive, in that they capture existing
systems, yet they bound the definition in different directions: To support
optimization, the model cannot be to strong.  To support coherence, the model
cannot be too weak\footnote{From this point of view, coherence is what
  distinguishes shared memory from a truly distributed system.}.

Concurrent models require that we identify events for read and write actions,
which might later allow for interaction with another thread.  Yet, the
correctness of transformations on single-threaded code are defined
semantically.  Getting these two world-views to play nice has proven
incredibly challenging.

Rather than a single canonical model, we have arrived at a spectrum of models
for SMP-DR.  These can be characterized as either \emph{strong} or
\emph{weak}.  The strong models invalidate some common compiler and hardware
optimizations and reorderings, such as common subexpression elimination or
load-buffering.

Sequential consistency (SC) \citep{Lamport:1979:MMC:1311099.1311750} is
universally accepted as the strongest sensible model of SMP-DR.  In addition
to invalidating optimizations, SC is also non-compositional.  Nevertheless,
SC and other strong models my work well in practice
\cite{Singh:2012:ESC:2337159.2337220,Dolan:2018:BDR:3192366.3192421,Ou:2018:TUC:3288538.3276506,Liu:2019:ASC:3314221.3314611}.

Attempts to define a weakest model for SMP-DR have largely failed.
\begin{enumerate}
\item \citet{Manson:2005:JMM:1047659.1040336} and similar models
  \cite{DBLP:conf/esop/JagadeesanPR10,DBLP:conf/popl/KangHLVD17} all require
  shenanigans for type safety
  \cite{DBLP:journals/toplas/Lochbihler13,DBLP:conf/tldi/GotoJPR12}.  This is
  a variant of the thin-air problem.
\item \citet{DBLP:conf/lics/JeffreyR16} is too strong.
\item C11 does not define a semantics for racy programs.  Even restricted to
  the defined subset, C11 still allows thin-air type unsafety on relaxed
  atomics. The attempts to repair C11 have lead to strong models.
\end{enumerate}

In this paper, we propose a candidate definition for
\begin{center}
  \emph{the weakest sensible model of SMP-DR}.
\end{center}

Our model uses Hoare logic for sequential code and partially-ordered sets of
read-write events for concurrent code.  As usual, the order denotes temporal
dependency between two events.

Our first insight is that these two approaches can be combined by associating
logical predicates with events in the partial order.  Sequencing affects the
predicates, as in Hoare logic.  Parallel composition, instead, combines the
events of two partial orders.

Our second insight is that when combining partial orders from the two
branches of a conditional, events may coalesce, resulting in weaker
predicates via disjunction.  This allows events that occur within both
branches of a conditional to float in the partial order, independent of the
condition itself.

There are several details to sort out, which we dutifully perform:
\begin{itemize}
\item give semantics of release and acquire fences
\item establish DRF-SC
\item establish logical reasoning
\item establish congruence properties for language constructs
\item establish that compiler optimizations are permitted
\item establish that hardware optimizations are permitted
\end{itemize}

Things we do not do:
\begin{itemize}
\item give semantics of read-modify-write operations
\item give semantics for non-terminating programs
\item prove type safety for a nontrivial language
\end{itemize}
About the last point, we do prove that our semantics disallows examples such
as Figure 27 of \cite{DBLP:journals/toplas/Lochbihler13}, which is closely
related to type safety.


% Local Variables:
% mode: latex
% TeX-master: "paper"
% End: