\section{Single-Threaded Optimizations}
\label{sec:opt}

When $\sem{\aCmd} \supseteq \sem{\aCmd'}$, we say that $\aCmd'$ is a
\emph{valid transformation} of $\aCmd$.  In this section, we show the
validity of specific optimizations and discuss ``optimizations'' that fail,
such as \ref{RFUB}.


\paragraph{Valid Rewrites}
%\label{sec:valid}

% To enable reasoning about program fragments, transformation validity must be
% preserved by \emph{contexts}.  In \textsection\ref{sec:model}, we defined the
% semantics by prefixing one action at a time.  This helps to make the
% semantics understandable, but it also creates impoverished contexts.

% To allow for richer contexts, we appeal to the alternate presentation of the
% language given in \textsection\ref{sec:semicolon}.  We refactor the syntax
% of commands and define contexts:
% \begin{align*}
%   \aCmd,\,\bCmd
%   \BNFDEF& \SKIP
%   \mkern-2mu\BNFSEP\mkern-2mu \FENCE^{\fmode}
%   \mkern-2mu\BNFSEP\mkern-2mu \aReg\GETS\aExp
%   % \mkern-2mu\BNFSEP\mkern-2mu \aReg\GETS \aLoc^{\amode} 
%   % \mkern-2mu\BNFSEP\mkern-2mu \aLoc^{\amode}\GETS\aExp
%   \mkern-2mu\BNFSEP\mkern-2mu \aReg\GETS \REF{\cExp}^{\amode} 
%   \mkern-2mu\BNFSEP\mkern-2mu \REF{\cExp}^{\amode}\GETS\aExp
%   \\[-.5ex]
%   \BNFSEP&\aCmd \PAR \bCmd
%   \mkern-2mu\BNFSEP\mkern-2mu\aCmd \SEMI \bCmd
%   \mkern-2mu\BNFSEP\mkern-2mu \VAR\aLoc\SEMI \aCmd
%   \mkern-2mu\BNFSEP\mkern-2mu \IF{\aExp} \THEN \aCmd \ELSE \bCmd \FI
%   \\
%   \aCtxt,\,\bCtxt
%   \BNFDEF& \hole{}
%   \mkern-2mu\BNFSEP\mkern-2mu \aCtxt \PAR \bCmd
%   \mkern-2mu\BNFSEP\mkern-2mu \aCmd \PAR \bCtxt
%   \mkern-2mu\BNFSEP\mkern-2mu \aCtxt \SEMI \bCmd
%   \mkern-2mu\BNFSEP\mkern-2mu \aCmd \SEMI \bCtxt
%   \mkern-2mu\BNFSEP\mkern-2mu \VAR\aLoc\SEMI \aCtxt
%   \\[-.5ex]
%   \BNFSEP& \IF{\aExp} \THEN \aCtxt \ELSE \bCmd \FI
%   \mkern-2mu\BNFSEP\mkern-2mu \IF{\aExp} \THEN \aCmd \ELSE \bCtxt \FI
% \end{align*}


% \begin{lemma}%\label{freadssatcomp}[Compositionality of $\freadsat$]
%   Let $\bCtxt$ be a context
%   % \footnote{%
%   %   The results of this section hold for contexts of the example language
%   %   given in \textsection\ref{sec:model} and extended in \textsection\ref{sec:variants}.
%   %   % \begin{math}
%   %   %   \begin{array}[t]{rcl}
%   %   %     \aCtxt,\,\bCtxt
%   %   %     &\BNFDEF& \hole{}
%   %   %     \BNFSEP \aReg\GETS\aExp\SEMI \aCtxt
%   %   %     \BNFSEP \aReg\GETS\aLoc^{\amode}\SEMI \aCtxt 
%   %   %     \BNFSEP \aLoc^{\amode}\GETS\aExp\SEMI \aCtxt
%   %   %     \\[-.5ex]
%   %   %     &\BNFSEP&\aCtxt \PAR \bCmd
%   %   %     \BNFSEP \aCmd \PAR \bCtxt
%   %   %     \BNFSEP \VAR\aLoc\SEMI \aCtxt
%   %   %     \\[-.5ex]
%   %   %     &\BNFSEP& \IF{\aExp} \THEN \aCtxt \ELSE \bCmd \FI
%   %   %     \BNFSEP \IF{\aExp} \THEN \aCmd \ELSE \bCtxt \FI
%   %   %   \end{array}
%   %   % \end{math}
%   %   The results of this section hold for contexts of the example language
%   %   given in \textsection\ref{sec:model} and extended in \textsection\ref{sec:variants}.
%   %   The results also hold for the more general contexts of
%   %   \textsection\ref{sec:semicolon}, which includes full sequential composition:
%   %   \begin{displaymath}
%   %     \begin{array}[t]{rcl}
%   %       \aCtxt,\,\bCtxt
%   %       &\BNFDEF& \hole{}
%   %       \BNFSEP \aCtxt \SEMI \bCmd
%   %       \BNFSEP \aCmd \SEMI \bCtxt
%   %       \BNFSEP \aCtxt \PAR \bCmd
%   %       \BNFSEP \aCmd \PAR \bCtxt
%   %       \BNFSEP 
%   %       \\[-.5ex]
%   %       &\BNFSEP& \VAR\aLoc\SEMI \aCtxt
%   %       \BNFSEP \IF{\aExp} \THEN \aCtxt \ELSE \bCmd \FI
%   %       \BNFSEP \IF{\aExp} \THEN \aCmd \ELSE \bCtxt \FI
%   %     \end{array}
%   %   \end{displaymath}}
% and $\readc\sem{\aCmd} \supseteq \readc\sem{\aCmd'}$:
% \begin{displaymath}
%   \readc\sem{\bCtxt\hole{\aCmd}} \supseteq \readc\sem{\bCtxt\hole{\aCmd'}}
% \end{displaymath}
% \end{lemma}

To discuss valid transformations without getting lost in notation, we present
them using simple locations, rather than calculated addresses.  The extension
is simple: For address expressions $\REF{\aExp}$ and $\REF{\bExp}$, replace
$\aLoc=\bLoc$ by provable equality of $\aExp$ and $\bExp$, and
$\aLoc\neq\bLoc$ by provable inequality.  Operations on
sets can be defined similarly.
%
Let $\free(\aCmd)$ be the set of locations and registers that occur in $\aCmd$.

The semantics validates several peephole optimizations.
\begin{align*}
  \taglabel{RL}
  \sem{\aReg \GETS \aLoc \SEMI \bReg  \GETS \aLoc\SEMI\aCmd} &\supseteq 
  \sem{\aReg \GETS \aLoc \SEMI \bReg  \GETS \aReg\SEMI\aCmd}
  \\
  \taglabel{DS}
  \sem{\aLoc \GETS \aExp \SEMI \aLoc  \GETS \bExp\SEMI\aCmd} &\supseteq 
  \sem{\aLoc \GETS \bExp\SEMI\aCmd}    
  \\
  \taglabel{WW}
  \sem{\aLoc \GETS \aExp \SEMI \bLoc  \GETS \bExp\SEMI\aCmd} &=
  \sem{\bLoc  \GETS \bExp\SEMI \aLoc \GETS \aExp\SEMI\aCmd} &&\text{if } \aLoc\neq\bLoc
  \\
  \taglabel{RW}
  \sem{\aReg \GETS \aLoc \SEMI \bLoc  \GETS \bExp\SEMI\aCmd} &=
  \sem{\bLoc  \GETS \bExp\SEMI \aReg \GETS \aLoc\SEMI\aCmd} &&\text{if } \disjoint{{\free(\aReg \GETS \aLoc)}}{{\free(\bLoc \GETS \bExp)}}
  \\
  \taglabel{RR}
  \sem{\aReg \GETS \aLoc \SEMI \bReg  \GETS \bLoc\SEMI\aCmd} &=
  \sem{\bReg  \GETS \bLoc\SEMI \aReg \GETS \aLoc\SEMI\aCmd} &&\text{if } \aReg\neq\bReg
\end{align*}
% \ref{WW}, \ref{RW} and \ref{RR} require that two sides of the semicolon
% have disjoint ids; for example, \ref{RW} requires $\disjoint{{\free(\aReg
%     \GETS \aLoc)}}{{\free(\bLoc \GETS \bExp)}}$. 
% \ref{RR} requires either $\aReg\neq\bReg$ or
%   $\aLoc=\bLoc$.  \ref{WW} and \ref{RW} require that two sides of the
% semicolon have disjoint ids; for example, \ref{RW} requires
% $\disjoint{{\free(\aReg \GETS \aLoc)}}{{\free(\bLoc \GETS \bExp)}}$.
Redundant load elimination \eqref{RL} follows
from item \ref{1} in the definition of prefixing, taking $\bEv\in\Event$.
Dead store elimination \eqref{DS} follows from Definition \ref{def:cover}, of
$\relfilt{}$.  The independent reorderings (\ref{WW}, \ref{RW} and \ref{RR})
follow from item \ref{item:5} in the definition of prefixing, since no order
is imposed.  Item \ref{item:5} also validates roach-motel reorderings.  For
example, the rules for relaxed writes are as follows:
  \begin{align*}
    \taglabel{AcqW} 
    \sem{x \GETS \aExp \SEMI\aReg \GETS y\ACQ \SEMI\aCmd} &\supseteq
    \sem{\aReg \GETS y\ACQ  \SEMI x\GETS \aExp \SEMI \aCmd} 
    &&\textif x\neq y&&\hbox{}
    \\
    \taglabel{RelW}
    \sem{y\REL \GETS \bExp \SEMI x \GETS \aExp \SEMI \aCmd} &\supseteq
    \sem{x \GETS \aExp \SEMI y\REL \GETS \bExp \SEMI \aCmd}
    &&\textif x\neq y&&\hbox{}
  \end{align*}

  % Reads on the same location are ordered in our model; thus, read optimizations
% are limited by the power of aliasing analysis
% \cite[\textsection2.3]{DBLP:conf/java/Pugh99}.  If $\aLoc\neq\bLoc$, then:
  Since item \ref{item:5} does not impose order between reads of the same
  location (recall \ref{Co2}), \ref{RR} can allow the possibility that
  $\aLoc=\bLoc$.  As a result, read optimizations are not limited by the
  power of aliasing analysis.   By composing \ref{RR} and \ref{RL}, we validate
  \ref{CSE1}, from \textsection\ref{sec:intro}:
\begin{align*}
  \taglabel{CSE}
  \sem{r_1\GETS \aLoc \SEMI
  s\GETS \bLoc \SEMI  
  r_2\GETS \aLoc\SEMI\aCmd}
  \supseteq
  \sem{r_1\GETS \aLoc \SEMI     
    r_2\GETS r_1\SEMI
    s\GETS \bLoc \SEMI\aCmd}
    &&\textif \aReg_2\neq\bReg&&\hbox{}
\end{align*}
%However, this fails if $\aLoc=\bLoc$.


% \begin{displaymathsmall}
%   \sem{r_1\GETS \REF{\aExp} \SEMI
%   s\GETS \REF{\bExp} \SEMI  
%   r_2\GETS \REF{\aExp}}
%   \supseteq
%   \sem{r_1\GETS \REF{\aExp} \SEMI
%   s\GETS \REF{\bExp} \SEMI  
%   r_2\GETS r_1}
% \end{displaymathsmall}
% This holds regardless of whether $\aExp=\bExp$.


% By induction on the length of the pomsets in $\aCmd$, we can use the
% reorderings to establish, more generally, that when $\aCmd$ and $\bCmd$ are
% assignment sequences and $\disjoint{\free(\aCmd)}{\free(\bCmd)}$:
% \begin{gather*}
%   \sem{\aCmd \SEMI \bCmd} = \sem{\bCmd \SEMI \aCmd} 
% \end{gather*}
% Appealing directly to the semantics, we can establish general properties for
% redundant load \ref{RLp}, store forwarding \ref{SFp}, and roach-motel
% \ref{A}, \ref{R}:
% Let $\aCmd$ be
% synchronization-free, with disjoint ids as before:
% \begin{align*}
%   % \taglabelp{RL}
%   % \sem{\aReg \GETS \aLoc  \SEMI \bReg \GETS \aLoc  \SEMI \aCmd} &\supseteq
%   % \sem{\aReg \GETS \aLoc \SEMI \aCmd[\aReg/\bReg]}
%   % \\
%   % \taglabelp{SF} 
%   % \sem{\aLoc \GETS \aExp \SEMI \bReg \GETS \aLoc \SEMI \aCmd} &\supseteq 
%   % \sem{\aLoc \GETS \aExp \SEMI \aCmd[\aExp/\bReg]}  
%   % \\
%   \taglabel{A}
%   \sem{\aCmd \SEMI \bReg \GETS \aLoc \ACQ} &\supseteq
%   \sem{\bReg \GETS \aLoc \ACQ\SEMI  \aCmd}
%   \\
%   \taglabel{R}
%   \sem{\aLoc \REL \GETS \aExp \SEMI \aCmd } &\supseteq
%   \sem{\aCmd \SEMI \aLoc \REL \GETS \aExp }
% \end{align*}
% Roach-motel reorderings that increase the scope of synchronization are valid.
% \begin{lemma}%[Reorderings]
%   Suppose $\cExp\neq\dExp$ is a tautology.  Then the following reorderings hold.
%   \begin{align*}
%     \tag{R-Acq}
%     \sem{\aReg \GETS \REF{\cExp} \SEMI \bReg \GETS \REF{\dExp}\ACQ \SEMI \aCmd} &\supseteq
%     \sem{\bReg \GETS \REF{\dExp}\ACQ\SEMI \aReg \GETS \REF{\cExp} \SEMI \aCmd}
% &&\textif \aReg\notin\free(\dExp) \textand \bReg\notin\free(\cExp)
%     \\
%     \tag{W-Acq} 
%     \sem{\REF{\dExp} \GETS \bExp \SEMI\aReg \GETS \REF{\cExp}\ACQ \SEMI\aCmd} &\supseteq
%     \sem{\aReg \GETS \REF{\cExp}\ACQ  \SEMI \REF{\dExp}\GETS \bExp \SEMI \aCmd} 
%     &&\textif \aReg\notin\free(\dExp) \cup\free(\bExp)
%     \\
% \tag{Rel-R} 
% \sem{\REF{\dExp}\REL \GETS \bExp \SEMI\aReg \GETS \REF{\cExp} \SEMI \aCmd} &\supseteq
%     \sem{ \aReg \GETS \REF{\cExp} \SEMI \REF{\dExp}\REL \GETS \bExp \SEMI \aCmd}
%    &&\textif \aReg\notin\free(\dExp) \cup\free(\bExp)
%     \\
%     \tag{Rel-W}
%     \sem{\REF{\cExp}\REL \GETS \aExp \SEMI \REF{\dExp} \GETS \bExp \SEMI \aCmd} &\supseteq
%     \sem{\REF{\dExp} \GETS \bExp \SEMI \REF{\cExp}\REL \GETS \aExp \SEMI \aCmd}
%   \end{align*}
% \begin{proof}
% The proof follows from noticing that the pomsets in the semantics of the right hand sides are augmentations of a pomset on the left hand side.  
% \end{proof}
% \end{lemma}

% Again, we suppose that $\aCmd$ is an assignment sequence and disjoint
% identifiers.
% \ref{A} and \ref{R} require
% disjoint names, as in \ref{WW} and
% \ref{RW}. In addition, \ref{A} and \ref{R} require that $\aCmd$ is
% synchronization-free.


Many laws hold for the conditional.  We show dead code elimination \eqref{DC}
and case analysis \eqref{CA}, which follows from disjunction closure
(Definition \ref{def:dis}).
\begin{align*}
  \taglabel{DC}
  \sem{\IF{\aExp}\THEN\aCmd\ELSE\bCmd\FI} &=
  \sem{\aCmd}
  \qquad \textif \aExp \text{ is a tautology}
  \\
  \taglabel{CA}
  \sem{\IF{\aExp}\THEN\aCmd\ELSE\aCmd\FI} &=
  \sem{\aCmd}
\end{align*}
% \ref{DC} requires that $\aExp$ be a tautology.

% Suppose $\cExp\neq\dExp$ is a tautology.  Then the following 
%reorderings hold.
%  \begin{align*}
% \begin{align*}
%    \tag{R-R}
%    \sem{\aReg \GETS \REF{\cExp} \SEMI \bReg \GETS \REF{\dExp} 
%\SEMI \aCmd} &=
%    \sem{\bReg \GETS \REF{\dExp} \SEMI \aReg \GETS \REF{\cExp} 
%\SEMI \aCmd}
%    &&\textif \aReg\notin\free(\dExp) \textand \bReg\notin\free(\cExp)
 %   \\
 %   \tag{R-W}
 %   \sem{\aReg \GETS \REF{\cExp} \SEMI \REF{\dExp} \GETS \bExp 
%\SEMI \aCmd} &=
%    \sem{\REF{\dExp} \GETS \bExp \SEMI \aReg \GETS \REF{\cExp} 
%\SEMI \aCmd}
%    &&\textif \aReg\notin\free(\dExp) \cup\free(\bExp)
%    \\
 %   \tag{W-W}
 %   \sem{\REF{\cExp} \GETS \aExp \SEMI \REF{\dExp} \GETS \bExp \SEMI 
%\aCmd} &=
 %   \sem{\REF{\dExp} \GETS \bExp \SEMI \REF{\cExp} \GETS \aExp \SEMI 
%\aCmd}
%  \end{align*}
% \begin{proof}
% The semantics of sequential composition $\aCmd \SEMI \bCmd$, where 
% the only enforced $\lt$ relationships come from conflict on locations or 
%release or acquire actions.    The roach-motel reorderings that increase 
%the scope of synchronization are valid because the pomsets in the 
%semantics of the right hand sides are augmentations of a pomset on the 
%left hand side. 
%\end{proof}
%\end{lemma}


% \begin{lemma}%[Reorderings]
% Suppose $\cExp\neq\dExp$ is a tautology.  Then the following 
%reorderings hold.
 % \begin{align*}
%    \tag{R-Acq}
 %   \sem{\aReg \GETS \REF{\cExp} \SEMI \bReg \GETS \REF{\dExp}\ACQ 
%\SEMI \aCmd} &\supseteq
 %   \sem{\bReg \GETS \REF{\dExp}\ACQ\SEMI \aReg \GETS \REF{\cExp} 
%\SEMI \aCmd}
% &&\textif \aReg\notin\free(\dExp) \textand \bReg\notin\free(\cExp)
 %   \\
 %   \tag{W-Acq} 
 %   \sem{\REF{\dExp} \GETS \bExp \SEMI\aReg \GETS \REF{\cExp}\ACQ 
%\SEMI\aCmd} &\supseteq
%    \sem{\aReg \GETS \REF{\cExp}\ACQ  \SEMI \REF{\dExp}\GETS \bExp 
%\SEMI \aCmd} 
%    &&\textif \aReg\notin\free(\dExp) \cup\free(\bExp)
 %   \\
% \tag{Rel-R} 
%\sem{\REF{\dExp}\REL \GETS \bExp \SEMI\aReg \GETS \REF{\cExp} 
%\SEMI \aCmd} &\supseteq
 %   \sem{ \aReg \GETS \REF{\cExp} \SEMI \REF{\dExp}\REL \GETS \bExp 
%\SEMI \aCmd}
%   &&\textif \aReg\notin\free(\dExp) \cup\free(\bExp)
 %   \\
 %   \tag{Rel-W}
 %   \sem{\REF{\cExp}\REL \GETS \aExp \SEMI \REF{\dExp} \GETS \bExp %\SEMI \aCmd} &\supseteq
 %   \sem{\REF{\dExp} \GETS \bExp \SEMI \REF{\cExp}\REL \GETS \aExp 
%\SEMI \aCmd}
%  \end{align*}
%\begin{proof}
%The proof follows from noticing that the pomsets in the semantics of the 
%right hand sides are augmentations of a pomset on the left hand side.  
%\end{proof}
%\end{lemma}

% \paragraph*{Compiler optimizations.} Reordering and peephole optimizations
% can be combined to describe common compiler optimizations.  We illustrate
% using common subexpression elimination,
% following~\citet{Dolan:2018:BDR:3192366.3192421}:
% Consider the command 
% \begin{math}
%   (\aReg \GETS \aLoc *2  \SEMI \aCmd \SEMI \bReg \GETS \aLoc * 2)
% \end{math}
% where $\aCmd$ is independent of $\aReg$.  Reordering yields
% \begin{math}
%   (\aCmd \SEMI \aReg \GETS \aLoc *2  \SEMI  \bReg \GETS \aLoc * 2),
% \end{math}
% followed by redundant load to yield
% \begin{math}
%   (\aCmd \SEMI \aReg \GETS \aLoc * 2 \SEMI  \bReg \GETS \aReg).
% \end{math}

% Similarly, the treatment of loop-invariant code motion, dead-store
% elimination and constant propagation
% from~\citet{Dolan:2018:BDR:3192366.3192421} follow.

% Since our model is more generous about permitted reorderings, we 
% permit optimizations that they forbid.  Consider:
%\begin{math}
 % (\aReg \GETS \aLoc \SEMI \bLoc \GETS \cLoc  \SEMI \aLoc \GETS 
%\aReg).
%\end{math}
%Reordering, permitted by us, but forbidden by them, yields
%\begin{math}
%  (\aReg \GETS \aLoc \SEMI \aLoc \GETS \aReg \SEMI \bLoc \GETS 
%\cLoc),
%\end{math}
%followed by the valid elimination of redundant load
%\begin{math}
%  (\aReg \GETS \aLoc \SEMI \aLoc \GETS \aReg \SEMI \bLoc \GETS %
%\cLoc).
%\end{math}

As expected, %sequential and
parallel composition commutes with conditionals and declarations, and
conditionals and declarations commute with each other.  We discussed scope
extrusion on page \pageref{page:extrusion}.
%For example,  % We show sequential scope extrusion \ref{SSE}, which concerns
% sequential composition and location binding:
% \begin{math}
%   % \taglabel{PSE}
%   \sem{\aCmd\PAR \VAR\aLoc\SEMI\bCmd}\allowbreak =
%   \sem{\VAR\aLoc\SEMI(\aCmd\PAR\bCmd)}
%   % \\
%   % \taglabel{SSE}
%   % \sem{\aCmd\SEMI \VAR\aLoc\SEMI\bCmd}\allowbreak &=
%   % \sem{\VAR\aLoc\SEMI(\aCmd\SEMI\bCmd)}
%   % \\
%   % \taglabel{CSE}
%   % \sem{\IF{\aExp}\THEN\aCmd\ELSE \VAR\aLoc\SEMI\bCmd\FI}\allowbreak &=
%   % \sem{\VAR\aLoc\SEMI \IF{\aExp}\THEN\aCmd\ELSE\bCmd\FI}
% \end{math}
% if $\aLoc$ does not appear in $\aCmd$.

%\ref{SSE} requires that $\aLoc$ does not appear in $\aCmd$.

\paragraph{Irrelevant Reads}
In our work on microarchitecture \citep{2019-sp}, irrelevant relaxed read
actions could be observed using cache effects.  Thus far, we have maintained
this perspective---for example, $\sem{r\GETS x}\not\supseteq\sem{\SKIP}$
since $\sem{\SKIP}$ has a pomset with only the terminal action, and there is
no such pomset in $\sem{r\GETS x}$.
% $\not\supseteq\sem{r\GETS x\SEMI r\GETS{}x}$.
This inequation holds even though there is no context in the
language of this paper that can distinguish these programs.

Some traditional optimizations are only valid ``modulo irrelevant reads.'' In
order to characterize these, we define \emph{read saturation} as follows:
%
% \labeltext{Let}{page:readsat} $\readc(\aPSS)$ be the set $\aPSS'$ where
% $\aPS'\in\aPSS'$ when $\exists\aPS\in\aPSS$ and $\exists D$ such that
% $\Event'= \Event\uplus D$, ${\le'} \supseteq{\le}$,
% ${\labeling'} \supseteq{\labeling}$, and for every $\bEv\in D$ there are
% $\aLoc$ and $\aVal$ such that
% $\labelingAct'(\bEv)=(\DR[\modeRLX]{\aLoc}{\aVal})$.
%
\labeltext{Let}{page:readsat} $\readc(\aPSS)$ be the set $\aPSS'$ where
$\aPS'\in\aPSS'$ when $\exists\aPS\in\aPSS$ and there exist
$\Event''\subset\Event$ and $D$ such that $\Event'= \Event''\uplus D$,
${\le'}\supseteq{\le\restrict{\Event''}}$,
${\labeling'} \supseteq{\labeling\restrict{\Event''}}$, and for every
$\bEv\in D$ there are $\aLoc$ and $\aVal$ such that
$\labelingAct'(\bEv)=(\DR[\modeRLX]{\aLoc}{\aVal})$.
%
% and
% %$\aPS[\aVal/\aLoc]$ is consistent.
% $\bigwedge_{\aEv\in\Event}\labelingForm(\aEv)[\aVal/\aLoc]$ is satisfiable.
% The last requirement ensures consistency with item \ref{4c} in the
% definition of prefixing.
%
Note that if $\aPSS'\supseteq\aPSS$, then
$\readc(\aPSS')\supseteq \readc(\aPSS)$.
We have:
\begin{align*}
  \taglabel{SF}
  \readc\sem{\aLoc \GETS \aExp \SEMI \aReg  \GETS \aLoc\SEMI\aCmd} &\supseteq 
  \readc\sem{\aLoc \GETS \aExp \SEMI \aReg  \GETS \aExp\SEMI\aCmd}  
  \\
  \taglabel{IRE}
  \readc\sem{\aReg  \GETS \aLoc\SEMI\aCmd} & \supseteq
  \readc\sem{\aCmd}  
  &&\textif \aReg\not\in\free(\aCmd)&&\hbox{}
\end{align*}
Like \ref{RL}, store forwarding \eqref{SF} follows from item \ref{1} in the
definition of prefixing, taking $\bEv\in\Event$.  Note that this inclusion
does not hold in the plain semantics: for example,
$\sem{\aLoc \GETS \aExp \SEMI \aReg \GETS \aExp}$ has a pomset containing
only a write and the terminal action, whereas
$\sem{\aLoc \GETS \aExp \SEMI \aReg \GETS \aLoc}$ contains no such pomset.

Irrelevant read elimination \eqref{IRE} follows from the definition
of $\readc$ and prefixing.

Irrelevant read introduction fails generally, since $\sem{\SKIP}$ has pomset
with only the termination action, whereas
.  Consider the following example
\cite[\textsection1.4.5]{SevcikThesis}:
\begin{align*}
  \readc\sem{\IF{\aReg} \THEN \bReg \GETS \aLoc \SEMI \bLoc \GETS \bReg \FI}
  &=
  \readc\sem{\bReg \GETS \aLoc \SEMI \IF{\aReg} \THEN \bLoc \GETS \bReg \FI}
\end{align*}
The right-hand program is derived from the left by introducing an irrelevant
read in the else-branch, and then moving the common code out of the
conditional.  Read saturation allows us to add actions of the form
$(\DR{x}{v})$ to the left-hand pomsets and $(r\neq0\mid\DR{x}{v})$ to the
right, thus equating them.


% \ref{RL} and \ref{SF} show that redundant accesses can be removed.  The same
% does not hold for \emph{non-redundant} access:
% $\sem{r\GETS x}\not\supseteq\sem{\SKIP}$ (consider a completed pomsets for
% $\SKIP$).
% % , since $\sem{\SKIP}$ includes the
% % pomset with a single event, labeled $\DSTOP$.


\paragraph{Invalid Rewrites}
%\label{sec:invalid}

Relevant read introduction is invalid:
\begin{align*}
  %\tag{Read-Intro-Invalid}
  \sem{\aReg \GETS \aLoc \SEMI \IF{\aReg {\neq} \aReg} \THEN \cLoc \GETS 1 \FI}
  &\not\supseteq
  \sem{\aReg \GETS \aLoc \SEMI \bReg \GETS \aLoc  \SEMI \IF{\aReg {\neq}\bReg} \THEN \cLoc \GETS 1 \FI}
\end{align*}
These are distinguished by the context
\begin{math}
  \hole{} \PAR x\GETS1\PAR x\GETS2.
\end{math}

% But even \emph{irrelevant} read introduction is invalid.
% Consider the following example \cite[\textsection1.4.5]{SevcikThesis}:
% \begin{align*}
%   \sem{\IF{\aReg} \THEN \bReg \GETS \aLoc \SEMI \bLoc \GETS \bReg \FI}
%   &\not\supseteq
%   \sem{\bReg \GETS \aLoc \SEMI \IF{\aReg} \THEN \bLoc \GETS \bReg \FI}
% \end{align*}
% This inequation holds even though there is no context in our
% language that distinguishes these programs.


\eqref{DS} shows that redundant writes can be eliminated.  General write
elimination is, of course, invalid; for example
\begin{math}
  \sem{\aLoc\GETS 1} 
  \not\supseteq
  \sem{\SKIP}.
\end{math}
Write introduction is also invalid:
\begin{align*}
  \sem{\aLoc \GETS 1} 
  &\not\supseteq
  \sem{\aLoc \GETS 1 \SEMI \aLoc \GETS 1}
\end{align*}
Observationally, these are distinguished by the context
\begin{math}
  \hole{} \PAR
  r\GETS x \SEMI
  x\GETS2 \SEMI
  s\GETS x\SEMI
  \IF{\aReg {=} \bReg} \THEN \cLoc \GETS 1 \FI.
\end{math}


% But even \emph{irrelevant} read introduction is invalid:
% $\sem{\SKIP}\not\supseteq\sem{r\GETS x}\not\supseteq\sem{r\GETS x\SEMI r\GETS x}$.  These inequations hold even though there is no context in our
% language that distinguishes these programs.
% % With weaker notions of coherence
% % \cite{Manson:2005:JMM:1047659.1040336}, these
% % commands are indistinguishable.



We discussed the invalidity of variable renaming on page
\pageref{page:extrusion} and thread inlining on page
\pageref{page:inlining}.


% // p and q might be aliased
% int i = p
% // concurrent write to p.x by another thread
% int j = q 
% int k = p

% Since p and q only might be aliased, but are not definitely aliased, then the
% use of q cannot be optimized away (if it were known that p and q pointed to
% the same object, then it would be legal to replace the assignments to j and k
% with assignments of the value of i). Consider the case where p and q are in
% fact aliased, and another thread writes to the memory location for p/q
% between the first use of p and the use of q; the use of q will see the
% new value. It will be illegal for the second use of p (stored into k) to
% get the same value as was stored into i. However, a fairly standard compiler
% optimization would involve eliminating the getfield for k and replacing it
% with a reuse of the value stored into i. Un- fortunately, that optimization
% is illegal in any language that requires Coherence.

% One way to think of it is that since a read of a memory location may cause
% the thread to become aware of a write by another thread, it must be treated
% in the compiler as a possible write.



\citet{BoehmOOTA} \labeltext{considers}{page:rfub} the following programs:
\begin{gather*}
  \tag{RFUB}\label{RFUB}
  \sem{r\GETS y\SEMI x\GETS r}
  \not\supseteq
  \sem{r\GETS y\SEMI \IF{r \NOTEQ 1} \THEN z\GETS 1\SEMI r\GETS 1\FI \SEMI x\GETS r}
\end{gather*}
The left command is half of \ref{OOTA3}, from
\textsection\ref{sec:logic}.  The right command is dubbed \rfub{}, for
\emph{Register assignment From an Unexecuted Branch}.
\citeauthor{BoehmOOTA} observes that in the context $x\GETS y \PAR \hole{}$,
these programs have different behaviors.  Yet the \oota{} example on the left
never writes $1$.  Why should the unexecuted branch change that?  As it turns
out, both branches of the conditional in \ref{RFUB} can execute, since the write
to $x$ is independent of the read from $y$.  Considering just the two threads
above, we have $\hoare{\TRUE}{\text{\rfub}}{x=1}$, but not
$\hoare{\TRUE}{\text{\oota}}{x=1}$.  As a result, it is expected that \ref{RFUB}
may have additional behaviors.  The change in the thread from \ref{OOTA3} to
\ref{RFUB} is not a valid refinement under Hoare logic and thus it is not valid
in our semantics.

% Let $\aCmd$ be the right
% thread in \ref{rfub}.
% \begin{align*}
%   \aCmd = r\GETS y\SEMI \IF{r \NOTEQ 1} \THEN z\GETS 1\SEMI r\GETS 1\FI  \SEMI x\GETS r 
% \end{align*}
% Then, $\hoare{\TRUE}{\aCmd}{x=1} $ and $\hoare{y \neq 1}{\bCmd}{z=1}$.  Thus, the traditional sequential semantics {\em does not} attribute any cause for the write of $1$ to $x$, as seen by the execution:
% \begin{tikzdisplay}[node distance=1em]
%  \event{ry}{\DR{y}{1}}{}
%  \event{wa}{\DW{z}{1}}{right=of ry}
%  \event{wx}{\DW{x}{1}}{right=of wa}
%  \rf{ry}{wa}
% \end{tikzdisplay}
% % Consider a fragment from example~\ref{rfub}.
% % \begin{align*}
% % \bCmd:  r \GETS y \SEMI \IF{r \NOTEQ 42} \THEN a \GETS 1 \SEMI s \GETS 42 \FI \SEMI x \GETS 42
% % \end{align*}
% % Then, $\hoare{\TRUE}{\bCmd}{x=42} $ and $\hoare{y \neq 42}{\bCmd}{a= 1} $.  Thus, the traditional sequential semantics {\em does not} attribute any cause for the write of $42$ to $x$, as seen by the execution:
% % \begin{tikzdisplay}[node distance=1em]
% %  \event{ry}{\DR{y}{41}}{}
% %  \event{wx}{\DW{x}{42}}{right=of ry}
% %  \event{wa}{\DW{a}{1}}{below=of rx}
% %  \rf{ry}{wa}
% % \end{tikzdisplay}
% Consider the
% following, where all locations are initialized to $0$.  In \textsection\ref{sec:logic} we provide machinery to prove that \oota{} is
% incapable of writing $1$.  The question is whether this should be possible
% for \rfub, which changes \oota{} only to include a \emph{Register assignment
%   From an Unexecuted Branch} \cite{BoehmOOTA}:?  
% \begin{align*}
%   \label{oota}  \tag{\textsc{oota}}
%   y\GETS x
%   \PAR&
%   r\GETS y\SEMI
%   x\GETS r
%   \\
%   \tag{\rfub1}
%   \label{rfub}
%   y\GETS x
%   \PAR&
%   r\GETS y\SEMI
%   \IF{r \NOTEQ 1} \THEN z\GETS 1\SEMI r\GETS 1\FI  \SEMI x\GETS r 
% \end{align*}
% %and the following \emph{Out Of Thin Air} litmus test:



