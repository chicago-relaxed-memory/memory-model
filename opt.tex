\section{Single-Threaded Optimizations}
\label{sec:opt}

When $\sem{\aCmd} \supseteq \sem{\aCmd'}$, we say that $\aCmd'$ is a
\emph{valid transformation} of $\aCmd$.  In this section, we show the
validity of specific optimizations and discuss ``optimizations'' that fail,
such as \ref{RFUB}.

\paragraph{Valid Rewrites.}
%\label{sec:valid}

% To enable reasoning about program fragments, transformation validity must be
% preserved by \emph{contexts}.  In \textsection\ref{sec:model}, we defined the
% semantics by prefixing one action at a time.  This helps to make the
% semantics understandable, but it also creates impoverished contexts.

% To allow for richer contexts, we appeal to the alternate presentation of the
% language given in \textsection\ref{sec:semicolon}.  We refactor the syntax
% of commands and define contexts:
% \begin{align*}
%   \aCmd,\,\bCmd
%   \BNFDEF& \SKIP
%   \mkern-2mu\BNFSEP\mkern-2mu \FENCE^{\fmode}
%   \mkern-2mu\BNFSEP\mkern-2mu \aReg\GETS\aExp
%   % \mkern-2mu\BNFSEP\mkern-2mu \aReg\GETS \aLoc^{\amode} 
%   % \mkern-2mu\BNFSEP\mkern-2mu \aLoc^{\amode}\GETS\aExp
%   \mkern-2mu\BNFSEP\mkern-2mu \aReg\GETS \REF{\cExp}^{\amode} 
%   \mkern-2mu\BNFSEP\mkern-2mu \REF{\cExp}^{\amode}\GETS\aExp
%   \\[-.5ex]
%   \BNFSEP&\aCmd \PAR \bCmd
%   \mkern-2mu\BNFSEP\mkern-2mu\aCmd \SEMI \bCmd
%   \mkern-2mu\BNFSEP\mkern-2mu \VAR\aLoc\SEMI \aCmd
%   \mkern-2mu\BNFSEP\mkern-2mu \IF{\aExp} \THEN \aCmd \ELSE \bCmd \FI
%   \\
%   \aCtxt,\,\bCtxt
%   \BNFDEF& \hole{}
%   \mkern-2mu\BNFSEP\mkern-2mu \aCtxt \PAR \bCmd
%   \mkern-2mu\BNFSEP\mkern-2mu \aCmd \PAR \bCtxt
%   \mkern-2mu\BNFSEP\mkern-2mu \aCtxt \SEMI \bCmd
%   \mkern-2mu\BNFSEP\mkern-2mu \aCmd \SEMI \bCtxt
%   \mkern-2mu\BNFSEP\mkern-2mu \VAR\aLoc\SEMI \aCtxt
%   \\[-.5ex]
%   \BNFSEP& \IF{\aExp} \THEN \aCtxt \ELSE \bCmd \FI
%   \mkern-2mu\BNFSEP\mkern-2mu \IF{\aExp} \THEN \aCmd \ELSE \bCtxt \FI
% \end{align*}


% \begin{lemma}%\label{freadssatcomp}[Compositionality of $\freadsat$]
%   Let $\bCtxt$ be a context
%   % \footnote{%
%   %   The results of this section hold for contexts of the example language
%   %   given in \textsection\ref{sec:model} and extended in \textsection\ref{sec:variants}.
%   %   % \begin{math}
%   %   %   \begin{array}[t]{rcl}
%   %   %     \aCtxt,\,\bCtxt
%   %   %     &\BNFDEF& \hole{}
%   %   %     \BNFSEP \aReg\GETS\aExp\SEMI \aCtxt
%   %   %     \BNFSEP \aReg\GETS\aLoc^{\amode}\SEMI \aCtxt 
%   %   %     \BNFSEP \aLoc^{\amode}\GETS\aExp\SEMI \aCtxt
%   %   %     \\[-.5ex]
%   %   %     &\BNFSEP&\aCtxt \PAR \bCmd
%   %   %     \BNFSEP \aCmd \PAR \bCtxt
%   %   %     \BNFSEP \VAR\aLoc\SEMI \aCtxt
%   %   %     \\[-.5ex]
%   %   %     &\BNFSEP& \IF{\aExp} \THEN \aCtxt \ELSE \bCmd \FI
%   %   %     \BNFSEP \IF{\aExp} \THEN \aCmd \ELSE \bCtxt \FI
%   %   %   \end{array}
%   %   % \end{math}
%   %   The results of this section hold for contexts of the example language
%   %   given in \textsection\ref{sec:model} and extended in \textsection\ref{sec:variants}.
%   %   The results also hold for the more general contexts of
%   %   \textsection\ref{sec:semicolon}, which includes full sequential composition:
%   %   \begin{displaymath}
%   %     \begin{array}[t]{rcl}
%   %       \aCtxt,\,\bCtxt
%   %       &\BNFDEF& \hole{}
%   %       \BNFSEP \aCtxt \SEMI \bCmd
%   %       \BNFSEP \aCmd \SEMI \bCtxt
%   %       \BNFSEP \aCtxt \PAR \bCmd
%   %       \BNFSEP \aCmd \PAR \bCtxt
%   %       \BNFSEP 
%   %       \\[-.5ex]
%   %       &\BNFSEP& \VAR\aLoc\SEMI \aCtxt
%   %       \BNFSEP \IF{\aExp} \THEN \aCtxt \ELSE \bCmd \FI
%   %       \BNFSEP \IF{\aExp} \THEN \aCmd \ELSE \bCtxt \FI
%   %     \end{array}
%   %   \end{displaymath}}
% and $\readc\sem{\aCmd} \supseteq \readc\sem{\aCmd'}$:
% \begin{displaymath}
%   \readc\sem{\bCtxt\hole{\aCmd}} \supseteq \readc\sem{\bCtxt\hole{\aCmd'}}
% \end{displaymath}
% \end{lemma}

To discuss valid transformations without getting lost in notation, we present
them using simple locations, rather than calculated addresses.  The extension
is simple: For address expressions $\REF{\aExp}$ and $\REF{\bExp}$, replace
$\aLoc=\bLoc$ by provable equality of $\aExp$ and $\bExp$, and
$\aLoc\neq\bLoc$ by provable inequality.  Operations on
sets can be defined similarly.
%
Let $\free(\aCmd)$ be the set of locations and registers that occur in $\aCmd$.

The semantics validates several peephole optimizations.
\begin{align*}
  \taglabel{RL}
  \sem{\aReg \GETS \aLoc \SEMI \bReg  \GETS \aLoc\SEMI\aCmd} &\supseteq 
  \sem{\aReg \GETS \aLoc \SEMI \bReg  \GETS \aReg\SEMI\aCmd}
  \\
  \taglabel{SF}
  \sem{\aLoc \GETS \aExp \SEMI \bReg  \GETS \aLoc\SEMI\aCmd} &\supseteq 
  \sem{\aLoc \GETS \aExp \SEMI \bReg  \GETS \aExp\SEMI\aCmd}
  \\
  \taglabel{DS}
  \sem{\aLoc \GETS \aExp \SEMI \aLoc  \GETS \bExp\SEMI\aCmd} &\supseteq 
  \sem{\aLoc \GETS \bExp\SEMI\aCmd}    
  \\
  \taglabel{WW}
  \sem{\aLoc \GETS \aExp \SEMI \bLoc  \GETS \bExp\SEMI\aCmd} &=
  \sem{\bLoc  \GETS \bExp\SEMI \aLoc \GETS \aExp\SEMI\aCmd} &&\text{if } \aLoc\neq\bLoc
  \\
  \taglabel{RW}
  \sem{\aReg \GETS \aLoc \SEMI \bLoc  \GETS \bExp\SEMI\aCmd} &=
  \sem{\bLoc  \GETS \bExp\SEMI \aReg \GETS \aLoc\SEMI\aCmd} &&\text{if } \disjoint{{\free(\aReg \GETS \aLoc)}}{{\free(\bLoc \GETS \bExp)}}
  \\
  \taglabel{RR}
  \sem{\aReg \GETS \aLoc \SEMI \bReg  \GETS \bLoc\SEMI\aCmd} &=
  \sem{\bReg  \GETS \bLoc\SEMI \aReg \GETS \aLoc\SEMI\aCmd} &&\text{if } \aReg\neq\bReg
\end{align*}
% \ref{WW}, \ref{RW} and \ref{RR} require that two sides of the semicolon
% have disjoint ids; for example, \ref{RW} requires $\disjoint{{\free(\aReg
%     \GETS \aLoc)}}{{\free(\bLoc \GETS \bExp)}}$. 
% \ref{RR} requires either $\aReg\neq\bReg$ or
%   $\aLoc=\bLoc$.  \ref{WW} and \ref{RW} require that two sides of the
% semicolon have disjoint ids; for example, \ref{RW} requires
% $\disjoint{{\free(\aReg \GETS \aLoc)}}{{\free(\bLoc \GETS \bExp)}}$.
Redundant load elimination \eqref{RL} and store forwarding \eqref{SF} follow
from item \ref{1} in the definition of prefixing, taking $\bEv\in\Event$.
Dead store elimination \eqref{DS} follows from Definition \ref{def:cover}, of
$\relfilt{}$.  The independent reorderings (\ref{WW}, \ref{RW} and \ref{RR})
follow from item \ref{item:5} in the definition of prefixing, since no order
is imposed.  Item \ref{item:5} also validates roach-motel reorderings.  For
example, the rules for relaxed writes are as follows:
  \begin{align*}
    \taglabel{AcqW} 
    \sem{x \GETS \aExp \SEMI\aReg \GETS y\ACQ \SEMI\aCmd} &\supseteq
    \sem{\aReg \GETS y\ACQ  \SEMI x\GETS \aExp \SEMI \aCmd} 
    &&\textif x\neq y&&\hbox{}
    \\
    \taglabel{RelW}
    \sem{y\REL \GETS \bExp \SEMI x \GETS \aExp \SEMI \aCmd} &\supseteq
    \sem{x \GETS \aExp \SEMI y\REL \GETS \bExp \SEMI \aCmd}
    &&\textif x\neq y&&\hbox{}
  \end{align*}

  % Reads on the same location are ordered in our model; thus, read optimizations
% are limited by the power of aliasing analysis
% \cite[\textsection2.3]{DBLP:conf/java/Pugh99}.  If $\aLoc\neq\bLoc$, then:
  Since item \ref{item:5} does not impose order between reads of the same
  location (recall \ref{Co2}), \ref{RR} can allow the possibility that
  $\aLoc=\bLoc$.  As a result, read optimizations are not limited by the
  power of aliasing analysis.   By composing \ref{RR} and \ref{RL}, we validate
  \ref{CSE1}, from \textsection\ref{sec:intro}:
\begin{align*}
  \taglabel{CSE}
  \sem{r_1\GETS \aLoc \SEMI
  s\GETS \bLoc \SEMI  
  r_2\GETS \aLoc\SEMI\aCmd}
  \supseteq
  \sem{r_1\GETS \aLoc \SEMI     
    r_2\GETS r_1\SEMI
    s\GETS \bLoc \SEMI\aCmd}
    &&\textif \aReg_2\neq\bReg&&\hbox{}
\end{align*}
%However, this fails if $\aLoc=\bLoc$.


% \begin{displaymathsmall}
%   \sem{r_1\GETS \REF{\aExp} \SEMI
%   s\GETS \REF{\bExp} \SEMI  
%   r_2\GETS \REF{\aExp}}
%   \supseteq
%   \sem{r_1\GETS \REF{\aExp} \SEMI
%   s\GETS \REF{\bExp} \SEMI  
%   r_2\GETS r_1}
% \end{displaymathsmall}
% This holds regardless of whether $\aExp=\bExp$.


% By induction on the length of the pomsets in $\aCmd$, we can use the
% reorderings to establish, more generally, that when $\aCmd$ and $\bCmd$ are
% assignment sequences and $\disjoint{\free(\aCmd)}{\free(\bCmd)}$:
% \begin{gather*}
%   \sem{\aCmd \SEMI \bCmd} = \sem{\bCmd \SEMI \aCmd} 
% \end{gather*}
% Appealing directly to the semantics, we can establish general properties for
% redundant load \ref{RLp}, store forwarding \ref{SFp}, and roach-motel
% \ref{A}, \ref{R}:
% Let $\aCmd$ be
% synchronization-free, with disjoint ids as before:
% \begin{align*}
%   % \taglabelp{RL}
%   % \sem{\aReg \GETS \aLoc  \SEMI \bReg \GETS \aLoc  \SEMI \aCmd} &\supseteq
%   % \sem{\aReg \GETS \aLoc \SEMI \aCmd[\aReg/\bReg]}
%   % \\
%   % \taglabelp{SF} 
%   % \sem{\aLoc \GETS \aExp \SEMI \bReg \GETS \aLoc \SEMI \aCmd} &\supseteq 
%   % \sem{\aLoc \GETS \aExp \SEMI \aCmd[\aExp/\bReg]}  
%   % \\
%   \taglabel{A}
%   \sem{\aCmd \SEMI \bReg \GETS \aLoc \ACQ} &\supseteq
%   \sem{\bReg \GETS \aLoc \ACQ\SEMI  \aCmd}
%   \\
%   \taglabel{R}
%   \sem{\aLoc \REL \GETS \aExp \SEMI \aCmd } &\supseteq
%   \sem{\aCmd \SEMI \aLoc \REL \GETS \aExp }
% \end{align*}
% Roach-motel reorderings that increase the scope of synchronization are valid.
% \begin{lemma}%[Reorderings]
%   Suppose $\cExp\neq\dExp$ is a tautology.  Then the following reorderings hold.
%   \begin{align*}
%     \tag{R-Acq}
%     \sem{\aReg \GETS \REF{\cExp} \SEMI \bReg \GETS \REF{\dExp}\ACQ \SEMI \aCmd} &\supseteq
%     \sem{\bReg \GETS \REF{\dExp}\ACQ\SEMI \aReg \GETS \REF{\cExp} \SEMI \aCmd}
% &&\textif \aReg\notin\free(\dExp) \textand \bReg\notin\free(\cExp)
%     \\
%     \tag{W-Acq} 
%     \sem{\REF{\dExp} \GETS \bExp \SEMI\aReg \GETS \REF{\cExp}\ACQ \SEMI\aCmd} &\supseteq
%     \sem{\aReg \GETS \REF{\cExp}\ACQ  \SEMI \REF{\dExp}\GETS \bExp \SEMI \aCmd} 
%     &&\textif \aReg\notin\free(\dExp) \cup\free(\bExp)
%     \\
% \tag{Rel-R} 
% \sem{\REF{\dExp}\REL \GETS \bExp \SEMI\aReg \GETS \REF{\cExp} \SEMI \aCmd} &\supseteq
%     \sem{ \aReg \GETS \REF{\cExp} \SEMI \REF{\dExp}\REL \GETS \bExp \SEMI \aCmd}
%    &&\textif \aReg\notin\free(\dExp) \cup\free(\bExp)
%     \\
%     \tag{Rel-W}
%     \sem{\REF{\cExp}\REL \GETS \aExp \SEMI \REF{\dExp} \GETS \bExp \SEMI \aCmd} &\supseteq
%     \sem{\REF{\dExp} \GETS \bExp \SEMI \REF{\cExp}\REL \GETS \aExp \SEMI \aCmd}
%   \end{align*}
% \begin{proof}
% The proof follows from noticing that the pomsets in the semantics of the right hand sides are augmentations of a pomset on the left hand side.  
% \end{proof}
% \end{lemma}

% Again, we suppose that $\aCmd$ is an assignment sequence and disjoint
% identifiers.
% \ref{A} and \ref{R} require
% disjoint names, as in \ref{WW} and
% \ref{RW}. In addition, \ref{A} and \ref{R} require that $\aCmd$ is
% synchronization-free.


Many laws hold for the conditional.  We show dead code elimination \eqref{DC}
and case analysis \eqref{CA}, which follows from disjunction closure
(Definition \ref{def:dis}).
\begin{align*}
  \taglabel{DC}
  \sem{\IF{\aExp}\THEN\aCmd\ELSE\bCmd\FI} &=
  \sem{\aCmd}
  \qquad \textif \aExp \text{ is a tautology}
  \\
  \taglabel{CA}
  \sem{\IF{\aExp}\THEN\aCmd\ELSE\aCmd\FI} &=
  \sem{\aCmd}
\end{align*}
% \ref{DC} requires that $\aExp$ be a tautology.

% Suppose $\cExp\neq\dExp$ is a tautology.  Then the following 
%reorderings hold.
%  \begin{align*}
% \begin{align*}
%    \tag{R-R}
%    \sem{\aReg \GETS \REF{\cExp} \SEMI \bReg \GETS \REF{\dExp} 
%\SEMI \aCmd} &=
%    \sem{\bReg \GETS \REF{\dExp} \SEMI \aReg \GETS \REF{\cExp} 
%\SEMI \aCmd}
%    &&\textif \aReg\notin\free(\dExp) \textand \bReg\notin\free(\cExp)
 %   \\
 %   \tag{R-W}
 %   \sem{\aReg \GETS \REF{\cExp} \SEMI \REF{\dExp} \GETS \bExp 
%\SEMI \aCmd} &=
%    \sem{\REF{\dExp} \GETS \bExp \SEMI \aReg \GETS \REF{\cExp} 
%\SEMI \aCmd}
%    &&\textif \aReg\notin\free(\dExp) \cup\free(\bExp)
%    \\
 %   \tag{W-W}
 %   \sem{\REF{\cExp} \GETS \aExp \SEMI \REF{\dExp} \GETS \bExp \SEMI 
%\aCmd} &=
 %   \sem{\REF{\dExp} \GETS \bExp \SEMI \REF{\cExp} \GETS \aExp \SEMI 
%\aCmd}
%  \end{align*}
% \begin{proof}
% The semantics of sequential composition $\aCmd \SEMI \bCmd$, where 
% the only enforced $\lt$ relationships come from conflict on locations or 
%release or acquire actions.    The roach-motel reorderings that increase 
%the scope of synchronization are valid because the pomsets in the 
%semantics of the right hand sides are augmentations of a pomset on the 
%left hand side. 
%\end{proof}
%\end{lemma}


% \begin{lemma}%[Reorderings]
% Suppose $\cExp\neq\dExp$ is a tautology.  Then the following 
%reorderings hold.
 % \begin{align*}
%    \tag{R-Acq}
 %   \sem{\aReg \GETS \REF{\cExp} \SEMI \bReg \GETS \REF{\dExp}\ACQ 
%\SEMI \aCmd} &\supseteq
 %   \sem{\bReg \GETS \REF{\dExp}\ACQ\SEMI \aReg \GETS \REF{\cExp} 
%\SEMI \aCmd}
% &&\textif \aReg\notin\free(\dExp) \textand \bReg\notin\free(\cExp)
 %   \\
 %   \tag{W-Acq} 
 %   \sem{\REF{\dExp} \GETS \bExp \SEMI\aReg \GETS \REF{\cExp}\ACQ 
%\SEMI\aCmd} &\supseteq
%    \sem{\aReg \GETS \REF{\cExp}\ACQ  \SEMI \REF{\dExp}\GETS \bExp 
%\SEMI \aCmd} 
%    &&\textif \aReg\notin\free(\dExp) \cup\free(\bExp)
 %   \\
% \tag{Rel-R} 
%\sem{\REF{\dExp}\REL \GETS \bExp \SEMI\aReg \GETS \REF{\cExp} 
%\SEMI \aCmd} &\supseteq
 %   \sem{ \aReg \GETS \REF{\cExp} \SEMI \REF{\dExp}\REL \GETS \bExp 
%\SEMI \aCmd}
%   &&\textif \aReg\notin\free(\dExp) \cup\free(\bExp)
 %   \\
 %   \tag{Rel-W}
 %   \sem{\REF{\cExp}\REL \GETS \aExp \SEMI \REF{\dExp} \GETS \bExp %\SEMI \aCmd} &\supseteq
 %   \sem{\REF{\dExp} \GETS \bExp \SEMI \REF{\cExp}\REL \GETS \aExp 
%\SEMI \aCmd}
%  \end{align*}
%\begin{proof}
%The proof follows from noticing that the pomsets in the semantics of the 
%right hand sides are augmentations of a pomset on the left hand side.  
%\end{proof}
%\end{lemma}

% \paragraph*{Compiler optimizations.} Reordering and peephole optimizations
% can be combined to describe common compiler optimizations.  We illustrate
% using common subexpression elimination,
% following~\citet{Dolan:2018:BDR:3192366.3192421}:
% Consider the command 
% \begin{math}
%   (\aReg \GETS \aLoc *2  \SEMI \aCmd \SEMI \bReg \GETS \aLoc * 2)
% \end{math}
% where $\aCmd$ is independent of $\aReg$.  Reordering yields
% \begin{math}
%   (\aCmd \SEMI \aReg \GETS \aLoc *2  \SEMI  \bReg \GETS \aLoc * 2),
% \end{math}
% followed by redundant load to yield
% \begin{math}
%   (\aCmd \SEMI \aReg \GETS \aLoc * 2 \SEMI  \bReg \GETS \aReg).
% \end{math}

% Similarly, the treatment of loop-invariant code motion, dead-store
% elimination and constant propagation
% from~\citet{Dolan:2018:BDR:3192366.3192421} follow.

% Since our model is more generous about permitted reorderings, we 
% permit optimizations that they forbid.  Consider:
%\begin{math}
 % (\aReg \GETS \aLoc \SEMI \bLoc \GETS \cLoc  \SEMI \aLoc \GETS 
%\aReg).
%\end{math}
%Reordering, permitted by us, but forbidden by them, yields
%\begin{math}
%  (\aReg \GETS \aLoc \SEMI \aLoc \GETS \aReg \SEMI \bLoc \GETS 
%\cLoc),
%\end{math}
%followed by the valid elimination of redundant load
%\begin{math}
%  (\aReg \GETS \aLoc \SEMI \aLoc \GETS \aReg \SEMI \bLoc \GETS %
%\cLoc).
%\end{math}

As expected, %sequential and
parallel composition commutes with conditionals and declarations, and
conditionals and declarations commute with each other.  We discussed scope
extrusion on page \pageref{page:extrusion}.
%For example,  % We show sequential scope extrusion \ref{SSE}, which concerns
% sequential composition and location binding:
% \begin{math}
%   % \taglabel{PSE}
%   \sem{\aCmd\PAR \VAR\aLoc\SEMI\bCmd}\allowbreak =
%   \sem{\VAR\aLoc\SEMI(\aCmd\PAR\bCmd)}
%   % \\
%   % \taglabel{SSE}
%   % \sem{\aCmd\SEMI \VAR\aLoc\SEMI\bCmd}\allowbreak &=
%   % \sem{\VAR\aLoc\SEMI(\aCmd\SEMI\bCmd)}
%   % \\
%   % \taglabel{CSE}
%   % \sem{\IF{\aExp}\THEN\aCmd\ELSE \VAR\aLoc\SEMI\bCmd\FI}\allowbreak &=
%   % \sem{\VAR\aLoc\SEMI \IF{\aExp}\THEN\aCmd\ELSE\bCmd\FI}
% \end{math}
% if $\aLoc$ does not appear in $\aCmd$.

%\ref{SSE} requires that $\aLoc$ does not appear in $\aCmd$.


\paragraph{Invalid Rewrites}
%\label{sec:invalid}

Write introduction is invalid:
\begin{align*}
  \sem{\aLoc \GETS 1} 
  &\not\supseteq
  \sem{\aLoc \GETS 1 \SEMI \aLoc \GETS 1}
\end{align*}
Observationally, these are distinguished by the context
\begin{math}
  \hole{} \PAR
  r\GETS x \SEMI
  x\GETS2 \SEMI
  s\GETS x\SEMI
  \IF{\aReg {=} \bReg} \THEN \cLoc \GETS 1 \FI.
\end{math}

Read introduction is also invalid.  Consider first a \emph{relevant} read:
\begin{align*}
  %\tag{Read-Intro-Invalid}
  \sem{\aReg \GETS \aLoc \SEMI \IF{\aReg {\neq} \aReg} \THEN \cLoc \GETS 1 \FI}
  &\not\supseteq
  \sem{\aReg \GETS \aLoc \SEMI \bReg \GETS \aLoc  \SEMI \IF{\aReg {\neq}\bReg} \THEN \cLoc \GETS 1 \FI}
\end{align*}
These are distinguished by the context
\begin{math}
  \hole{} \PAR x\GETS1\PAR x\GETS2.
\end{math}

But even \emph{irrelevant} read introduction is invalid:
$\sem{\SKIP}\not\supseteq\sem{r\GETS x}\not\supseteq\sem{r\GETS x\SEMI r\GETS x}$.  These inequations hold even though there is no context in our
language that distinguishes these programs.
% With weaker notions of coherence
% \cite{Manson:2005:JMM:1047659.1040336}, these
% commands are indistinguishable.

\ref{RL} and \ref{SF} show that redundant accesses can be removed.  The same
does not hold for \emph{non-redundant} access:
$\sem{r\GETS x}\not\supseteq\sem{\SKIP}$ (consider a completed pomsets for
$\SKIP$).
% , since $\sem{\SKIP}$ includes the
% pomset with a single event, labeled $\DSTOP$.


We discussed the invalidity of variable renaming on page
\pageref{page:extrusion} and thread inlining on page
\pageref{page:inlining}.


% // p and q might be aliased
% int i = p
% // concurrent write to p.x by another thread
% int j = q 
% int k = p

% Since p and q only might be aliased, but are not definitely aliased, then the
% use of q cannot be optimized away (if it were known that p and q pointed to
% the same object, then it would be legal to replace the assignments to j and k
% with assignments of the value of i). Consider the case where p and q are in
% fact aliased, and another thread writes to the memory location for p/q
% between the first use of p and the use of q; the use of q will see the
% new value. It will be illegal for the second use of p (stored into k) to
% get the same value as was stored into i. However, a fairly standard compiler
% optimization would involve eliminating the getfield for k and replacing it
% with a reuse of the value stored into i. Un- fortunately, that optimization
% is illegal in any language that requires Coherence.

% One way to think of it is that since a read of a memory location may cause
% the thread to become aware of a write by another thread, it must be treated
% in the compiler as a possible write.



\citet{BoehmOOTA} \labeltext{considers}{page:rfub} the following programs:
\begin{gather*}
  \tag{RFUB}\label{RFUB}
  \sem{r\GETS y\SEMI x\GETS r}
  \not\supseteq
  \sem{r\GETS y\SEMI \IF{r \NOTEQ 1} \THEN z\GETS 1\SEMI r\GETS 1\FI \SEMI x\GETS r}
\end{gather*}
The left command is half of \ref{OOTA3}, from
\textsection\ref{sec:logic}.  The right command is dubbed \rfub{}, for
\emph{Register assignment From an Unexecuted Branch}.
\citeauthor{BoehmOOTA} observes that in the context $x\GETS y \PAR \hole{}$,
these programs have different behaviors.  Yet the \oota{} example on the left
never writes $1$.  Why should the unexecuted branch change that?  As it turns
out, both branches of the conditional in \ref{RFUB} can execute, since the write
to $x$ is independent of the read from $y$.  Considering just the two threads
above, we have $\hoare{\TRUE}{\text{\rfub}}{x=1}$, but not
$\hoare{\TRUE}{\text{\oota}}{x=1}$.  As a result, it is expected that \ref{RFUB}
may have additional behaviors.  The change in the thread from \ref{OOTA3} to
\ref{RFUB} is not a valid refinement under Hoare logic and thus it is not valid
in our semantics.

% Let $\aCmd$ be the right
% thread in \ref{rfub}.
% \begin{align*}
%   \aCmd = r\GETS y\SEMI \IF{r \NOTEQ 1} \THEN z\GETS 1\SEMI r\GETS 1\FI  \SEMI x\GETS r 
% \end{align*}
% Then, $\hoare{\TRUE}{\aCmd}{x=1} $ and $\hoare{y \neq 1}{\bCmd}{z=1}$.  Thus, the traditional sequential semantics {\em does not} attribute any cause for the write of $1$ to $x$, as seen by the execution:
% \begin{tikzdisplay}[node distance=1em]
%  \event{ry}{\DR{y}{1}}{}
%  \event{wa}{\DW{z}{1}}{right=of ry}
%  \event{wx}{\DW{x}{1}}{right=of wa}
%  \rf{ry}{wa}
% \end{tikzdisplay}
% % Consider a fragment from example~\ref{rfub}.
% % \begin{align*}
% % \bCmd:  r \GETS y \SEMI \IF{r \NOTEQ 42} \THEN a \GETS 1 \SEMI s \GETS 42 \FI \SEMI x \GETS 42
% % \end{align*}
% % Then, $\hoare{\TRUE}{\bCmd}{x=42} $ and $\hoare{y \neq 42}{\bCmd}{a= 1} $.  Thus, the traditional sequential semantics {\em does not} attribute any cause for the write of $42$ to $x$, as seen by the execution:
% % \begin{tikzdisplay}[node distance=1em]
% %  \event{ry}{\DR{y}{41}}{}
% %  \event{wx}{\DW{x}{42}}{right=of ry}
% %  \event{wa}{\DW{a}{1}}{below=of rx}
% %  \rf{ry}{wa}
% % \end{tikzdisplay}
% Consider the
% following, where all locations are initialized to $0$.  In \textsection\ref{sec:logic} we provide machinery to prove that \oota{} is
% incapable of writing $1$.  The question is whether this should be possible
% for \rfub, which changes \oota{} only to include a \emph{Register assignment
%   From an Unexecuted Branch} \cite{BoehmOOTA}:?  
% \begin{align*}
%   \label{oota}  \tag{\textsc{oota}}
%   y\GETS x
%   \PAR&
%   r\GETS y\SEMI
%   x\GETS r
%   \\
%   \tag{\rfub1}
%   \label{rfub}
%   y\GETS x
%   \PAR&
%   r\GETS y\SEMI
%   \IF{r \NOTEQ 1} \THEN z\GETS 1\SEMI r\GETS 1\FI  \SEMI x\GETS r 
% \end{align*}
% %and the following \emph{Out Of Thin Air} litmus test:



% {\bf MOVE TO APPENDIX??.  

% Given $\aCmd$, define 
% $\linV{\aCmd}$ as follows.
% \begin{definition}
% With each shared variable $\aLoc$, we associate a thread local variable $\aLocLoc$ and two boolean variables $\aRead,\aChanged$ that will be local to the program fragment.  

% \begin{align*}
% \linV{\aCmd}= & \VAR\vec{\aLocLoc} \SEMI \VAR \vec{\aRead} \SEMI \VAR \vec{\aChanged} \SEMI  \\
% & \vec{\aRead} =\vec{0} \SEMI \vec{\aChanged} = \vec{0} \SEMI \\
% & \aCmd' \SEMI   \\
% & \overline{\IF \aChanged\  \aLoc= \aLocLoc}
% \end{align*}
% where:
% $\aCmd'$ is derived from $\aCmd$ by replacing:
% \begin{itemize}
% \item  each read $\aReg = \aLoc$ by $\aRead = 1 \SEMI \IF {(\aChanged \lor\ \aRead)} \THEN\  \aReg= \aLocLoc \ELSE\  \aReg= \aLoc \SEMI \FI $,
% \item each write $\aLoc = \aExp$ by $\aChanged =1 \SEMI \aLocLoc = \aExp$
% \end{itemize}
% \end{definition}
% From the soundness and completeness of Hoare logic for sequential operational semantics (eg. see formalization in~\citet{gordonHoare}), we deduce for any $\aForm, \bForm$: 
% \[ \hoare{\aForm}{\aCmd}{\bForm}  \Longleftrightarrow\  \hoare{\aForm}{\linV{\aCmd}}{\bForm} \]
% }

\section{Relation to Hoare Logic}
Note that if $\aPSS'\supseteq\aPSS$, then
$\readc(\aPSS')\supseteq \readc(\aPSS)$.

% We give an abstract characterization of valid optimizations, establishing
% sufficient conditions to replace any command $\aCmd$ by an equivalent
% $\bCmd$: If $\aCmd$ and $\bCmd$ are synchronization\hyp{}free and
% sequentially equivalent, and furthermore $\bCmd$ is \emph{linear}---performs
% at most one read and at most one write on each location in each
% execution---then $\aCmd$ can be refined to $\bCmd$.

A program is \emph{sequential} if it lacks $\!\!\PAR\!\!$, and
\emph{synchronization-free} if lacks fences and $\modeRA/\modeSC$ access.  We
argue that our model is fully flexible with respect to optimization of such
programs, as long as the optimizations do not introduce new writes or
``relevant'' reads.  To do so, we first formalize a
relation between Hoare triples and pomsets with preconditions.
We then isolate a fragment of our
language and show
the soundness of {\em all} transformations of synchronization-free sequential
programs into this fragment.

%
\paragraph{Pomsets for Hoare Logic}
%\label{sec:hoare}
In \textsection\ref{sec:model}, we used Hoare logic
\cite{Hoare:1969:ABC:363235.363259,gordonHoare} to reason about
\emph{dependency analysis} for sequential code.  Here, we use an alternative
interpretation of Hoare logic to establish the soundness of \emph{program
  transformations}.  We develop a pomset semantics for pairs of formulae
$\semRW{\aForm}{\bForm}$, and show a relation between $\sem{\aCmd}$ and
$\semRW{\aForm}{\bForm}$ for valid Hoare triples
$\hoare{\aForm}{\aCmd}{\bForm}$.  In \textsection\ref{sec:model},
preconditions were discharged by read events.  Here, instead, preconditions
are derived from the read events themselves.

\begin{definition}
  \label{def:prepost}
  Let $\aPS\in\semRW{\aForm\land\cForm}{\bForm}$
  %We say that $\aPS$ \emph{satisfies} precondition $\aForm$ and postcondition $\bForm$ (notation $\aPS\in\semRW{\aForm}{\bForm}$)
  when it is possible to satisfy the following:

  Let $\aLocs$ be the set of locations such that $\aLoc\in\aLocs$ exactly
  when $\aForm$ depends on $\aLoc$.  For each $\aLoc\in\aLocs$, choose
  $\bEv_\aLoc\in\Event$ that reads $\aLoc$.
  Let $\bEvs_\aLocs=\{\bEv_\aLoc\mid\aLoc\in\aLocs\}$.
  Let $\aSub$ be the substitution generated % from $\aLocs$
  as follows:
  % For $\aLoc\in\aLocs$, 
  $\aLoc\aSub = \aVal$ exactly when $\bEv_\aLoc$ reads $\aVal$ from $\aLoc$.

  Let $\bLocs$ be the set of locations such that $\bLoc\in\bLocs$ exactly
  when $\bForm$ depends on $\bLoc$.  For each $\bLoc\in\bLocs$, choose
  $\aEv_\bLoc\in\Event$ that writes $\bLoc$.
  Let $\aEvs_\bLocs=\{\aEv_\bLoc\mid\bLoc\in\bLocs\}$.
  Let $\bSub$ be the substitution generated % from $\bLocs$
  as follows:
  % For $\bLoc\in\bLocs$, 
  $\bLoc\bSub = \aVal$ exactly when $\aEv_\bLoc$ writes $\aVal$ to $\bLoc$.

  Require that $\aForm\aSub$ and $\bForm\bSub$ are satisfiable.

  Require that $\labelingForm(\aEv_\bLoc)$ implies $\cForm$ (for each $\aEv_\bLoc$).
  
  Require that if $\cEv\le\aEv_\bLoc$ and $\cEv$ is a read, then $\cEv\in\bEvs_\aLocs$.

  Require that if $\aEv_\bLoc\le\cEv$ and $\cEv$ is a write to $\bLocs$, then $\cEv\in\aEvs_\bLocs$.
\end{definition}
Pictorially, we have:
\begin{tikzdisplay}[node distance=.1ex and 2em]
  \event{r}{\bEvs_\aLocs}{}
  \event{w2}{\cForm\mid\aEvs_\bLocs}{below right=of r}
  \event{w1}{\cEvs_\bLocs}{above right=of w2}
  \po{r}{w2}
  \wk{w1}{w2}
\end{tikzdisplay}
Here, $\aEvs_\bLocs$ are the final writes to $\bLocs$, with precondition $\cForm$.
$\cEvs_\bLocs$ are other writes to $\bLocs$, which must be ordered before $\aEvs_\bLocs$.
$\bEvs_\aLocs$ are the reads that the writes depend upon.

Precondition strengthening in Hoare logic validates read introduction, which
is invalid in our semantics.  Thus, in this section we consider pomsets that
are saturated with irrelevant reads.

% Recall that postconditions are properties of \emph{completed} executions.
% For example, in $\hoare{\TRUE}{x\GETS1\SEMI x\GETS2}{x{=}2}$, the
% postcondition does not hold for the prefix $x\GETS1$.  % As a result, we
% restrict attention to completed executions.  % A pomset
% $\aPS\in\aPSS$ is \emph{maximal} if there is no $\aPS'\in\aPSS$ such that
% $\aPS$ is a proper prefix of $\aPS'$.


\labeltext{Let}{page:readsat} $\readc(\aPSS)$ be the set $\aPSS'$ where $\aPS'\in\aPSS'$ when
$\exists\aPS\in\aPSS$ and  $\exists D$ such that $\Event'= \Event'\uplus D$,
${\le'} \supseteq{\le}$, $\labeling'(\aEv) = \labeling(\aEv)$, and for every
$\bEv\in D$ there are $\aLoc$ and $\aVal$ such that
$\labelingAct'(\bEv)=(\DR[\modeRLX]{\aLoc}{\aVal})$.

\todo{Let xs=ws, ys=vs be such that xi=wi not in ys vs. Then:
   {ys=vs} C {xs=ws}  <==>  [ys=vs | xs=ws] cap read(C) not= emptyset}
\begin{theorem}
  \label{thm:hoare}
  If $\aCmd$ is synchronization-free and sequential,
  $\hoare{\aForm}{\aCmd}{\bForm}\Longleftrightarrow\notdisjoint{\semRW{\aForm}{\bForm}}{\readsem{\aCmd}}$.
\begin{proof}
  The proof proceeds by induction on the derivation of the Hoare triple.

  We first consider the structural rules.  Precondition strengthening follows
  from augmentation closure.   The structural rule for
  disjunction
  {\small\begin{math}
    \frac{\hoare{\aForm_1}{\aCmd}{\bForm_1}\quad  \hoare{\aForm_2}{\aCmd}{\bForm_2}}{ \hoare{\aForm_1 \lor \aForm_2}{\aCmd}{\bForm_1\lor \bForm_2}} 
  \end{math}}
follows from disjunction closure (Definition~\ref{def:dis}). The structural rule for conjunction
  {\small\begin{math}
    \frac{\hoare{\aForm_1}{\aCmd}{\bForm_1}\quad \hoare{\aForm_2}{\aCmd}{\bForm_2}}{ \hoare{\aForm_1 \land \aForm_2}{\aCmd}{\bForm_1\land \bForm_2}} 
  \end{math}}
  follows from the fact that pomsets have only concurrency and no conflict.  

  The remaining cases follow directly from the semantics.  The only subtlety
  is the write rule, which uses $\parallel$ to ensure disjunction closure.
\end{proof}
\end{theorem}
% To illustrate multiple writes consider:
% \begin{gather*}
%   \hoare{\TRUE}{x\GETS 1 \SEMI x \GETS 2}{x=2}
%   \\
%   \hbox{\begin{tikzinline}[node distance=1em]
%       \event{w1}{\DW{x}{1}}{}
%       \event{w2}{\DW{x}{2}}{right=of w1}
%       \wk{w1}{w2}
%     \end{tikzinline}}
% \end{gather*}

Preconditions can be placed in $\aForm$ or $\cForm$ in
Definition~\ref{def:prepost}, resulting in different pomsets:
\begin{gather*}
  \hoare{x=1}{y \GETS x}{y=1}
  \\
    \hbox{\begin{tikzinline}[node distance=1em]
        \event{r}{\DR{x}{1}}{}
        \event{w}{\DW{y}{1}}{right=of r}
        \po{r}{w}
      \end{tikzinline}}
    \qquad\qquad
    \hbox{\begin{tikzinline}[node distance=1em]
        \event{r}{\DR{x}{1}}{}
        \event{w}{x=1 \mid \DW{y}{1}}{right=1ex of r}
      \end{tikzinline}}
\end{gather*}
Control dependencies are calculated correctly:
\begin{gather*}
  \hoare{\TRUE}{\IF{x} \THEN y \GETS 1 \ELSE y \GETS 1 \FI}{y=1} 
  \\
    \hbox{\begin{tikzinline}[node distance=1em]
        \event{r}{\DR{x}{1}}{}
        \event{w}{\DW{y}{1}}{right=1ex of r}
      \end{tikzinline}}
\end{gather*}
% In contrast, in a semantics that forbids load buffering, where the best that one can prove is
% $\hoare{x=v} {r \GETS x \SEMI \IF{r=1} \THEN y \GETS 1 \ELSE y \GETS 1 \FI}{y= 1}
% $.

For any consistent set of preconditions, we can always find a single pomset:
\begin{gather*}
  \hoare{x_1=1}{y_1 \GETS x_1}{y_1=1}
  \qquad
  \hoare{x_2=1}{y_2 \GETS x_2}{y_2=1}
  \\
    \hbox{\begin{tikzinline}[node distance=1em]
        \event{rx}{\DR{x_1}{1}}{}
        \event{wy}{\DW{y_1}{1}}{right=of rx}
        \event{ru}{\DR{x_2}{1}}{right=1ex of wy}
        \event{wv}{\DW{y_2}{1}}{right=of ru}
        \rf{rx}{wy}
        \rf{ru}{wv}
      \end{tikzinline}}
\end{gather*}
% \begin{corollary}
%   If $\bigwedge_{i\in I}\aForm_i$ is satisfiable,
%   \begin{math}
%     \textstyle\bigwedge_{i\in I}\hoare{\aForm_i}{\aCmd}{\bForm_i} \Longleftrightarrow
%     \notdisjoint{\textstyle\bigcap_{i\in I}\semRW{\aForm_i}{\bForm_i}}{\readsem{\aCmd}}.
%   \end{math}
% \end{corollary}

\paragraph{Linearity}
%\label{sec:linear}
A command $\aCmd$ is \emph{linear} if for every $\aPS\in\sem{\aCmd}$, (1) there is
at most one read and at most one write on any location, and (2) if there is
both a read and a write to the same location, then the reads is ordered
before the write.

Intuitively, this means that the context around $\aCmd$ is unable to
interfere with the atomic execution of $\aCmd$; dually, neither can the
atomic execution of $\aCmd$ interfere with the context.  From an arbitrary
command, it is a straightforward exercise to construct a linear command that
is sequentially equivalent.

We say that $\aCmd$ and $\aCmd'$ \emph{satisfy the same Hoare triples} when
$\hoare{\aForm}{\aCmd}{\bForm}$ if and only if
$\hoare{\aForm}{\aCmd'}{\bForm}$, for every $\aForm$ and $\bForm$.

\begin{corollary}\label{seqcompleteness}
  Let $\aCmd$ and $\aCmd'$ be synchronization-free and sequential.  Further,
  let $\aCmd'$ be linear.
  Then $\aCmd$ and $\aCmd'$ satisfy the same Hoare
  triples if and only if $\readsem{\aCmd} \supseteq \readsem{\aCmd'}$.
  % If $\aCmd$ and $\aCmd'$ satisfy the same Hoare
  % triples then $\readsem{\aCmd} \supseteq \readsem{\aCmd'}$.
%   \begin{proof}
%     The pomsets in the semantics of a linear sequential program fragment,
%     such as $\sem{\bCmd}$, are generated by the augmentation closure of
%     pomsets that have a special format that only include edges of the
%     form: \begin{tikzdisplay}[node distance=1em]
%       \event{r}{\smash{\vec{\bForm}}\mid\DR{\vec{\bLoc}}{\vec{\aVal}}}{}
%       \eventl{\aEv}{w}{\aForm\mid\DW{\aLoc}{\bVal}}{below right=of r}
%       \po{r}{w}
%     \end{tikzdisplay}
%     where $\vec{\bLoc}\GETS \vec{\aLoc}$ has no conflicting assignments to the same variable, and where each $\aLoc$ appears in at most one write event. Thus, using  theorem~\ref{hoareGen}, we deduce that $\sem{\bCmd}$ is completely determined by the Hoare triples satisfied by $\bCmd$.  
%    
%     By the hypothesis of this theorem, $\aCmd$ satisfies the same Hoare triples as $\bCmd$.  Using  theorem~\ref{hoareGen}, we deduce that $\sem{\bCmd} \subseteq \fsat(\sem{\aCmd})$. 
% \end{proof}
\end{corollary}

The linearity restriction ensures that the context cannot interfere with the
atomic execution of the command, and, dually, that the atomic execution of
the command cannot interfere with the context.  To see the need for this,
consider that the introduction of redundant relevant reads is valid sequentially, but
not valid concurrently. For example,
$\aReg \GETS \aLoc \SEMI \IF{\aReg \NOTEQ \aReg}\THEN \bLoc \GETS 1 \FI$
cannot be refined to
$\aReg \GETS \aLoc \SEMI \bReg \GETS \aLoc \SEMI \IF{\aReg \NOTEQ \bReg}
\THEN \bLoc \GETS 1 \FI$.  In a concurrent context, the latter program may
see different values for the two reads.

Order is necessary to exclude optimizations such as
$x\GETS 1 \SEMI r \GETS x$ to $x\GETS 1$. They are both linear. Satisfy the
same Hoare triples.  \todo{But have different read(C).}

\endinput 




\subsection{Full abstraction for synchronization free threads}
Our semantics is complete for reasoning about full-thread optimizations of synchronization free programs.

In the rest of this section, we only consider commands $\aCmd,\bCmd$ that are
restriction-free, composition-free (ie. single-threaded), and
synchronization-free (ie. no acquire, release, fence).

In order to develop the proof, 

This closure permits us to  describe a normal form for the top-level pomsets that arise in single-threaded and synchronization free code.  
\begin{definition}
$\aPS$ is in normal form if:
\begin{itemize}
% \item All preconditions on events are tautologies.
\item If $\bEv \lt \aEv$ and  $\bEv$ is a write, then $\aEv$ is a write or read on the same variable.
\item If $\bEv \lt \aEv$ and  $\aEv$ is a read, then $\bEv$ is a write on the same variable. 
    
% \item $\aEv \gtN \bEv$ only if $ \aEv\ (\lt \cup \reco)^{\star} \  \bEv$.
\item If $\aEv, \aEv'$ have the same read action label, then there exists a write $\bEv$ on the same location such that $\aEv \gtN \bEv \gtN \aEv'$.
%\item  If $\aEv, \aEv'$ have the same write action label, then 
%there exists a event $\bEv$ on the same location as $\aEv$ such 
%that  $\bEv$ has a different action label and $\aEv \gtN \bEv \gtN 
%\aEv'$.
\end{itemize}
\end{definition}
In a normal form pomset, the successors of a write event (resp. the predecessors of a read) are related in the coherence order to the event.  Any two events with the same read label are separated by a write in the $\reco$ order. 

It suffices to consider normal forms when distinguishing single-threaded, synchronization free code at the top level.  The normal forms determine the full semantics by the closure properties of the semantics.

\begin{lemma}\label{unrhd}
Every top-level pomset of $\freadsat\sem{\aCmd}$ is a top-level pomset of $\freadsat\sem{\bCmd}$ if and only if 
every top-level normal form pomset of $\freadsat\sem{\aCmd}$ is a top-level normal form pomset of $\freadsat\sem{\bCmd}$.
\begin{proof}
The proof proceeds by induction on structure.  The key case is prefixing.  The only edges $\lt$-edges out of writes  and into read actions enforced by prefixing are $\reco$ edges.  Read prefixing permits the reuse of events to ensure that distinct read events not separated by $\reco$ have distinct read labels.  
\end{proof}
\end{lemma}

We develop testers for top-level pomsets in normal form.  We  follow~\citet{Plotkin:1997:TSP:266557.266600}, albeit in a concrete form appropriate to our setting.   

Let $\aPS$ be a top-level pomset in normal form with events $\aEv_1, \ldots,
\aEv_n$.  For all $i$, we assume a new location $b_i$.    Let $\freshaval$ be a fresh value that does not occur in $\aPS$.  

Let $\vec{\bEv}$ be all  the predecessors of $\aEv_i$ in $\lt$. Let $\vec{c}$ be the corresponding sub vector of $b_i$'s. 
\begin{itemize}
\item 
If $\aEv_i$ has label $\DR{\aLoc}{\aVal}$, we define a program $\testP{\aPS}{i}$ as follows:  
\[
  \testP{\aPS}{i} = (
  % \vec{\aReg} \GETS \vec{c}  \SEMI
  \IF{\vec{c}=\vec{1}} \THEN x \GETS \aVal \SEMI  \FENCE \SEMI b_i \GETS 1 \FI)
\]
\item 
If $\aEv_i$ has label $\DW{\aLoc}{\aVal}$, we define a program $\testP{\aPS}{i}$ as follows:
\[
  \testP{\aPS}{i} = (
  % \vec{\aReg} \GETS \vec{c}  \SEMI
  \IF{\vec{c}=\vec{1}} \THEN %\aReg = \aLoc \SEMI
  \IF{\aLoc = \aVal} \THEN \aLoc = \freshaval \SEMI \FENCE \SEMI b_i \GETS 1  \FI \FI)
\]
\end{itemize}
$\testP{\aPS}{i}$ is as expected, matching the reads (resp. writes) in $\aPS$ by writes (resp. reads).  The fresh value is used to flush out prior writes and reset to see fresh writes.    

\begin{definition}[Tester for $\aPS$]\label{testAPS} 
The tester context for $\aPS$, $\testP{\aPS}{}\hole{}$ is 
\[
b \GETS b_{1} \land  \cdots \land b_{n}
\PAR  \testP{\aPS}{1} 
\PAR   \cdots 
\PAR  \testP{\aPS}{n} 
\PAR \hole{}
\]
\end{definition}
The constraints of the normal form ensure that if the labels of any two events are the same, then one is a predecessor of the other. 

\begin{lemma}\label{tester}
$\sem{\testP{\aPS}{}\hole{\aCmd}}$  has a pomset that sets $b$ to $1$ iff $\aPS \in \freadsat\sem{\aCmd}$.
\begin{proof}
It suffices to prove that there is a pomset that sets $b$ to $1$ in
 $\freadsat\sem{b \GETS b_{1} \land  \cdots \land b_{n}
\PAR  \testP{\aPS}{1} 
\PAR   \cdots 
\PAR  \testP{\aPS}{n}} \parallel \bPS$ iff $\aPS$ is an augmentation of $\bPS$. 

We prove by induction on $\lt$ that $\testP{\aPS}{i}\hole{\aCmd}$  sets $b_i$ to $1$ iff the event $\aEv_i$ is enabled. Result follows.
\end{proof}
\end{lemma}

\begin{theorem}
  Let $\aCmd_1$ and $\aCmd_2$ be restriction-free, composition-free and
  synchronization free.  Then all top-level pomsets of
  $\freadsat\sem{\aCmd_1}$ are also pomsets of $\freadsat\sem{\aCmd_2}$ if
  and only if for all parallel contexts $\bCmd$, all top-level pomsets of
  $\freadsat\sem{\aCmd_1 \PAR \bCmd}$ are also pomsets of
  $\freadsat\sem{\aCmd_2 \PAR \bCmd}$
\begin{proof}
  The forward implication follows from the compositionality of the semantics.

  For the converse, pick a top level pomset in normal form in
  $\aPS \in \freadsat\sem{\aCmd_1} \setminus \freadsat\sem{\aCmd_2} $.  The
  required context is given by the tester context $\testP{\aPS}{}\hole{}$.
 \end{proof}
\end{theorem}

The full abstraction theorem applies only to full threads.  The reason for this limited statement is that our impoverished language does not permit  parallel composition to be used freely as the continuation in arbitrary sequential contexts.  A full abstraction theorem that applies also to commands can be achieved with these richer distinguishing contexts.  
