For non-MCA:
\begin{itemize}
\item ${\le}$ is causal order
\item \hl{${\leloc}$} is per-location order
\item also need program order or \hl{thread ids}
\end{itemize}

In data model:
\begin{itemize}
\item \hl{a set of \emph{thread ids} $\Thrd$, ranged over by $\aThrd$ and
  $\bThrd$}
\end{itemize}
There are partial functions $\rreads$ and
$\rwrites: \Act \fun (\Thrd \times \Loc \times \Val)$, and there are subsets
of $\Act$: $\Acq$, $\Rel$, $\SC$, and $\Term$.  We derive the partial
functions $\fthrd: \Act \fun \Thrd$ and $\floc: \Act \fun \Loc$ from
$\rreads$ and $\rwrites$.  For example, define $\floc(\aAct)= \aLoc$ if
$\rreads(\aAct) = (\dontcare,\aLoc,\dontcare)$ or
$\rwrites(\aAct) = (\dontcare,\aLoc,\dontcare)$.

\begin{definition}[\ref{def:mmpomset}]
  A \emph{pomset with preconditions} is a tuple
  $(\Event, {\le}, \hbox{\hl{$\leloc$}},
  \labeling)$, such that
  \begin{itemize}
  \item $\Event$ is a set of \emph{events},
  \item ${\le} \subseteq (\Event\times\Event)$ and \hl{${\leloc} \subseteq (\Event\times\Event)$} are partial orders, 
  \item $\labeling: \Event \fun (\Formulae\times\Act)$ is a \emph{labeling},
    from which we derive functions $\labelingForm:\Event\fun\Formulae$ and $\labelingAct:\Event\fun\Act$,
  \item $\bigwedge_{\aEv}\labelingForm(\aEv)$ is satisfiable
    \emph{(consistency)},
  \item if $\bEv\le\aEv$ then $\labelingForm(\aEv)$ implies
    $\labelingForm(\bEv)$ \emph{(causal strengthening)}, and
  \item \hl{if $\floc(\labelingAct(\bEv))=\floc(\labelingAct(\aEv))$ and
    $\bEv\le\aEv$ then $\bEv\leloc\aEv$ \emph{(observation)}}.
  \end{itemize}
  % A pomset is \emph{completed} if $\exists\aEv.\;\labelingAct(\aEv)=\DSTOP$. % and $\forall\bEv.\;\bEv\le\aEv$.
\end{definition}


\begin{candidate}[\ref{def:prefix}]
  Let $(\aForm \mid \aAct) \prefix \aPSS$ be the set
  $\PRE{\aPSS'}$ %$\aPSS'$ 
  where
$\aPS'\in\aPSS'$ when 
there is $\aPS\in\aPSS$ such that
\hbox{{\labeltextsc[P1]{(P1)}{1}} $\Event' = \Event \cup \{\bEv\}$,}
{\labeltextsc[P2]{(P2)}{2}}  ${\le'}\supseteq{\le}$, % if $\bEv \le \aEv$ then $\bEv \le' \aEv$,
{\labeltextsc[P3]{(P3a)}{3}}  $\labelingAct'(\aEv) = \labelingAct(\aEv)$, 
\labeltextsc[P3b]{(P3b)}{3b} $\labelingAct'(\bEv) = \aAct$,
\begin{enumerate}
% \item[(1)]\label{pre-E} $\Event' = \Event \uplus \{\bEv\}$,
% \item[(2)]\label{pre-le} ${\le'}\supseteq{\le}$, % if $\bEv \le \aEv$ then $\bEv \le' \aEv$,
% \item[(3a)] $\labelingAct'(\bEv) = \aAct$,
% \item[(3b)] $\labelingForm'(\bEv)$ implies $\aForm$,
% \item[(4a)] $\labelingAct'(\aEv) = \labelingAct(\aEv)$,
\item[{\labeltextsc[P4a]{(P4a)}{4a}}]{\labeltextsc[P4]{}{4}}%
  if $\bEv\in\Event$ then $\labelingForm'(\bEv)$ implies
  $\aForm\lor\labelingForm(\bEv)$, otherwise $\labelingForm'(\bEv)$ implies $\aForm$,
\item[{\labeltextsc[P4b]{(P4b)}{4b}}]
  if $\bEv$ does not read then either $\aEv=\bEv$ or
  $\labelingForm'(\aEv)$ implies $\labelingForm(\aEv)$, 
\item[{\labeltextsc[P4c]{(P4c)}{4c}}]
  if $\bEv$ reads $\aVal$ from $\aLoc$ then either $\aEv=\bEv$ or
  $\labelingForm'(\aEv)$ implies $\labelingForm(\aEv)[\aVal/\aLoc]$,
\item[{\labeltextsc[P5a]{(P5a)}{5a}}]\labeltextsc[P5]{}{5}%
  if $\bEv$ reads and $\aEv$ writes then either $\aEv=\bEv$ or $\labelingForm'(\aEv)$
  implies $\labelingForm(\aEv)$ or $\bEv\le'\aEv$,
% \item[5a.]  if $\aEv$ writes then either $\labelingForm'(\aEv)$ implies
%   $\labelingForm(\aEv)$, or some $\cEv\le'\aEv$ reads $\aVal$
%   from $\aLoc$ and $\labelingForm'(\aEv)$ implies $\labelingForm(\aEv)[\aVal/\aLoc]$,
% \item[5a.] if %$\bEv$ \externally reads and
%   $\labelingForm'(\aEv)$ does not imply $\labelingForm(\aEv)$ and $\aEv$ writes, then
%   $\bEv\le'\aEv$,
\item[{\labeltextsc[P5b]{(P5b)}{5b}}]
  if $\bEv$ and $\aEv$ are \external actions in conflict then \hl{$\bEv\leloc'\aEv$}, %$\bEv \gtN' \aEv$,
\item[{\labeltextsc[P5c]{(P5c)}{5c}}]
  if $\bEv$ is an acquire or $\aEv$ is a release then $\bEv \le' \aEv$, and
\item[{\labeltextsc[P5d]{(P5d)}{5d}}]
  if $\bEv$ is an SC write and $\aEv$ is an SC read then $\bEv \le' \aEv$.
%\item[5e.] if $\aEv$ is a termination, then $\bEv\le'\aEv$.
% \item[6.] if $\bEv$ is a release, $\aEv_1$ is an acquire, $\aEv_1\le\aEv_2$, then $\labelingForm(\aEv_2)$
%   is location independent.
\end{enumerate}
\end{candidate}

\begin{definition}[\ref{def:rf}]
  We say $\bEv$ \emph{fulfills $\aEv$} (on $\aLoc$) if
  \labeltextsc[F1]{(F1)}{rf1} $\bEv$ \externally writes $\aVal$ to $\aLoc$,
  \labeltextsc[F2]{(F2)}{rf2} $\aEv$ \externally reads $\aVal$ from $\aLoc$,
  \labeltextsc[F3]{(F3)}{rf3} $\bEv \lt \aEv$, and
  \labeltextsc[F4]{(F4)}{rf4} for every conflicting write $\cEv$, either
  \hl{$\cEv \leloc \bEv$ or $\aEv \leloc \cEv$ or
    $\fthrd(\aEv)=\fthrd(\cEv)$}.
\end{definition}
Last case in fulfillment motivated by this example (\textsection\ref{sec:arm}):
\begin{gather*}
  %\taglabel{flowing/pop}
  \begin{gathered}
  r\GETS x\SEMI x\GETS 1
  \PAR
  y\GETS x 
  \PAR
  x\GETS y 
  \\[-1.2ex]
  \hbox{\begin{tikzinline}[node distance=1.5em]
      \event{a}{\DR{x}{1}}{}
      \event{b}{d:\DW{x}{1}}{right=of a}
      \wk{a}{b}
      \event{c}{\DR{x}{1}}{right=3em of b}
      \event{d}{\DW{y}{1}}{right=of c}
      \po{c}{d}
      \event{e}{\DR{y}{1}}{right=3em of d}
      \event{f}{e:\DW{x}{1}}{right=of e}
      \po{e}{f}
      \rf{b}{c}
      \rf{d}{e}
      \rf[out=172,in=8]{f}{a}
    \end{tikzinline}}
\end{gathered}
\end{gather*}

