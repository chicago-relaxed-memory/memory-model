\section{Introduction}
\citet{Manson:2005:JMM:1047659.1040336} identify the central problem in the design of software relaxed memory models: ``The memory model must strike a balance between ease-of-use for programmers and implementation flexibility for system designers.''   \emph{None} of the extant memory models validate both ``implementation flexibility'' and ``ease-of-use''.   This paper provides a solution.

%In order to sharpen the criteria to evaluate memory models, we first outline desiderata  that concretize the above prescription.  

There are two dimensions to ``implementation flexibility''.  Firstly, the model should be realizable on modern hardware with minimal synchronization.  A canonical example is that the relaxed atomics of C11 (or the plain variables of Java) should not require any extra synchronization.  Secondly, the  model should facilitate compiler transformations.    Ideally, the model should support the known transformations used to optimize  synchronization free single threaded code.  A canonical example is the  commuting of independent statements.


There are also two dimensions to ``ease-of-use''.  Firstly,the \emph{data race free-sequentially consistent (\drfsc)} criterion~\cite{DBLP:journals/tpds/AdveH93, DBLP:conf/isca/AdveH90} permits the programmer to forget about relaxed memory for correctly synchronized programs.    Secondly, all programs, including those with data races,  should support compositional reasoning on temporal safety properties~\cite{PnueliSafety,Misra:1981:PNP:1313338.1313770,StarkSafety,Abadi:1993:CS:151646.151649} \footnote{The simplest form of such a composition principle is:
\[
  \frac{
      \afo, \aPSS_1 \models\afo
      \qquad
      \afo, \aPSS_2 \models\afo
    }{\aPSS_1 \parallel \aPSS_2 \models \afo}
\]
where $\afo$ is a temporal safety property on read and write actions, and $\afo, \aPSS_1 \models\afo$ indicates that $\aPSS_1$ satisfies $\afo$ if the environment satisfies $\afo$. } to permit us to reason separately about individual threads validating safety properties. 

``Out Of Thin Air'' (\oota) executions invalidate the 
composability  of safety properties.    Consider the ``type unsafety'' example, from
\citep[Figure 8]{DBLP:journals/toplas/Lochbihler13}:
\begin{equation}
  \label{types}
  %x\GETS 0 \SEMI  y\GETS 0  \SEMI  b\GETS 0 \SEMI  
  x\GETS y
  \PAR
  \aReg\GETS x \SEMI \IF{b} \THEN \bReg \GETS \NEW \texttt{C()} \ELSE \aReg \GETS \NEW \texttt{D()} \FI  \SEMI y \GETS \aReg \SEMI  
  \PAR
  b \GETS 1   
\end{equation}
Prima facie, the allocation operation can pick the same address for the two new objects, because only one of them occurs in any one execution.   This causes type safety to break for racy programs in the JMM, forcing~\citeauthor{DBLP:journals/toplas/Lochbihler13} to include the type information in the address itself.
Informally, all threads satisfy the invariant that conjoins ``a creation of a \texttt{C} object is preceded by a read of \texttt{b} as $1$'' and ``a creation of a \texttt{D} object is preceded by a read of \texttt{b} as $0$''.  If composability of safety were to hold, the full program satisfies the invariant,   and facilitates the proof of a temporal invariant that forbids``pointer forging''  within a single execution.  

Models that validate ``ease-of-use'' include Sequential Consistency (SC)~\citet{Lamport:1979:MMC:1311099.1311750}, \citet{Dolan:2018:BDR:3192366.3192421}, \citet{DBLP:conf/pldi/LahavVKHD17}, \citet{DBLP:conf/lics/JeffreyR16}, RC11 model\cite{DBLP:conf/pldi/LahavVKHD17} and~\citet{BoehmOOTA}.  However, {\em all} of them invalidate reordering of independent statements.  \cite{Dolan:2018:BDR:3192366.3192421,BoehmOOTA,DBLP:conf/lics/JeffreyR16} forbid breaking of the program order from reads to writes and thus require extra fences after read actions in hardware implementations.  The results of \citet{Boehm:2014:OGA:2618128.2618134} show that the RC11 model\cite{DBLP:conf/pldi/LahavVKHD17} forces a dependency or a fence between a relaxed atomic read and a subsequent relaxed atomic write.  


Whereas the JMM was designed with these two aims, the JMM and the related models of~\citet{DBLP:conf/esop/JagadeesanPR10}, \citet{DBLP:conf/popl/KangHLVD17} and~\citet{Chakraborty:2019} invalidate compositional reasoning, as demonstrated by variants of \citeauthor{DBLP:journals/toplas/Lochbihler13}'s type unsafety example.  


Our approach has two key ingredients.  

First, we focus on  multi-copy atomicity (\mca) in a software memory model. \footnote{\mca\ is traditionally explored in hardware memory models.   \tso\ (see, e.g.~\cite{DBLP:journals/cacm/SewellSONM10}) and recent architectures, such as \armeight\ (see, e.g.~\cite{DBLP:journals/pacmpl/PulteFDFSS18}), are \mca, but not older architectures, such as \ppc\ (see, e.g.~\cite{DBLP:conf/pldi/SarkarSAMW11}) or \armseven\ (see, e.g.~\cite{DBLP:conf/popl/AlglaveFIMSSN09}).}  As envisioned in chapter 3.3 of~\citet{AlglaveThesis},  this permits us to address cross-thread dependencies simply by a partial order in a pomset model~\citep{GISCHER1988199,Plotkin:1997:TSP:266557.266600}, with the acylicity of the pomset providing a global notion of time.  

Second, we weaken the program-order within a thread to a preferred program order that only captures essential dependencies.  We use classical Hoare logic~\citep{Hoare:1969:ABC:363235.363259}, as presented in~\citet{gordonHoare}.  
Consider the following sequential program fragments
\begin{align*}
& \aCmd_1: \aLoc \GETS 1 \SEMI \bLoc \GETS 1 \\
& \aCmd_2: \aReg \GETS \aLoc \SEMI \IF{\aReg == 0} \THEN\ \bLoc \GETS 1 \ELSE \bLoc \GETS 1  \FI \\
& \aCmd_3: \aLoc \GETS 1 \SEMI \IF{\aReg != 0} \THEN \bLoc \GETS 1 \FI \\
&\aCmd_4:  \bLoc \GETS 1 \SEMI \aLoc \GETS 1
\end{align*}
All the above program fragments satisfy the same Hoare triples.  In particular, $\hoare{\TRUE}{\aCmd_i}{\bLoc =1}$ holds for all $i=1,\ldots,4$, capturing the essence of the idea that the program order into $\bLoc \GETS 1$ can be removed and that the write of $1$ to $\bLoc$ can be commuted to the beginning\footnote{Thus, the model fully reaps the benefits of viewing a memory model in terms of (sequential) program transformations, eg. see~\citep{Saraswat:2007:TMM:1229428.1229469,DBLP:conf/fm/LahavV16,
DBLP:conf/popl/DemangeLZJPV13,DBLP:conf/esop/FerreiraFS10}, without explicitly being formalized as such.} .


We show the following.
\begin{itemize}
\item The model can account for the full suite of features found in the C11 memory model, including relaxed/release-acquire and SC atomics, fences, and RMW.

\item The model satisfies the \drfsc\ criterion  (\textsection\ref{sec:sc}).

\item The model compiles to \armeight\ and \tso\ {\em without} extra synchronization for raw variables  (\textsection\ref{sec:arm}).
However, a semantic version of \mca\ in our model means that a compilation to \armseven\ or \ppc\ requires extra synchronization.  

\item The model validates compositional reasoning on safety properties (\textsection\ref{sec:logic}).  Thus, it supports the infrastructure required to prove a realistic type safety theorem that includes racy programs without requiring that that the type information is included in the address itself.

\item The model validates a variety of single threaded transformations including reordering of independent statements and roach motel laws for synchronization  (\textsection\ref{sec:opt}).
\end{itemize}
On the last point, we demonstrate a further completeness result.  Whereas the JMM aims to validate {\em all} sequential optimizations, it is clear that is impossible.  For example, the introduction of redundant reads  but not valid concurrently. Thus, $\aReg \GETS \REF{\aLoc} \SEMI \IF{\aReg != \aReg} \cLoc \GETS 1 \FI$ cannot be replaced by $\aReg \GETS \aLoc \SEMI \bReg \GETS \aLoc  \SEMI 
\IF{\aReg != \bReg} \THEN \cLoc \GETS 1 \FI$.  

% Similarly, for redundant writes; $\aLoc \GETS 1$ cannot be 
% replaced by $\aLoc \GETS 1 \SEMI \aLoc \GETS 1 $ in a model 
% with coherence.  

Our model does the best possible under these constraints.   Call a program fragment ``linear'' if it does at most one read and at most one write on any location in any execution.  Thus, the context is unable to interfere with the atomic execution of the command; dually, neither can the atomic execution of the command interfere with the context.  We show that if sequential and synchronization free $\aCmd$ and $\bCmd$ are sequentially equivalent, and furthermore $\bCmd$ is ``'linear'' in this sense, then $\aCmd$ can be validly replaced by $\bCmd$.  




%\paragraph*{Rest of the paper. }  We begin with an informal introduction to the modeling ideas in \textsection\ref{sec:model:intro}, developing the precise formalities in \textsection\ref{sec:model}.   \textsection\ref{sec:sc} proves the DRF theorem, whereas \textsection\ref{sec:arm} provides a compilation into \armeight\ and \tso.  Single threaded optimizations, and the associated completeness theorems are addressed in \textsection\ref{sec:opt}.  \textsection\ref{sec:logic} describes a temporal logic, and a compositional proof principle for proving safety properties.  \textsection\ref{sec:examples}.develops more illustrative examples.  We address related work in \textsection\ref{sec:ldrf} and conclude after.  An appendix contains details of proofs and further examples.

We give an informal introduction to the model in \textsection\ref{sec:model:intro} before presenting the precise formalities in \textsection\ref{sec:model}.
\textsection\ref{sec:sc} proves the DRF theorem, whereas \textsection\ref{sec:arm} provides a compilation into \armeight\ and \tso.  Single threaded optimizations, and the associated completeness theorems are addressed in \textsection\ref{sec:opt}.  \textsection\ref{sec:logic} describes a temporal logic, and a compositional proof principle for proving safety properties.
We %present additional examples in \textsection\ref{sec:examples} and
end with
a discussion of related work in \textsection\ref{sec:ldrf}.
% An appendix
% contains details of proofs and further examples.


\endinput

To the reader interested in models that forbid load buffering, we provide a way to adapt our model to forbid the relaxing of the program order from reads to writes, thus modeling~\cite{Dolan:2018:BDR:3192366.3192421,BoehmOOTA}.  Our new contributions for such a reader are an approach to validating data-sensitive compiler optimizations and compositional reasoning of temporal properties.   


We illustrate the last criterion with two examples.   
First, consider the well-known ``Out Of Thin Air'' (\oota) litmus test, with all variables initialized to $0$:
\begin{equation}
  \label{oota1}
  %x\GETS0\SEMI y\GETS0\SEMI
  (y\GETS x \PAR x \GETS y)
\end{equation}
Informally, both threads satisfy the invariant that conjoins ``A write of 1  to x  requires a prior read  of 1 from y'' and ``A write of 1  to y  requires a prior read  of 1 from x ''.  If composition holds, the full program satisfies the invariant.  Since the variable declaration closes the program from other writes to $x,y$, we  deduce the conjunction of  ``A write of 1  to x  requires a prior write  of 1 to x'' and ``A write of 1  to y  requires a prior write  of 1 from x'' . Thus, we deduce that ``A write of 1  to x  requires a prior write  of 1 to x'', and consequently ``there is no write of 1 to x''. 

provides an  {\em objectively  falsifiable} measurement of \oota\ in a memory model. 



