\section{Introduction}
\label{sec:intro}
\citet{Manson:2005:JMM:1047659.1040336} identify the central problem in the design of software relaxed memory models: ``The memory model must strike a balance between ease-of-use for programmers and implementation flexibility for system designers.''   \emph{None} of the extant memory models validate both ``implementation flexibility'' and ``ease of use.''   This paper provides a solution.

%In order to sharpen the criteria to evaluate memory models, we first outline desiderata  that concretize the above prescription.  

There are two dimensions to ``implementation flexibility.''  First, the model should be realizable on modern hardware with minimal synchronization.  A canonical example is that the relaxed atomics of C11 (or the plain variables of Java) should not require any extra synchronization.

Second, the  model should facilitate compiler transformations.    Ideally, the model should support the known transformations used to optimize  synchronization\hyp{}free single\hyp{}threaded code.  A canonical example is that independent statements should commute.


There are also two dimensions to ``ease of use.''  First, the \emph{data race free-sequentially consistent (\drfsc)} criterion~\cite{DBLP:journals/tpds/AdveH93, DBLP:conf/isca/AdveH90} permits the programmer to forget about relaxed memory for correctly synchronized programs.

Second, all programs---including those with data races---should support compositional reasoning on temporal safety properties~\cite{PnueliSafety,Misra:1981:PNP:1313338.1313770,StarkSafety,Abadi:1993:CS:151646.151649}\nofootnote{TOO MUCH FOR HERE The simplest form of such a composition principle is:
\[
  \frac{
      \afo, \aPSS_1 \models\afo
      \qquad
      \afo, \aPSS_2 \models\afo
    }{\aPSS_1 \parallel \aPSS_2 \models \afo}
\]
where $\afo$ is a temporal safety property on read and write actions, and $\afo, \aPSS_1 \models\afo$ indicates that $\aPSS_1$ satisfies $\afo$ if the environment satisfies $\afo$.}.
%to permit us to reason separately about individual threads validating safety properties. 
% ``Out Of Thin Air'' (\oota) executions invalidate the 
% composability  of safety properties.
The following program ``is type correct if it declares $x$, $y$ and $r$ of type \texttt{D}. However, it has a legal execution [in Java] where they reference a \texttt{C} object'' \citep[Fig.~8]{DBLP:journals/toplas/Lochbihler13}:
%\vspace{-.5ex}
\begin{gather}
  \label{lochbihler}
  \begin{aligned}
    &z \GETS 1
    \PAR %\\[-1ex] \PAR&
    y\GETS x
    \\[-1ex] \PAR&
    \aReg\GETS y \SEMI \IF{z} \THEN \bReg \GETS \NEW \texttt{C} \ELSE \aReg \GETS \NEW \texttt{D} \FI  \SEMI x \GETS \aReg 
  \end{aligned}
\end{gather}
 Informally, all threads satisfy the invariant ``allocation at type \texttt{C} is preceded by reading $1$ for $z$'' and ``allocation at type \texttt{D} is preceded by reading $0$ for $z$.''  If composability of safety were to hold, the full program would satisfy both invariants.
\citeauthor{DBLP:journals/toplas/Lochbihler13} argues that composability is necessary to prove type safety of racy Java programs without partitioning memory by type---an unrealistic assumption for any allocator. %    and facilitates the proof of a temporal invariant that forbids``pointer forging''  within a single execution.  
 % Prima facie, the allocation operation can pick the same address for the two new objects, because only one of them occurs in any one execution.   This causes type safety to break for racy programs in the JMM, forcing~\citeauthor{DBLP:journals/toplas/Lochbihler13} to include the type information in the address itself.

Models that validate ``ease-of-use'' include Sequential Consistency (SC)~\citep{Lamport:1979:MMC:1311099.1311750}, RC11 \citep{DBLP:conf/pldi/LahavVKHD17}, and others \citep{Dolan:2018:BDR:3192366.3192421,DBLP:conf/pldi/LahavVKHD17,DBLP:conf/lics/JeffreyR16,Boehm:2014:OGA:2618128.2618134}.  However, all of these models invalidate reordering of independent statements.  Several require extra fences after read actions in hardware implementations. % \cite{Dolan:2018:BDR:3192366.3192421,BoehmOOTA,DBLP:conf/lics/JeffreyR16} forbid breaking of the program order from reads to writes and thus require extra fences after read actions in hardware implementations.
% \citep{Boehm:2014:OGA:2618128.2618134} show that the RC11 model %\cite{DBLP:conf/pldi/LahavVKHD17}
% forces a dependency or a fence between a relaxed atomic read and a subsequent relaxed atomic write.  


% Whereas the JMM was designed with these two aims,
Java \citep{Manson:2005:JMM:1047659.1040336} and  related models \citep{DBLP:conf/esop/JagadeesanPR10,DBLP:conf/popl/KangHLVD17,DBLP:journals/pacmpl/ChakrabortyV19} are intended to validate more optimizations.  However, all of these models invalidate compositional reasoning, as in \eqref{lochbihler}.  Thus, they cannot support both type safety and realistic memory allocation.


Our approach has two key ingredients.  

First, we weaken the program-order within a thread to capture only \emph{essential dependencies}.
Previous language models have used syntactic notions of dependency \cite{Batty:2011:MCC:1926385.1926394}.
Instead, we embed logic formulae in events, using classical
\citet{Hoare:1969:ABC:363235.363259} logic to compute dependencies between read and write events. %, as presented by~\citet{}.  
Consider the following  program fragments:
\begingroup
\allowdisplaybreaks
\begin{align*}
  & \aCmd_1: \aLoc \GETS 1 \SEMI \bLoc \GETS 1
  \\[-1ex] & \aCmd_2: \aReg \GETS \aLoc \SEMI \IF{\aReg} \THEN\ \bLoc \GETS 1 \ELSE \bLoc \GETS 1  \FI
  \\[-1ex] & \aCmd_3: \aLoc \GETS 1 \SEMI \IF{\aReg} \THEN \bLoc \GETS 1 \FI
  %\\[-1ex] &\aCmd_4:  \bLoc \GETS 1 \SEMI \aLoc \GETS 1
\end{align*}
\endgroup
All these fragments satisfy  $\hoare{\TRUE}{\aCmd_i}{\bLoc =1}$; thus, in each case, the write of $y$ is independent of
any code that precedes it in program order.  This allows a compiler or processor to reorder the write with respect to the code that precedes it\nofootnote{IN RELATED Thus, the model fully reaps the benefits of viewing a memory model in terms of (sequential) program transformations, eg. see~\citep{Saraswat:2007:TMM:1229428.1229469,DBLP:conf/fm/LahavV16,
DBLP:conf/popl/DemangeLZJPV13,DBLP:conf/esop/FerreiraFS10}, without explicitly being formalized as such.}.

Second, we focus on \emph{multi-copy atomicity} (\mca), which holds that when a write becomes visible to one thread it must become visible to all\nofootnote{\mca\ is traditionally explored in hardware memory models.   \tso\ (see, e.g.~\cite{DBLP:journals/cacm/SewellSONM10}) and recent architectures, such as \armeight\ (see, e.g.~\cite{DBLP:journals/pacmpl/PulteFDFSS18}), are \mca, but not older architectures, such as \ppc\ (see, e.g.~\cite{DBLP:conf/pldi/SarkarSAMW11}) or \armseven\ (see, e.g.~\cite{DBLP:conf/popl/AlglaveFIMSSN09}).} \citep{DBLP:journals/pacmpl/PulteFDFSS18}.  As envisioned in \cite[\textsection3.3]{AlglaveThesis},  this allows us to capture cross-thread dependencies  in a partial order.  We use \emph{partially ordered multisets (pomsets)}~\citep{GISCHER1988199,Plotkin:1997:TSP:266557.266600}, with the acyclicity of the pomset providing a global notion of time.  
Just as \mca{} dramatically simplifies the programmer model for hardware, % \citep{DBLP:journals/pacmpl/PulteFDFSS18},
this global notion of time dramatically simplifies our model for the language level.
% As far as we are aware, ours is the first language-level model to capture \mca.   The appeal of \mca{} in hardware is the dramatically simpler programmer model \citep{DBLP:journals/pacmpl/PulteFDFSS18}.  We believe it has the same appeal at the language-level.  

We show that the model:
\begin{itemize}
\item captures all C11 concurrency features  %, including relaxed, release-acquire and SC atomics, fences, and RMW
  (\textsection\ref{sec:variants}),

\item allows compositional reasoning for safety %properties
  (\textsection\ref{sec:logic}),

\item compiles to \armeight\ and \tso\ {\em without} extra synchronization for relaxed-atomic access\nofootnote{Compilation to \armseven\ or \ppc\ requires extra synchronization.} (\textsection\ref{sec:arm}),

\item  satisfies the \drfsc\ criterion  (\textsection\ref{sec:sc}), and

\item validates single-threaded compiler optimizations
  (\textsection\ref{sec:opt}).

\end{itemize}

We discuss compiler optimizations both concretely and abstractly.
Concretely, we show the validity of specific optimizations, such as the roach motel laws for synchronization.
Abstractly, we establish sufficient conditions to replace any command $\aCmd$ by an equivalent $\bCmd$: if
$\aCmd$ and $\bCmd$ are synchronization\hyp{}free and sequentially equivalent, and furthermore $\bCmd$ is \emph{linear}---performs at most one read and at most one write on each location in each execution---then $\aCmd$ can be refined to $\bCmd$.  

The linearity restriction ensures that the context cannot interfere with the atomic execution of the command, and, dually, that the atomic execution of the command cannot interfere with the context.   To see the need for this, consider that the introduction of redundant reads is valid sequentially, but not valid concurrently. For example, $\aReg \GETS \aLoc \SEMI \IF{\aReg \NOTEQ \aReg}\THEN \bLoc \GETS 1 \FI$ cannot be refined to $\aReg \GETS \aLoc \SEMI \bReg \GETS \aLoc  \SEMI  \IF{\aReg \NOTEQ \bReg} \THEN \bLoc \GETS 1 \FI$.  In a concurrent context, the latter program may see different values for the two reads.
% On the last point, we demonstrate a further completeness result.  Whereas the JMM aims to validate {\em all} sequential optimizations, it is clear that is impossible.  For example, the introduction of redundant reads  but not valid concurrently. Thus, $\aReg \GETS \REF{\aLoc} \SEMI \IF{\aReg != \aReg} \cLoc \GETS 1 \FI$ cannot be replaced by $\aReg \GETS \aLoc \SEMI \bReg \GETS \aLoc  \SEMI 
% \IF{\aReg != \bReg} \THEN \cLoc \GETS 1 \FI$.  

% Our model does the best possible under these constraints.   Call a program fragment ``linear'' if it does at most one read and at most one write on any location in any execution.  Thus, the context is unable to interfere with the atomic execution of the command; dually, neither can the atomic execution of the command interfere with the context.  We show that if sequential and synchronization free $\aCmd$ and $\bCmd$ are sequentially equivalent, and furthermore $\bCmd$ is ``'linear'' in this sense, then $\aCmd$ can be validly replaced by $\bCmd$.  
% Similarly, for redundant writes; $\aLoc \GETS 1$ cannot be 
% replaced by $\aLoc \GETS 1 \SEMI \aLoc \GETS 1 $ in a model 
% with coherence.





%\paragraph*{Rest of the paper. }  We begin with an informal introduction to the modeling ideas in \textsection\ref{sec:model:intro}, developing the precise formalities in \textsection\ref{sec:model}.   \textsection\ref{sec:sc} proves the DRF theorem, whereas \textsection\ref{sec:arm} provides a compilation into \armeight\ and \tso.  Single threaded optimizations, and the associated completeness theorems are addressed in \textsection\ref{sec:opt}.  \textsection\ref{sec:logic} describes a temporal logic, and a compositional proof principle for proving safety properties.  \textsection\ref{sec:examples}.develops more illustrative examples.  We address related work in \textsection\ref{sec:ldrf} and conclude after.  An appendix contains details of proofs and further examples.

% We give an informal introduction to the model in \textsection\ref{sec:model:intro} before presenting the precise formalities in \textsection\ref{sec:model}.
% \textsection\ref{sec:sc} proves the DRF theorem, whereas \textsection\ref{sec:arm} provides a compilation into \armeight\ and \tso.  Single threaded optimizations, and the associated completeness theorems are addressed in \textsection\ref{sec:opt}.  \textsection\ref{sec:logic} describes a temporal logic, and a compositional proof principle for proving safety properties.
% We %present additional examples in \textsection\ref{sec:examples} and
% end with
% a discussion of related work in \textsection\ref{sec:ldrf}.
% An appendix
% contains details of proofs and further examples.

In the main paper, we present the model, examples, the results concerning
compositional reasoning  and optimization, and a discussion of
related work.  The details of \armeight/\tso-compilation and \drfsc\ may be found in the appendix.

\endinput

To the reader interested in models that forbid load buffering, we provide a way to adapt our model to forbid the relaxing of the program order from reads to writes, thus modeling~\cite{Dolan:2018:BDR:3192366.3192421,BoehmOOTA}.  Our new contributions for such a reader are an approach to validating data-sensitive compiler optimizations and compositional reasoning of temporal properties.   


We illustrate the last criterion with two examples.   
First, consider the well-known ``Out Of Thin Air'' (\oota) litmus test, with all variables initialized to $0$:
\begin{equation}
  \label{oota1}
  %x\GETS0\SEMI y\GETS0\SEMI
  (y\GETS x \PAR x \GETS y)
\end{equation}
Informally, both threads satisfy the invariant that conjoins ``A write of 1  to x  requires a prior read  of 1 from y'' and ``A write of 1  to y  requires a prior read  of 1 from x ''.  If composition holds, the full program satisfies the invariant.  Since the variable declaration closes the program from other writes to $x,y$, we  deduce the conjunction of  ``A write of 1  to x  requires a prior write  of 1 to x'' and ``A write of 1  to y  requires a prior write  of 1 from x'' . Thus, we deduce that ``A write of 1  to x  requires a prior write  of 1 to x'', and consequently ``there is no write of 1 to x''. 

provides an  {\em objectively  falsifiable} measurement of \oota\ in a memory model. 



