\section{Compiling to hardware}

\citet{DBLP:journals/pacmpl/PodkopaevLV19} define the \emph{Intermediate
  Memory Model (IMM)} and provide efficient implementations of the IMM into
several processor architectures, including TSO, ARMv8 and Power.

In this section, we show that any execution allowed by a sublanguage of the
IMM is also allowed by our semantics.  The sublanguage we consider bans
loops, read-modify-write (RMW) operations, and fences.  In addition, we take
the set of memory locations, $\Loc$, to be finite.  Syntactically, we drop
the superscript \textsf{rlx} on relaxed reads and writes; in addition, we use
structured conditionals rather than the more general \textsf{goto}.  We refer
to this sublanguage as $\muIMM$.

$\muIMM$ programs sit in the restriction-free fragment of our language, where
all memory locations are initialized to $0$ and parallel-composition occurs
only at top level.  In other words, $\muIMM$ programs have the form
\begin{displaymath}
  {\aLoc_1}\GETS{0}\SEMI
  \cdots\SEMI
  {\aLoc_m}\GETS{0}\SEMI
  (\aCmd^1 \PAR \cdots \PAR \aCmd^n)
\end{displaymath}
where $\aCmd^1$, \ldots, $\aCmd^n$ do not include either composition or
restriction.

Due to space limitations, we do not include a full description of the IMM.
The broad strokes of the argument given here should be clear, but interested
readers will need to refer to \citep{DBLP:journals/pacmpl/PodkopaevLV19} for
details.

Let $G$ be an execution graph for a $\muIMM$ program satisfying the
consistency requirements in \textsection3.4 of
\citep{DBLP:journals/pacmpl/PodkopaevLV19}. (Because the source language
lacks RMW operations, the ``is exclusive'' flag on every read will be
\textsf{not-ex} and the RMW mode on every write will be \textsf{normal}.)

Let $R^*$ denotes the reflexive and transitive closure of relation $R$.  Let
$R;S$ denote the composition of relations $R$ and $S$.

Given an execution graph $G$, we say that $\aEv$ is an \emph{internal read} if
$\aEv\in\fcodom(G.\mathsf{po}\cap G.\mathsf{rf})$.

From $G$ we construct a candidate pomset $\aPS$ as follows:
\begin{itemize}
\item $\Event= G.\textsf{E}$,
\item $\labelingAct(\aEv)=\tau G.\mathsf{lab}(e)$, if $\aEv$ is a relaxed
  internal read, 
\item $\labelingAct(\aEv)=G.\mathsf{lab}(e)$, if $\aEv$ is not a relaxed
  internal read,
\item $\labelingForm(\aEv)=\TRUE$,
\item ${\leq} = G.{\rar}^*$, and
\item ${\gtN} = {\leq} \cup {\leq}; {G.{\reco}} \cup {G.{\reco}} ; {\leq}$
\end{itemize}

We show that $\aPS$ is a top-level pomset, reasoning as follows.
First, we establish the criteria for a 3-valued pomset (Definition~\ref{def:3valued}).
\begin{itemize}
\item ${\le}$ is a partial order.  This holds since $G.{\rar}$ is acyclic.
\item If $\bEv \le \aEv$ then $\bEv \gtN \aEv$.  By construction.
\item If $\bEv \le \aEv$ and $\aEv \gtN \bEv$ then $\bEv = \aEv$.  ????
\item If $\cEv \le \bEv \gtN \aEv$ or $\cEv \gtN \bEv \le \aEv$ then
  $\cEv \gtN \aEv$. By construction.
\end{itemize}

Next, we establish the criteria for a 3-valued pomset with preconditions (Definition~\ref{def:3pre}).
\begin{itemize}
\item $\labelingForm(\aEv)$ implies $\labelingForm(\bEv)$ whenever
  $\bEv\le\aEv$.   Trivial, since every formula is $\TRUE$.
\item $\aPS$ is $\aLoc$-coherent; that is, when restricted to events that
  read or write $\aLoc$, $\gtN$ forms a partial order.
\end{itemize}

Finally, we establish the criteria for a top-level pomset
(Definition~\ref{def:x-closed}).
\begin{itemize}
\item $\aEv$ is location independent. Trivial, since every formula is $\TRUE$.
\item If $\aEv$ reads $\aVal$ from $\aLoc$, then there is some $\bEv$ such that
  \begin{itemize}
  \item $\bEv \lt \aEv$,  
  \item $\bEv$ writes $\aVal$ to $\aLoc$, and
  \item if $\cEv$ writes to $\aLoc$
    then either $\cEv \gtN \bEv$ or $\aEv \gtN \cEv$.
  \end{itemize}    
\end{itemize}
