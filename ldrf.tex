\section{Comparison to related work}\label{sec:ldrf}
A memory consistency model for a shared-memory multiprocessor defines the
values that a read may return.  There has been extensive research on hardware
and software memory models. \citet{DBLP:journals/pacmpl/PodkopaevLV19}
provided formal foundations to bridge the gap between the two.
\citet{AlglaveThesis} provided a historical and conceptual perspective on
hardware memory models.

This paper focusses on software memory models.  For a survey and history of
the Java Memory Model, see \citep{DBLP:journals/toplas/Lochbihler13}; for a
survey and history of C11, see \citep{DBLP:phd/ethos/Batty15}.

This paper follows a line of work on relaxed memory models in the framework
of true concurrency
\cite{DBLP:conf/lics/JeffreyR16,Pichon-Pharabod:2016:CSR:2837614.2837616,DBLP:conf/esop/CenciarelliKS07}.

We have already discussed related work in context earlier in the paper.
Here, we discuss in more detail the relationship with the most closely
related work.

\subsection{Memory models via program transformations}
We discuss prior approaches to definining memory models using program
transformations in chronological order.  The general flavor in these
approaches is to consider SC executions, albeit with the threads subject to
sequential program transformation.

\citet{Saraswat:2007:TMM:1229428.1229469}
aimed to describe a \jmm-like model this way.  \drfsc\ holds, but \citet{SP} 
discovered that it permits \oota\ behavior.

\citet{DBLP:conf/esop/FerreiraFS10} described a relaxed memory model as being
parametrized by available program transformations.  They showed \drfsc, but do not
study \oota, relate to concrete models, or demonstrate compilation results.

\citet{DBLP:conf/popl/DemangeLZJPV13} developed \textsc{bmm}, a model whose only
permitted reordering is of a relaxed write with a following relaxed read.
The paper develops an axiomatic and studies effects on compilation.  By
design, it is designed as a restriction of the \jmm\ that invalidates several
compiler optimizations.

\citet{DBLP:conf/fm/LahavV16} characterized \tso\ as being derived by
considering Write-Read (\textsc{wr}) reordering and Read-After-Write (\textsc{raw})
elimination.  They also showed that the release acquires of C11 are less
expressive than considering \textsc{wr}, \textsc{raw} and thread-inlining.  Our paper is
inspired by their implicit challenge: ``Some memory models can be defined via
transformations.  But there is more to weak memory than transformations.''

\subsection{Modeling the invalidation of load buffering} 

In this subsection, we argue that our paper is relevant even to the reader
who desires a semantics that invalidates ``load buffering,'' or ``reads from the future.''
There are two related lines of research that suggest to prohibit the
reordering of loads with following stores.
\begin{itemize}
\item \citet{Dolan:2018:BDR:3192366.3192421} imposed this restriction in
  order to ``bound data races in space and time.''
  Consider the following program:
  % In the following program $r,s$ are registers, and $x,y,z,u,v,w$
  % are shared variables.
  % \begin{align*}
  %   & (u \GETS 0 \SEMI y \GETS 1 \SEMI v\REL  \GETS 1 \SEMI z \GETS 2 \SEMI r \GETS u \SEMI y \GETS 3) \\[-.5ex]
  %   \PAR & (y \GETS 2 \SEMI  \IF{v\ACQ} \THEN w \GETS 1 + y -y \SEMI u \GETS 1 \SEMI x \GETS 1  \FI \SEMI s \GETS u \SEMI  y \GETS 2)
  % \end{align*}
\begin{displaymatharrayrl}
  &x\GETS1\SEMI y\GETS1\SEMI v\REL\GETS1\SEMI z\GETS1\SEMI
  \IF{w}\THEN x\GETS2\FI
  \\\PAR
  &y\GETS2\SEMI \aReg{}\GETS v\ACQ\SEMI z\GETS2\,\SEMI
  \IF{\aReg{}}\THEN w\GETS{y-y+x}\FI
\end{displaymatharrayrl}
\begin{displaymath}
    \begin{tikzpicture}[node distance=1.2em and .8em,baselinecenter]
    \event{a1}{\DW{x}{1}}{}
    \event{a2}{\DW{y}{1}}{right=of a1}
    \event{a3}{\DWRel{v}{1}}{right=of a2}
    \event{a4}{\DW{z}{1}}{right=of a3}
    \po{a1}{a2}
    \po{a2}{a3}
    \po{a3}{a4}
    \event{b1}{\DW{y}{2}}{below=of a2}
    \event{b2}{\DRAcq{v}{1}}{below=of a3}
    \event{b3}{\DW{z}{2}}{right=of b2}
    %\co{b1}[right]{a2}
    \event{b4}{\DR{y}{?}}{right=of b3}
    \event{b5}{\DR{y}{?}}{right=of b4}
    %\co{b3}[right]{a4}
    \event{b6}{\DR{x}{1}}{right=of b5}
    \po{b1}{b2}
    \po{b2}{b3}
    \po{b3}{b4}
    \po{b4}{b5}
    \po{b5}{b6}
    \rf{a3}{b2}
    \event{a5}{\DR{w}{1}}{right=of a4}
    \event{a6}{\DW{x}{2}}{right=of a5}
    \po{a4}{a5}
    \po{a5}{a6}
    \event{b7}{\DW{w}{1}}{right=of b6}
    \po{b6}{b7}
    \rf{b7}{a5}
    %\race{b7}{a5}
    %\race{a6}{b6}
  \end{tikzpicture}
\end{displaymath}
In their model, one can deduce that $w$ must be assigned $1$, despite the
past race on $y$, concurrent race on $z$ and future race on $x$.
% In their model, one can reason sequentially about the the assignment to $w$
% and deduce that $x=1$ implies $z= 1$ despite the concurrent races on $u$, and
% the past and future races on $y$, where past and future are stated with
% respect to the assignment to $w$.

\item In the C11 family of models, eg. see
  \citep{Boehm:2014:OGA:2618128.2618134,DBLP:conf/pldi/LahavVKHD17,DBLP:conf/oopsla/VafeiadisN13},
  load buffering is forbidden to avoid \oota. The elegant discussion in
  \citet{BoehmOOTA} elucidates the issues via a series of ``\oota'' like
  examples.  In particular, to rule out example~\ref{rfub}.
\end{itemize}

Our model can easily capture models that forbid load buffering.  It
necessitates two changes.
\begin{itemize}
\item Modify the semantic rule for read to remove the possibility of an
  internal read:
  \begin{align*}
    \sem{\aReg\GETS\REF{\cExp}\SEMI \aCmd} & =
    \textstyle\bigcup_{\aLoc=\REF{\bVal}}\; ({\cExp}=\bVal) \guard \textstyle\bigcup_\aVal\; (\DR\aLoc\aVal) \prefix \sem{\aCmd}[\aLoc/\aReg] 
  \end{align*}
\item Modify item \ref{pre-read}b in the definition of prefixing to require
  order from a read to subsequent write.  Item \ref{pre-read}b$'$ becomes: if
  $\aEv$ is a write then $\cEv\lt'\aEv$.  In \ref{pre-read}b$'$ we removed
  the ``or or $\labelingForm'(\aEv)$ implies $\labelingForm(\aEv)$.''
\end{itemize}

While \citet{Dolan:2018:BDR:3192366.3192421} already describes DRF and the
compilation to \armeight, our approach yields the following new insights.
\begin{itemize}
\item Compositional treatment of temporal invariants
  from \textsection\ref{sec:logic} holds mutatis mutandis, since all
  executions of the restricted model are already accounted for in the general
  model.

\item With regards to single-threaded optimizations in
  \textsection\ref{sec:opt}, our approach provides different methods to prove
  optimizations, which complement the extant methods of
  \citet{Dolan:2018:BDR:3192366.3192421} with data sensitivity; for example,
  our methods provides a simple proof of the transformation of
  $\IF{\aExp}\THEN x \GETS 1 \SEMI \aCmd \ELSE x\GETS 1 \SEMI \bCmd\FI$ to
  $x \GETS 1 \SEMI \IF{\aExp}\THEN \aCmd \ELSE \bCmd\FI$.

\end{itemize}

\subsection{Revisiting \oota}
\citet{DBLP:conf/esop/BattyMNPS15} describe the problem of thin-air
executions. \citet{BoehmOOTA} revisits community discussions on ``\oota''
like examples.  We classify the models on the basis of their behavior on
three examples: \eqref{oota1}, \eqref{alan} and \eqref{rfub}.
\begin{itemize}
\item RC11\cite{DBLP:conf/pldi/LahavVKHD17}, \jmm\ 
  \cite{Manson:2005:JMM:1047659.1040336} and the related models of
  \citet{DBLP:conf/esop/JagadeesanPR10} and \citet{DBLP:conf/popl/KangHLVD17}
  forbid only \eqref{oota1}.
\item The model of this paper forbids \eqref{oota1} and \eqref{alan}.
\item Forbidding load buffering as per
  \citep{Dolan:2018:BDR:3192366.3192421,BoehmOOTA} forbids all three
  examples.  RC11\cite{DBLP:conf/pldi/LahavVKHD17} also forbids all three
  examples.
\end{itemize}

One of our novel contributions to this line of research is the desideratum of
compositional reasoning of safety properties. The presence of such
compositional reasoning suffices to remove the undesirable effects of \oota,
eg. type safety.  And while \oota\ in a model is an elusive creature, and
hence hard to prove or disprove, the support of a model for a compositional
proof principle is objectively provable or falsifiable.

\subsection{Relationship to models of speculation}
\citet{2019-sp} is a close formal cousin to our approach.  In the spectrum
from micro architectures to architectures to software,whereas \citet{2019-sp}
lies between the micro and architectural levels, we aim to be as abstract as
possible to bridge the gap to programming languages.  This is reflected in
the several differences in the formal model.

We consider a single global acyclic order in contrast to the two studied in
\citet{2019-sp}.  We do not track false branches of conditionals, as seen in
our monotonicity axiom, whereby any event with a precondition $\FALSE$ can
only have $\FALSE$ successors in $\lt$, whereas in that paper, speculation on
false branches can lead to observable effects.  We are also less accurate
about tracking reads, as reflected in the removal of program order between
reads and reads.

\citet{2019-sp} does not study the properties as a memory model.  We
conjecture that our proofs of DRF and compositional reasoning can be carried
over to that setting. \citet{2019-sp} also preserves some $\rpox$ between
reads, whereas we do not; so, we certainly support more compiler
optimizations of single threaded code.


\section{Conclusions}
We have defined a relaxed memory model, using standard tools from the
semantics and demonstrated that it satisfies the fundamental criterion for
software memory models, namely ``the memory model must strike a balance
between ease-of-use for programmers and implementation flexibility for system
designers.''

One key lesson we draw is that any software relaxed memory model should
support the cyclical composition proof rule of
\citet{Abadi:1993:CS:151646.151649} for temporal properties.  We posit that all
the benefits of \oota, such as typing guarantees, can be derived from this
principle. 

We conclude with a suggestion for the consideration of programming language
designers.  The semantics of a programming language should support a clear
and executable prescription of dependency in a program.  Such a specification
provides a nice interface between the language user and the compiler writer,
and a clear bridge from the compiler writer to architectures.

Our current paper is to be viewed as a step towards this goal.  We will
explore developing an executable specification based on these ideas in future
work.
 



