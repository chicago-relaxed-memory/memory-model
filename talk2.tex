\documentclass[t,aspectratio=169]{beamer} %tlmgr install translator
%\usepackage{media9}

\usepackage{macros}
\usepackage{talk2}

%\includeonlyframes{oota-pomsets}  
\usepackage{pdfcomment}
% \setbeameroption{show notes on second screen}
\newcommand<>{\pdfnote}[1]{%
  \only#2{\marginnote{\pdfcomment[icon=note,hspace=10pt,color=white,hoffset=30cm]{#1}}}%
  \note#2[item]{#1}%
}
\usecolortheme{seahorse}
\usetheme{default}
\setbeamertemplate{navigation symbols}{\insertframenumber/\inserttotalframenumber}
\setbeamertemplate{itemize item}[circle]
\setbeamertemplate{itemize subitem}[circle]
\setbeamertemplate{itemize subsubitem}[circle]


\begin{comment}
  
  While developing slides, use
  \includeonlyframes{current}  

  Then latex will only process slides with
  \begin{frame}[label=y]
  \end{frame}  
  
  \begin{frame}
    \frametitle{}
    \begin{itemize}
    \item 
    \end{itemize}
  \end{frame}
  \begin{itemize}
  \item 
  \end{itemize}

  \begin{framecenter}
  \end{framecenter}

  Great Beamer guide here:
  https://www.overleaf.com/learn/latex/Beamer_Presentations:_A_Tutorial_for_Beginners_(Part_4)—Overlay_Specifications

  How to present on mac without blurring:
  Use firefox, or Presentation.app
  https://tex.stackexchange.com/questions/431423/os-x-blurry-beamer-presentations-on-sierra-high-sierra

  With Presentation.app:
  \usepackage{pdfcomment}
  \newcommand{\pdfnote}[1]{\marginnote{\pdfcomment[icon=note]{#1}}}

  To convert from PDF to KeyNOTE: use the automator suggestion here
  https://apple.stackexchange.com/questions/95856/import-multiple-pages-from-a-pdf-as-separate-slides-in-keynote
  I put the automator app here: https://www.dropbox.com/sh/1o36nes45cydmbx/AADPVWpeEBZj2vemVXkGKQXVa?dl=0
  - Copy that to your machine somewhere.
  - In the finder, drop talk.pdf onto "PDF Save Pages as Images.app".
  - This will create a bunch of png files on the desktop
  - Now, open keynote with a blank document.
  - In the finder, select the files, drop onto the leftmost "slides" pane of Keynote.

  Another display option is to use http://dspdfviewer.danny-edel.de

  For Beamer notes:
  \usepackage{pgfpages}
  \setbeameroption{show notes on second screen}

  Skim is not great for this:

  To give a presentation with the Skim reader (http://skim-app.sourceforge.net) %on OSX so
  that you see the notes on your laptop and the slides on the projector, do %the following:
  
  1. Generate just the presentation (hide notes) and save to slides.pdf
  2. Generate onlt the notes (show only nodes) and save to notes.pdf
  3. With Skim open both slides.pdf and notes.pdf
  4. Click on slides.pdf to bring it to front.
  5. In Skim, under "View -> Presentation Option -> Synhcronized Noted Document"
  select notes.pdf.
  6. Now as you move around in slides.pdf the notes.pdf file will follow you.
  7. Arrange windows so that notes.pdf is in full screen mode on your laptop
  and slides.pdf is in presentation mode on the projector.
  You can use the keys "t" and "d" to get different things on the presenter screen

  \pause
  \onslide<>

  \visible<>{during}                 \begin{visibleenv}<>during\end{visibleenv}
  \invisible<>{other}                \begin{invisibleenv}<>other \end{invisibleenv}
  
  \onslide<>{during}
  \uncover<>{during}                 \begin{uncoverenv}<>during\end{uncoverenv}
  
  \only<>{during}                    \begin{onlyenv}<>during\end{onlyenv}
  \alt<>{during}{other}              \begin{altenv}<>{during-before}{during-after}{other-before}{other-after}text\end{altenv}
  \temporal<>{before}{during}{after}
  \setbeamercovered{transparent=35}


  \begin{itemize}[<+-| alert@+>]
  \end{itemize}

  \begin{columns}
    \begin{column}{0.5\textwidth}
    \end{column}
    \begin{column}{0.5\textwidth}
    \end{column}
  \end{columns}

\end{comment}

\title{Pomsets with Preconditions}
\subtitle{A Simple Model of Relaxed Memory}
\author{Radha Jagadeesan$^\dagger$ \and \alert{Alan Jeffrey}$^*$ \and \alert{James Riely}$^\dagger$}
% \date{\href{https://www.cs.kent.ac.uk/research/groups/plas/seminarcalendar.html?view_by=month\&date=20201019\&category=\&tag=plas}{PLaS seminar, University of Kent}\\19 October 2020}
\date{OOPSLA\\November 2020}
\institute{$^\dagger$DePaul University \and $^*$Roblox}




\begin{document}
\begin{frame}[label=title]
  \maketitle
\end{frame}

\begin{frame}
  \frametitle{A Simple Task!}
  A model of memory that supports:
  \begin{itemize}
  \item<+-> concurrency (with data races)
  % \item<+-> data races
  \item<+-> hardware reorderings
  \item<+-> compiler reorderings
  \item<+-> temporal reasoning (compositionally)
  \item<+-> a few other things:
    \begin{itemize}
    \item java causality test cases
    \item if-introduction (aka, case analysis)
    \item redundant read elimination
    \item implementation on x86-64 and Arm8
    \item address calculation
    \item release/acquire and SC access
    \item fences and RMWs
    \end{itemize}
  \end{itemize}
\end{frame}

\begin{frame}
  \frametitle{What does this mean?}
  \huge
  \begin{displaymath}
    \PW{x}{0}
    \SEMI
    \PR{x}{r}
    \SEMI
    \PW{y}{r+1}
  \end{displaymath}
  \pause
  \begin{displaymatharray}{rll}
    \hoare{\TRUE}{&\PW{x}{0}&}{x=0}\\\pause
    \hoare{x=0}{&\PR{x}{r}&}  {r=0}\\\pause
    \hoare{r=0}{&\PW{y}{r+1}&}{y=1}
  \end{displaymatharray}
  \pause
  \bigX
\end{frame}

\begin{frame}
  \frametitle{What about concurrency? ($r$ is a register)}
  \huge
  \begin{displaymath}
    \PW{x}{0}
    \SEMI
    \PR{x}{r}
    \SEMI
    \PW{y}{r+1}
    \PAR
    \PW{x}{5}
  \end{displaymath}
  \pause
  \begin{displaymath}
    \scalebox{2}{    
      \begin{tikzinline}[node distance=1.5em]
        \event{a}{\DW{x}{0}}{}
        \event{b}{\DR{x}{\alt<-4>{5}{0}}}{right=of a}
        \event{c}{\DW{y}{\alt<-4>{6}{1}}}{right=of b}
        \event{d}{\DW{x}{5}}{right=3em of c}      
        \visible<1-3>{\po{a}{b}}
        \po{b}{c}
        \visible<3-4>{\rf[out=-160,in=-20]{d}{b}}
        \visible<4-4>{\wk[out=20,in=160]{a}{d}}
        \visible<5->{\rfi{a}{b}}
        \visible<5->{\wk[out=20,in=160]{b}{d}}
      \end{tikzinline}}
  \end{displaymath}

  \onslide<6->
  \normalsize
  Complete pomset: $\forall\DR{x}{v}$
  \begin{itemize}
  \item $\exists\DW{x}{v}<\DR{x}{v}$
  \item $\forall\DW{x}{v'}$ either $\DW{x}{v'}\leq\DW{x}{v}$ or $\DR{x}{1}<\DW{x}{v'}$
  \end{itemize}
  \onslide<7->
  \bigX
\end{frame}

\imagepage{images/what-if.jpg}{}
%\imagepage{images/ss.gif}{}

\begin{frame}
  \frametitle{What store buffering?}
  \huge
  \begin{displaymath}
    \PW{x}{0}
    \SEMI
    \PW{x}{1}
    \SEMI
    \PR{y}{0}
    \PAR
    \PW{y}{0}
    \SEMI
    \PW{y}{1}
    \SEMI
    \PR{x}{0}
  \end{displaymath}
  \begin{displaymath}
    \scalebox{1.70}{    
      \begin{tikzinline}[node distance=1.5em]
        \event{a}{\DW{x}{0}}{}
        \event{b}{\DW{x}{1}}{right=of a}
        \event{c}{\DR{y}{0}}{right=of b}
        \event{d}{\DW{y}{0}}{right=3em of c}
        \event{e}{\DW{y}{1}}{right=of d}
        \event{f}{\DR{x}{0}}{right=of e}
        \po{a}{b}
        \po{b}{c}
        \po{d}{e}
        \po{e}{f}
      \end{tikzinline}}
  \end{displaymath}
\end{frame}


\end{document}

% Local Variables:
% mode: latex
% TeX-master: t
% End:
        
